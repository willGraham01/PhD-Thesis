\documentclass[a4paper]{report}

%Import thesis style file
\usepackage{baththesis}
%Define thesis style macros and names
\title{Asymptotic and Numerical Analysis of Wave Propagation in Thin-Structure Waveguides}
\author{William Michael Graham}
\degree{Doctor of Philosophy}
\department{Department of Mathematical Sciences}
\degreemonthyear{June 2022}

%Image and page layout packages
\usepackage{graphicx}
\usepackage{subcaption} %allows subfigures
\usepackage[bottom]{footmisc} %footnotes go below figures
\usepackage{parskip} %adds line space between paragraphs by default
%Declare location of image files
\usepackage{appendix} %can declare chapter-local appendices

\DeclareGraphicsRule{.tif}{png}{.png}{`convert #1 `dirname #1`/`basename #1 .tif`.png}
\graphicspath{{./Diagrams/Diagram_PDFs/} {./Diagrams/Numerical_Results/}}

%Allows for more options when enumerating lists (roman, lettering, etc)
\usepackage{enumerate}

%\input imports all commands from the target files
%The idea behind this file is that it will be used to store all the maths-related macros that I concoct; so that I can import all the commands by \input{this file} in the preamble of any file that I want to use them in.
%This should make the top-level files look a lot cleaner, and the preamble much shorter!

\usepackage{amssymb}
\usepackage{amsmath}

%% EDITING FOR TYPO/ADDITIONS THAT ARE REQUIRED
\usepackage{xcolor}
\newcommand{\tstk}[1]{\textcolor{red}{\textbf{#1}} \newline}

%% THEOREMS AND LEMMAS, ETC using amsthm

\usepackage{amsthm}
\theoremstyle{definition}
\newtheorem{definition}{Definition}[section]
\theoremstyle{definition}
\newtheorem{problem}{Problem}[section]
\theoremstyle{plain}
\newtheorem{theorem}{Theorem}[section]
\theoremstyle{plain}
\newtheorem{lemma}[theorem]{Lemma}
\theoremstyle{plain}
\newtheorem{prop}[theorem]{Proposition}
\theoremstyle{plain}
\newtheorem{cory}[theorem]{Corollary}
\theoremstyle{definition}
\newtheorem{convention}[theorem]{Convention}
\theoremstyle{definition}
\newtheorem{assumption}[theorem]{Assumption}
\theoremstyle{definition}
\newtheorem{conjecture}[theorem]{Conjecture}

%% EQUATION FORMATING THROUGHOUT DOCUMENT

% allows equations in the same align environment to split over multiple pages.
\allowdisplaybreaks 

%% EQUATION FUNCTIONALITY - SHORTHAND, LARGE OPERATOR FORMATTING, ETC

% UNITS AND OPERATORS
\newcommand{\rmi}{\mathrm{i}}					% imaginary unit i (RoMan font i)
\newcommand{\e}{\mathrm{e}}						% Euler number e (Roman font e)
\newcommand{\md}{\mathrm{d}}					% differential d (Roman d)
\newcommand{\charFunc}[1]{\mathcal{I}_{#1}}	% characteristic function of a set
\newcommand{\sgn}{\mathrm{sgn}}				% sign function of an input

% MATRIX DISPLAY: adds extra functionality to pmatrix, vmatrix, bmatrix etc by allowing you to pass an optional argument in [FACTOR] to multiply the default spacing between elements by FACTOR
\makeatletter
\renewcommand*\env@matrix[1][\arraystretch]{%
  \edef\arraystretch{#1}%
  \hskip -\arraycolsep
  \let\@ifnextchar\new@ifnextchar
  \array{*\c@MaxMatrixCols c}
  }
\makeatother

% SHORTHAND
\newcommand{\recip}[1]{\frac{1}{#1}}								% writes 1/{#1}
\renewcommand{\vec}[1]{\mathbf{#1}}								% vectors are bold, no overhead arrow
\newcommand{\lconv}[1]{\overset{#1}{\longrightarrow}}				% convergence with #1 above arrow
\newcommand{\toInfty}[1]{ \ \text{as} \ #1 \rightarrow\infty}		% writes out "as #1 tends to infty"

% (standard) INTEGRAL AND DIFFERENTIAL OPERATORS
% integral display: \integral{domain}{integrand}{measure/integral variable}
\newcommand{\integral}[3]{\int_{#1}#2 \ \mathrm{d}#3}
% derivatives of various types
\newcommand{\diff}[2]{\dfrac{\mathrm{d}#1}{\mathrm{d}#2}}				% complete derivative d#1/d#2
\newcommand{\pdiff}[2]{\dfrac{\partial #1}{\partial #2}}				% partial derivative p#1/p#2
\newcommand{\ddiff}[2]{\dfrac{\mathrm{d}^2 #1}{\mathrm{d} {#2}^2}}	% 2nd complete deriv
\newcommand{\pddiff}[2]{\dfrac{\partial^2 #1}{\partial {#2}^2}}		% 2nd partial derivative
\newcommand{\grad}{\nabla}												% gradient operator
\newcommand{\laplacian}{\Delta}										% Laplacian
\newcommand{\curl}[1]{\mathrm{curl}_{#1}}								% curl with subscript #1
\newcommand{\dmap}{\Gamma_0}											% Dirichlet map
\newcommand{\nmap}{\Gamma_1}											% Neumann map
\newcommand{\dtn}{\mathcal{D}}											% Dirichlet-to-Neumann map
\newcommand{\gelfand}{\mathfrak{G}}									% Gelfand transform

% BRACKETS AND NORMS - encloses inputs in relevent bracers
\newcommand{\bracs}[1]{\left( #1 \right)}					% ( brackets )
\newcommand{\sqbracs}[1]{\left[ #1 \right]}				% [ square bracers ]
\newcommand{\clbracs}[1]{\left\{ #1 \right\}}				% { curly bracers }
\newcommand{\abs}[1]{\left\lvert #1 \right\rvert}			% | absolute value |
\newcommand{\norm}[1]{\lvert\lvert #1 \rvert\rvert}		% || norm ||
\newcommand{\setVert}{\ \middle\vert \ }					% vertical bar for the middle of sets
\newcommand{\ip}[2]{\left\langle #1 , #2 \right\rangle}	% < inner , product >

% STANDARD SETS
\newcommand{\naturals}{\mathbb{N}}			% natural numbers
\newcommand{\integers}{\mathbb{Z}}			% integers
\newcommand{\rationals}{\mathbb{Q}}		% rational numbers
\newcommand{\reals}{\mathbb{R}}			% real numbers
\newcommand{\complex}{\mathbb{C}}			% complex numbers

\newcommand{\smooth}[1]{C^{\infty}\bracs{#1}}				% smooth functions
\newcommand{\psmooth}[1]{C^{\infty}_{\#}\bracs{#1}}		% periodic smooth functions
\newcommand{\csmooth}[1]{C^{\infty}_{0}\bracs{#1}}			% compactly supported smooth functions
\newcommand{\ltwo}[2]{L^{2}\bracs{#1,\mathrm{d}#2}}		% L^2(domain, measure) space
\newcommand{\htwo}[1]{H^2_\mathrm{grad}\bracs{#1}}			% H^2_{gradients}(domain)
\newcommand{\supp}{\mathrm{supp}}							% support of a function
\newcommand{\dom}{\mathrm{dom}}							% domain of an operator

%% MACROS FOR THESIS NOTATION

%creates the closed interval from 0 to the length of the input #1, denoted by absolute value
\newcommand{\interval}[1]{\sqbracs{0,\abs{#1}}}

% SYMBOLS, CONSTANTS, COMMON VARIABLES
\newcommand{\eps}{\varepsilon}					% pretty epsilons to distingush from physical constants
\newcommand{\epsFS}{\epsilon_0}				% permittivity of free space
\newcommand{\muFS}{\mu_0}						% permeability of free space
\newcommand{\wavenumber}{\kappa}				% fourier variable or wavenumber
\newcommand{\qm}{\theta}						% quasi-momentum parameter
\newcommand{\kt}{\bracs{\wavenumber, \qm}}		% (\wavenumber, \qm) pair
\newcommand{\effFreq}{\Lambda}					% Effective frequency sqrt(w^2-\wavenumber^2)

% DOMAINS + MEASURES
\newcommand{\dddom}{\widetilde{\Omega}}		% 3D domain notation
\newcommand{\ddom}{\Omega}						% 2D domain notation

\newcommand{\massMes}{\nu}						% vertex point mass supporting measure
\newcommand{\ddmes}{\mu}						% 2D sing. measure
\newcommand{\dddmes}{\widetilde{\ddmes}}		% 2D sing. measure + point masses
\newcommand{\compMes}{\lambda_2^{\ddmes}}		% 2D composite Leb + sing measure
\newcommand{\ccompMes}{\lamda_2^{\dddmes}}		% 2D composite Leb + sing + point mass measure

% GRAPHS
\newcommand{\graph}{\mathbb{G}}					% graph variable
\newcommand{\vertSet}{\mathcal{V}}					% set of vertices rather than big V
\newcommand{\edgeSet}{\mathcal{E}}					% set of edges rather than large E, \graph = (V,E)
\newcommand{\dgmap}{\dmap^{\graph}}				% Dirichlet map with graph superscript
\newcommand{\ngmap}{\nmap^{\graph}}				% Neumann map with graph superscript
\newcommand{\ag}{\mathcal{A}_\graph}				%caligraphic A with graph script for distinction

% GRAPH CONNECTIONS
\newcommand{\conLeft}{\stackrel{\rightarrow}{\smash{\sim}\rule{0pt}{0.4ex}}}	% j connects to k, j left
\newcommand{\conRight}{\stackrel{\leftarrow}{\smash{\sim}\rule{0pt}{0.4ex}}}	% j connects to k, j right
\newcommand{\con}{\sim} 														% j connects to k

% SHIFTED GRADIENT OPERATORS
\newcommand{\ograd}{\grad^{(0)}}								% grad operator with 0 superscript
\newcommand{\tgrad}{\nabla^{\qm}}								% grad operator with qm superscript
\newcommand{\kgrad}{\grad^{(\wavenumber)}}						% grad with wavenumber superscript
\newcommand{\ktgrad}{\grad^{\kt}}								% grad with wavenumber, qm superscript
\newcommand{\kcurl}[1]{\mathrm{curl}_{#1}^{(\wavenumber)}}	 	% k-curl with measure subscript #1
\newcommand{\ktcurl}[1]{\mathrm{curl}_{#1}^{\kt}}				% k,theta-curl with measure subscript #1

% GRAD/CURL OF ZERO SETS: format is \command{domain}{measure}
\newcommand{\gradZero}[2]{\mathcal{G}_{ #1, \mathrm{d}#2}\bracs{0}}					% gradients of zero
\newcommand{\kgradZero}[2]{\mathcal{G}_{ #1, \mathrm{d}#2}^{(\wavenumber)}\bracs{0}}	% k-gradients of zero
\newcommand{\curlZero}[2]{\mathcal{C}_{ #1, \mathrm{d}#2}\bracs{0}}					% curls of zero
\newcommand{\kcurlZero}[2]{\mathcal{C}_{ #1, \mathrm{d}#2}^{(\wavenumber)}\bracs{0}}	% k-curls of zero

% SOBOLEV SPACES: format is \command{domain}{measure}
\newcommand{\gradSob}[2]{H^1_\mathrm{grad}\bracs{#1, \mathrm{d}#2}}			% gradient SS
\newcommand{\tgradSob}[2]{H^1_{\qm, \mathrm{grad}}\bracs{#1, \mathrm{d}#2}} 	% t-gradient SS
\newcommand{\gradSobQM}[2]{\tgradSob{#1}{#2}}									% t-gradient SS, legacy

%\wavenumber,\qm-gradient Sobolev space
\newcommand{\ktgradSob}[2]{H^1_{\wavenumber,\qm,\mathrm{grad}}\bracs{#1, \mathrm{d}#2}}	

\newcommand{\curlSob}[2]{H^1_\mathrm{curl}\bracs{#1, \mathrm{d}#2}}						% curl SS
\newcommand{\tcurlSob}[2]{H^1_{\qm, \mathrm{curl}}\bracs{#1, \mathrm{d}#2}}				% t-curl SS
\newcommand{\kcurlSob}[2]{H^1_{\wavenumber,\mathrm{curl}}\bracs{#1, \mathrm{d}#2}}		% k-curl SS

%\wavenumber,\qm-curl Sobolev space
\newcommand{\ktcurlSob}[2]{H^1_{\wavenumber,\qm,\mathrm{curl}}\bracs{#1, \mathrm{d}#2}}
%\wavenumber,\qm-curl, divergence-free Sobolev space
\newcommand{\ktcurlSobDivFree}[2]{\mathcal{H}^{\kt}\bracs{#1, \mathrm{d}#2}}	
				
%H^2(domain, measure), only really used for Leb. measure
\newcommand{\gradgradSob}[2]{H^2_\mathrm{grad}\bracs{#1, \mathrm{d}#2}}  %maths commands, variables, and other packages

%labelling hacks
\newcommand\labelthis{\addtocounter{equation}{1}\tag{\theequation}}

%can use the \includeonly{} command to specify which chapters to update and include in the document, to save on rendering time. This will also retain equation/section numbering from skipped sections too.
\includeonly{
%./Chapters/Introduction/Intro,
%./Chapters/Theory-Prelims/Theory-Prelims,
%./Chapters/Scalar-System/Scalar-System,
%./Chapters/Curl-Curl/Curl-Curl,
./Chapters/SingInclusions/SingInclusions,
}

\begin{document}

%Title page --- provided the information sent to baththesis is correct, this should auto-generate
\maketitle

%Number pages prior to the start of the thesis with roman numerals
\pagenumbering{roman}

%Licensing and authorship declaration comes next --- setting are in the baththesis style
\License

\chapter*{Summary}
The study of periodic composite materials is highly relevant in the design of photonic crystal fibres.
Under specific relations between the material properties, one can demonstrate the emergence of spectral band gaps (as well as other metamaterial behaviour) which give rise to frequencies that do not support any modes of light in the crystal.
These frequency gaps --- band gaps --- can then be exploited in the design of waveguides.
Similar contrast effects have also been shown to be the result of geometric contrast on so-called thin-structure domains.

Singular structures, and composite domains with one of the composite materials being singular, represent an intuitive visual limit of the domains described above.
However it is another matter to determine whether a given system of equations on a composite domain converges (in some sense) to a system of equations on such a singular structure.
The present methods of establishing such convergence results rely on having an \emph{a priori} guess of the expected limiting equations and structure to hand.
Whilst there has been some success in establishing convergence results of this kind for various materials exhibiting contrasts, there has been little progress in the context of electromagnetism.

The work conducted in this thesis aims to answer and explore a number of questions raised by the above.
We examine variational problems posed on singular structures, the form of which are motivated by the equations of electromagnetism.
Such problems constitute very intuitive analogues of their thin-structure or composite-domain counterparts, however require us to establish lower-dimensional analogues of the gradient, curl, and divergence operators --- the study of which comprises much of our analysis.
However through this analysis we are able to realise such variational problems as more familiar and tractable problems, which we can then look to solve either analytically or numerically.
These variational problems also provide candidates for the effective problems of their thin-structure counterparts, potentially aiding in establishing the aforementioned convergence results.  

We first focus on the study of the acoustic approximation on a singular structure represented by an embedded graph; demonstrating that it can be realised as a quantum graph problem, which coincides with the known limit of analogous equations on shrinking thin structures.
The geometric contrast between the edges and vertices of our singular structure that is encoded into the singular measures we use to setup our variational problem also emerges explicitly in the resulting quantum graph problem.
Furthermore, we also demonstrate that the resulting problem can be readily analysed and solved through use of the $M$-matrix, and provide an explicit form for it for any underlying graph.
This analysis then serves as motivation for us to study both the curl-of-the-curl equation on a singular structure, and the acoustic approximation on a composite domain containing singular inclusions.
In both cases we will to setup a variational problem on a singular structure representing a physical material under contrast, reduce it to a more tractable problem (ideally on the underlying graph), and suggest and implement numerical methods for solving it.


\chapter*{Acknowledgements}
Write your acknowledgements here

%Create the contents
\tableofcontents

\cleardoublepage %ensure that page numbering is correct
\addcontentsline{toc}{chapter}{\listfigurename}
\listoffigures %caption[list of figures text]{in body text} can be used to customise the text provided here

%Switch to arabic numbering for research content
\cleardoublepage
\pagenumbering{arabic}

%Introductory chapter begins
\chapter{Introduction}
\tstk{make this title more focused when the time comes.}

\tstk{Introductory chapter, containing the motivation, literature review, and overview of the work carried out and presented.
Also establishes the story of the thesis, and the organisation of content.}

\section{Physical Motivation} \label{sec:PhysMot}
Optical fibres are the \textit{de facto} industry standard for large telecommunications systems, thanks to their ability to transmit information quickly and with far less signal loss than other methods (such as metal cables).
The technology has rapidly developed since the first optical fibres were fabricated in the 1970s \cite{knight2003photonic} and optical fibres in use today present a balance between several competing factors to deliver a reliable performance.
Some of these factors such as (optical) loss are inherent, brought about by the materials needed to build the fibres,
Other factors can be influenced by the fibres design (group-velocity dispersion) or fabrication process, which can lead to imperfections and polarisation effects.
The common design of an optical fibre will consist of a ``core" made of a dielectric (non-conducting) material with a given refractive index, surrounded by ``cladding", another dielectric material of a lower refractive index.
In practice this is normally achieved by choosing a material for the core and then using a doped version of said material for the cladding, with silica being a common choice, which leads to typical differences in refractive indices of the core and cladding of around $0.001$.
By ensuring the cladding material has a lower refractive index than the core material, modes of light\footnote{A mode of light is a mono-frequency solution to the governing equations of electromagnetism in the fibre.} can be confined to the core of an optical fibre via the phenomenon of Total Internal Reflection (TIR), illustrated in figure \ref{fig:Diagram_OpticalFibre}.
\begin{figure}[h]
	\centering
	\includegraphics[scale=1.0]{Diagram_OpticalFibre.pdf}
	\caption{\label{fig:Diagram_OpticalFibre} A schematic diagram of a weakly-guiding optical fibre. A core is surrounded by a cladding, giving a cross section composed of concentric circles. Light propagates along the fibre axis, confined to the core by the process of TIR. Note: this is illustrated with a ray picture of light.}
\end{figure} 
One can imagine light entering the core of the fibre at one end, and (totally internally) reflecting off the boundaries of core and cladding as it moves along the fibre (although this intuition uses the ray description of light rather than the wave description).
Wave guidance in fibres using TIR is known as weak guiding, and typically allows light to be propagated over tens of kilometres before a signal boost is required.
However despite the majority of modern optical fibres utilising this method of guidance, all improvements to the technology have been incremental and largely centre around the manufacturing process. \newline

The first significant departure from this traditional setup came with the realisation that the ability to 
structure a material on the same scale as the wavelength of optical light will drastically alter its optical properties.
With the advancement of fabrication techniques to allow for these structures to be physically manufactured, the development and study of materials known as ``photonic crystals" begun in earnest.
These photonic crystals are materials composed of a periodic microstructure (on the aforementioned length scale), typically a periodic arrangement (``matrix") of one dielectric material within another.
They can be used as cladding for a traditional optical fibre in the same manner as described above, by encasing a core material of a higher refractive index within the photonic crystal --- such fibres are referred to as ``photonic crystal fibres" (PCFs).
However more radical designs for PCFs utilise the fact that, depending on their frequency (or equivalently wavelength), certain modes of light cannot propagate through a given photonic crystal.
Due to the inhomogeneity of the photonic crystal, waves propagating (or attempting to propagate) within it undergo multiple scattering and are subject to interference.
At certain frequencies, a wave attempting to propagate through the crystal may be subject to cumulative destructive inference off the inclusions, effectively forbidding wave propagation within the crystal at that frequency.
These intervals or ranges of frequencies at which light cannot propagate within the crystal are referred to as \emph{photonic} or \emph{frequency band gaps}, and their compliment as \emph{bands}.
These photonic band gaps offer an alternative method of ``guidance", rather than having light undergo TIR within a core material, instead these modes can be confined to the core by virtue of having a frequency within one of these band gaps.
Since such a mode of light cannot propagate in the surrounding crystal, it is confined, and can propagate along the core (provided its propagation \emph{is} supported by the core).
From these ideas came several new designs for PCFs, the first being the design and manufacture of ``hollow-core"\footnote{The name coming from the fact that the first such fibres were typically made from a photonic crystal of glass periodically punctured with air holes, surrounding a larger air cavity that played the role of the core. The core consisting of air rather than a solid dielectric material gave rise to the term ``hollow".} fibres schematically illustrated in figure \ref{fig:Diagram_PCF-AirHollowConfine.pdf}.
\begin{figure}[b!]
	\centering
	\begin{subfigure}[t]{0.45\textwidth}
		\centering
		\includegraphics[scale=0.5]{Diagram_PCF-AirHollowConfine.pdf}
		\caption{\label{fig:Diagram_PCF-AirHollowConfine.pdf} A hollow-core fibre formed from a material (typically a glass) punctured with air holes.}
	\end{subfigure}
	~
	\begin{subfigure}[t]{0.45\textwidth}
		\centering
		\includegraphics[scale=0.5]{Diagram_PCF-LowIndexConfine.pdf}
		\caption{\label{fig:Diagram_PCF-LowIndexConfine.pdf} An all-solid PCF, whose core is a defect in the periodic structure.}
	\end{subfigure}
	\caption{\label{fig:Diagram_PCF} Schematic illustration of the cross-section of a PCF. A photonic crystal surrounds a core, whose bandgaps confine certain modes of light to the cores, and support their propagation down the fibre. Darker colours represent higher refractive indices.}
\end{figure}
These PCFs confine light of a given frequency to this core at the centre, using the surrounding periodic arrangement of low-index inclusions.
However, this band-gap confinement can also be used to confine light to a low refractive index core via the placement of higher-index inclusions in the crystal (in figure \ref{fig:Diagram_PCF-LowIndexConfine.pdf}), in ``solid-core" PCFs --- see for example the study \cite{luan2004allsolid}.
Further investigations into PCF designs have even shown confinement and propagation of light using metallic reflection \cite{hou2008metallic}.
Importantly, all of these PCF designs do not utilise TIR to ensure that light is confined to the core, breaking tradition with established optical fibre technology \cite{knight2003photonic, russell2003photonic}.

The physical theory underlying the process by which they operate means that PCFs have the potential to replace traditional weakly guiding optical fibres as the industry standard.
PCFs have an advantage over conventional optical fibres in that their applications are not limited to telecommunications, with alternative applications including non-linear optics (where they offer high optical intensities per unit power, making them highly efficient) and particle guidance (dielectric particles can be guided by the dipole forces exerted by light).
Knowing how a photonic crystals band gaps depend on its geometry and microstructure is of great importance in the design process, as these gaps determine the range of frequencies at which the fibre can operate as a waveguide.
Hence there has been much to motivate study of the optical properties of PCFs; and understanding this interplay serves as the motivation for the study of the systems considered in this work, albeit these systems represent an approximation to these 2D photonic crystals.

\subsection{Exisiting Models for PCFs} \label{ssec:ExistingPCFModels}
Before moving on to introduce the systems to be studied in this work, we briefly outline some of the existing modelling techniques for PCFs.
The most conceptually straightforward of these is the use of numerical techniques to solve Maxwell's equations to determine modes that are supported by the PCF.
These modes can then be used to construct an approximate band-gap plot for the fibre, with the level of precision in the computational results coming at the cost of increasing the computing time.
Whilst this can provide detailed information about a fibre, it does not provide any insight into how the structure of the photonic crystal has contributed or affected the resulting band-gap plot.
Furthermore numerical schemes (particularly those based off finite elements) are known to require special treatment when being used to solve Maxwell's equations, but this in itself has been studied in detail (see for example \cite{monk2003finite}).
We will attempt to steer clear of these kinds of solvers, primarily because they will not be directly applicable to the (variational) problems we will be considering (section \ref{sec:TP-DomainSetup}), and the aforementioned lack of insight into the dependence of the band-gap structure on the underlying fibre geometry. \newline

There has been some success in developing approximate models for wave guidance in PCFs that retain key features of experimental band-gap plots.
Typically one reduces (by means of an appropriate transform or use of Bloch's theorem) the analysis of the governing equations for wave guidance in the photonic crystal to that on a period cell, and reassembles the band-gap structure through the analysis of the family of problems on the unit cell.
The ARROW model \tstk{Litchinitser, N., Dunn, S., Steinvurzel, P., Eggleton, B., White, T., McPhedran,
R., and De Sterke, C. Application of an arrow model for designing tunable photonic
devices. Opt. Express 12, (8) (2004), 15401550. Also references within \cite{birks2006approximate}} has been highly effective at explaining confinement of light to the core in certain solid-core fibres through anti-resonant reflections off the higher-index inclusions.
Models such as \cite{birks2006approximate}, simplify the geometry of the period cell and solve the (scalar) wave equation in the resulting structure, reporting good agreement with numerical solvers for the types of PCF the approximate model considers.
Other works such as \cite{birks2004scaling} look to understand how band-gaps for hollow-core fibres scale with the contrast in refractive index.
Such approaches useful when speed is more important than high accuracy, or when an intuitive picture of the band-structure is desired.
\tstk{now Cooper's pre-print too? not entirely sure how to transition here}
\tstk{also, with mention of ARROW, now's probably a good time to mention resonance for the first time!}

\section{The Maxwell Equations and Derived Systems} \label{sec:Intro-Maxwell}
Wave propagation in electromagnetic contexts is governed by the system of Maxwell equations;
\begin{align*}
	\grad\cdot \mathbf{D} = \rho_f, &\qquad
	\curl{}\mathbf{E} = -\pdiff{\mathbf{B}}{t}, \\
	\grad\cdot \mathbf{B} = 0, &\qquad
	 \curl{}\mathbf{H} = J_f + \pdiff{\mathbf{D}}{t},
\end{align*}
where the vector fields $\mathbf{E}$, $\mathbf{D}$, $\mathbf{H}$, and $\mathbf{B}$ represent (respectively) the electric field, electric displacement field, magnetic field, and magnetic induction field, and the functions $\rho_f$ and $J_f$ are the (free) electric charge density and (free) electric current density respectively \cite{jackson1999classical, cessenat1996mathematical}.
This system is incomplete without constitutive relations informing us how $\mathbf{D}$ and $\mathbf{B}$ depend on $\mathbf{E}$ and $\mathbf{H}$, and we will concern ourselves with the linear approximations
\begin{align*}
	\mathbf{D} = \epsilon_m \mathbf{E}, \qquad \mathbf{B} = \mu_{m}\mathbf{H},
\end{align*}
where $\epsilon_m$ (respectively $\mu_m$) is the electric permittivity (magnetic permeability) of the material.
Further to our consideration of photonic crystals, we also treat $\epsilon_m$ and $\mu_m$ as time-independent, scalar-valued functions of position $x$, which is suitable for studying inhomogeneous, isotropic media such as photonic crystals\footnote{For a more general material the constitutive relations can be much more complex, potentially being non-linear and spatially varying. 
Other effects such as hysteresis in ferromagnets can introduce time dependencies, whilst Lorentz materials have material parameters that depend on the frequency of incident electromagnetic radiation.}.
Under these constitutive relations, and in the absence of free charges and currents, the Maxwell system reduces to
\begin{align*}
	\grad\cdot \epsilon_m \mathbf{E} = 0, &\qquad
	\curl{}\mathbf{E} = -\mu_m\pdiff{\mathbf{H}}{t}, \\
	\grad\cdot \mu_m \mathbf{H} = 0, &\qquad
	\curl{}\mathbf{H} = \epsilon_m \pdiff{\mathbf{E}}{t}.
\end{align*}
One can then seek time-harmonic solutions (by taking a Fourier transform in time), 
\begin{align*}
	\mathbf{E}\bracs{x,t} = E\bracs{x}e^{-\rmi\omega t},
	&\qquad \mathbf{H}\bracs{x,t} = H\bracs{x}e^{-\rmi\omega t},
\end{align*}
obtaining system
\begin{align*}
	\grad\cdot \epsilon_m \mathbf{E} = 0,
	&\qquad \recip{\rmi\mu_m}\curl{}E = \omega H, \\ 
	\grad\cdot \mu_m \mathbf{H} = 0,
	&\qquad -\recip{\rmi\epsilon_m}\curl{}H = \omega E,
\end{align*}
as a result.
The equations involving the curl can be written in matrix form, and correspond to the spectral problem for the ``operator" 
\begin{align*}
	\mathcal{M} &:=
	\begin{pmatrix}
		0 & -\recip{\rmi\epsilon_m}\curl{} \\
		\recip{\rmi\mu_m}\curl{} & 0
	\end{pmatrix}.
\end{align*}
We will come to call $\mathcal{M}$ the Maxwell operator, however we need to be slightly careful in defining it so that we respect the divergence-free conditions, and end up with a self-adjoint operator.
Typically this not difficult if the domain we are considering is smooth (or the whole of $\reals^d$), even the regularity of $\epsilon_m$ and $\mu_m$ tends not to be an issue in these contexts.
However there is significantly more work that needs to be done for domains with (possibly non-smooth) metallic inclusions or boundaries, with the work of \cite{birman1987l2, birman1989selfadjoint} providing a detailed study of the Maxwell operator in these contexts.
For the purposes of this review, we can define the Maxwell operator $\mathcal{M}$ by assigning it the domain
\begin{align*}
	\dom{\mathcal{M}} = \left\{ \bracs{E,H} \ \middle\vert \right. 
	&
	\left. E\in L^2\bracs{\ddom, \epsilon_m\md x}^3, \ H\in L^2\bracs{\ddom, \mu_m\md x}^3, \right. \\
	&
	\left. \curl{}E, \ \curl{}H\in\ltwo{\ddom}{x}^3, \right. \\
	&
	\left. \grad\cdot \epsilon_m E = 0, \ \grad\cdot \mu_m H = 0 \right\},
\end{align*}
where the derivatives are understood in the weak (or distributional) sense, and $\ddom\subset\reals^3$ is our domain.
The weighted spaces $L^2\bracs{\ddom, w\md x}^3$ consist of the square-integrable functions $f:\reals^3\rightarrow\complex^3$ equipped with the norm
\begin{align*}
	\norm{ f }_{L^2\bracs{\ddom, w\md x}^3} &= \integral{\ddom}{\abs{f(x)}^2 w(x)}{x}.
\end{align*}
Under this setup, the Maxwell operator $\mathcal{M}$ is self-adjoint and the \emph{Maxwell system} is the spectral problem for this operator.
The spectrum of $\mathcal{M}$ determines the frequencies of light that support modes, with any intervals that are absent from the spectrum of $\mathcal{M}$ corresponding to band gaps.

Even with the Maxwell operator being self-adjoint, it is still not elliptic and has eigenvalues that extend across the whole real line (that is, to both $\pm\infty$), making direct analysis of it difficult.
To tackle this, $\mathcal{M}$ can be applied to itself to produce a positive-definite operator whose action decouples the $E$ and $H$ fields, resulting in the \emph{curl-of-the-curl} spectral problems
\begin{subequations} \label{eq:Intro-CurlCurlEqns}
	\begin{align}
		\epsilon_m^{-1}\curl{}\bracs{\mu_m^{-1}\curl{}E} &= \omega^2 E, \\
		\mu_m^{-1}\curl{}\bracs{\epsilon_m^{-1}\curl{}H} &= \omega^2 H,
	\end{align}
\end{subequations}
the eigenvalues $\omega^2$ of either problem then determining the eigenvalues of $\mathcal{M}$.
A further reduction is possible if the material properties of the medium are independent of one of the coordinate directions (say $x_3$), so $\epsilon_m(x)=\epsilon_m(x_1,x_2)$ and $\mu_m(x)=\mu_m(x_1,x_2)$ only.
One can then one can consider waves propagating ``in the $\bracs{x_1,x_2}$-plane" by taking a Fourier transform in the $x_3$ direction and setting the corresponding propagation constant equal to zero.
Upon doing so, the action of $\mathcal{M}$ decouples into separate actions on the transverse electric (TM) field $\bracs{E_1,E_2,0,0,0,H}^\top$ and transverse magnetic (TM) field $\bracs{0,0,E_3,H_1,H_2,0}^\top$, yielding the equations
\begin{subequations} \label{eq:Intro-AcousticApprox}
	\begin{align}
		-\mu_m^{-1}\grad\cdot\epsilon_m^{-1}\grad E_3(x_1,x_2) &= \omega^2 E_3(x_1,x_2), \\
		-\epsilon_m^{-1}\grad\cdot\mu_m^{-1}\grad H_3(x_1,x_2) &= \omega^2 H_3(x_1,x_2).
	\end{align}
\end{subequations}
These equations are be referred to as \emph{acoustic approximations}, as they also appear when studying acoustic waves in periodic media.

Having provided an outline of the important governing equations (and corresponding operators) that describe wave propagation in PCs, we move on to a brief discussion of the structure of the eigenvalues of differential equations with periodic coefficients.
An excellent introduction to this topic (with a particular emphasis on applications to PCs) is provided in the survey \cite{kuchment2001mathematics}, and we will go into more details in section \ref{sec:TP-GelfandTransform}.
Let us suppose we have an operator $\mathcal{A}$ with periodic coefficients\footnote{Think of $\mathcal{A}$ as being one of \eqref{eq:Intro-CurlCurlEqns} or \eqref{eq:Intro-AcousticApprox} with $\epsilon_m$ and $\mu_m$ periodic.} with respect to a period cell $\ddom=[0,1)^d$.
A natural transform to apply is the Gelfand transform, which first requires us to introduce the dual cell (or Brillouin zone) $B=[-\pi,\pi)^d$ to $\ddom$.
Then we can define the Gelfand transform $\gelfand u$ of a function $u$ as 
\begin{align*}
	\gelfand u(x, \qm) &= \sum_{n\in\integers^d} u(x+n)\e^{\rmi\qm(x+n)},
\end{align*}
where the variable $\qm$ is called the \emph{quasi-momentum}, the analogue of the dual variable in the Fourier transform.
The Gelfand transform allows us to obtain the spectrum of $\mathcal{A}$ from the spectrum of each member of a family of operators $\mathcal{A}_{\qm}$ parametrised by the quasi-momentum.
Importantly, each $\mathcal{A}_{\qm}$ acts on the period cell (which is a compact domain) rather than the whole of $\reals^d$ that the original medium fills, so under ellipticity assumptions the $\mathcal{A}_{\qm}$ will possess discrete spectra.
This allows us to order the eigenvalues $\lambda_j\bracs{\qm}$, $j\in\naturals$, of $\mathcal{A}_{\qm}$ in ascending order in $j$.
The (continuous) functions $\lambda_j$ of $\qm$ are called \emph{dispersion branches}, \emph{dispersion relations}, or \emph{(spectral) band functions}.
It holds that the spectrum of $\mathcal{A}$, $\sigma\bracs{\mathcal{A}}$, is equal to the union of the spectra of the $\mathcal{A}_\qm$.
Equivalently, the projection of the graphs $y=\lambda_j(\qm)$ onto the $y$ axis provides the spectrum of $\mathcal{A}$.
Explicitly, we have that
\begin{align*}
	\sigma\bracs{\mathcal{A}} &= \bigcup_{\qm\in B} \sigma\bracs{\mathcal{A}_{\qm}}
	= \bigcup_{j\in\naturals} \lambda_j\bracs{B}
	= \bigcup_{j\in\naturals} \sqbracs{ \min_{\qm\in B}\lambda_j, \max_{\qm\in B}\lambda_j },
\end{align*}
the final equality coming from the fact that the branches $\lambda_j$ are continuous functions of $\qm$.
The interval $\sqbracs{ \min_{\qm}\lambda_j, \max_{\qm}\lambda_j }$ is often referred to as the $j^{\text{th}}$ \emph{spectral band} of $\mathcal{A}$.
This highlights where the band-gap structure of these composite materials comes from, whenever 
\begin{align} \label{eq:Intro-DispersionBranchGapIneq}
	\max_{\qm}\lambda_{j-1} < \min_{\qm}\lambda_{j},
\end{align}
there is a corresponding gap between the end of the $(j-1)^{\text{th}}$ band and the beginning of the $j^{\text{th}}$.
It is known that for second-order ordinary differential equations \eqref{eq:Intro-DispersionBranchGapIneq} holds with a non-strict inequality, so the spectral bands cannot overlap but may touch \cite[chapter XIII]{reed1978iv}, so it is conceivable that alterations to the material may open up gaps between adjacent bands.
In two dimensions (or higher) the spectral bands can (and usually do) overlap, which makes the task of designing a material with band gaps harder though not impossible.

\subsection{Non-dimensionalisation of the Maxwell System} \label{ssec:Intro-NonDimMax}
We now turn specifically towards domains that model PCs, the process of non-dimensionalising the governing equations, and the various free parameters that emerge from this process.
Regrettably, this process is often skipped over in the mathematical literature, with most works electing to start from a non-dimensionalised system.
Whilst there is nothing wrong with choosing such a starting point, there are several sets of non-dimensional parameters that one can choose to express the resulting system in, and it is important to keep track of what each set represents when interpreting the physical behaviours the non-dimensionalised system is describing.
This can also lead to some rather confusing (and seemingly conflicting) language in the literature, so we provide an explicit account of this process.

A periodic medium can be modelled by specifying a period cell $Q=[0,L)^d$, $L>0$, and then taking a union of translated copies of this period cell
\begin{align*}
	\bigcup_{n\in\integers^d} (Q + nL),
\end{align*}
to fill the whole of $\reals^d$.
Note that the requirement that $Q$ be (hyper-) cubic in shape is not restrictive, as long as the material is periodic in $d$ linearly independent directions one can apply a linear transform to produce a (hyper-) cubic period cell.
For composite media such as PCs, $Q$ is then further divided into two regions, the \emph{inclusions} $Q_0$ with $\overline{Q_0}\subset Q$ and the \emph{bulk} (also called \emph{matrix} or \emph{background}) $Q_1:=Q\setminus \overline{Q_0}$.
The electric permittivity $\epsilon_m$ and magnetic permeability $\mu_m$ of the medium are then defined as the $Q$-periodic functions with
\begin{align*}
	\epsilon_m(x) = \begin{cases} \epsilon^{\mathrm{inc}} & x\in Q_0, \\ \epsilon^{\mathrm{bulk}} & x\in Q_1, \end{cases}
	\qquad
	\mu_m(x) = \begin{cases} \mu^{\mathrm{inc}} & x\in Q_0, \\ \mu^{\mathrm{bulk}} & x\in Q_1. \end{cases}
\end{align*}
The materials that make up PCs tend to have similar (if not identical) permeabilities, so one typically adopts the assumption that the materials are non-magnetic and takes $\mu^{\mathrm{inc}}$ and $\mu^{\mathrm{bulk}}$ to both be equal to the permeability of free space, $\muFS$.
We also assign $Q_0$ a characteristic size $L-l$, $0<l<L$, giving the period cell illustrated in figure \ref{fig:Diagram_ScalingDimensionfull}.
\begin{figure}[b!]
	\centering
	\begin{subfigure}[t]{0.45\textwidth}
		\centering
		\includegraphics[scale=1.0]{Diagram_ScalingDimensionfull.pdf}
		\caption{\label{fig:Diagram_ScalingDimensionfull} Schematic illustration of the period cell of a non-magnetic, periodic, composite medium, prior to non-dimensionalisation.}
	\end{subfigure}
	~
	\begin{subfigure}[t]{0.45\textwidth}
		\centering
		\includegraphics[scale=1.0]{Diagram_ScalingDimensionless.pdf}
		\caption[Illustration of the relationship between physical properties and dimensionless parameters for composite media.]{\label{fig:Diagram_ScalingDimensionless} The period cell of the domain of the non-dimensionalised problem.}
	\end{subfigure}
	\caption[Dimensionfull and dimensionless (non-magnetic) composite domains.]{\label{fig:Diagram_ScalingND} Dimensionfull and dimensionless (non-magnetic) composite domains.}
\end{figure}

We now proceed to write the system of Maxwell equations with respect to dimensionless variables.
Let $\lambda_0$ be the wavelength in vacuum of some light incident onto the structure described by $\ddom$, and define (the reference frequency in vacuum as) $\omega_0 = \lambda_0^{-1}\bracs{\epsFS\muFS}^{-\recip{2}}$.
Introduce the dimensionless electric and magnetic fields $\tilde{E}$ and $\tilde{H}$ and corresponding dimensionless spatial variable $\tilde{x}$ by
\begin{align*}
	H = \hat{h}\tilde{H}, 
	\qquad E = \hat{h}\bracs{\frac{\muFS}{\epsFS}}^{\recip{2}}\tilde{E},
	\qquad x = \lambda_0\tilde{x},
\end{align*}
where $\hat{h}$ is a constant with the units of magnetic field (Amperes per metre, $\mathrm{A}\mathrm{m}^{-1}$), and $\epsFS$ the permittivity of free space.
The Maxwell system then becomes
\begin{align*}
	\curl{\tilde{x}}\tilde{E} 
	&= \rmi\omega\lambda_0\epsFS^{\recip{2}}\mu_m\muFS^{-\recip{2}} \tilde{H}, \\
	\curl{\tilde{x}}\tilde{H} 
	&= -\rmi\omega\lambda_0\muFS^{\recip{2}}\eps_m\epsFS^{-\recip{2}} \tilde{E}.
\end{align*}
Defining the (non-dimensional) frequency $z := \omega_0^{-1}\omega = \omega\lambda_0\bracs{\epsFS\muFS}^{\recip{2}}$ and dimensionless permittivity $\epsilon_{r}=\epsFS^{-1}\epsilon_m$, we have the system
\begin{align} \label{eq:Intro-NonDimMaxwell}
	\curl{\tilde{x}}\tilde{E} 
	&= \rmi z \tilde{H}, \\
	\epsilon^{-1}_{r} \ \curl{\tilde{x}}\tilde{H} 
	&= -\rmi z \tilde{E}.
\end{align}
This results in the domain setup illustrated in figure \ref{fig:Diagram_ScalingDimensionless}: we are now studying the (dimensionless) equations \eqref{eq:Intro-NonDimMaxwell} for $\tilde{x}\in\reals^d$, with a period cell $\widetilde{Q} = \lambda_0^{-1}Q$ of size $\tilde{L}:=\lambda_0^{-1}L$, a bulk region $\widetilde{Q}_1 = \lambda_0^{-1}Q_1$ of size $\delta:=\lambda_0^{-1}l$, an inclusion $\widetilde{Q}_0 = \lambda_0^{-1}Q_0$ of size $\tilde{L}-\delta$, and dimensionless (or ``relative") permittivity
\begin{align*}
	\epsilon_{r}\bracs{\tilde{x}} = 
	\begin{cases} 
		\epsFS^{-1}\epsilon^{\mathrm{inc}} & \tilde{x}\in\widetilde{Q}_0, \\
		\epsFS^{-1}\epsilon^{\mathrm{bulk}} & \tilde{x}\in\widetilde{Q}_1.
	\end{cases}
\end{align*}
From here, one can derive the dimensionless curl-of-the-curl equations and acoustic approximation.
The values of the (dimensionless) parameters
\begin{align*}
	\tilde{L} = \lambda_0^{-1}L, \quad
	\delta = \lambda_0^{-1}l, \quad
	\epsFS^{-1}\epsilon^{\mathrm{inc}}, \quad
	\epsFS^{-1}\epsilon^{\mathrm{bulk}},
\end{align*}
characterise the various behaviours that the system can exhibit.

Now we highlight that our choice of non-dimensionalisation is far from unique, and there are several alternatives one could choose from, arriving at a different set of dimensionless parameters --- however in each case the number of (independent) parameters would be the same.
One other selection of dimensionless parameters worth highlighting are the following;
\begin{align} \label{eq:Intro-NonDimLengthScales}
	\tilde{L} = \lambda_0^{-1}L, \quad
	\delta = \lambda_0^{-1}l, \quad
	\lambda^{\mathrm{inc}} := \lambda_0\omega_0\bracs{\epsilon^{\mathrm{inc}}\muFS}^{-\recip{2}}, \quad
	\lambda^{\mathrm{bulk}} := \lambda_0\omega_0\bracs{\epsilon^{\mathrm{bulk}}\muFS}^{-\recip{2}},
\end{align}
where each of the dimensionless parameters are now ratios of lengths.
The parameters $\lambda^{\mathrm{inc}}$ and $\lambda^{\mathrm{bulk}}$ (respectively) represent the ratio of the wavelength of light in the inclusion (bulk) to that in free space, and satisfy $\epsilon_{r} = \lambda_{r}^{-2}$.
The equivalent non-dimensional system would then be
\begin{subequations} \label{eq:Intro-NonDimMaxwellLengths}
	\begin{align}
		\curl{}\tilde{E} &= -\rmi z\tilde{H}, \\
		\lambda_{r}^2\curl{}\tilde{H} &= \rmi z\tilde{E},
	\end{align}
\end{subequations}
where
\begin{align*}
	\lambda_r\bracs{\tilde{x}} = 
	\begin{cases} 
		\lambda^{\mathrm{inc}} & \tilde{x}\in\widetilde{Q}_0, \\
		\lambda^{\mathrm{bulk}} & \tilde{x}\in\widetilde{Q}_1.
	\end{cases}
\end{align*}
Having all the dimensionless parameters represent ratios of lengths is typically done to make the description of the phenomenons of \emph{resonance} and \emph{critical contrast} more intuitive, which we elaborate on in section \ref{ssec:Intro-CritContrast}.
However as mentioned previously, in the mathematical literature the non-dimensional Maxwell system (under the non-magnetic assumption) \eqref{eq:Intro-NonDimMaxwellLengths} is simply introduced as \emph{the} Maxwell system, and one typically sees the notation ``$\eps^{-1}$" for $\lambda_r^2$ in \eqref{eq:Intro-NonDimMaxwellLengths}.
This can lead to some confusion on the part of a new researcher in the field, as it is common to see such an ``$\eps$" used (and treated) implicitly as both the (reciprocal square) ratio of lengths $\lambda_r^2$ (as it appears in \eqref{eq:Intro-NonDimMaxwellLengths}) and as the relative permittivity $\epsilon$ (as it appears in \eqref{eq:Intro-NonDimMaxwell}).
As such, studies may talk about $\eps$ as if it were a ratio of length scales, whilst others may refer to seemingly the same quantity as a contrast between material permittivites.

The benefits of non-dimensionalising are that all possible behaviours of the Maxwell system are characterised by the values these dimensionless parameters take, and the (non-dimensional) spectrum is qualitatively representative of the physical frequencies, which can be obtained through ``re-dimensionalising" the spectral parameter $z$.
It is natural to now consider the various (combinations of) asymptotic limits of these free parameters and the resulting behaviours they describe.
Typically these asymptotic (or effective) problems and can reveal effects that are otherwise hard to notice, and are easier to work with analytically and numerically when compared to the original system.
A discussion of these asymptotic limits and their relation to \emph{resonance} and homogenisation of so called \emph{critical contrast} materials is the focal point of section \ref{ssec:Intro-CritContrast}.

Upon completing this section, we drop the overhead tilde notation on the dimensionless electric and magnetic fields and spatial variables for brevity, and adopt $\omega$ as the symbol for the non-dimensional spectral parameter over $z$.
We will also use the symbol $u$ to represent (one of) the fields $E$ or $H$ in the curl-of-the-curl equation, and $E_3$ or $H_3$ in the acoustic approximation.
We now move on to a description of the various effects one can expect from asymptotic regimes of these dimensionless parameters.

\section{Asymptotic Limits and Dispersive Effects} \label{sec:AsymptoticStudies}
We have introduced a typical domain setup and set of dimensionless parameters that one might consider when studying (electromagnetic) wave propagation in PCs.
Now we turn our attention to the behaviour of such waves, and the structure of the corresponding spectra and band-gaps, in various regimes of these parameters.
There is an important distinction for us to make here between geometric and material properties.
Using \eqref{eq:Intro-NonDimMaxwellLengths} as an example, the parameters $\tilde{L}, \delta$ pertain to a description of the geometry of the PC --- changing these parameters does not alter the properties intrinsic to the materials that make up the composite, only their relative size within the period cell compared to the wavelength of incident light.
Conversely, the bulk and inclusion wavelengths correspond to material properties of the PC and how these interact with incoming light --- changing these corresponds to changing the physical materials that the PC is made from, but does not change the arrangement of the materials within the period cell itself.
This also provides a natural split between the various regimes that can be considered; fixed material properties with the geometry varying, fixed geometry with varying permittivities, or simultaneous limits in both.
The former consideration takes us to the realm of thin structures shrinking to singular structures, and the quantum graph problems that describe the limiting behaviour, which is the topic of section \ref{ssec:Intro-ThinStructures}.
To study the other regimes, we are naturally guided us towards the techniques of homogenisation theory to derive ``limiting" problems and study the resulting effects that emerge, as a result of the contrast between the dimensionless parameters.

% include the discussion on thin structures
\subsection{Thin Structures in the $\delta\rightarrow0$ limit} \label{ssec:Intro-ThinStructures}

Graphically, the ``limit" $\delta\rightarrow0$ in these domain setups is rather intuitive --- one can visualise the region $Q_1$ becoming increasingly fine, getting closer to a collection of connected line segments as $\delta$ decreases to 0.
This raises the natural question as to whether it is reasonable (or even possible) to appeal to this visual intuition and approximate a photonic crystal as a \emph{singular structure} embedded into a 2 or 3 dimensional matrix.
Figure \ref{fig:Diagram_ShrinkToSingularSquareCell} illustrates this process for a 2-dimensional geometry that is extruded into 3 dimensions.
\begin{figure}[h]
	\centering
	\includegraphics[scale=0.5]{Diagram_ShrinkToSingularSquareCell.pdf}
	\caption{\label{fig:Diagram_ShrinkToSingularSquareCell} A visual illustration of the $\delta\rightarrow0$ limit in a ``fibre-like" geometry. The cross sectional geometry (dark cross) shrinks to a collection of line segments, resulting in a union of planes whose cross-section is a graph-like structure.}
\end{figure}
The resulting ``singular structure" is a region embedded into a space in which it has no volume (or area in two dimensions); in the illustration the resulting planes have no volume from the perspective of $\reals^3$.
If we remain true to our photonic crystal setup, the rest of $\reals^d$ upon taking the ``limit" $\delta\rightarrow0$ is taken up by the ``inclusion" $Q_0$.
However, relevant to our studies in \tstk{sections} will be the study of so called \emph{thin-structures}.
Such thin-structures $G_{\delta}$ consist only of the region (formed by translations of) $Q_1$, that is $G_{\delta} = \bigcup_{l\in\integers^d}(l+Q_1)$.
In almost all contexts one is typically looking at a thin-structure domain that is akin to a ``thickened graph"; one takes a graph $\graph$, embeds it into $\reals^d$, and inflates the edges into tubes of radius $\sim\delta$ and the vertices into junctions regions connecting the ends of these tubes.
In the interest of avoiding addressing several technicalities and detracting from the focus of our review, one can think of $G_{\delta}$ as being the surface of (or volume enclosed by) the resulting ``inflated" structure --- the illustration in figure \ref{fig:Diagram_ThickenedGraph} provides a sufficient visualisation.
\begin{figure}[t]
	\centering
	\includegraphics[scale=1.0]{Diagram_ThickenedGraph.pdf}
	\caption{\label{fig:Diagram_ThickenedGraph} A schematic illustration of a thickened graph $G_{\delta}$, consisting of tubes whose central axes are the edges of the underlying graph $\graph$, meeting at inflated vertex regions. The graph $\graph$ consists of the dashed edges and connecting vertex.}
\end{figure}

By design, the thin-structure $G_{\delta}$ converges in the eyeball norm as $\delta\rightarrow0$ to the (metric) graph $\graph$, whose edges are correspond to the central axes of the tubes and whose vertices correspond to the centre of the collapsing junction regions.
Since $\graph$ is embedded into real space, each edge $I_{jk}$ which connects the vertex $v_j$ to $v_k$ can be bestowed a length $l_{jk}>0$ and a corresponding interval $\sqbracs{0,l_{jk}}$ where we associate $0\in\sqbracs{0,l_{jk}}$ to $v_j$ and $l_{jk}\in\sqbracs{0,l_{jk}}$ to $v_k$.
A function defined on a quantum graph $u$ is then specified by its restriction to each of the $I_{jk}$, denoted by $u^{(jk)}$, and the values $u$ takes at the vertices of the graph.
With this notion of length, one can now define a derivative for $u$ along the edges, and thus can equip $\graph$ with (the analogue of) a differential operator.
Since the intervals associated to the edges $I_{jk}$ do not convey the connectivity of $\graph$, the ``boundary conditions" for these differential operators come in the form of matching conditions at the vertices.
A graph equipped with such an operator is called a \emph{quantum graph}, and the system of equations and ``boundary conditions" defined by this operator a \emph{quantum graph problem}.
A more precise description and introduction is provided in section \tstk{section ref}, and a more complete overview of the field can be found in \cite{berkolaiko2013introduction}, however it is sufficient for this introduction to think of such a quantum graph problem as consisting of an ODE on each of the intervals associated to $I_{jk}$, coupled through matching conditions at the vertices.
Standard Neumann or Dirichlet conditions on $u$ at each vertex can be used as the vertex conditions, however these tend to neglect (or to an extent, suppress) the underlying graph structure, which allows for more adventurous conditions to be used.
More common choices of vertex conditions are continuity of the function $u$ though the vertices\footnote{That is whenever two edges $I_{jk}$ and $I_{kl}$ share a common vertex, the restrictions $u^{(jk)}$ and $u^{(kl)}$ must take the same value at the shared vertex $v_k$.} and the Kirchoff condition,
\begin{align*}
	\sum_{v_k \text{ connects to } v_j} 
	\pdiff{u^{(jk)}}{n}\bracs{v_j} = \alpha_j u(v_j),
\end{align*}
where $\pdiff{u^{(jk)}}{n}\bracs{v_j}$ denotes the \emph{incoming} derivative to $v_j$, and $\alpha_j$ is a (``coupling") constant. 
The Kirchoff condition corresponds relates the flux at each vertex to the function value, when the coupling constant (or function value) at $v_j$ is zero, it's form and interpretation is analogous to Kirchoff's first law of ``net zero current" through junctions in electric circuits.

The use of quantum graphs as approximations to physical processes and the study of the spectra of the associated operators has a rather rich history thanks \tstk{get refs for this, Kuch-Zheng is a good place to start} to interests from mesoscopic physics\footnote{Broadly speaking, this is a branch of condensed matter physics focusing on materials whose size ranges from a few molecules or atoms to micrometres.}.
\tstk{Although we do not delve into the details, such quantum graph problems are known to describe thin superconducting structures called ``quantum wires", ``molecular wires", and free-electron theory of conjugated molecules.}
In each of these contexts there is difficulty in identifying the \emph{correct} quantum graph problem. Whilst the graph itself is easily identifiable, the resulting vertex conditions are not due to complications in coming from the behaviour of the solutions within the junction regions of $G_{\delta}$ as $\delta$ decreases.
Heuristic arguments and physical intuition led to the standard practice of imposing continuity and Kirchoff conditions as the vertex conditions in the ``approximating" quantum graphs for these applications.
These long-standing assumptions seemed justified with the study of \tstk{\cite{kuchment2001convergence} themselves extending J. Rubinstein and M. Schatzman}, which established convergence of the spectrum of the Neumann laplacian on $G_{\delta}$ (which we will denote by $\mathcal{G}_{\delta}$) to the spectrum of a quantum graph problem (defined by an operator we denote by $\mathcal{G}$).
Precisely, the result
\begin{align*}
	\lim_{\delta\rightarrow0} \lambda_n\bracs{\mathcal{G}_{\delta}} = \lambda_n\bracs{\mathcal{G}}
\end{align*}
we proved for every $n\in\naturals$, where $\lambda_n\bracs{\mathcal{G}_{\delta}}$ and $\lambda_n\bracs{\mathcal{G}}$ are the eigenvalues of the respective operators ordered in ascending order.
The argument that was utilised revolved around representing the eigenvalues of $\mathcal{G}_{\delta}$ and $\mathcal{G}$ via the minimax principle, and a method of translating functions on $G_{\delta}$ to $\graph$ and back so that the increase in the Rayleigh quotient could be controlled.
Convergence of the resolvent was proved in \tstk{Satio - in \cite{exner2005convergence}, [27]th reference} when the underlying graph had no loops or cycles.
In particular, the vertex conditions associated with $\mathcal{G}$ were continuity of $u$ at the vertices of $\graph$, and a Kirchoff condition at each vertex\footnote{This is a simplification in the interest of providing the reader with an overall idea of the key concepts. The study \cite{kuchment2001convergence} allowed for a slightly more general, weighted domain $G_{\delta}$ which results in a weighted sum in the Kirchoff condition, and also allows the presence of an external field.}, justifying the heuristic arguments that had come before.

However, the study \cite{exner2005convergence} later demonstrated that this was not the complete story.
Starting from the domain\footnote{The study \cite{exner2005convergence} actually considers the more general setup where one is working with manifolds, and differential operators on these manifolds. Here, we present the results in a manner contextualized to our review.} $G_{\delta}$, one can define the volume of the inflated edges $V_{\mathrm{edge}}$ as a function of the thickness $\delta$ and volume of the junction regions $V_{\mathrm{vertex}}$, which also scales with $\delta$.
In the limit $\delta\rightarrow0$, the spectrum of $\mathcal{G}_{\delta}$ coincides with the spectrum of an operator $\tilde{\mathcal{G}}$ which depends on the relative scaling between $V_{\mathrm{vertex}}$ and $V_{\mathrm{edge}}$.
\begin{itemize}
	\item (``Thick vertex" setup) If $V_{\mathrm{edge}}\ll V_{\mathrm{vertex}}\rightarrow0$ as $\delta\rightarrow0$, then $\tilde{\mathcal{G}}$ defines a quantum graph problem with (homogeneous) Dirichlet conditions at each of the vertices.
	This case is referred to as Dirichlet decoupling; the resulting limit problem is just a collection of independent ODEs along the edges of the graph $\graph$, intuitively the result of the vertices being ``too big" when compared to the edges and thus preventing any interaction between the edges.
	\item (``Thick edge" setup) If $V_{\mathrm{vertex}}\ll V_{\mathrm{edge}}\rightarrow0$ as $\delta\rightarrow0$, the edges dominate in the limit problem and one obtains $\tilde{\mathcal{G}} = \mathcal{G}$ from the study \cite{kuchment2001convergence}.
	Intuitively, since the vertices ``disappear" before the edges as $\delta\rightarrow0$, the resulting problem (and it's solutions) must have strict matching conditions where the edges meet --- resulting in continuity and a net flux of 0 (Kirchoff condition) at each vertex.
	\item (``Borderline/critical" case) Of particular interest is when $\frac{V_{\mathrm{edge}}}{V_{\mathrm{vertex}}}\rightarrow c>0$ as $\delta\rightarrow0$, that is when the volume of the tubes and junctions are of the same order of $\delta$.
	This is an intermediary for the other two cases, the effect of decoupling the edges (large junctions) is balanced by the need for consistency between edges (large tubes).
	The spectral problem for the resulting operator $\tilde{\mathcal{G}}$ can be shown to define a quantum graph problem whose vertex conditions are continuity at each of the vertices, but along with a ``non-standard" Kirchoff condition at the vertices,
	\begin{align*}
	\sum_{v_k \text{ connects to } v_j} 
	\pdiff{u^{(jk)}}{n}\bracs{v_j} = \lambda\alpha_j u(v_j).
	\end{align*}
	Here, $\alpha_j$ is the aforementioned coupling constant and $\lambda$ is the \emph{spectral parameter} for $\tilde{\mathcal{G}}$ --- in physical terms, one can say that the strength of any coupling at a vertex is proportional to the system's energy.
	Mathematically, the operator $\tilde{\mathcal{G}}$ is said to have a generalised resolvent --- we elaborate on this in what follows.
\end{itemize}
Of particular interest is the operator $\tilde{\mathcal{G}}$ in the borderline case; it is not an operator on a quantum graph per say, but its spectral problem $\tilde{\mathcal{G}}\lambda = \lambda u$ can be interpreted as a quantum graph problem with the spectral parameter appearing in the boundary (or vertex) conditions.
More precisely, the operator $\tilde{\mathcal{G}}$ acts in an ``extended" (or ``larger") function space, rather than the standard function spaces for functions on metric graphs, \tstk{and is said to possess a ``generalised resolvent" - define this, what is it?}
The quantum graph problem with the non-standard Kirchoff condition belongs to the class of problems with generalised resolvents, \tstk{the refs for this?}, and the operator $\tilde{\mathcal{G}}$ is the so-called ``Strauss extension" or ``dilation" of this problem.
\tstk{as we will see, our starting point addresses multiple issues here: we can ``start" from an intuitive operator and problem to arrive at this non-standard QG problem. Also, this makes the job of ``guessing" the limiting problem much easier!}

To conclude our discussion of the literature, we highlight that at present there are no results which extend the studies \cite{kuchment2001convergence, exner2005convergence} into the Maxwell setting (that is, to either the curl-of-the-curl equation or the full Maxwell system).
By contrast, there has been considerable interest in the study of the equations of elasticity on thin-structures and the resulting ``singular" limits.
The work \cite{zhikov2002homogenization} studied the problem of homogenisation for periodic thin-structure of thickness $\delta$, looking to derive the resulting effective problem in the ``long-wavelength" regime.
In this work determination of the effective problem in the so-called critical scaling regime $\delta\sim\eps$ of the thickness $\delta$ with the period $\eps$ was noted as being an open problem, with the resulting homogenised operator being different to that in the non-critical scaling regimes.
This work was followed up in \cite{zhikov2003homogenization}, in which these differences were identified and the effective problem in the critical setting presented.
A method for passing to the limit in the equations of elasticity on a bounded thin-structure domain as the thickness $\delta\rightarrow0$ was also developed in \cite{zhikov2006derivation}, the resulting system being a quantum graph problem with Kirchoff matching conditions at each of the vertices.
We highlight these results because, akin to \cite{kuchment2001convergence}, these studies do not account for the possible ``borderline" scaling case --- only the edge thickness $\delta$ is present in the studies.
For this reason it is likely that results similar to those presented by \cite{exner2005convergence} can be derived within the context of elasticity too, however this remains unexplored in the literature.

The work of chapters \tstk{1 and 2} will be particularly relevant to the aforementioned studies and the ``limiting" quantum graph problems.
In these chapters we will take the ``limiting" singular structure as our starting point, and pose a variational problem upon this singular structure.
This will require us to setup and study appropriate function spaces \tstk{section ref}, before then demonstrating the relationship between the abstract variational problem and the various ``limiting" quantum graph problems discussed above --- including the generalised resolvent problem.
We will also go into detail about how one can explicitly solve the resulting quantum graph problems, before looking to extend our results into the Maxwell setting, and indicating some of the challenges that this brings.
The theory we establish in these chapters will also be valuable to us in chapter \tstk{composite measure}, when we come to study a ``photonic crystal with singular inclusions".

% now begin the discussion of critical contrast
\subsection{Critical, or High Contrast Composites} \label{ssec:Intro-CritContrast}
We have seen that there are a variety of effective problems that one can obtain from the $\delta\rightarrow0$ limit of a thin structure in section \ref{ssec:Intro-ThinStructures}.
Now we put the shoe on the other foot, and focus on the various behaviours that can emerge when considering asymptotic limits of the material properties (or rather, limits of the dimensionless quantities that are derived from them).
The studies discussed here typically concern periodic, composite materials whose domains are described in the manner introduced in section \ref{ssec:Intro-NonDimMax}, with the size of the inclusion $\delta$ fixed.
Such domains are more reflective of physical photonic crystals than the thin domains considered previously in section \ref{ssec:Intro-ThinStructures}.
Despite our interest being primarily in singular structures and the resulting spectra of operators on them, it is nonetheless important for us to highlight how asymptotic models for materials at ``critical-contrast" can also give rise to dispersive effects.
Indeed, we shall see that the dispersive effects that arise from these models mirror those that emerge from limits of thin structures.

Studies of such materials typically rely on techniques from homogenisation theory to determine ``effective" material properties in the asymptotic limit of interest.
For example, consider the operator $\mathcal{A}=-\grad\cdot A\grad$ on such a composite domain --- the (possibly matrix-valued) function $A$ describes the ``non-dimensional" material parameters, and throughout we will assume it is $Q$-periodic.
One regime of interest is the ``long wave" (or low frequency) regime, where one attempts to garner information about the lowest spectral band of a composite material.
Accordingly, the (small) parameter $\eps\ll 1$ is introduced and we consider the (sequence of) operators $\mathcal{A}^{\eps} = -\grad\cdot A\bracs{\eps^{-1}x}\grad$ in the limit $\eps\rightarrow0$.
We then derive an ``effective" operator $\mathcal{A}^{\mathrm{hom}} = -\grad\cdot A^{\mathrm{hom}}\grad$ that describes the resulting limit, and look to formalise the relationship between $\mathcal{A}^{\mathrm{hom}}$ and the $\mathcal{A}^{\eps}$.
The matrix $A^{\mathrm{hom}}$ is sometimes labelled or interpreted as the ``effective" material properties.
Typically one desires convergence of $\mathcal{A}^{\eps}$ to $\mathcal{A}^{\mathrm{hom}}$ as $\eps\rightarrow0$ in the norm-resolvent sense, namely that the family of solutions $u_{\eps}$ to the problems $\mathcal{A}^{\eps}u_{\eps}=f_{\eps}$ converges to the solution $u$ of the effective problem $\mathcal{A}^{\mathrm{hom}}u = f$, for any suitable initial data $f_{\eps}\rightarrow f$.
However one may also establish weaker results, for example coincidence of the spectrum of $\mathcal{A}^{\mathrm{hom}}$ with the limit of $\sigma\bracs{\mathcal{A}^{\eps}}$ (akin to the results for thin structures discussed in section \ref{ssec:Intro-ThinStructures}) which would otherwise be implied by norm-resolvent convergence.
This convergence in turn ensures that the effective problem faithfully describes the behaviour of the composite material within this regime, and one typically hopes that the effective problem is more mailable to analysis than its counterpart.
These studies can also reveal interesting material effects within such regimes that might otherwise be lost or go unnoticed.
The design of materials to open spectral band gaps is one such effect that is desired, however other dispersive effects such as metamaterial behaviour \tstk{ref?} --- where the effective material properties appear to take unphysical values --- can also emerge from these limits.
In these settings, one is typically expecting to obtain an effective problem where the effective material properties $A^{\mathrm{hom}}$ depend non-trivially on the frequency variable $\omega$.

The asymptotic treatment of composite materials has seen considerable interest due to the aforementioned motivations.
However, it has been known to both the mathematical and physical community for some time that the toolbox of standard homogenisation theory is unable to capture the correct asymptotic behaviour for composite materials.
It has been reported \tstk{proved, no? Reported in K-E-K-Nab Unified Approach, pointing to the references: M. Sh. Birman and T. A. Suslina, Second order periodic differential operators. Threshold properties and homogenization, St. Petersburg Math. J. 15 (2004), no. 5, 639–714 and V. V. Zhikov, Spectral approach to asymptotic diffusion problems. (Russian), Differentsial’nye uravneniya 25 (1989), no. 1, 44–50. } in the uniformly elliptic setting, that is where the matrix $\mathcal{A}(\eps^{-1}x)$ and its inverse are uniformly bounded, that the resulting effective problem has $A^{\mathrm{hom}}$ being a constant matrix, which leaves no room for time dispersive effects.
If one drops the assumption of uniform ellipticity, the resulting analysis becomes more complicated and the standard toolbox of asymptotic study is no longer sufficient for describing the effective problem.
Indeed the inadequacy of standard techniques in such contexts has been reported in works such as \cite{nicorovici1995photonic} \tstk{later paper implies this is part of a series of works though}, which highlighted ``non-commuting limits" when considering wave scattering off a periodic composite material.
In such a problem there are two scales to consider; the wavenumber of the incident wave $\wavenumber$ and the refractive index of the inclusions within the composite, $N$ (which is proportional to the square root of the relative electric permittivity of the inclusions, to maintain a connection to the dimensionless quantities introduced in section \ref{sec:Intro-Maxwell}).
In order to study the first spectral band\footnote{This is also referred to as the lowest spectral band, or the ``acoustic band" in the literature.} in the case when the inclusions are highly conducting; we can either consider the case when the inclusions are perfectly conducting ($N\rightarrow\infty$) for fixed $\wavenumber$ and then take the ``static-limit" $\wavenumber\rightarrow0$, \emph{or} consider the static limit ($\wavenumber\rightarrow0$) at a high but finite refractive index and then take $N\rightarrow\infty$.
However the two approaches produce different effective materials --- the explanation provided in \cite{movchan2001noncommuting} is that the trajectory in $\bracs{N^{-1},\wavenumber}$-space that one takes as these values approach the origin must be of the form $N^{-1}=\wavenumber^p$, with $p<1$ for homogenisation to be satisfactory, otherwise the problem being considered is ``singularly perturbed" and boundary layer effects emerge near the inclusion-matrix interface which are not handled by the homogenisation process.
However, there have been a number of developments aimed at expanding the tools of homogenisation theory to handle contexts such as the above, where the material parameters scale in particular relation to the size of an incident wave or the size of the period cell.

Let us begin with the study of \cite{zhikov2000extension}, that was concerned with the homogenisation problem for the operator $\mathcal{A}^{\eps}$,
\begin{align*}
	\mathcal{A}^{\eps} := -\grad\cdot A^{\eps}(\eps^{-1}x)\grad,
\end{align*}
in the limit as $\eps\rightarrow0$, where $A^{\eps}(y)=\eps^2$ on $Q_0$ and $1$ on $Q_1$\footnote{This setup corresponds to a periodic structure with low-index inclusions, as opposed to the high-index inclusions considered in \cite{movchan2001noncommuting} \tstk{and other McPhedran papers?}.}, see the domain on the left in figure \ref{fig:Diagram_HL-Zhikov}.
\begin{figure}[b]
	\centering
	\includegraphics[scale=1.0]{Diagram_HL-Zhikov.pdf}
	\caption[Relation between materials under critical contrast and at resonance.]{\label{fig:Diagram_HL-Zhikov} A material under critical contrast (left) as studied in \cite{zhikov2000extension}, and a material at resonance studied in \cite{hempel2000spectral} (right).}
\end{figure}
Materials such as this are said to be under ``critical contrast"; the (dimensionless) material properties scale in certain proportion to the size of the period cell $\eps$, analogous to (although not exactly) the relation between refractive index and incident wavenumber mentioned earlier.
A limiting operator $\mathcal{A}$, and the convergence of the solutions to $\mathcal{A}^{\eps}u^{\eps}=f^{\eps}$ to the solution $u$ of $\mathcal{A}u=f$ is obtained --- however this convergence is in the so-called two-scale sense rather than the standard ``norm-resolvent" sense of classical homogenisation.
The spectrum of the limiting operator $\mathcal{A}$ is also shown to be the Hausdorff limit of the spectra of the operators $\mathcal{A}^{\eps}$ as $\eps\rightarrow0$, and demonstrates dissipative properties.
It is argued (and demonstrated) that two-scale convergence ensures that not only do the solutions $u^{\eps}$ converge to $u$, but the energy of the system also converges (in this two-scale sense) to the energy of the effective problem --- something which is not always guaranteed by classical homogenisation in contexts such as this. \tstk{isn't there also a study that does this in the scattering context - would be good to have this since McPhedran was doing scattering too? G. Bouchitté and D. Felbacq, Homogenization near resonances and artificial magnetism from dielectrics, C. R. Math. Acad. Sci. Paris 339 (2004), no. 5, 377–382.}
Furthermore, it is also demonstrated that if $A^{\eps}(y)\propto\eps^p$ on $Q_0$ for $p<2$, the effective problem coincides with that obtained through classical homogenisation, whilst for $p>2$ the effective problem is again the ``two-scale limit" of the original problem.
This allows us to obtain an effective problem for materials under critical contrast, where previous homogenisation techniques fell short --- however the convergence obtained is not the standard norm-resolvent convergence we are used to from classical homogenisation.

In a related development, the study \cite{hempel2000spectral} proved the band-gap structure of the resulting spectrum, although starting from a different domain, illustrated in right of figure \ref{fig:Diagram_HL-Zhikov}.
The period cell and inclusion are of a fixed size in this study, and the operator considered is $\widetilde{\mathcal{A}}^\sigma:=-\grad\cdot\tilde{A}\grad$ is studied\footnote{The study \cite{hempel2000spectral} actually concerns a slightly more general version of this operator.} where $\widetilde{A}=1$ on $Q_0$ and $\widetilde{A}=\sigma\gg 1$ on $Q_1$.
The asymptotic limit $\kappa\rightarrow\infty$ was then explored, with the deduction that the endpoints of the spectral bands correspond to the Dirchlet and (the limits of the) Neumann eigenvalues of the Laplacian on the inclusions $Q_0$.
Additional results about the density of states were also proven, the main takeaway being that the spectrum concentrates near the right-endpoints of the bands as $\sigma$ increases. 
These results are also proved in \cite{friedlander2002density} under slightly stricter assumptions concerning smoothness of the boundary of the inclusion $Q_0$, but providing an asymptotic form for the integrated density of states. 
This asymptotic form for the integrated density of states is further refined in \cite{selden2005periodic}, in the setup of \cite{friedlander2002density}. 
Upon identifying $\sigma=\eps^{-2}$ (so $\sigma\rightarrow\infty \Leftrightarrow \eps\rightarrow0$) it is clear why the effective problems for $\mathcal{A}^{\eps}$ and $\widetilde{\mathcal{A}}^{\sigma}$ (under the respective limits) possess the same spectra, although it should be noted that this is one of the only common features of the two problems --- in most other respects they behave differently.
This association between $\mathcal{A}^{\eps}$ and $\widetilde{\mathcal{A}}^{\sigma}$ also proves a good time for us to introduce another term (or concept) that is often thrown around in the literature and hard to nail down --- that of resonance.
We can observe that, with the size of the period cell being unity in the setup of \cite{hempel2000spectral}, the material considered cannot be under critical contrast.
However, this material is under (or at) \emph{resonance} --- exactly one of the non-dimensional parameters describing this material ($\sigma$) is large, whilst the others are of order unity.
The term ``resonance" typically brings with it connotations from more physically minded contexts, one things of resonance as a phenomenon where the frequency of a periodic driving force is close to the natural frequency of the system on which it acts, causing a large spike in the amplitude of the response of the system.
This intuition still holds; observe that in terms of the dimensionless parameters \eqref{eq:Intro-NonDimLengthScales} we have $\delta\in\bracs{0,1}$, $\hat{l}^{\mathrm{inc}}=1$, and $\hat{l}^{\mathrm{bulk}}=\sigma\gg 1$, the wavelength within the bulk $\lambda^{\mathrm{bulk}}$ has been ``driven" in response to the periodic structure of the material.
Through these two studies one can see a correspondence between materials ``under resonance" and those at ``critical contrast", both phenomena can result in the emergence of band-gaps or other dissipative effects, despite seeming to model different material setups.

The studies discussed above, whilst illustrative and helpful in demonstrating how it is possible for dissipative effects to emerge, still fall short of definitively establishing these effects as a direct consequence of being under critical contrast (or at resonance).
The work of \cite{cherednichenko2019unified, cherednichenko2019time} attempts to rectify this, looking to provide a more general theory analogous to that which already exists for the moderate (non-critical) contrast setting.
An key observation made in these works is that models with frequency dependent boundary conditions can be extended to conservative systems \tstk{via the analysis of generalised resolvents --- link to where this was first mentioned}) , and these ``extended" systems are precisely the asymptotic limits of models of high (or critical) contrast media.
We have seen (section \ref{ssec:Intro-ThinStructures}) that such frequency-dependent problems also emerge in the asymptotic limit of thin structure problems as the thickness of the structure tends to zero \tstk{Kirill-Sasha's Kronig-Penney diople paper even highlights this link right?}.
This would seem to imply that the same asymptotic properties, and thus dispersive effects such as metamaterial behaviour or existence of band gaps, can be obtained from either geometric contrast (manifesting in the correct vertex-to-edge volume scaling, section \ref{ssec:Intro-ThinStructures}) or from high contrast materials (that is, contrast between the material properties).
From a physical design perspective, having the ability to obtain such effects through two different means is helpful in contexts where one such contrast cannot be guaranteed, or is difficult to manufacture.
This correspondence also firmly establishes the utility of quantum graph problems as effective models for a variety of media.
Furthermore, the relevance of our singular structure problems now extends beyond applications to thin structures --- if geometric contrast can mirror the effects of material contrast, then insights from our singular structure problems will also be informative and useful in the critical contrast context.

% finally, talk about when both $\delta$ and $\eps$ are present, and lead into Research Overview
\subsection{Double limits in geometric and material parameters} \label{ssec:Intro-DoubleLimits}
So far we have considered the results of a shrinking domain thickness given fixed material properties (section \ref{ssec:Intro-ThinStructures}), and then various asymptotic limits of high contrast materials with fixed inclusion sizes (section \ref{ssec:Intro-CritContrast}). 
It is natural to ask whether any consideration has been put into considering the effects that might emerge when both the (relative) size $\delta$ of the region $Q_1$ shrinks to zero \emph{and} the contrast in the material parameters simultaneously increases to infinity.
The series of papers \cite{figotin1996band-scalar, figotin1996band-maxwell, figotin1998spectral} was dedicated to investigating the spectral properties that emerge in the aforementioned limits, for a variety of operators relevant to models of wave propagation.
This series begins with \cite{figotin1996band-scalar}, studying the acoustic equation on a non-magnetic composite material whose period cell $Q$ is the unit square in $\reals^2$, and whose inclusion $Q_0$ is a smaller square of side length $1-\delta$ --- the setup is illustrated in figure \ref{fig:Diagram_KF-DoubleLimitStudy}.
\begin{figure}[b!]
	\centering
	\includegraphics[scale=1.0]{Diagram_KF-DoubleLimitStudy.pdf}
	\caption[Illustration of the domains studied in \cite{figotin1996band-scalar, figotin1996band-maxwell, figotin1998spectral}, under simultaneous limits of geometric- and material parameters.]{\label{fig:Diagram_KF-DoubleLimitStudy} Illustration of the domains studied in the series \cite{figotin1996band-scalar, figotin1996band-maxwell, figotin1998spectral}. Simultaneous limits of the material parameters and geometric parameters were considered, although contrast between the edge and vertex volumes of the region $\widetilde{Q}_1$ were absent.}
\end{figure}
The region $Q_1$ is therefore shrinking as $\delta\rightarrow0$ (visually, to a singular structure) and possesses (relative) dielectric constant $\epsilon_r$ which is equal to $\eps$ on $Q_1$ and $1$ on $Q_0$.
For sufficiently small values of $\delta$, $\bracs{\delta\eps}^{-1}$, and $\eps\delta^2$, it is demonstrated that spectral gaps open, that the spectrum concentrates near the eigenvalues of the Neumann Laplacian on (the union of all periodic translations of) $Q_0$, and an estimate for the width of the spectral bands is also provided.
This was followed up \cite{figotin1996band-maxwell} by a consideration of the curl-of-the-curl equation in $\reals^3$, on a domain that was the extrusion into 3 dimensions of the 2D domain described previously.
In the asymptotic regime $\eps\delta^{\frac{3}{2}}\ll 1$, $\eps\delta\gg1$, gaps in the spectrum once again begin to open --- details on the structure of the spectrum and estimates regarding the width of the spectral bands are again provided.
Finally, analysis of the acoustic equation when the region $Q_0$ was a polygon \cite{figotin1998spectral} was considered, in the limit as $\delta\rightarrow0$, and $\bracs{\delta\eps}^{-1}\rightarrow W<\infty$.
Note that since the region $Q_0$ is a polygon, the region $Q_1$ can be thought of as a thickening of some graph $\graph$ embedded into $\reals^2$.
Under these assumptions, it is shown that the spectrum of the acoustic equation on such a medium converges to the spectrum of the problem
\begin{align} \label{eq:Intro-KuchFigQGLimit}
	-\laplacian u &= \lambda\bracs{\delta_{\graph} + W}u,
\end{align}
where $\lambda$ is the spectral parameter and $\delta_{\graph}$ is the delta function supporting the graph $\graph$.
Note that this convergence is not in the norm-resolvent sense, and only demonstrates coincidence of the limit of the acoustic equation as $\delta\rightarrow0$, $\bracs{\delta\eps}^{-1}\rightarrow W<\infty$ with the spectrum of the problem \eqref{eq:Intro-KuchFigQGLimit}.
This is of particular interest to our work in chapter \ref{ch:SingInc}, where we will consider a composite medium with one of the components being a singular structure, and look to obtain (and solve) a non-local quantum graph problem from our singular structure formulation.
A similar setup has been studied for materials at high contrast in the context of linear elasticity in \cite{cherednichenko2019homogenisation}; again the domain consists of a periodic structure with $Q_1$ composed of thin rods of width $\delta$ and whose material parameter $\lambda^{\mathrm{bulk}}=O(\delta^2)$, whilst $\lambda^{\mathrm{inc}}=O(1)$ in the surrounding inclusions $Q_0$.
This work establishes convergence of the solutions to the resolvent problem in the two-scale sense (introduced in \cite{zhikov2000extension}), and conducts further analysis to establish convergence of the spectra to that of the ``two-scale" limit operator.

None of the studies thus described not consider the possible relations between the vertex and edge volume of the ``thickened graph" region $Q_1$ as $\delta\rightarrow0$, and as a result such a geometric parameter is absent from the analysis and resulting equations.
Our approach via singular structures is motivated by such geometric contrasts, although we shall see that we can incorporate material contrast into the problems we study in chapter \ref{ch:SingInc}.
We also have a more geometrically motivated starting point in mind; our choice to start from a measure-theoretic formulation reflects the different length scales of the components of the ``limit" material, and we expect to see effects due to contrast emerge through the interactions of these length scales.

\section{Overview of Research Presented in this Thesis} \label{sec:Intro-ProblemIntroduction}
As we have alluded to sections \ref{ssec:Intro-ThinStructures}-\ref{ssec:Intro-DoubleLimits}, our interest is not in employing asymptotic techniques to study composite materials in various parameter regimes.
Instead, our interest is in studying problems involving singular structures, motivated by the ``visual limit" as the width of one of the components of a composite medium shrinks to zero.
Whilst we postpone the more involved and formal definitions to chapter \ref{ch:TheoryPrelims}; here we elaborate on the problems that we shall consider, our motivation for doing so, and their connection to the asymptotic studies we have already discussed.
Before we begin, let us establish some basic notation and concepts to enable our discussion.
A singular structure is a subset $S\subset\reals^d$ with $d$-dimensional Lebesgue measure zero, although more generally such structures can be described as a regions that have no area from the perspective of the space into which they are is embedded.
We will always choose to represent singular structures through a graph $\graph=\bracs{\vertSet,\edgeSet}$, where each vertex is associated with a point in $\reals^d$ and each edge to the line segment joining the (points associated to the) vertices at each endpoint.

Now that we have a singular structure, we then have to ask how we can define differential equations (or more precisely, operators) on them.
To motivate this thought process, let us take a singular structure (represented by a graph) $\graph$ and consider the more familiar setup of a differential equation complimented by boundary conditions on the corresponding thickened graph $G_{\delta}$.
For illustrative purposes, we will consider the system
\begin{subequations} \label{eq:Intro-ThinToSingularSys}
	\begin{align}
		-\grad\cdot\grad u &= f, \qquad x\in G_{\delta}, \label{eq:Intro-ThinToSingularDE} \\
		\pdiff{u}{n}\big\vert_{\partial G_{\delta}} &= 0, \label{eq:Intro-ThinToSingularBC}
	\end{align}
\end{subequations}
for twice-differentiable $u$ and square-integrable $f$.
Now consider which pieces of this formulation either change or cease to have clear meaning as the thickness $\delta$ shrinks to zero:
\begin{enumerate}
	\item The most notable effect is the loss of a distinction between the boundary $\partial G_{\delta}$ and the interior of $G_{\delta}$ when $\delta$ ``reaches" zero and becomes the singular structure $\graph$ --- the edges of this graph correspond to the overlap between what was previously two parts of the boundary of $G_{\delta}$.
	\item The graph $\graph$ itself has no interior from the perspective of $\reals^d$, differential equation \eqref{eq:Intro-ThinToSingularDE} becomes meaningless.
	Where or how do we pose this if our new domain has no interior?
\end{enumerate}
We can address our first observation by realising that it is only our explicit separation of the boundary condition \eqref{eq:Intro-ThinToSingularBC} and differential equation \eqref{eq:Intro-ThinToSingularDE} that leads to this problem.
Instead we can consider the variational formulation for \eqref{eq:Intro-ThinToSingularSys}, to find $u$ such that
\begin{align} \label{eq:Intro-ThinToSingularVar}
	\integral{G_{\delta}}{ \grad u\cdot \overline{\grad\phi} }{x} 
	&= \integral{G_{\delta}}{ f\overline{\phi} }{x}, \qquad\forall\phi\in V_{\mathrm{test}},
\end{align}
for some suitable set of test functions $V_{\mathrm{test}}$ (typically the set of smooth functions on the domain $G_{\delta}$).
A ``weak" formulation such as \eqref{eq:Intro-ThinToSingularVar} provides us with a formulation of \eqref{eq:Intro-ThinToSingularSys} where the boundary conditions are implicitly incorporated into the formulation --- they of course can be recovered from \eqref{eq:Intro-ThinToSingularDE} by considering the test functions whose support touches the boundary of $G_{\delta}$.
This allows us to circumnavigate the coincidence of the interior of our singular structure with what was previously the boundary of our thin structure, however we are not out of the woods yet.
If we simply replace the domain of integration $G_{\delta}$ with $\graph$ in \eqref{eq:Intro-ThinToSingularVar}, both sides of the equation are identically zero as a result of the singular nature of  $\graph$.
To address this we have to stop viewing our singular structure from the perspective of the Lebesgue measure, and instead in a manner that acknowledges that our structure is of a lower dimension to the surrounding space.
Further to this, the classical gradient is no longer appropriate for studying problems on singular structures --- there is difficulty in defining this object in the classical sense involving two-sided limits.
If we want to consider problems motivated by the Maxwell system, we run into similar problems regarding the curl and divergence of vector fields.
This leads us to introduce the notion of singular measures (section \ref{sec:SingularMeasures}) that support graphs embedded into $\reals^d$, and pose variational problems of the form \eqref{eq:Intro-ThinToSingularVar} with respect to these measures, and on the singular domain $\graph$.
Changing measure of course changes the notion of area, integration and thus differentiation too.
The variational problems we are able to define with these reworked concepts have a striking resemblance to their classical analogues --- in fact, the only differences being the measure in the variational problem and the ``Sobolev space" that our solution $u$ belongs to.
This motivates our study of the so-called ``non-classical" Sobolev spaces (section \ref{sec:BorelMeasSobSpaces}) and later generalisation to curls and divergences in chapter \ref{ch:CurlCurl} and to composite materials with singular regions in chapter \ref{ch:SingInc}.

Consideration of variational problems with respect to Borel measures is not an unexplored concept, although considerations for the explicit solution of such problems (both analytically and numerically) is largely untouched in the literature.
Works such as \cite{bouchitte2001homogenization, zhikov2000extension} introduced the notion of two-scale convergence with respect to Borel measures into the existing toolbox of homogenisation theory.
Other works have since begun studying variational problems with respect to measures, within the context of homogenisation problems in elasticity one can see the studies \cite{zhikov2002homogenization, zhikov2003homogenization, cherednichenko2019homogenisation}.
The studies \cite{cherednichenko2018operator, cherednichenko2022operator, cherednichenko2020order} have considered the homogenisation of problems within the context of electromagnetism, posed with respect to arbitrary (Borel) measures, and derived the appropriate effective problems.
For an example of an explicit study of a variational problem on a singular structure, the study \cite{zhikov2013spectrum} considers the spectrum of the operator of transverse displacements (within the context of elasticity), proving that it possesses a discrete spectrum.
However we reiterate that none of these studies touch upon how one might proceed with the explicit solution of the effective problem that is obtained --- whilst having such analytic properties are useful, the limiting models are not much use if they are as hard to solve as the problems they are approximating.
Furthermore, none of these studies consider setups in which there the structure shrinking to zero possesses contrast between its vertex and edge volumes.
These are two particular topics we look to address in chapters \ref{ch:ScalarSystem} and \ref{ch:SingInc}.

The singular structure problems we consider remain true to the idea that they coincide with the ``visual limit" of a composite domain with the thickness of one component shrinking to zero.
Our variational problems with respect to singular measures are motivated by their classical counterparts (which employ the Lebesgue measure), however the appropriate spaces and formulations are very easy to set up.
Appealing to analogy, it is natural to question whether (in the parameter regimes where the effective problem is known) these variational problems, or the behaviours they display, will coincide with those of the effective problems for composite media.
If this is the case, we can look to extend our approaches into situations where the analysis of zero-thickness limits has not been conducted.
We have already highlighted that such analysis has not been conducted for the curl-of-the-curl equation nor indeed the complete Maxwell system, whilst only spectral convergence for the acoustic approximation has been considered.
Singular structure problems thus offer us a potential tool to aid in obtaining candidates for effective problems in certain parameter regimes --- such candidates are required before one can begin to consider proving convergence results.
Finally, we also wish to explore how problems on singular structures can be explicitly solved (numerically or analytically) to recover the spectrum and eigenfunctions.
Whilst an valid topic to explore in its own right, this line of investigation also allows us to predict the behaviours that may emerge from the zero-thickness limits of thin-structures (and composite materials) should the former association prove fruitful.

We now outline the arrangement of the content of this work.
Concepts and standing assumptions that will be utilised across all chapters are detailed in chapter \ref{ch:TheoryPrelims}; notably the concepts of singular measures (section \ref{sec:SingularMeasures}), an overview of the Gelfand transform (section \ref{sec:TP-GelfandTransform}) and the ``shifted" gradient operators, an introduction to ``non-classical" Sobolev spaces (section \ref{sec:BorelMeasSobSpaces}), quantum graphs (section \ref{sec:QuantumGraphs}) and the notation we shall employ for handling singular structures.

Chapter \ref{ch:ScalarSystem} looks to firmly establish the connection between singular structure problems and quantum graph problems that are the limit of thin structure problems, in the sense of the discussion in section \ref{ssec:Intro-ThinStructures}.
We shall consider the variational formulation \eqref{eq:SingularScalarWaveEqn-VariationalForm} motivated by the acoustic approximation \eqref{eq:Intro-AcousticApprox}; and through analysis of the aforementioned non-classical Sobolev spaces (conducted in section \ref{sec:3DGradSobSpaces} and summarised in section \ref{ssec:3DGradGeometric}), demonstrate its coincidence with to the quantum graph problem \eqref{eq:SingularWaveEqnQGProblem} in theorem \ref{thm:ScalarDerivation-Theorem}.
The system \eqref{eq:SingularWaveEqnQGProblem} possesses Wentzell vertex conditions like those obtained in the zero thickness limit of thin-structures, and the connection between the non-classical Sobolev spaces and the extended spaces introduced in section \ref{ssec:Intro-ThinStructures} is made explicit in the discussion of section \ref{ssec:ExtendedSpaces}.
Finally we provide explicit expressions for the entries of the $M$-matrix in proposition \ref{prop:M-MatrixEntries} and discuss how it can be utilised in explicit solution of the aforementioned quantum graph problem, complimenting this with some examples in section \ref{sec:ScalarExamples}.
This will provide solidify our approach via singular measures as a basis for us to extend to more complex situations --- namely to the curl-of-the-curl equation and composite media.

The objective of chapter \ref{ch:CurlCurl} is to use the foundations established in chapter \ref{ch:ScalarSystem} to further explore the analogue of the curl-of-the-curl equation \eqref{eq:Intro-CurlCurlEqns} on singular structures.
We will again establish a variational problem on a singular structure (equation \eqref{eq:SingularCurlEquation-VariationalForm}) and similarly to chapter \ref{ch:ScalarSystem}, look to derive an alternative problem which we can then look to solve or analyse.
This requires us to extend our previous analysis concerning the definition of appropriate ``gradient-like" objects from chapter \ref{ch:ScalarSystem} to incorporate curls of vector fields, the analysis of which is conducted in section \ref{sec:CC-CurlAnalysis} and summarised in section \ref{sec:CC-Geometric}.
We also highlight that all vector fields are curls of zero when considering graph-like singular structures embedded into 3-dimensional space (proposition \ref{prop:3DGraph-CurlsAreZero}).
Section \ref{sec:DivFreeCondition} discusses the implications of imposing a divergence-free condition on vector fields, and in particular characterises the properties of divergence-free fields through proposition \ref{prop:DivFree-AllGradsConditions}.
Further examination reveals that two of these properties are equivalent to orthogonality to gradients of zero, whilst the remaining properties are bestowed by orthogonality to tangential gradients (corollary \ref{cory:DivFree-TangGradsConditions}).
In principle, the variational problem \eqref{eq:SingularCurlEquation-VariationalForm} can provide a candidate for the ``limit" of a sequence of thin structure problems for the curl-of-the-curl equation, analogously to those which are currently known for the acoustic approximation.
We are looking to bridge the existing gap in the theory concerning the behaviour of the equations of electromagnetism on thin structures as $\delta\rightarrow0$, which was highlighted in section \ref{ssec:Intro-ThinStructures} and the convergence results presently unexplored.
However we shall observe (theorem \ref{thm:CurlCurlReduction-Theorem}) that the singular analogue of the curl-of-the-curl equation in fact reduces to that of the acoustic approximation, the implications of which we discuss in section \ref{sec:CC-Discussion}.

Finally, in chapter \ref{ch:SingInc} we consider a composite medium consisting of a background material interlaced with a singular ``skeleton", again studying a variational problem (equation \eqref{eq:SI-WeakWaveEqn}) representing the acoustic approximation.
This moves us away from considerations of the limits of thin structures, and towards the setting of PCs and the ``limits" of the domains in section \ref{sec:Intro-Maxwell}, where the inclusion material is shrinking to a singular structure, and there is contrast between the vertex and edge regions of the shrinking structure.
Following our analysis of the appropriate function spaces (the focus of section \ref{sec:CompSobSpaces}, whose key results are summarised in section \ref{sec:SI-ProblemFormulation}), we explore several equivalent forms of the problem \eqref{eq:SI-WeakWaveEqn} as \eqref{eq:SI-VarProb} (section \ref{sec:SI-VarProbMethod}) and later \eqref{eq:SI-StrongForm} (section \ref{sec:SI-StrongDerivation}), attempting to again reduce the variational problem to a problem more tractable to solution (theorem \ref{thm:SingInc-DerivationTheorem}).
Our final push will leave us with the quantum graph problem \eqref{eq:SI-NonLocalQG} --- although in this case, the resulting operator is non-local.
We also suggest and implement a couple of numerical schemes for an example geometry; directly from the in-max principle through the problem \ref{prob:DiscVarProb}, via the ``strong formulation" \eqref{eq:SI-StrongForm} in section \ref{ssec:SI-FDMMethod}, and a method solely on the singular skeleton section \ref{ssec:SI-GraphMethod}, to explore the potential behaviours of our original problem \eqref{eq:SI-WeakWaveEqn}.

%Theory and Preliminaries chapter begins
\chapter{Preliminaries and Theory} \label{ch:TheoryPrelims}
\tstk{make this title more focused when the time comes.}

Purpose of this chapter is largely to introduce the two major tools we employ (plus any more that the singular-inclusion chapter/1st order Maxwell throws up) --- Quantum Graphs (and the M-matrix) and the singular structure spaces/definitions.

In hindsight, this section will likely be made into two chapters (or at least, multiple sections) to accomodate the quantum graph preliminaries.
The Sob Space stuff isn't as bloated, since our interest in a general measure quickly pales once we establish our measures $\dddmes, \ddmes, \nu$.

\section{Quantum Graphs}

some stuff about quantum graphs :)

\section{Singular Measures} \label{sec:SingularMeasures}
Now that we have established how we can represent singular structures through embedded graphs, we move on to the introduction of the singular measures which will allow us to pose meaningful variational problems on these structures.
The introduction of these measures also necessitates a change in the way we view gradients, curls, and divergences, which is the subject of section \ref{sec:BorelMeasSobSpaces}, and is our solution to the problems discussed in section \ref{sec:Intro-ProblemIntroduction}.

Let $\ddom\subset\reals^d$, and let $\mathcal{B}_{\ddom}$ denote the Borel sigma algebra for $\ddom$ (with the topology inherited from $\reals^d$).
A (Borel) measure $\rho$ on the measurable space $\bracs{\ddom,\mathcal{B}_{\ddom}}$ is called \emph{singular} if there exist two disjoint sets $S_1,S_2\in \mathcal{B}_{\ddom}$ such that $\rho$ is zero on all (measurable) subsets of $S_1$ and the $d$-dimensional Lebesgue measure $\lambda_d$ is zero on all subsets of $S_2$.
The textbook example of a singular measure is the point-mass (or ``$\delta$-function") measure placed at a point $x_0\in\ddom$,
\begin{align*}
	\delta_{x_0}\bracs{B} &= \begin{cases} 1 & x_0\in B, \\ 0 & x_0\not\in B, \end{cases}
	\qquad \forall B\in\mathcal{B}_{\ddom},
\end{align*}
for which we can take $S_1=\ddom\setminus\clbracs{x_0}$ and $S_2=\clbracs{x_0}$.
In addition to point masses, we will also want to consider singular measures which support our singular structures.
Let $\graph=\bracs{\vertSet, \edgeSet}$ be a metric graph embedded into $\reals^d$.
For each $I_{jk}\in\edgeSet$ define the (Borel) measure $\lambda_{jk}$ as the measure which supports the one-dimensional measure $\lambda_1$ on (or ``along") $I_{jk}$,
\begin{align*}
	\lambda_{jk}\bracs{B} = \lambda_{1}\bracs{r_{jk}^{-1}\bracs{B \cap I_{jk}}},
	&\quad\text{for all Borel } B.
\end{align*}
Here, $r_{jk}$ is the parametrisation of the edge $I_{jk}$ as per assumption \ref{ass:MeasTheoryProblemSetup}.
The measure $\lambda_{jk}$ will be referred to as the \emph{singular measure that supports $I_{jk}$}, or just the \emph{singular measure on $I_{jk}$}.
Since the sum of two singular measures is another singular measure, we can then define the following singular measures;
\begin{align*}
	\ddmes\bracs{B} = \sum_{v_j\in \vertSet}\sum_{j\conLeft k} \lambda_{jk}\bracs{B},
	\quad
	\nu\bracs{B} = \sum_{v_j\in\vertSet}\alpha_j\delta_j\bracs{B},
	\quad
	\dddmes\bracs{B} = \ddmes\bracs{B} + \nu\bracs{B},
\end{align*}
where we refer to $\ddmes$ as the ``\emph{singular measure} that supports $\graph$", and $\nu$ is the singular measure supporting the masses $\alpha_j$ at the vertices of $\graph$.
For graphs embedded into two and three dimensions, figure \ref{fig:Diagram_SingularMeasure2D} illustrates the $\dddmes$ ``measure" of Borel sets.
\begin{figure}[b!]
	\centering
	\begin{subfigure}[t]{0.45\textwidth}
		\centering
		\includegraphics[scale=0.85]{Diagram_SingularMeasure2D.pdf}
		\caption{\label{fig:Diagram_SingularMeasure2D} For a graph embedded in $\reals^2$, the $\ddmes$-measure of any Borel set $B$ is obtained from summing the contributions of each $\lambda_{jk}$, as indicated by the thickened lines. Sets that do not intersect $\graph$ have zero measure.}
	\end{subfigure}
	~
	\begin{subfigure}[t]{0.45\textwidth}
		\centering
		\includegraphics[scale=0.5]{Diagram_SingularMeasure3D.pdf}
		\caption{\label{fig:Diagram_SingularMeasure3D} An extension of the concept of a singular measure to 3 dimensions. The red portions of each edge indicate the contributions from each edge to the measure of the cube $B$.}
	\end{subfigure}
\end{figure}
In chapter \ref{ch:SingInc}, we will reintroduce the background material into which our singular structure is embedded.
This will require us to consider the sum of singular measures and the ``background" Lebesgue measure $\lambda_d$, so for a singular measure $\rho$ we shall write \tstk{change overhead tilde notation on $\dddmes$ to distinguish!!!} $\tilde{\rho} = \rho + \lambda_d$.

\section{Sobolev Spaces with Respect to Borel Measures}

some more words on these wonderful monstrosities!

%Scalar equation chapter begins
\chapter{Scalar Equation} \label{ch:ScalarSystem}
\tstk{title name of chapter needs redoing once you have an idea for what to change it to! As does the title of the next section!!!}

\section{Chapter Introduction} \label{sec:ScalarEqnChapterIntro}
This chapter aims to provide a concrete introduction to variational problems on singular structures, and their link to the quantum graph problems that arise in the asymptotic limit of thin structures as the thickness shrinks to zero.
We will concern ourselves with the study of the acoustic equation on a singular structure, and the corresponding non-classical Sobolev spaces of functions possessing gradients with respect to the measure $\dddmes$.
By understanding this function space we will be able to derive a quantum graph problem, or more precisely a problem of generalised resolvent type, from our original variational problem.
This will solidify the link between our variational problems and the limits of thin structure problems as discussed in section \ref{ssec:Intro-ThinStructures}, with our non-classical Sobolev spaces forming the natural ``extended" spaces, and providing us with a springboard for our work in the later chapters \ref{ch:CurlCurl} and \ref{ch:SingInc}.
To conclude this section, we will turn our attention to the use of the $M$-matrix to analyse the spectrum of the resulting quantum graph problems, providing an explicit formula for its entries.
We shall conclude with an examination of some specific singular structure geometries, complimenting some of the discussion points in sections \ref{ssec:MMatrix} and \ref{sec:ScalarDiscussion}.

Let us now formulate the problem that we wish to consider.
Let $\tgradSob{\ddom}{\dddmes}$ be the non-classical Sobolev space introduced in section \ref{sec:BorelMeasSobSpaces}, and let $\graph=\bracs{\vertSet, \edgeSet}$ be the period graph of a (periodic) metric graph embedded into $\reals^2$, with unit cell $\ddom\subset\reals^2$, in accordance with the notation and setup of section \ref{sec:TP-DomainSetup}.
In this chapter, we will concern ourselves with the analysis of the \emph{acoustic equation}\footnote{In order to avoid sounding monotonous in the text, we will also refer to \eqref{eq:SingularScalarWaveEqn} as the \emph{wave equation}.}
\begin{align} \label{eq:SingularScalarWaveEqn}
	-\laplacian_{\dddmes}^{\qm} u = \omega^2 u, \quad\text{in } \ddom,
\end{align}
where $u\in\tgradSob{\ddom}{\dddmes}$.
A detailed study of the functions that lie in the space $\tgradSob{\ddom}{\dddmes}$ is left to section \ref{sec:3DGradSobSpaces}, however we provide a geometric interpretation of the gradients of zero and tangential gradients $\tgrad_{\dddmes}u$ in section \ref{ssec:3DGradGeometric}.
Perhaps a more pressing issue is that we must provide a suitable definition of the operator $-\laplacian_{\dddmes}^{\qm}$ on the left hand side of \eqref{eq:SingularScalarWaveEqn}.
With the knowledge (from section \ref{sec:BorelMeasSobSpaces}) that $\tgradSob{\ddom}{\dddmes}$ is a Hilbert space, we consider the bilinear form $b_{\qm}$ defined by
\begin{align*}
	\dom\bracs{b_{\qm}} &= \tgradSob{\ddom}{\dddmes}\times\tgradSob{\ddom}{\dddmes}, \\
	b_{\qm}\bracs{u,v} &= \integral{\ddom}{ \tgrad_{\dddmes}u\cdot\overline{\tgrad_{\dddmes}v} }{\dddmes}.
\end{align*}
Clearly $b_{\qm}\bracs{u,u}\geq 0$ with equality only when $u$ is the zero function (whose tangential gradient is also the zero function). 
\tstk{Defining $\ip{u}{v}_{b_{\qm}} = b_{\qm}\bracs{u,v}+\ip{u}{v}_{\tgrad{\ddom}{\dddmes}}$, we can see that the norms $\norm{\cdot}_{b_{\qm}}$ and $\norm{\cdot}_{\tgradSob{\ddom}{\dddmes}}$ are equivalent (we have that $\recip{2}\norm{\cdot}_{b_{\qm}} \leq \norm{\cdot}_{\tgradSob{\ddom}{\dddmes}} \leq \norm{\cdot}_{b_{\qm}}$, and thus by Kato's representation theorem we have that... }
Therefore, there exists a self-adjoint operator $-\laplacian_{\dddmes}^{\qm}$ defined by
\begin{align*}
	\dom\bracs{ -\laplacian_{\dddmes}^{\qm} } 
	&= \clbracs{ u\in\tgradSob{\ddom}{\dddmes} \setVert \exists f\in\ltwo{\ddom}{\dddmes} \text{ s.t. } \right.
	\\
	& \qquad \labelthis\label{eq:AcousticOperatorDefinition}
	\left. b_{\qm}\bracs{u,v} = \ip{f}{v}_{\tgradSob{\ddom}{\dddmes}}, \quad \forall v\in\tgradSob{\ddom}{\dddmes} },
%	\left. \integral{\ddom}{ \tgrad_{\dddmes}u\cdot\overline{\tgrad_{\dddmes}v} }{\dddmes} = \integral{\ddom}{ f\overline{v}}{\dddmes}, \quad \forall v\in\tgradSob{\ddom}{\dddmes} },
\end{align*}
with action $-\laplacian_{\dddmes}^{\qm} u = f$, with $u$ and $f$ related as in \eqref{eq:AcousticOperatorDefinition}.
The acoustic equation \eqref{eq:SingularScalarWaveEqn} is then the eigenvalue problem for this operator, that is the problem of finding $\omega^2>0$ and $u\in\tgradSob{\ddom}{\dddmes}$ such that
\begin{align} \label{eq:SingularScalarWaveEqn-VariationalForm}
	\integral{\ddom}{ \tgrad_{\dddmes}u\cdot\overline{\tgrad_{\dddmes}\phi} }{\dddmes}
	&= \omega^2 \integral{\ddom}{ u\overline{\phi} }{\dddmes},
	\qquad\forall\phi\in\psmooth{\ddom}.
\end{align}
Whenever we refer to the problem \eqref{eq:SingularScalarWaveEqn}, we understand this problem in the sense of \eqref{eq:SingularScalarWaveEqn-VariationalForm}.

Recall the periodic graph $\hat{\graph}$ and measure $\upsilon$ from section \ref{sec:TP-DomainSetup}.
We can follow a similar construction to define the operator $-\laplacian_{\upsilon}$ on $\gradSob{\reals^2}{\upsilon}$, using the form
\begin{align*}
	\integral{\reals^2}{ \grad_{\upsilon}u\cdot\overline{\grad_{\upsilon}v} }{\upsilon},
	\qquad u,v\in\gradSob{\reals^2}{\upsilon}.
\end{align*}
Through the use of a Gelfand transform (and via passage through approximating sequences of smooth functions), the family of operators $-\laplacian_{\dddmes}^{\qm}$ decomposes $-\laplacian_{\upsilon}$ as
\begin{align*}
	-\laplacian_{\upsilon} &= \int_{[-\pi,\pi)^2}^{\bigoplus} -\laplacian_{\dddmes}^{\qm} \ \md\qm,
\end{align*}
and so we have in particular that
\begin{align*}
	\sigma\bracs{ -\laplacian_{\upsilon} } &= \bigcup_{\qm\in[-\pi,\pi)^2} \sigma\bracs{ -\laplacian_{\dddmes}^{\qm} }.
\end{align*}
That is to say, we are studying the spectrum of the periodic operator $-\laplacian_{\upsilon}$ --- the analogue of the acoustic equation for our singular structure domain --- through the family of operators $-\laplacian_{\dddmes}^{\qm}$ defined on periodic functions with domain $\ddom$.
Each member of this family is defined on a space of functions with a compact domain, so provided they are (uniformly) elliptic they will each posses discrete eigenvalues, and we can obtain dispersion branches as detailed in section \ref{sec:TP-GelfandTransform}.
It will be clear from both our examples and the derived quantum graph problem that this is the case for the operators $-\laplacian_{\dddmes}^{\qm}$.

Further to this point, in section \ref{sec:ScalarDerivation} we demonstrate that solutions to \eqref{eq:SingularScalarWaveEqn} satisfy the following quantum graph problem:
\begin{subequations} \label{eq:SingularWaveEqnQGProblem}
	\begin{align}
		-\bracs{\diff{}{y} + \rmi\qm_{jk}}^2 u^{(jk)} &= \omega^2 u^{(jk)}, \quad &y\in\interval{I_{jk}}, \ \forall I_{jk}\in\edgeSet, \label{eq:SingularWaveEqnQGProblem-1} \\
		u \text{ is continuous at } & v_j, \quad &\forall v_j\in\vertSet, \label{eq:SingularWaveEqnQGProblem-2} \\
		\sum_{j\con k}\bracs{\pdiff{}{n} + \rmi\qm_{jk}}u^{(jk)}\bracs{v_j} &= \omega^2\alpha_j u\bracs{v_j}, \quad &\forall v_j\in\vertSet. \label{eq:SingularWaveEqnQGProblem-3}
	\end{align}
\end{subequations}
The $\qm_{jk}$ are rotations of the quasi-momentum $\qm$, and can be computed given the orientation of the edge $I_{jk}\in\edgeSet$.
Note that \eqref{eq:SingularWaveEqnQGProblem} possesses a generalised resolvent --- the spectral parameter $\omega^2$ is present in the boundary conditions at each vertex \eqref{eq:SingularWaveEqnQGProblem-3}.
As such, section \ref{ssec:GRandSELinks} will also make explicit the link between our variational formulation and the Strauss extension for the problem \eqref{eq:SingularWaveEqnQGProblem}.

Having derived \eqref{eq:SingularWaveEqnQGProblem} and made the connection between its extension and our starting variational problems, we move on to how one can analyse the spectrum of such problems (both analytically and numerically).
This culminates in section \ref{sec:ScalarDiscussion} with an explicit expression for the $M$-matrix of \eqref{eq:SingularWaveEqnQGProblem}, opening a discussion into the methodology and considerations for using it to analyse the spectrum of \eqref{eq:SingularWaveEqnQGProblem}, which we then employ in section \ref{sec:ScalarExamples}, before concluding.
At the end of this chapter we will have an understanding of the relationship between our variational problems with respect to singular measures and quantum graph problems of generalised resolvent type, and the considerations one needs to make when defining derivatives on such structures.
This will form the basis from which we develop our framework for handling the analogues of the curl and divergence operator on singular-structures, naturally bringing us towards the analysis of the curl-of-the-curl equation in chapter \ref{ch:CurlCurl} and an investigation into the first-order Maxwell system.
Furthermore, our understanding of tangential gradients will be invaluable to our analysis in chapter \ref{ch:SingInc}, when we consider a singular structure surrounded by a bulk material.
We will also have reviewed the use of the $M$-matrix as a tool for analysing the spectrum of the resulting quantum graph problems, providing access numerically to the spectrum of our starting singular structure problem.

\section{Derivation of quantum graph problem} \label{sec:ScalarDerivation}
In this section we provide an overview of a system of the form \eqref{eq:SingularWaveEqnQGProblem} is obtained from \eqref{eq:SingularScalarWaveEqn}, which will setup our discussion revolving around the methods we employ for solving \eqref{eq:SingularWaveEqnQGProblem} in section \ref{sec:ScalarDiscussion}.

Precise definition and analysis of the ``Sobolev spaces" used here can be found in section \ref{sec:3DGradSobSpaces}, although we provide a short intuitive idea of the object $\tgrad_{\ddmes}u$ here.
The central idea behind understanding $\tgrad_{\ddmes}u$ is that the singular measure $\ddmes$ only supports the edges of $\graph$, and so cannot ``see" any changes in the function $u$ ``across" (in the direction perpendicular to) the edge $I_{jk}$.
So at any point $x\in I_{jk}$, the ``gradient" $\tgrad_{\ddmes}u(x)$ encapsulates the rate of change of the function $u$ at $x\in I_{jk}$ \emph{only} in the direction along $I_{jk}$.
As a result, it is not inaccurate to think of $\tgrad_{\ddmes}u(x) = \bracs{u^{(jk)}}'(x)e_{jk}$ (up to an appropriate shift due to the presence of the quasi-momentum $\qm$) for $x\in I_{jk}$, where $\bracs{u^{(jk)}}' = \pdiff{u^{(jk)}}{e_{jk}}$.
This also provides us with an intuitive understanding of how $\tgrad_{\ddmes}u$ behaves on each edge of the graph $\graph$, which is crucial for deriving the set of ``edge ODEs" \eqref{eq:SingularWaveEqnQGProblem} and providing a meaning to the $\diff{}{x}$ operator that appears in those equations.
As discussed in section \ref{ssec:FunctionSpaces}, the coupling constants attached to the vertices of the graph as well as the connectivity of the graph itself then dictate that these ``edge-wise" components $u^{(jk)}$ and $\bracs{u^{(jk)}}'$ adhere to certain matching conditions at the vertices.
The functions $u\in\gradSobQM{\ddom}{\dddmes}$ and their gradients $\tgrad_{\dddmes}u$ can be thought of as possessing the following properties (precise statements can be found in section \ref{sec:3DGradSobSpaces}):
\begin{enumerate}[(a)]
	\item The function $u$ is continuous at all vertices $v_j\in\vertSet$.
	\item On each edge $I_{jk}\in\edgeSet$; $\tgrad_{\dddmes}u = \tgrad_{\lambda_{jk}}u$, and $\tgrad_{\lambda_{jk}}u = \bracs{\bracs{u^{(jk)}}' + \rmi\qm_{jk}u^{(jk)}}e_{jk}$.
	The function $\bracs{u^{(jk)}}'$ being the derivative (in the $H^1$-sense) of the function $u^{(jk)}\circ r_{jk}$.
	\item At each vertex $v_j\in\vertSet$, we have $\tgrad_{\dddmes}u=0$, however $\lim_{x\rightarrow v_j}\tgrad_{\lambda_{jk}}u$ need not be zero.
\end{enumerate}

We can now provide a conceptual argument for how a system of the form \eqref{eq:SingularWaveEqnQGProblem} arises from \eqref{eq:SingularScalarWaveEqn}.
A function $u\in\gradSobQM{\ddom}{\dddmes}$ is said to be a solution to \eqref{eq:SingularScalarWaveEqn} if
\begin{align} \label{eq:SingularWaveEqnWeakForm}
	\integral{\ddom}{\tgrad_{\dddmes}u\cdot\overline{\tgrad_{\dddmes}\phi}}{\dddmes} &= \omega^2\integral{\ddom}{u\overline{\phi}}{\dddmes}, \quad\forall \phi\in\smooth{\ddom}.
\end{align}
We first note that whenever the equality in \eqref{eq:SingularWaveEqnWeakForm} holds for all smooth functions $\phi$, it holds in particular when $\phi$ has support that intersects the interior of a single edge $I_{jk}\in\edgeSet$ and no other parts of $\graph$.
Combined with the fact that $\dddmes$ is a sum of the edge measures and point masses at the vertices, and that $\tgrad_{\dddmes}u=\tgrad_{\lambda_{jk}}u$ on the edge $I_{jk}$, equation \eqref{eq:SingularWaveEqnWeakForm} implies
\begin{align*}
	0 &= \integral{\ddom}{ \bracs{\tgrad_\ddmes u \cdot \overline{\tgrad\phi} - \omega^2 u\overline{\phi}} }{\ddmes}
	= \integral{I_{jk}}{ \bracs{\tgrad_{\lambda_{jk}}u \cdot \overline{\tgrad\phi} - \omega^2 u^{(jk)}\overline{\phi}} }{\lambda_{jk}} \\
	&= \integral{I_{jk}}{ \clbracs{ \bracs{\bracs{u^{(jk)}}' + \rmi\qm_{jk} u^{(jk)}}\bracs{\overline{\phi}' - \rmi\qm_{jk} \overline{\phi} } - \omega^2 u^{(jk)}\overline{\phi} } }{\lambda_{jk}}.
\end{align*}
Now using the change of variables $r_{jk}$ and denoting $\tilde{u}^{(jk)} = u^{(jk)} \circ r_{jk}$ and $\varphi = \phi\circ r_{jk}$, we arrive at
\begin{align*}
	0 &= \int_{0}^{\abs{I_{jk}}} \bracs{\bracs{\tilde{u}^{(jk)}}' + \rmi\qm_{jk} \tilde{u}^{(jk)}}\bracs{\overline{\varphi}' - \rmi\qm_{jk} \overline{\varphi} } - \omega^2 \tilde{u}^{(jk)}\overline{\varphi} \ \md y. \\
	\implies
	\int_{0}^{\abs{I_{jk}}} \bracs{\tilde{u}^{(jk)}}'\overline{\varphi}' \ \md y &=
	\int_{0}^{\abs{I_{jk}}} \clbracs{ \omega^2\tilde{u}^{(jk)} + 2\rmi\qm_{jk}\bracs{\tilde{u}^{(jk)}}' + \bracs{\rmi\qm_{jk}}^2\tilde{u}^{(jk)} } \overline{\varphi} \ \md y.
\end{align*}
This holds for all smooth $\varphi$ with support contained in the interior of $\interval{I_{jk}}$, and as such we can deduce that $\tilde{u}^{(jk)}$ is twice (weakly) differentiable, and obtain the (strong) equation
\begin{align*}
	-\bracs{\diff{}{x} + \rmi\qm_{jk}}^2 \tilde{u}^{(jk)} &= \omega^2 \tilde{u}^{(jk)}, \quad x\in\interval{I_{jk}}.
\end{align*}

Now we turn our attention to the derivation of the vertex conditions.
Fix a vertex $v_j\in \vertSet$, and consider functions $\phi\in\smooth{\ddom}$ whose support intersects $\graph$ in a neighbourhood of $v_j$ that only contains edges which connect to $v_j$ (which can be, for example, a ball of sufficiently small radius centred on $v_j$).
Using the change of variables $r_{jk}$ on each edge and writing $\tilde{u}^{(jk)} = u^{(jk)} \circ r_{jk}$, $\varphi_{jk} = \phi\circ r_{jk}$ for each $k\con j$, we can work from \eqref{eq:SingularWaveEqnWeakForm} to obtain
\begin{align*}
	0 &= \sum_{k: \ k\con j} \integral{I_{jk}}{ \bracs{ \tgrad_\ddmes u \cdot \overline{\tgrad\phi} - \omega^2 u\overline{\phi} } }{\lambda_{jk}} 
	+ \integral{\ddom}{ \bracs{ \tgrad_{\dddmes}u\cdot\overline{\tgrad_{\dddmes}\phi}-\omega^2 u\overline{\phi} } }{\nu} \\
	&= \sum_{k: \ k\con j} \int_{0}^{\abs{I_{jk}}} \clbracs{ \bracs{\bracs{\tilde{u}^{(jk)}}' + \rmi\qm_{jk} \tilde{u}^{(jk)}}\bracs{\overline{\varphi}' - \rmi\qm_{jk} \overline{\varphi} } - \omega^2 \tilde{u}^{(jk)}\overline{\varphi} } \ \md y \\
	&\qquad + \alpha_j\left.\bracs{ \tgrad_{\dddmes}u\cdot\overline{\tgrad_{\dddmes}\phi}-\omega^2 u\overline{\phi} }\right\vert_{v_j} \\
	&= \sum_{k: \ k\con j} \int_{0}^{\abs{I_{jk}}} \clbracs{ \bracs{\bracs{\tilde{u}^{(jk)}}' + \rmi\qm_{jk} \tilde{u}^{(jk)}}\bracs{\overline{\varphi}' - \rmi\qm_{jk} \overline{\varphi} } - \omega^2 \tilde{u}^{(jk)}\overline{\varphi} } \ \md y
	 - \alpha_j \omega^2 u\bracs{v_j}\overline{\phi}\bracs{v_j}.
\end{align*}
Here we have used the fact that $\tgrad_{\dddmes}u\bracs{v_j}=0$ (see section \ref{sec:3DGradSobSpaces}).
Given that (from before) $u$ is twice differentiable on each $I_{jk}$, it follows that
\begin{align*}
	\alpha_j\omega^2 u\bracs{v_j}\overline{\phi}\bracs{v_j} 
	&= - \sum_{k: \ k\con j} \int_{0}^{\abs{I_{jk}}} \bracs{ \bracs{\diff{}{x} + \rmi\qm_{jk}}^2 \tilde{u}^{(jk)} +\omega^2 \tilde{u}^{(jk)} }\overline{\varphi} \ \md x \\
	&\qquad + \sum_{k: \ k\con j}\overline{\varphi}\bracs{v_j}\bracs{\pdiff{}{n} + \rmi\qm_{jk}}\tilde{u}^{(jk)}\bracs{v_j} \\
	&= \overline{\varphi}\bracs{v_j}\sum_{k: \ k\con j}\bracs{\pdiff{}{n} + \rmi\qm_{jk}}\tilde{u}^{(jk)}\bracs{v_j}. \labelthis\label{eq:DerivationVertexConditionWeak}
\end{align*}
Given that \eqref{eq:DerivationVertexConditionWeak} holds for every smooth $\varphi$, and that $\overline{\varphi}\bracs{v_j}=\overline{\phi}\bracs{v_j}$, we arrive at the condition
\begin{align*}
	\alpha_j\omega^2 u\bracs{v_j} &= \sum_{j\con k}\bracs{\pdiff{}{n} + \rmi\qm_{jk}}\tilde{u}^{(jk)}\bracs{v_j}, \quad \forall v_j \in \vertSet.
\end{align*}
Repeating the argument for each $v_j\in \vertSet$ then provides us with a condition of this form at each vertex.
One should note the presence of $\omega^2$ in this equation, so this is not a standard Robin condition on the derivatives of the edge-wise components of $u$, but rather indicates that our problem belongs to the class of problems with generalised resolvents, as mentioned in section \ref{ssec:DiffOpsOnGraphs}.
The result of theorem \ref{thm:dddmesTangGradImplication} tells us that functions $u\in\gradSobQM{\ddom}{\dddmes}$ are also continuous at each vertex $v_j$, and thus the following problem (precisely \eqref{eq:SingularWaveEqnQGProblem}) has been derived:
\begin{subequations}
	\begin{align}
		-\bracs{\diff{}{y} + \rmi\qm_{jk}}^2 u^{(jk)} &= \omega^2 u^{(jk)}, \quad &y\in\interval{I_{jk}}, \ \forall I_{jk}\in\edgeSet, \\
		u \text{ is continuous at } & v_j, \quad &\forall v_j\in\vertSet,  \\
		\sum_{j\con k}\bracs{\pdiff{}{n} + \rmi\qm_{jk}}u^{(jk)}\bracs{v_j} &= \omega^2\alpha_j u\bracs{v_j}, \quad &\forall v_j\in\vertSet,
	\end{align}
\end{subequations}
where we henceforth drop the overhead tilde notation and simply write $u^{(jk)}$ for brevity (appealing to the obvious association between $u^{(jk)}$ and $\tilde{u}^{(jk)}$).
Solving for the eigenvalues $\omega^2$ will net us the eigenvalues of our original problem \eqref{eq:SingularScalarWaveEqn}, and taking the union of the eigenvalues over $\qm$ will provide the spectrum of \eqref{eq:SingularScalarWaveEqnWholeSpace}.
As will be made clear in the discussion that follows, the quantum graph problem \eqref{eq:SingularWaveEqnQGProblem} is much easier to handle both analytically and numerically thanks to the utility of the $M$-matrix.

\section{General formula for the $M$-Matrix of a Finite, Periodic Graph} \label{sec:ScalarDiscussion}
Having obtained the quantum graph problem \eqref{eq:SingularWaveEqnQGProblem}, we turn our attention to determining the eigenvalues $z := \omega^2$.
In this section we contextualise the theory introduced in section \ref{ssec:MMatrix}, showing how it is employed for studying $\sigma\bracs{-\laplacian_{\dddmes}^{\qm}}$.
Our key result will be the provision of an explicit formula for (the entries of) the $M$-matrix in terms of the underlying (period) graph on which \eqref{eq:SingularWaveEqnQGProblem} is posed.
We will follow up on this in section \ref{sec:ScalarExamples}, where we provide some explicit examples of quantum graph problems whose spectra can be determined through the study of the $M$-matrix.

One will note that only $\alpha_j$ appears in the Wentzell condition in \eqref{eq:GraphLaplacianExample}, but $\alpha_j\omega^2$ is present on the right hand side of \eqref{eq:SingularWaveEqnQGProblem}.
As was raised in section \ref{ssec:Intro-ThinStructures}, the problem \eqref{eq:SingularWaveEqnQGProblem} belongs to the class of problems with generalised resolvents.
In section \ref{ssec:MMatrix} we introduced the $M$-matrix in a more familiar setting, with no $\omega^2$-dependence in the vertex conditions.
However the analysis of the spectrum of $-\laplacian_{\dddmes}^{\qm}$ can be carried out by replacing the matrix $B=-\alpha$ (in section \ref{ssec:MMatrix}) with $\omega^2 B$, as will be done in section \ref{ssec:MMatrixConsequences}.
Justification lies in the observation that introducing explicit $\omega^2$-dependence should not affect the (structure of) the arguments in the supporting theory\footnote{See \cite[page 1846]{cherednichenko2018effective} in reference to the results of \cite{ryzhov2009weyl} and references therein.} and consequentially similar conclusions can be drawn with this alteration to $B$.
However we highlight that a formal presentation of these arguments has not been conducted in the literature, and remains open.
If one has reservations about this ``gap" in the available theory, an alternative to analysing a problem with generalised resolvents directly is explored in \cite[Section 6]{cherednichenko2017norm}.
One could look to transform a problem with $\omega^2$-dependent $\delta$-type vertex condition (like \eqref{eq:SingularWaveEqnQGProblem}) into a problem with $\omega^2$-independent $\delta'$-type vertex conditions.
This comes at the cost of having to determine the appropriate (unitary) transform to apply to \eqref{eq:SingularWaveEqnQGProblem}, but the theory of section \ref{ssec:MMatrix} would apply to the transformed problem, could be used to analyse the spectrum, and then the inverse transform applied.

Define the (Dirichlet) map $\dmap$ as in \eqref{eq:GraphDNMapDef}.
The action of the Neumann map $\nmap$ is defined sightly differently to how it appears in \eqref{eq:GraphDNMapDef} (by virtue of our need to take a Gelfand transform), however ultimately conveys the same meaning:
\begin{align} \label{eq:GraphNMapQM}
	\bracs{\nmap u}_j = -\sum_{j\con k} \bracs{\pdiff{}{n} + \rmi\qm_{jk} }u(v_j).
\end{align}
With these definitions, the Green's identity \eqref{eq:GraphGreensIdentity} 
\begin{align*}
	\ip{ -\bracs{\diff{}{y}+\rmi\qm_{jk}}^2 u }{ v }_{L^2\bracs{\graph}} - \ip{ u }{ -\bracs{\diff{}{y}+\rmi\qm_{jk}}^2 v }_{L^2\bracs{\graph}}
	&= \ip{\nmap u}{\dmap v}_{\complex^{\abs{\vertSet}}} - \ip{\dmap u}{\nmap v}_{\complex^{\abs{\vertSet}}},
\end{align*}
holds, and the map $u\mapsto\bracs{\dmap u, \nmap u}$ is clearly surjective onto $\complex^{\abs{\vertSet}}\times\complex^{\abs{\vertSet}}$.
As such, we can define the $M$-matrix for the problem \eqref{eq:SingularWaveEqnQGProblem} in the manner described in section \ref{ssec:MMatrix}, and in particular know that $\omega^2$ is an eigenvalue of $-\laplacian_{\qm}^{\dddmes}$ whenever $M\bracs{\omega^2}-\omega^2 B$ possesses a zero eigenvalue.
In the abstract, this is not particularly helpful for explicitly determining the eigenvalues --- however the nature of the problem \eqref{eq:SingularWaveEqnQGProblem} allows us to compute the entries of $M\bracs{\omega^2}$ explicitly.

\subsection{General formula for the $M$-matrix} \label{ssec:MMatrixResult}
The relatively simple nature of the structure provided by the metric graph $\graph$, and the action on each edge of the graph, allows us to explicitly compute the entries of the $M$-matrix.
The proposition we present allows for $\graph$ to possess looping edges, although this is largely for completeness because we will want to introduce \emph{artificial vertices} (section \ref{ssec:ArtificialVertices}) to break these loops.
The following proposition provides the entries of the $M$-matrix.
\begin{prop}[$M$-matrix entries] \label{prop:M-MatrixEntries}
	Let $\graph=\bracs{\vertSet,\edgeSet}$ be an embedded graph on which the problem \eqref{eq:SingularWaveEqnQGProblem} is posed.
	Suppose that $\dmap u = e_k$ where $e_k$ is the $k$\textsuperscript{th} canonical unit vector in $\complex^{\abs{\vertSet}}$.
	Then the $j$\textsuperscript{th} entry of $\nmap u$, and hence the $jk$\textsuperscript{th} entry in the $M$-matrix, is given by
	\begin{align*}
		\bracs{\nmap u}_j &= 
		\begin{cases}
			0,	
			& j \not\con k, \\[5pt]
			\sum_{j\conLeft k} \omega \e^{\rmi\qm_{jk}l_{jk}} \csc\bracs{l_{jk}\omega} 
			+ \sum_{j\conRight k} \omega \e^{-\rmi\qm_{kj}l_{kj}} \csc\bracs{l_{kj}\omega},
			& j\neq k, \ j\con k, \\[5pt]
			- \sum_{\substack{j\con l \\ j\neq l}} \omega\cot\bracs{l_{jl}\omega}
			- 2\omega\sum_{j\conLeft j} \clbracs{ \cot\bracs{l_{jj}\omega} - \cos\bracs{\qm_{jj}l_{jj}}\csc\bracs{l_{jj}\omega} },
			& j=k.
		\end{cases}
	\end{align*}
\end{prop}
Note the choice of $j\conLeft j$ in the contributions from loops is simply a convention, $j\conRight j$ is equivalent here.
Also recall the convention for summing over $j\con k$:
\begin{align*}
	\sum_{j\con k} \omega\cot\bracs{l_{jk}\omega} &= \sum_{j\conLeft k} \omega\cot\bracs{l_{jk}\omega}	+ \sum_{j\conRight k} \omega\cot\bracs{l_{kj}\omega}
\end{align*}
\begin{proof}
	The proof below is an explicit computation, similar to that in \cite{ershova2014isospectrality} with adjustments for the dependence on $\qm$.
	
	We first write the general form of the edge solution $u^{(jk)}$ from \eqref{eq:SingularWaveEqnQGProblem-1}:
	\begin{align} \label{eq:EdgeEqnGeneralSolution}
		u^{(jk)} &= \e^{-\rmi\qm_{jk}y}\bracs{ C_{+}^{(jk)}\e^{-\rmi\omega y} + C_{-}^{(jk)}\e^{\rmi\omega y} },
		\quad C_{+}^{(jk)}, C_{-}^{(jk)}\in\complex.
	\end{align}
	Since the $M$-matrix maps $\complex^{\abs{\vertSet}}$ to $\complex^{\abs{\vertSet}}$, it is sufficient to determine its action on the canonical basis of $\complex^{\abs{\vertSet}}$.
	So for each fixed $k\in\clbracs{1,...,\abs{\vertSet}}$ we set $\dmap u = e_k$.
	This provides us with sufficient Dirichlet data to solve \eqref{eq:SingularWaveEqnQGProblem-1} on each edge and eliminate the constants $C_{+}^{(jk)}$, $C_{-}^{(jk)}$ in \eqref{eq:EdgeEqnGeneralSolution}, obtaining
	\begin{align*}
		j\not\con k &\implies
		\begin{cases}
			u_{jk}(x) = 0, \\
			u_{kj}(x) = 0,
		\end{cases} \\
		j\neq k, \ j\con k &\implies
		\begin{cases}
			u_{jk}(x) = \e^{-\rmi\qm_{jk}\bracs{x-l_{jk}}}\csc\bracs{\omega l_{jk}}\sin\bracs{\omega x}, \\
			u_{kj}(x) = \e^{-\rmi\qm_{kj}x}\csc\bracs{\omega l_{kj}}\sin\bracs{\omega \bracs{l_{kj}-x}},
		\end{cases} \\
		j = k &\implies 
		\begin{cases}
			u_{jj}(t) = \e^{-\rmi\qm_{jj}x} \bracs{ \e^{-\rmi\omega x} + \sqbracs{\e^{\rmi\qm_{jj}l_{jj}}-\e^{-\rmi\omega l_{jj}}}\csc\bracs{\omega l_{jj}}\sin\bracs{\omega x}  },
		\end{cases}
	\end{align*}
	This in turn enables us to explicitly differentiate the expressions for $u_{jk}$, and read off the values of $\bracs{\pdiff{}{n}+\rmi\qm_{jk}}u_{jk}$ at the vertices.
	In the case $j\not\con k$, we obviously get zero contribution from the edges $I_{jk}$ and $I_{kj}$.
	The case $j\neq k, \ j\con k$, yields the following contributions from the edges $I_{jk}$ and $I_{kj}$:
	\begin{align*}
		\bracs{\pdiff{}{n}+\rmi\qm_{jk}}u^{(jk)}\bracs{v_j} = -\omega \e^{\rmi\qm_{jk}l_{jk}}\csc\bracs{\omega l_{jk}}, 
		&\qquad \bracs{\pdiff{}{n}+\rmi\qm_{jk}}u^{(jk)}\bracs{v_k} = \omega\cot\bracs{\omega l_{jk}}, \\
		\bracs{\pdiff{}{n}+\rmi\qm_{kj}}u^{(kj)}\bracs{v_j} = -\omega \e^{-\rmi\qm_{kj}l_{kj}}\csc\bracs{\omega l_{kj}}, 
		&\qquad \bracs{\pdiff{}{n}+\rmi\qm_{kj}}u^{(kj)}\bracs{v_k} = \omega\cot\bracs{\omega l_{kj}}.
	\end{align*}
	Finally, when considering the case $j=k$, the contribution to $\bracs{\nmap u}_j$ from loops $I_{jj}$ in the graph also requires us to compute
	\begin{align*}
		-\lim_{x\rightarrow0}\bracs{\bracs{u^{(jj)}}'+i\qm_{jj}u^{(jj)}}(x) + \lim_{x\rightarrow l_{jj}} & \bracs{\bracs{u^{(jj)}}'+i\qm_{jj}u^{(jj)}}(x) \\
		&\qquad = 2\omega\bigl( \cot\bracs{\omega l_{jj}} - \cos\bracs{\qm_{jj}l_{jj}}\csc\bracs{\omega l_{jj}} \bigr).	
	\end{align*}
	We then use the formula \eqref{eq:GraphNMapQM} to obtain the desired result for $\bracs{\nmap u}_j$.
	Since $M(\dmap u) = \nmap u$, and the $e_k$ are a basis for $\complex^{\abs{V}}$, we have also deduced the $k^{\text{th}}$ column of the $M$-matrix.
\end{proof}
Proposition \ref{prop:M-MatrixEntries} also demonstrates how the $M$-matrix is parametrised by $\qm$, and so we shall denote it by $M_{\qm}$ henceforth.
The dependence of $M_\qm$ on $\qm$ is due to our decision to specify our singular structure as an embedded, periodic metric graph and then apply the Gelfand transform (see section \ref{ssec:MMatrix}).
In the following section, we continue our analysis of this family of $M$-matrices and how it can be used to recover the eigenvalues of \eqref{eq:SingularWaveEqnQGProblem}.

\subsection{Consequences of proposition \ref{prop:M-MatrixEntries}} \label{ssec:MMatrixConsequences}
Whilst proposition \ref{prop:M-MatrixEntries} provides an explicit form for the entries of the $M$-matrix,  it is not the most convenient when looking for a method for determining the spectrum of \eqref{eq:SingularWaveEqnQGProblem}.
Proposition \ref{prop:M-MatrixEntries} does however show that $M_\qm$ is meromorphic, and thus has the following decomposition:
\begin{cory} \label{cory:M-MatrixEntriesNoPoles}
	Let $G^{(1)}_\qm\bracs{\omega}$ have entries defined by
	\begin{align*}
		\bracs{G^{(1)}_\qm}_{jk} &= 
		\begin{cases}
			\!\begin{aligned}
				&0,
			\end{aligned}			
			& j \not\con k, \\
			\!\begin{aligned}
				&\sum_{j\conLeft k} \bracs{ \e^{\rmi\qm_{jk}l_{jk}} \prod_{v_l\in\vertSet}\prod_{\substack{ l\conLeft m \\ \bracs{l,m} \neq \bracs{j,k} }}\sin\bracs{l_{lm}\omega} }
				\\ &\quad + \sum_{j\conRight k} \bracs{ \e^{-\rmi\qm_{kj}l_{kj}} \prod_{v_l\in\vertSet}\prod_{\substack{l\conLeft m \\ \bracs{l,m} \neq \bracs{k,j} }}\sin\bracs{l_{lm}\omega} },
			\end{aligned}
			& j\neq k, \ j\con k, \\
			\!\begin{aligned}
				&- \sum_{\substack{j\con l \\ j\neq l}} \bracs{ \cos\bracs{l_{jl}\omega}\prod_{v_m\in\vertSet}\prod_{\substack{ m\conLeft n \\ \bracs{m,n}\neq\bracs{j,l} }}\sin\bracs{l_{mn}\omega} }
				\\ &\quad - 2\sum_{j\conLeft j} \bracs{ \sqbracs{ \prod_{v_l\in\vertSet}\prod_{\substack{l\conLeft m \\ \bracs{l,m}\neq\bracs{j,j} }}\sin\bracs{l_{lm}\omega} }\bigl[ \cos\bracs{l_{jj}\omega} - \cos\bracs{\qm_{jj}l_{jj}} \bigr] },
			\end{aligned}
			& j=k,
		\end{cases}
	\end{align*}
	and set
	\begin{align*}
		G^{(2)}\bracs{\omega} &= \prod_{v_j\in\vertSet} \prod_{j\conLeft k}\sin\bracs{l_{jk}\omega}.
	\end{align*}
	Further define
	\begin{align*}
		H^{(1)}_{\qm}(z) &:= 
		\begin{cases} 
			\omega G_\qm^{(1)}(\omega), & \abs{\edgeSet} \text{ is even}, \\
			G_\qm^{(1)}(\omega), & \abs{\edgeSet} \text{ is odd},
		\end{cases} \\
		H^{(2)}(z) &:=
		\begin{cases}
			G^{(2)}(\omega), & \abs{\edgeSet} \text{ is even}, \\
			\omega^{-1} G^{(2)}(\omega), & \abs{\edgeSet} \text{ is odd}.
		\end{cases}
	\end{align*}
	Then the functions $H^{(1)}_{\qm}(z)$ and $H^{(2)}(z)$ are analytic in $z:=\omega^2$ and we have
	\begin{align*}
		M_\qm\bracs{z} &= \bracs{ H^{(2)}\bracs{z} }^{-1} H^{(1)}_\qm\bracs{z}.
	\end{align*}
\end{cory}
The product notation should be understood analogously to the summation notation over $j\con k$ introduced in section \ref{sec:QuantumGraphs}.
The zeros of $H^{(2)}$ exactly coincide with the poles of $M_\qm$, both $H^{(1)}_\qm$ and $H^{(2)}$ are analytic, and the matrix $H^{(1)}_\qm$ even has its entry at position $jk$ bounded (uniformly in $\omega$) by the number of (direct) connections between $v_j$ and $v_k$.

Recall that (section \ref{ssec:MMatrix}) we need to determine those $z$ for which the matrix $M_\qm(z)-B(z)$ has at least one zero eigenvalue, where we have set $B(z) = -z\alpha$.
Now let $\beta_j^{\qm}\bracs{z}, j\in\clbracs{1,...,\abs{\vertSet}}$ denote the eigenvalue branches of the matrix $M_\qm(z)-B(z)$.
Also set 
\begin{align*}
	\mathfrak{M}_\qm(z) = H^{(1)}_\qm(z) - H^{(2)}(z)B(z),
\end{align*}
and let $\widetilde{\beta}_j^{\qm}\bracs{z}, j\in\clbracs{1,...,\abs{\vertSet}}$ denote the eigenvalue branches of $\mathfrak{M}_\qm$.
With the poles removed, and the entries of $\mathfrak{M}_\qm$ being continuous (even smooth) functions of $\omega$ and $\qm$, the $\widetilde{\beta}_j^{\qm}$ are also continuous with respect to $\omega$ and $\qm$.
For each fixed $\qm$, the matrix $\mathfrak{M}_\qm$ is analytic and so has at least one zero eigenvalue at those $z$ for which there exists a $w\in\complex^{\abs{\vertSet}}\setminus\clbracs{0}$ such that
\begin{align} \label{eq:QGGenEvalSolveNoPoles}
	\mathfrak{M}_\qm\bracs{z}w = 0.
\end{align}
We could also chose to determine these $z$ via solution to 
\begin{align} \label{eq:QGDetSolveCondition}
	\det\mathfrak{M}_\qm\bracs{z} &= 0,
\end{align}
the merits of each approach (via \eqref{eq:QGGenEvalSolveNoPoles} or \eqref{eq:QGDetSolveCondition}) we will discuss in section \ref{ssec:ApproachConsiderations}.
If $z_0$ solves \eqref{eq:QGGenEvalSolveNoPoles} (or equivalently \eqref{eq:QGDetSolveCondition}), then $M_{\qm}\bracs{z_0}-B\bracs{z_0}$ has a zero eigenvalue when
\begin{align} \label{eq:EigenvalueBranchLimit}
	\lim_{z\rightarrow z_0} \beta_j^{\qm}\bracs{z} = \lim_{z\rightarrow z_0} \bracs{ H^{(2)}\bracs{z} }^{-1} \widetilde{\beta}_j^{\qm}\bracs{z} = 0
\end{align}
for at least one $j$ with $\widetilde{\beta}_j^{\qm}\bracs{z_0}=0$.
Checking the limit in \eqref{eq:EigenvalueBranchLimit} is not necessary for all solutions $z_0$ to \eqref{eq:QGDetSolveCondition}; indeed it is only necessary when $H^{(2)}\bracs{z_0} = 0$, which by corollary \ref{cory:M-MatrixEntriesNoPoles} occurs when
\begin{align} \label{eq:H2ZerosEqn}
	z_0 = \bracs{ \frac{n\pi}{l_{jk}} }^2, \quad j\conLeft k, \ n\in\naturals_{0},
\end{align}
which is a countable set of isolated points.
Furthermore the limit \eqref{eq:H2ZerosEqn} is essentially a residue, so one also has the possibility of using numerical methods derived from techniques in complex analysis to evaluate the limit.
As we will discuss in section \ref{ssec:ApproachConsiderations}, one may be able to circumvent checking the limit \eqref{eq:EigenvalueBranchLimit} by exploiting \eqref{eq:TP-DenseQMSubsetSuffices}.
Considerations concerning which of the two equations \eqref{eq:QGGenEvalSolveNoPoles} or \eqref{eq:QGDetSolveCondition} should be used for determining the spectrum $\sigma_\qm$ are also discussed in section \ref{ssec:ApproachConsiderations}.
In any event, \eqref{eq:SingularScalarWaveEqn} has now been reduced to a more accessible (family of) matrix-eigenvalue problems for $\mathfrak{M}_\qm$.

\subsection{Artificial vertices and splitting edges} \label{ssec:ArtificialVertices}
As noted in section section \ref{ssec:MMatrix}, it is required that the underlying graph $\graph$ contains no looping edges and has all edge-lengths pairwise irrationally-related\footnote{More precisely, it was noted that the operator we wish to study is required to be simple, which in the context of quantum graphs is equivalent to these conditions \cite{ashurova2014simplicity}.}.
Failure to ensure that this condition is met results in the ``loss" of certain eigenvalues when using the $M$-matrix to determine the spectrum of \eqref{eq:SingularWaveEqnQGProblem} --- these are highlighted explicitly in section \ref{ssec:Example1DLoop}.
Of course, the graphs motivated by physical applications generally do not adhere to these restrictions --- typically being highly symmetric or and potentially consisting of loops.
It is thus necessary to introduce \emph{artificial vertices} (or \emph{dummy vertices}) to split the edges of the original graph, removing any loops and ensuring all (new) edges have irrationally-related lengths.
Here it is relevant to highlight the discussions in \cite[examples 1.4.3 and 2.1.12]{berkolaiko2013introduction}; which consider a star-graph with Neumann vertex conditions, and demonstrate that the edge lengths being pairwise irrationally-related is also sufficient for all the eigenvalues to have multiplicity one.
The result \cite[theorem 3.7.1]{berkolaiko2013introduction} handles more general graph settings, and using techniques surrounding so-called bond scattering matrices, culminates with \cite[theorem 3.7.1]{berkolaiko2013introduction} relating the multiplicity of the eigenvalues to the multiplicity of the roots of the so-called \emph{secular equation}.
A discussion is provided in \cite[section 3.5]{berkolaiko2013introduction} that relates this framework with the to the Dirichlet-to-Neumann map, and hence our approach via \eqref{eq:QGGenEvalSolveNoPoles} and \eqref{eq:QGDetSolveCondition}.

Introducing an artificial vertex to split an edge is as intuitive as it sounds.
Suppose $\graph=\bracs{\vertSet, \edgeSet}$ and one wishes to split the edge $I_{jk}$ (where it may be the case that $j=k$ and we have a loop).
Place a vertex $v_l$ at some point along the edge $I_{jk}$, and replace $I_{jk}$ with the edges $I_{jl}$ and $I_{lk}$, to obtain a new graph $\graph^*$.
The total length of the edges must be preserved, so $l_{jk} = l_{jl} + l_{lk}$, but the lengths of the new edges should be chosen bearing in mind the aforementioned requirements.
Furthermore, a zero coupling constant should be placed at the artificial vertex $v_l$ --- this ensures matching of the solution $u$ and its derivative at the artificial vertex, as would have been the case along the original edge if it had not been split.
The quasi-momentum parameters should also satisfy $\qm_{jl}=\qm_{lk}=\qm_{jk}$ (although this is a by-product of having straight edges, see assumption \ref{ass:MeasTheoryProblemSetup}), and the direction of the original edge should be preserved.
This ensures (via \eqref{eq:SingularWaveEqnQGProblem-2} and \eqref{eq:SingularWaveEqnQGProblem-3}) that any eigenvalues $\omega^2$ of \eqref{eq:SingularWaveEqnQGProblem} on $\graph^*$ are also eigenvalues of \eqref{eq:SingularWaveEqnQGProblem} on $\graph$, with the eigenfunction $u^{(jk)}$ being related to $u^{(jl)}$ and $u^{(lk)}$ in the obvious manner.
This process can be executed iteratively, splitting edges to remove rational-relations between edge lengths and any loops.
Doing so means that any graph representing a singular-structure can now be treated in the manner described in section \ref{ssec:MMatrixConsequences}.

We highlight here that there are a number of considerations to make regarding the design of a \emph{general algorithm} to create a graph with pairwise irrationally-related edge lengths, from a graph without this quality.
Design of such an algorithm lies outside the objectives of this thesis, however we hope to provide readers interested in developing such a proceedure with a helpful starting point for such considerations.
In the event that one begins with a graph with $L$ edges all of which are \emph{all} pairwise-rationally related, then one has the following proceedural method for creating a graph with $2L$ edges of pairwise irrationally-related lengths:
\begin{enumerate}
   \item Split the edge $I_1$ of length $l_1$ into two edges of lengths $\frac{l_1}{\sqrt{2}}$ and $l_1\bracs{1 - \frac{1}{\sqrt{2}}}$.
   These edges now have lengths that are irrationally related to each other, and all other edges in the graph.
   \item Split the edge $I_2$ of length $l_2$ into two edges of lengths $\frac{l_2}{\sqrt{3}}$ and $l_2\bracs{1-\frac{1}{\sqrt{3}}}$.
   These edges now have lengths that are irrationally related to each other, all edges $I_n$, $2<n<L$, and the two new edges created in step 1.
   \item Proceeding iteratively, one splits the edge $I_n$ of length $l_n$ into two edges of length $\frac{l_n}{\sqrt{p_n}}$ and $l_n\bracs{1-\frac{1}{\sqrt{p_n}}}$ where $p_n$ is the $n^{\text{th}}$ prime number.   
\end{enumerate}
Here we have are dependent on the following result:
\begin{lemma} \label{lem:NumTheory-GraphSplitAlgorithm}
    Suppose that $a,b>0$ and there exists $q\in\rationals$ such that $\frac{a}{b}=q$.
    Let $p_1$ and $p_2$ be distinct prime numbers. 
    Then the values
    \begin{align} \label{eq:NumTheory-PairsWithoutAssumption}
        \frac{a}{\sqrt{p_1}}, \quad
        a\bracs{1-\frac{1}{\sqrt{p_1}}}, \quad
        b,    
    \end{align}
    are pairwise-irrationally related.
    If additionally $\sqrt{\frac{p_1}{p_2}}\frac{\sqrt{p_2}-1}{\sqrt{p_1}-1}$ is irrational\footnote{It is not known to the author whether this condition is always satisfied for primes $p_1$ and $p_2$, so the result may be true without the need to add this additional assumption.}, the values
    \begin{align} \label{eq:NumTheory-PairsWithAssumption}
        \frac{a}{\sqrt{p_1}}, \quad
        a\bracs{1-\frac{1}{\sqrt{p_1}}}, \quad
        \frac{b}{\sqrt{p_2}}, \quad
        b\bracs{1-\frac{1}{\sqrt{p_2}}},
    \end{align}
    are pairwise-irrationally related.
\end{lemma}
\begin{proof}
    This proof is simply a case of comparing the quotients of the possible pairs in \eqref{eq:NumTheory-PairsWithoutAssumption} and \eqref{eq:NumTheory-PairsWithAssumption}, and realising they are irrational.
\end{proof}
For the purposes of completing the proceedure listed above, it would be sufficient to demonstrate the weaker conclusion that there exist at least $L$ such pairs of primes --- because after $L$ steps we would be left with $2L$ edges of irrationally-related lengths.
Things are further complicated however if the original graph contains a mixture of rationally and irrationally related edge lengths, as one might now split an edge in the above manner only to find it then rationally relates to another edge that it previously did not.
A stronger version of lemma \ref{lem:NumTheory-GraphSplitAlgorithm} would be needed in this case (dropping the assumption that $a$ and $b$ are rationally related); or one would have to produce a less-n\"{i}ave approach to the order in which edges are split, as it may no longer be sufficient to simply use reciprocal roots of prime numbers to ensure edges are irrationally-related.

\subsection{Considerations for the approach to solving \eqref{eq:QGGenEvalSolveNoPoles} or \eqref{eq:QGDetSolveCondition}} \label{ssec:ApproachConsiderations}
In this section we briefly discuss some considerations for recovering the spectrum $\sigma\bracs{-\laplacian_{\upsilon}}=:\sigma$, through determination the spectra $\sigma_{\qm}:=\sigma\bracs{-\laplacian_{\dddmes}^{\qm}}$ via \eqref{eq:QGGenEvalSolveNoPoles} or \eqref{eq:QGDetSolveCondition}.
%Recall that our use of the Gelfand transform informs us that $\sigma = \bigcup_{\qm}\sigma_{\qm}$.
We take as given that one has to hand an appropriate numerical scheme for handling the generalised eigenvalue problem \eqref{eq:QGGenEvalSolveNoPoles} (a good introduction to which can be found in \cite{guttel2017nonlinear}), and so do not delve into the details of how such an algorithm would operate.
However it is worth highlighting that $\mathfrak{M}_\qm$ is Hermitian\footnote{Our requirement that $\alpha_j>0$ is implicitly used here --- in general $\mathfrak{M}_{\qm}$ is Hermitian provided $\alpha_j\in\reals$. The $M$-matrix $M_{\qm}$ is Hermitian regardless of the values of the coupling constants.}, from which most numerical schemes benefit.
The element-wise derivative with respect to $\omega$ is also easily computed (this derivative is required for generalised eigenvalue solvers based on Newton's method), although the resulting expressions are often cumbersome.
If we are intent on working directly with the determinant, then we can use corollary \ref{cory:M-MatrixEntriesNoPoles} to prove the following result about its form.
\begin{prop} \label{prop:MMatrixDetForm}
	Given a graph $\graph = \bracs{\vertSet, \edgeSet}$ containing no loops and whose edge-lengths are pairwise irrationally related, there exists a function $F\bracs{\qm,\omega}$ that is analytic in both its arguments, such that
	\begin{align} \label{eq:MMatrixDetForm}
		\det\mathfrak{M}_\qm\bracs{\omega^2} = \bracs{ \omega H^{(2)}\bracs{\omega^2} }^{\abs{\vertSet}-2} F\bracs{\qm,\omega}.
	\end{align}
\end{prop}
A proof of this result can be found in section \ref{sec:ProofOfProp}, but is just book-keeping the number of times a given factor can appear in the expression for the determinant.
Note that a graph with only one vertex can \emph{only} have looping edges, which must be broken via an artificial vertex to produce a graph with (at least two) vertices.
We can then use proposition \ref{prop:MMatrixDetForm} to prove the following result.
\begin{cory} \label{cory:ScalarSetInclusions}
	Define the sets
	\begin{align*}
		F_0^{\qm} := \clbracs{\omega \setVert F\bracs{\qm, \omega}=0},
		\qquad
		F_0 := \bigcup_{\qm\in B}F_0^{\qm},
		\qquad
		H_0 := \clbracs{\omega \setVert H^{(2)}\bracs{\omega^2}=0}.
	\end{align*}		
	Then it holds that
	\begin{enumerate}[(i)]
		\item $F_0\setminus H_0 = \sigma\setminus H_0$,
		\item $\overline{F_0\setminus H_0} \subset \sigma$.
	\end{enumerate}
\end{cory}
\begin{proof}
	\begin{enumerate}[(i)]
		\item This follows from the relation $\widetilde{\beta}_j^{\qm} = H^{(2)}\beta_j^\qm$ between the eigenvalue branches $\widetilde{\beta}_j^{\qm}$ of $\mathfrak{M}_{\qm}$ and $\beta_j^{\qm}$ of $M_{\qm}-B$.
		If $\omega_0\in F_0\setminus H_0$, there exists some $\qm\in B$ such that $F\bracs{\qm,\omega_0}=0$, and so proposition \ref{prop:MMatrixDetForm} implies that $\det\mathfrak{M}_\qm\bracs{\omega_0^2}=0$.
		So there is in particular one eigenvalue branch $\widetilde{\beta}_j^{\qm}$ with $\widetilde{\beta}_j^{\qm}\bracs{\omega_0^2}=0$, and since $\omega_0\not\in H_0$ we have $\beta_j^\qm\bracs{\omega_0^2}=0$ too, so $\omega_0\in\sigma$.
		Conversely, if $\omega_0\in \sigma\setminus H_0$, there exists some $\qm\in B$ and an eigenvalue branch with $\beta_j^\qm\bracs{\omega_0^2}=0$.
		Since $\omega_0\not\in H_0$, $\widetilde{\beta}_j^{\qm}\bracs{\omega_0^2}=0$ too, so $\det\mathfrak{M}_{\qm}\bracs{\omega_0^2}=0$, and thus by proposition \ref{prop:MMatrixDetForm} so is $F\bracs{\qm,\omega_0}$.
		\item By (i), $F_0\setminus H_0 \subset \sigma$.
		Therefore, $\overline{F_0\setminus H_0}\subset\overline{\sigma}=\sigma$, since the spectrum $\sigma$ is closed.
	\end{enumerate}
\end{proof}
Corollary \ref{cory:M-MatrixEntriesNoPoles} demonstrates that the majority of the spectrum can thus be found through examination of the function $F$.
The preimage $F^{-1}\bracs{\clbracs{0}}$ coincides with $\sigma$ up to the roots of $H^{(2)}\bracs{\omega^2}$ --- in particular, it dictates the spectral bands.
This also allows us to avoid manually checking the limit \eqref{eq:EigenvalueBranchLimit} at roots of $H^{(2)}\bracs{\omega^2}$ within the spectral bands --- such values are necessarily within the closure of $F_0\setminus H_0$.
All that remains are then those $\omega_0\in F_0\setminus H_0$ that are not within $\overline{F_0\setminus H_0}$ --- a set of isolated points along the positive real line.
Our example in section \ref{ssec:ExampleCrossInPlane} illustrates that the inclusion in (ii) can indeed be strict.

Finding $\sigma$ (up to checking isolated roots of $H^{(2)}$) now becomes a question of obtaining $F_0$ efficiently.
There is always brute force; computing $F_0^{\qm}$ for each $\qm$ (or for each $\qm$ in a suitable mesh if working numerically), and then taking the union over $\qm$ to obtain $F_0$.
Alternatives are available if $F$ admits additional properties, however these will rely on our ability to find an analytic expression for $F$ (one is obtained in the proof of proposition \ref{prop:MMatrixDetForm}, however it is rather cumbersome to use).
The ideal case being when we can write $F\bracs{\qm,\omega} = F_1\bracs{\qm} - F_2\bracs{\omega}$ for continuous $F_1$ and $F_2$, then \eqref{eq:QGDetSolveCondition} implies $F_0$ can be found simply by examining
\begin{align*}
	\min_{\qm}\clbracs{F_1(\qm)} \leq F_2\bracs{\omega} \leq \max_{\qm}\clbracs{F_1(\qm)},
\end{align*} 
although such a separation of $F$ will not generally be possible.
Of course, if an analytic expression for $F$ can be obtained, we can always then employ a root-finding algorithm to determine the set $F_0$.
Further to this, (the proof of) proposition \ref{prop:MMatrixDetForm} informs us that (for each fixed $\qm$) such an expression for $F$ will be solely in terms of $\omega^2$, sine and cosine functions of $\omega$.

We should not ignore the elephant in the room concerning \emph{how practical} it is to obtain, and work with, the matrix $\mathfrak{M}_\qm$ and the function $F$ in the first place.
Corollary \ref{cory:M-MatrixEntriesNoPoles} affords some insight into the sparsity of $\mathfrak{M}_{\qm}$; for every pair of vertices $v_j$ and $v_k$ that are directly connected by (at least one) edge, there are two non-zero off-diagonal entries introduced to $\mathfrak{M}_{\qm}$, whilst the diagonal entries are always not identically zero (for a connected, periodic graph).
For a given number of vertices in a connected periodic graph without loops, we thus minimise the number of non-zero off-diagonal entries when the graph is a chain --- each vertex $v_j$ has precisely two direct connections to $v_{j-1}$ and $v_{j+1}$.
The ratio $\frac{\abs{\edgeSet}}{\abs{\vertSet}}$ provides a good indicator of how ``close" to a chain graph $\graph$ is (and consequentially, how close to a diagonal matrix $\mathfrak{M}_\qm$ is) --- if $\frac{\abs{\edgeSet}}{\abs{\vertSet}} = 1$, the graph is necessarily a chain.
The opposite extreme $\frac{\abs{\edgeSet}}{\abs{\vertSet}} = \recip{2}(\abs{\vertSet}-1)$ corresponds to a fully connected graph and thus dense $\mathfrak{M}_\qm$.
For graphs with $\frac{\abs{\edgeSet}}{\abs{\vertSet}} \approx 1$, $\mathfrak{M}_{\qm}$ is relatively sparse and computing $F$ is more feasible, although still difficult analytically if $\abs{\vertSet}$ is large.
Therefore for denser $\mathfrak{M}_\qm$ and larger (in the sense of number of edges) graphs, one begins to lean toward numerical schemes based on \eqref{eq:QGGenEvalSolveNoPoles}.

Another topical question worth addressing at this point is whether, given the expectation that $\sigma$ has a band-gap structure, an alternative to finding all $\omega\in F_0$ is to only compute the spectral edges and then reconstruct $F_0$ from them.
It is known for (second order) periodic PDE problems that the edges of the spectral bands occur at the symmetry values of the quasi-momentum --- those values of $\qm$ which correspond to the periodic and anti-periodic problems (in each axis direction) on the unit cell.
If the above statement were true for the problem \eqref{eq:SingularWaveEqnQGProblem}, then the dimensionality of the problem of computing $F_0$ (hence $\sigma$) could be reliably reduced to determining $F_0^\qm$ for the aforementioned symmetry values of $\qm$.
However as discussed in \cite[Chapter 4.6]{berkolaiko2013introduction}, whilst this has been experimentally observed to be true for quantum graph problems motivated by physical structures, it is in fact untrue in general.
This extends to \eqref{eq:SingularWaveEqnQGProblem}, however \cite[Chapter 4.6]{berkolaiko2013introduction} remarks that adopting the approach of assuming the spectral edges lie at the symmetry points of the quasi-momentum does not often lead to errors in practice.
It is unclear why this is the case; the current hypothesis is that the underlying (period) graph needs to be made highly asymmetric to move the spectral edges away from the symmetry points of the quasi-momentum, and since most physical structures of interest display symmetries in their unit cells, the assumption ``holds in practice".
In our examples in section \ref{sec:ScalarExamples}, we will also see that preserving some underlying symmetries in our graphs indeed results in the symmetry values of the quasi-momentum being the boundaries of the spectral gaps.

\section{Dispersion relations for concrete graph topologies} \label{sec:ScalarExamples}
In this section we provide examples to demonstrate how the spectrum $\sigma=\sigma\bracs{-\laplacian_{\upsilon}}$ is obtained via the analysis of the $M$-matrix for the quantum graph problem \eqref{eq:SingularWaveEqnQGProblem}.
The examples are chosen to highlight the methodology and some of the remarks discussed in sections \ref{sec:QuantumGraphs} and \ref{sec:ScalarDiscussion}.
In each of these examples we will encounter a number of terms with similar forms, so to avoid repetition we define these quantities here.
Let $a\in\bracs{0,1}$ and define
\begin{align} \label{eq:SE-CommonTrigFns}
	s_a\bracs{\omega} &= \sin\bracs{a\omega}, \quad 
	c_a\bracs{\omega} = \cos\bracs{a\omega}, \quad
	\tilde{s}_a\bracs{\omega} = \sin\bracs{(1-a)\omega}, \quad 
	\tilde{c}_a\bracs{\omega} = \cos\bracs{(1-a)\omega}.
\end{align}
The quantities in \eqref{eq:SE-CommonTrigFns} are $\omega$-dependent, however throughout this section we will suppress this dependency unless providing important formulae. 

\subsection{One-Dimensional Loop} \label{ssec:Example1DLoop}
We begin with the simplest example, a chain of vertices periodic in one direction, to demonstrate how one takes the period graph of a physical singular-structure and employs proposition \ref{prop:M-MatrixEntries} to construct the $M$-matrix and extract the spectral information.
We also use the simplicity of this example to highlight the necessity of splitting edges via the introduction of artificial vertices to remove loops and edge-lengths that are rationally-related, to compliment section \ref{ssec:ArtificialVertices}.

Consider the graph $\hat{\graph}$ periodic in one direction in $\reals\times\sqbracs{0,1}$, with vertices $v_j = \bracs{j + \recip{2}, 0}^\top$ and edges $I_{j\bracs{j+1}}, \ j\in\integers$.
Since $\hat{\graph}$ is parallel to- and periodic in the $x_1$-direction, the period cell lies in $S^1$ and the quasi-momentum $\qm\in\left[-\pi,\pi\right)$ is scalar\footnote{One can simply set $\qm_2=0$ in \eqref{eq:SingularWaveEqnQGProblem} when constructing the $\qm_{jk}$ to account for the lack of periodicity in $x_2$.}.
Place identical coupling constants at each vertex, with $\alpha_j = \alpha_1>0 \ \forall v_j\in\vertSet$.
The period graph $\graph$ of $\hat{\graph}$ consists of a single vertex $v$ (without loss of generality placed at the origin) with a looping edge $I$ of length 1, with quasi-momentum $\qm_I=\qm$ on $I$ and coupling constant $\alpha_1$ at $v$.
We must introduce an artificial vertex (section \ref{ssec:ArtificialVertices}) to break the loop $I$ into two edges with irrationally-related edge lengths, producing a new metric graph $\graph^*=\bracs{\vertSet, \edgeSet}$ (on which \eqref{eq:SingularWaveEqnQGProblem} is posed) with
\begin{align*}
	\vertSet = \clbracs{ v_1 , v_2 }, \quad \edgeSet = \clbracs{ I_{12}, I_{21} },
	&\qquad \abs{I_{12}} = a, \quad \abs{I_{21}} = 1-a,  \\
	\alpha_1 >0, \quad \alpha_2 = 0,
	&\qquad \qm_{12} = \qm_{21} = \qm,
\end{align*}
where $a$ and $1-a$ are irrationally related --- taking $a = \recip{\sqrt{2}}$ would suffice, for example.
The process of moving from $\hat{\graph}$ through $\graph$ to $\graph^*$ is illustrated in figure \ref{fig:Diagram_1DExample}.
\begin{figure}[t!]
	\centering
	\begin{subfigure}[t]{0.3\textwidth}
		\centering
		\includegraphics[scale=2]{Diagram_1DLineGraph.pdf}
		\caption[]{\label{fig:Diagram_1DLineGraph} The graph $\hat{\graph}$, periodic in one dimension, consisting of integer-spaced vertices.}
	\end{subfigure}
	~
	\begin{subfigure}[t]{0.3\textwidth}
		\centering
		\includegraphics[scale=2]{Diagram_1DLineQuantumGraph.pdf}
		\caption[]{\label{fig:Diagram_1DLineQuantumGraph} The metric graph corresponding to the period graph $\graph$, containing one (looping) edge of length 1}
	\end{subfigure}
	~
	\begin{subfigure}[t]{0.3\textwidth}
		\centering
		\includegraphics[scale=2]{Diagram_1DLineComputationGraph.pdf}	
		\caption[]{\label{fig:Diagram_1DLineComputationGraph} The metric graph $\graph^*$ obtained by the introduction of a dummy vertex $v_2$, which has coupling constant 0.}
	\end{subfigure}
	\caption[The 1D chain graph studied in the example of section \ref{ssec:Example1DLoop}.]{\label{fig:Diagram_1DExample} The graphs $\hat{\graph}$, $\graph$, and $\graph^*$.}
\end{figure} \newline

Using proposition \ref{prop:M-MatrixEntries} and corollary \ref{cory:M-MatrixEntriesNoPoles} we find that
\begin{align*}
	\mathfrak{M}_{\qm} = 
	\begin{pmatrix}[1.75]
		-\omega c_a \tilde{s}_a - \omega s_a \tilde{c}_{a} + \omega^2\alpha_1 s_a \tilde{s}_a &
		\omega e^{\rmi\qm a} \tilde{s}_a + \omega e^{-\rmi\qm(1-a)} s_a \\
		\omega e^{-\rmi\qm a} \tilde{s}_a + \omega e^{\rmi\qm(1-a)} s_a &
		-\omega c_a \tilde{s}_a - \omega s_a \tilde{c}_a
	\end{pmatrix},
	\qquad
	H^{(2)} = s_a \tilde{s}_a,\\
\end{align*}
using the notation \eqref{eq:SE-CommonTrigFns}.
Since $\graph^*$ is a chain graph with only two vertices, it is easy enough to compute $\det\mathfrak{M}_{\qm}$ and solve \eqref{eq:QGDetSolveCondition} analytically, yielding
\begin{align} \label{eq:1DChainDetEqual0}
	0 = 2\omega^2 s_a\bracs{\omega} \tilde{s}_a\bracs{\omega} \bracs{ \cos\omega - \frac{\omega\alpha_1}{2}\sin\omega - \cos\qm }.
\end{align}
Notice that the factor in front of the brackets in \eqref{eq:1DChainDetEqual0} is $2\omega^2 H^{(2)}\bracs{\omega^2}$, so is zero at $\omega=0$ and at the roots of $H^{(2)}$.
Let us define $\Xi\bracs{\qm,\omega} = \cos\omega - \frac{\omega\alpha_1}{2}\sin\omega - \cos\qm$.
Since $\cos\qm$ attains every value in $\sqbracs{-1,1}$ for $\qm\in\left[-\pi,\pi\right)$, the bracketed term in \eqref{eq:1DChainDetEqual0} implies that any $\omega$ satisfying
\begin{align*}
	-1 \leq \cos\omega - \frac{\omega\alpha_1}{2}\sin\omega \leq 1,
\end{align*}
is part of the spectrum of \eqref{eq:SingularWaveEqnQGProblem}.

We now consider solutions of \eqref{eq:1DChainDetEqual0} that are also zeros of $H^{(2)}$ --- let $\omega_0$ denote one of these values, so $\omega_0\in\clbracs{\frac{n\pi}{a}, \frac{n\pi}{1-a} \setVert n\in\naturals }$.
The eigenvalue branches of $\mathfrak{M}_{\qm}$ can be computed,
\begin{align*}
	\widetilde{\beta}_{\pm, \qm}\bracs{\omega^2} &= -\omega\sin\omega + \frac{\omega^2\alpha_1}{2}s_a \tilde{s}_a \pm \omega\sqrt{ \sin^2\omega + \frac{\omega^2\alpha_1^2}{4}s_a^2 \tilde{s}_a^2 - 2s_a \tilde{s}_a\bracs{\cos\omega+\cos\qm} },
\end{align*}
however only $\widetilde{\beta}_{+, \qm}\bracs{\omega_0^2}=0$.
As such, $\omega_0$ is part of the spectrum of \eqref{eq:SingularWaveEqnQGProblem} when
\begin{align*}
	\lim_{\omega\rightarrow\omega_0}\bracs{ H^{(2)}\bracs{\omega^2} }^{-1}\widetilde{\beta}\bracs{\omega^2}_{+} = 0,
\end{align*}
which (after applying L'h\^{o}spital's rule) only occurs when
\begin{align*}
	\exists\qm_0\in\left[-\pi,\pi\right) \text{ s.t. } \cos\omega_0 - \frac{\omega_0\alpha_1}{2}\sin\omega_0 - \cos\qm_0 = 0,
\end{align*}
that is precisely when there is some $\qm_0$ such that $\Xi\bracs{\qm_0,\omega_0}=0$.
Therefore, the spectrum of \eqref{eq:SingularWaveEqnQGProblem} is fully described by those $\omega$ satisfying
\begin{align*}
	\-1 \leq \cos\omega - \frac{\omega\alpha_1}{2}\sin\omega \leq 1.
\end{align*}
We remark here that in the notation of proposition \ref{prop:MMatrixDetForm} and corollary \ref{cory:ScalarSetInclusions}, $F\bracs{\omega, \qm}$ is defined by the right-hand side of equation \eqref{eq:1DChainDetEqual0}, and we have that $\overline{F_0\setminus H_0}=\sigma$ in this case.

Breaking the looping edge and ensuring that the resulting edge-lengths are irrationally-related is necessary to obtain a full description of $\sigma$ --- failure to do so results in the loss of certain Dirichlet eigenvalues.
By way of illustration, if the looping edge is not broken then one obtains
\begin{align*}
	\det\mathfrak{M}_{\qm}\bracs{\omega^2} &= \cos\omega - \frac{\omega\alpha_1}{2}\sin\omega - \cos\qm, \\
	H^{(2)}\bracs{\omega^2} &= \omega\sin\omega, \\
	\widetilde{\beta}_{\qm}\bracs{\omega^2} &= \cos\omega - \frac{\omega\alpha_1}{2}\sin\omega - \cos\qm.
\end{align*}
This means that $\widetilde{\beta}_{0}\bracs{(2k\pi)^2}=0$ and $\widetilde{\beta}_{-\pi}\bracs{(2(k-1)\pi)^2}=0$ for $k\in\naturals$, but 
\begin{align*}
	\lim_{\omega\rightarrow 2k\pi}\bracs{ H^{(2)}\bracs{\omega^2} }^{-1}\widetilde{\beta}_{0}\bracs{\omega^2} &= \alpha_1\bracs{2k\pi}^2 \neq 0, \\
	\lim_{\omega\rightarrow 2(k-1)\pi}\bracs{ H^{(2)}\bracs{\omega^2} }^{-1}\widetilde{\beta}_{-\pi}\bracs{\omega^2} &= \alpha_1\bracs{2(k-1)\pi}^2 \neq 0,
\end{align*}
which leads one to falsely exclude $\omega=n\pi, \ n\in\naturals$ from the spectrum.
These $\omega=2k\pi, \qm=0$ and $\omega=2(k-1)\pi, \qm=-\pi$ are in fact the eigenvalues of the Dirichlet problem
\begin{align*}
	-\bracs{\diff{}{t} + \rmi\qm_{jk}}^2 \tilde{u}^{(jk)} = \omega^2 \tilde{u}^{(jk)}, \quad & y\in\interval{I_{jk}}, \quad \forall I_{jk}\in \edgeSet, \\
	u \text{ is continuous at each } &v_j \in \vertSet, \\
	u\bracs{v_j} = 0, \quad &v_j\in\vertSet,
\end{align*}
which are also solutions to \eqref{eq:SingularWaveEqnQGProblem}.
Similar problems occur when one breaks the loop into two edges of rationally-related edge lengths --- taking $a=\recip{2}$ results in similar ``loss" of the eigenvalues $\omega=2k\pi, \ k\in\naturals$, for example.
Provided that the value of $a$ is correctly (adhering to pairwise-irrational edge lengths) however, the value used does not affect the resulting conclusions, as can be seen in this example.

\subsection{``Decorated" Graph with Dependencies Arising from the Embedding} \label{ssec:EmbeddingDependentExample}
We next provide an explicit example to complement the discussion that concluded section \ref{ssec:MMatrix}, concerning our decision to bestow our graphs with an embedding prior to determination of the $M$-matrix. 
To avoid confusion in this section, the term \emph{quantum graph} will be prefixed with \emph{embedded} when we are discussing a quantum graph that has been equipped with an embedding, and prefixed with \emph{abstract} when referring to a quantum graph that has not been assigned an embedding.

Consider the embedded graph $\hat{\graph}$ in $\reals\times\sqbracs{0,1}$, with vertices
\begin{align*}
	v_1^m = \bracs{m + \recip{2}, \recip{2}}^\top, 
	&\quad v_2^m = \bracs{m + \recip{2}\bracs{1+\cos\beta}, \recip{2}\bracs{1+\sin\beta}}^\top,
\end{align*}
for a fixed angle $\beta\in\bracs{0,\pi}$, and edges $I_{1}^{m} = \sqbracs{v_1^m, v_1^{m+1}}$ and $I_{12}^m=\sqbracs{v_1^m, v_2^m}$ for $m\in\integers$.
Place a coupling constant $\alpha_1$ at each $v_1^m$, and let $v_2^m$ have zero coupling constant, for each $m$.
Then let $\graph$ be the period graph of $\hat{\graph}$ and let $\graph_{\mathcal{Q}}=\bracs{\vertSet_{\mathcal{Q}}, edgeSet_{\mathcal{Q}}}$ be the abstract quantum graph described by
\begin{align*}
	\vertSet_{\mathcal{Q}} = \clbracs{v_1, v_2},
	\qquad
	\edgeSet_{\mathcal{Q}} = \clbracs{ I_1=\sqbracs{v_1,v_1}, \ I_2=\sqbracs{v_1,v_2} },
	\qquad
	l_{11} = 1, \ l_{12} = \recip{2}.
\end{align*}
Note that $\graph_{\mathcal{Q}}$ is the abstract quantum graph that which can be embedded into $\sqbracs{0,1}^2$ to obtain $\graph$.
Further to this, $\graph_{\mathcal{Q}}$ does not contain any reference to the angle $\beta$ at which the edges $I^m_{12}$ are orientated at --- this is entirely an artefact of our decision to embed $\graph_{\mathcal{Q}}$ into $\reals\times\sqbracs{0,1}$ and obtain $\graph$.
Upon introducing an artificial vertex to break the looping edge (see example \ref{ssec:Example1DLoop}), the embedded quantum graph $\graph^*$ which we study is
\begin{align*}
	&\graph^* = \bracs{\vertSet^*, \edgeSet^*}, \quad
	\vertSet^* = \clbracs{ v_1, v_2, v_3 }, \quad
	\edgeSet^* = \clbracs{ I_{12}, I_{13}, I_{31} }, \\
	&v_1 = \bracs{\recip{2},\recip{2}}, \quad
	v_2 = \recip{2}\bracs{\cos\beta, \sin\beta}, \quad
	v_3 = \bracs{\recip{2}+a, \recip{2}},
\end{align*}
where $a$ is chosen so that the lengths of the edges ($a, 1-a,$ and $\recip{2}$) are pairwise irrationally related.
We again have scalar quasi-momentum $\qm\in\left[-\pi,\pi\right)$ with
\begin{align*}
	\qm_{13} = \qm_{31} = \qm, \quad \qm_{12} = \qm\cos\beta,
\end{align*}
and the coupling constant at $v_1$ is $\alpha_1$, whilst the coupling constants at $v_2$ and $v_3$ are zero.
We again illustrate the embedded periodic graph $\hat{\graph}$, as well as the abstract quantum graphs $\graph_{\mathcal{Q}}$ and that corresponding to $\graph^*$ in figure \ref{fig:Diagram_1DAngledEdgeExample}.
\begin{figure}[b!]
	\centering
	\begin{subfigure}[t]{0.3\textwidth}
		\centering
		\includegraphics[scale=1.85]{Diagram_1DAngledEdge-Embedded.pdf}
		\caption[]{\label{fig:Diagram_1DAngledEdge-Embedded} The graph $\hat{\graph}$, periodic in one dimension, consisting of integer-spaced vertices with an edge ``hanging" at an angle $\beta$.}
	\end{subfigure}
	~
	\begin{subfigure}[t]{0.3\textwidth}
		\centering
		\includegraphics[scale=1.85]{Diagram_1DAngledEdge-Quantum.pdf}
		\caption[]{\label{fig:Diagram_1DAngledEdge-Quantum} The abstract quantum graph $\graph_{\mathcal{Q}}$ corresponding to the embedded period graph $\graph$.}
	\end{subfigure}
	~
	\begin{subfigure}[t]{0.3\textwidth}
		\centering
		\includegraphics[scale=1.85]{Diagram_1DAngledEdge-Computation.pdf}	
		\caption[]{\label{fig:Diagram_1DAngledEdge-Computation} The abstract quantum graph corresponding to the embedded, periodic graph $\graph^*$ that we study.}
	\end{subfigure}
	\caption[The graph and period graph studied in the example of section \ref{ssec:EmbeddingDependentExample}.]{\label{fig:Diagram_1DAngledEdgeExample} The graphs $\graph$, $\graph_{\mathcal{Q}}$, and $\graph^*$.}
\end{figure}

Applying proposition \ref{prop:M-MatrixEntries} and corollary \ref{cory:M-MatrixEntriesNoPoles} yields
\begin{align*} 
	\mathfrak{M}_\qm &=
	\begin{pmatrix}[2.5]
		\begin{split}
			&-c_a \tilde{s}_a s_{\recip{2}} 
			- s_a \tilde{c}_a s_{\recip{2}}  \\
			&- s_a \tilde{s}_a c_{\recip{2}}
			+ \omega\alpha_1 s_a \tilde{s}_a s_{\recip{2}}
		\end{split} &
		\exp\bracs{\dfrac{\rmi\qm\cos\beta}{2}}s_a \tilde{s}_a &
		e^{\rmi\qm a}\tilde{s}_a s_{\recip{2}} + e^{-\rmi\qm(1-a)}s_a s_{\recip{2}} \\
		\begin{split}		
			& \exp\bracs{-\dfrac{\rmi\qm\cos\beta}{2}}s_a \tilde{s}_a 
		\end{split} &
		-s_a \tilde{s}_a c_{\recip{2}} &
		0 \\
		\begin{split}
			& e^{-\rmi\qm a}\tilde{s}_a s_{\recip{2}} + e^{\rmi\qm(1-a)}s_a s_{\recip{2}} 
		\end{split} &
		0 &
		-\bracs{c_a \tilde{s}_a s_{\recip{2}} + s_a \tilde{c}_a s_{\recip{2}}}
	\end{pmatrix}, \\
	H^{(2)} &= \omega^{-1} s_a \tilde{s}_a s_{\recip{2}}.
\end{align*}
The angle $\beta$ has entered into the form of the $M$-matrix due to our embedding, however we shall see that the spectrum $\sigma$ is independent of $\beta$, as one would expect from examining its abstract periodic quantum graph in the alternative manner described in section \ref{ssec:MMatrix}.

Solving \eqref{eq:QGDetSolveCondition} yields
\begin{align} \label{eq:EmbeddedGraphDetSolveCondition}
	0 = 2s_a^2\bracs{\omega} \tilde{s}_a^2\bracs{\omega} s_2^2\bracs{\omega} c_2\bracs{\omega}
	\sqbracs{ \cos\qm + \recip{2} - \frac{3}{2}\cos\omega + \frac{\alpha_1\omega}{2}\sin\omega }.
\end{align}
We can identify 
\begin{align*}
	F\bracs{\qm,\omega} = 2 s_a \tilde{s}_a c_2
	\sqbracs{ \cos\qm + \recip{2} - \frac{3}{2}\cos\omega + \frac{\alpha_1\omega}{2}\sin\omega },
\end{align*}
however it will be slightly more convenient for us to work with $\Xi\bracs{\omega} = \frac{3}{2}\cos\omega - \frac{\alpha_1\omega}{2}\sin\omega$.
The rest of the spectrum is determined by the zeros of $F$, which occur at either $\omega=(2k-1)\pi$ for $k\in\naturals$ or when there exists at least one $\qm$ such that $\Xi\bracs{\omega} = \cos\theta + \recip{2}$.
The latter part of the spectrum consists of those $\omega$ such that
\begin{align*}
	\min_{\qm\in\left[-\pi,\pi\right)}\bracs{\cos\theta + \recip{2}} &\leq \Xi\bracs{\omega} 
	\leq \max_{\qm\in\left[-\pi,\pi\right)} \bracs{\cos\theta + \recip{2}}, \\
	\Leftrightarrow & \abs{ 3\cos\omega - \alpha_1\omega\sin\omega + 1 } \leq 2, 
\end{align*}
these points are visualised in figure \ref{fig:1DDecoratedGraph}.
Note that the points $\omega=(2k-1)\pi$ are included here too.
\begin{figure}[b!]
	\centering
	\begin{subfigure}[t]{0.45\textwidth}
		\centering
		\includegraphics[scale=0.525]{1DDecoratedGraph_alpha1.pdf}
		\caption[]{\label{fig:1DDecoratedGraph_alpha1} The values of $\omega$ which solve \eqref{eq:EmbeddedGraphDetSolveCondition} with $\alpha_1=1$. No zeros of $H^{(2)}$ form part of the spectrum in this case.}
	\end{subfigure}
	~
	\begin{subfigure}[t]{0.45\textwidth}
		\centering
		\includegraphics[scale=0.525]{1DDecoratedGraph_alpha0-25.pdf}
		\caption[]{\label{fig:1DDecoratedGraph_alpha0-25} The values of $\omega$ which solve \eqref{eq:EmbeddedGraphDetSolveCondition} with $\alpha_1=\recip{4}$. With $\alpha_1$ this small, some of the zeros of $H^{(2)}$ form part of the spectrum.}
	\end{subfigure}
	\caption[The spectrum of \eqref{eq:SingularScalarWaveEqn} on the geometry of section \ref{ssec:EmbeddingDependentExample}, and the corresponding poles of the determinant of the $M$-matrix.]{\label{fig:1DDecoratedGraph} The values of $\omega$ which solve \eqref{eq:EmbeddedGraphDetSolveCondition}, using $a=\recip{\sqrt{2}}$. Changing the value of $\alpha$ effects how many zeros of $H^{(2)}$ are included in the spectrum.}
\end{figure}

Zeros of $H^{(2)}$ occur at $\omega= 2n\pi, \frac{n\pi}{a}, \frac{n\pi}{1-a}$, but examining the limit \eqref{eq:EigenvalueBranchLimit} reveals that if $H^{(2)}\bracs{\omega_0^2}=0$, $\omega_0$ is part of the spectrum only when there exists a $\qm_0\in\left[-\pi,\pi\right)$ such that $\Xi\bracs{\omega_0}=\cos\theta_0+\recip{2}$.
That is, we once again require $F\bracs{\qm,\omega}=0$ for a root of $H^{(2)}$ to be part of the spectrum.
It is also worth noting that (in figure \ref{fig:1DDecoratedGraphEvalBranches-Thetas}) there are two eigenvalue branches $\widetilde{\beta}_j^{\qm}$ which are zero at $\omega_0$ (the third being non-zero at $\omega_0$).
The limit \eqref{eq:EigenvalueBranchLimit} exists for both branches, however only for one is it zero --- these branches and the corresponding limits in the vicinity of root $\omega_0=\pi\sqrt{2}$ are plotted in figure \ref{fig:1DDecoratedGraphEvalBranches-Thetas}.
\begin{figure}[b!]
	\centering
	\includegraphics[width=\textwidth]{1DDecoratedGraphEvalBranches-Thetas.pdf}
	\caption[Eigenvalue branches of the $M$-matrix near a pole of the determinant, for the geometry of section \ref{ssec:EmbeddingDependentExample}.]{\label{fig:1DDecoratedGraphEvalBranches-Thetas} Eigenvalue branches of the matrix $\mathfrak{M}_{\qm}$ near $\omega_0 = \pi\sqrt{2}$, which is a root of $H^{(2)}$. The value $\qm_0\approx-0.691\pi$ solves $\Xi\bracs{\omega_0}=\cos\qm_0+\recip{2}$, and the limit \eqref{eq:EigenvalueBranchLimit} is zero. For all other values of $\qm$ however, the limit \eqref{eq:EigenvalueBranchLimit} is non-zero.}
\end{figure}

In summary, $\sigma$ consists of those $\omega^2$ such that
\begin{align*}
	\abs{ 3\cos\omega - \alpha_1\omega\sin\omega + 1 } \leq 2,
\end{align*}
which again is precisely those $\omega$ such that there is some $\qm$ such that $\bracs{\qm,\omega}\in F^{-1}\bracs{\clbracs{0}}$.
Furthermore, and as expected, the spectrum does not depend on $\beta$ despite the fact that the $M$-matrix for each operator on $\graph^*$ does.
The spectrum does of course differ from the spectrum of the quantum graph in section \ref{ssec:Example1DLoop} due to the presence of the ``decoration" $I_{12}$, although this difference only depends on the length this additional edge (and were the coupling constant at $v_2$ non-zero, this too).

\subsection{Cross in the plane geometry} \label{ssec:ExampleCrossInPlane}
Our final example is a two-dimensional graph whose period cell represents a lattice-like structure in $\reals^2$.
Consider the embedded, periodic graph $\hat{\graph}$ defined as follows --- for each $\bracs{n,m}\in\integers^2$ define
\begin{align*}
	v^{(n,m)} &= \bracs{n+\recip{2}, m+\recip{2}}, \quad
	I_{\mathrm{left}}^{\bracs{n,m}} = \sqbracs{v^{\bracs{n,m}}, v^{\bracs{n+1,m}}}, \quad
	I_{\mathrm{up}}^{\bracs{n,m}} = \sqbracs{v^{\bracs{n,m}}, v^{\bracs{n,m+1}}}, \\
	\hat{\vertSet} &= \clbracs{v^{\bracs{n,m}} \setVert \bracs{n,m}\in\integers^2}, \quad
	\hat{\edgeSet} = \clbracs{ I_{\mathrm{l}}^{\bracs{n,m}}, I_{\mathrm{u}}^{\bracs{n,m}} \setVert \bracs{n,m}\in\integers^2}, \quad
	\hat{\graph} = \bracs{\hat{\vertSet}, \hat{\edgeSet}}.
\end{align*}
Place a coupling constant $\alpha^{\bracs{n,m}} =: \alpha_3>0$ at each $v^{(n,m)}$.
The period graph $\graph$ occupies $\left[0,1\right)^2$ and consists of a single vertex with two looping edges of length 1.
Breaking the loops by introducing two artificial vertices takes us to the quantum graph
\begin{align*}
	\vertSet^* = \clbracs{v_1, v_2, v_3}, \quad
	\edgeSet^* = \clbracs{I_{13}, I_{31}, I_{23}, I_{32}}, \quad
	\graph^* = \bracs{\vertSet^*, \edgeSet^*},
\end{align*}
with
\begin{align*}
	l_{13} = b, \quad l_{31} = \tilde{b} := 1-b, \quad 
	l_{23} = a, \quad l_{32} = \tilde{a} := 1-a, \qquad
	\qm_{13} = \qm_{31} = \qm_2, \quad \qm_{23} = \qm_{32} = \qm_1,
\end{align*}
and coupling constant $\alpha_3$ at $v_3$ (and zero coupling constants at the dummy vertices $v_1$ and $v_2$).
Using corollary \ref{cory:M-MatrixEntriesNoPoles} we set $H^{(2)}\bracs{\omega^2} = s_a\bracs{\omega} s_b\bracs{\omega} \tilde{s}_a\bracs{\omega} \tilde{s}_b\bracs{\omega}$, and obtain
\begin{align*}
	\mathfrak{M}_{\qm} & \bracs{\omega^2} = \\
	&
	\begin{pmatrix}[2.5]
		-\omega s_a \tilde{s}_a \bracs{ s_b \tilde{c}_b + c_b \tilde{s}_b } &
		0 &
		\begin{split}
			&\omega s_a \tilde{s}_a \bracs{ e^{\rmi\qm_2\tilde{b}}s_b + e^{-\rmi\qm_2 b}\tilde{s}_b }
		\end{split} \\
		0 &
		-\omega s_b \tilde{s}_b \bracs{ s_a \tilde{c}_a + c_a \tilde{s}_a } &
		\begin{split}
			&\omega s_b \tilde{s}_b \bracs{ e^{\rmi\qm_1\tilde{a}}s_a + e^{-\rmi\qm_1 a}\tilde{s}_a } 
		\end{split} \\
		\omega s_a \tilde{s}_a \bracs{ e^{-\rmi\qm_2\tilde{b})}s_b + e^{\rmi\qm_2 b}\tilde{s}_b } &
		\omega s_b \tilde{s}_b \bracs{ e^{-\rmi\qm_1\tilde{a}}s_a + e^{\rmi\qm_1 a}\tilde{s}_a } &
		\begin{split}
			&-\omega ( s_a s_b \tilde{s}_a \tilde{c}_b 
			+ s_a s_b \tilde{c}_a \tilde{s}_b \\ 
			& + s_a c_b \tilde{s}_a \tilde{s}_b
			+ c_a s_b \tilde{s}_a \tilde{s}_b \\
			& - \omega\alpha_3 s_a s_b \tilde{s}_a \tilde{s}_b )
		\end{split}
	\end{pmatrix}.
\end{align*}

Examining \eqref{eq:QGDetSolveCondition} yields
\begin{align} \label{eq:ExampleThickVertexSolution}
	0 = \omega^3 \bracs{H^{(2)}\bracs{\omega^2}}^2 \tilde{s}_b^2\bracs{\omega} \sin\bracs{\omega} 
	\bracs{ 4\cos\bracs{\frac{\qm_1+\qm_2}{2}}\cos\bracs{\frac{\qm_1-\qm_2}{2}} + \omega\alpha_3\sin\omega - 4\cos\omega }
\end{align}
although for ease we also define
\begin{align*}
	\Xi\bracs{\omega} := \cos\omega - \frac{\alpha_3\omega}{4}\sin\omega.
\end{align*}
Note that any $\omega_0$ for which $-1\leq\Xi\bracs{\omega_0}\leq 1$, there is some $\qm_0$ such that the bracketed factor in \eqref{eq:ExampleThickVertexSolution} is zero --- this again encompasses the case when $\sin\omega=0$, as in this case $\Xi\bracs{\omega} = \pm 1$.
Examination of the eigenvalue branches then produces a familiar conclusion; if $H^{(2)}\bracs{\omega_0^2}=0$, $\omega_0$ forms part of the spectrum of \eqref{eq:SingularWaveEqnQGProblem} if and only if there exists a $\qm_0$ such that
\begin{align} \label{eq:ExampleThickVertexSolutionReduced}
	\Xi\bracs{\omega}=\cos\bracs{\frac{\qm_1+\qm_2}{2}}\cos\bracs{\frac{\qm_1-\qm_2}{2}},
\end{align}
that is when $F\bracs{\qm_0,\omega_0}=0$.
As a result, the spectrum consists of exactly those $\omega$ such that
\begin{align*}
	-1 \leq \cos\omega - \frac{\alpha_3\omega}{4}\sin\omega \leq 1.
\end{align*}
If $\alpha_3=0$, we observe that the spectral bands touch and there are no spectral gaps.
By introducing (geometric) contrast through the coupling constants, gaps between the spectral edges open.
For any $\alpha_3>0$, the spectral bands $I_n$ satisfy $I_n\subset\sqbracs{(n-1)\pi, n\pi}$, each with left-endpoint $(n-1)\pi$ and a right-endpoint strictly less than $n\pi$.
This behaviour is reversed for $\alpha_3<0$, the bands having right-endpoint $n\pi$ and left-endpoint strictly greater than $(n-1)\pi$.
For $\alpha\leq-2$ there is even a gap between an isolated eigenvalue at $0$ and the beginning of the band $I_1$ --- this provides us with a case where the inclusion $\overline{F_0\setminus H_0}\subset\sigma$ in corollary \ref{cory:ScalarSetInclusions} is strict.
However as mentioned in section \ref{sec:ScalarEqnChapterIntro}, $\alpha<0$ does not correspond to any physical material.

In addition to recovering the spectrum of \eqref{eq:SingularWaveEqnQGProblem}, we can use \eqref{eq:ExampleThickVertexSolution} and \eqref{eq:QGGenEvalSolveNoPoles} to recover the eigenfunctions too.
For a given $\qm$, equation \eqref{eq:ExampleThickVertexSolutionReduced} can be solved for (a solution) $\omega=\omega_0$ (of course, we could instead choose an eigenvalue $\omega$ and compute the corresponding quasi-momentum for which $\omega\in\sigma_{\qm}$).
This $\omega_0$ corresponds to an eigenvalue $\omega_0^2$ of \eqref{eq:SingularWaveEqnQGProblem}, but also implies that there exists a $w\in\complex^{\abs{\vertSet}}$ such that \eqref{eq:QGGenEvalSolveNoPoles} holds at $\omega=\omega_0$.
We can compute the eigenvector(s) $w\in\complex^{\abs{\vertSet}}$ of $\mathfrak{M}_{\qm}\bracs{\omega_0^2}$ corresponding to its zero eigenvalue.
Identifying $w = \dmap u$ as the Dirichlet data of the eigenfunction $u$, and given \eqref{eq:EdgeEqnGeneralSolution}, the edge functions $u^{(jk)}$ can be obtained.
Some examples of the result of this process are plotted in figure \ref{fig:CrossInPlane-EdgePlot}; one can observe continuity of the eigenfunctions at the central vertex, whilst their incoming derivatives adhere to the Wentzell condition.
\begin{figure}[t!]
	\centering
	\begin{subfigure}[t]{0.45\textwidth}
		\centering
		\includegraphics[width=\textwidth]{CrossInPlane_EdgePlot-R-a1.pdf}
		\caption[]{\label{fig:CrossInPlane_EdgePlot-R-a1} The real part of the eigenfunction corresponding to $\omega_0=0.63936, \alpha=1$.}
	\end{subfigure}
	~
	\begin{subfigure}[t]{0.45\textwidth}
		\centering
		\includegraphics[width=\textwidth]{CrossInPlane_EdgePlot-I-a1.pdf}
		\caption[]{\label{fig:CrossInPlane_EdgePlot-I-a1} The imaginary part of the eigenfunction corresponding to $\omega_0=0.63936, \alpha=1$.}
	\end{subfigure}
	\vskip\baselineskip
	\begin{subfigure}[t]{0.45\textwidth}
		\centering
		\includegraphics[width=\textwidth]{CrossInPlane_EdgePlot-R-a4.pdf}
		\caption[]{\label{fig:CrossInPlane_EdgePlot-R-a4} The real part of the eigenfunction corresponding to $\omega_0=0.44812, \alpha=4$.}
	\end{subfigure}
	~
	\begin{subfigure}[t]{0.45\textwidth}
		\centering
		\includegraphics[width=\textwidth]{CrossInPlane_EdgePlot-I-a4.pdf}
		\caption[]{\label{fig:CrossInPlane_EdgePlot-I-a4} The imaginary part of the eigenfunction corresponding to $\omega_0=0.44812, \alpha=4$.}
	\end{subfigure}	
	\caption[Eigenfunctions of \eqref{eq:SingularScalarWaveEqn} on the cross-in-the-plane geometry.]{\label{fig:CrossInPlane-EdgePlot} Plots of the eigenfunctions corresponding to the eigenvalue $\omega_0=0.63936$  when $\alpha=1$, and $\omega_0=0.44812$ when $\alpha=4$. Both eigenvalues are attained when $\qm=\bracs{\frac{\pi}{4},\frac{\pi}{4}}^\top$, and the edge functions are plotted above the graph $\graph$ in the $\bracs{x_1,x_2}$-plane. We observe the expected continuity at the vertices, adherence to the Wentzell condition at $v_3$, and matching derivatives at the artificial vertices used to split the edges.}
\end{figure}
At the artificial vertices, we have matching of the incoming edge functions \emph{and} their derivatives, consistent with the zero coupling constant placed at dummy vertices.
To round off the analysis, quantities such as the integrated density of states (IDoS) and density of states (DoS) can also be estimated from \eqref{eq:ExampleThickVertexSolution}, as shown in figure \ref{fig:CrossInPlane_ScalarDoS} along with a display of the band-gap structure of the spectrum.
\begin{figure}[t!]
	\begin{subfigure}[t]{0.45\textwidth}
		\centering
		\includegraphics[scale=0.5]{CrossInPlane_ScalarDoS_alpha1-00.pdf}
		\caption[]{\label{fig:CrossInPlane_ScalarDoS_alpha1-00} The (relative) integrated density of states (IDoS), density of states (DoS) and spectrum for the system with $\alpha_3=1$.}
	\end{subfigure}
	~
	\begin{subfigure}[t]{0.45\textwidth}
		\centering
		\includegraphics[scale=0.5]{CrossInPlane_ScalarDoS_alpha4-00.pdf}
		\caption[]{\label{fig:CrossInPlane_ScalarDoS_alpha4-00} The (relative) integrated density of states (IDoS), density of states (DoS) and spectrum for the system with $\alpha_3=4$.}
	\end{subfigure}	
	\caption[The spectrum and density of states for the problem \eqref{eq:SingularScalarWaveEqn} on the cross-in-the-plane geometry.]{\label{fig:CrossInPlane_ScalarDoS} The (relative) IDoS, DoS, and spectrum for the graph topology in section \ref{ssec:ExampleCrossInPlane}.
	The relative IDoS at the value $x$ is defined as the IDoS at the value $x$ minus $\left\lfloor\frac{x}{\pi}\right\rfloor\bracs{2\pi}^2$.}
\end{figure}
We observe that the spectrum concentrates in the centre of each band, as is to be expected from the symmetry of the geometry.

\tstk{the Sob spaces stuff is all done with $\kt$ grads, even though in this chapter we only need $\theta$ grads. This is largely to avoid repetition in the following chapter, but does mean that we are doing more work than necessary here. It also means things appear backwards, as we first did $\theta$-grads as motivation/a basis for then considering $\kt$-grads, curls, and divergence (free). TLDR; I could change this to $\theta$-grads and move $\kt$-grads to chapter 4's appendix, only do $\kt$-grads here and not do them in chapter 4, or any other sensible policy.}

\section{Sobolev functions on the edges of an embedded graph} \label{ssec:ScalarSobSpaces}
\tstk{proper introductory section here? Also, need to change from $x_1$-parallel to $x_2$-parallel throughout! Otherwise things are going to get really annoying}

\tstk{have done some setup to be able to lead with...}

Thematic throughout our analysis of gradients of zero and Sobolev functions will be the following procedure; we will first aim to understand gradients of zero on a single edge $I_{jk}$ that is parallel to (one of) the co-ordinate axes, and then employ rotation ideas to generalise our arguments to edges at any angle to the axes.
Next, we demonstrate that gradients of zero on $\graph$ can be built up from those on the individual edges --- this is unsurprising given that $\ddmes$ is just the sum of the individual singular measures supporting each edge.
Once we understand gradients of zero on edges, we can then understand the tangential gradients on the edges and on $\graph$ by following a similar line of reasoning --- working on the edges first and then looking at the implications for functions on the entire graph.
We will also have to consider (and analyse) the behaviour of gradients of zero and tangential gradients at the vertices, induced by $\nu$.
This finally allows us to understand Sobolev functions and their tangential gradients (and gradients of zero) on $\graph$ with respect to the measure $\dddmes$.
This approach will also guide us when we come to examine the Sobolev spaces of curls in section \tstk{ref!}.

As promised, we begin with an examination of gradients of zero. \tstk{we will come to consider $\kappa$-gradients in a later chapter, so to save time we might as well deal with them here. We can set $\kappa=0$, or ignore the third component in this section without any harm coming to us.}

\subsection{Gradients of Zero} \label{ssec:muGradZero}
In this section we will characterise gradients of zero with respect to the measures $\ddmes, \nu$, and $\dddmes$, in that order.
Throughout, we denote by $\ograd$ the $\ktgrad$ operator with $\kt=\bracs{0,0}$.
Given proposition \ref{prop:ZeroInvariantUnderQM-Wavenumber}, without loss of generality we can always take any approximating sequence $\phi_n$ for a gradient of zero $g$ to be such that $\ograd\phi_n\rightarrow g$, as opposed to $\ktgrad\phi_n\rightarrow g$.
\begin{prop}[Gradients of Zero on a Segment Parallel to the $x_2$-axis] \label{prop:3DGradZeroParallel}
	Suppose that the edge $I_{jk}$ is parallel to the $x_2$-axis.
	Then 
	\begin{align*}
		\gradZero{\ddom}{\lambda_{jk}} &= 
		\clbracs{ \bracs{g,0,0}^\top	\setVert g\in\ltwo{\ddom}{\lambda_{jk}} }.
	\end{align*}
\end{prop}
\begin{proof}
	This is a version of the argument in \cite[Section~3.1]{zhikov2000extension}, given proposition \ref{prop:GradZeroInvarientUnderQM} we can consider (without loss of generality) $\kt=\bracs{0,0}$ throughout.
	This argument is also one particular version of the argument in the proof of proposition \ref{prop:RotationOfEdgeGradients}, which we present in detail below.
\end{proof}

\begin{prop} \label{prop:3DGradZeroRotated}
	Let $I_{jk}$ be an edge of $\graph$.
	Then
	\begin{align*}
		\gradZero{\ddom}{\lambda_{jk}} 
		&= \clbracs{ g_{jk}\hat{n}_{jk} \setVert g_{jk}\in\ltwo{\ddom}{\lambda_{jk}} } \\
		&= \clbracs{ \begin{pmatrix} R_{jk}^{\top} & 0 \\ 0 & 1 \end{pmatrix} \bracs{g_{jk},0,0}^\top \setVert g_{jk}\in\ltwo{\ddom}{\lambda_{jk}} } \\
		&= \clbracs{ g_{jk}\in\ltwo{\ddom}{\ddmes}^2 \setVert g_{jk}\cdot e_{jk} = 0 }.
	\end{align*}
\end{prop}
Note that the three sets on the right hand side are all equal by definition of $n_{jk}, e_{jk}$, and $R_{jk}$.
As such, we will demonstrate the equality on the first line in the proof. 
\begin{proof}
	Clearly, if $g=\bracs{0,0,g_3}^\top\in\gradZero{\ddom}{\lambda_{jk}}$, then \eqref{eq:GradZeroSequenceDef} implies that $g_3=0$.
	
	Next, suppose that $g=g_{jk}\hat{e}_{jk}\in\gradZero{\ddom}{\lambda_{jk}}$, and take an approximating sequence $\phi_n$ for $g$.
	Given proposition \ref{prop:ZeroInvariantUnderQM-Wavenumber}, this implies that
	\begin{align*}
		\phi_n\lconv{\ltwo{\ddom}{\lambda_{jk}}}0, \quad
		\ograd\phi_n \lconv{\ltwo{\ddom}{\lambda_{jk}}^3} g\hat{e}_{jk}.
	\end{align*}
	Therefore, $\ograd\phi_n\cdot\hat{e}_{jk}\rightarrow g$, and thus
	\begin{align*}
		\int_0^{\abs{I_{jk}}} \abs{\diff{\phi_n}{y}\bracs{r_{jk}(y)} - g\bracs{r_{jk}(y)} }^2 \ \md y
		&= \integral{I_{jk}}{ \abs{\ograd\phi_n\cdot\hat{e}_{jk} - g}^2 }{\lambda_{jk}} \toInfty{n}, \\
		\implies \diff{\phi_n}{y}\bracs{r_{jk}(y)} &\lconv{\ltwo{\interval{I_{jk}}}{y}} g\bracs{r_{jk}(y)}.
	\end{align*}
	We also observe that $\phi_n\bracs{r_{jk}(y)}\rightarrow 0$ in $\ltwo{\interval{I_{jk}}}{y}$, and thus $g\bracs{r_{jk}(y)}$ is the derivative (in the classical $\gradSob{\interval{I_{jk}}}{y}$-sense) of the zero function, and thus $g=0$.
	
	Finally, suppose that $g\in\smooth{\ddom}$ and consider the smooth function $\phi(x) = \bracs{\bracs{x-v_j}\cdot n_{jk}}g(x)$.
	Then we have that $\ograd\phi_n = \bracs{\bracs{x-v_j}\cdot n_{jk}}\ograd g + g\hat{n}_{jk}$, and notice that $\bracs{x-v_j}\cdot n_{jk}	=0$ when $x\in I_{jk}$.
	It is now clear that $\phi=0$ and $\ograd\phi = g\hat{n}_{jk}$ on $I_{jk}$, so $g\hat{n}_{jk}\in\gradZero{\ddom}{\lambda_{jk}}$.
	By density of $\smooth{\ddom}$ in $\ltwo{\ddom}{\lambda_{jk}}$, we can conclude that $g\hat{n}_{jk}\in\gradZero{\ddom}{\lambda_{jk}}$ for every $g\in\ltwo{\ddom}{\lambda_{jk}}$.
\end{proof}
Taking $R_{jk}$ as the identity matrix to recover proposition \ref{prop:3DGradZeroParallel}.

We now focus on demonstrating that the set of gradients of zero on the entirety of $\graph$ is formed from gradients of zero on each edge.
That is, we look to prove the following characterisation of $\gradZero{\ddom}{\ddmes}$:
\begin{prop}[Characterisation of $\gradZero{\ddom}{\ddmes}$] \label{prop:3DGradZeroChar}
	For an embedded graph $\graph$ in $\ddom$, we have that
	\begin{align*}
		\gradZero{\ddom}{\ddmes} &= \clbracs{ g\in\ltwo{\ddom}{\ddmes}^3 \setVert g^{(jk)}\in\gradZero{\ddom}{\lambda_{jk}} \ \forall I_{jk}\in\edgeSet }.
	\end{align*}
\end{prop}
We will denote $B := \clbracs{ g\in\ltwo{\ddom}{\ddmes}^3 \setVert g^{(jk)}\in\gradZero{\ddom}{\lambda_{jk}} \ \forall I_{jk}\in\edgeSet }$ for the time being.
In order to prove proposition \ref{prop:3DGradZeroChar} we will need some supporting results, however the argument can be sketched out like so.
Showing that $\gradZero{\ddom}{\ddmes}\subset B$ is straightforward due to the definition of $\ddmes$ and that the norms we are interested in are related:
\begin{align*}
	\norm{\cdot}_{\ltwo{\ddom}{\lambda_{jk}}} &\leq \norm{\cdot}_{\ltwo{\ddom}{\ddmes}}, \\
	\norm{\cdot}_{\ltwo{\ddom}{\ddmes}}^2 &= \sum_{v_j\in\vertSet}\sum_{j\conLeft k}\norm{\cdot}_{\ltwo{\ddom}{\lambda_{jk}}}^2.
\end{align*}
The reverse inclusion is slightly more technical due to the fact that we have to form an approximating sequence (that converges on all of $\graph$) from a set of approximating sequences that each converge on one particular edge.
However we cannot simply extend an approximating sequence on an edge $I_{jk}$ by zero to the whole graph (as it is no longer guaranteed to be smooth), so have to smooth this sequence to zero over some small region around $I_{jk}$.
This ``smoothing" requires us to always have some non-zero distance between the edge $I_{jk}$ and all other edges of $\graph$, which is a non-trivial process if the support of the approximating sequence is close to one of the vertices $v_j$ or $v_k$.
Having overcome this obstacle, one can show that any $g_{jk}\in\gradZero{\ddom}{\lambda_{jk}}$ \emph{can} be extended by zero to obtain a function $g\in\gradZero{\ddom}{\ddmes}$, and then using linearity of the subspace $\gradZero{\ddom}{\ddmes}$, the proof will be complete.

Before continuing, we introduce two families of smooth functions that we shall make use of during the proof of proposition \ref{prop:3DGradZeroChar}.
Let $\eta\in\smooth{\ddom}$ have the properties
\begin{align*}
	0 \leq \eta(x) \leq 1, \quad
	\eta(x) = 0 \text{ when } \abs{x} \leq 1, \quad
	\eta(x) = 1 \text{ when } \abs{x} \geq 2.
\end{align*}
Then define
\begin{align*}
	\eta_j(x) = \eta\bracs{x-v_j}, \quad
	\eta_j^n(x) = \eta_j\bracs{nx},
\end{align*}
which are both smooth functions by composition.
The functions $\eta_j^n$ will enable us to extend functions defined on one edge $I_{jk}$ to the whole of $\graph$ without worrying about proximity to the vertex $v_j$.
Notice that $\eta_j^n\rightarrow 1 \toInfty{n}$ in $\ltwo{\ddom}{\ddmes}$ since
\begin{align*}
	\integral{\ddom}{ \abs{\eta_j^n - 1}^2 }{\ddmes} &\leq
	\integral{ B_{2/n}\bracs{v_j} }{}{\ddmes}
	= \ddmes\bracs{ B_{2/n}\bracs{v_j} } \leq \frac{4\abs{\edgeSet}}{n}.
\end{align*}
Additionally, we have that $\eta_j^n$ also converges in $\ltwo{\ddom}{\dddmes}$ to the function
\begin{align*}
	\tilde{\charFunc{j}} = \begin{cases} 1 & x\neq v_j, \\ 0 & x=v_j, \end{cases}
\end{align*}
since
\begin{align*}
	\norm{\eta_j^n - \tilde{\charFunc{j}}}_{\ltwo{\ddom}{\ddmes}}^2
	&= \norm{\eta_j^n - 1}_{\ltwo{\ddom}{\ddmes}}^2
	+ \integral{\ddom\setminus\clbracs{v_j}}{ \abs{\eta_j^n - 1}^2 }{\nu} \\
	&= \norm{\eta_j^n - 1}_{\ltwo{\ddom}{\ddmes}}^2 \rightarrow 0.
\end{align*}
Unsurprisingly we also need a family of smooth functions to help us ``smooth off" any approximating sequences.
Let $I_{jk}\in\edgeSet$, $\eps>0$, and set
\begin{align} \label{eq:ShortenedEdgeDef}
	I_{jk}^\eps := \clbracs{ x\in I_{jk} \setVert \mathrm{dist}\bracs{x,\partial I_{jk}}\leq\recip{\eps} }.
\end{align}
Let $\chi_{jk}^\eps\in\smooth{\ddom}$ be such that
\begin{align*}
	0 \leq \chi_{jk}^n \leq 1, \quad
	\chi_{jk}^\eps(x) = 1 \text{ when } \mathrm{dist}\bracs{x, I_{jk}^\eps}\leq \recip{3\eps}, \quad
	\chi_{jk}^\eps(x) = 0 \text{ when } \mathrm{dist}\bracs{x, I_{jk}^\eps}\geq \frac{2}{3\eps}.
\end{align*}
As $\graph$ is finite, we can assume without loss of generality that the only edge of $\graph$ that $\supp\bracs{\chi_{jk}^\eps}$ intersect is $I_{jk}$ (otherwise, we just rescale the argument).
We can also assemble $\chi_{jk}^\eps$ such that $\abs{ \ograd\chi_{jk}^{\eps} } \leq c\eps$ for some $c\geq 0$ independent of $\eps$.
We can check the convergence of $\chi_{jk}^\eps \toInfty{\eps}$ in $\ltwo{\ddom}{\ddmes}$ to the characteristic function of the edge $I_{jk}$, denoted by $\charFunc{jk}$;
\begin{align*}
	\integral{\ddom}{ \abs{\chi_{jk}^\eps - \charFunc{jk}}^2 }{\ddmes}
	&= \integral{I_{jk}}{ \abs{\chi_{jk}^\eps - \charFunc{jk}}^2 }{\lambda_{jk}}
	\leq \integral{I_{jk}\cap\clbracs{\chi_{jk}^\eps\leq 1}}{}{\lambda_{jk}}
	= \frac{2}{3\eps} \rightarrow 0 \toInfty{\eps}.
\end{align*}
We also have that $\chi_{jk}^\eps$ converges to the characteristic function $\charFunc{jk}^\circ$ of the interior of $I_{jk}$ in $\ltwo{\ddom}{\dddmes}$, since
\begin{align*}
	\integral{\ddom}{ \abs{\chi_{jk}^\eps - \charFunc{jk}^\circ}^2 }{\dddmes}
	&= \integral{\ddom}{ \abs{\chi_{jk}^\eps - \charFunc{jk}^\circ}^2 }{\ddmes}
	= \integral{\ddom}{ \abs{\chi_{jk}^\eps - \charFunc{jk}}^2 }{\ddmes}.
\end{align*}

We can now prove proposition \ref{prop:3DGradZeroChar} --- the inclusion $\gradZero{\ddom}{\ddmes}\subset B$ follows immediately.
\begin{lemma} \label{lem:3DGradZeroSubsetB}
	\begin{align*}
		\gradZero{\ddom}{\ddmes} \subset B.
	\end{align*}
\end{lemma}
\begin{proof}
	This is a direct consequence of $\lambda_{jk}$ being a restriction of $\ddmes$ to a given edge.
	Indeed, let $g\in\gradZero{\ddom}{\ddmes}$ and let $\phi_n$ be an approximating sequence for $g$.
	Then clearly
	\begin{align*}
		\norm{\phi_n}_{\ltwo{\ddom}{\lambda_{jk}}} &\leq \norm{\phi_n}_{\ltwo{\ddom}{\ddmes}} \rightarrow 0, \\
		\norm{\ograd\phi_n - g^{(jk)}}_{\ltwo{\ddom}{\lambda_{jk}}} &\leq \norm{\ograd\phi_n - g}_{\ltwo{\ddom}{\ddmes}} \rightarrow 0,
	\end{align*}
	thus $g\in\ltwo{\ddom}{\lambda_{jk}}$ for every $I_{jk}$, so $g\in B$.
\end{proof}
Turning our attention to the reverse inclusion, we first demonstrate that so long as a gradient of zero on an edge $I_{jk}$ has support contained within the interior of the $I_{jk}$, we can extend it to a gradient of zero on the whole graph.
\begin{lemma}[Extension lemma for Gradients of Zero] \label{lem:3DExtensionLemmaGrads}
	Let $n\in\naturals$ and $I_{jk}^n$ be as in \eqref{eq:ShortenedEdgeDef}.
	Suppose that $g_{jk}\in\gradZero{\ddom}{\lambda_{jk}}$ with $\supp\bracs{g_{jk}}\subset I_{jk}^n$.
	Define the functions $g\in\ltwo{\ddom}{\ddmes}$ and $\tilde{g}\in\ltwo{\ddom}{\dddmes}$ by
	\begin{align*}
		g =	\begin{cases} g_{jk} & \mathrm{on} \ I_{jk}, \\ 0 & \mathrm{otherwise}, \end{cases} 
		&\quad
		\tilde{g} =	\begin{cases} g_{jk} & \mathrm{on} \ I_{jk}\setminus\clbracs{v_j,v_k}, \\ 0 & \mathrm{otherwise}. \end{cases}
	\end{align*}
	Then $g\in\gradZero{\ddom}{\ddmes}$ and $\tilde{g}\in\gradZero{\ddom}{\dddmes}$.
\end{lemma}
\begin{proof}
	Let $\phi_n$ be an approximating sequence for $g_{jk}$, and consider the sequence (of smooth functions) $\psi_l = \chi_{jk}^n\phi_l$.
	We have that
	\begin{align*}
		\norm{\psi_l}_{\ltwo{\ddom}{\ddmes}} 
		&= \norm{\chi_{jk}^n\phi_l}_{\ltwo{\ddom}{\lambda_{jk}}}
		\leq \norm{\phi_l}_{\ltwo{\ddom}{\lambda_{jk}}} \rightarrow 0 \toInfty{l}.
	\end{align*}
	Furthermore,
	\begin{align*}
		\norm{\ograd\psi_l - g}_{\ltwo{\ddom}{\ddmes}^2}^2
		&= \norm{\chi_{jk}^n\ograd\phi_l + \phi_l\ograd\chi_{jk}^n - g_{jk}}_{\ltwo{\ddom}{\lambda_{jk}}^2}^2 \\
		&\leq 2\norm{\phi_l\ograd\chi_{jk}^n}_{\ltwo{\ddom}{\lambda_{jk}}^2}^2 + 2\norm{\chi_{jk}^n\ograd\phi_l - g_{jk}}_{\ltwo{\ddom}{\lambda_{jk}}^2}^2 \\
		&\leq 2\sup_{I_{jk}}\abs{\ograd\chi_{jk}^n}^2 \norm{\phi_l}_{\ltwo{\ddom}{\lambda_{jk}}^2}^2
		+ 2\norm{\ograd\phi_l - g_{jk}}_{\ltwo{\ddom}{\lambda_{jk}}^2}^2 \\
		&\rightarrow 0 \toInfty{l}.
	\end{align*}
	Therefore, $\psi_l$ is an approximating sequence for $g$, and thus $g\in\gradZero{\ddom}{\ddmes}$, as required.
	
	Next, notice that $\psi_l\bracs{v_j} = \psi_l\bracs{v_k} = 0$, and $\ograd\psi_l\bracs{v_j} = \ograd\psi_l\bracs{v_k} = 0$ for every $l\in\naturals$.
	Therefore,
	\begin{align*}
		\norm{\psi_l}_{\ltwo{\ddom}{\nu}} = 0 = \norm{\ograd\psi_l}_{\ltwo{\ddom}{\nu}^2},
	\end{align*}
	for every $l\in\naturals$, and so
	\begin{align*}
		\norm{\psi_l}_{\ltwo{\ddom}{\dddmes}} &= \norm{\psi_l}_{\ltwo{\ddom}{\ddmes}} \rightarrow 0, \\
		\norm{\ograd\psi_l - \tilde{g}}_{\ltwo{\ddom}{\dddmes}^2} &= \norm{\ograd\psi_l - g}_{\ltwo{\ddom}{\ddmes}^2}^2 \rightarrow 0,
	\end{align*}
	and thus $\tilde{g}\in\gradZero{\ddom}{\dddmes}$.
\end{proof}
The hypothesis that $\supp\bracs{g_{jk}}\subset I_{jk}^n$ is essential for the inequality 
\begin{align*}
	\norm{\ograd\phi_l - g_{jk}}_{\ltwo{\ddom}{\lambda_{jk}}^2} \leq \norm{\ograd\phi_l - g_{jk}}_{\ltwo{\ddom}{\lambda_{jk}}^2}
\end{align*} 
to hold, and so that we can use the function $\chi_{jk}^n$ to ensure that our approximating sequence $\psi_l$ is ``restricted" to the edge $I_{jk}$ only, on which we know that $\phi_l$ has the properties we need.

We can now use the fact that the space of gradients of zero is closed to complete the proof of proposition \ref{prop:3DGradZeroChar}.
\begin{prop} \label{prop:3DBSubsetGradZero}
	We have that
	\begin{align*}
		\gradZero{\ddom}{\ddmes} \supset B.
	\end{align*}
	Furthermore, for any $g\in B$, let $\tilde{g}\in\ltwo{\ddom}{\dddmes}$ be defined by
	\begin{align*}
		\tilde{g}(x) &= \begin{cases} 0 & x\in\vertSet, \\ g & \mathrm{otherwise}. \end{cases}
	\end{align*}
	Then $\tilde{g}\in\gradZero{\ddom}{\dddmes}$.
\end{prop}
\begin{proof}
	Let $g\in B$, and define a family of functions $g_n$ by
	\begin{align*}
		g_n &= \sum_{v_j\in\vertSet}\sum_{j\conLeft k}\eta_j^n \eta_k^n g^{(jk)}.
	\end{align*}
	The graph $\graph$ is finite, so the sum converges.
	For each $j,k$ in the sum, the function $\eta_j^n \eta_k^n g^{(jk)}$ is an element of $\ltwo{\ddom}{\lambda_{jk}}$ with support in $I_{jk}^n$, so satisfies the hypothesis of the Extension lemma \ref{lem:3DExtensionLemmaGrads}.
	Therefore, $\eta_j^n \eta_k^n g^{(jk)}\in\gradZero{\ddom}{\ddmes}$ and since $\gradZero{\ddom}{\ddmes}$ is a linear subspace, $g_n\in\gradZero{\ddom}{\ddmes}$ for every $n\in\naturals$.
	Furthermore, we can see that $g_n\rightarrow g \toInfty{n}$ in $\ltwo{\ddom}{\ddmes}$ by the algebra of limits.
	Since $\gradZero{\ddom}{\ddmes}$ is closed, we conclude that $g\in\gradZero{\ddom}{\ddmes}$ too.
	
	Similarly, we can conclude by the Extension lemma \ref{lem:3DExtensionLemmaGrads} that the functions
	\begin{align*}
		\tilde{g}_n &= \begin{cases} 0 & x\in\vertSet, \\ g_n & \mathrm{otherwise}, \end{cases}
	\end{align*}
	form elements of $\gradZero{\ddom}{\dddmes}$ for each $n\in\naturals$.
	In addition, $\tilde{g}_n$ converges to $\tilde{g}$, and by closure, we have $\tilde{g}\in\gradZero{\ddom}{\dddmes}$.
\end{proof}
Proposition \ref{prop:3DBSubsetGradZero} and lemma \ref{lem:3DGradZeroSubsetB} then constitute the proof of proposition \ref{prop:3DGradZeroChar}.
The fact that we can also extend elements of $\gradZero{\ddom}{\ddmes}$, and hence $\gradZero{\ddom}{\lambda_{jk}}$, to elements of $\gradZero{\ddom}{\dddmes}$ will also contribute to our characterisation of the set $\gradZero{\ddom}{\dddmes}$.

Having dealt with the behaviour of gradients of zero on the edges of the graph $\graph$, we turn our attention to their behaviour at the vertices, induced by the measure $\nu$.
This analysis is far more straightforward than for $\ddmes$, in no small part due to the fact that the vertices of $\graph$ are isolated from each other, and so there are no problems centred around one vertex's proximity to another.
Let us begin by defining some useful functions and constants.
Set $d := \recip{2}\min\clbracs{\abs{I_{jk}} \setVert I_{jk}\in\edgeSet}$, which exists and is strictly greater than 0 since $\graph$ is finite.
For $c\in\complex$, let $\varphi_c:\reals^2\rightarrow\complex$ be a smooth function such that
\begin{align} \label{eq:NuSmoothVertexFunctionDef}
	\varphi_c(0) = 0, \quad
	\grad\varphi_c(0) = c, \quad
	\supp\bracs{\varphi_c}\subset B_d(0),
\end{align}
where $B_d(0)$ is the ball of radius $d$ centred at the origin.
Finally, set $N = \abs{\vertSet}$ and for each $v_j\in\vertSet$ define
\begin{align*}
	g_1^j(x) = \begin{cases} \bracs{1,0,0}^\top, & x=v_j, \\ 0 & x\neq v_j, \end{cases}
	\quad
	g_2^j(x) = \begin{cases} \bracs{0,1,0}^\top, & x=v_j, \\ 0 & x\neq v_j, \end{cases}
	\quad
	g_3^j(x) = \begin{cases} \bracs{0,0,1}^\top, & x=v_j, \\ 0 & x\neq v_j. \end{cases}
\end{align*}
We will now demonstrate that $\ltwo{\ddom}{\nu}^3$ is isomorphic to $\complex^{3N}$. 
\begin{lemma} \label{lem:L2nuIsomCN}
	The space $\ltwo{\ddom}{\nu}^3$ is isomorphic to $\complex^{3N}$.
	Furthermore, the collection 
	\begin{align*}
		B_{\nu} = \clbracs{g_1^j, g_2^j, g_3^j \setVert v_j\in\vertSet}
	\end{align*} forms a basis of $\ltwo{\ddom}{\nu}^3$.
\end{lemma}
\begin{proof}
	It is sufficient to notice that any $f\in\ltwo{\ddom}{\nu}^3$ is entirely determined by the values it takes at the vertices $v_j$.
	As such, we can define the map
	\begin{align*}
		\iota:\ltwo{\ddom}{\nu} \rightarrow \complex^{3N}, \quad
		\iota(f) = \bracs{\frac{f\bracs{v_1}}{\sqrt{\alpha_1}}, \frac{f\bracs{v_2}}{\sqrt{\alpha_2}}, \hdots, \frac{f\bracs{v_N}}{\sqrt{\alpha_N}}}^\top,
	\end{align*}
	where we have vertically concatenated the collection of three-vectors $f\bracs{v_j}, v_j\in\vertSet$ 	(and use the principle square root if $\alpha_j$ has non-zero imaginary part).
	Clearly $\iota$ is a bijection, and additionally for $f,g\in\ltwo{\ddom}{\nu}^3$ we have that
	\begin{align*}
		\ip{f}{g}_{\ltwo{\ddom}{\nu}^3} &= \integral{\ddom}{f\cdot\overline{g}}{\nu}
		= \sum_{v_j\in\vertSet}\alpha_j f\bracs{v_j}\overline{g\bracs{v_j}}
		= \iota(f)\cdot\overline{\iota(g)}
		= \ip{\iota(f)}{\iota(g)}_{\complex^{3N}},
	\end{align*}
	so $\iota$ is an isometry.
	Furthermore, the image of $B_{\nu}$ under $\iota$ is the canonical basis of $\complex^{3N}$, and thus the collection $B_{\nu}$ forms a basis of $\ltwo{\ddom}{\nu}^3$.
\end{proof}
Of course, if any of the $\alpha_j=0$, then we have that $\ltwo{\ddom}{\nu}^2$ is isomorphic to $\complex^{2(N-M)}$, where there are $M$ such $\alpha_j=0$ --- the obvious adjustment can be made to the map $\iota$.

The reason for observing that the collection $B_{\nu}$ is a basis of $\ltwo{\ddom}{\nu}^3$ is so that characterising the space $\gradZero{\ddom}{\nu}$ is now an easy task.
\begin{prop} \label{prop:NuGradZeroChar}
	We have that 
	\begin{align*}
		\gradZero{\ddom}{\nu} &= \mathrm{span}\clbracs{g_1^j, g_2^j \setVert j\in\vertSet }
		= \clbracs{ g\in\ltwo{\ddom}{\nu}^3 \setVert g_3 = 0}.
	\end{align*}
\end{prop}
\begin{proof}
	Notice that for any $g\in\gradZero{\ddom}{\nu}$, any approximating sequence $\phi_n$ is such that
	\begin{align*}
		\phi_n \rightarrow 0, \quad \partial_1\phi_n \rightarrow g_1, 
		\quad \partial_2\phi_n\rightarrow g_2, \quad \rmi\wavenumber\phi_n \rightarrow g_3,
	\end{align*}
	and therefore $g_3 = 0$ since $\rmi\wavenumber\phi_n$ converges to $g_3$ and the zero function.
	
	Now take $c=\bracs{1,0,0}^\top$ and $v_j\in\vertSet$, and let $\phi(x) = \varphi_c\bracs{x-v_j}$ for $\varphi_c$ as in \eqref{eq:NuSmoothVertexFunctionDef}.
	The function $\phi$ is smooth, has support contained in $B_d\bracs{v_j}$, and is such that $\ograd\phi(x) = \ograd\varphi_c\bracs{x-v_j}$.
	Clearly
	\begin{align*}
		\integral{\ddom}{\abs{\phi}^2}{\nu} = 0, \quad
		\integral{\ddom}{\abs{\ograd\phi - g_1^j}^2}{\nu} = 0,
	\end{align*}
	hence $g_1^j\in\gradZero{\ddom}{\nu}$.
	By a similar construction, we can show that $g_2^j\in\gradZero{\ddom}{\nu}$ too, and since $\gradZero{\ddom}{\nu}$ is a closed linear subspace of $\ltwo{\ddom}{\nu}^3$, we have the desired result.
\end{proof}

Given that the measure $\dddmes = \ddmes + \nu$, propositions \ref{prop:3DGradZeroChar} and \ref{prop:NuGradZeroChar} allow us to understand $\gradZero{\ddom}{\dddmes}$.
Intuitively, $\gradZero{\ddom}{\dddmes}$ is made up of linear combinations of gradients of zero with respect to $\ddmes$ and $\nu$, although we will qualify this statement since the functions (or rather, equivalence classes of functions) that live in $\gradZero{\ddom}{\ddmes}$ and $\gradZero{\ddom}{\nu}$ are defined on different parts of $\graph$.
\begin{theorem}[``Characterisation" of $\gradZero{\ddom}{\dddmes}$]
	Let $\tilde{g}\in\ltwo{\ddom}{\dddmes}^3$ and define
	\begin{align*}
		g_\mu(x) = \begin{cases} \tilde{g}(x) & x\neq v_j, \ \forall v_j\in\vertSet, \\ 0 & x=v_j, \ v_j\in\vertSet, \end{cases}
		\qquad
		g_\nu(x) = \begin{cases} 0 & x\neq v_j, \ \forall v_j\in\vertSet, \\ \tilde{g}\bracs{v_j} & x=v_j, \ v_j\in\vertSet. \end{cases}
	\end{align*}
	Then
	\begin{align*}
		\tilde{g}\in\gradZero{\ddom}{\dddmes} \quad\Leftrightarrow\quad
		g_\mu\in\gradZero{\ddom}{\ddmes} \text{ and } g_\nu\in\gradZero{\ddom}{\nu}.
	\end{align*}
\end{theorem}
\begin{proof}
	$\bracs{\Rightarrow}$ For the right-directed implication, it is sufficient to notice that
	\begin{align*}
		\norm{\cdot}_{\ltwo{\ddom}{\dddmes}^3}^2 &= \norm{\cdot}_{\ltwo{\ddom}{\ddmes}^3}^2 + \norm{\cdot}_{\ltwo{\ddom}{\nu}^3}^2,
	\end{align*}
	so any approximating sequence for $\tilde{g}$ also converges to $g_\mu$ in $\ltwo{\ddom}{\ddmes}^3$ and $g_\nu$ in $\ltwo{\ddom}{\nu}^3$.
	
	$\bracs{\Leftarrow}$ For the left-directed implication,
\end{proof}

\subsection{Tangential Gradients} \label{ssec:3DTangGradients}

\subsection{Geometric Interpretation} \label{ssec:3DGradGeometric}

\section{Proof of Proposition \ref{prop:MMatrixDetForm}} \label{sec:ProofOfProp}
The objective of this section is to prove proposition \ref{prop:MMatrixDetForm}, restated below for reference. \newline
\textit{Given a graph $\graph = \bracs{\vertSet, \edgeSet}$ containing no loops and whose edge-lengths are pairwise irrationally related, there exists a function $F\bracs{\qm,\omega}$ that is analytic in both its arguments, such that
\begin{align*}
	\det\mathfrak{M}_\qm\bracs{\omega^2} = \bracs{ \omega H^{(2)}\bracs{\omega^2} }^{\abs{\vertSet}-2} F\bracs{\qm,\omega}.
\end{align*}
} \newline
In the interest of brevity, we will suppress explicit dependencies on $\omega$ and $\qm$ throughout this section.

Throughout this section, let $\graph=\bracs{\vertSet, \edgeSet}$ be a graph with pairwise irrationally-related edge lengths and no loops (edges of the form $I_{jj}$), and set $N=\abs{\vertSet}$.
Interpret empty sums as evaluating to zero, empty products as evaluating to one, and for each $I_{jk}\in\edgeSet$ let
\begin{align*}
	s_{jk}\bracs{\omega} = \sin\bracs{l_{jk}\omega}, \quad
	c_{jk}\bracs{\omega} = \cos\bracs{l_{jk}\omega}, \quad
	e_{jk}^+\bracs{\qm} = \e^{\rmi\qm_{jk}l_{jk}}, \quad
	e_{jk}^-\bracs{\qm} = \e^{-\rmi\qm_{jk}l_{jk}}.
\end{align*}
Given $n\in\naturals$, let $S_n$ denote the symmetric group --- that is the group whose elements are the permutations of the integers $1, ..., n$.
Additionally, for $s\in S_n$ we use the notation $s_j := s(j), \ j=1,...,n$, and write $\mathrm{sgn}(s)$ for the signature of $s$.
Let $S_{\graph}$ denote the subset of $S_N$ defined as follows,
\begin{align*}
	S_{\graph} &= \clbracs{ s\in S_N \setVert j\con s_j \text{ or } j=s_j \ \forall j=1,...,N}.
\end{align*}
That is, an element $s\in S_\graph$ sends all integers $j$ to the index of another vertex $v_{s_j}$ that (directly) connects to $v_j$, or sends $j$ to itself.

Using the Leibniz formula for the determinant, we have that
\begin{align*}
	\det\mathfrak{M}_{\qm} &= \sum_{s\in S_N} \bracs{ \mathrm{sgn}(s) \prod_{j=1}^N \mathfrak{M}_{j s_j} },
\end{align*}
however if $s\not\in S_\graph$, then there exists at least one $\mathfrak{M}_{j s_j}=0$, so this term contributes nothing to the sum.
As such, we can write
\begin{align*}
	\det\mathfrak{M}_{\qm} &= \sum_{s\in S_\graph} \bracs{ \mathrm{sgn}(s) \prod_{j=1}^N \mathfrak{M}_{j s_j} }, \\
	&= \sum_{s\in S_\graph} \bracs{ \mathrm{sgn}(s) \prod_{\substack{j=1,...,N, \\ j\neq s_j}} \mathfrak{M}_{j s_j} \prod_{\substack{j=1,...,N, \\ j = s_j}} \mathfrak{M}_{jj} }.
\end{align*}
Using corollary \ref{cory:M-MatrixEntriesNoPoles}, and with $s\in S_\graph$ with $s_j = k$, we have that
\begin{align*}
	\mathfrak{M}_{jk} &= \omega H^{(2)} \bracs{ \sum_{j\conLeft k}e_{jk}^+ s_{jk}^{-1} + \sum_{j\conRight k}e_{kj}^- s_{kj}^{-1} }, \quad j\neq k, \\
	\mathfrak{M}_{jj} &= \omega H^{(2)} \bracs{ -\sum_{j\con l}c_{jl}s_{jl}^{-1} + \omega\alpha_j }, \quad j=k.
\end{align*}
For $s\in S_\graph$, define
\begin{align*}
	C\bracs{s} 
%	&= \prod_{\substack{j=1,...,N, \\ j\neq s_j}} \mathfrak{M}_{j s_j} \prod_{\substack{j=1,...,N, \\ j = s_j}} \mathfrak{M}_{jj} \\
	&= \prod_{\substack{j=1,...,N, \\ j\neq s_j}} \bracs{ \sum_{j\conLeft k}e_{jk}^+ s_{jk}^{-1} + \sum_{j\conRight k}e_{kj}^- s_{kj}^{-1} } \prod_{\substack{j=1,...,N, \\ j = s_j}} \bracs{ -\sum_{j\con l}c_{jl}s_{jl}^{-1} + \omega\alpha_j }, \labelthis\label{eq:PropProof-ProductLabel}
\end{align*}
so that 
\begin{align*}
	\det\mathfrak{M} &= \sum_{s\in S_\graph} \mathrm{sgn}(s) \bracs{ \omega H^{(2)} }^N  C(s)
\end{align*}

We refer to the product over $j\neq s_j$ in \eqref{eq:PropProof-ProductLabel} as the ``connection product" and the product over $j=s_j$ as the ``diagonal product".
All that remains for us to do is, for a given edge $I_{jk}$ and $s\in S_{\graph}$, to compute the highest power of $s_{jk}^{-1}$ that appears in $C(s)$.
To this end, let us fix $I_{jk}\in\edgeSet$ (with $j\neq k$ since $\graph$ has no loops) and $s\in S_{\graph}$.
Note that $s_{jk}^{-1}$ can only appear in at most 2 terms in the diagonal product (when $s_j=j$ and  $s_k=k$), and likewise can only appear in at most 2 terms in the connection product (when $s_j=k$ and $s_k=j$).
Furthermore, this also prevents $s_{jk}^{-1}$ from appearing in a total of 3 or greater terms across both products --- if it did, there would be two possibilities:
\begin{enumerate}
	\item $s_{jk}^{-1}$ appears in 2 terms in the connection product and 1 term in the diagonal product.
	Appearing twice in the connection product requires that $s_j=k$ and $s_k=j$.
	But appearing in the diagonal product requires one of $s_j=j$ or $s_k=k$, which cannot happen simultaneously with $s_j=k$ and $s_k=j$ as $j\neq k$.
	\item $s_{jk}^{-1}$ appears in 1 term in the connection product and 2 terms in the diagonal product.
	Appearing twice in the diagonal product requires that $s_j=j$ and $s_k=k$.
	But appearing in the connection product requires one of $s_j=k$ or $s_k=j$, which cannot happen simultaneously with $s_j=j$ and $s_k=k$ as $j\neq k$.
\end{enumerate}
As such, if one were to expand out the sums and products of $C(s)$, each term would contain a factor of $s_{jk}^{0}, s_{jk}^{-1}$, or $s_{jk}^{-2}$.
Given the formulae for $C(s), H^{(2)}, s_{jk}, c_{jk}, e_{jk}^+$, and $e_{jk}^-$, we can see that the function $\bracs{\qm, \omega}\rightarrow\bracs{ \omega H^{(2)} }^2 C(s)$ is therefore analytic in $\qm$ and $\omega$.
Hence we can write
\begin{align*}
	\det\mathfrak{M}_{\qm} &= \bracs{ \omega H^{(2)} }^{N-2} \sum_{s\in S_\graph} \bracs{ \mathrm{sgn}(s) \bracs{ \omega H^{(2)} }^2 C(s) },
\end{align*}
and upon identifying
\begin{align*}
	F\bracs{\qm, \omega} &= \sum_{s\in S_\graph} \bracs{ \mathrm{sgn}(s) \bracs{ \omega H^{(2)} }^2 C(s) },
\end{align*}
proposition \ref{prop:MMatrixDetForm} is proved.

%\begin{lemma} \label{lem:DetFormProofLemmaLess3}
%	Fix $I_{jk}\in\edgeSet$ and $s\in S_{\graph}$.
%	Then the term $s_{jk}^{-1}$ appears a combined total of strictly less than three times in the terms of the two products in $C(s)$.
%\end{lemma}
%\begin{proof}
%	Suppose for contradiction that $s_{jk}^{-1}$ appears a combined total of 3 times in the terms of the two products.
%	Note that $j\con k$ implies that $j\neq k$ since $\graph$ is assumed to have no loops.
%	Also note that $s_{jk}^{-1}$ can only appear in at most 2 terms in the diagonal product (when $s_j=j$ and  $s_k=k$), and likewise can only appear in at most 2 terms in the connection product (when $s_j=k$ and $s_k=j$).
%	There are two possibilities:
%	\begin{enumerate}
%		\item $s_{jk}^{-1}$ appears in 2 terms in the connection product and 1 term in the diagonal product.
%		Appearing twice in the connection product requires that $s_j=k$ and $s_k=j$.
%		But appearing in the diagonal product requires one of $s_j=j$ or $s_k=k$, which cannot happen simultaneously with $s_j=k$ and $s_k=j$ as $j\neq k$, so we have a contradiction.
%		\item $s_{jk}^{-1}$ appears in 1 term in the connection product and 2 terms in the diagonal product.
%		Appearing twice in the diagonal product requires that $s_j=j$ and $s_k=k$.
%		But appearing in the connection product requires one of $s_j=k$ or $s_k=j$, which cannot happen simultaneously with $s_j=j$ and $s_k=k$, so we have a contradiction.
%	\end{enumerate}
%	Similar reasoning holds for $s_{kj}^{-1}$.
%\end{proof}
%Lemma \ref{lem:DetFormProofLemmaLess3} demonstrates that, if one were to expand out the sums and products of $C(s)$, each term would have a factor of $s_{jk}^{0}, s_{jk}^{-1}$, or $s_{jk}^{-2}$.
%This allows us to notice that the least power of $s_{jk}$ that appears in the terms of the expression $\bracs{ \omega H^{(2)} }^2 C(s)$ is zero (and the greatest is $2$).
%Given the formulae for $C(s), H^{(2)}, s_{jk}, c_{jk}, e_{jk}^+$, and $e_{jk}^-$, we can see that the function $\bracs{\qm, \omega}\rightarrow\bracs{ \omega H^{(2)} }^2 C(s)$ is analytic in $\qm$ and $\omega$.
%Therefore, we can write
%\begin{align*}
%	\det\mathfrak{M} &= \bracs{ \omega H^{(2)} }^{N-2} \sum_{s\in S_\graph} \bracs{ \mathrm{sgn}(s) \bracs{ \omega H^{(2)} }^2 C(s) },
%\end{align*}
%and identifying
%\begin{align*}
%	F\bracs{\qm, \omega} &= \sum_{s\in S_\graph} \bracs{ \mathrm{sgn}(s) \bracs{ \omega H^{(2)} }^2 C(s) },
%\end{align*}
%proposition \ref{prop:MMatrixDetForm} is proved.


%Curl-curl equation plus 1st order Maxwell chapter begins
\chapter{Curl-of-the-Curl Equation} \label{ch:CurlCurl}
\tstk{make this title more focused when the time comes.}

%\input{./Chapters/Curl-Curl/Introduction}

\section{Geometric Interpretation of Tangential Curls} \label{sec:CC-Geometric}
Unlike the gradient, the curl of a vector field does not have a natural one-dimensional analogue that we can appeal to when looking to work on singular structure domains.
This makes the analysis of section \ref{sec:CC-CurlAnalysis}, which takes the procedure laid out in section \ref{sec:3DGradSobSpaces} and extends it to curls, crucial to our understanding of the problem \eqref{eq:SingularCurlEquation} and the tangential curls themselves.
In the interests of providing the reader an intuitive idea of what the tangential curl with respect to a singular measure is, we provide a geometric interpretation in this section prior along with a summary of the key behaviours $\ktcurl{\dddmes}u$ exhibits.
We will also address a remark made in section \ref{sec:TP-DomainSetup}, concerning why we do not consider a singular structure modelled by a (periodic) graph embedded into $\reals^3$, and instead only consider a domain formed from the extrusion into 3 dimensions of a periodic graph embedded in $\reals^2$.

Let us begin by demonstrating why curls of zero on a singular structure embedded into $\reals^3$ do not give rise to any interesting variational problems.
Consider a segment $I_{jk}\subset\reals^3$, with $I_{jk} = \clbracs{0}\times\sqbracs{0,l_{jk}}\times\clbracs{0}$, which we can view as one edge of a graph embedded into $\reals^3$ --- the illustration in figure \ref{fig:Diagram_SingularMeasure3D} is representative of the situation.
We can then prove the following result:
\begin{prop} \label{prop:3DGraph-CurlsAreZero}
	Every vector field is a curl of zero, that is
	\begin{align*}
		\curlZero{\reals^3}{\lambda_{jk}} = \ltwo{\reals^3}{\lambda_{jk}}^3,
	\end{align*}
	and consequentially every vector field $u\in\ltwo{\reals^3}{\lambda_{jk}}^3$ is also an element of $\ktcurlSob{\reals^3}{\lambda_{jk}}$ with tangential curl equal to zero.
\end{prop}
\begin{proof}
	We construct explicit approximating sequences for any function in $\ltwo{\reals^3}{\lambda_{jk}}^3$, by first considering $\phi\in\csmooth{\reals^3}$, and considering the vector field $\varphi^{(2)} = -x_1\phi\widehat{x}_3\in\csmooth{\reals^3}^3$.
	We observe that $\varphi^{(2)}=0$ on $I_{jk}$ (since $x_1=0$ here) and 
	\begin{align*}
		\curl{}\varphi^{(2)} = -x_1\partial_2\phi\widehat{x}_1 + \bracs{\phi + x_1\partial_1\phi}\widehat{x}_2,
	\end{align*}
	which is equal to $\phi\widehat{x}_2$ on $I_{jk}$.
	This implies that $\varphi^{(2)}$ serves as an approximating ``sequence" with
	\begin{align*}
		\varphi^{(2)} \lconv{\ltwo{\reals^3}{\lambda_{jk}}^3} 0, 
		\qquad
		\curl{}\varphi^{(2)} \lconv{\ltwo{\reals^3}{\lambda_{jk}}^3} \phi\widehat{x}_2,
	\end{align*}
	so $\phi\widehat{x}_2\in\curlZero{\reals^3}{\lambda_{jk}}$.
	Consideration of the functions $\varphi^{(1)} = -x_3\phi\widehat{x}_2$ and $\varphi^{(3)}=x_1\phi\widehat{x}_2$ demonstrates that $\phi\widehat{x}_1$ and $\phi\widehat{x}_3$ are also elements of $\curlZero{\reals^3}{\lambda_{jk}}$.
	By linearity we have that $\csmooth{\reals^3}^3\subset\curlZero{\reals^3}{\lambda_{jk}}$, and then by a density argument we obtain
	\begin{align*}
		\curlZero{\reals^3}{\lambda_{jk}} = \ltwo{\reals^3}{\lambda_{jk}}^3.
	\end{align*}
	We can also infer that any smooth vector field $\Phi\in\csmooth{\reals^3}{\lambda_{jk}}^3$ is an element of $\ktcurlSob{\reals^3}{\lambda_{jk}}$ with $\ktcurl{\lambda_{jk}}\Phi = 0$ since $\ltwo{\reals^3}{\lambda_{jk}}^{\perp} = \clbracs{0}$.
	Density of smooth functions in $\ltwo{\reals^3}{\lambda_{jk}}^3$ then demonstrates that any $u\in\ltwo{\reals^3}{\lambda_{jk}}^3$ is also an element of $\ktcurlSob{\reals^3}{\lambda_{jk}}$ with tangential curl equal to zero.
\end{proof}
The assumption on the position and orientation of $I_{jk}$ is not restrictive, in the sense that the result we prove will generalise to any line segment (and thus edge) in $\reals^3$.
This result also holds when we replace $\reals^3$ with a finite domain $U\subset\reals^3$ (with non-zero $\lambda_3$-measure) and consider $I_{jk}\subset U$.
We can also obtain a similar conclusion when $U$ is considered to be the period cell of some 3-dimensional singular structure, and $I_{jk}$ a piece of this structure.
Arguments similar to those in section \ref{sec:CC-CurlAnalysis} can then be used to deduce that tangential curls with respect to $\ddmes$ will inherit the behaviour of tangential curls with respect to each $I_{jk}$ on the respective edges.
This culminates in the realisation that a singular structure in 3 dimensions represented by a graph does not give rise to a non-trivial notion of the (tangential) curl of a vector field, and consequentially there are no non-trivial solutions to \eqref{eq:SingularCurlEquation} on such a structure.

This trivial behaviour is consistent with the interpretation of tangential curls that we obtain from our analysis in section \ref{sec:CC-CurlAnalysis} of the space $\ktcurlSob{\ddom}{\dddmes}$.
Classically, one can interpret the curl $c$ of a vector field $u$ by identifying $c(x)$ as the axis of rotation that an (infinitesimally small) spherical body would undergo if placed in the field $u$ at position $x$ (with the angular speed of the rotation proportional to the magnitude of $c(x)$).
Our intuition and geometric interpretation for tangential gradients (section \ref{ssec:3DGradGeometric}) was largely based on the idea that the singular measure $\lambda_{jk}$ cannot perceive changes in functions \emph{across} the edges $I_{jk}$.
The interpretation we have for tangential curls (and curls of zero) also appeals to this idea; the measure $\lambda_{jk}$ does not have any concept of normal derivative across $I_{jk}$, and correspondingly the product measure $\lambda_{jk}\times\lambda_1$ does not have any concept of a derivative \emph{outward} from the plane $P_{jk}$.
When we are restricted to only observing the values of $u$ in the plane $P_{jk}$, any changes in $u$ in the direction of the outward normal $\widehat{n}_{jk}$ to $P_{jk}$ cannot be observed.
Such changes in $u$ only affect the components of $c(x)$ in the directions orthogonal to $\widehat{n}_{jk}$ --- but if we do not know how we are rotating in these directions, we do not know how to orient \emph{the axis} about which we rotate.
As a result, any ``rotations" with axes parallel to the $\widehat{x}_3$ and $\widehat{e}_{jk}$ directions must correspond to curls of zero (corollary \ref{cory:CurlZero-Rotated}).
On the other hand, the only component of $c(x)$ that isn't derived from changes in $u$ in the direction $\widehat{n}_{jk}$ is the component along the direction $\widehat{n}_{jk}$ itself, which is the only component in tangential curls that is non-zero (corollary \ref{cory:TangCurlEdgeRotated}).
We can apply these ideas to our previous consideration of singular structures represented by graphs embedded into $\reals^3$ (proposition \ref{prop:3DGraph-CurlsAreZero}) ---  $\lambda_{jk}$ has no concept of derivative across the edge $I_{jk}$, and in three dimensions this means there are two linearly independent directions in which rates of change cannot be seen.
Consequentially, we can never accurately reconstruct the axis about which we are rotating, as all three components of any ``curl" require information about the rate of change in $u$ in at least one of the directions normal to $I_{jk}$.
The measure $\massMes$ tells a similar story: in this case we have thrown away the ability to perceive change in \emph{every} axial direction, and like the case for tangential gradients we find that $\ktcurl{\massMes}u=0$ for any vector field.

For a visual interpretation of tangential curls one can imagine the following scenario, which is illustrated in figure \ref{fig:Diagram_CurlGeometric} using the language of the commentary that follows.
\begin{figure}[t!]
	\centering
	\includegraphics[scale=1.0]{./Diagram_CurlGeometric.pdf}
	\caption[Geometric interpretation of tangential curls with respect to $\dddmes$.]{\label{fig:Diagram_CurlGeometric} An illustration of the notation of tangential curl and curls of zero on $P_{jk}$. Tangential curls cause points on an (infinitesimally small) sphere to rotate within the plane $P_{jk}$, whilst curls of zero induce rotations that move points on the sphere out of the plane.}
\end{figure}
Imagine an infinitesimally small sphere whose centre is at a point $x\in P_{jk}$, and then place another point $p$ on the intersection between the surface of the sphere and the plane $P_{jk}$.
Then consider the (closed) path $\gamma_p$ that $p$ traces out under the rotation induced by the vector field $u$.
Under the axis of rotation provided by $\ktcurl{\lambda_{jk}}u$, the curve $\gamma_p$ will be entirely contained in the plane $P_{jk}$, and thus the rotation is visible to\footnote{Or for want of a better phrase, the rotation \emph{traces out a path that can be followed by}.} the measure $\lambda_{jk}\times\lambda_1$.
However under the axis of rotation provided by a curl of zero, the curve $\gamma_p$ is not contained in the plane $P_{jk}$ --- in fact, it's intersection with the plane will be at most a finite number of points.
If instead $P_{jk}\cap\gamma_p\supset\clbracs{p,p^*}$ for $p^*\neq p$, the measure $\lambda_{jk}\times\lambda_1$ still has no way of knowing \emph{how} $p$ travelled to $p^*$, since the path $\gamma_p$ is not visible (and, from the 3-dimensional perspective, there are infinitely many curls of varying magnitudes that could rotate $p$ to $p^*$ and back).
In the event that $P_{jk}\cap\gamma_p=\clbracs{p}$, then it appears that the sphere is not rotating at all.
That is to say, curls of zero induce rotations ``out of the plane" $P_{jk}$, which the measure $\lambda_{jk}\times\lambda_1$ cannot observe, whilst tangential curls ensure that all rotation happens ``within" the plane $P_{jk}$.

Similarly to chapter \ref{ch:ScalarSystem}, it is the case that the behaviours of tangential curls and curls of zero on the edges of our singular structure (that is, with respect to $\lambda_{jk}$) are inherited by vector fields with tangential curls with respect to $\ddmes$ and $\dddmes$.
These arguments are the focus of section \ref{sec:CC-CurlAnalysis}, however we again provide the reader with a summary of the key properties of the tangential curl so that the manipulations in section \ref{sec:3DSystemDerivation} can be followed.
To this end, a function $u\in\ktcurlSob{\ddom}{\dddmes}$ has the following properties:
\begin{enumerate}[(i)]
	\item On a given edge $I_{jk}$, we have that $\ktcurl{\dddmes}u = \ktcurl{\lambda_{jk}}u$.
	\item Along the edge $I_{jk}$, only the ``in-plane" components of $u$ are relevant.
	To this end we set $U^{(jk)} = R_{jk}\begin{pmatrix} u^{(jk)}_1 \\ u^{(jk)}_2 \end{pmatrix}$, so that $U_2^{(jk)} = \bracs{u_1^{(jk)},u_2^{(jk)}}^\top e_{jk}$, and we have that
	\begin{align*}
		\ktcurl{\lambda_{jk}}u = \bracs{\bracs{u_3^{(jk)}}' + \rmi\qm_{jk}u_3^{(jk)} - \rmi\wavenumber U^{(jk)}_2}\widehat{n}_{jk}.
	\end{align*}
	The constants $\qm_{jk}=\qm\cdot e_{jk}$ are the same constants from section \ref{sec:ScalarDerivation}, and $u_3'$ is the derivative in the $H^1$ sense of $u^{(jk)}_3\circ r_{jk}$.
	The term $\rmi\wavenumber U^{(jk)}_2$ resembles the information about any rotation induced by changes in $u$ along the $x_3$-axis that the measure $\lambda_{jk}\times\lambda_1$ can see.
	\item We have that $u_3\in\ktgradSob{\ddom}{\dddmes}$, so in particular it is continuous at each vertex $v_j$.
	\item At each $v_j$, we have that $\ktcurl{\dddmes}u = 0$.
	However much like the case with gradients, the incoming curls to a vertex from the connecting edges do not have to have a limit of zero into the vertex!
\end{enumerate}

%\section{Derivation of quantum graph problem} \label{sec:3DSystemDerivation}
We now derive the system \eqref{eq:3DQGFullSystem} from \eqref{eq:PeriodCellCurlCurlStrongForm}.
Let $u=\bracs{u_1,u_2,u_3}^\top\in\ktcurlSob{\ddom}{\dddmes}$ and let $\Phi=\bracs{\phi_1,\phi_2,\phi_3}^\top\in\psmooth{\ddom}^3$.
Also define for each $I_{jk}\in\edgeSet$,
\begin{align*}
	U^{(jk)} := R_{jk} \begin{pmatrix} u_1^{(jk)} \\ u_2^{(jk)} \end{pmatrix},
	\qquad
	\Psi^{(jk)} := R_{jk} \begin{pmatrix} \phi_1^{(jk)} \\ \phi_2^{(jk)} \end{pmatrix}.
\end{align*}
We use an overhead tilde to denote composition with $r_{jk}$, and for a function $v$ with $\widetilde{v}^{(jk)}\in\ktgradSob{\sqbracs{0,l_{jk}}}{y}$ write $\bracs{v^{(jk)}}' := \bracs{\widetilde{v}^{(jk)}}' \circ r_{jk}^{-1}$.
Finally, for a given quasi-momentum $\qm$, set $\qm_{jk} = \bracs{ R_{jk}\qm }_2 = \qm\cdot e_{jk}$.
%, and let
%\begin{align*}
%	\ktcurlSobDivFree{\ddom}{\dddmes} &=
%	\clbracs{ u\in\ktcurlSob{\ddom}{\dddmes} \setVert u \text{ is } \kt\text{-divergence-free with respect to } \dddmes}.
%\end{align*}
A function $u\in\ktcurlSob{\ddom}{\dddmes}$ is a solution to \eqref{eq:PeriodCellCurlCurlStrongForm} if
\begin{align} \label{eq:PeriodCellCurlCurlWeakForm}
	\integral{\ddom}{ \ktcurl{\dddmes}u\cdot\overline{\ktcurl{\dddmes}\Phi} }{\dddmes} 
	&= \omega^2 \integral{\ddom}{ u\cdot\overline{\Phi} }{\dddmes},
	\quad\forall \Phi\in\psmooth{\ddom}^3.
\end{align}
We immediately have that $u$ is $\kt$-divergence free (in the sense of definition \ref{def:DivFree-TangGradients}), since for any $v\in\ktgradSob{\ddom}{\dddmes}$ we can take an approximating sequence $\Phi^n$ for $v$ and obtain
\begin{align*}
	\omega^2\integral{\ddom}{ u\cdot\overline{\ktgrad_{\dddmes}v} }{\dddmes}
	&= \omega^2 \lim_{n\rightarrow\infty} \integral{\ddom}{ u\cdot\overline{\ktgrad\Phi^n} }{\dddmes} \\
	&= \lim_{n\rightarrow\infty} \integral{\ddom}{ \ktcurl{\dddmes}u\cdot\overline{\ktcurl{\dddmes}\ktgrad\Phi^n} }{\dddmes}
	= 0,
\end{align*}
through using lemma \ref{lem:CurlOfGradSmoothFunctions}.
Note that we can similarly show that $u$ is orthogonal to all gradients of zero via the same trick, however we will see that the additional information (the conditions (i) and (iv) in proposition \ref{prop:DivFree-AllGradsConditions}) is obtained more directly in what follows.

Now let us fix an edge $I_{jk}$ and take $\Phi$ such that $\supp\Phi\cap\graph\subset I_{jk}^{\circ}$.
The equation \eqref{eq:PeriodCellCurlCurlWeakForm} reduces to
\begin{align*}
	\omega^2 \integral{I_{jk}}{ u\cdot\overline{\Phi} }{\lambda_{jk}}
	&= \integral{I_{jk}}{ \ktcurl{\ddmes}u\cdot\overline{\ktcurl{\ddmes}\Phi} }{\lambda_{jk}} \\
	&= \integral{I_{jk}}{ \bracs{ \bracs{u_3^{(jk)}}' + \rmi\qm_{jk} u_3^{(jk)} - \rmi\wavenumber U_2^{(jk)} }\overline{\bracs{ \bracs{\phi_3^{(jk)}}' + \rmi\qm_{jk} \phi_3^{(jk)} - \rmi\wavenumber \Psi_2^{(jk)} }} }{\lambda_{jk}}.
\end{align*}
Using the change of variables $r_{jk}$ we obtain
\begin{align*}
	\int_0^{l_{jk}} & \bracs{ \bracs{\widetilde{u}_3^{(jk)}}' + \rmi\qm_{jk} \widetilde{u}_3^{(jk)} - \rmi\wavenumber \widetilde{U}_2^{(jk)} } \bracs{ \bracs{\overline{\widetilde{\phi}}_3^{(jk)}}' - \rmi\qm_{jk} \overline{\widetilde{\phi}}_3^{(jk)} + \rmi\wavenumber \overline{\widetilde{\Psi}}_2^{(jk)} } \ \md y \\
	&= \omega^2 \int_0^{l_{jk}} \widetilde{U}_1^{(jk)}\overline{\widetilde{\Psi}}_1^{(jk)} + \widetilde{U}_2^{(jk)}\overline{\widetilde{\Psi}}_2^{(jk)} + \widetilde{u}_3^{(jk)}\overline{\widetilde{\phi}}_3^{(jk)} \ \md y.
\end{align*}
Now suppose we have $\varphi_1,\varphi_2\in\csmooth{\sqbracs{0,l_{jk}}}$.
Since the system
\begin{align*}
	\begin{pmatrix} \phi_1 \\ \phi_2 \end{pmatrix} &= R_{jk}^\top \begin{pmatrix} \varphi_1 \\ \varphi_2 \end{pmatrix}
\end{align*}
has a unique solution, we can construct a $\Phi\in\csmooth{\ddom}$ with $\supp\Phi\cap\graph\subset I_{jk}^{\circ}$, $\widetilde{\Psi}_1^{(jk)}=\varphi_1$, and $\widetilde{\Psi}_2^{(jk)}=\varphi_2$.
We thus observe that
\begin{align*}
	\int_0^{l_{jk}} & \bracs{ \bracs{\widetilde{u}_3^{(jk)}}' + \rmi\qm_{jk} \widetilde{u}_3^{(jk)} - \rmi\wavenumber \widetilde{U}_2^{(jk)} } \bracs{ \overline{\varphi_3}' - \rmi\qm_{jk} \overline{\varphi_3} + \rmi\wavenumber \overline{\varphi_2} } \ \md y \\
	&= \omega^2 \int_0^{l_{jk}} \widetilde{U}_1^{(jk)}\overline{\varphi_1} + \widetilde{U}_2^{(jk)}\overline{\varphi_2} + \widetilde{u}_3^{(jk)}\overline{\varphi_3} \ \md y
\end{align*}
holds for any $\varphi_1,\varphi_2,\varphi_3\in\csmooth{\sqbracs{0,l_{jk}}}$.
Therefore it must hold that for any $\varphi\in\csmooth{\sqbracs{0,l_{jk}}}$,
\begin{subequations}
	\begin{align*}
		0 &= \int_0^{l_{jk}} \overline{\varphi} \widetilde{U}_1^{(jk)} \ \md y, \labelthis\label{eq:CurlCurlWeakFormPhi1} \\
		0 &= \int_0^{l_{jk}} \overline{\varphi} \bracs{ \rmi\wavenumber\bracs{\widetilde{u}_3^{(jk)}}' + \bracs{\wavenumber^2 - \omega^2}\widetilde{U}_2^{(jk)} - \wavenumber\qm_{jk}\widetilde{u}_3^{(jk)}  } \ \md y, \labelthis\label{eq:CurlCurlWeakFormPhi2} \\
		0 &= \int_0^{l_{jk}} \overline{\varphi}' \bracs{ \bracs{\widetilde{u}_3^{(jk)}}'
		- \rmi\wavenumber\widetilde{U}_2^{(jk)} + \rmi\qm_{jk}\widetilde{u}_3^{(jk)} } \\
		&\qquad -\rmi\qm_{jk}\overline{\varphi}\bracs{ \bracs{\widetilde{u}_3^{(jk)}}' - \rmi\wavenumber\bracs{\widetilde{U}_2^{(jk)}}' + \rmi\qm_{jk}\widetilde{u}_3^{(jk)} }
		- \omega^2 \widetilde{u}_3^{(jk)}\overline{\varphi} \ \md y. \labelthis\label{eq:CurlCurlWeakFormPhi3}
	\end{align*}
\end{subequations}
The equation \eqref{eq:CurlCurlWeakFormPhi1} implies that $U_1^{(jk)}=0$ (almost everywhere) on $I_{jk}$, whilst \eqref{eq:CurlCurlWeakFormPhi2} implies that
\begin{align*}
	\rmi\wavenumber\bracs{\diff{}{y} + \rmi\qm_{jk}}\widetilde{u}_3^{(jk)} + \wavenumber^2\widetilde{U}_2^{(jk)} &= \omega^2\widetilde{U}_2^{(jk)},
\end{align*}
on $I_{jk}$.
Given that $u$ is divergence-free, we know that $\widetilde{U}_2^{(jk)}$ is weakly differentiable in the $\gradSob{\sqbracs{0,l_{jk}}}{y}$-sense and can manipulate \eqref{eq:CurlCurlWeakFormPhi3} to obtain
\begin{align*}
	\int_0^{l_{jk}} \overline{\varphi}' \widetilde{u}'_{3,jk} \ \md y
	&= -\int_0^{l_{jk}} \overline{\psi} \bracs{ \rmi\wavenumber\widetilde{U}'_{2,jk} - \wavenumber\qm_{jk}\widetilde{U}_{2,jk} - 2\rmi\qm_{jk}\widetilde{u}'_{3,jk} + \qm_{jk}^2\widetilde{u}_{3,jk} - \omega^2\widetilde{u}_{3,jk} } \ \md y.,
\end{align*}
for all $\varphi\in\csmooth{\sqbracs{0,l_{jk}}}$.
Thus we can conclude that $\widetilde{u}_3\in\gradgradSob{\sqbracs{0,l_{jk}}}{y}$, from which we deduce that
\begin{align*}
	-\bracs{ \diff{}{y} + \rmi\qm_{jk} }^2\widetilde{u}_3^{(jk)} + \rmi\wavenumber\bracs{ \diff{}{y} + \rmi\qm_{jk} }\widetilde{U}_2^{(jk)} &= \omega^2 \widetilde{u}_3^{(jk)}
\end{align*}
on $I_{jk}$.
This provides us with a system of coupled ODEs on each of the edges $I_{jk}$ (and the information that the component $\widetilde{U}_1^{(jk)}=0$ on each edge).

Now we look at how these edge ODEs are coupled at the vertices.
Fix a vertex $v_j$ and consider a $\Phi\in\csmooth{\ddom}$ whose support contains $v_j$ in its interior, and no other vertices of $\graph$.
Testing against such $\Phi$ in \eqref{eq:PeriodCellCurlCurlWeakForm} implies that
\begin{align*}
	\alpha_j\omega^2 u(v_j)\cdot\overline{\Phi}(v_j)
	&= \integral{\ddom}{ \ktcurl{\ddmes}u\cdot\overline{\ktcurl{\ddmes}\Phi} - \omega^2 u\cdot\overline{\Phi} }{\ddmes} \\
	&= \sum_{j\con k}\integral{I_{jk}}{ \bracs{ \bracs{u_3^{(jk)}}' + \rmi\qm_{jk}u_3^{(jk)} - \rmi\wavenumber U_2^{(jk)}} \bracs{ \bracs{\overline{\phi}_3^{(jk)}}' - \rmi\qm_{jk}\overline{\phi}_3^{(jk)} + \rmi\wavenumber \overline{\Psi}_2^{(jk)}} \\
	&\qquad - \omega^2 U_1^{(jk)}\overline{\Psi}_1 - \omega^2 U_2^{(jk)}\overline{\Psi}_2 - \omega^2 u_3^{(jk)}\overline{\phi}_3 }{\lambda_{jk}} \\
	&= \sum_{j\con k}\int_0^{l_{jk}} \bracs{ \bracs{\widetilde{u}_3^{(jk)}}' + \rmi\qm_{jk}\widetilde{u}_3^{(jk)} - \rmi\wavenumber \widetilde{U}_2^{(jk)}} \bracs{ \bracs{\overline{\widetilde{\phi}}_3^{(jk)}}' - \rmi\qm_{jk}\overline{\widetilde{\phi}}_3^{(jk)} + \rmi\wavenumber \overline{\widetilde{\Psi}}_2^{(jk)}} \\
	&\qquad - \bracs{\bracs{\widetilde{u}_3^{(jk)}}' + \rmi\qm_{jk}\widetilde{u}_3^{(jk)} - \rmi\wavenumber \widetilde{U}_2^{(jk)}}\rmi\wavenumber\overline{\widetilde{\Psi}_2^{(jk)}} - \omega^2 \widetilde{u}_3^{(jk)}\overline{\widetilde{\phi}}_3 \ \md y \\
	&= -\sum_{j\con k}\int_0^{l_{jk}} \overline{\widetilde{\phi}}_3
	\sqbracs{ \bracs{\widetilde{u}_3^{(jk)}}'' + 2\rmi\qm_{jk}\bracs{\widetilde{u}_3^{(jk)}}' + \bracs{\rmi\qm_{jk}}^2\widetilde{u}_3^{(jk)} + \omega^2\widetilde{u}_3^{(jk)} } \\
	&\qquad - \overline{\widetilde{\phi}}_3\sqbracs{\rmi\wavenumber\bracs{ \bracs{\widetilde{U}_2^{(jk)}}' + \rmi\qm_{jk}\widetilde{U}_2^{(jk)} } } \ \md y \\
	&\qquad + \overline{\phi}_3(v_j)\sqbracs{ \sum_{j\con k}\bracs{\pdiff{}{n}+\rmi\qm_{jk}}u_3^{(jk)}(v_j) + \sum_{j\conRight k}U_2^{(kj)}(v_j) - \sum_{j\conLeft k}U_2^{(jk)}(v_j) } \\
	&= \overline{\phi}_3(v_j)\sqbracs{ \sum_{j\con k}\bracs{\pdiff{}{n}+\rmi\qm_{jk}}u_3^{(jk)}(v_j) + \sum_{j\conRight k}U_2^{(kj)}(v_j) - \sum_{j\conLeft k}U_2^{(jk)}(v_j) },
\end{align*}
where $U_2^{(jk)}(v_j)$ is the trace of $U_2^{(jk)}$ to the vertex $v_j$.
Identifying that this holds for all $\phi_1(v_j), \phi_2(v_j), \phi_3(v_j)\in\complex$, we must conclude that
\begin{align*}
	u_1(v_j) &= 0, \\
	u_2(v_j) &= 0, \\
	\alpha_j\omega^2u_3(v_j) &= \sum_{j\con k}\bracs{\pdiff{}{n}+\rmi\qm_{jk}}u_3^{(jk)}(v_j) + \sum_{j\conRight k}U_2^{(kj)}(v_j) - \sum_{j\conLeft k}U_2^{(jk)}(v_j).
\end{align*}
We are thus presented with the following set of equations on each edge $I_{jk}$;
\begin{subequations} \label{eq:QGRawSystem}
	\begin{align*}
		\widetilde{U}_1^{(jk)} &= 0 \\
		\rmi\wavenumber \bracs{ \diff{}{y} + \rmi\qm_{jk} }\widetilde{u}_3^{(jk)} + \wavenumber^2\widetilde{U}_2^{(jk)} &= \omega^2\widetilde{U}_2^{(jk)}, \labelthis\label{eq:QGPhi2} \\
		-\bracs{ \diff{}{y} + \rmi\qm_{jk} }^2\widetilde{u}_3^{(jk)} + \rmi\wavenumber\bracs{ \diff{}{y} + \rmi\qm_{jk} }\widetilde{U}_2^{(jk)} &= \omega^2 \widetilde{u}_3^{(jk)}, \labelthis\label{eq:QGPhi3} 
	\end{align*}
	complemented by the vertex conditions
	\begin{align*}
		\widetilde{u}_3 \text{ is continuous at } v_j &\quad\forall v_j\in\vertSet, \labelthis\label{eq:QGContinuity} \\
		u_1(v_j) &= 0, \\
		u_2(v_j) &= 0, \\
		\alpha_j \omega^2 u_3\bracs{v_j}
		&= \sum_{j\con k} \bracs{ \pdiff{}{n} + \rmi\qm_{jk} }u_3^{(jk)}\bracs{v_j} - \rmi\wavenumber\bracs{ \sum_{j\conRight k} U_2^{(jk)} - \sum_{j\conLeft k} U_2^{(jk)} }. \labelthis\label{eq:QGVertexCondition}
	\end{align*}
\end{subequations}

Determining the solution to \eqref{eq:QGRawSystem} requires us to find the function $u_3$ and the edge-functions $U_2^{(jk)}$.
However we can eliminate the $U_2^{(jk)}$ from the system \eqref{eq:QGRawSystem} via substitution of \eqref{eq:QGPhi2} into \eqref{eq:QGPhi3}, and use of corollary \ref{cory:DivFree-TangGradsConditions}(v), to obtain
\begin{subequations} \label{eq:CurlCurl-ScalarQGProblem}
	\begin{align}
		-\bracs{ \diff{}{y} + \rmi\qm_{jk} }^2 u_3^{(jk)} &= \bracs{\omega^2-\wavenumber^2}u,
		\qquad &\text{on each } I_{jk}, \\
		u_3 \text{ is continuous at } & v_j \qquad &\forall v_j\in\vertSet, \\
		\sum_{j\con k}\bracs{\pdiff{}{n}+\rmi\qm_{jk}}u_3^{(jk)}(v_j) &= \alpha_j\bracs{\omega^2-\wavenumber^2}u_3(v_j), \qquad &\forall v_j\in\vertSet.
	\end{align}
\end{subequations}
That is, \tstk{curl of curl equation} simply reduces to the scalar system obtained in section \ref{sec:ScalarDerivation}, except with $\omega^2-\wavenumber^2$ playing the role of the spectral parameter.
We will elaborate further on the implications of this result in section \tstk{discussion}.

\tstk{remarks: the fact that you have obtained a (scalar) equation for the "polarised" Maxwell by a series of nontrivial manipulations on the curl demonstrates that your approach is sound overall and gives further credibility to the work on the first part of the thesis. This "polarised Maxwell" chapter will not have much new stuff on the numerics side though, as we have got back to the scalar case, which will have been looked at numerically in the preceding chapter.}

\tstk{UNCHANGED FROM PREVIOUS NOTES!!!! Also needs an introductory sentence, and will probably go in the discussion section of this chapter.}

\subsection{Remarks on the Calder\'on Operator} \label{ssec:CalderonOp}
In this section we look to draw parallels between the classical curl-of-the-curl problem (described by \eqref{eq:CurlCurlEqn} on a suitable domain with boundary conditions) for a polarised electromagnetic field, and the system \eqref{eq:QGRawSystem}.
It is not known whether the problem \eqref{eq:PeriodCellCurlCurlStrongForm} is the ``limit" of a thin-structure problem with thick vertices, in the sense of \tstk{scalar SS problem to scalar TS problem, via KZ and EP}.
However, we can demonstrate that the $M$-operator associated to the (operator which defines the) problem \eqref{eq:QGRawSystem} (hence \eqref{eq:PeriodCellCurlCurlStrongForm}) is a direct analogue of the Calder\'on operator for problems like \eqref{eq:CurlCurlEqn}.
This will be is done by showing that the Dirichlet and Neumann maps for the classical problem motivate ``natural" definitions for their counterparts for the problem \eqref{eq:QGRawSystem}, and that the resulting maps form a boundary triple, providing us with an $M$-operator.
We will also see that the vertex condition \eqref{eq:QGVertexCondition} can be written in a familiar form \tstk{see the BC in scalar paper, or \eqref{eq:DispersiveBC}} relating the Dirichlet and Neumann maps, implying a similar solving approach to \tstk{scalar discussion chapter} can be undertaken.

First, we quickly review/ reintroduce the Calder\'on operator in the classical setting. \tstk{might be the first time we talk about this, or we might move it into the $M$-matrix section of QGs, or even into the introductory section of this chapter.}
Let $\dddom\subset\reals^3$ be a domain, and consider the curl-of-the-curl problem (or polarised Maxwell system)
\begin{subequations} \label{eq:Maxwell3D}
	\begin{align} 
		\curl{}\bracs{\curl{}u}u - \beta u = 0 &\qquad\text{in } \dddom, \\
		\hat{n}\wedge\curl{}u = m &\qquad\text{on } \partial\dddom,
	\end{align}
\end{subequations}
where $\beta>0$ and $m$ is a given function, and $\hat{n}$ is the exterior normal to the surface $\partial\dddom$.
The Calder\'on operator associated to the problem \eqref{eq:Maxwell3D} is then the operator $\mathcal{C}$ that acts on solutions $u$ to \eqref{eq:Maxwell3D}, sending
\begin{align*}
	u\vert_{\partial\dddom} \rightarrow \hat{n}\wedge\bracs{\curl{}u}\vert_{\partial\dddom}.
\end{align*}
Defining $\mathcal{A}$ as the operator
\begin{align*}
	\mathrm{dom}\bracs{\mathcal{A}} &= \clbracs{ u\in H^2_{\mathrm{curl}}(\dddom) \setVert \hat{n}\wedge\curl{}u\vert_{\partial\dddom} = m }, \\
	\mathcal{A}u &= \curl{}\bracs{\curl{}u},
\end{align*}
and the Dirichlet and Neumann maps ($\dmap$ and $\nmap$ respectively) by
\begin{align} \label{eq:ClassicalEM-DNMaps}
	\dmap u = u\vert_{\partial\dddom}, \qquad
	\nmap u = \hat{n}\wedge\curl{u}\vert_{\partial\dddom},
\end{align}
we can validate Green's identity for the triple $\bracs{\ltwo{\partial\dddom}{S}^3, \dmap, \nmap}$:
\begin{align*}
	\integral{\dddom}{ \mathcal{A}u \cdot \overline{v} - u \cdot \overline{\mathcal{A}v} }{x}
	&= \integral{\dddom}{ \curl{\curl{u}}\cdot\overline{v} - u\cdot\overline{\curl{\curl{v}}} }{x} \\
	&= \integral{\dddom}{ \curl{u}\cdot\overline{\curl{v}} - \curl{u}\cdot\overline{\curl{v}} }{x} \\
	&\quad + \integral{\partial\dddom}{ \hat{n}\wedge\curl{u}\cdot\overline{v} - u\cdot\hat{n}\wedge\overline{\curl{v}} }{S} \\
	&= \integral{\partial\dddom}{ \nmap u \cdot \overline{\dmap v} - \dmap u \cdot \overline{\nmap v} }{S}.
\end{align*}
We can thus conclude that $\bracs{\ltwo{\partial\dddom}{S}^3, \dmap, \nmap}$ is a boundary triple for the operator $\mathcal{A}$.
The Calder\'on operator is the corresponding Dirichlet-to-Neumann map, or $M$-operator, \tstk{QG chapter} for the problem \eqref{eq:Maxwell3D}.

Now let us return to the quantum graph problem \eqref{eq:QGRawSystem}.
If we expect \eqref{eq:SingStrucCurlCurl} to be some ``limit" of a thin-structure problem with thick vertices, then we expect that the vertex condition \eqref{eq:QGVertexCondition} will be of the form
\begin{align} \label{eq:DispersiveBC}
	\dgmap u &= -\omega^2 \tilde{\alpha} \ngmap u,
\end{align}
which is the form of the vertex conditions in \tstk{scalar problem}.
In the context of \eqref{eq:QGRawSystem}, we should expect $\tilde{\alpha}$ be akin to the diagonal matrix of coupling constants (rather than precisely the matrix $\alpha$), whilst $\dgmap, \ngmap$ will be the Dirichlet and Neumann maps for the quantum graph problem \eqref{eq:QGRawSystem}.

Given the definition of $\dmap$ in \eqref{eq:ClassicalEM-DNMaps}, we should expect that
\begin{align} \label{eq:DGMapDef}
	\dgmap u &= 
	\begin{pmatrix}
		u\bracs{v_1} \\ u\bracs{v_2} \\ \vdots \\ u\bracs{v_N}
	\end{pmatrix}
	\in\complex^{3N},
\end{align}
where we have stacked the 3-vectors on top of each other and set $N=\abs{\vertSet}$.
As for $\ngmap u$, this should be the analogue of $\nmap$ in \eqref{eq:ClassicalEM-DNMaps} --- only now the boundary of our domain is the vertices of $\graph$.
Define the functions \tstk{might be worth moving into our usual setup assumption for ease of use?}
\begin{align*}
	\sgn_{jk}: \clbracs{v_j, v_k} \rightarrow \clbracs{-1,0,1}, 
	&\qquad
	\sgn_{jk}(x) = \begin{cases} -1 & x=v_j, \\ 1 & x=v_k, \end{cases}
	&\qquad
	\hat{\sigma}_{jk} &= \sgn_{jk}\widehat{e}_{jk},
\end{align*}
so $\hat{\sigma}_{jk}$ is the ``exterior normal" to the edge $I_{jk}$.
The natural candidate for $\ngmap$ is then 
\begin{align} \label{eq:NGMapDef}
	\ngmap u &= 
	\begin{pmatrix}
		\sum_{1\con k} \hat{\sigma}_{1k}\wedge\ktcurl{\dddmes}u\vert_{v_1} \\
		\sum_{2\con k} \hat{\sigma}_{2k}\wedge\ktcurl{\dddmes}u\vert_{v_2} \\
		\vdots \\
		\sum_{N\con k} \hat{\sigma}_{Nk}\wedge\ktcurl{\dddmes}u\vert_{v_N}
	\end{pmatrix}
	\in\complex^{3N},
\end{align}
where we have again stacked the 3-vectors vertically.
From our analysis of $\kt$-tangential curls \tstk{section}, we know that
\begin{align*}
	\ktcurl{\dddmes}u &= \bracs{ \bracs{ u_3^{(jk)} }' + \rmi\qm_{jk}u_3^{(jk)} - \rmi\wavenumber U_2^{(jk)} }\widehat{n}_{jk},
\end{align*}
on each edge $I_{jk}$.
Therefore, 
\begin{align*}
	\widehat{e}_{jk}\wedge\ktcurl{\dddmes}u &= -
	\begin{pmatrix} 
	0 \\
	0 \\
	\bracs{ u_3^{(jk)} }' + \rmi\qm_{jk}u_3^{(jk)} - \rmi\wavenumber U_2^{(jk)}
	\end{pmatrix},
\end{align*}
on $I_{jk}$, and hence (for a fixed $v_j\in\vertSet$)
\begin{align*}
	\sum_{j\con k} \hat{\sigma}_{jk} \ \wedge \ &\ktcurl{\dddmes}u\vert_{v_j} = \\ 
	&\begin{pmatrix}
	0 \\
	0 \\	
	- \sum_{j\con k}\bracs{\pdiff{}{n} + \rmi\qm_{jk}}u_3^{(jk)}\bracs{v_j}
	+ \rmi\wavenumber\bracs{ \sum_{j\conRight k} U_2^{(kj)}\bracs{v_j} - \sum_{j\conLeft k} U_2^{(jk)}\bracs{v_j} }
	\end{pmatrix}.
\end{align*}
\tstk{we could write
\begin{align*}
	\sum_{j\conRight k} U_2^{(kj)}\bracs{v_j} - \sum_{j\conLeft k} U_2^{(jk)}\bracs{v_j} &=
	\sum_{j\con k} \sgn_{jk}U_2^{(jk)}\bracs{v_j}
\end{align*} 
using our definitions and conventions from the QG chapter.}
The vertex conditions for the system \eqref{eq:QGRawSystem} can be written as
\begin{align} \label{eq:VertConditionExplicit}
	\alpha_j\omega^2 u\bracs{v_j} &=
	\begin{pmatrix}
	0 \\
	0 \\	
	\bracs{\pdiff{}{n} + \rmi\qm_{jk}}u_3^{(jk)}\bracs{v_j}
	- \rmi\wavenumber\bracs{ \sum_{j\conRight k} U_2^{(kj)}\bracs{v_j} - \sum_{j\conLeft k} U_2^{(jk)}\bracs{v_j} }
	\end{pmatrix},
\end{align}
at each $v_j\in\vertSet$ --- note that the first two components are just the conditions $u_1\bracs{v_j}=u_2\bracs{v_j}=0$.
We can identify \eqref{eq:VertConditionExplicit} as being of the form \eqref{eq:DispersiveBC} where
\begin{align*}
	\tilde{\alpha} = 
	\mathrm{diag}\bracs{\alpha_1, \alpha_1, \alpha_1, \alpha_2, \alpha_2, \alpha_2, ..., \alpha_N, \alpha_N, \alpha_N} \in \complex^{3N\times 3N},
\end{align*}
and $\dgmap, \ngmap$ are as in \eqref{eq:DGMapDef}, \eqref{eq:NGMapDef}.
To complete the analogy, define the operator $\ag$ via the action
\begin{align*}
	\ag u &= 
	\begin{pmatrix}
		\sqbracs{ \rmi\wavenumber\bracs{\diff{}{y} + \rmi\qm_{jk} }u_3^{(jk)} + \wavenumber^2 U_2^{(jk)} }e_{jk}
		+ U_1^{(jk)} n_{jk} \\
		- \bracs{\diff{}{y} + \rmi\qm_{jk} }^2 u_3^{(jk)} + \rmi\wavenumber \bracs{\diff{}{y} + \rmi\qm_{jk} }U_2^{(jk)}
	\end{pmatrix}
\end{align*}
on each edge, where $\mathrm{dom}\bracs{\ag}$ consists of all functions $u$ with the following properties:
\begin{align*}
	u\in\mathrm{dom}\bracs{\ag} \quad\Leftrightarrow\quad &
	\begin{cases}
	u\in L^2\bracs{\graph}\times L^2\bracs{\graph}\times H^2\bracs{\graph}, \\
	\begin{pmatrix} u_1 \\ u_2 \end{pmatrix}\cdot e_{jk}\in \gradSob{I_{jk}}{y}, & \forall I_{jk}\in\edgeSet, \\
	u \text{ is continuous at } v_j, & \forall v_j\in\vertSet, \\
	\text{\eqref{eq:VertConditionExplicit} is satisfied at } v_j, & \forall v_j\in\vertSet.
	\end{cases}
\end{align*}
Then we have that
\begin{align*}
	\integral{I_{jk}}{ \ag u \cdot \overline{v} }{y} - \integral{I_{jk}}{ u \cdot \overline{\ag v} }{y}
	&= \sqbracs{ -u'_3 v_3 + u_3 v_3' - 2\rmi\qm_{jk}u_3 v_3 + \rmi\wavenumber\bracs{U_2 v_3 + u_3 V_2} }_{v_j}^{v_k} \\
	&= -\sqbracs{ \overline{v}_3\bracs{ \bracs{\diff{}{y} + \rmi\qm_{jk} }u_3 - \rmi\wavenumber U_2 } }_{v_j}^{v_k} \\
	&\qquad + \sqbracs{ u_3\overline{\bracs{ \bracs{\diff{}{y} + \rmi\qm_{jk} }v_3 - \rmi\wavenumber V_2 }} }_{v_j}^{v_k}.
\end{align*}
Which implies that
\begin{align*}
	&\ip{\ag u}{v}_{L^2\bracs{\graph}^3} - \ip{u}{\ag v}_{L^2\bracs{\graph}^3}
	= \sum_{v_j\in\vertSet}\sum_{j\conLeft k} \integral{I_{jk}}{ \ag u \cdot \overline{v} - u \cdot \overline{\ag v} }{y} \\
	&\quad = \sum_{v_j\in\vertSet}\sum_{j\conLeft k} -\sqbracs{ \overline{v}_3\bracs{ \bracs{\diff{}{y} + \rmi\qm_{jk} }u_3 - \rmi\wavenumber U_2 } }_{v_j}^{v_k}
	+ \sqbracs{ u_3\overline{\bracs{ \bracs{\diff{}{y} + \rmi\qm_{jk} }v_3 - \rmi\wavenumber V_2 }} }_{v_j}^{v_k} \\
	&\quad = \sum_{v_j\in\vertSet} u_3\bracs{v_j}\overline{\bracs{ \sum_{j\con k}\bracs{\pdiff{}{n} + \rmi\qm_{jk}}v_3 - \rmi\wavenumber\bracs{ \sum_{j\conRight k} V_2^{(kj)}\bracs{v_j} - \sum_{j\conLeft k} V_2^{(jk)}\bracs{v_j} } }} \\
	&\quad + \sum_{v_j\in\vertSet} \overline{v}_3\bracs{v_j}\bracs{ \sum_{j\con k}\bracs{\pdiff{}{n} + \rmi\qm_{jk}}u_3 - \rmi\wavenumber\bracs{ \sum_{j\conRight k} U_2^{(kj)}\bracs{v_j} - \sum_{j\conLeft k} U_2^{(jk)}\bracs{v_j} } } \\
	&\quad = \ngmap u \cdot \overline{\dgmap v} - \dgmap u \cdot \overline{\ngmap v}
	= \ip{\ngmap u}{\dgmap v}_{\complex^{3N}} - \ip{\dgmap u}{\ngmap v}_{\complex^{3N}},
\end{align*}
and so Green's identity holds. \tstk{do we even define a boundary triple in the QG chapter? If so, saying "green's identity" doesn't make much sense!}
Therefore, $\bracs{\complex^{3N}, \dgmap, \ngmap}$ is a boundary triple for the operator $\ag$.
Given the motivations for the definitions \eqref{eq:DGMapDef} and \eqref{eq:NGMapDef}, the $M$-operator associated with \eqref{eq:QGRawSystem} can be thought of as an analogue (or ``graph-version") of the Calder\'on operator for the problem \eqref{eq:SingStrucCurlCurl}.

% After derivation, we talk about how we failed to find a new system b/c curls are kinda redundant
%\section{Discussion and Research Outlook} \label{sec:CC-Discussion}
In this chapter we have performed a detailed analysis of the equation \eqref{eq:SingularCurlEquation} and the associated function spaces.
By extending the definitions and techniques from chapter \ref{ch:ScalarSystem}, we have been able to provide a definition for a curl-like object in one dimension (the tangential curl), and understand the properties of said object on our singular structure.
Complementing this analysis is the analysis and discussion of section \ref{sec:DivFreeCondition}, on the condition of being divergence free in our singular structure context.
It is possible to define a ``strong" notion of the divergence of a vector field in a similar manner to gradients and curls, \tstk{for which de Rham works}, but forces us to accept that the divergence of any vector field is the zero function.
Were we considering a problem in which the divergence appeared as an interior operator, this approach would be necessary for our understanding of the variational form of said problem, as we would need to explicitly work with the object $\ktdiv{\dddmes}u$.
However the problem \eqref{eq:SingularCurlEquation} does not require us to provide such a ``strong" definition of the divergence, and we instead interpret the divergence free condition weakly, as is done classically.
This enables us to characterise divergence-free vector fields, and its utility extends into exploration of the first order Maxwell system, which we discuss shortly.

Our efforts to turn \eqref{eq:SingularCurlEquation} into a more tractable problem in section \ref{sec:3DSystemDerivation} culminate in us arriving at the acoustic approximation.
This marks a departure from the classical setting when one considers the curl of the curl equation \eqref{eq:Intro-CurlCurlEqns} on a periodic medium in the $\bracs{x_1,x_2}$-plane that is extruded into $x_3$.
In such a setup, we only obtain the acoustic approximation from \eqref{eq:Intro-CurlCurlEqns} under the assumption of oblique wave propagation --- when the propagation constant in one of the axial directions is zero.
This would imply that we should only obtain \eqref{eq:SingularScalarWaveEqn} from \eqref{eq:SingularCurlEquation} when we set $\wavenumber=0$, however we obtain \eqref{eq:SingularScalarWaveEqn} regardless of the value of $\wavenumber$, and even have $\wavenumber$ appearing as part of an effective spectral parameter in $\omega^2-\wavenumber^2$.
The reason for this reduction lies in the nature of our variational problem itself --- specifically the dynamics induced by the tangential curls.
Let us imagine we now have a wave $u = u_0\e^{\rmi\vec{k}\cdot x}$ propagating in our material with wave-vector $\vec{k} = \qm_1\widehat{x}_1 + \qm_2\widehat{x}_2 + \wavenumber\widehat{x}_3$.
For each edge $I_{jk}$ (or plane $P_{jk}$) we can write this wave-vector using the local frame of reference provided by $\widehat{e}_{jk}$ and $\widehat{n}_{jk}$, obtaining $\vec{k}_{jk} := \qm_{jk}\widehat{e}_{jk} + \qm_{jk}^\perp\widehat{n}_{jk} + \wavenumber\widehat{x}_3$, where $\qm_{jk}^\perp := \qm\cdot n_{jk}$.
Now $\ktcurl{\dddmes}u$ on $I_{jk}$ is directed in the $\widehat{n}_{jk}$ direction, effectively only inducing dynamics or changes in $u$ in the $\widehat{e}_{jk}$ and $\widehat{x}_3$ directions.
There is no propagation out of the plane $P_{jk}$ as a result, which corresponds to when we have the component of $\vec{k}$ in the $\widehat{n}_{jk}$ direction being zero --- $\vec{k}\cdot\widehat{n}_{jk}=0$, which happens only if $\qm_{jk}^\perp=0$.
By analogy with the classical setting, one component of our wave-vector being zero will then result in the curl of the curl equation reducing to the acoustic approximation.
This occurs for each plane $P_{jk}$, and thus we find that our edge ODEs always reduce to the (singular analogue of the) acoustic approximation.
We also highlight that this is a direct consequence of the behaviour of the tangential curls on the singular structure geometry --- a union of planes induced by the extrusion of $\hat{\graph}$ --- which we have chosen to study.
There is no way to avoid such a reduction since it is tied to the nature of our singular structure being \emph{singular} --- we are always going to effectively ``loose" the ability to observe wave propagation in one direction.

This last observation does provide an idea that might prevent this collapse to the acoustic approximation:  we need part of our domain to be non-singular, which can be done by ``filling" the regions between the edges $I_{jk}$ with some background material.
Doing so will move us closer towards a true ``fibre-like" geometry, where we have a background material interlaced with periodic inclusions, and then extruded into three dimensions.
Pursuit of this idea naturally leads us on to the work in chapter \ref{ch:SingInc}.
Before moving on however, we highlight some further considerations and open problems that have been bought to the forefront of our attention in light of the results of this section.

\subsection{Open Questions} \label{ssec:CC-OpenQuestions}
Obtaining \eqref{eq:SingularScalarWaveEqn} from \eqref{eq:SingularCurlEquation} ultimately fulfils one of the objectives established at the start of this chapter: to provide a candidate for (the spectrum of) the limit of the curl of the curl equation on a thin structure as the thickness tends to zero.
With the success of our analogous variational approach in chapter \ref{ch:ScalarSystem} in obtaining the known limit of the acoustic equation on thin structures, we can put forward \eqref{eq:QGRawSystem} as a candidate for the limit of the aforementioned curl of the curl problems.
We have highlighted already that determining this limit (or the limit of the spectra) is currently an open problem in the literature.
By providing a candidate for the limit we motivate and invite analysis akin to \cite{kuchment2001convergence, kuchment2003asymptotics, exner2005convergence}; either to confirm that sending the thickness of a structure to zero results in the same effective problem as the acoustic setting, or to indicate the effects that are present in the thin structure setting that do not disappear in the singular limit but which are not captured by our variational problems.
The former being confirmed would further cement our approach via singular measures as a tool for probing limits of thin structures, and provide a useful justification for physical modelling of such structures in an electromagnetic context.
It could also motivate further analysis into the first order Maxwell system in such limits, of which the curl of the curl equation a consequence, similar to how the acoustic approximation is a consequence of a particular case of the curl of the curl equations. 
Confirmation of the latter case would also provide a number of insights into why our classically-motivated variational approach fails.
It may simply be the case that \eqref{eq:SingularCurlEquation} is not the correct analogue of \eqref{eq:Intro-CurlCurlEqns}, and with adjustments any effects that are lost can be recovered --- with the explanation as to how and why lying in the aforementioned analysis.
Or it may be that there is an alternative approach to defining gradients and curls to the one we have adopted, which is not so restrictive and preserves any effects our current approach looses.
In any event, this would shed further light onto the behaviour of the curl of the curl operator on increasingly fine thin structures.

Given that our investigation into the curl of the curl equation has resulted in us obtaining the acoustic approximation, it is natural to consider stepping back to a variational problem for the general Maxwell system;
\begin{subequations} \label{eq:CC-Maxwell1stOrder}
	\begin{align}
		\ktdiv{\dddmes} E = 0,
		&\qquad
		\ktcurl{*}E = \rmi\omega\mu_m H, \\
		\ktdiv{\dddmes} H = 0,
		&\qquad
		\ktcurl{*}H = -\rmi\omega\epsilon_m E,
	\end{align}
\end{subequations}
for the purpose of illustration, we will assume the material parameters are constants. 
A thorough analysis would establish a formal definition for the Maxwell operator, the eigenvalue problem for which corresponds to \eqref{eq:CC-Maxwell1stOrder}, which we interpret as the problem of finding divergence-free $E,H\in\pltwo{\ddom}{\dddmes}^3$ such that
\begin{subequations} \label{eq:CC-Maxwell1stOrderVariational}
	\begin{align} 
		\integral{\ddom}{ E\cdot\overline{\ktcurl{}\phi} - \rmi\omega\mu_m H\cdot\overline{\phi} }{\dddmes} &= 0, 
		\qquad &\forall \phi\in\psmooth{\ddom}^3, \\
		\integral{\ddom}{ H\cdot\overline{\ktcurl{}\psi} + \rmi\omega\epsilon_m E\cdot\overline{\psi} }{\dddmes} &= 0,
		\qquad &\forall \psi\in\psmooth{\ddom}^3.
	\end{align}
\end{subequations}
One has to answer a number of new questions when attempting to examine the first order Maxwell operator.
Definition \ref{def:DivFree-AllGradients} translates directly into the first order context, but we write the ``curls" in \eqref{eq:CC-Maxwell1stOrder} with a subscript asterisk to distinguish them from the tangential curls that belong to fields in $\ktcurlSob{\ddom}{\dddmes}$.
Unlike in \eqref{eq:SingularCurlEquation}, the curl appears as a first-order operator in \eqref{eq:CC-Maxwell1stOrder}, and so there is no guarantee that the object $\ktcurl{*}E$ coincides with $E\in\ktcurlSob{\ddom}{\dddmes}$.
Indeed, the objects $\ktcurl{*}E$ and $\ktcurl{*}H$ in \eqref{eq:CC-Maxwell1stOrder} are only really ``defined" upon determining a solution $E,H$ to \eqref{eq:CC-Maxwell1stOrderVariational}.
Circumnavigating these issues and performing an analysis of the resulting system is the natural next step in pursuit of understanding the Maxwell system on singular structures, and the behaviour of the modes supported by such structures.
There is also the related question of whether the singular curl of the curl equation \eqref{eq:SingularCurlEquation} would still be a consequence of the more general (singular) Maxwell system \eqref{eq:CC-Maxwell1stOrder} that is defined.
More precisely, is it true that a solution $E,H$ to \eqref{eq:CC-Maxwell1stOrder} gives rise to a solution $\bracs{u, \ktcurl{\dddmes}u }\in\ktcurlSob{\ddom}{\dddmes}$ to \eqref{eq:SingularCurlEquation}? \tstk{here}

% Then there's a conclusions chapter

% Now the appendicies, with the curl-curl theory :) ...

%\input{./Chapters/Curl-Curl/Old_CC_Appendix}

\section{Sobolev Spaces of Functions with Curls} \label{sec:CC-CurlAnalysis}
In similar vein to section \ref{sec:3DGradSobSpaces}, our focus for this section will be the analysis of curls of zero and tangential curls with respect to the measures $\ddmes$, $\massMes$ and $\dddmes$.
We will again utilise the thematic approach of the aforementioned section; first looking to understand curls of zero and tangential curls on (planes induced by an) edge $I_{jk}$, then show that we can combine such edge functions together to form functions on the whole graph $\graph$.

\subsection{Curls of Zero} \label{ssec:CurlsOfZero}
Utilising proposition \ref{prop:ZeroInvariantUnderQM-Wavenumber}, for any curl of zero $c$ we will (without loss of generality) take approximating sequences $\phi_n$ for which $\phi_n\rightarrow0$ and $\ocurl{}\phi\rightarrow c$.
Here, $\ocurl{}$ is the operator $\ktcurl{}$ with $\kt=\bracs{0,0}$.
Our analysis begins, analogously to how it did in section \ref{ssec:GradZero}, with an examination of $\curlZero{\ddom}{\lambda_{jk}}$.
\begin{prop} \label{prop:CurlZero-Parallel}
	Let $I_{jk}$ be an edge in $\ddom$ with $e_{jk} = \widehat{x}_2$.
	Then
	\begin{align*}
		\curlZero{\ddom}{\lambda_{jk}} &= \clbracs{ \bracs{0,c_2,c_3}^\top \setVert c_2,c_3\in\pltwo{\ddom}{\lambda_{jk}} }.
	\end{align*}
\end{prop}
The proof of this result is one particular case of the more general proof that we present for the following corollary, when we no longer assume that the edge $I_{jk}$ is parallel to the $x_2$-axis.
However we note that for any $c\in\pltwo{\ddom}{\lambda_{jk}}^3$, the function $\bracs{0,c_2,c_3}^{\top}$ is simply the function $c$ with its $\widehat{x}_1$ component removed, and in the setup of proposition \ref{prop:CurlZero-Parallel} $\widetilde{n}_{jk} = \widehat{x}_1$.
\begin{cory}[Curls of Zero on an Edge] \label{cory:CurlZero-Rotated}
	Let $I_{jk}$ be an edge of $\graph$.
	Then we have that
	\begin{align*}
		\curlZero{\ddom}{\lambda_{jk}} &= \clbracs{ c_{jk}\widehat{e}_{jk} + c_3\widehat{x}_3 \setVert c_{jk},c_3\in\pltwo{\ddom}{\lambda_{jk}} }.
	\end{align*}
\end{cory}
\begin{proof}
	Write $C = \clbracs{ c_{jk}\widehat{e}_{jk} + c_3\widehat{x}_3 \setVert c_{jk},c_3\in\pltwo{\ddom}{\lambda_{jk}} }$.
	
	First, suppose that $c\widehat{n}_{jk}\in\curlZero{\ddom}{\lambda_{jk}}$ and take an approximating sequence $\Phi^n$ for it.
	This means that we have the following convergences in $\ltwo{\ddom}{\lambda_{jk}}$:
	\begin{align*}
		\Phi^n_1 \rightarrow 0, \quad
		\Phi^n_2 &\rightarrow 0, \quad
		\Phi^n_3 \rightarrow 0, \\
		\partial_2\Phi^n_3 - \rmi\wavenumber\Phi^n_2 \rightarrow c n_1^{(jk)} = c e_2^{(jk)}, \quad
		\partial_1\Phi^n_3 - \rmi\wavenumber\Phi^n_1 &\rightarrow -c n_2^{(jk)} = c e_1^{(jk)}, \quad
		\partial_1\Phi^n_2 - \partial_2\Phi^n_1 \rightarrow 0.
	\end{align*}
	Therefore, we have that
	\begin{align*}
		\grad^{(0)}\Phi^n_3 \lconv{\ltwo{\ddom}{\lambda_{jk}}^2} c \widehat{e}_{jk},
	\end{align*}
	and $\Phi^n_3\rightarrow 0$ in $\ltwo{\ddom}{\lambda_{jk}}$.
	As such, $c\widehat{e}_{jk}\in\gradZero{\ddom}{\lambda_{jk}}$, and then proposition \ref{prop:3DGradZeroRotated} implies that $c=0$.
	This informs us that 
	\begin{align*}
		\curlZero{\ddom}{\lambda_{jk}}\cap \bracs{\ltwo{\ddom}{\lambda_{jk}}^3\setminus C} = \clbracs{0}.
	\end{align*}
	
	Now, let $c\in\psmooth{\ddom}$.
	Take $\Phi, \Psi\in\csmooth{\ddom}$ which, within some open neighbourhood of $I_{jk}$, have the form
	\begin{align*}
		\Phi = \bracs{v_j - x}\cdot n_{jk} c\widehat{x}_3, &\quad
		\Psi = \bracs{v_j - x}\cdot n_{jk} c\widehat{e}_{jk}.
	\end{align*}
	This ensures that both $\Phi$ and $\Psi$ are zero on $I_{jk}$, as for $x\in I_{jk}$ we have that $v_j-x$ is some scalar multiple of $e_{jk}$.
	Furthermore, we also have that
	\begin{align*}
		\ocurl{}\Phi &= \begin{pmatrix} \bracs{v_j-x}\cdot n_{jk} \partial_2 c - c n_2^{(jk)} \\ -\bracs{v_j-x}\cdot n_{jk} \partial_1 c + c n_1^{(jk)} \\ 0 \end{pmatrix}
		= \begin{pmatrix} \bracs{v_j-x}\cdot n_{jk} \partial_2 c + c e_1^{(jk)} \\ -\bracs{v_j-x}\cdot n_{jk} \partial_1 c + c e_2^{(jk)} \\ 0 \end{pmatrix}, \\
		\ocurl{}\Psi &= \begin{pmatrix} 0 \\ 0 \\ \bracs{\partial_1 c + \partial_2 c}\bracs{x-v_j}\cdot n_{jk} + c\norm{n_{jk}}_2 \end{pmatrix}
		= \begin{pmatrix} 0 \\ 0 \\ \bracs{\partial_1 c + \partial_2 c}\bracs{x-v_j}\cdot n_{jk} + c \end{pmatrix},
	\end{align*}
	so again for $x\in I_{jk}$, we have that
	\begin{align*}
		\ocurl{}\Phi = c\widehat{e}_{jk}, \qquad \ocurl{}\Psi = c\widehat{x}_3,
	\end{align*}
	whenever $x\in I_{jk}$, and thus $c\widehat{e}_{jk}, c\widehat{x}_3\in\curlZero{\ddom}{\lambda_{jk}}$.
	Density of $\psmooth{\ddom}$ in $\pltwo{\ddom}{\lambda_{jk}}$ then implies that $C\subset\curlZero{\ddom}{\lambda_{jk}}$.
	Combined with our earlier deduction, the proof is complete.
\end{proof}

Corollary \ref{cory:CurlZero-Rotated} provides the backbone for our geometric interpretation of curls of zero and tangential curls, given in section \ref{sec:CC-Geometric}.
The curls of zero always lie in the plane induced by $I_{jk}$, as vectors within the plane only serve to induce rotations perpendicular to said plane.
In three dimensions (that is, with the usual Lebesgue measure) we would be able to observe such rotations, however with our limited view (through $\lambda_{jk}$), we cannot.
As one might expect, the behaviour of curls of zero on $\graph$ can be characterised by the behaviour of in the individual edges, as was the case for gradients of zero and tangential gradients.
\begin{theorem}[Characterisation of $\curlZero{\ddom}{\ddmes}$] \label{thm:CurlZeroChar}
	We have that
	\begin{align*}
		\curlZero{\ddom}{\ddmes} = \clbracs{ c\in\pltwo{\ddom}{\ddmes}^3 \setVert c\in\curlZero{\ddom}{\lambda_{jk}}, \ \forall I_{jk}\in\edgeSet }.
	\end{align*}
\end{theorem}
For ease, let us denote $C = \clbracs{ c\in\pltwo{\ddom}{\ddmes}^3 \setVert c\in\curlZero{\ddom}{\lambda_{jk}}, \ \forall I_{jk}\in\edgeSet }$, and we note that thanks to corollary \ref{cory:CurlZero-Rotated} we have that
\begin{align*}
	C &= \clbracs{ c\in\pltwo{\ddom}{\ddmes}^3 \setVert c\cdot\widehat{n}_{jk}\vert_{I_{jk}}=0, \ \forall I_{jk}\in\edgeSet }.
\end{align*}

The method of proof proceeds in much the same way as that of proposition \ref{prop:3DGradZeroChar}, the inclusion $\curlZero{\ddom}{\ddmes}\subset C$ is immediate, the reverse direction requires us to deal with the issue of reconciling the different edge functions of curls of zero as we approach the vertices.
Our first target is the aforementioned easier inclusion:
\begin{lemma} \label{lem:CurlZeroInC}
	\begin{align*}
		\curlZero{\ddom}{\ddmes} \subset C.
	\end{align*}
\end{lemma}
\begin{proof}
	Let $c\in\curlZero{\ddom}{\ddmes}$, and take an approximating sequence $\Phi^n$.
	Clearly we have that
	\begin{align*}
		\norm{ \Phi^n }_{\ltwo{\ddom}{\lambda_{jk}}^3} &\leq \norm{ \Phi^n }_{\ltwo{\ddom}{\ddmes}^3} \rightarrow 0, \\
		\norm{ \ocurl{}\Phi^n - c }_{\ltwo{\ddom}{\lambda_{jk}}^3} & \leq \norm{ \ocurl{}\Phi^n - c }_{\ltwo{\ddom}{\ddmes}^3} \rightarrow 0,
	\end{align*}
	and so $c\in C$.
\end{proof}

Next, we require an extension lemma for curls of zero (compare this result with the role of lemma \ref{lem:3DExtensionLemmaGrads} for gradients of zero), and from this we will use closure of $\curlZero{\ddom}{\ddmes}$ to show the inclusion $\curlZero{\ddom}{\ddmes}\supset C$.
\begin{lemma}[Extension lemma for $\curlZero{\ddom}{\lambda_{jk}}$] \label{lem:CurlZeroExtensionLemma}
	Let $n\in\naturals$ and $I_{jk}^n$ be as in \eqref{eq:ShortenedEdgeDef}.
	Suppose that $c\in\curlZero{\ddom}{\lambda_{jk}}$ with $c=0$ on $I_{jk}\setminus I_{jk}^n$.
	Then
	\begin{align*}
		c\in\curlZero{\ddom}{\ddmes}.
	\end{align*}
	Furthermore, define $\tilde{c}$ via
	\begin{align*}
		\tilde{c} &= \begin{cases} c & x\not\in\vertSet, \\ 0 & x\in\vertSet. \end{cases}
	\end{align*}
	Then we also have that $\tilde{c}\in\curlZero{\ddom}{\dddmes}$.
\end{lemma}
\begin{proof}
	Let $\Phi^l$ be an approximating sequence for $c$ as in \eqref{eq:CurlZeroSequenceDef}, and $\chi_{jk}^n$ be the smooth function defined in \eqref{eq:SmoothChiDef}.
	Recall that we can choose $\chi_{jk}^n$ such that $\sup\abs{\grad\chi_{jk}^n}\leq kn$ for some constant $k$.
	Consider the sequence $\Psi^l := \chi_{jk}^n \Phi^l$, by construction we have that
	\begin{align*}
		\integral{\ddom}{ \abs{\Psi^l}^2 }{\ddmes} \leq \integral{I_{jk}}{ \abs{\Phi^l}^2 }{\lambda_{jk}} \rightarrow 0 \toInfty{l}.
	\end{align*}
	Since $\chi_{jk}^n$ and $\Phi^l$ are smooth, we can use the identity
	\begin{align*}
		\ocurl{}\Psi^l &= \ograd\chi_{jk}^n\wedge\Phi^l + \chi_{jk}^n\ocurl{}\Phi^l,
	\end{align*}
	to deduce that
	\begin{align*}
		\integral{\ddom}{ \abs{ \ocurl{}\Psi^l - c }^2 }{\ddmes}
		&\leq 2\integral{I_{jk}}{ \abs{ \chi_{jk}^n\ocurl{}\Phi^l - c }^2 }{\lambda_{jk}}
		+ 2\sup\abs{\ograd\chi_{jk}^n}^2 \integral{I_{jk}}{ \abs{ \Phi^l }^2 }{\lambda_{jk}} \\
		&\leq2\integral{I_{jk}}{ \abs{ \ocurl{}\Phi^l - c }^2 }{\lambda_{jk}}
		+ 2(kn)^2 \integral{I_{jk}}{ \abs{ \Phi^l }^2 }{\lambda_{jk}} \\
		&\rightarrow 0 \toInfty{l}.
	\end{align*}
	The sequence $\Psi^l$ now serves as the required approximating sequence, giving us that $c\in\curlZero{\ddom}{\ddmes}$.
	Furthermore, we also notice that $\Psi^l\bracs{v_j}=0$ and $\curl{}\Psi^l\bracs{v_j}=0$ for every $v_j\in\vertSet$, so we additionally have that
	\begin{align*}
		\norm{\Psi^l}_ {\ltwo{\ddom}{\massMes}^3} = 0
		= \norm{\ocurl{}\Psi^l}_{\ltwo{\ddom}{\massMes}^3},
	\end{align*}
	and thus
	\begin{align*}
		\Psi^l \lconv{\ltwo{\ddom}{\dddmes}^3} 0, 
		\qquad
		\ocurl{}\Psi^l \lconv{\ltwo{\ddom}{\dddmes}^3} \tilde{c},
	\end{align*}
	which demonstrates that $\tilde{c}\in\curlZero{\ddom}{\dddmes}$.
\end{proof}

\begin{lemma} \label{lem:CInCurlZero}
	\begin{align*}
		C \subset \curlZero{\ddom}{\ddmes}.
	\end{align*}
	Furthermore, for $c\in C$ the function
	\begin{align*}
		\tilde{c} &= \begin{cases} c & x\not\in\vertSet, \\ 0 & x\in\vertSet, \end{cases}
	\end{align*}
	is an element of $\curlZero{\ddom}{\dddmes}$.
\end{lemma}
\begin{proof}
	Let $\eta_j^n$ be as in \eqref{eq:SmoothEtaDef}, take $c\in C$, and define two families of functions $c_n$, $\tilde{c}_n$ by
	\begin{align*}
		c_n = \sum_{v_j\in\vertSet}\sum_{j\conLeft k} \eta_j^n \eta_k^n c_{jk},
		\qquad
		\tilde{c}_n = \begin{cases} c_n & x\not\in\vertSet, \\ 0 & x\in\vertSet. \end{cases}
	\end{align*}
	For each $n\in\naturals$ and $j\conLeft k$, the function $\eta_j^n \eta_k^n c_{jk}$ satisfies the hypothesis of the extension lemma \ref{lem:CurlZeroExtensionLemma}, and so is an element of $\curlZero{\ddom}{\ddmes}$, and its extension by zero the the vertices of $\graph$ is an element of $\curlZero{\ddom}{\dddmes}$.
	As both $\curlZero{\ddom}{\ddmes}$ and $\curlZero{\ddom}{\dddmes}$ are linear subspaces, $c_n\in\curlZero{\ddom}{\ddmes}$ and $\tilde{c}_n\in\curlZero{\ddom}{\dddmes}$ for every $n\in\naturals$.
	Since $\eta_j^n\rightarrow 1$ in $\ltwo{\ddom}{\ddmes}$, we conclude that $c_n\rightarrow c$ in $\ltwo{\ddom}{\ddmes}^3$, and since $\curlZero{\ddom}{\ddmes}$ is closed, we have that $c\in\curlZero{\ddom}{\ddmes}$.
	Similarly we can conclude that $\tilde{c}$ is the limit of $\tilde{c}_n$, and thus is an element of $\curlZero{\ddom}{\dddmes}$.
\end{proof}
Lemmas \ref{lem:CurlZeroInC}, \ref{lem:CurlZeroExtensionLemma}, and \ref{lem:CInCurlZero} then form the proof of theorem \ref{thm:CurlZeroChar}.

Our next brief stop will be an examination of the set $\curlZero{\ddom}{\massMes}$, before we then look to fully characterise curls of zero with respect to $\dddmes$.
Much of the same intuition that we had for gradients also holds for curls in the context of the measure $\massMes$ --- given that $\massMes$ only respects a finite number of points in $\ddom$, there is no way to quantify rotations at these points, and so all $\ltwo{\ddom}{\massMes}^3$-vector fields turn out to be curls of zero.
\begin{prop} \label{prop:VertexCurlZero}
	\begin{align*}
		\curlZero{\ddom}{\massMes} &= \pltwo{\ddom}{\massMes}^3.
	\end{align*}
\end{prop}
\begin{proof}
	For each $j\in\vertSet$ and $k\in\clbracs{1,2,3}$, let $c^k_1\in\pltwo{\ddom}{\massMes}^3$ be the function
	\begin{align*}
		c^j_k\bracs{x} &= \begin{cases} e_k & x=v_j, \\ 0 & x\neq v_j, \end{cases}
	\end{align*}
	where $e_k$ is the $k$\textsuperscript{th} canonical unit vector in $\complex^3$.
	Note that the collection 
	\begin{align*}
		\mathcal{C} = \clbracs{c^j_k \ \setVert v_j\in\vertSet, k\in\clbracs{1,2,3}}
	\end{align*}
	is a basis for $\pltwo{\ddom}{\massMes}^3$, so it is sufficient to show that $\mathcal{C}\subset\curlZero{\ddom}{\massMes}$, since $\curlZero{\ddom}{\massMes}$ is a closed linear subspace of $\ltwo{\ddom}{\massMes}^3$.
	To this end, let $d=\min_\edgeSet\abs{I_{jk}}$ and $\psi:\reals\rightarrow\reals$ be a smooth function with 
	\begin{align*}
		\psi(0) = 0, \quad \psi'(0) = 1, \quad \supp\bracs{\psi} \subset \bracs{-d,d}.
	\end{align*}
	Then for each $v_j\in\vertSet$, the function $\Phi_j\in\csmooth{\ddom}^3$ via 
	\begin{itemize}
		\item $\Phi_j(x) = \bracs{0, 0, \psi\bracs{x_2 - v_j^{(2)}}}^\top$ is such that $\Phi_j = 0, \ \ocurl{}\Phi_j = c^j_1$ in $\ltwo{\ddom}{\massMes}^3$,
		\item $\Phi_j(x) = \bracs{0, 0, \psi\bracs{v_j^{(1)} - x_1}}^\top$ is such that $\Phi_j = 0, \ \ocurl{}\Phi_j = c^j_2$ in $\ltwo{\ddom}{\massMes}^3$,
		\item $\Phi_j(x) = \bracs{0, \psi\bracs{x_1 - v_j^{(1)}}, 0}^\top$ is such that $\Phi_j = 0, \ \ocurl{}\Phi_j = c^j_3$ in $\ltwo{\ddom}{\massMes}^3$,
	\end{itemize}
	and thus (possibly after multiplying by a smooth cut-off function equal to unity on some neighbourhood of $v_j$) we see that $\mathcal{C}\subset\curlZero{\ddom}{\massMes}$, and the result follows.
\end{proof}

Although we usually conduct a study of tangential curls after completing our study of the curls of zero (akin to our methodology for studying gradients of zero and tangential gradients), there is little point in delaying the obvious characterisation for $\ktcurlSob{\ddom}{\massMes}$.
\begin{cory} \label{cory:VertexCurlSob}
	\begin{align*}
		\ktcurlSob{\ddom}{\massMes} &= \clbracs{ \bracs{u,0} \setVert u\in\pltwo{\ddom}{\massMes}^3 }.
	\end{align*}
\end{cory}
\begin{proof}
	($\subset$) If $u\in\ktcurlSob{\ddom}{\massMes}$ with $\ktcurl{\massMes}u=\bracs{v_1,v_2,v_3}^\top$ then the requirement that $\ktcurl{\massMes}u \perp \curlZero{\ddom}{\massMes}$ and proposition \ref{prop:VertexCurlZero} imply that $v_1=v_2=v_3=0$.
	
	($\supset$) Conversely if $u\in\ltwo{\ddom}{\massMes}^3$, for each $v_j\in\vertSet$ let $\psi_j$ denote the smooth ``bump" function centred on $v_j$ with
	\begin{align*}
		\psi_j\bracs{v_j} = 1, \quad
		\psi_j'\bracs{v_j} = 0, \quad
		\supp\bracs{\psi_j} \subset B_d\bracs{v_j},
	\end{align*}
	where $d=\min_{\edgeSet}\abs{I_{jk}}$.
	Setting $\Phi(x) = \sum_{v_j\in\vertSet}u\bracs{v_j}\psi_j\bracs{x}$, we have that $\Phi=u$ in $\ltwo{\ddom}{\massMes}^3$ so $\bracs{u,\ktcurl{}\Phi}\in W^{\kt}_{\mathrm{curl}}$.
	Therefore, there exists some $c\in\curlZero{\ddom}{\massMes}^{\perp}$ such that $\bracs{u,c}\in\ktcurlSob{\ddom}{\massMes}$, however from the ``$\subset$" inclusion we can then conclude that $c = 0$.
\end{proof}

We now combine our understanding of $\curlZero{\ddom}{\ddmes}$ and $\curlZero{\ddom}{\massMes}$ to show that linear combinations of functions in these spaces make up the set $\curlZero{\ddom}{\dddmes}$.
\begin{prop} \label{prop:ThickVertexCurlZeroCharacterisation}
	Let $\tilde{c}\in\ltwo{\ddom}{\dddmes}^3$ where
	\begin{align*}
		\tilde{c} &= \begin{cases} c_{\ddmes} & x\not\in\vertSet, \\ c_{\massMes} & x\in\vertSet, \end{cases}
	\end{align*}
	for $c_{\ddmes}\in\pltwo{\ddom}{\ddmes}^3$ and $c_{\massMes}\in\pltwo{\ddom}{\massMes}^3$.
	Then we have that
	\begin{align*}
		c\in\curlZero{\ddom}{\dddmes} \quad\Leftrightarrow\quad 
		& c_{\ddmes}\in\curlZero{\ddom}{\ddmes} \text{ and } c_{\massMes}\in\curlZero{\ddom}{\massMes}.
	\end{align*}
\end{prop}
\begin{proof}
	($\Rightarrow$) For the right-directed implication it is sufficient to notice that 
	\begin{align*}
		\norm{\cdot}_{\ltwo{\ddom}{\dddmes}}^2 &= \norm{\cdot}_{\ltwo{\ddom}{\ddmes}}^2 + \norm{\cdot}_{\ltwo{\ddom}{\massMes}}^2,
	\end{align*}
	so any approximating sequence for $c$ that converges in $\ltwo{\ddom}{\dddmes}^3$ also converges in $\ltwo{\ddom}{\ddmes}^3$ to $c_{\ddmes}$ and in $\ltwo{\ddom}{\massMes}^3$ to $c_{\massMes}$.
	
	($\Leftarrow$) For the left-directed implication it is sufficient for us to demonstrate the implication holds when $c_{\massMes}=0$, then when $c_{\ddmes}=0$ and $c_{\massMes}\neq0$ at precisely one vertex $v$.
	Having shown the implication in these cases, linearity of $\curlZero{\ddom}{\dddmes}$ will then complete the proof.
	
	The case with $c_{\massMes}=0$ is handled by the extension lemma \ref{lem:CurlZeroExtensionLemma} and lemma \ref{lem:CInCurlZero}.
	
	So consider the case when $c_{\ddmes}=0$, and when $c_{\massMes}=0$ at all vertices except $v\in\vertSet$, with $v=\bracs{v_1, v_2}\in\ddom$ and with $c_{\massMes}(v) = \bracs{c_1, c_2, c_3}^\top$.
	For each $n\in\naturals$ take a smooth function $\phi_n:\reals\rightarrow\sqbracs{-1,1}$ with the properties
	\begin{align*}
		\phi_n(0) = 0,
		&\quad	\phi'_n(0) = 1, \\
		\supp\bracs{\phi_n} &\subset\bracs{-\frac{2}{n},\frac{2}{n}}, \\
		%\phi_n(t) = 0, &\quad t\not\in B_{\frac{2}{n}}(0), \\
		\abs{\phi_n(y)} \leq \recip{n}, &\quad y\in \bracs{-\frac{2}{n},\frac{2}{n}}.
		% B_{\frac{2}{n}}(0).
	\end{align*}
	Since $\abs{\phi_n(y)} \leq \recip{n}$, $\phi_n$ can be chosen so that exists a constant $K$ independent of $n$ such that $\abs{\phi_n'(y)} \leq K$ for any $y\in\supp\bracs{\phi_n}$.
	Define the functions $\Phi^n\in\csmooth{\ddom}$ for all $n\geq\recip{d}$ as follows;
	\begin{align*}
		\Phi^n(x) = 
		\begin{pmatrix} 
			0 \\ 
			c_3\phi_n\bracs{x_1 - v_1} \\ 
			c_1\phi_n\bracs{x_2-v_2} + c_2\phi_n\bracs{v_1-x_1} 
		\end{pmatrix},
		&\qquad
		\ocurl{}\Phi_n(x) =
		\begin{pmatrix}
			c_1\phi'_n\bracs{x_2-v_2} \\
			c_2\phi'_n\bracs{v_1-x_1} \\
			c_3\phi'_n\bracs{x_1-v_1}
		\end{pmatrix}.
	\end{align*}
	Then we have the following:
	\begin{align*}
		\integral{\ddom}{ \abs{ \Phi^n }^2 }{\dddmes}
		&\leq \bracs{c_3^2 + \bracs{c_1 + c_2}^2} \bracs{ \integral{B_{\frac{2}{n}}(v)}{ \recip{n^2} }{\ddmes}
		+ \alpha_v\abs{\phi_n(0)}^2 } \\
		&= \frac{2\mathrm{deg}(v)}{n^3}\bracs{\abs{c_3}^2 + \abs{c_1 + c_2}^2} \rightarrow 0 \toInfty{n}, \\
		\integral{\ddom}{ \abs{ \ocurl{}\Phi^n - c }^2 }{\dddmes}
		&= \integral{\ddom}{ \abs{ \ocurl{}\Phi^n}^2 }{\ddmes}
		+ \integral{\ddom}{ \abs{ \ocurl{}\Phi^n - c_{\massMes} }^2 }{\massMes} \\
		&= \abs{c_1}^2\integral{\ddom}{ \abs{\phi'_n\bracs{x_2-v_2}}^2 }{\ddmes}
		+ \abs{c_2}^2\integral{\ddom}{ \abs{\phi'_n\bracs{v_1-x_1}}^2 }{\ddmes} \\
		&\quad + \abs{c_3}^2\integral{\ddom}{ \abs{\phi'_n\bracs{x_1-v_1}}^2 }{\ddmes}
		+ \alpha_v\abs{c(v)}^2\abs{\phi'_n(0)-1}^2 \\
		&\leq \abs{c(v)}^2 \bracs{ K^2\integral{B_{\frac{2}{n}}(v)}{ }{\ddmes}
		+ \alpha_v\abs{\phi'_n(0)-1}^2 } \\
		&= \frac{2\mathrm{deg}(v)}{n}\abs{c(v)}^2 K^2 \rightarrow 0 \toInfty{n},
	\end{align*}
	where $\mathrm{\deg}(v)$ is the degree of the vertex $v$.
	We thus conclude that $c\in\curlZero{\ddom}{\dddmes}$.
	With the linearity of $\curlZero{\ddom}{\dddmes}$ and the previous case, the proof is complete.
\end{proof}

\subsection{Tangential Curls} \label{ssec:TangCurls}
With theorem \ref{thm:CurlZeroChar} in hand, we are ready to explore the form of tangential curls with respect to the measures $\ddmes$ and $\dddmes$.
Note that we have already identified the curls of zero with respect to $\massMes$, through corollary \ref{cory:VertexCurlSob}.
We begin, as has become standard, with an edge parallel to one of the coordinate axes.
\begin{prop} \label{prop:TangCurlEdgeParallel}
	Let $I_{jk}$ be parallel to the $x_2$ axis, and write $\gradSob{\sqbracs{0,l_{jk}}}{y}$ for the classical Sobolev space of functions on $\sqbracs{0,l_{jk}}$ with respect to the Lebesgue measure.
	Suppose that $u\in\ktgradSob{\ddom}{\lambda_{jk}}$ and set $\tilde{u}=u\circ r_{jk}$ on $I_{jk}$.
	Then $\tilde{u}_3\in\gradSob{\sqbracs{0,l_{jk}}}{y}$ and 
	\begin{align*}
		\ktcurl{\lambda_{jk}}u &= \bracs{ u_3' + \rmi\qm_2 u_3 - \rmi\wavenumber u_2 }\widehat{x}_1,
	\end{align*}
	where $u_3'=\tilde{u}_3'\circ r_{jk}^{-1}$.
\end{prop}

The proof follows from the general argument presented in corollary \ref{cory:TangCurlEdgeRotated}.
However proposition \ref{prop:TangCurlEdgeParallel} is useful for validating our expectations that the tangential curl is directed ``out of" the plane induced by $I_{jk}$, when the geometry of the edge simplified by being aligned to the coordinate axes.
We also observe that we have the operation $\partial_2+\rmi\qm_2$ acting on $u_3$ and $\rmi\wavenumber$ acting on $u_2$ --- effectively the expression $\grad u_3\cdot\widehat{e}_{jk} - \partial_3 u_2$, which resembles the ``in-plane" part of the classical curl.
\begin{cory} \label{cory:TangCurlEdgeRotated}
	Let $I_{jk}\in\edgeSet$, and let $u\in\ktcurlSob{\ddom}{\lambda_{jk}}$.
	Define
	\begin{align*}
		U(x) = R_{jk}\begin{pmatrix} u_1(x) \\ u_2(x) \end{pmatrix},
		\qquad
		\tilde{u}_3 &= u_3\circ r_{jk}, 
		\qquad
		\qm_{jk} = \qm\cdot e_{jk}.
	\end{align*}
	Then $\tilde{u}_3\in\gradSob{\sqbracs{0,l_{jk}}}{y}$ and
	\begin{align*}
		\ktcurl{\lambda_{jk}}u &= \bracs{ u_3' + \rmi\qm_{jk}u_3 - \rmi\wavenumber U_2 }\widehat{n}_{jk},
	\end{align*}
	where $u_3' := \tilde{u}_3'\circ r_{jk}^{-1}$.
\end{cory}
It is worth remarking that $U_2 = \bracs{u_1,u_2}^{\top}e_{jk}$ on $I_{jk}$, so the tangential curl only retains the projection of the components of $u$ onto the plane $P_{jk}$.
\begin{proof}
	Let us write $c = \ktcurl{\lambda_{jk}}u$.
	The requirement that $c\perp\curlZero{\ddom}{\lambda_{jk}}$, combined with theorem \ref{thm:CurlZeroChar}, implies that
	\begin{align*}
		c\cdot\widehat{e}_{jk} = 0 = c\cdot\widehat{x}_3,
	\end{align*}
	so we can write $c = c_{jk}\widehat{n}_{jk}$, for some $c_{jk}\in\pltwo{\ddom}{\lambda_{jk}}$ to be determined.
	To this end, let us now take an approximating sequence $\Phi^n$ for $u$, which provides us with the following convergences in $\ltwo{\ddom}{\lambda_{jk}}$:
	\begin{align*}
		\Phi^n_1 \rightarrow u_1, \quad 
		\Phi^n_2 &\rightarrow u_2, \quad
		\Phi^n_3 \rightarrow u_3, \\
		\bracs{\partial_2+\rmi\qm_2}\Phi^n_3 - \rmi\wavenumber\Phi^n_2 &\rightarrow c_{jk}n_1^{(jk)} = c_{jk}e_2^{(jk)}, \\
		\bracs{\partial_1+\rmi\qm_1}\Phi^n_3 - \rmi\wavenumber\Phi^n_1 &\rightarrow -c_{jk}n_2^{(jk)} = c_{jk}e_1^{(jk)}, \\
		\bracs{\partial_1 + \rmi\qm_1}\Phi^n_2 - \bracs{\partial_2+\rmi\qm_2}\Phi^n_1 &\rightarrow 0.
	\end{align*}
	In particular we notice that
	\begin{align*}
		\grad\Phi^n_3\cdot e_{jk} \rightarrow c_{jk} + \rmi\wavenumber U\cdot e_{jk} - \rmi u_3\qm\cdot e_{jk},
	\end{align*}
	and thus the sequence $\varphi_n := \Phi^n_3\circ r_{jk}$ is such that
	\begin{align*}
		\varphi_n\rightarrow\tilde{u}_3, \quad
		\varphi_n'\rightarrow c_{jk}\circ r_{jk} + \rmi\wavenumber\bracs{U\circ r_{jk}}\cdot e_{jk} - \rmi\tilde{u}_3\qm\cdot e_{jk}, \quad
		\text{in } \ltwo{\sqbracs{0,l_{jk}}}{y}.
	\end{align*}
	As such, we have that $\tilde{u}_3\in\gradSob{\sqbracs{0,l_{jk}}}{y}$, and upon rearrangement and undoing the change of variables $r_{jk}$, we obtain
	\begin{align*}
		c_{jk}(x) &= u_3'(x) + \rmi\qm_{jk}u_3(x) - \rmi\wavenumber U(x)\cdot e_{jk}
		= u_3'(x) + \rmi\qm_{jk}u_3(x) - \rmi\wavenumber U_2(x),
	\end{align*}
	providing the result.
\end{proof}

Our goal now is to characterise tangential curls belonging to functions in the spaces $\ktcurlSob{\ddom}{\ddmes}$ and $\ktcurlSob{\ddom}{\dddmes}$ using our edge-wise understanding from corollary \ref{cory:TangCurlEdgeRotated} and vertex-characterisation from corollary \ref{cory:VertexCurlSob}.
A simple comparison of norms shows us that the edge-wise behaviour must be inherited:
\begin{prop} \label{prop:TC-dddmesImpliesOthers}
	Suppose that $\bracs{u,\ktcurl{\dddmes}u}\in\ktcurlSob{\ddom}{\dddmes}$.
	Then we have that $\bracs{u,\ktcurl{\dddmes}u}\in\ktcurlSob{\ddom}{\ddmes}$ and $\bracs{u,\ktcurl{\dddmes}u}\in\ktcurlSob{\ddom}{\massMes}$ too.
\end{prop}
\begin{proof}
	Simply take an approximating sequence $\Phi^n$ for $u$, and notice that
	\begin{align*}
		\norm{\cdot}_{\ltwo{\ddom}{\dddmes}^3} &= \norm{\cdot}_{\ltwo{\ddom}{\ddmes}^3} + \norm{\cdot}_{\ltwo{\ddom}{\massMes}^3}.
	\end{align*}
	This implies that $\Phi^n$ and $\ktcurl{}\Phi^n$ also converge in these norms to (functions almost everywhere equal to) $u$ and $\ktcurl{\dddmes}u$ respectively.
	We are also assured that $\ktcurl{\dddmes}u$ is orthogonal to $\curlZero{\ddom}{\ddmes}$ and $\curlZero{\ddom}{\massMes}$ by theorem \ref{thm:CurlZeroChar}, which completes the proof.
\end{proof}

The proof of corollary \ref{cory:TangCurlEdgeRotated} hints at a connection between the third component $u_3$ of a function $u\in\ktcurlSob{\ddom}{\ddmes}$ and the space $\ktgradSob{\ddom}{\ddmes}$.
Indeed in this argument we saw that an approximating sequence for the curl also provides us with an approximating sequence for the third component \emph{and its gradient}.
Knowing that the third component of a function $u\in\ktcurlSob{\ddom}{\dddmes}$ is very informative, since we have theorem \ref{thm:CharOfSobSpaces} informing us that these functions are precisely those that can be constructed from edge functions, provided these edge functions are continuous at the vertices.
\begin{prop} \label{prop:TC-3rdComponentIFF}
	Let $u_3^{(jk)}\in\pltwo{\ddom}{\lambda_{jk}}$, $u_3\in\pltwo{\ddom}{\ddmes}$, and $\tilde{u}_3\in\pltwo{\ddom}{\dddmes}$.
	Then the following equivalences hold.
	\begin{align*}
		\mathrm{(i)} \ &
		u^{(jk)}:= \bracs{0,0,u_3^{(jk)}}^\top\in\ktcurlSob{\ddom}{\lambda_{jk}}
		\quad &\Leftrightarrow &\quad
		u_3^{(jk)}\in\ktgradSob{\ddom}{\lambda_{jk}}, \\
		\mathrm{(ii)} \ &
		u:= \bracs{0,0,u_3}^\top\in\ktcurlSob{\ddom}{\ddmes}
		\quad &\Leftrightarrow &\quad
		u_3\in\ktgradSob{\ddom}{\ddmes}, \\
		\mathrm{(iii)} \ &
		\tilde{u} := \bracs{0,0,\tilde{u}_3}^\top\in\ktcurlSob{\ddom}{\dddmes}
		\quad &\Leftrightarrow &\quad
		\tilde{u}_3\in\ktgradSob{\ddom}{\dddmes}.
	\end{align*}
\end{prop}
\begin{proof}
	We will explicitly prove the statement (ii), since the other statements (i) and (iii) follow essentially the same argument --- taking an approximating sequence, observing the convergences in $L^2$ that we obtain, and appealing the the characterisation of gradients or curls of zero depending on which implication is being shown.
	
	$(\Leftarrow)$ For the left-directed implication, let $\phi_n$ be an approximating sequence for $u_3$ and write $g = \ktgrad_{\ddmes}u$.
	Define the sequence $\Phi^n = \bracs{0,0,\phi_n}^\top\in\psmooth{\ddom}^3$, and notice that
	\begin{align*}
		\Phi^n\lconv{\ltwo{\ddom}{\ddmes}^3}u, \qquad
		\ktcurl{}\Phi^n = 
		\begin{pmatrix} 
			\bracs{\partial_2+\rmi\qm_2}\phi_n \\ -\bracs{\partial_1 + \rmi\qm_1}\phi_n \\ 0 
		\end{pmatrix}
		\lconv{\ltwo{\ddom}{\ddmes}^3}
		\begin{pmatrix}
			g_2 \\ -g_1 \\ 0
		\end{pmatrix}
		=: c.
	\end{align*}
	Note that, since $g\perp\gradZero{\ddom}{\ddmes}$, we have $g\perp\widehat{e}_{jk}$ on every edge $I_{jk}$, implying that $c\perp\widehat{n}_{jk}$ on each $I_{jk}$ and thus $c\perp\curlZero{\ddom}{\ddmes}$ by theorem \ref{thm:CurlZeroChar}.
	Therefore, $\Phi^n$ serves as an approximating sequence for $\bracs{u,c}\in\ktcurlSob{\ddom}{\ddmes}$.
	
	$(\Rightarrow)$ For the right-directed implication, let $\Phi^n$ be an approximating sequence for $u$ and write $c=\ktcurl{}u$.
	Define the sequence $\phi_n = \Phi^n_3$; similarly to the above, it can be shown that
	\begin{align*}
		\phi_n \lconv{\ltwo{\ddom}{\ddmes}} u_3, \qquad
		\ktgrad\phi_n \lconv{\ltwo{\ddom}{\ddmes}^3} \begin{pmatrix} -c_2 \\ c_1 \\ \rmi\wavenumber u_3 \end{pmatrix}
		=: g.
	\end{align*}
	Again by recognising $c\perp\curlZero{\ddom}{\ddmes}$ implies that $g\perp\gradZero{\ddom}{\ddmes}$, we can conclude that $\phi_n$ serves as an approximating sequence for $\bracs{u,g}\in\ktgradSob{\ddom}{\ddmes}$.
\end{proof}

Proposition \ref{prop:TC-3rdComponentIFF} also allows us to invoke theorem \ref{thm:CharOfSobSpaces} to conclude that the component $u_3$ is continuous at the vertices of $\graph$.
This just leaves questions concerning the behaviour of the components $u_1$ and $u_2$.
We have already seen from proposition \ref{prop:TC-dddmesImpliesOthers} that a function on the whole graph breaks down edge-wise, however we lack continuity of $u_1$ and $u_2$ at the vertices (and even weak differentiability along the edges), so cannot utilise an argument that will allow us to combine these edge-functions together like we did with scalar functions possessing gradients.
The best we can do (utilising only knowledge of curls) comes when we have a vector field whose first and second components have support contained in the interior of an edge, analogous to lemma \ref{lem:ExtensionLemmaEdgeFunctions}.
\begin{prop}
	Suppose that $=\bracs{u_1,u_2,0}^\top\in\ktcurlSob{\ddom}{\lambda_{jk}}$ with $u=0$ outside $I_{jk}^\eps$ for some $\eps>0$.
	Then $u\in\ktcurlSob{\ddom}{\ddmes}$ and $u\in\ktcurlSob{\ddom}{\dddmes}$ with
	\begin{align*}
		\ktcurl{\ddmes}u =
		\begin{cases} 
			\ktcurl{\lambda_{jk}}u & x\in I_{jk}, \\
			0 & x\not\in I_{jk}, 
		\end{cases}
		\qquad
		\ktcurl{\dddmes}u = 
		\begin{cases} 
			\ktcurl{\lambda_{jk}}u & x\in I_{jk}, \\
			0 & x\not\in I_{jk}, \\
			0 & x\in\vertSet.
		\end{cases}
	\end{align*}
\end{prop}
\begin{proof}
	Take an approximating sequence $\Phi^n$ for $u$, and consider the smooth cut-off function $\chi_{jk}^{\eps}$ in \eqref{eq:SmoothChiDef}.
	Construct the sequence $\Psi^n = \chi_{jk}^{\eps}\Phi^n$, and note that for each $n\in\naturals$, 
	\begin{align*}
		\Psi^n\bracs{v_j}=0, \qquad
		\ktcurl{}\Psi_n = \chi_{jk}^{\eps}\ktcurl{}\Phi^n + \grad\chi_{jk}^{\eps}\wedge\Phi^n, \qquad
		\ktcurl{}\Psi_n\bracs{v_j} = 0.
	\end{align*}
	We then observe that
	\begin{align*}
		\integral{\ddom}{ \abs{\Psi^n-u}^2 }{\ddmes}
		&= \integral{\ddom}{ \abs{\chi_{jk}^\eps\Phi^n-u}^2 }{\lambda_{jk}} \\
		&\leq \integral{\ddom}{ \abs{\Phi^n-u}^2 }{\lambda_{jk}} \rightarrow 0 \toInfty{n}.
	\end{align*}
	Furthermore, since $\grad\chi_{jk}^\eps$ is only non-zero outside the support of $u$, we have that
	\begin{align*}
		\recip{2}\integral{\ddom}{ & \abs{\ktcurl{}\Psi^n - \ktcurl{\ddmes}u}^2 }{\ddmes} \\
		&\leq \integral{\ddom}{ \abs{\chi_{jk}^{\eps}\ktcurl{}\Phi^n - \ktcurl{\ddmes}u}^2 }{\ddmes}
		+ \integral{\ddom}{ \abs{\grad\chi_{jk}^{\eps}\wedge\Phi^n}^2 }{\lambda_{jk}} \\
		&\leq \integral{\ddom}{ \abs{\ktcurl{}\Phi^n - \ktcurl{\ddmes}u}^2 }{\lambda_{jk}} %\\
%		&\qquad 
		+ \integral{\ddom}{ \abs{\grad\chi_{jk}^\eps}^2\abs{\Phi^n - u}^2 }{\lambda_{jk}} \\
		&\leq \norm{\ktcurl{}\Phi^n - \ktcurl{\ddmes}u}_{\ltwo{\ddom}{\lambda_{jk}}^3} \\
		&\qquad + \sup\abs{\grad\chi_{jk}^\eps}^2\norm{\Phi^n - u}_{\ltwo{\ddom}{\lambda_{jk}}^3} \\
		&\rightarrow0 \toInfty{n}.
	\end{align*}
	This gives us the required convergences for $u\in\ktcurlSob{\ddom}{\ddmes}$, we also observe that $u\in\ktcurlSob{\ddom}{\dddmes}$ through use of the same approximating sequence as
	\begin{align*}
		\norm{\Psi^n-u}_{\ltwo{\ddom}{\massMes}^3} = 0 = \norm{\ktcurl{}\Psi^n}_{\ltwo{\ddom}{\massMes}^3}
	\end{align*}
	for every $n\in\naturals$.
\end{proof}

The issue preventing us from establishing the converse implication to that in proposition \ref{prop:TC-dddmesImpliesOthers} is the aforementioned lack of control over the components $u_1$ and $u_2$ near the vertices of a graph.
In the case of tangential gradients, the we had to depend on continuity of the incoming edge functions at the vertex to circumnavigate this issue, and this continuity does not hold for the components $u_1$ and $u_2$ on approach to shared vertices.
Knowing the field $u$ possesses a tangential curl is not a strong enough condition to establish whether these components have matching traces into the vertices, or even directional derivatives along the edges.
Consideration of what it means for a field to be divergence-free with respect to $\dddmes$ (section \ref{sec:DivFreeCondition}) goes some of the way towards to remedying this --- providing $u_1$ and $u_2$ with some additional properties.

\section{The Divergence-Free Condition} \label{sec:DivFreeCondition}
As was mentioned in section \ref{sec:Intro-Maxwell}, the definition of the Maxwell operator requires us to pay respect to the divergence-free nature of the electric and magnetic fields; in particular the vector fields that make up the space on which the curl of the curl equation is posed on are required to be divergence-free.
This divergence-free condition adds extra constraints to the fields, and the analysis of section \ref{sec:CC-CurlAnalysis} indicates that we require an analogue of this divergence-free condition to constrain the behaviour of vector fields with $\dddmes$-tangential curls near the vertices.
Given the motivation for our consideration of the curl of the curl equation on a singular structure, it is natural for us to consider what the analogue of a divergence free field would be.
Using the classical notion of a divergence-free vector field as our starting point, we find ourselves presented with several reasonable definitions for the notion of a divergence free vector field with respect to our singular measures, which we will discuss in turn below.
However, we will see that a ``strong" definition through approximating sequences fails to provide a rather underwhelming\footnote{If not entirely unexpected.} description (see lemma \ref{lem:DivZero-Everything}), whilst the ``weaker" definitions \ref{def:DivFree-AllGradients} and \ref{def:DivFree-TangGradients} do provide some additional constraints on our vector fields, but section \ref{sec:3DSystemDerivation} will show that these constraints are also directly obtainable from \eqref{eq:SingularCurlEquation}.

Given how we have chosen to define gradients and curls with respect to our singular measures, it is natural to ask whether it would be possible to define the tangential divergence through an approximation via smooth functions in a similar manner.
A short investigation proves that this is not be the case.
If we follow our motivations for gradients and curls, for a (Borel) measure $\rho$ we could define
\begin{align*}
	W_{\rho,\mathrm{div}}^{\kt} = \overline{\clbracs{ \bracs{\phi,\ktgrad\cdot\phi} \setVert \phi\in\psmooth{\ddom}^3 }},
\end{align*}
with the closure taken in $\ltwo{\ddom}{\rho}^3\times\ltwo{\ddom}{\rho}$, and define the space of divergences of zero (with respect to $\rho$) as
\begin{align*}
	\divZero{\ddom}{\rho} &= \clbracs{d \setVert \bracs{0,d}\in W_{\rho,\mathrm{div}}^{\kt}}.
\end{align*}
Similar arguments to those in section \ref{sec:BorelMeasSobSpaces} establish that $\divZero{\ddom}{\rho}$ does not depend on the value of $\kt$, and is a closed, linear subspace of $\pltwo{\ddom}{\rho}$.
Hence we can decompose $\pltwo{\ddom}{\rho}$ as the direct sum of $\divZero{\ddom}{\rho}$ and its orthogonal complement.
In turn, we would find that a function has at most one ``tangential divergence", and thus could define the Sobolev space of functions possessing divergences as
\begin{align*}
	\ktdivSob{\ddom}{\rho}
	&= \clbracs{ \bracs{u,d}\in W_{\rho,\mathrm{div}}^{\kt} \setVert d\perp\mathcal{D}_{\ddom,\md\rho}\bracs{0} }.
\end{align*}
Then any vector field whose tangential divergence is zero, we would label as a divergence free field.
However once we consider the divergences of zero on a single edge of our singular structure, we will quickly run into an unpleasant consequence:
\begin{lemma} \label{lem:DivZero-Everything}
	For any edge $I_{jk}$, we have that $\divZero{\ddom}{\lambda_{jk}}=\pltwo{\ddom}{\lambda_{jk}}$.
\end{lemma}
\begin{proof}
	Let $\phi\in\psmooth{\ddom}$, and let $\varphi\in\csmooth{\ddom}$ be a smooth function with $\varphi(x)=\bracs{x-v_j}\cdot n_{jk}\phi(x)$ in some neighbourhood of $I_{jk}$.
	Then the smooth function $\varphi\widehat{n}_{jk}\in\csmooth{\ddom}^3$ is such that $\varphi=0$ and $\grad\cdot\varphi\widehat{n}_{jk}=\phi$ on $I_{jk}$, so $\phi\in\divZero{\ddom}{\lambda_{jk}}$.
	A density argument then completes the proof.
\end{proof}
Further to this, it is not difficult to prove an analogy of the extension results (proposition \ref{prop:3DGradZeroChar} and theorem \ref{thm:CurlZeroChar}) for these divergences of zero.
Neither is it difficult to show that $\divZero{\ddom}{\massMes}=\pltwo{\ddom}{\massMes}$, and consequentially that an analogue of theorem \ref{thm:3DdddmesCharGradZero} and proposition \ref{prop:ThickVertexCurlZeroCharacterisation} holds, so every tangential divergence is zero!
We should not be surprised at this result --- the divergence is a scalar representing the net flow of a vector field out of a given point $x$.
However our singular measure cannot see any such flux across the edges $I_{jk}$ (or out of the planes $P_{jk}$), thus any divergence induced by such a flux must be a divergence of zero.
However it is then impossible to know ``how much" of the scalar divergence would be due to these unobservable fluxes across $I_{jk}$, and how much was from the observable portions of the vector field along $I_{jk}$.

Naturally, one might find it dissatisfying that all vector fields are divergence free in lieu of the above computation, and instead look for other possible definitions or interpretations.
A characterising feature of (classical) divergence-free fields is that they are orthogonal to all potential vector fields (that is, any vector field equal to the gradient of a scalar function).
Utilising this analogue with the classical divergence, we could adopt the following definition for our analogue of divergence-free fields:
\begin{definition}[$\dddmes$-Divergence Free] \label{def:DivFree-AllGradients}
	A field $u\in\pltwo{\ddom}{\dddmes}^3$ is divergence free (with respect to $\dddmes$) if (and only if)
	\begin{align*}
		\ip{u}{g}_{\ltwo{\ddom}{\dddmes}^3} = \integral{\ddom}{ u\cdot\overline{g} }{\dddmes} = 0,
	\end{align*}
	for every $g\in W_{\mathrm{grad}}^{\kt}\bracs{\ddom,\md\dddmes}$.
\end{definition}
That is, a field is divergence free only if it is orthogonal in $\ltwo{\ddom}{\dddmes}$ to any gradient field, including the gradients of zero.
Whilst this ``weak" definition breaks the tradition of our approach via smooth approximations, it will nonetheless be valuable for us to examine the implications of definition \ref{def:DivFree-AllGradients} (and the later definition \ref{def:DivFree-TangGradients}) when we come to the analysis of \eqref{eq:SingularWaveEqnQGProblem}).
Definition \ref{def:DivFree-AllGradients} admits the following characterisation.
\begin{prop} \label{prop:DivFree-AllGradsConditions}
	Let $u\in\pltwo{\ddom}{\dddmes}^3$ and define (for each edge $I_{jk}$)
	\begin{align*}
		U^{(jk)} = R_{jk}\begin{pmatrix} u_1^{(jk)} \\ u_2^{(jk)} \end{pmatrix}, 
		\qquad \qm_{jk} = \qm\cdot e_{jk}, 
		\qquad \widetilde{U}^{(jk)} = U^{(jk)}\circ r_{jk}.
	\end{align*}		
	Then $u$ is divergence-free if and only if
	\begin{align*}
		\mathrm{(i)} \ & U_1^{(jk)} = 0, \quad \forall I_{jk}\in\edgeSet \\
		\mathrm{(ii)} \ & \widetilde{U}_2^{(jk)}\in\gradSob{\sqbracs{0,l_{jk}}}{y}, \quad \forall I_{jk}\in\edgeSet, \\
		\mathrm{(iii)} \ & \bracs{U_2^{(jk)}}' + \rmi\qm_{jk}U_2^{(jk)} + \rmi\wavenumber u_3^{(jk)} = 0 \text{ on } I_{jk}, \quad \forall I_{jk}\in\edgeSet, \\
		\mathrm{(iv)} \ & u_1\bracs{v_j} = u_2\bracs{v_j} = 0, \quad \forall v_j\in\vertSet, \\
		\mathrm{(v)} \ & \sum_{j\conRight k}U_2^{(kj)}\bracs{v_j} - \sum_{j\conLeft k}U_2^{(jk)}\bracs{v_j} = \rmi\wavenumber\alpha_j u_3\bracs{v_j}, \quad \forall v_j\in\vertSet,
	\end{align*}
	where $\bracs{U_2^{(jk)}}' = \bracs{\widetilde{U}_2^{(jk)}}'\circ r_{jk}^{-1}$.
\end{prop}
\begin{proof}
	First assume that the field $u$ is divergence-free (in the sense of definition \ref{def:DivFree-AllGradients}).
	\begin{enumerate}[(i)]
		\item We first utilise theorem \ref{thm:3DdddmesCharGradZero} to deduce that $\ip{u}{g}_{\ltwo{\ddom}{\lambda_{jk}}}=0$ for every $g\in\gradZero{\ddom}{\lambda_{jk}}$.
		Then since we know that $\gradZero{\ddom}{\lambda_{jk}}$ consists of all functions of the form $g_{jk}\hat{n}_{jk}$ (proposition \ref{prop:3DGradZeroChar}), we must conclude that
		\begin{align*}
			0 &= \integral{\ddom}{ \overline{g}_{jk}\begin{pmatrix} u_1^{(jk)} \\ u_2^{(jk)} \end{pmatrix}\cdot n_{jk} }{\lambda_{jk}}, \qquad \forall g_{jk}\in\pltwo{\ddom}{\lambda_{jk}},
		\end{align*}
		so $U_1^{(jk)}=0$ on $I_{jk}$.
		\item Let $\phi\in\csmooth{\sqbracs{0,l_{jk}}}$.
		Utilising suitable cut-off functions, we can then construct a function $\varphi\in\csmooth{\ddom}\cap\psmooth{\ddom}$ such that $\varphi(x)=\phi\bracs{r_{jk}^{-1}(x)}$ when $x\in I_{jk}$, and with $\varphi=0$ on all other parts of $\graph$.
		Since such $\varphi$ are smooth, they are elements of $\ktgradSob{\ddom}{\lambda_{jk}}$ and by lemma \ref{lem:ExtensionLemmaEdgeFunctions} also elements of $\ktgradSob{\ddom}{\dddmes}$.
		As such, we have that
		\begin{align*}
			0 &= \integral{\ddom}{ u\cdot\overline{\ktgrad_{\dddmes}\varphi} }{\dddmes}
			= \integral{I_{jk}}{ u\cdot\bracs{ \bracs{\overline{\varphi}' - \rmi\qm_{jk}\overline{\varphi}}\widehat{e}_{jk} - \rmi\wavenumber\overline{\varphi}\widehat{x}_3} }{\lambda_{jk}} \\
			&= \integral{I_{jk}}{ \bracs{\overline{\varphi}' - \rmi\qm_{jk}\overline{\varphi}}U_2^{(jk)} - \rmi\wavenumber\overline{\varphi}u_3^{(jk)} }{\lambda_{jk}} \\
			&= \int_0^{l_{jk}} \bracs{\overline{\phi}' - \rmi\qm_{jk}\overline{\phi}}\widetilde{U}_2^{(jk)} - \rmi\wavenumber\overline{\phi}\widetilde{u}_3^{(jk)} \ \md y, \\
			\implies
			\int_0^{l_{jk}} \overline{\phi}'\widetilde{U}_2^{(jk)} \ \md y
			&= \int_0^{l_{jk}} \overline{\phi}\bracs{ \rmi\qm_{jk}\widetilde{U}_2^{(jk)} + \rmi\wavenumber\widetilde{u}_3^{(jk)} } \ \md y.
		\end{align*}
		This holds for every $\phi\in\csmooth{\sqbracs{0,l_{jk}}}$, and so we can conclude that $\widetilde{U}_2^{(jk)}\in\gradSob{\sqbracs{0,l_{jk}}}{y}$.
		\item Following immediately on from the above, we have that
		\begin{align*}
			\bracs{U_2^{(jk)}}' = - \bracs{ \rmi\qm_{jk}U_2^{(jk)} + \rmi\wavenumber u_3^{(jk)} },
		\end{align*}
		which upon rearrangement gives the result.
		\item We again notice through theorem \ref{thm:3DdddmesCharGradZero} we have that $\ip{u}{g}_{\ltwo{\ddom}{\massMes}}=0$ for every $g\in\gradZero{\ddom}{\massMes}$.
		Fix $v_j\in\vertSet$, and take $g=\bracs{g_1,g_2,0}^\top$ at $v_j$ and zero elsewhere.
		Given the characterisation in proposition \ref{prop:NuGradZeroChar}, we can conclude that
		\begin{align*}
			0 &= \integral{\ddom}{ u_\cdot\overline{g} }{\massMes}
			= \alpha_j\bracs{\overline{g_1}u_1 + \overline{g_2}u_2},
		\end{align*}
		for all $g_1,g_2\in\complex$.
		Therefore, we must have that $u_1\bracs{v_j}=u_2\bracs{v_j}=0$.
		\item Finally, take $\varphi\in\csmooth{\ddom}$ with support
		\begin{align*}
			\supp\bracs{\varphi} \subset \mathcal{J}\bracs{v_j}\setminus\clbracs{v_k\in\vertSet \setVert v_k\neq v_j}.
		\end{align*}
		Since this $\varphi$ is smooth, it is an element of $\ktgradSob{\ddom}{\dddmes}$ by definition, and we have that
		\begin{align*}
		0 &= \integral{\ddom}{ u\cdot\overline{\ktgrad_{\dddmes}\varphi} }{\dddmes} \\
		&= \sum_{j\con k}\integral{I_{jk}}{ U_2^{(jk)}\bracs{\overline{\varphi}' - \rmi\qm_{jk}\overline{\varphi}} - \rmi\wavenumber u_3^{(jk)}\overline{\varphi} }{\lambda_{jk}} 
		+ \integral{\ddom}{ u\cdot\overline{\ktgrad_{\dddmes}\varphi} }{\massMes} \\
		&= \sum_{j\con k}\int_0^{l_{jk}} \widetilde{U}_2^{(jk)}\bracs{\overline{\widetilde{\varphi}}' - \rmi\qm_{jk}\overline{\widetilde{\varphi}}} - \rmi\wavenumber u_3^{(jk)}\overline{\widetilde{\varphi}} \ \md y
		-\rmi\wavenumber\alpha_j u_3\bracs{v_j}\overline{\varphi}\bracs{v_j}.
		\end{align*}
		Rearranging and using (ii) and (iii), we find that
		\begin{align*}
			\rmi\wavenumber\alpha_j u_3\bracs{v_j}\overline{\varphi}\bracs{v_j}
			&= - \sum_{j\con k}\int_0^{l_{jk}} \overline{\widetilde{\varphi}}\bracs{ \widetilde{U}_2^{(jk)} + \rmi\qm_{jk}\widetilde{U}_2^{(jk)} + \rmi\wavenumber\widetilde{u}_3^{(jk)} } \ \md y \\
			&\quad + \overline{\varphi}\bracs{v_j}\bracs{ \sum_{j\conRight k} U_2^{(kj)}\bracs{v_j} - \sum_{j\conLeft k} U_2^{(jk)}\bracs{v_j} } \\
			&= \overline{\varphi}\bracs{v_j}\bracs{ \sum_{j\conRight k} U_2^{(kj)}\bracs{v_j} - \sum_{j\conLeft k} U_2^{(jk)}\bracs{v_j} }.
		\end{align*}
		This holds for every such smooth $\varphi$, and so we must have that
		\begin{align*}
			\rmi\wavenumber\alpha_j u_3\bracs{v_j} &= \sum_{j\conRight k} U_2^{(kj)}\bracs{v_j} - \sum_{j\conLeft k} U_2^{(jk)}\bracs{v_j}.
		\end{align*}
	\end{enumerate}
	
	Now let us suppose that (i)-(v) hold.
	Conditions (i) and (iv) ensure that $u$ is orthogonal to $\gradZero{\ddom}{\dddmes}$ via theorem \ref{thm:3DdddmesCharGradZero}.
	Now let $v\in\ktgradSob{\ddom}{\dddmes}$, and consider an approximating sequence $\phi_n$ for $v$.
	Clearly,
	\begin{align*}
		\ip{u}{\ktgrad_{\dddmes}v}_{\ltwo{\ddom}{\dddmes}^3}
		&= \lim_{n\rightarrow\infty}\ip{u}{\ktgrad\phi_n}_{\ltwo{\ddom}{\dddmes}^3},
	\end{align*}
	from which we can use (ii), (iii), and (v) to deduce that $\ip{u}{\ktgrad\phi_n}_{\ltwo{\ddom}{\dddmes}^3}=0$ for every $n\in\naturals$, providing orthogonality to tangential gradients.
\end{proof}
We highlight that a field $u$ being divergence-free provides us with some additional information about the regularity of the components of $u$, however still falls short of establishing any kind of continuity of the in-plane components $u_1$ and $u_2$ near the vertices.
The condition (ii) provides us with regularity of (some linear combination of) these components along each edge, but we do not obtain any information about their traces, other than the condition (v) --- which does not imply the traces must match even when $u_3(v_j)=0$.

We can make some remarks about each of the conditions (i)-(v) in proposition \ref{prop:DivFree-AllGradsConditions} to draw parallels with the usual notion of divergence free.
The left hand side of condition (iii) bears close resemblance to the expression for the three dimensional divergence; recalling our use of a Fourier transform in the $\widehat{x}_3$ direction, the $\rmi\wavenumber u_3$ term is the result of the presence of a $\partial_3 u_3$ term in non-Fourier space.
Furthermore, we can also associate
\begin{align*}
	\bracs{U_2^{(jk)}}' + \rmi\qm_{jk}U_2^{(jk)} &= \bracs{\pdiff{}{e_{jk}} + \rmi\qm\cdot e_{jk}}\bracs{U^{(jk)}\cdot e_{jk}},
\end{align*}
which is the (shifted) derivative along the edge $I_{jk}$ (which exists thanks to (ii)).
Given that the vectors $\widehat{e}_{jk}$ and $\widehat{x}_3$ span the plane $P_{jk}$ induced by the edge $I_{jk}$, the left hand side of (iii) amounts to an ``in-plane" divergence (using the local coordinate frame $y_{jk}$ over the axial reference frame in the $\bracs{x_1,x_2}$-plane) which, for a divergence free field is required to be zero as one might expect.
It is not surprising that the condition (iii) does not involve any rates of change in the $\hat{n}_{jk}$ direction (``out of the plane" $P_{jk}$), since our singular measure cannot see such changes as they occur across the edges $I_{jk}$.
The presence of the conditions (i) and (iv) is slightly more perplexing.
Condition (i) informs us that our vector field $u$ is \emph{not} the result of changes across the edges $I_{jk}$, that our singular measure cannot observe.
We can take this as the only assurance $\lambda_{jk}$ can provide that our vector field $u$ is not inducing any flux or flow in the direction normal to $I_{jk}$ --- whilst we cannot see changes across the edges, we can at least observe the direction of the field \emph{on} the edge, and make the ``out-of-plane" component zero.
The condition (iv) fills the same role at the vertices --- recall that $\massMes$ cannot observe the function $u$ outside of the vertices, so $\massMes\times\lambda_1$ can only observe changes along the line induced by $v_j$ and thus the ``in-plane" components $u_1$ and $u_2$ must be zero at each vertex.
The condition (v) is then the result of the interaction of the incoming fields at the vertices\footnote{The condition (v) even bears resemblance to a $\delta'$-type vertex condition in a quantum graph problem --- the ``derivative" $\rmi\wavenumber u_3$ at $v_j$ must equal a multiple of the incoming edge functions.}.
To be divergence free at the vertex $v_j$; the sum of the ``flux" seen by the measure $\ddmes$, represented by the trace values of the $U_2^{(jk)}$ (the sign depending on whether the corresponding edge is directed into or out of the vertex $v_j$), and the ``flux" observed by $\massMes\times\lambda_1$ along the line induced by $v_j$, must balance.

This interpretation does raise the question as to whether we can define some kind of \emph{weak divergence} of a vector field with respect to our singular measures.
However we can note from proposition \ref{prop:DivFree-AllGradsConditions} that only conditions (iii) and (v) actually affect what such a ``weak divergence" would look like --- the requirement that a field be orthogonal to gradients of zero to be divergence-free directly affects the field itself.
Classically of course, there are no ``gradients of zero" and so there is no distinction between tangential gradients and gradients of zero.
Consequentially, divergence free functions being orthogonal to gradients is functionally the same as them being orthogonal to all tangential gradients.
In pursuit of a ``weak divergence", we first put forward a weaker definition of ``divergence-free", which for ease of language we will refer to as \emph{tangentially divergence-free}.
\begin{definition}[Tangentially $\dddmes$-Divergence Free] \label{def:DivFree-TangGradients}
	A field $u\in\pltwo{\ddom}{\dddmes}^3$ is tangentially divergence free with respect to $\dddmes$ if (and only if)
	\begin{align*}
		\ip{u}{\ktgrad_{\dddmes}v}_{\ltwo{\ddom}{\dddmes}^3} = \integral{\ddom}{ u\cdot\overline{\ktgrad_{\dddmes}v} }{\dddmes} = 0,
	\end{align*}
	for every $v\in\ktgradSob{\ddom}{\dddmes}$.
\end{definition}

As one might expect from the differences between the definitions \ref{def:DivFree-AllGradients} and \ref{def:DivFree-TangGradients}, a characterisation of tangentially divergence free can be obtained by recycling the proof of proposition \ref{prop:DivFree-AllGradsConditions}.
Indeed, this characterisation retains precisely the conditions (ii), (iii), and (v):
\begin{cory} \label{cory:DivFree-TangGradsConditions}
	Let $u\in\pltwo{\ddom}{\dddmes}^3$ and define (for each edge $I_{jk}$)
	\begin{align*}
		U^{(jk)} = R_{jk}\begin{pmatrix} u_1^{(jk)} \\ u_2^{(jk)} \end{pmatrix}, 
		\qquad \qm_{jk} = \qm\cdot e_{jk}, 
		\qquad \widetilde{U}^{(jk)} = U^{(jk)}\circ r_{jk}.
	\end{align*}		
	Then $u$ is tangentially divergence-free if and only if
	\begin{align*}
		\mathrm{(ii)} \ & \widetilde{U}_2^{(jk)}\in\gradSob{\sqbracs{0,l_{jk}}}{y}, \quad \forall I_{jk}\in\edgeSet, \\
		\mathrm{(iii)} \ & \bracs{U_2^{(jk)}}' + \rmi\qm_{jk}U_2^{(jk)} + \rmi\wavenumber u_3^{(jk)} = 0 \text{ on } I_{jk}, \quad \forall I_{jk}\in\edgeSet, \\
		\mathrm{(v)} \ & \sum_{j\conRight k}U_2^{(kj)}\bracs{v_j} - \sum_{j\conLeft k}U_2^{(jk)}\bracs{v_j} = \rmi\wavenumber\alpha_j u_3\bracs{v_j}, \quad \forall v_j\in\vertSet,
	\end{align*}
	where $\bracs{U_2^{(jk)}}' = \bracs{\widetilde{U}_2^{(jk)}}'\circ r_{jk}^{-1}$.
\end{cory}
We purposefully retain the same labels as proposition \ref{prop:DivFree-AllGradsConditions}.
Similarly to how tangential gradients and curls only encode information about changes of functions or fields in the plane, a field being tangentially divergence free only imposes balance between the changes (in their respective directions) of the in-plane components of a vector field $u$.

This brings us closer to the idea of defining a ``weak divergence" in a similar manner to that of gradients and curls in section \ref{sec:BorelMeasSobSpaces}.
If we bring each of the terms in (v) over to the left-hand side, we now have a function $F\in\pltwo{\ddom}{\dddmes}$ with
\begin{align} \label{eq:ktDivergence-Guess}
	F(x) &= 
	\begin{cases} 
		\bracs{U_2^{(jk)}}'(x) + \rmi\qm_{jk}U_2^{(jk)}(x) + \rmi\wavenumber u_3(x) &
		x\in I_{jk}\setminus\clbracs{v_j,v_k}, \\
		\sum_{j\conRight k}U_2^{(kj)}\bracs{v_j} - \sum_{j\conLeft k}U_2^{(jk)}\bracs{v_j} -\rmi\wavenumber\alpha_j u_3\bracs{v_j} &
		x=v_j\in\vertSet,
	\end{cases}
\end{align}
and that suggestively satisfies $u$ is tangentially divergence free if and only if $F=0$.
Indeed, we could elect to define the $\kt$-divergence with respect to $\dddmes$ of a vector field $u$ in the following manner:
\begin{definition}[$\kt$-Divergence (with respect to $\dddmes$)] \label{def:dddmesDivergence}
	Let $u\in\pltwo{\ddom}{\dddmes}^3$.
	If there exists a function $d\in\pltwo{\ddom}{\dddmes}$ such that
	\begin{align*}
		\integral{\ddom}{ u\cdot\overline{\ktgrad_{\dddmes}\phi} }{\dddmes}
		&= -\integral{\ddom}{ d\overline{\phi} }{\dddmes}, \qquad\forall\phi\in\psmooth{\ddom},
	\end{align*}
	we say that $u$ admits (or possesses) a weak $\kt$-divergence with respect to $\dddmes$.
	Then we call $d:=\ktgrad_{\dddmes}\cdot u$ the $\kt$-divergence with respect to $\dddmes$ of the vector field $u$.
	If $\ktgrad_{\dddmes}\cdot u=0$, then $u$ is weakly divergence free.
\end{definition}
Clearly we could produce a similar definition for any Borel measure $\rho$ on $\ddom$.
The notions of $\kt$-tangentially divergence free (definition \ref{def:DivFree-TangGradients}) and weakly $\kt$-divergence free as in definition \ref{def:dddmesDivergence} coincide.
Furthermore, the weak $\kt$-divergence of a field $u$ is unique --- if there are two such divergences, the difference integrated against all $\phi\in\psmooth{\ddom}$ must be zero, from which we can deduce the difference is zero almost everywhere.
And finally, we can conclude that the weak $\kt$-divergence has the same form as the function $F$ in \eqref{eq:ktDivergence-Guess}, up to the position of the constant $\alpha_j$.
\begin{cory}
	Let $u\in\pltwo{\ddom}{\dddmes}^3$, and define (for each edge $I_{jk}$)
	\begin{align*}
		U^{(jk)} = R_{jk}\begin{pmatrix} u_1^{(jk)} \\ u_2^{(jk)} \end{pmatrix}, 
		\qquad \qm_{jk} = \qm\cdot e_{jk}, 
		\qquad \widetilde{U}^{(jk)} = U^{(jk)}\circ r_{jk}.
	\end{align*}	
	Then
	\begin{enumerate}[(i)]
		\item If $u$ admits a $\kt$-divergence, then $\widetilde{U}_2^{(jk)}\in\ktgradSob{\sqbracs{0,l_{jk}}}{y}$ for each $I_{jk}$ and 
		\begin{align} \label{eq:dddmesDivergence-Form}
			\ktgrad_{\dddmes}\cdot u &= 
			\begin{cases} 
				\bracs{U_2^{(jk)}}' + \rmi\qm_{jk}U_2^{(jk)} + \rmi\wavenumber u_3 &
				x\in I_{jk}\setminus\clbracs{v_j,v_k}, \\
				-\recip{\alpha_j}\bracs{\sum_{j\conRight k}U_2^{(kj)}\bracs{v_j} - \sum_{j\conLeft k}U_2^{(jk)}\bracs{v_j}} + \rmi\wavenumber u_3\bracs{v_j} &
				x=v_j\in\vertSet,
			\end{cases}
		\end{align}
		where $\bracs{U_2^{(jk)}}' = \bracs{\widetilde{U}_2^{(jk)}}'\circ r_{jk}^{-1}$.
		\item If $\widetilde{U}_2^{(jk)}\in\ktgradSob{\sqbracs{0,l_{jk}}}{y}$ for each $I_{jk}$, then $u$ admits a $\kt$-divergence $\ktgrad_{\dddmes}\cdot u$ defined as in \eqref{eq:dddmesDivergence-Form}.
	\end{enumerate}
\end{cory}
The proof of this corollary follows the same arguments as those for the conditions (ii), (iii), and (v) in proposition \ref{prop:DivFree-AllGradsConditions}.

\tstk{now do the de Rham stuff. Also, the curl-curl equation automatically gives us the other, weak definitions of divergence free to play with, so de Rham is a kind of tie-breaker in this respect.}
\begin{theorem}[de Rham]
	The following statements hold.
	\begin{enumerate}[(a)]
		\item If $u\in\ktgradSob{D}{\rho}$ then $\ktgrad_{\rho}u\in\ktcurlSob{\ddom}{\rho}$ with $\ktcurl{\rho}\bracs{\ktgrad_{\rho}u}=0$.
		\item If $g\in\gradZero{D}{\rho}$ then $g\in\ktcurlSob{\ddom}{\rho}$ with $\ktcurl{\rho}g=0$.
		\item Suppose that $v\in\ktcurlSob{D}{\rho}$ with $\ktcurl{\rho}v=0$.
		Then there exist unique $u\in\ktgradSob{D}{\rho}$ and $g\in\gradZero{D}{\rho}$ such that $v = \ktgrad_{\rho}u + g$.
		\item If $v\in\ktcurlSob{D}{\rho}$ then $\ktcurl{\rho}v\in H^1_{\qm,\wavenumber,\mathrm{div}}\bracs{\ddom,\md\rho}$
	\end{enumerate}
\end{theorem}

%Sing Inclusions chapter begins
\chapter{Composite Medium w/ Singular Inclusions} \label{ch:SingInc}
\tstk{title name of chapter needs redoing once you have an idea for what to change it to! As does the title of the next section!!!}

% Chapter introduction
\section{Chapter Introduction} \label{sec:SingIncChapterIntro}
The focus of chapters \ref{ch:ScalarSystem} and \ref{ch:CurlCurl} was the postulation, analysis, and solution of variational problems on a solely singular structure, including analysis pertaining to how to understand the notion of a derivative (or curl) on such a structure.
In each of these problems, the ``remainder" of our period cell ($\ddom\setminus\graph$) was ignored by the singular measures we consider --- this is an observation that we exploit in the analysis of the various Sobolev spaces.
Such domains and problems are more akin to the contexts highlighted in section \ref{ssec:Intro-ThinStructures}, rather than the setting of photonic crystals.
There is the heuristic argument to be made that by choosing to ignore $\ddom\setminus\graph$, the variational problems reflect a physical situation where we expect there to be no field (or wave propagation) in this region.
In the context of electromagnetism, this would represent some (singular) dielectric material surrounded by conductors --- there would be no field in the conducting regions, and only ``along" the dielectric materials.
However PCs (and PCFs) do not consist of a (thin, periodic) dielectric encased in a conducting material, rather they are composed of a (thin, periodic) dielectric material surrounded by \emph{another} dielectric material (section \ref{sec:PhysMot}).
We also discovered (in chapter \ref{ch:CurlCurl}) that the ``singular" curl-of-the-curl problem always reduces to the acoustic approximation, and there are a number of problems that plague the postulation of a first-order system due to the lower-dimensionality of the singular structure itself.
These factors provide the motivation for the final problem we examine --- the acoustic approximation on a two-dimensional composite domain with one of the components being singular.

These structures represent the ``visual" zero-thickness limit of a domain with thin-structure inclusions, like those illustrated in figure \ref{fig:Diagram_ScalingDimensionless} as $\delta\rightarrow0$.
We have discussed the limited amount of literature concerning such problems in section \ref{ssec:Intro-DoubleLimits}, highlighting that none of these problems consider the possibility of additional geometric contrast between the vertex and edge regions of the thin-structure inclusions.
In this respect, our final problem is setups so as to mimic the zero-thickness limit of such domains where this geometric contrast is present.
Postulation of an appropriate problem requires us to examine a ``composite measure", and opens up the potential for interactions between three different length scales: the ``bulk regions", the singular structure, and the vertices.
Although we do not include any material contrast in our problem initially, we will later remark at how one can introduce a parameter representing this contrast.

Let us formalise the notation and terminology we will us to describe the problem we will be working on in this section.
As usual, we take $\graph$ to be the period graph of a periodic, embedded graph $\hat{\graph}$ in $\reals^2$ with unit cell $\ddom$.
The graph $\graph$ naturally breaks $\ddom$ into a collection of disjoint union of (open) polygonal regions (or subdomains); we label these $\ddom_i$ for $i\in\Lambda$ for some appropriate (finite) index set $\Lambda$, and we then have that $\ddom = \graph\cup\bigcup_{i\in\Lambda}\ddom_i$.
We will refer to the graph $\graph$ and its constituent edges $I_{jk}\in\edgeSet$ as the \emph{(singular) skeleton}, and refer to the $\ddom_i$ as the \emph{bulk (regions)} or \emph{dielectric regions}\footnote{We neglect to use the term ``inclusion" for either $\ddom_i$ or $\graph$, to sidestep a philosophical argument concerning which material is being ``included" in the other.}.
Additionally, recall that we denote by $\lambda_2$ the two-dimensional Lebesgue measure on $\ddom$ and write
\begin{align*}
	\ccompMes := \lambda_2 + \dddmes = \lambda_2 + \ddmes + \massMes,
\end{align*}
where $\ccompMes$ shall we referred to as the \emph{composite measure} on $\ddom$ with respect to the graph $\graph$.
Whenever we refer to $\ddom$ as a \emph{composite domain}, we refer to $\ddom$ equipped with the measure $\ccompMes$.
Additionally, we define $\lcompMes = \lambda_2 + \lambda_{jk}$ for each edge $I_{jk}$.

We shall observe that our singular skeleton provides effects that are distinct from those induced by having interface conditions at the common boundaries of the $\ddom_i$.
At such an interface, one has matching conditions (appropriate to the modelling context) between the solutions (to a suitable differential problem) approaching from either side of the interface.
The interface itself has no size or bulk, and there are no dynamics happening along these interfaces beyond the matching conditions imposed --- the behaviour of the solution is determined in the bulk, and then matched to the expected (or physically relevant) interface conditions.
In contrast, our skeleton is bestowed a notion of length by $\ddmes$, and thus has the potential to (and does) give rise to dynamics along the edges of $\graph$, which will be coupled to the dynamics in the composite regions $\ddom_i$\footnote{The measure $\massMes$ has an analogous effect between the skeletal edges and the vertices --- we have already seen this effect manifest in the Wentzell conditions obtained in the effective problem in chapter \ref{ch:ScalarSystem}.}.
The behaviour of a solution is thus no longer purely determined by the behaviour in the bulk regions, then by matching to the other regions via the interfaces which separate them.
In fact, we will see that it is even possible to reformulate a problem on the composite domain into a problem posed solely on the skeleton $\graph$, where the interplay between the solution in the bulk and on the edges is encoded in the non-locality of the resulting problem.

The focus of this chapter will be on the acoustic approximation
\begin{align} \label{eq:SI-WaveEqn}
	-\laplacian_{\ccompMes}^\qm u = \omega^2 u \qquad\text{in } \ddom,
\end{align}
now posed on our composite domain and respecting our singular skeleton\footnote{See section \ref{sec:SI-ProblemFormulation} for a precise definition of what is meant by this equation, although the meaning assigned is analogous to our previous approaches in chapters \ref{ch:ScalarSystem} and \ref{ch:CurlCurl}.}.
Our objectives are similar to those of previous chapters; we are interested in studying (and explicitly obtaining) the spectrum of \eqref{eq:SI-WaveEqn}, and what behaviours we can expect to see emerging as a result of the geometric contrast.
Once again, we will approach this objective by attempting to find alternative formulations for \eqref{eq:SI-WaveEqn}, which are easier to analyse and can be solved numerically.
In fact, we will discover that \eqref{eq:SI-WaveEqn} possesses several equivalent formulations, each of which can be the basis of a numerical scheme with benefits and hindrances relative to the other formulations.
The first such formulation we consider (section \ref{sec:SI-VarProbMethod}) comes directly from the variational problem for the operator that defines \eqref{eq:SI-WaveEqn} (see section \ref{sec:SI-ProblemFormulation}), and the second (section \ref{ssec:SI-FDMMethod}) comes from the corresponding ``strong form" that we can derive using analysis of the function space that $u$ lives in (section \ref{sec:CompSobSpaces}).
These formulations still require us to work with the unfamiliar gradients ($\tgrad_{\ccompMes}u$) or handle interplay between the solution in the bulk and on the skeleton, which brings us to the third formulation in section \ref{sec:SI-NonLocalQG} where we reformulate \eqref{eq:SI-WaveEqn} into a problem on the skeleton only.
Our investigation into each of these formulations will highlight several ``trade-offs" that are made as we move between the various formulations or numerical approaches to solving \eqref{eq:SI-WaveEqn}.
We will conclude with a discussion of further extensions to this work --- notably alternative numerical approaches, and analysis of more general problems from electromagnetism than the acoustic approximation.
%For example, moving towards a problem on the skeleton only allows us to avoid handling tangential gradients with respect to $\compMes$ (and other non-classical objects) and theoretically reduces the dimensionality of any numerical scheme we want to employ because the skeleton is 1D.
%On the other hand, moving onto the skeleton also results in the introduction of non-local effects into the equations on each $I_{jk}$, to compensate for the effect of the bulk regions, which complicates the solutions process.
%We will also establish a link between the first and second formulations by means of \tstk{motivated by the use of a Strauss dilations, extended spaces}, and speculate on the affect of introducing non-zero coupling constants at the vertices.

% Formulation of the problem, and transition to classical coupled PDEs
\section{Problem Formulation} \label{sec:SI-ProblemFormulation}
Let us begin defining the objects in \eqref{eq:SI-WaveEqn} accurately.
This requires us to first analyse the tangential gradients of functions that live in the space $\tgradSob{\ddom}{\ccompMes}$ in the same vein as did with $\ktgradSob{\ddom}{\dddmes}$ and $\ktcurlSob{\ddom}{\dddmes}$ before, and is the focus of section \ref{sec:CompSobSpaces}.
As we might expect from the previous chapters, we find that:
\begin{itemize}
	\item The tangential gradient $\tgrad_{\ccompMes}u$ of $u$ is such that
	\begin{align*}
		\tgrad_{\ccompMes}u = 
		\begin{cases} 
			\grad u + \rmi\qm u & x\in\ddom\setminus\graph, \\ 
			\tgrad_{\lambda_{jk}}u & x\in I_{jk}, \ \forall I_{jk}\in\edgeSet, \\
			0 & x\in\vertSet,			
		\end{cases}
	\end{align*}
	where $\grad u$ denotes the weak gradient of $u\in H^1\bracs{\ddom}\cong\gradSob{\ddom}{\lambda_2}$.
	This is to say, in the bulk regions the function $u$ and its tangential gradient coincide with the familiar notion of a weak derivative (with respect to the Lebesgue measure).
	\item The function $u$ lives in $\gradSob{\ddom_i}{\lambda_2}$ for each of the bulk regions, and the traces of $u$ from $\ddom_i$ onto the inclusions $I_{jk}$ coincide with the values of $u^{(jk)}$ on the inclusions.
	This is as close to a condition of ``continuity across the inclusions" as we can get.
	Additionally, the $u^{(jk)}$ are continuous at the vertices of $\graph$, as was the case for functions in $\ktgradSob{\ddom}{\dddmes}$.
\end{itemize}

As with the variational problems of chapters \ref{ch:ScalarSystem} and \ref{ch:CurlCurl}, we understand \eqref{eq:SI-WaveEqn} in the variational sense: the problem of finding $\omega^2>0$ and non-zero $u\in\tgradSob{\ddom}{\ccompMes}$ such that
\begin{align} \label{eq:SI-WeakWaveEqn}
	\integral{\ddom}{ \tgrad_{\ccompMes}u\cdot\overline{\tgrad_{\ccompMes}\phi} }{\ccompMes}
	&= \omega^2 \integral{\ddom}{ u\overline{\phi} }{\ccompMes}, \quad\forall\phi\in\psmooth{\ddom}.
\end{align}
However as before we highlight that we could consider (for a fixed $\qm$) the bilinear map $b_{\qm}$ defined on pairs $(u,v)\in\tgradSob{\ddom}{\ccompMes}\times\tgradSob{\ddom}{\ccompMes}$ where
\begin{align*}
	b_{\qm}(u,v) &= \integral{\ddom}{ \tgrad_{\ccompMes}u\cdot\overline{\tgrad_{\ccompMes}v} }{\compMes}
	= \ip{\tgrad_{\ccompMes}u}{\tgrad_{\ccompMes}v}_{\ltwo{\ddom}{\ccompMes}^2},
\end{align*}
and use $b_{\qm}$ to define the self-adjoint operator $-\laplacian_{\ccompMes}^\qm$ by
\begin{align*} 
	\dom\bracs{ -\laplacian_{\ccompMes}^\qm } &= \clbracs{ u\in\tgradSob{\ddom}{\ccompMes} \setVert \exists f\in\ltwo{\ddom}{\ccompMes} \text{ s.t. } \right.
	\\
	& \qquad
	\left. \integral{\ddom}{ \tgrad_{\ccompMes}u\cdot\overline{\tgrad_{\ccompMes}v} }{\ccompMes} = \integral{\ddom}{ f\overline{v}}{\ccompMes}, \quad \forall v\in\tgradSob{\ddom}{\ccompMes} }, \labelthis\label{eq:CompLaplaceOpDom}
\end{align*}
with action $-\laplacian_{\ccompMes}^\qm = f$, where $u$ and $f$ are related as in \eqref{eq:CompLaplaceOpDom}.
Equation \eqref{eq:SI-WaveEqn} is then the eigenvalue equation for the operator $-\laplacian_{\ccompMes}^\qm$.
With $\ddom$ being bounded and $-\laplacian_{\ccompMes}^\qm$ self-adjoint, the spectrum of each $-\laplacian_{\ccompMes}^\qm$ consists of a discrete set of values $\omega^2\in\reals$.
We can even utilise the min-max principle to write down a variational formulation whose solution determines the eigenvalues (and eigenfunctions) of $-\laplacian_{\ccompMes}^\qm$, which will form the basis of our first approach to numerically solving this problem.
Through our use of the Gelfand transform, taking the union of the spectra over $\qm$ will provide us with the spectrum of a periodic operator on $\reals^2$ with period cell $\ddom$.

Despite their useful analytic properties (as operators), \eqref{eq:CompLaplaceOpDom} and \eqref{eq:SI-WeakWaveEqn} do not lend themselves particularly well to explicit analytic solution, nor provide any insight into how to handle objects like $\tgrad_{\ccompMes}u$ numerically.
Indeed, the tangential gradients and the integrals in \eqref{eq:SI-WeakWaveEqn} with respect to $\compMes$ are unfamiliar both from an analytic and numerical standpoint --- we know some of their properties when restricted to different regions of $\ddom$, but not how to work with them directly to obtain a solution (or approximation thereof) to \eqref{eq:SI-WeakWaveEqn}.
The complications this gives rise to will motivate us to continue our search of an alternative (and more informative) realisation of \eqref{eq:SI-WeakWaveEqn}, leading to the ``strong formulation" obtained in section \ref{sec:SI-StrongDerivation}.
We will take this idea further in section \ref{sec:SI-NonLocalQG} when we attempt to reformulate our problem on the skeleton, and discard the bulk regions.

Our investigation into the acoustic approximation on composite domains will lean on the cross-in-the-plane geometry from the example in section \ref{ssec:ExampleCrossInPlane} (now equipped with $\ccompMes$) to ground our discussion of each of our numerical approaches, and illustrate their implementation.
For convenience, we have translated the period graph by $\bracs{-\recip{2},-\recip{2}}$, which allows us to avoid carrying constant terms around.
The period cell of this geometry now consists of a skeleton $\graph$ within $\ddom=\left[0,1\right)^2$, with a single vertex $v_0=\bracs{0,0}^\top$ with coupling constant $\alpha_0$, and two ``looping" edges $I_h = \sqbracs{0,1}\times\clbracs{0}$, $I_v=\clbracs{0}\times\sqbracs{0,1}$.
The quasi-momentum parameters $\qm_{jk}$ are easily computable as $\qm_h = \qm_1$ and $\qm_v = \qm_2$, and we only have a single bulk region, $\ddom_1 = \ddom^{\circ}=\bracs{0,1}^2$.
There are a number of properties that we expect of our eigenvalues due to the symmetric nature of our geometry, notably the following:
\begin{prop}[Cross in the plane symmetries] \label{prop:CrossInPlaneSymmetries}
	If $\omega^2, u\bracs{x_1,x_2}$ is a solution to \eqref{eq:SI-WaveEqn} at $\qm = \bracs{\qm_1,\qm_2}\in\left[-\pi,\pi\right)^2$, then:
	\begin{itemize}
		\item $\omega^2, u\bracs{1-x_1,x_2}$ is a solution to \eqref{eq:SI-WaveEqn} at $\qm = \bracs{-\qm_1, \qm_2}$.
		\item $\omega^2, u\bracs{x_1,1-x_2}$ is a solution to \eqref{eq:SI-WaveEqn} at $\qm = \bracs{\qm_1, -\qm_2}$.
		\item $\omega^2, u\bracs{x_2,x_1}$ is a solution to \eqref{eq:SI-WaveEqn} at $\qm = \bracs{\qm_2, \qm_1}$.
	\end{itemize}
\end{prop}

Additionally for the cross-in-the-plane geometry, we can obtain analytic expressions for a subset of the eigenfunctions and eigenvalues --- namely those inherited from the Dirichlet Laplacian on $\bracs{0,1}^2$.
Observe that for $n,m\in\naturals$, the function
\begin{align*}
	u_{n,m}(x) &= \e^{-\rmi\qm\cdot x}\sin\bracs{n\pi x_1}\sin\bracs{m\pi x_2},
\end{align*}
is an eigenfunction of the Dirichlet Laplacian $-\laplacian^{\qm}_0$, solving the system
\begin{align*}
	-\laplacian^{\qm}_0 u 
	:= \bracs{\partial_1+\rmi\qm_1}^2 u + \bracs{\partial_2+\rmi\qm_2}^2 u
	&= \omega_{n,m}^2 u, \qquad \text{on } \bracs{0,1}^2, \\
	u\bracs{0,x_2} = u\bracs{1,x_2} = u\bracs{x_1,0} = u\bracs{x_1,1} &= 0,
\end{align*}
where $\omega_{n,m}^2 = \bracs{n^2+m^2}\pi^2$.
It is clear that $u_{n,m}\in\tgradSob{\ddom}{\ccompMes}$ and has $u=0$ on the skeleton.
In the event that both
\begin{itemize}
	\item ($n$ is even and $\qm_1=0$) or ($n$ is odd and $\qm_1=-\pi$),
	\item ($m$ is even and $\qm_2=0$) or ($m$ is odd and $\qm_2=-\pi$),
\end{itemize}
hold, then $u_{n,m}$ solves \eqref{eq:SI-WaveEqn} with $\omega^2 = \omega_{n,m}^2$ --- the phase-factor $\e^{-\rmi\qm\cdot x}$ is required to ensure that certain terms in the variational formulation cancel out\footnote{Specifically, the normal derivative traces onto the skeleton (see \eqref{eq:SI-InclusionEqn}) only cancel under these assumptions.}.
This provides us with some suitable test cases for our numerical schemes.
It will be clear (upon viewing figure \ref{fig:CompositeCross-VP-SpectralBands}) that these eigenvalues of the Dirichlet Laplacian are also not the \emph{only} eigenvalues of the problem \eqref{eq:SI-WaveEqn} --- the two spectra are distinct and neither is contained in the other.
Knowing that \eqref{eq:SI-WaveEqn} shares eigenvalues with the Dirichlet Laplacian will also have consequences for us in section \ref{sec:SI-NonLocalQG}, when we attempt to reduce \eqref{eq:SI-WaveEqn} to a quantum graph problem.

% Ideas for numerical solution of the problem
\section{Numerical Approach to the Variational Problem and Strong Formulation} \label{sec:SI-VPandFDM}
\tstk{fix this section's name once you decide what makes the cut. Also make a proper introduction here, or at least a lead-in.}

In this section we turn to numerical schemes for handling the variational problem \eqref{eq:SI-VarProb} and the strong formulation \eqref{eq:SI-BulkEqn}-\eqref{eq:SI-VertexCondition}.
For each formulation, we will outline the numerical scheme we have chosen to pursue and, importantly, the methods in which we can handle the non-standard integrals and gradients with respect to $\compMes$ numerically.
Where appropriate we will use the "cross-in-the-plane" geometry from the example in section \tstk{ref!!}, now equipped with $\compMes$, to illustrate the results of these methods.

\subsection{Variational Problem} \label{ssec:SI-VP}
We begin by examining the variational problem \eqref{eq:SI-VarProb},
\begin{align*} 
	\omega_n^2 &:= \min_{u}\clbracs{ \integral{\ddom}{ \abs{\tgrad_{\compMes}u}^2 }{\compMes} \setVert \norm{u}_{\ltwo{\ddom}{\compMes}}=1, \ u\perp u_l \ \forall 1\leq l\leq n-1 }. \tag{\eqref{eq:SI-VarProb} restated}
\end{align*}
Our interest is in determining the eigenvalues $\omega_n^2$, however we also need to determine the eigenfunctions $u_n$ since we need $u_n$ to be orthogonal to each of $u_l, 1\leq l\leq n-1$.
Given that we can obtain the eigenvalue $\omega_n^2$ from the eigenfunction $u_n$ by evaluating the integral in \eqref{eq:SI-VarProb}, we will focus our discussion on the approximation (and computation) of the eigenfunctions.
We also drop the explicit subscript $n$, and just consider the problem of determining the function $u\in\tgradSob{\ddom}{\compMes}$ which solves the optimisation problem
\begin{subequations} \label{eq:SI-MinProblem}
	\begin{align}
		\text{Minimise} \quad & \quad \integral{\ddom}{ \abs{\tgrad_{\compMes}u}^2 }{\compMes} \\
		\text{Subject to} \quad & \quad \integral{\ddom}{ \abs{u}^2 }{\compMes} = 1, \\
		& \quad \integral{\ddom}{ u\cdot\overline{u}_l }{\compMes} = 0, \ 1\leq l\leq n-1,
	\end{align}
\end{subequations}
where $n\in\naturals$, $u_l, 1\leq l\leq n-1$ are given (pairwise) orthogonal functions.
The traditional idea when attempting to approximate a minimising function is to represent the minimising function $u$ in a basis $\clbracs{\varphi_m}_{m\in\naturals_0}\subset\tgradSob{\ddom}{\compMes}$, truncate the basis expansion at some index $M$,
\begin{align} \label{eq:SI-VPTruncatedBasis}
	u &\approx \sum_{m=0}^M u_m \varphi_m,
\end{align}
and solve the minimisation problem (that arises from substituting \eqref{eq:SI-VPTruncatedBasis} into \eqref{eq:SI-MinProblem}) in the coefficients of the basis expansion that remain --- the choice of $M$ determines the accuracy in the approximate eigenfunction (and hence eigenvalue).
This minimisation problem will be discrete (solving for the $M+1$ independent variables $u_m$), and can be handled using optimisation methods.

None of the steps above are prohibited for the problem \eqref{eq:SI-MinProblem}, and so we can proceed with the aforementioned ideas.
We will illustrate this implementation for the "cross-in-the-plane" geometry first introduced in \tstk{example reference}\footnote{For convenience, we have translated the period cell by $\bracs{\recip{2},\recip{2}}$ with regards to how this geometry was handled in \tstk{ref}. This just allows us to avoid carrying additional constant terms around in our computations, and we would obtain the same results as if we didn't apply any translation.}; we take $\ddom=\left[0,1\right)^2$, and let $\graph$ be the period graph with a single vertex $v_0=\bracs{0,0}^\top$, and two ``looping" edges $I_h = \sqbracs{0,1}\times\clbracs{0}$, $I_v=\clbracs{0}\times\sqbracs{0,1}$.
Our first task is to decide on the basis functions $\varphi_m$ that we want to use to approximate $u$.
From the standpoint of accuracy (and typically speed) of the numerical solution there are several properties that it is desirable for this basis to have; orthonormality between the $\varphi_m$, similar shape to that expected of $u$, and of course periodicity.
This is where problems concerning the unfamiliar nature of our space $\tgradSob{\ddom}{\compMes}$ begin to arise, as the choice of basis is considerably more complex --- as choosing the behaviour of $\varphi_m$ in the bulk regions then restricts what $\varphi_m$ can do on the skeleton, and vice-versa. 
The geometry of the skeleton can also compound this issue, particularly if there are a large number of bulk regions $\ddom_i$, if they have irregular shapes, or if their shapes are significantly different (in terms of size or shape) from each other.
In general, one can choose a basis in similar fashion to how this is done for finite element schemes; mesh $\ddom$ into a union of simplexes (usually triangles) by placing nodes $\tau_i$, ensuring that none of the simplexes straddle any parts of the skeleton (that is, the interior of a simplex never has non empty intersection with part of the skeleton).
Then, use ``tent" or ``hat" functions centred on each node $\tau_i$ for the truncated basis functions $\varphi_m$.
This allows sufficient flexibility in the behaviour of $u$ on the edges and in the bulk regions, at the expense of requiring a new mesh for every new graph geometry.

Fortunately, the geometry of our ``cross-in-the-plane" example is rather simple, since the two edges $I_h, I_v$ are aligned with the coordinate axes.
This makes computing integrals on the skeleton much simpler, and traces from the bulk region can be computed with relative ease, so we can avoid taking the approach of meshing $\ddom$ as described above.
Instead, we can opt to choose a basis in a way more akin to spectral methods --- by choosing ``global" basis functions rather than the ``local" tent-basis functions that meshing $\ddom$ would utilise.
Combined with the fact that we only have one bulk region that spans the entire period cell, the natural candidate for our basis functions would be the 2D Fourier basis $\e^{2\pi\rmi(\alpha x + \beta y}$.
These functions are orthogonal in $\ltwo{\ddom}{\compMes}$, have period cell $\ddom$, and on each of the edges of $\graph$ reduce to a 1D-Fourier series.
However, these functions also possess a continuous (in the sense of matching traces) normal derivative across the skeleton, which functions in $\tgradSob{\ddom}{\compMes}$ are not required to have, and so we cannot use the Fourier basis.
Instead, we will look to use 2D polynomials to approximate our function $u$, by taking $M\in\naturals$ and setting
\begin{align} \label{eq:2DPolyBasisDef}
	\varphi_m(x,y) &= x^{i_m} y^{j_m}, \quad m = j_m + Mi_m, \ i,j\in\clbracs{0,...,M-1}.
\end{align}
These functions are not periodic by definition, so we are required to add the additional constraints
\begin{align*}
	u\bracs{0,y} = u\bracs{1,y}, \ \forall y\in\sqbracs{0,1}, 
	\qquad 
	u\bracs{x,0} = u\bracs{x,1}, \ \forall x\in\sqbracs{0,1},
\end{align*}
to our minimisation problem to account for this.
With this choice of basis, and writing $U = \bracs{u_0,...,u_{M^2-1}}^\top$, we are tasked with solving the following problem.
\begin{problem}[Discrete Variational Problem] \label{prob:DiscVarProb}
	Let $M,n\in\naturals$ and $\varphi_m$ be as in \eqref{eq:2DPolyBasisDef}.
	Given coefficients $U_l=\bracs{u^l_0,...,u^l_{M^2-1}}^\top$ for $1\leq l\neq n-1$, find values $U=\bracs{u_0,...,u_{M^2-1}}^\top$ that:
	\begin{subequations} \label{eq:SI-ExampleMinProb}
		\begin{align}
			\text{Minimise} \quad & \quad J\sqbracs{U} := \sum_{m=0}^{M^2-1}\sum_{n=0}^{M^2-1}u_m\overline{u}_n\ip{\tgrad_{\compMes}\varphi_m}{\tgrad_{\compMes}\varphi_n}_{\ltwo{\ddom}{\compMes}^2} 
			\label{eq:SI-EMPObjectiveFn} \\
			\text{Subject to} \quad & \quad \sum_{m=0}^{M^2-1}\sum_{n=0}^{M^2-1}u_m\overline{u}_n\ip{\varphi_m}{\varphi_n}_{\ltwo{\ddom}{\compMes}} = 1, 
			\label{eq:SI-EMPNormConstraint} \\
			& \quad \sum_{i_m=1}^{M-1}u_{j_m+Mi_m} = 0, \ \forall j_m\in\clbracs{0,...,M-1}, 
			\label{eq:SI-EMPxPeriodicity} \\
			& \quad \sum_{j_m=1}^{M-1}u_{j_m+Mi_m} = 0, \ \forall i_m\in\clbracs{0,...,M-1},
			\label{eq:SI-EMPyPeriodicity} \\
			& \quad \sum_{m=0}^{M^2-1}\sum_{n=0}^{M^2-1}u_m\overline{u}^l_n\ip{\varphi_m}{\varphi_n}_{\ltwo{\ddom}{\compMes}} = 0, \ \forall 1\leq l\leq n-1.
			\label{eq:SI-EMPOrthogonality}
		\end{align}
	\end{subequations}
\end{problem}
The minimiser $U$ of problem \ref{prob:DiscVarProb} then provides our approximation of $u$, and we have that $\omega^2 \approx J[U]$.
Equation \eqref{eq:SI-EMPNormConstraint} is the norm constraint on the eigenfunction $u$, \eqref{eq:SI-EMPxPeriodicity} (respectively \eqref{eq:SI-EMPyPeriodicity}) are the constraints that ensure periodicity of the eigenfunction in the $x$ ($y$) directions, and \eqref{eq:SI-EMPOrthogonality} forces $u$ to be orthogonal to the previously computed eigenfunctions $u_l$.
Due to our truncation, we are only ever able to compute approximations to the lowest $M^2 - \bracs{2M + 1} + 1$ eigenvalues due to the number of constraints in the problem \ref{prob:DiscVarProb}.

\tstk{time to display some nice figures, maybe some comparisons between the two methods etc? Also do a run of this with $\alpha_3\neq0$ just to see what on earth happens!}

\subsection{Finite Difference Scheme} \label{ssec:FDMSingInc}
\tstk{This is the content summary of \texttt{CompositeMedium\_PeriodicFDM.ipynb}}
As an alternative to working directly from the variational problem \ref{prob:DiscVarProb}, we can instead choose to work from our ``strong formulation" \eqref{eq:SI-BulkEqn}-\eqref{eq:SI-VertexCondition}.
Before we do so, it is convenient to notice that we can move the ``trace" terms in \eqref{eq:SI-InclusionEqn} to the left-hand-side to obtain the slightly nicer looking system
\begin{align*}
	-\laplacian_\qm u 
	&= \omega^2 u 
	&\text{in } \ddom_i, \\
	- \bracs{\diff{}{y} + \rmi\qm_{jk}}^2u^{(jk)}  - \bracs{\bracs{\grad u\cdot n_{jk}}^+ - \bracs{\grad u\cdot n_{jk}}^-}
	&= \omega^2 u^{(jk)},
	&\text{in } I_{jk}, \\
	\sum_l \bracs{\pdiff{}{n}+\rmi\qm_{jk_l}} u^{(jk_l)}(v_j) 
	&= 0 
	&\text{at } v_j\in\vertSet.
\end{align*}
We have now placed all differential operators on the left hand side of the equations, and our goal now is to devise a numerical scheme to approximate the action of the ``operators" on the left-hand-side of the above equations, and from that determine the spectrum of the problem.
Since all the objects in the formulation above are classical, we can look to employ a n{\"i}ave finite-difference based numerical scheme to approximate the spectrum of our problem.

\tstk{rewrite from here: it's going to be much easier to first discuss problems with meshing in general. IE placement of nodes, the need to adhere to the coordinate axis in the bulk but the local edge coordinate system on the edges, the need to place a node at every vertex and enough nodes along each edge, and that we can't cross edges when in a bulk region, so have to guarantee we have nearest neighbours contained in each $\overline{\ddom}_i$ whenever we're in a bulk region. All of this makes it hard to come up with a mesh in the first place, and you won't have a constant mesh-width throughout. This also means that centred differences are off the table since we don't have the same inter-nodal difference. But, having accounted for these things, you can still draw up a scheme (and we should translate the more general equations below into such a framework, IE nodes $\tau_i$ and use something like $\tau_i^{++}$ etc for it's neighbour ``up and to the left".}
We first discretise $\ddom$ into a \emph{suitable} uniform mesh consisting of $N\times N$ nodes $\tau_{p,q} = \bracs{(p-1)h,(q-1)h}$ for $p,q\in\clbracs{1,...,N}$, with a mesh width of $h = \recip{N-1}$, and write $u_{p,q} = u\bracs{\tau_{p,q}}$.
Conditions for a ``suitable" mesh will be highlighted as we proceed with the description, and discussed afterwards.
Also note that there is no need to place nodes along both of the periodic edges of $\ddom$, so long as we keep track of which nodes are connected by periodicity, but for notational purposes it is convenient to include such nodes in our description.
Proceeding with the discretisation of the above equations, at points $\tau_{p,q}\in\ddom_i$ in one of the bulk regions, we discretise as
\begin{align*}
	-\laplacian_\qm u\bracs{\tau_{p,q}} &\approx 
	\bracs{\abs{\qm}^2 + 4h^{-2}}u_{p,q}
	-h^{-1}\bracs{h^{-1} + \rmi\qm_1}u_{p+1,q}
	-h^{-1}\bracs{h^{-1} - \rmi\qm_1}u_{p-1,q} \\
	&\qquad -h^{-1}\bracs{h^{-1} + \rmi\qm_2}u_{p,q+1}
	-h^{-1}\bracs{h^{-1} - \rmi\qm_2}u_{p,q-1}, \labelthis\label{eq:SI-FDMBulkDiscretise}
\end{align*}
using centred differences --- of course, using forward (also known as left) or backward (a.k.a right) differences would also be a valid approach.

For nodes $\tau_{p,q}\in I_{jk}$, our finite difference approximations become slightly more complex, as we are forced to consider the nearest nodes to $\tau_{p,q}$ that lie in the directions $e_{jk}$ and $n_{jk}$.
To condense the notation we let;
\begin{itemize}
	\item $\tau_{p,q}^{\pm e_{jk}}$ denote the nearest node to $\tau_{p,q}$ in the direction $\pm e_{jk}$ from $\tau_{p,q}$, and set $h_{p,q}^{\pm e_{jk}} = \abs{ \tau_{p,q} - \tau_{p,q}^{\pm e_{jk}} }$.
	\item $\tau_{p,q}^{\pm n_{jk}}$ denote the nearest node to $\tau_{p,q}$ in the direction $\pm n_{jk}$ from $\tau_{p,q}$, and set $h_{p,q}^{\pm n_{jk}} = \abs{ \tau_{p,q} - \tau_{p,q}^{\pm n_{jk}} }$.
	\item $u_{p,q}^{\pm e_{jk}} = u\bracs{\tau_{p,q}^{\pm e_{jk}}}$ and $u_{p,q}^{\pm n_{jk}} = u\bracs{\tau_{p,q}^{\pm n_{jk}}}$.
\end{itemize}
Note that to be able to perform the above we must require that each of the nodes $\tau_{p,q}^{\pm e_{jk}}$, $\tau_{p,q}^{\pm n_{jk}}$ exist --- an important consideration for a suitable mesh.
Additionally, let us assume that $h_{p,q}^{e_{jk}} := h_{p,q}^{+e_{jk}}=h_{p,q}^{-e_{jk}}$ so that we can use central differences in the direction along $I_{jk}$ --- this is not a necessary property the mesh must have, as we could simply elect to use either forward or backward differences if $h_{p,q}^{+e_{jk}} \neq h_{p,q}^{-e_{jk}}$.
Then we have that 
\begin{align*}
	&- \bracs{\diff{}{y} + \rmi\qm_{jk}}^2u^{(jk)}\bracs{\tau_{p,q}} - \bracs{\bracs{\grad u\bracs{\tau_{p,q}}\cdot n_{jk}}^+ - \bracs{\grad u\bracs{\tau_{p,q}}\cdot n_{jk}}^-} \\
	&\qquad \approx \bracs{\qm_{jk}^2 + 2\bracs{h_{p,q}^{e_{jk}}}^{-1} + 2\bracs{h_{p,q}^{e_{jk}}}^{-2}}u_{p,q} \\
	&\qquad - \bracs{h_{p,q}^{e_{jk}}}^{-1}\bracs{ \bracs{h_{p,q}^{e_{jK}}}^{-1} + \rmi\qm_{jk} }u_{p,q}^{+e_{jk}}
	- \bracs{h_{p,q}^{e_{jk}}}^{-1}\bracs{ \bracs{h_{p,q}^{e_{jk}}}^{-1} - \rmi\qm_{jk} }u_{p,q}^{-e_{jk}} \\
	&\qquad - \bracs{h_{p,q}^{+n_{jk}}}^{-1}u_{p,q}^{+n_{jk}}
	- \bracs{h_{p,q}^{-n_{jk}}}^{-1}u_{p,q}^{-n_{jk}}, \labelthis\label{eq:SI-GeneralEdgeDiscretise}
\end{align*}
which serves as our approximation at $\tau_{p,q}$ --- we have taken ``centred" differences along the edge $I_{jk}$ and one-sided derivatives from the adjacent regions to approximate the traces of the normal derivatives.

For those $\tau_{p,q}$ that are placed at the vertices $v_j\in\vertSet$, our finite difference approximation already assumes continuity of $u$ at these nodes, so instead we must enforce the vertex conditions here.
To this end, we find that
\begin{subequations} \label{eq:SI-FDMGeneralVertex}
	\begin{align}
		\bracs{ \pdiff{}{n} + \rmi\qm_{jk} } u^{(jk)}(v_j)
		&= h_{p,q}^{+e_{jk}}\bracs{ u_{p,q} - u_{p,q}^{+e_{jk}} } - \rmi\qm_{jk}u_{p,q}, \\
		\bracs{ \pdiff{}{n} + \rmi\qm_{jk} } u^{(kj)}(v_j)
		&= h_{p,q}^{-e_{kj}}\bracs{ u_{p,q} - u_{p,q}^{-e_{kj}} } + \rmi\qm_{kj}u_{p,q},	
	\end{align}
\end{subequations}
which will allow us to compute the approximation to \eqref{eq:SI-VertexCondition}.
These approximations then allow us to form a finite difference matrix $\mathcal{F}$\footnote{One final consideration we then have to make is to ensure that we adhere to periodic boundary conditions for any nodes that lie on $\partial\ddom$ when constructing $\mathcal{F}$.}, which acts on the vector $U$ with $U_{i} = u_{p,q}$ where $i = q + Np$.
Then, we solve the (generalised) eigenvalue problem
\begin{align} \label{eq:FDM-MatrixSystem}
	\mathcal{F}U = \beta\bracs{\omega^2}U,
\end{align}
for $\omega^2>0, U\in\complex^{N^2}$, where 
\begin{align*}
	\bracs{\beta\bracs{\omega^2}}_{nm} &= 
	\begin{cases}
 		\omega^2 & n=m, n=q + Np, \text{ and } \tau_{p,q}\not\in\vertSet, \\
 		0 & \text{otherwise. }
	\end{cases}
\end{align*}
The matrix-valued function $\beta$ ensures that the vertex condition is satisfied when solving \eqref{eq:FDM-MatrixSystem}, which can be done with a suitable generalised eigenvalue solver (note that $\beta\bracs{\omega^2}$ is easily computable and positive semi-definite, being a diagonal matrix with non-negative entries).

We must place (enough) nodes on each of the edges $I_{jk}$ to ensure that \eqref{eq:SI-InclusionEqn} is discretised correctly and reflected in the finite difference matrix $\mathcal{F}$.
Thus, the major requirement that we make of our mesh is that the nodes $\tau_{p,q}^{\pm e_{jk}}$ and $\tau_{p,q}^{\pm n_{jk}}$ exist whenever we have $\tau_{p,q}\in I_{jk}$.
Of course, this is not ideal if the edges of $\graph$ are not aligned with the coordinate axes, as using a uniform mesh no longer guarantees that such nodes will exist (nor does it, in general, ensure that at least one node is placed on every edge 
Even if such nodes exist, there is no guarantee that they are ``near" the original node as figure \ref{fig:Diagram_CompMesMeshNearestNeighbours} illustrates.
\begin{figure}
	\centering
	\includegraphics[scale=1.0]{Diagram_CompMesMeshNearestNeighbours.pdf}
	\caption{\label{fig:Diagram_CompMesMeshNearestNeighbours} With a uniform mesh and general skeleton $\graph$, there is no guarantee that the ``nearest neighbours" of a node are close by, which in turn leads to a poor approximation of the gradient at such nodes. Indeed, there isn't even any guarantee that each $I_{jk}$ will have at least one node placed on it.}
\end{figure}
Indeed, it is very possible that the nodes $\tau_{p,q}^{\pm e_{jk}}$ and $\tau_{p,q}^{\pm n_{jk}}$ are significantly further away from $\tau_{p,q}$ than other nodes in the mesh.
However, the requirement that we use a uniform mesh is not a necessity, we can use a non-uniform mesh at the expense of abandoning centred differences in the bulk regions and along the edges $I_{jk}$.
If taking this option, nodes should be placed along the edges $I_{jk}$ first (including placing nodes at each of the vertices), and then placed non-uniformly in the bulk regions.
The equation \eqref{eq:SI-FDMBulkDiscretise} invariably changes (since $\laplacian_{\qm}u_{p,q}$ is no longer discretised with centred differences), but ultimately one can still assemble a system of the form \eqref{eq:FDM-MatrixSystem} that approximates \eqref{eq:SI-BulkEqn}-\eqref{eq:SI-VertexCondition}.
Alternatively, if looking to avoid a complex mesh, one could instead approximate the values like $u_{p,q}^{+e_{jk}}$ using interpolations of nearby nodal values, which would reduce the sparsity of the resulting matrix $\mathcal{F}$, but might avoid the need to use additional nodes, or keep track of nodes in a non-uniform mesh.

To illustrate the above, we perform the steps above using the cross-in-the-plane geometry \tstk{ref}.
Take $N\in\naturals$ to be odd, and place nodes $\tau_{p,q} = \bracs{(p-1)j, (q-1)h}$ for $p,q\in\clbracs{1,...,N}$, giving a mesh-width $h=\recip{N-1}$.
Uniform mesh is suitable for this geometry because our edges are aligned with the coordinate axes, so the nearest-neighbours $\tau_{p,q}^{\pm e_{jk}}, \tau_{p,q}^{\pm n_{jk}}$ always exist when looking at a node $\tau_{p,q}\in I_{jk}$, and are just the 4 adjacent nodes $\tau_{p-1,q},\tau_{p+1,q},\tau_{p,q-1},\tau_{p+1}$.
Encoding the periodic boundary conditions amounts to ``enslaving" the nodal values $u_{p,N}$ to $u_{p,0}$ (and similarly $u_{N,q}$ to $u_{0,q}$) --- as such, we also adopt the convention that $u_{p,-1}=u_{p,N-1}$ and $u_{-1,q}=u_{N-1,q}$.
Furthermore, the nodes $\tau_{\frac{N-1}{2},q}$ and $\tau_{p,\frac{N-1}{2}}$ are precisely the nodes that lie on the singular inclusions, and $\tau_{\frac{N-1}{2},\frac{N-1}{2}}$ is placed at the vertex $v_0$, with all other nodes lying in the (interior of the) bulk regions.
Upon discretising, we retain \eqref{eq:SI-FDMBulkDiscretise} at the nodes $\tau_{p,q}$ that lie in the bulk regions.
For those $\tau_{p,q}$ that lie on the horizontal edge $I_h$ (that is, $\tau_{p,\frac{N-1}{2}}$ for $p\neq\frac{N-1}{2}$) the discretisation \eqref{eq:SI-GeneralEdgeDiscretise} becomes
\begin{align*}
	- \bracs{\diff{}{y} + \rmi\qm_{jk}}^2u^{(jk)}_{p,q} - \bracs{\bracs{\grad u_{p,q}\cdot n_{jk}}^+ - \bracs{\grad u_{p,q}\cdot n_{jk}}^-}
	& \approx \bracs{\qm_1^2 + 2h^{-1} + 2h^{-2}}u_{p,q} \\
	& \quad - h^{-1}\bracs{h^{-1} + \rmi\qm_1}u_{p+1,q} \\
	& \quad - h^{-1}\bracs{h^{-1} - \rmi\qm_1}u_{p-1,q} \\
	& \quad - h^{-1}u_{p,q+1} - h^{-1}u_{p,q-1}. \labelthis\label{eq:SI-FDMHorzEdgeDiscretise}
\end{align*}
Our use of a uniform mesh resulting in \eqref{eq:SI-FDMHorzEdgeDiscretise} being much simpler than \eqref{eq:SI-GeneralEdgeDiscretise}.
Similarly, for nodes $\tau_{p,q}$ on the vertical edge $I_v$ (that is, $\tau_{\frac{N-1}{2},q}$ for $q\neq\frac{N-1}{2}$) we have that
\begin{align*}
	- \bracs{\diff{}{y} + \rmi\qm_{jk}}^2u^{(jk)}_{p,q} - \bracs{\bracs{\grad u_{p,q}\cdot n_{jk}}^+ - \bracs{\grad u_{p,q}\cdot n_{jk}}^-}
	& \approx \bracs{\qm_2^2 + 2h^{-1} + 2h^{-2}}u_{p,q} \\
	& \quad - h^{-1}\bracs{h^{-1} + \rmi\qm_2}u_{p,q+1} \\
	& \quad - h^{-1}\bracs{h^{-1} - \rmi\qm_2}u_{p,q-1} \\
	& \quad - h^{-1}u_{p+1,q} - h^{-1}u_{p-1,q}. \labelthis\label{eq:SI-FDMVertEdgeDiscretise}
\end{align*}
Finally, at the central vertex $v_0 = \tau_{\frac{N-1}{2},\frac{N-1}{2}}$, we have that
\begin{align} \label{eq:SI-FDMVertCond}
	\sum_{j\con k}\bracs{ \pdiff{}{n} + \rmi\qm_{jk} }u^{(jk)}\bracs{v_0}
	&\approx h^{-1} \bracs{ 4u_{p,q} - u_{p+1,q} - u_{p-1,q} - u_{p,q+1} - u_{p,q-1} },
\end{align}
where $p = q = \frac{N-1}{2}$.
The matrix-valued $\beta\bracs{\omega^2}$ is also easily computed as
\begin{align*}
	\beta:\complex\rightarrow\complex^{(N-1)\times(N-1)}, 
	\qquad \bracs{\beta\bracs{\omega^2}}_{jk} = \begin{cases} 1 & j=k\neq\frac{N-1}{2}, \\ 0 & \text{otherwise}. \end{cases}
\end{align*}

\tstk{this gives us the numerical results in the file \texttt{CompositeMedium\_PeriodicFDM.ipynb} - I can export the results we want to pdfs and import them as images here.
Things to note are that the finite-difference matrix $\mathcal{F}$ is not Hermitian, but the eigenvalues appear to all be real.
The eigenfunction plots are also quite nice, but we don't have anything to compare them to (true values, etc).
We could do a crude sweep over the quasi-momentum values to see if the symmetry properties hold, and what the spectrum in general looks like as we vary the quasimomentum?
It might also be worth timing the runs for comparison with the ``spectral" method, or working out the cost in FLOPs, time, etc for the scheme to run?
Also need to mention how we have wasted a lot of our effort in solving in the bulk regions, when in reality we don't actually need the form of the function here...}

% Derivation of a non-local quantum graph problem
\section{Reformulation of \eqref{eq:SI-WaveEqn} as a Non-Local Quantum Graph Problem} \label{sec:SI-NonLocalQG}
Both of the methods employed in sections \ref{sec:SI-VarProbMethod} and \ref{ssec:SI-FDMMethod} force us to handle the interaction between the bulk regions and skeleton in some way; either implicitly through the global approximation we use (if working from \eqref{eq:SI-VarProb}) or explicitly in how we tie the various approximations in the bulk regions and skeleton together (in the approach of section \ref{ssec:SI-FDMMethod}).
We now ask the question as to whether we can go further, and find a formulation for our problem that allows us to remove the equations in the bulk regions, replacing them with suitable terms in the equations along the skeleton.
Our objective is to arrive at a quantum graph problem; we have seen the effectiveness of the techniques involving the $M$-matrix for extracting the spectrum of the problems in chapter \ref{ch:ScalarSystem} (and by extension chapter \ref{ch:CurlCurl}), and hope to find similar success by transforming the problem \eqref{eq:SI-StrongForm}.
From the perspective of solution (either numerically or analytically) we also expect to see a reward in dimension reduction if we can reduce \eqref{eq:SI-StrongForm} to a problem on the one-dimensional skeleton.

\subsection{Formulation of the non-local quantum graph problem} \label{ssec:SI-ToQG}
In order to investigate how to reduce \eqref{eq:SI-WaveEqn} to a quantum graph problem, we need a method of reconstructing the behaviour of a solution $u$ in the bulk regions from its behaviour on the skeleton.
We already know that the eigenfunction itself is continuous across the skeleton, so the function values on the boundary of each bulk region coincide with the function values on the skeleton.
Furthermore, we also observe that the only information that the skeleton needs about the solution in the bulk regions are the traces of the normal derivative.
So the question becomes whether there is a method or function through which we can extract the (traces of the) normal derivatives from the bulk regions from the (boundary values of the) function on the skeleton, given that we know \eqref{eq:SI-BulkEqn} is satisfied in each bulk region.
Our attention is naturally drawn to the Dirichlet-to-Neumann map for the Helmholtz equation on $\ddom_i$.

For each bulk region $\ddom_i$, fix $\omega^2>0$ and a solution $u\in H^2\bracs{\ddom_i}$ to the problem
\begin{align*}
	\bracs{ \laplacian^\qm + \omega^2 }u &= 0, \qquad\text{on } \ddom_i, \\
	u\vert_{\partial\ddom_i} &= g,
\end{align*}
where $g\in\ltwo{\partial\ddom_i}{S}$.
If $\omega^2$ is an eigenvalue of the Dirichlet Laplacian $-\laplacian^\qm_0$ on $\ddom_i$, then notice that the solution $u$ to the above problem is not unique --- we can add any multiple of the eigenfunction $u_f$ of $-\laplacian^\qm_0$ corresponding to the eigenvalue $\omega^2$ to $u$ and obtain another solution.
Consequentially the Neumann data of a solution is not uniquely determined by its Dirichlet data $u\vert_{\partial\ddom_i}$ at such values of $\omega^2$, and thus the Dirichlet to Neumann map $\dtn^i_\omega$ for the operator $\laplacian_{\qm}+\omega^2$ is not well-defined at this value of $\omega^2$.
For those values of $\omega^2$ that are not eigenvalues of $-\laplacian^\qm_0$ however, we can define the Dirichlet to Neumann map $\dtn^i_\omega$ by first letting
\begin{align*}
	\dom\bracs{\dtn^i_\omega} = \{ \bracs{g,h}\in\ltwo{\partial\ddom_i}{S}\times\ltwo{\partial\ddom_i}{S} \ \vert \
	& \exists v\in\gradgradSob{\ddom_i}{\lambda_2} \text{ s.t. } \\
	&\quad \bracs{\laplacian_\qm + \omega^2}v = 0, \\
	&\quad v\vert_{\partial\ddom_i}=g, \ \left.\bracs{\tgrad v\cdot n}\right\vert_{\partial\ddom_i} = h \},
	\labelthis\label{eq:SI-DtNMapRegion}
\end{align*}
and assigning the action $\dtn^i_\omega g = h$ where $g,h$ are related as in \eqref{eq:SI-DtNMapRegion}.

Now suppose we have a solution $u, \omega^2$ to \eqref{eq:SI-StrongForm}, with $\omega^2$ not being an eigenvalue of $-\laplacian^\qm_0$ on any of the bulk regions $\ddom_i$.
The boundary of $\ddom_i$ is a union of a (finite number of) the skeletal edges $I_{jk}$, and since $u\in\tgradSob{\ddom}{\ccompMes}$, we also have that $u\in\tgradSob{\ddom}{\dddmes}$ and consequentially $u\in\gradSob{\partial\ddom_i}{S}$.
Given continuity of $u$ across the skeleton, and $u\in\gradSob{\ddom_i}{\lambda_2}$, we have that $u$ serves the role of $v$ in \eqref{eq:SI-DtNMapRegion}, with $g=u\vert_{\partial\ddom_i}$.
Therefore,
\begin{align*}
	\bracs{\dtn^+_\omega u}\vert_{I_{jk}} = - \bracs{\tgrad u\cdot n_{jk}}^+,
	&\quad
	\bracs{\dtn^-_\omega u}\vert_{I_{jk}} = \bracs{\tgrad u\cdot n_{jk}}^-,
\end{align*}
where we have used $\dtn^\pm_\omega$ to denote the Dirichlet-to-Neumann maps for the regions $\ddom^{\pm}_{jk}$ corresponding to the edge $I_{jk}$.
Substituting into \eqref{eq:SI-InclusionEqn} gives us the new equation
\begin{align*}
	- \bracs{\diff{}{y} + \rmi\qm_{jk}}^2u^{(jk)} 
	&= \omega^2 u^{(jk)} - \bracs{ \dtn^+_\omega u + \dtn^-_\omega u },
\end{align*}
on each $I_{jk}$.

Now consider a solution $\omega^2>0$, $u\in H^2\bracs{\graph}$ of the problem
\begin{subequations} \label{eq:SI-NonLocalQG}
	\begin{align}
		- \bracs{\diff{}{y} + \rmi\qm_{jk}}^2u^{(jk)} 
		&= \omega^2 u^{(jk)} - \bracs{ \dtn^+_\omega u^{(jk)} + \dtn^-_\omega u^{(jk)}},
		&\qquad\text{on } I_{jk}, \label{eq:SI-NonLocalQGEdgeEquation}  \\
		\sum_{j\con k} \bracs{\pdiff{}{n}+\rmi\qm_{jk}} u^{(jk)}(v_j) &= \alpha_j\omega^2 u(v_j),
		&\qquad\text{at every } v_j\in\vertSet, \label{eq:SI-NonLocalQGVertexDeriv} \\
		u \text{ is continuous,} & 
		&\qquad\text{at every } v_j\in\vertSet, \label{eq:SI-NonLocalQGVertexCont}
	\end{align}
\end{subequations}
where again we assume $\omega^2$ is not in the spectrum of $-\laplacian^\qm_0$ on any $\ddom_i$.
Then this eigenpair of \eqref{eq:SI-NonLocalQG} also defines a solution $\omega^2, \tilde{u}$ to \eqref{eq:SI-StrongForm}; take $\tilde{u}=u$ on the skeleton $\graph$, and in the bulk region $\ddom_i$ assign $\tilde{u}$ the values of the function $v$ in \eqref{eq:SI-DtNMapRegion} for $g=u$.
So excluding the possibility for eigenvalues being shared with $-\laplacian^\qm_0$ on one of the bulk regions $\ddom_i$, the system \eqref{eq:SI-NonLocalQG} shares the same eigenvalues as \eqref{eq:SI-StrongForm}.

The problem \eqref{eq:SI-NonLocalQG} is inherently non-local; for each edge $I_{jk}$ the terms $\dtn^{\pm}_\omega u$ require information about the values of the edge functions $u^{(lm)}$ for all edges $I_{jk}\subset\ddom_{jk}^{\pm}$, and these edges are not necessarily directly connected to $I_{jk}$ itself.
Indeed, the appearance of the Dirichlet to Neumann map in the ODEs along the skeleton is the price that we must pay for removing the bulk regions, but still needing to retain the information about how their presence induces interactions between non-adjacent edges.
Our previous approaches that used the $M$-matrix to access the eigenvalues of quantum graph problems (sections \ref{sec:ScalarDiscussion} and \ref{sec:ScalarExamples}) largely depended upon computing an explicit form for the entries of the $M$-matrix.
This is currently unfeasible as there is no closed form for the action of the map $\dtn^i_\omega$ on a general polygonal domain.
This lack of a closed form, and the fact that $\dtn^i_\omega$ depends on our spectral parameter $\omega^2$  also has serious drawbacks when we hypothesise about solving \eqref{eq:SI-NonLocalQG} numerically, which is the subject of section \ref{ssec:SI-GraphMethod}.

We have already mentioned that, when $\omega^2$ is an eigenvalue of $-\laplacian^\qm_0$ on any of the bulk regions $\ddom_i$ we cannot apply the same reasoning to reduce \eqref{eq:SI-StrongForm} to a problem on the skeleton.
It is however worth briefly exploring how much we can do to reduce \eqref{eq:SI-StrongForm} to the skeleton if we happen to land on a value $\omega_i^2$ which \emph{is} an eigenvalue of $-\laplacian^\qm_0$ in one of the bulk regions $\ddom_i$.
Let $\varphi^{(i)}_n$ be the eigenfunction(s) corresponding to this eigenvalue (where the index $n$ ranges from $1$ to the multiplicity $N_i$ of $\omega_i^2$), and write $S_i = \mathrm{span}\clbracs{\varphi^{(i)}_n}$.
As we have already seen, we can add linear combinations of these $\varphi^{(i)}_n$ to a function $u$ that satisfies \eqref{eq:SI-BulkEqn} in $\ddom_i$ and still end up with a function $\tilde{u} := u + \sum_{n}c_n\varphi^{(i)}_n$ (where $c_n\in\complex$) that satisfies \eqref{eq:SI-BulkEqn} with $\tilde{u}\vert_{\partial\ddom_i} = u\vert_{\partial\ddom_i}$.
However the the traces of the normal derivatives of $u$ and $\tilde{u}$ onto $\partial\ddom_i$ no longer match, and we have some additional ``freedom" in \eqref{eq:SI-InclusionEqn} when the edge $I_{jk}$ forms part of the boundary of $\ddom_i$.
Indeed, if there exists a $\phi\in S_i$ such that $u$ satisfies
\begin{subequations}
	\begin{align*}
		-\laplacian_{\qm}u &= \omega_i^2 u, &\text{in } \ddom_i, \\
		-\bracs{\diff{}{y} + \rmi\qm_{jk}}^2u^{(jk)}  
		&= \omega_i^2 u^{(jk)} + \bracs{\bracs{\grad u\cdot n_{jk}}^+ - \bracs{\grad u\cdot n_{jk}}^-} + \mathrm{T}\phi,
		&\text{on } I_{jk}\subset\partial\ddom_i, \\
		- \bracs{\diff{}{y} + \rmi\qm_{jk}}^2u^{(jk)} 
		&= \omega^2 u^{(jk)} - \bracs{ \dtn^+_\omega u^{(jk)} + \dtn^-_\omega u^{(jk)}},
		&\text{on } I_{jk}\cap\partial\ddom_i=\emptyset, \\
		\sum_{k} \bracs{\pdiff{}{n}+\rmi\qm_{jk}} u^{(jk)}(v_j) 
		&= \alpha_j\omega^2 u(v_j),
		&\text{at each } v_j\in\vertSet,
	\end{align*}
\end{subequations}
where
\begin{align*}
	\mathrm{T}\phi &=
	\begin{cases}
		0, & \ddom_i\neq\ddom_{jk}^{\pm}, \\
		\bracs{\grad\phi\cdot n_{jk}}^+, & \ddom_i = \ddom_{jk}^+, \\
		-\bracs{\grad\phi\cdot n_{jk}}^-, & \ddom_i = \ddom_{jk}^-, \\
		\bracs{\grad\phi\cdot n_{jk}}^+ -\bracs{\grad\phi\cdot n_{jk}}^-, & \ddom_i = \ddom_{jk}^+ \text{ and } \ddom_i = \ddom_{jk}^-,
	\end{cases}
\end{align*}
then $u, \omega_i^2$ will be a solution to \eqref{eq:SI-StrongForm}.
The $\mathrm{T}\phi$ term encodes this aforementioned freedom that comes from our ability to add eigenfunctions of $-\laplacian^\qm_0$ to $u$ in $\ddom_i$, and the knock-on affect on the equations on the skeletons.
If one wishes to look for a solution to this problem, then we have to accept that we will be solving for a tuple $\bracs{u, \clbracs{c_n}}\in H^2\bracs{\graph}\times\complex^{N_i}$ where $\phi=\sum_n c_n\varphi^{(i)}_n$, and incorporate this accordingly into our solution method.

\subsection{Considerations for numerical solution on the skeleton} \label{ssec:SI-GraphMethod}
Although we have succeeded in obtaining the problem \eqref{eq:SI-NonLocalQG} which is posed on a (collection of) one-dimensional domains from the two-dimensional problem \eqref{eq:SI-StrongForm} on $\ddom$, the price we have paid is rather severe.
The handling of the Dirichlet-to-Neumann map term is the main complication with any numerical approach to solving \eqref{eq:SI-NonLocalQG}.
The non-locality in \eqref{eq:SI-NonLocalQGEdgeEquation} rules out numerical methods that rely on local approximations to the solution, such as the regional finite differences similar to those employed in \ref{ssec:SI-FDMMethod}, as we have no way of expressing the action of $\dtn^i_\omega$ \emph{solely} in terms of nearby function values.
It also rules out our previous approach involving the $M$-matrix\footnote{There is also the pressing theoretical question as to whether \eqref{eq:SI-NonLocalQG} admits a boundary triple.} since we can no longer construct the entries of the $M$-matrix just from the knowledge of the function values at the vertices.

Given the non-locality that $\dtn^i_\omega$ introduces, we must again turn to ideas from spectral methods to approximate a solution $u$ to \eqref{eq:SI-NonLocalQG}.
So let us suppose that we have some finite dimensional subspace $V$ spanned by functions $\psi_m$, $1\leq m\leq M$, in which we will represent our solution by the formula in \eqref{eq:SI-VPTruncatedBasis}.
The difficulty that we face with any numerical scheme is in how we approximate the action of the maps $\dtn^i_\omega$ on these basis functions $\psi_m$.
This problem is compounded by the fact that $\dtn^i_\omega$ themselves depend on the eigenvalue $\omega^2$, so however we choose to compute $\dtn^i_\omega \psi_m$ has to be general enough to account for the possibility of evaluation at different values of $\omega$.
That is, we need to know the action of $\dtn_{\omega}^i$ for every bulk region $\ddom_i$, on \emph{every} basis function $\psi_m$, or at the least be able to approximate this.
Asking for an analytic expression is a tall order, especially given the wide variety of shapes that the bulk regions $\ddom_i$ can take --- in general, $\ddom_i$ could be any bounded, polygonal domain.
If we only need a numerical approximation, our obvious option for computing $\dtn^i_\omega\psi_m$ is solving the equation \eqref{eq:SI-BulkEqn} subject to the boundary conditions $u=\psi_m$, and reading off the Neumann data.
However this takes us back to numerically solving each of the problems
\begin{align*}
	-\laplacian_{\qm}v &= \omega^2 v &\text{on } \ddom_i, \\
	v\vert_{\partial\ddom_i} &= \psi_m,
\end{align*}
for every $m=1,...,M$ and bulk region $\ddom_i$, and then reading off the Neumann data of the computed solution.
Doing so means that we are no longer only considering a one-dimensional problem: we might as well employ the techniques of section \ref{ssec:SI-FDMMethod} and directly compute approximations over the whole of $\ddom$, rather than solve $M$ second-order equations on each of the bulk regions $\ddom_i$ to extract $\dtn^i_\omega\psi_m$ and then solve a one-dimensional problem to construct $u$ from the $\psi_m$.

Possible ideas for making progress with a numerical scheme might include representing $u$ on each $\partial\ddom_i$ as a sum of the eigenfunctions of the map $\dtn^i_\omega$.
This makes evaluation of the action of $\dtn^i_\omega$ cheap, provided one has access to the eigenvalues $w^{(i)}_n$ and eigenfunctions $\varphi^{(i)}_n$ of $\dtn^i_\omega$.
These are the so-called Steklov eigenvalues and eigenfunctions, that solve the system (Steklov eigenvalue problem)
\begin{align*}
	\bracs{\laplacian_\qm + \omega^2}\varphi^{(i)}_n &= 0, \qquad\text{in } \ddom_i, \\
	\varphi^{(i)}_n\vert_{\partial\ddom_i} &= w^{(i)}_n\pdiff{\varphi^{(i)}_n}{n}\vert_{\partial\ddom_i}. 
\end{align*}
However one then needs to be able to reconcile different representations of $u$ on edges common to two different bulk regions, and still faces the prospect of computing solutions to the Steklov problem in each of the bulk regions.
Further details and the questions that require answers pertaining to these points, from our attempts to implement such a method, are available for the interested reader in the appendix \ref{sec:SIApp-NonLocalSolve}.
Ultimately however, short of obtaining an analytic expression (or an expression that can be easily approximated numerically) for the action of $\dtn^i_{\omega}$ in terms of $\omega$, there is no simple way to deal with the action of $\dtn^i_{\omega}$ that avoids consideration of PDEs in the bulk regions.

%If we are to accept this non-locality we must use the ideas from spectral methods; decide on a finite-dimensional subspace $V$ in which to approximate the solution to $u$, determine a suitable basis for this space, the expand our approximate solution in terms of this basis and choose the coefficients of the basis expansion to satisfy the problem \eqref{eq:SI-NonLocalQG} in $V$.
%For the time being, we will simply let $V\subset\htwo{\graph}$ be a finite-dimensional subspace with dimension $M$ and basis functions $\clbracs{\psi_m}_{m=1}^{M}$.
%For the purposes of our discussion we will also take $\alpha_j=0$ at each of the vertices, since the major problems with a numerical approach to \eqref{eq:SI-NonLocalQG} are present whether or not the coupling constants are non-zero.
%Write the approximate solution $u_V\in V$ to \eqref{eq:SI-NonLocalQG} as
%\begin{align*}
%	u_V &= \sum_{m=1}^M u_m\psi_m, \qquad u_m\in\complex,
%\end{align*}
%for basis coefficients $u_m$ to be determined.
%We then multiply write \eqref{eq:SI-NonLocalQG} in the (``weak") form
%\begin{align*}
%	\sum_{v_j\in\vertSet}\sum_{j\conLeft k} 
%	\clbracs{ 
%	\integral{I_{jk}}{\tgrad_{\lambda_{jk}}u\cdot\overline{\tgrad_{\lambda_{jk}}\phi}}{\lambda_{jk}} 
%	+ \integral{I_{jk}}{\overline{\phi}\dtn^+_{\omega}u + \overline{\phi}\dtn^-_{\omega}u}{\lambda_{jk}} 
%	}
%	&= \omega^2 \sum_{v_j\in\vertSet}\sum_{j\conLeft k}\integral{I_{jk}}{u\overline{\phi}}{\lambda_{jk}},
%\end{align*}
%which holds for each $\phi$ in a suitable space of test functions.
%From here, we replace $u$ with $u_V$, substitute the basis expansion, and choose $\phi=\psi_n$ for each $n\in\clbracs{1,...,M}$ to obtain
%\begin{align*}
%	0 &= 
%	\sum_{m=1}^{M} u_m \bracs{ A_{n,m} + L_{n,m} - \omega^2 B_{n,m} }, \\
%	A_{n,m} &= \sum_{v_j\in\vertSet}\sum_{j\conLeft k} \ip{\tgrad_{\lambda_{jk}}\psi_m}{\tgrad_{\lambda_{jk}}\psi_n}_{L^2\bracs{I_{jk}}}, \\
%	L_{n,m}\bracs{\omega} &= \sum_{v_j\in\vertSet}\sum_{j\conLeft k} \ip{\dtn^+_{\omega}\psi_m + \dtn^-_{\omega}\psi_m}{\psi_n}_{L^2\bracs{I_{jk}}}, \\
%	B_{n,m} &= \sum_{v_j\in\vertSet}\sum_{j\conLeft k} \ip{\psi_m}{\psi_n}_{L^2\bracs{I_{jk}}}.
%\end{align*}
%Letting $M\bracs{\omega^2}$ be the matrix-valued function with entries
%\begin{align} \label{eq:SI-NonLocalQGNumericalDisc}
%	\bracs{M\bracs{\omega^2}}_{n,m} &= A_{n,m} + L_{n,m}\bracs{\omega} - \omega^2 B_{n,m},
%\end{align}
%our coefficients $u_m$ and approximate eigenvalues $\omega^2$ are then determined by the solution to the generalised eigenvalue problem
%\begin{align*}
%	M\bracs{\omega^2}U &= 0,
%	\qquad
%	U = \bracs{u_1, u_2, ... , u_M}^{\top}.
%\end{align*}
%Insofar we have done nothing different to a standard spectral or finite element method, with the only non-standard terms appearing in our matrix $M\bracs{\omega^2}$ being the $L_{n,m}$ terms coming from the Dirichlet to Neumann map.
%These depend in a non-trivial way on the spectral parameter $\omega^2$, and form the major obstacle to the implementation of any numerical scheme.
%In order to evaluate $L_{n,m}$, we need to know the action of $\dtn_{\omega}^i$ (for every bulk region $\ddom_i$) on \emph{every} basis function $\psi_m$, or at the least be able to approximate this.
%Asking for an analytic expression is a tall order, especially given the wide variety of shapes that the bulk regions $\ddom_i$ can take --- in general, $\ddom_i$ could be any bounded, polygonal domain.
%On the numerical approximation side, our obvious option for computing $\dtn^i_\omega\psi_m$ is solving the equation \eqref{eq:SI-BulkEqn} subject to the boundary conditions $u=\psi_m$, and reading off the Neumann data.
%However this forces us back to numerically solving each of the problems
%\begin{align*}
%	-\laplacian_{\qm}v &= \omega^2 v &\text{on } \ddom_i, \\
%	v\vert_{\partial\ddom_i} &= \psi_m,
%\end{align*}
%for every $m=1,...,M$ and bulk region $\ddom_i$, and then reading off the Neumann data of the computed solution.
%Furthermore, we will have to perform these computations every time we want to evaluate $M(\omega^2)$ whilst solving the generalised eigenvalue problem.
%Ultimately, short of obtaining an analytic expression (or an expression that can be easily approximated numerically) for the action of $\dtn^i_{\omega}$ in terms of $\omega$, there is no simple way to deal with
%the inherent influence of \eqref{eq:SI-BulkEqn} on \eqref{eq:SI-NonLocalQGEdgeEquation}.
%Possible ideas for alternatives include expressing $L_{n,m}$ in terms of the eigenvalues and eigenfunctions of the $\dtn^i_\omega$, or exploring whether the Dirichlet-to-Neumann map admits an ``expansion" --- these are briefly elaborated on in section \ref{sec:SIApp-NonLocalSolve}.
%Ultimately however, a deeper understanding of the properties of $\dtn_\omega^i$ are required to make any further progress without falling back to solving a two-dimensional problem in the bulk regions.

% Concluding remarks and future work
\section{Conclusions and Further Exploration} \label{sec:SI-Conc}
We conclude our investigation into the various formulations of \eqref{eq:SI-WaveEqn}, our attempts to access the spectrum, and the insights we have obtained with a summary and survey of open questions motivated by this work.
Recall that the motivation for studying \eqref{eq:SI-WaveEqn} was \tstk{many-fold};
\begin{itemize}
	\item Study the ``intuitive" limit of thin-structure problems, with a view to motivating rigorous analysis and further generalisations to problems in electromagnetism
	\item Perform preliminary investigations into the effects introduced by the presence of geometric contrast
	\item Investigate numerical approaches to the solution of such problems
\end{itemize} 

Each of the sections \ref{sec:SI-VarProbMethod}, \ref{sec:SI-StrongDerivation}, and \ref{sec:SI-NonLocalQG} has provided us with an alternative formulation of \eqref{eq:SI-WaveEqn}.
We demonstrated in section \ref{sec:SI-VarProbMethod} that we can attempt to solve \eqref{eq:SI-WeakWaveEqn} directly through use of the min-max principle, appealing only to the understanding of $\tgradSob{\ddom}{\ccompMes}$ that we gained from section \ref{sec:CompSobSpaces}.
Whilst approximation of the eigenvalues, eigenfunctions, and dispersion relations was possible using this method, the formulation itself does not provide us with any explicit insights into the interactions between the skeleton and the bulk, nor the effect of the coupling constants $\alpha_j$.
We also highlighted that the lack of a priori knowledge concerning the explicit functions that live in $\tgradSob{\ddom}{\ccompMes}$, and the subset of these that are solutions to \eqref{eq:SI-WaveEqn}, forces us to use inefficient choices for our basis functions.
Indeed, having to hand an explicit orthonormal basis for $\tgradSob{\ddom}{\ccompMes}$, or some a priori information about the form of the eigenfunctions would go a number of ways towards fixing this issue.
We also highlight that if one is intent on solving \eqref{eq:SI-WaveEqn} directly from the variational problem, one might wish to consider a finite element based approach.
There would still be a number of issues surrounding explicitly determining a suitable basis to use in computations, however \eqref{eq:SI-WeakWaveEqn} is already in the form $b_{\qm}(u,\phi)=\ip{u}{\phi}$, so already invites solution via finite elements after checking the bilinear form $b_{\qm}$ is suitably well-behaved\footnote{That is, $b_{\qm}$ is bounded and elliptic on $\tgradSob{\ddom}{\ccompMes}$ so that one has convergence of such a scheme guaranteed.}.

The difficulties we came across when working in $\tgradSob{\ddom}{\ccompMes}$ lead us to the decision to derive a formulation of \eqref{eq:SI-WaveEqn} that does not involve the rather unfamiliar tangential gradients with respect to $\ccompMes$.
This lead us to the ``strong" formulation \eqref{eq:SI-StrongForm}, which clarifies the interactions between the bulk regions, skeleton, and vertices.
We observe that the inclusion of a background material causes a coupling between the solution in the bulk regions and on the skeleton through the \tstk{Ventcel'?} condition \eqref{eq:SI-InclusionEqn}, and bears additional similarities to effective problems derived in the context of elasticity (without geometric contrast between the edges and vertices).
Furthermore, the geometric contrast provided by the coupling constants $\alpha_j$ is again present in a non-classical Kirchoff condition at the vertices.
Whilst the physical material that we have been considering does not possess any material contrast between the bulk and skeleton, we highlight that we may ``add mass" to the measure $\lambda_2$ to capture the effects of such contrast in our strong formulation too.
The observations justify our approach via singular measures and physically motivated variational problems as a useful tool for providing an insight into the effective problems one can expect to obtain from the study of composite materials with geometric and/or material contrast.
However, the question is still open in the literature as to whether the (operator whose spectral problem is the) strong formulation \eqref{eq:SI-StrongForm} is the limit (either in the spectral or norm-resolvent sense) of a sequence of operators on a composite, periodic medium with both geometric and material contrast.
On the numerical side, we can exploit the additional information gained from \eqref{eq:SI-StrongForm} to approximate the action of the operator (that defines the left hand side of \eqref{eq:SI-FDMEquationsToDisc}) on $u$ in each of the bulk regions and on each edge of the skeleton, then couple the various ``regional" approximations via the flux across the edges and the non-classical Kirchoff condition at the vertices.
Such a scheme allows us to avoid the problems surrounding the space $\tgradSob{\ddom}{\ccompMes}$ that we ran into previously, albeit not without introducing a number of other considerations in setting up such a scheme.
Our example concerning the cross-in-the-plane geometry also demonstrates that the introduction of geometric contrast (that is, ensuring that $\alpha_j\neq0$) is sufficient to cause band-gaps to emerge in the spectrum of $-\bracs{\tgrad_{\ccompMes}}^2$. \tstk{alpha going to infinity causes Dirichlet decoupling maybe? do some more runs...}
There also appears to be good agreement with the approximations obtained from solution to \eqref{eq:SI-VarProb}, and convergence to the eigenfunctions and eigenvalues shared with the Dirichlet Laplacian.

The final step in our investigation looks into whether there is anything to be gained by attempting by attempting to ``remove" the bulk regions from our formulation entirely.
In light of chapter \ref{ch:ScalarSystem}, we already have a solid understanding quantum graph problems and how to analyse them, and additionally expect that solving a system of ODEs on the edges will be less complex than a system of PDEs coupled to ODEs (that themselves obey a non standard condition at the vertices).
It is possible for us to make such a reduction, provided that we stay away from those $\omega^2$ that are eigenvalues of the Dirichlet Laplacian on one of the bulk regions $\ddom_i$.
This is not ideal as the cross in the plane geometry demonstrates that such eigenvalues can form part of the spectrum of $-\bracs{\tgrad_{\ccompMes}}^2$.
For the eigenvalues that remain, we can realise \eqref{eq:SI-WaveEqn} as a quantum graph problem.
However this comes at the expense of introducing a non-local term involving Dirichlet-to-Neumann maps into the resulting system of edge ODEs.
Unfortunately, solution to such a problem numerically requires us to have the ability to evaluate (at least approximately) these Dirichlet-to-Neumann maps, the only reliable way being to return to solving PDEs in the bulk regions.
With a deeper understanding of the action of these Dirichlet-to-Neumann maps, a numerical approach to this non-local quantum graph problem might become tractable.
However at present, price we pay for removing the PDEs in the bulk regions from our formulation is not worth the complexity introduced by the non-local effects in the resulting ODEs on the skeleton.

In summary, the strong formulation \eqref{eq:SI-StrongFrom} provides a clear picture of how the bulk regions, skeleton, and geometric contrast each interact with one another.
We also observe that by adding mass to the relevant measures in our formulation, effects from other forms of contrasts in the material being modelled also appear in the resulting system modelled by \eqref{eq:SI-WaveEqn}.
As such, our ``intuition-motivated" variational problems have provided us with a useful tool for predicting the effective problems for composite materials under contrast.
In pursuit of methods to numerically obtain the eigenvalues $\omega^2$, one can work with either \eqref{eq:SI-VarProb}, or through finite difference approximations in each of the bulk regions coupled along the skeleton.
Each method has a number of considerations to take into account, discussed in the respective sections.
And whilst it is possible to reduce \eqref{eq:SI-WaveEqn} to a problem on the skeleton, it is not possible to do so to recover all values of $\omega^2$ (those that correspond to Dirichlet eigenvalues) nor numerically solve the resulting system without a deeper understanding of the Dirichlet-to-Neumann map.

\subsection{Extensions: Limits of Materials Under Contrast}

\subsection{Extensions: Curl-of-the-curl and the Maxwell System}

%
%\subsection{Non-Zero Coupling Constants, and the Relation Between \eqref{eq:SI-WeakWaveEqn} and \eqref{eq:SI-NonLocalQG}} \label{ssec:SI-Strauss}
%
%\tstk{copy-paste from ExtendedSpaceAlready.tex}
%Let $\ddom=\left[0,1\right)^2$ be our usual domain filled with a singular structure $\graph$, separated by $\graph$ into the pairwise-disjoint connected components $\ddom_i, i\in\Lambda$ for some finite index set $\Lambda$.
%Set $N = \abs{\vertSet}$ to be the number of vertices, and $L=\abs{\Lambda}$ be the number of bulk regions.
%Also denote by $\compMes = \lambda_2 + \ddmes$, and for coupling constants $\alpha_j>0$ at the vertices $v_j$ let $\nu = \sum_{v_j\in\vertSet}\alpha_j\delta_{v_j}$ be a weighted sum of point-mass measures centred at the vertices.
%On an edge $I_{jk}$, we denote by $\ddom_+$ the bulk region in the direction $n_{jk}$ from $I_{jk}$, and $\ddom_-$ the bulk region in the direction $-n_{jk}$ from $I_{jk}$.
%Denote by $n^{\pm}$ the unit exterior normal to $\ddom_{\pm}$ (noting that $n^{\pm}=\mp n_{jk}$), and write $\pdiff{u^{\pm}}{n^{\pm}}$ to be the normal derivative on $\partial\ddom_{\pm}$ of the function $u$ restricted to $\ddom_{\pm}$.
%
%The ``strong formulation" of our composite medium problem is
%\begin{subequations} \label{eq:StrongForm}
%	\begin{align}
%		-\laplacian_{\qm}u &= \omega^2 u, &\qquad\text{in } \ddom_i, \ \forall i\in\Lambda, \\
%		-\bracs{\diff{}{y}+\rmi\qm_{jk}}^2 u_{jk} - \bracs{\pdiff{u^+}{n^+} + \pdiff{u^-}{n^-}} &= \omega^2 u_{jk},  &\qquad\text{on every } I_{jk}\in\edgeSet, \\
%		\sum_{j\con k}\bracs{\pdiff{}{n}+\rmi\qm_{jk}}u_{jk}(v_j) &= 0, &\qquad\text{at every } v_j\in\vertSet,
%	\end{align}
%\end{subequations}
%where the function $u$ is $\gradgradSob{\ddom_i}{\lambda_2}$ for every $i$, $H^2(I_{jk})$ for every $I_{jk}$, is continuous (in the sense of traces) across $I_{jk}$, and is continuous at the vertices $v_j$.
%Recall that this problem was derived from the variational problem of finding $u\in\tgradSob{\ddom}{\compMes}$ such that
%\begin{align} \label{eq:WeakForm}
%	\integral{\ddom}{ \tgrad_{\compMes}u\cdot\overline{\tgrad_{\compMes}} }{\compMes} &=
%	\omega^2\integral{\ddom}{ u\overline{\phi} }{\compMes}, \quad\forall\phi\in\smooth{\ddom}.
%\end{align}
%
%We are interested in constructing an extended space $\mathcal{H}$ in which the problem \eqref{eq:StrongForm} reads as a standard eigenvalue problem $\mathcal{A} u = \omega^2 u$ for some operator $\mathcal{A}$ on $\mathcal{H}$.
%With this in mind, consider the space
%\begin{align*}
%	\mathcal{H} &= \bracs{\bigoplus_{i\in\Lambda}\gradgradSob{\ddom_i}{\lambda_2}} \oplus H^2\bracs{\graph} \oplus \ltwo{\ddom}{\nu},
%\end{align*}
%viewed as a subspace of
%\begin{align*}
%	\mathcal{L} := \bracs{\bigoplus_{i\in\Lambda}\ltwo{\ddom_i}{\lambda_2}} \oplus L^2\bracs{\graph} \oplus \ltwo{\ddom}{\nu},
%\end{align*}
%where we denote an element $u$ of $\mathcal{L}$ by the $\bracs{L+\abs{\edgeSet}+N}$-``vector"
%\begin{align*}
%	u &= \bracs{\clbracs{u_{\ddom_i}}_{i\in\Lambda}, \clbracs{u_{jk}}_{I_{jk}\in\edgeSet}, \clbracs{u(v_j)}_{v_j\in\vertSet}}^\top.
%\end{align*}
%Note that $\ltwo{\ddom}{\nu}\cong\complex^N$.
%Define the operator $\mathcal{A}$ by
%\begin{align*}
%	\dom\bracs{\mathcal{A}} &= \mathcal{H}, \\
%	\mathcal{A} u &= 
%	\begin{pmatrix}	
%	\clbracs{-\laplacian_{\qm}u_{\ddom_i}}_{i\in\Lambda} \\
%	\clbracs{-\bracs{\diff{}{y}+\rmi\qm_{jk}}^2 u_{jk} - \bracs{\tgrad u_{\ddom_+}\cdot n^+ + \tgrad u_{\ddom_-}\cdot n^-} }_{I_{jk}\in\edgeSet} \\
%	\clbracs{\sum_{j\con k}\bracs{\pdiff{}{n}+\rmi\qm_{jk}}u_{jk}(v_j)}_{v_j\in\vertSet}
%	\end{pmatrix}
%	\in\mathcal{L},
%\end{align*}
%and (setting aside questions about the nature of the spectrum, etc) consider the eigenvalue problem
%\begin{align*}
%	\mathcal{A} u = \omega^2 u,
%\end{align*}
%which without much effort can be shown to be equivalent to
%\begin{subequations} \label{eq:aopEvalProb}
%	\begin{align}
%		-\laplacian_{\qm}u &= \omega^2 u, &\qquad\text{in } \ddom_i, \ \forall i\in\Lambda, \\
%		-\bracs{\diff{}{y}+\rmi\qm_{jk}}^2 u_{jk} - \bracs{\tgrad u_{\ddom_+}\cdot n^+ + \tgrad u_{\ddom_-}\cdot n^-} &= \omega^2 u_{jk},  &\qquad\text{on every } I_{jk}\in\edgeSet, \\
%		\sum_{j\con k}\bracs{\pdiff{}{n}+\rmi\qm_{jk}}u_{jk}(v_j) &= \omega^2\alpha_j u(v_j), &\qquad\text{at every } v_j\in\vertSet.		
%	\end{align}
%\end{subequations}
%Here, we denote by $\tgrad u_{\ddom_+}\cdot n^+$ the trace of the $\qm$-shifted gradient $\tgrad$ of the function $u_{\ddom_{\pm}}\cdot n^{\pm}$ onto $I_{jk}$.
%
%Note that \eqref{eq:aopEvalProb} is not equivalent to \eqref{eq:StrongForm}, as functions in $\mathcal{H}$ are not required to adhere to continuity across the skeleton (hence the $\tgrad$ terms rather than just normal derivatives) nor continuity at the vertices (the notation $u(v_j)$ is just some value in $\complex$ that $u\in\mathcal{H}$ takes at each $v_j$).
%We can then ask if there is a form from which $\mathcal{A}$ can be defined, with this in mind, notice that
%\begin{align*}
%	\ip{\mathcal{A} u}{\phi}_{\mathcal{L}} &= \sum_{i\in\Lambda}\integral{\ddom_i}{ \tgrad u_{\ddom_i}\cdot\overline{\tgrad\phi} }{\lambda_2}
%	+ \sum_{v_j\in\vertSet}\sum_{j\conLeft k}\integral{\ddom}{ \bracs{\diff{}{y}+\rmi\qm_{jk}}u\overline{\bracs{\diff{}{y}+\rmi\qm_{jk}}\phi} }{\lambda_{jk}} \\
%	&= \sum_{i\in\Lambda}\ip{ \tgrad u }{ \tgrad\phi }_{\ltwo{\ddom_i}{\lambda_i}}
%	+ \sum_{v_j\in\vertSet}\sum_{j\conLeft k}\ip{ \bracs{\diff{}{y}+\rmi\qm_{jk}}u }{ \bracs{\diff{}{y}+\rmi\qm_{jk}}\phi }_{\ltwo{I_{jk}}{\lambda_{jk}}},
%\end{align*}
%so if we define (for $u\in\mathcal{H}$)
%\begin{align*}
%	\tgrad_{\mathcal{H}} u &:= \bracs{\clbracs{\tgrad u_{\ddom_i}}_{i\in\Lambda}, \clbracs{\bracs{\diff{}{y}+\rmi\qm_{jk}}u_{jk}}_{I_{jk}\in\edgeSet}, \clbracs{0}_{v_j\in\vertSet}}^\top,
%\end{align*}
%we have that
%\begin{align*}
%	\ip{\mathcal{A} u}{\phi}_{\mathcal{L}} &= \ip{\tgrad_{\mathcal{H}} u}{\tgrad_{\mathcal{H}}\phi}_{\mathcal{L}}.
%\end{align*}
%Thus, we can define $\mathcal{A}$ from the bilinear form $b(u,v) = \ip{\tgrad_{\mathcal{H}} u}{\tgrad_{\mathcal{H}}\phi}_{\mathcal{L}}$ for $u,v\in\mathcal{H}$, and write the eigenvalue problem for $\mathcal{A}$ as
%\begin{align} \label{eq:aopWeakEvalProb}
%	\ip{\tgrad_{\mathcal{H}} u}{\tgrad_{\mathcal{H}}\phi}_{\mathcal{L}} &= \omega^2\ip{u}{\phi}_{\mathcal{L}}.
%\end{align}
%
%The similarities between \eqref{eq:StrongForm} and \eqref{eq:aopEvalProb} are apparent --- if we take $\alpha_j=0$ for every $j$, then \eqref{eq:WeakForm} and \eqref{eq:aopWeakEvalProb} are the same problem, and our ``definition" of $\tgrad_{\mathcal{H}} u$ for $u\in\mathcal{H}$ coincides with $\tgrad_{\compMes}u$ for $u\in\tgradSob{\ddom}{\compMes}$.
%We even have that the set of $u\in\tgradSob{\ddom}{\compMes}$ that solve \eqref{eq:WeakForm} is a subset of $\mathcal{H}$.
%
%Furthermore, if we then ``switch on" the $\alpha_j>0$, then $\tgrad_{\mathcal{H}} u$ still coincides with what we expect the tangential gradient in $\tgradSob{\ddom}{(\compMes+\nu)}$ to be.
%The equations \eqref{eq:WeakForm} (replacing $\compMes$ with $\compMes+\nu$) and \eqref{eq:aopWeakEvalProb} are identical, and the problem \eqref{eq:aopEvalProb} (restricted to $\tgradSob{\ddom}{(\compMes+\nu)}$) reduces to what my current ``guess" at the analogue of \eqref{eq:StrongForm} with the point masses included would be.

% Chapter appendix begins
\begin{subappendices}

% Compite measure Sobolev space
%\section{The Space $\tgradSob{\ddom}{\compMes}$} \label{sec:CompSobSpaces}
Our analysis of the relevant Sobolev spaces in this section follows a similar approach to that we performed in chapter \ref{ch:ScalarSystem}.
However, we now have to worry about the behaviour of our approximating sequences on the bulk regions --- previously we did not have to pay particular care to what they were doing in these regions, since $\lambda_2$ was not present in the problem.

Throughout this section, let $\xi:\reals\rightarrow\reals$ be a smooth function such that
\begin{align*}
	\supp\bracs{\xi} = \sqbracs{-1,1}, \quad -1\leq\xi(y)\leq1 \ \forall y\in\sqbracs{-1,1}, \quad \xi(0)=0, \quad \xi'(0)=1.
\end{align*}
Note that we can choose $\xi$ such that there exists a $c>0$ s.t. $\abs{\xi'(y)}\leq c$ for all $y\in\sqbracs{-1,1}$ (and given the support of $\xi$, all $y\in\reals$).
For some of the estimates we want to make below, we also need the following notation.
\begin{definition}[Width in direction]
	Let $\ddom\subset\reals^2$ and $\clbracs{n, e}$ be a pair of orthonormal vectors in $\reals^2$.
	For each $\alpha\in\reals$, let $L_{\alpha} = \clbracs{x\in\ddom \setVert x = \alpha n + \beta e, \beta\in\reals}$ be the segment parallel to $e$ a ``signed distance" $\alpha$ from the diagonal $e$.
	Let $\lambda_{\alpha}$ be the singular measure that supports $L_{\alpha}$.
	Then the \emph{width of $\ddom$ in the direction $e$}, denoted $\ddom_e$, is defined to be $\ddom_e := \sup_{\alpha}\clbracs{\lambda_{\alpha}(L_{\alpha})}$.
\end{definition}
Note that $\ddom$ is always assumed bounded so the supremum exists, and if $\ddom$ is closed then is is a maximum and is attained for some $\alpha'$.
Furthermore, for $\alpha\in\reals$ let $\mathbf{I}^{\alpha} = \cup_{\abs{\alpha'}\leq\abs{\alpha}}L_{\alpha'}$.

\begin{lemma} \label{lem:SI-SmoothFunctionsResults}
	Let $I_{jk}\in\edgeSet$, and for $n\in\naturals$ define the function 
		\begin{align*}
			\xi_n:\ddom\rightarrow\reals, \qquad \xi_n(x) = \recip{n}\xi\bracs{n\bracs{x-v_j}\cdot n_{jk}}
		\end{align*}
		Then $\xi_n\in\psmooth{\ddom}\cap\csmooth{\ddom}$ for any\footnote{It may be required that we discard all  $n$ less than or equal to some $N\in\naturals$ for this to hold, depending on the geometry of the edge $I_{jk}$. Regardless, the result concerning the limit holds, and we can just use the subsequence $\bracs{\xi_n}_{n>N}$.} $n\in\naturals$.
	\begin{enumerate}[(i)]
		\item  We have that
		\begin{align*}
			\xi_n \lconv{\ltwo{\ddom}{\lcompMes}} 0, \qquad
			\grad\xi_n \lconv{\ltwo{\ddom}{\lcompMes}^2} n_{jk}\charFunc{jk},
		\end{align*}
		$\toInfty{n}$.
		\item For every $v_j\in\vertSet$, the functions $\eta_n^j$ defined in \eqref{eq:SmoothEtaDef} are such that
		\begin{align*}
			\eta_n^j \lconv{\ltwo{\ddom}{\compMes}} 1,
			&\qquad
			\eta_n^j \lconv{\ltwo{\ddom}{\ccompMes}} \charFunc{\ddom\setminus\clbracs{v_j}} := \begin{cases} 1 & x\neq v_j, \\ 0 & x=v_j. \end{cases}
		\end{align*}
	\end{enumerate}
\end{lemma}
\begin{proof}
	\begin{enumerate}[(i)]
		\item The $\xi_n$ are clearly smooth by construction, so we move onto establishing the convergences.
		Note that for any $x\in I_{jk}$, we have that $x = v_j + \beta e_{jk}$ for some $\beta\geq0$, and thus $\bracs{x-v_j}\cdot n_{jk} = 0$.
		Therefore, $\xi_n(x) = \recip{n}\xi(0) = 0$ whenever $x\in I_{jk}$.
		Additionally for $x\in I_{jk}$, we have that$\grad\xi_n(x) = n_{jk}\xi'\bracs{n\bracs{x-v_j}\cdot n_{jk}}$ so $\grad\xi_n(x) = n_{jk}$.
		Now,
		\begin{align*}
			\integral{\ddom}{ \abs{\xi_n(x)}^2 }{\lcompMes}
			&= \integral{\ddom}{ \recip{n^2}\abs{\xi\bracs{n\bracs{x-v_j}\cdot n_{jk}}}^2 }{\lambda_2}
			+ \integral{\ddom}{ \abs{\xi_n(0)}^2 }{\lambda_{jk}} \\
			&= \recip{n^2}\integral{\mathbf{I}^{1/n}}{ \abs{\xi\bracs{n\bracs{x-v_j}\cdot n_{jk}}}^2 }{\lambda_2}
			\leq \recip{n^2}\integral{\mathbf{I}^{1/n}}{ }{\lambda_2} \\
			&= \frac{2\ddom_{e_{jk}}}{n^3} \rightarrow 0,
		\end{align*}
		so we have that the $\xi_n$ converge to 0 in $\ltwo{\ddom}{\lcompMes}$.
		Since $n_{jk}\charFunc{jk}=0$ $\lambda_2$-almost-everywhere, we also have the estimate
		\begin{align*}
			\integral{\ddom}{ \abs{ \grad\xi_n(x) - n_{jk}\charFunc{jk} }^2 }{\lambda_2}
			&= \integral{\ddom}{ \abs{ \grad\xi_n(x) }^2 }{\lambda_2}
			= \integral{\ddom}{ \abs{n_{jk}}^2\abs{\xi'\bracs{n\bracs{x-v_j}\cdot n_{jk}}}^2 }{\lambda_2} \\
			&\leq \integral{\mathbf{I}^{1/n}}{ c^2 }{\lambda_2} 
			= \frac{2c^2\ddom_{e_{jk}}}{n},
		\end{align*}
		since $\abs{\xi'(y)}\leq c$ for all $y\in\reals$.
		We also have that
		\begin{align*}
			\integral{\ddom}{ \abs{ \grad\xi_n(x) - n_{jk}\charFunc{jk} }^2 }{\lambda_{jk}}
			&= \integral{I_{jk}}{ \abs{n_{jk}}^2\abs{ \xi'(0)-1 }^2 }{\lambda_{jk}} = 0,
		\end{align*}
		and so
		\begin{align*}
			\integral{\ddom}{ \abs{ \grad\xi_n(x) - n_{jk}\charFunc{jk} }^2 }{\lcompMes}
			&\leq \frac{2c^2\ddom_{e_{jk}}}{n} \rightarrow 0 \toInfty{n},
		\end{align*}
		and we are done.
		\item For these results we can observe that
		\begin{align*}
			\integral{\ddom}{\abs{ \eta_n^j - 1 }^2}{\lambda_2}
			&\leq \integral{B_{1/n}(v_j)}{}{\lambda_2} = \frac{\pi}{n^2}, \\
			\integral{\ddom}{\abs{ \eta_n^j - 1 }^2}{\lambda_{jk}}
			&\leq \integral{I_{jk} \cap B_{1/n}(v_j)}{}{\lambda_{jk}} = \frac{1}{n}, \\
			\integral{\ddom}{\abs{ \eta_n^j - 1 }^2}{\ddmes}
			&= \sum_{j\con k} \integral{\ddom}{\abs{ \eta_n^j - 1 }^2}{\lambda_{jk}} \leq \frac{\abs{\edgeSet}}{n},
		\end{align*}
		which demonstrate that $\eta_n^j \lconv{\ltwo{\ddom}{\compMes}} 1$.
		For the other convergence, observe that for $n \geq 2d$ where $d$ is as in \eqref{eq:HalfMinDistBetweenVertsDef}, we have that $\eta_n^j(v_k) = 1$ when $k\neq j$.
		Thus for $n \geq 2d$ we have
		\begin{align*}
			\integral{\ddom}{\abs{ \eta_n^j - \charFunc{\ddom\setminus\clbracs{v_j}} }^2}{\nu}
			&= \sum_{\substack{v_k\in\vertSet, \\ k\neq j}}\alpha_k\abs{\eta_n^j(v_k) - 1}^2
			+ \alpha_j\abs{ \eta_n^j(v_j) - 0 }^2
			= 0,
		\end{align*}
		and hence
		\begin{align*}
			\integral{\ddom}{\abs{ \eta_n^j - \charFunc{\ddom\setminus\clbracs{v_j}} }^2}{\ccompMes}
			&= \integral{\ddom}{\abs{ \eta_n^j - 1 }^2}{\lambda_2}
			+ \integral{\ddom}{\abs{ \eta_n^j - 1 }^2}{\ddmes}
			+ \integral{\ddom}{\abs{ \eta_n^j - \charFunc{\ddom\setminus\clbracs{v_j}} }^2}{\nu} \\
			&\leq \frac{\pi}{n^2} + \frac{1}{n} + 0 \qquad\text{when } n\geq 2d, \\
			&\rightarrow 0 \toInfty{n},
		\end{align*}
		providing the other convergence result.
	\end{enumerate}
\end{proof}

Lemma \ref{lem:SI-SmoothFunctionsResults} demonstrates to us that, similarly to the case in chapter \ref{ch:ScalarSystem}, the measure $\compMes$ cannot see changes ``across" the edges $I_{jk}$.
\begin{lemma} \label{lem:SI-SmoothGradZero}
	Suppose that $\phi\in\psmooth{\ddom}$, and let $I_{jk}\in\edgeSet$.
	Then the function
	\begin{align*}
		\tilde{g} &= \begin{cases} 0 & x\in\ddom\setminus I_{jk}, \\ \phi n_{jk} & x\in I_{jk}, \end{cases}
		\in\gradZero{\ddom}{\lcompMes}.
	\end{align*}
\end{lemma}
\begin{proof}
	Take $\xi_n$ to be the sequence in lemma \ref{lem:SI-SmoothFunctionsResults}, and set $\phi_n(x) = \phi(x)\xi_n(x)$ for each $n\in\naturals$.
	Note that
	\begin{align*}
		\grad\phi_n(x) &= \xi_n(x)\grad\phi(x) + \phi(x)\xi'\bracs{n\bracs{x-v_j}\cdot n_{jk}}n_{jk},
	\end{align*}
	so $\phi_n(x) = 0$ and $\grad\phi_n(x) = \phi(x)n_{jk}$ when $x\in I_{jk}$, due to the properties of $\xi_n$.
	Then $\phi_n\rightarrow0$ and $\grad\phi_n\rightarrow\tilde{g}$ by lemma \ref{lem:SI-SmoothFunctionsResults}(i).
\end{proof}

Our next result demonstrates that we gradients of zero with respect to $\lambda_{jk}$ can be elevated to be gradients of zero with respect to $\lcompMes$.
\begin{lemma} \label{lem:SI-GradZeroEdgeToComposite}
	Let $I_{jk}\in\edgeSet$, and $g\in\gradZero{\ddom}{\lambda_{jk}}$.
	Then the function
	\begin{align*}
		\tilde{g} = \begin{cases} 0 & x\in\ddom\setminus I_{jk}, \\ g & x\in I_{jk}, \end{cases}
	\end{align*}
	is an element of $\gradZero{\ddom}{\lcompMes}$.
\end{lemma}
\begin{proof}
	By the characterisation proposition \ref{prop:3DGradZeroRotated} we can write $g = g_{jk}n_{jk}$ for some $g_{jk}\in\ltwo{\ddom}{\lambda_{jk}}$.
	Let $\phi_n$ be an approximating sequence for $g$, and notice that since $\grad\phi_n\rightarrow g$, we have that
	\begin{align*}
		\grad\phi_n\cdot n_{jk} \rightarrow g\cdot n_{jk} = g_{jk}, 
		&\implies
		\bracs{ \grad\phi_n\cdot n_{jk} }n_{jk} \rightarrow g_{jk}n_{jk} = g,
	\end{align*}
	in $\ltwo{\ddom}{\lambda_{jk}}^2$.
	Each of the functions $\grad\phi_n\cdot n_{jk}$ are smooth, and thus by lemma \ref{lem:SI-SmoothGradZero} we have that the functions
	\begin{align*}
		\tilde{g}_n &= \begin{cases} 0 & x\in\ddom\setminus I_{jk}, \\ \bracs{ \grad\phi_n\cdot n_{jk} }n_{jk} & x\in I_{jk}, \end{cases}
	\end{align*}
	are all elements of $\gradZero{\ddom}{\lcompMes}$.
	Furthermore, we can observe that
	\begin{align*}
		\integral{\ddom}{\abs{ \tilde{g}_n - \tilde{g} }^2}{\lcompMes}
		&= \integral{\ddom}{\abs{ \tilde{g}_n - \tilde{g} }^2}{\lambda_{jk}}
		= \integral{I_{jk}}{\abs{ \bracs{ \grad\phi_n\cdot n_{jk} }n_{jk} - g }^2}{\lambda_{jk}} \\
		&\rightarrow 0 \toInfty{n},
	\end{align*}
	and so $\tilde{g}_n\lconv{\ltwo{\ddom}{\lcompMes}^2}\tilde{g}$.
	Since $\gradZero{\ddom}{\lcompMes}$ is closed, and $\tilde{g}_n\in\gradZero{\ddom}{\lcompMes}$ for every $n\in\naturals$, we thus conclude that the limit $\tilde{g}\in\gradZero{\ddom}{\lcompMes}$, and we are done.
\end{proof}

Now that we can ``lift" (via lemma \ref{lem:SI-GradZeroEdgeToComposite}) gradients of zero on an edge up to a gradients of zero with respect to the composite measure $\lcompMes$, and that all gradients of zero with respect to $\lambda_2$ are zero, this effectively gives us our characterisation of $\gradZero{\ddom}{\lcompMes}$.
\begin{cory}[Characterisation of $\gradZero{\ddom}{\lcompMes}$.] \label{cory:SI-GradZeroEdgeChar}
	The following characterisations hold:
	\begin{enumerate}[(i)]
		\item Let $I_{jk}\in\edgeSet$, suppose that $g\in\ltwo{\ddom}{\lambda_{jk}}$ and $\tilde{g}\in\ltwo{\ddom}{\lcompMes}$ where
		\begin{align*}
			\tilde{g} = \begin{cases} 0 & x\in\ddom\setminus I_{jk}, \\ g & x\in I_{jk}. \end{cases}
		\end{align*}
		Then
		\begin{align*}
			\tilde{g}\in\gradZero{\ddom}{\lcompMes} \quad\Leftrightarrow\quad 
			& g\in\gradZero{\ddom}{\lambda_{jk}}.
		\end{align*}
		\item
		\begin{align*}
			\gradZero{\ddom}{\lcompMes} 
			&= \clbracs{ \tilde{g}\in\ltwo{\ddom}{\lcompMes} \setVert \exists g\in\gradZero{\ddom}{\lambda_{jk}} \text{ s.t. } \tilde{g} = \begin{cases} 0 & x\in\ddom\setminus I_{jk}, \\ g & x\in I_{jk}. \end{cases}}
		\end{align*}
	\end{enumerate}
\end{cory}
\begin{proof}
	\begin{enumerate}[(i)]
		\item ($\Rightarrow$) The right-directed implication holds since any approximating sequence for $\tilde{g}$ in $\ltwo{\ddom}{\lcompMes}$ also serves as an appropriate approximating sequence for $g$ in $\ltwo{\ddom}{\lambda_{jk}}$. \newline
		($\Leftarrow$) The left-directed implication is simply the result of lemma \ref{lem:SI-GradZeroEdgeToComposite}.
		\item The ``$\supset$" inclusion holds by part (i) above.
		For the ``$\subset$" inclusion, take $\tilde{g}\in\gradZero{\ddom}{\lcompMes}$.
		Then $\tilde{g}\in\gradZero{\ddom}{\lambda_2}$ and $\tilde{g}\in\gradZero{\ddom}{\lambda_{jk}}$ since any approximating sequence for $\tilde{g}$ in $\ltwo{\ddom}{\lcompMes}$ also serves as the approximating sequence in the other two spaces.
		However, $\tilde{g}\in\gradZero{\ddom}{\lambda_2}$ implies that $\tilde{g}=0$ $\lambda_2$-almost-everywhere,	and thus $\tilde{g}$ has the form
		\begin{align*}
			\tilde{g} = \begin{cases} 0 & x\in\ddom\setminus I_{jk}, \\ g & x\in I_{jk}, \end{cases}
		\end{align*}
		where $g = \tilde{g}$ $\lambda_{jk}$-almost-everywhere.
		Since $g = \tilde{g}\in\gradZero{\ddom}{\lambda_{jk}}$, have shown the desired inclusion.
	\end{enumerate}
\end{proof}

Now we look to demonstrate that the set $\gradZero{\ddom}{\compMes}$ is constructed from sums of edge functions belonging to $\gradZero{\ddom}{\lcompMes}$.
\begin{prop}[Relation between $\gradZero{\ddom}{\lcompMes}$ and $\gradZero{\ddom}{\compMes}$.] \label{prop:SI-GradZeroExtensionAndChar}
	Fix $I_{jk}\in\edgeSet$.
	\begin{enumerate}[(i)]
		\item Suppose that $g\in\gradZero{\ddom}{\lcompMes}$ with
		\begin{align*}
			g = \begin{cases} 0 & x\in\ddom\setminus I_{jk}, \\ g_{jk} & x\in I_{jk}, \end{cases}
		\end{align*}
		and $\supp(g_{jk})\subset I_{jk}^n$ (where $I_{jk}^n$ is as in \eqref{eq:ShortenedEdgeDef}).
		Then the function
		\begin{align*}
			\tilde{g}(x) &:= \begin{cases} 0 & x\in\ddom\setminus\graph, \\ 0 & x\in\graph\setminus I_{jk}, \\ g_{jk}(x) & x\in I_{jk}, \end{cases}
		\end{align*}
		is an element of $\gradZero{\ddom}{\compMes}$.
		\item Now only assume that $g\in\gradZero{\ddom}{\lcompMes}$ with
		\begin{align*}
			g = \begin{cases} 0 & x\in\ddom\setminus I_{jk}, \\ g_{jk} & x\in I_{jk}. \end{cases}
		\end{align*}
		Then the function
		\begin{align*}
			\tilde{g}(x) &:= \begin{cases} 0 & x\in\ddom\setminus\graph, \\ 0 & x\in\graph\setminus I_{jk}, \\ g_{jk}(x) & x\in I_{jk}, \end{cases}
		\end{align*}
		is an element of $\gradZero{\ddom}{\compMes}$.
		\item We have that
		\begin{align*}
			\gradZero{\ddom}{\compMes} &= \clbracs{ \tilde{g}\in\ltwo{\ddom}{\compMes} \setVert \tilde{g}^{(jk)}\in\gradZero{\ddom}{\lcompMes} \ \forall I_{jk}\in\edgeSet }.
		\end{align*}
		\item Let $\tilde{g}\in\ltwo{\ddom}{\compMes}^2$, where
	\begin{align*}
		\tilde{g} &= \begin{cases} g_{\lambda_2} & x\in\ddom\setminus\graph, \\ g_{\ddmes} & x\in\graph. \end{cases}
	\end{align*}
	Then
	\begin{align*}
		\tilde{g}\in\gradZero{\ddom}{\compMes}
		\quad\Leftrightarrow\quad &
		g_{\lambda_2} = 0 \text{ and } g_{\ddmes}\in\gradZero{\ddom}{\ddmes}.
	\end{align*}
	\end{enumerate}
\end{prop}
\begin{proof}
	\begin{enumerate}[(i)]
		\item Take an approximating sequence $\phi_l$ for $g$, and set $\psi_l = \chi_{jk}^n\phi_l\in\psmooth{\ddom}$ for each $l\in\naturals$, where $\chi_{jk}^n$ is as in \eqref{eq:SmoothChiDef}.
		We then have that $\grad\psi_l = \chi_{jk}^n\grad\phi_l + \phi_l\grad\chi_{jk}^n$, and immediately see that $\psi_l\lconv{\ltwo{\ddom}{\compMes}}0$ since
		\begin{align*}
			\integral{\ddom}{\abs{ \psi_l }^2}{\compMes}
			&\leq \integral{\ddom}{\abs{ \phi_l }^2}{\compMes}.
		\end{align*}
		Additionally, we have the estimates
		\begin{align*}
			\recip{2}\integral{\ddom}{\abs{ \grad\psi_l - \tilde{g} }^2}{\ddmes}
			&= \recip{2}\integral{I_{jk}}{\abs{ \chi_{jk}^n\grad\phi_l - g  + \phi_l\grad\chi_{jk}^n }^2}{\lambda_{jk}} \\
			&\leq \integral{I_{jk}}{\abs{ \chi_{jk}^n\grad\phi_l - g }^2}{\lambda_{jk}}
			+ \integral{I_{jk}}{\abs{ \phi_l\grad\chi_{jk}^n }^2}{\lambda_{jk}} \\
			&\leq \integral{I_{jk}}{\abs{ \grad\phi_l - g }^2}{\lambda_{jk}}
			+ \sup_{\ddom}\abs{\grad\chi_{jk}^n}^2 \integral{I_{jk}}{\abs{ \phi_l }^2}{\lambda_{jk}}, \\
			\recip{2}\integral{\ddom}{\abs{ \grad\psi_l - \tilde{g} }^2}{\lambda_2}
			&\leq \integral{\ddom}{\abs{ \chi_{jk}^n\grad\phi_l - \tilde{g} }^2}{\lambda_2}
			+ \sup_{\ddom}\abs{\grad\chi_{jk}^n}^2 \integral{\ddom}{\abs{ \phi_l }^2}{\lambda_2} \\
			&\leq \integral{\ddom}{\abs{ \grad\phi_l }^2}{\lambda_2}
			+ \sup_{\ddom}\abs{\grad\chi_{jk}^n}^2 \integral{\ddom}{\abs{ \phi_l }^2}{\lambda_2}.
		\end{align*}
		Therefore,
		\begin{align*}
			\recip{2}\integral{\ddom}{\abs{ \grad\psi_l - \tilde{g} }^2}{\compMes}
			&\leq \integral{\ddom}{\abs{ \grad\phi_l }^2}{\lambda_2}
			+ \integral{I_{jk}}{\abs{ \grad\phi_l - g }^2}{\lambda_{jk}} \\
			&\quad + \sup_{\ddom}\abs{\grad\chi_{jk}^n}^2 
			\bracs{ \integral{\ddom}{\abs{ \phi_l }^2}{\lambda_2} 	+ \integral{I_{jk}}{\abs{ \phi_l }^2}{\lambda_{jk}} } \\
			&= \integral{\ddom}{\abs{ \grad\phi_l - g }^2}{\lcompMes}
			+ \norm{\phi_l}_{\ltwo{\ddom}{\lcompMes}} \sup_{\ddom}\abs{\grad\chi_{jk}^n}^2 \\
			&\rightarrow 0 \toInfty{l},
		\end{align*}
		and we conclude that $\tilde{g}\in\gradZero{\ddom}{\compMes}$.
		\item The function $g_{jk}\in\gradZero{\ddom}{\lambda_{jk}}$ by corollary \ref{cory:SI-GradZeroEdgeChar}, and thus for any $n\in\naturals$ the function $\eta_n^j\eta_n^k g_{jk}\in\gradZero{\ddom}{\lambda_{jk}}$.
		By lemma \ref{lem:SI-GradZeroEdgeToComposite}, the functions
		\begin{align*}
			g_n := \begin{cases} 0 & x\in\ddom\setminus I_{jk}, \\ \eta_n^j\eta_n^k g_{jk} & x\in I_{jk} \end{cases}
		\end{align*}
		are elements of $\gradZero{\ddom}{\lcompMes}$ too.
		By part (i), we then have that
		\begin{align*}
			\tilde{g}_n := 	\begin{cases} 0 & x\in\ddom\setminus\graph, \\ 0 & x\in\graph\setminus I_{jk}, \\ \eta_n^j\eta_n^k g_{jk} & x\in I_{jk}, \end{cases}
		\end{align*}
		is an element of $\gradZero{\ddom}{\compMes}$, $\forall n\in\naturals$.
		As such, by closure of $\gradZero{\ddom}{\compMes}$, it suffices to show that the sequence $\tilde{g}_n$ converges in $\ltwo{\ddom}{\compMes}$ to $\tilde{g}$.
		Given the result of lemma \ref{lem:SI-SmoothFunctionsResults}(ii), we indeed have that $\tilde{g}_n\rightarrow\tilde{g}$ and thus $\tilde{g}\in\gradZero{\ddom}{\compMes}$.
		\item The ``$\subset$" inclusion is simply a result of any approximating sequence for $\tilde{g}$ also serving as an approximating sequence in $\ltwo{\ddom}{\lcompMes}$ for each edge $I_{jk}$. \newline
		The ``$\supset$" inclusion follows from part (ii) and linearity of the space $\gradZero{\ddom}{\compMes}$; each $\tilde{g}^{(jk)}\in\gradZero{\ddom}{\lcompMes}$ can be extended by zero (using part (ii)) to a function in $\gradZero{\ddom}{\compMes}$.
		Summing up these extended functions then yields the function $\tilde{g}$ again, which is an element of $\gradZero{\ddom}{\compMes}$ by linearity of the subspace.
		\item ($\Leftarrow$) The left-directed implication follows from proposition \ref{prop:3DGradZeroChar}, corollary \ref{cory:SI-GradZeroEdgeChar}(ii), and part (iii). \newline
	($\Rightarrow$) The right-directed implication is simply a result of any approximating sequence for $\tilde{g}$ also serving as an approximating sequence for $g_{\lambda_2}$ and $g_{\ddmes}$.
	This also implies that $g_{\lambda_2} = 0$ (being a gradient of zero with respect to the Lebesgue measure), which completes the proof.
	\end{enumerate}
\end{proof}

Having understood the set of gradients of zero, we are now able to deduce some important properties of the functions that live in $\tgradSob{\ddom}{\compMes}$.
The most direct result that follows from proposition \ref{prop:SI-GradZeroExtensionAndChar}(iv) is that the tangential gradient $\tgrad_{\compMes}$ is the usual weak derivative (with respect to the Lebesgue measure) in the bulk regions, and equal to the tangential derivative with respect to $\ddmes$ on $\graph$.
\begin{cory}
	Suppose that $u\in\tgradSob{\ddom}{\compMes}$
	Then 
	\begin{align*}
		\tgrad_{\compMes}u = \begin{cases} \grad u + \rmi\qm u & x\in\ddom\setminus\graph, \\ \tgrad_{\ddmes}u & x\in\graph, \end{cases}
	\end{align*}
	where $\grad u$ denotes the weak derivative of $u\in\gradSob{\ddom}{\lambda_2}$.
\end{cory}
\begin{proof}
	The requirement that $\tgrad_{\compMes}u$ be orthogonal to $\gradZero{\ddom}{\compMes}$ and theorem \ref{thm:SI-GradZeroChar} implies that $\tgrad_{\compMes}u = \tgrad_{\ddmes}u$ on $\graph$.
	For the bulk regions, we again notice that any approximating sequence $\phi_n$ for $u\in\tgradSob{\ddom}{\compMes}$ also serves as an approximating sequence in for $u\in\tgradSob{\ddom}{\lambda_2}$, and $\tgrad_{\lambda_2}u = \grad u + \rmi\qm u$.
	This implies that $\tgrad_{\compMes}u = \grad u + \rmi\qm u$ ($\lambda_2$-almost-everywhere) in $\ddom\setminus\graph$, completing the proof.
\end{proof}

Functions in $\tgrad{\ddom}{\compMes}$ also exhibit additional regularity when approaching the edges of $\graph$ from the bulk regions $\ddom_i$, analogously to how functions in $\ktgradSob{\ddom}{\ddmes}$ exhibit continuity on approach the the vertices.
This is not immediately obvious from the definition (by approximation of smooth functions) of $\tgradSob{\ddom}{\compMes}$; however now that we understand that our tangential gradients are essentially the familiar (weak) gradients with respect to $\lambda_2$ in the bulk and are equal to tangential gradients with respect to $\ddmes$ on $\graph$, we can establish this connection between the bulk and inclusions.
\begin{theorem} \label{thm:SI-SobFuncEdgeContinuity}
	Let $u\in\tgradSob{\ddom}{\compMes}$, and fix a bulk region $\ddom_i$.
	Take $I_{jk}\in\edgeSet$ such that $\partial\ddom_i\cap I_{jk}\neq\emptyset$.
	Denote the trace operator (that extends the classical trace of functions in $\smooth{\overline{\ddom}_i}\cap\gradSob{\ddom_i}{\lambda_2}$) on $\gradSob{\ddom_i}{\lambda_2}$ into $\ltwo{\partial\ddom_i}{S}$ by $\mathrm{Tr}_i$.
	Then $\mathrm{Tr}_i(u)$ exists and $\mathrm{Tr}_i(u) = u^{(jk)}$ on $I_{jk}$.
\end{theorem}
\begin{proof}
	Take an approximating sequence $\phi_n$ for $u\in\tgradSob{\ddom}{\compMes}$; clearly this sequence is also such that
	\begin{align*}
		\phi_n \lconv{\ltwo{\ddom_i}{\lambda_2}} u, &\qquad \grad\phi_n\lconv{\ddom_i}{\lambda_2}\grad u, \\
		\phi_n \lconv{\ltwo{\ddom}{\lambda_{jk}}} u, &\qquad \tgrad\phi_n\lconv{\ddom}{\lambda_{jk}}\ktgrad_{\lambda_{jk}} u.
	\end{align*}
	Furthermore, since $\phi_n\in\psmooth{\ddom}$ and $\overline{\ddom}_i\subset\ddom$ (since the boundary of $\ddom_i$ consists of a union of edges of $\graph\subset\ddom$), we have $\phi_n\in\smooth{\overline{\ddom}_i}$.
	The regions $\ddom_i$ are also Lipschitz so the trace operator
	\begin{align*}
		\mathrm{Tr}_i: \gradSob{\ddom_i}{\lambda_2} \rightarrow \ltwo{\partial\ddom_i}{S},
	\end{align*}
	is linear and continuous.
	Because $\phi_n\rightarrow u$ in $\gradSob{\ddom_i}{\lambda_2}$ and $\phi_n\in\smooth{\overline{\ddom}_i}\cap\gradSob{\ddom_i}{\lambda_2}$, we have that
	\begin{align*}
		u\vert_{\partial\ddom_i} := \mathrm{Tr}_i(u) = \lim_{n\rightarrow\infty}\phi_n\vert_{\partial\ddom_i},
	\end{align*}
	where the limit is taken in $\ltwo{\partial\ddom_i}{S}$.
	In particular, we have that $\phi_n\lconv{\ltwo{I_{jk}}{S}}u\vert_{\partial\ddom_i}$ since $I_{jk}\subset\partial\ddom_i$.
	However,
	\begin{align*}
		\norm{ \phi_n - u\vert_{\partial\ddom_i} }_{\ltwo{I_{jk}}{S}}
		&= \integral{I_{jk}}{\abs{ \phi_n - u\vert_{\partial\ddom_i} }^2}{S}
		= \int_0^{l_{jk}} \abs{ \phi_n - u\vert_{\partial\ddom_i} }^2 \ \md y \\
		&= \integral{\ddom}{\abs{ \phi_n - u\vert_{\partial\ddom_i} }^2}{\lambda_{jk}}
		= \norm{ \phi_n - u\vert_{\partial\ddom_i} }_{\ltwo{\ddom}{\lambda_{jk}}},
	\end{align*}
	and so $\phi_n\rightarrow u\vert_{\partial\ddom_i}$ in $\ltwo{\ddom}{\lambda_{jk}}$.
	But we know that $\phi_n\rightarrow u^{(jk)}$ in $\ltwo{\ddom}{\lambda_{jk}}$ too, so it must be the case that $u^{(jk)} = u\vert_{\partial\ddom_i}$, and we are done.
\end{proof}
Clearly theorem \ref{thm:SI-SobFuncEdgeContinuity} also implies that traces from any two bulk regions $\ddom_1$ and $\ddom_2$ onto a common inclusion (that is, a shared boundary) must coincide with each other, as they both coincide with the values of $u^{(jk)}$ on the shared edges.
Also worth noting is that $u$ (restricted to the graph $\graph$) is continuous at the vertices (see theorem \ref{thm:3DTangGradGraph}), and thus combined with the previous remark will be continuous in the vicinity of any graph edges too.
Much like how theorem \ref{thm:3DTangGradGraph} established continuity of a function at the vertices of a graph, theorem \ref{thm:SI-SobFuncEdgeContinuity} will play an important part in our reformulation of the problem \eqref{eq:SI-WeakWaveEqn}.

\end{subappendices}

%Create bibliography, starting on new page
\newpage
\bibliographystyle{acm}
\bibliography{./BibFiles/Thesis_MasterBib.bib}

\end{document}