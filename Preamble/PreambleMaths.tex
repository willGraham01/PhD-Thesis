%The idea behind this file is that it will be used to store all the maths-related macros that I concoct; so that I can import all the commands by \input{this file} in the preamble of any file that I want to use them in.
%This should make the top-level files look a lot cleaner, and the preamble much shorter!

\usepackage{amssymb}
\usepackage{amsmath}

%% EDITING FOR TYPO/ADDITIONS THAT ARE REQUIRED
\usepackage{xcolor}
\newcommand{\tstk}[1]{\textcolor{red}{\textbf{#1}} \newline}

%% THEOREMS AND LEMMAS, ETC using amsthm

\usepackage{amsthm}
\theoremstyle{definition}
\newtheorem{definition}{Definition}[section]
\theoremstyle{definition}
\newtheorem{problem}{Problem}[section]
\theoremstyle{plain}
\newtheorem{theorem}{Theorem}[section]
\theoremstyle{plain}
\newtheorem{lemma}[theorem]{Lemma}
\theoremstyle{plain}
\newtheorem{prop}[theorem]{Proposition}
\theoremstyle{plain}
\newtheorem{cory}[theorem]{Corollary}
\theoremstyle{definition}
\newtheorem{convention}[theorem]{Convention}
\theoremstyle{definition}
\newtheorem{assumption}[theorem]{Assumption}
\theoremstyle{definition}
\newtheorem{conjecture}[theorem]{Conjecture}

%% EQUATION FORMATING THROUGHOUT DOCUMENT

% allows equations in the same align environment to split over multiple pages.
\allowdisplaybreaks 

%% EQUATION FUNCTIONALITY - SHORTHAND, LARGE OPERATOR FORMATTING, ETC

% UNITS AND OPERATORS
\newcommand{\rmi}{\mathrm{i}}					% imaginary unit i (RoMan font i)
\newcommand{\e}{\mathrm{e}}						% Euler number e (Roman font e)
\newcommand{\md}{\mathrm{d}}					% differential d (Roman d)
\newcommand{\charFunc}[1]{\mathcal{I}_{#1}}	% characteristic function of a set
\newcommand{\sgn}{\mathrm{sgn}}				% sign function of an input

% MATRIX DISPLAY: adds extra functionality to pmatrix, vmatrix, bmatrix etc by allowing you to pass an optional argument in [FACTOR] to multiply the default spacing between elements by FACTOR
\makeatletter
\renewcommand*\env@matrix[1][\arraystretch]{%
  \edef\arraystretch{#1}%
  \hskip -\arraycolsep
  \let\@ifnextchar\new@ifnextchar
  \array{*\c@MaxMatrixCols c}
  }
\makeatother

% SHORTHAND
\newcommand{\recip}[1]{\frac{1}{#1}}								% writes 1/{#1}
\renewcommand{\vec}[1]{\mathbf{#1}}								% vectors are bold, no overhead arrow
\newcommand{\lconv}[1]{\overset{#1}{\longrightarrow}}				% convergence with #1 above arrow
\newcommand{\toInfty}[1]{ \ \text{as} \ #1 \rightarrow\infty}		% writes out "as #1 tends to infty"

% (standard) INTEGRAL AND DIFFERENTIAL OPERATORS
% integral display: \integral{domain}{integrand}{measure/integral variable}
\newcommand{\integral}[3]{\int_{#1}#2 \ \mathrm{d}#3}
% derivatives of various types
\newcommand{\diff}[2]{\dfrac{\mathrm{d}#1}{\mathrm{d}#2}}				% complete derivative d#1/d#2
\newcommand{\pdiff}[2]{\dfrac{\partial #1}{\partial #2}}				% partial derivative p#1/p#2
\newcommand{\ddiff}[2]{\dfrac{\mathrm{d}^2 #1}{\mathrm{d} {#2}^2}}	% 2nd complete deriv
\newcommand{\pddiff}[2]{\dfrac{\partial^2 #1}{\partial {#2}^2}}		% 2nd partial derivative
\newcommand{\grad}{\nabla}												% gradient operator
\newcommand{\laplacian}{\Delta}										% Laplacian
\newcommand{\curl}[1]{\mathrm{curl}_{#1}}								% curl with subscript #1
\newcommand{\dmap}{\Gamma_0}											% Dirichlet map
\newcommand{\nmap}{\Gamma_1}											% Neumann map
\newcommand{\dtn}{\mathcal{D}}											% Dirichlet-to-Neumann map
\newcommand{\gelfand}{\mathfrak{G}}									% Gelfand transform

% BRACKETS AND NORMS - encloses inputs in relevent bracers
\newcommand{\bracs}[1]{\left( #1 \right)}					% ( brackets )
\newcommand{\sqbracs}[1]{\left[ #1 \right]}				% [ square bracers ]
\newcommand{\clbracs}[1]{\left\{ #1 \right\}}				% { curly bracers }
\newcommand{\abs}[1]{\left\lvert #1 \right\rvert}			% | absolute value |
\newcommand{\norm}[1]{\lvert\lvert #1 \rvert\rvert}		% || norm ||
\newcommand{\setVert}{\ \middle\vert \ }					% vertical bar for the middle of sets
\newcommand{\ip}[2]{\left\langle #1 , #2 \right\rangle}	% < inner , product >

% STANDARD SETS
\newcommand{\naturals}{\mathbb{N}}			% natural numbers
\newcommand{\integers}{\mathbb{Z}}			% integers
\newcommand{\rationals}{\mathbb{Q}}		% rational numbers
\newcommand{\reals}{\mathbb{R}}			% real numbers
\newcommand{\complex}{\mathbb{C}}			% complex numbers

\newcommand{\smooth}[1]{C^{\infty}\bracs{#1}}				% smooth functions
\newcommand{\psmooth}[1]{C^{\infty}_{\#}\bracs{#1}}		% periodic smooth functions
\newcommand{\csmooth}[1]{C^{\infty}_{0}\bracs{#1}}			% compactly supported smooth functions
\newcommand{\ltwo}[2]{L^{2}\bracs{#1,\md #2}}				% L^2(domain, measure) space
\newcommand{\pltwo}[2]{L^2_{\#}\bracs{#1,\md #2}}			% L^2(domain, measure) periodic space
\newcommand{\htwo}[1]{H^2_\mathrm{grad}\bracs{#1}}			% H^2_{gradients}(domain)
\newcommand{\supp}{\mathrm{supp}}							% support of a function
\newcommand{\dom}{\mathrm{dom}}							% domain of an operator

%% MACROS FOR THESIS NOTATION

%creates the closed interval from 0 to the length of the input #1, denoted by absolute value
\newcommand{\interval}[1]{\sqbracs{0,\abs{#1}}}

% SYMBOLS, CONSTANTS, COMMON VARIABLES
\newcommand{\eps}{\varepsilon}					% pretty epsilons to distingush from physical constants
\newcommand{\epsFS}{\epsilon_0}				% permittivity of free space
\newcommand{\muFS}{\mu_0}						% permeability of free space
\newcommand{\wavenumber}{\kappa}				% fourier variable or wavenumber
\newcommand{\qm}{\theta}						% quasi-momentum parameter
\newcommand{\kt}{\bracs{\wavenumber, \qm}}		% (\wavenumber, \qm) pair
\newcommand{\effFreq}{\Lambda}					% Effective frequency sqrt(w^2-\wavenumber^2)

% DOMAINS + MEASURES
\newcommand{\dddom}{\widetilde{\Omega}}		% 3D domain notation
\newcommand{\ddom}{\Omega}						% 2D domain notation

\newcommand{\massMes}{\nu}							% vertex point mass supporting measure
\newcommand{\ddmes}{\mu}							% 2D sing. measure
\newcommand{\dddmes}{\widetilde{\ddmes}}			% 2D sing. measure + point masses
\newcommand{\compMes}{\lambda_2^{\ddmes}}			% 2D composite Leb + sing measure
\newcommand{\lcompMes}{\lambda_2^{\lambda_{jk}}}	% 2D composite Leb + sing measure
\newcommand{\ccompMes}{\lamda_2^{\dddmes}}			% 2D composite Leb + sing + point mass measure

% GRAPHS
\newcommand{\graph}{\mathbb{G}}					% graph variable
\newcommand{\vertSet}{\mathcal{V}}					% set of vertices rather than big V
\newcommand{\edgeSet}{\mathcal{E}}					% set of edges rather than large E, \graph = (V,E)
\newcommand{\dgmap}{\dmap^{\graph}}				% Dirichlet map with graph superscript
\newcommand{\ngmap}{\nmap^{\graph}}				% Neumann map with graph superscript
\newcommand{\ag}{\mathcal{A}_\graph}				%caligraphic A with graph script for distinction

% GRAPH CONNECTIONS
\newcommand{\conLeft}{\stackrel{\rightarrow}{\smash{\sim}\rule{0pt}{0.4ex}}}	% j connects to k, j left
\newcommand{\conRight}{\stackrel{\leftarrow}{\smash{\sim}\rule{0pt}{0.4ex}}}	% j connects to k, j right
\newcommand{\con}{\sim} 														% j connects to k

% SHIFTED GRADIENT OPERATORS
\newcommand{\ograd}{\grad^{(0)}}								% grad operator with 0 superscript
\newcommand{\tgrad}{\nabla^{\qm}}								% grad operator with qm superscript
\newcommand{\kgrad}{\grad^{(\wavenumber)}}						% grad with wavenumber superscript
\newcommand{\ktgrad}{\grad^{\kt}}								% grad with wavenumber, qm superscript
\newcommand{\kcurl}[1]{\mathrm{curl}_{#1}^{(\wavenumber)}}	 	% k-curl with measure subscript #1
\newcommand{\ktcurl}[1]{\mathrm{curl}_{#1}^{\kt}}				% k,theta-curl with measure subscript #1

% GRAD/CURL/DIV OF ZERO SETS: format is \command{domain}{measure}
\newcommand{\gradZero}[2]{\mathcal{G}_{ #1, \mathrm{d}#2}\bracs{0}}					% gradients of zero
\newcommand{\kgradZero}[2]{\mathcal{G}_{ #1, \mathrm{d}#2}^{(\wavenumber)}\bracs{0}}	% k-gradients of zero
\newcommand{\curlZero}[2]{\mathcal{C}_{ #1, \mathrm{d}#2}\bracs{0}}					% curls of zero
\newcommand{\kcurlZero}[2]{\mathcal{C}_{ #1, \mathrm{d}#2}^{(\wavenumber)}\bracs{0}}	% k-curls of zero
\newcommand{\divZero}[2]{\mathcal{D}_{#1,\mathrm{d}#2}\bracs{0}} 						% divs of zero

% SOBOLEV SPACES: format is \command{domain}{measure}
\newcommand{\gradSob}[2]{H^1_\mathrm{grad}\bracs{#1, \mathrm{d}#2}}			% gradient SS
\newcommand{\tgradSob}[2]{H^1_{\qm, \mathrm{grad}}\bracs{#1, \mathrm{d}#2}} 	% t-gradient SS
\newcommand{\gradSobQM}[2]{\tgradSob{#1}{#2}}									% t-gradient SS, legacy

\newcommand{\ktgradSob}[2]{H^1_{\wavenumber,\qm,\mathrm{grad}}\bracs{#1, \mathrm{d}#2}}	% kt-grad SS
\newcommand{\curlSob}[2]{H^1_\mathrm{curl}\bracs{#1, \mathrm{d}#2}}						% curl SS
\newcommand{\tcurlSob}[2]{H^1_{\qm, \mathrm{curl}}\bracs{#1, \mathrm{d}#2}}				% t-curl SS
\newcommand{\kcurlSob}[2]{H^1_{\wavenumber,\mathrm{curl}}\bracs{#1, \mathrm{d}#2}}		% k-curl SS
\newcommand{\ktcurlSob}[2]{H^1_{\wavenumber,\qm,\mathrm{curl}}\bracs{#1, \mathrm{d}#2}}	% kt-curl SS
\newcommand{\ktdivSob}[2]{H^1_{\qm,\wavenumber,\mathrm{div}}\bracs{#1,\mathrm{d}#2}}		% kt-div SS
%\wavenumber,\qm-curl, divergence-free Sobolev space
\newcommand{\ktcurlSobDivFree}[2]{\mathcal{H}^{\kt}\bracs{#1, \mathrm{d}#2}}	
				
%H^2(domain, measure), only really used for Leb. measure
\newcommand{\gradgradSob}[2]{H^2_\mathrm{grad}\bracs{#1, \mathrm{d}#2}} 