\section{Chapter Introduction} \label{sec:ScalarEqnChapterIntro}
This chapter aims to provide a concrete introduction to variational problems on singular structures, and their link to the quantum graph problems that arise in the asymptotic limit of thin structures as the thickness shrinks to zero.
We will concern ourselves with the study of the acoustic (or wave) equation on a singular structure, and the corresponding non-classical Sobolev spaces of functions possessing gradients with respect to the measure $\dddmes$.
By understanding this function space we will be able to derive a quantum graph problem, or more precisely a problem of generalised resolvent type, from our original variational problem.
This will solidify the link between our variational problems and the limits of thin structure problems as discussed in section \ref{ssec:Intro-ThinStructures}, with our non-classical Sobolev spaces forming the natural ``extended" spaces.
Finally, we will turn our attention to the use of the $M$-matrix to analyse the spectrum of the resulting quantum graph problems, and provide some examples for specific singular structure geometries.

We now precisely formulate the problem that we wish to consider. \tstk{redo this by defining the operator $-\laplacian_{\dddmes,\qm}$ first, then saying it relates to the full space problem}
Let $\graph$ be the period graph of a (periodic) metric graph embedded into $\reals^2$, with unit cell $\ddom\subset\reals^2$ (see section \ref{sec:TP-OurProblems}).
In this chapter, we will concern ourselves with the analysis of the acoustic equation\footnote{In order to avoid sounding monotonous, we will also refer to \eqref{eq:SingularScalarWaveEqnWholeSpace} as the wave equation.}
\begin{align} \label{eq:SingularScalarWaveEqnWholeSpace}
	-\grad_{\dddmes}^2 u = \omega^2 u, \quad\text{in } \reals^2,
\end{align}
which we transform into a family of problems on the unit cell $\ddom$ by means of a Gelfand transform, providing us with
\begin{align} \label{eq:SingularScalarWaveEqn}
	-\bracs{\tgrad_{\dddmes}}^2 u = \omega^2 u, \quad\text{in } \ddom,
\end{align}
for each $\qm$ in the dual cell of $\ddom$.
Of course, we interpret \eqref{eq:SingularScalarWaveEqn} in a variational (or weak) sense, as the problem of finding $u\in\ktgradSob{\ddom}{\dddmes}, \omega^2>0$ such that
\begin{align*}
	\integral{\ddom}{ \tgrad_{\dddmes}u\cdot\overline{\tgrad_{\dddmes}\phi} }{\dddmes} 
	&= \omega^2\integral{\ddom}{ u\overline{\phi} }{\dddmes}, \quad\forall\phi\in\smooth{\ddom}.
\end{align*}
We will demonstrate that solutions to \eqref{eq:SingularScalarWaveEqn} satisfy the quantum graph problem
\begin{subequations} \label{eq:SingularWaveEqnQGProblem}
	\begin{align}
		-\bracs{\diff{}{y} + \rmi\qm_{jk}}^2 u^{(jk)} &= \omega^2 u^{(jk)}, \quad &y\in\interval{I_{jk}}, \ \forall I_{jk}\in\edgeSet, \label{eq:SingularWaveEqnQGProblem-1} \\
		u \text{ is continuous at } & v_j, \quad &\forall v_j\in\vertSet, \label{eq:SingularWaveEqnQGProblem-2} \\
		\sum_{j\con k}\bracs{\pdiff{}{n} + \rmi\qm_{jk}}u^{(jk)}\bracs{v_j} &= \omega^2\alpha_j u\bracs{v_j}, \quad &\forall v_j\in\vertSet. \label{eq:SingularWaveEqnQGProblem-3}
	\end{align}
\end{subequations}
The $\qm_{jk}$ are rotations of the quasi-momentum $\qm$, and can be computed given the orientation of the edge $I_{jk}\in\edgeSet$.
It should be noted that problems like \eqref{eq:SingularWaveEqnQGProblem} belong to the class of problems with generalised resolvents, since the spectral parameter $\omega^2$ appears in the boundary condition \eqref{eq:SingularWaveEqnQGProblem-3}, which we will elaborate on later \tstk{be sure to elaborate on this later! Also the refs!}.
Our interest lies in those problems for which $\Re\bracs{\alpha_j}>0$ (physically, this ensures causality in our problem) and $\Im\bracs{\alpha_j} = 0$.
Of course there is nothing that mathematically forbids the coupling constants $\alpha_j$ being complex-valued, but the restrictions ensure we retain a link to physical applications. 
Indeed, taking $\Im\bracs{\alpha_j} > 0$ (respectively $\Im\bracs{\alpha_j} < 0$) would correspond to placing an energy sink (respectively source) at the vertex $v_j$, whist $\Re\bracs{\alpha_j}<0$ would ``reverse causality" in the solution --- physically speaking, the problem would correspond to a system in which the past depends on the future.

The analysis of the Sobolev space $\gradSobQM{\ddom}{\dddmes}$, and in particular understanding the tangential gradient $\tgrad_{\dddmes}u$, is central to the derivation of \eqref{eq:SingularWaveEqnQGProblem}.
However, the analysis involved is lengthy and so can be found at the end of the chapter in section \ref{sec:3DGradSobSpaces}.
We will provide a short overview of this analysis in section \ref{sec:ScalarDerivation}, before going through the derivation of \eqref{eq:SingularWaveEqnQGProblem} from \eqref{eq:SingularScalarWaveEqn}.
Section \ref{sec:ScalarDiscussion} then provides an explicit expression for the $M$-matrix of \eqref{eq:SingularWaveEqnQGProblem}, and opens a discussion into the methodology and considerations for using it to analyse the spectrum of \eqref{eq:SingularWaveEqnQGProblem}.
We then employ this approach with a selection of examples in section \ref{sec:ScalarExamples}, before concluding.
By the end of this chapter we will have a solid basis from which to develop our framework for handling ``derivatives" on singular-structures, naturally bringing us towards the analysis of the curl-of-the-curl equation \tstk{ref} and an investigation into the first-order Maxwell system.
