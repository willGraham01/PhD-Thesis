\section{Concluding Remarks} \label{sec:Scalar-Conc}
The work of this chapter provides us with a concrete foundation upon which to build, as we turn our attention towards extensions to systems of vector equations on singular structures (chapter \ref{ch:CurlCurl}), taking us closer to an analogue of the Maxwell system.
The analysis of section \ref{sec:3DGradSobSpaces} provides us with an understanding, both intuitively and rigorously, of the derivative-like objects that our variational problem requires us to work with.
We have seen that gradients of zero emerge due to our singular measures inability to determine the behaviour of functions on $\ddom$ outside $\graph$, and that the tangential gradient corresponds to the 1 dimensional analogue of the gradient --- the derivative --- directed along each of the edges of $\graph$.
This is an expected behaviour; the singular measure $\dddmes$ only respects 1 dimensional lengths along $\graph$, effectively throwing away the information in the direction normal to the edges of $\graph$, and as a result the 2 dimensional gradient is replace with the 1 dimensional tangential gradient.
Whilst for gradients this one-dimensional analogue is fairly easy to accept intuitively, note that there is no such analogue for the curl (or divergence) operator acting on a vector field.
However, the analysis of $\ktgradSob{\ddom}{\dddmes}$ has also provided us with the method to identify what the analogous tangential curls (and notions of divergence) should be, which will be the focus of \tstk{curl analysis section}.
Our knowledge of $\ktgradSob{\ddom}{\dddmes}$ will also prove valuable in chapter \ref{ch:SingInc}, when we reintroduce the ``background" material that surrounds the graph $\graph$.

Section \ref{sec:ScalarDerivation} establishes the link between the singular structure problem \eqref{eq:SingularScalarWaveEqn} and the class of problems that emerge from thin-structure problems in the limit that the structure becomes singular, with the coupling constants $\alpha_j$ taking on the role of the ratio of vertex-to-edge volumes (see section \ref{ssec:Intro-ThinStructures}).
We also observed how the space $\tgradSob{\ddom}{\dddmes}$ allows us to construct these ``limiting" problems as the realisation of a self-adjoint operator ($-\laplacian_{\dddmes}^{\qm}$), defined through a bilinear form motivated by analogy with the ``weak formulation" of the classical acoustic equation.
Maintaining this analogy will be the reason behind the variational problems we will later consider in chapters \ref{ch:CurlCurl} and \ref{ch:SingInc}.
Our derivation in section \ref{sec:ScalarDerivation} also allows us to use the theory surrounding the $M$-matrix for explicitly determining the spectrum of $-\laplacian_{\dddmes}^{\qm}$, as discussed in section \ref{sec:ScalarDiscussion}.
It is worth reiterating that a formal analysis in the use of the $M$-matrix for the solution of problems bearing generalised resolvents has yet to be carried out (see the introduction to this chapter), although the expected procedure and conclusions are apparent.
Here we provide an explicit expression for the $M$-matrix of \eqref{eq:SingularWaveEqnQGProblem} on any graph topology, and cover a number of considerations one might wish to take on board when using the $M$-matrix to analyse the spectrum.
We built on our discussion with the examples of section \ref{sec:ScalarExamples}; including how one can also gain access to the eigenfunctions and (integrated) density of states (section \ref{ssec:ExampleCrossInPlane}), the necessity of artificial vertices (section \ref{ssec:Example1DLoop}), and how potential freedoms in the choice of embedding for a quantum graph do not affect the resulting spectrum (section \ref{ssec:EmbeddingDependentExample}).

Before we move on to considering systems of vector equations (chapter \ref{ch:CurlCurl} and composite domains (chapter \ref{ch:SingInc}), we highlight the possibility of extending the analysis carried out thus far to embedded graphs whose edges are not assumed to be straight line segments.

\subsection{Extensions: Curved Edges} \label{ssec:CurvedEdges}
Our analysis has been, and in the following chapters will continue to be, carried out within the context of the standing assumptions \ref{ass:MeasTheoryProblemSetup} --- this includes the assumption that the edges of our singular structure are straight lines.
Of course, the cross sections of physical photonic crystals are not limited to unions of straight struts, nor are the applications of quantum graphs to \tstk{introduction context, quantum wires, chemical reactions, etc}.
With our analysis of $\gradZero{\ddom}{\dddmes}$ and $\ktgradSob{\ddom}{\dddmes}$ complete, we can predict through analogy the changes that would arise if this restriction was dropped, to help motivate any future work in this direction.

To this end, let us now suppose that the edges $I_{jk}$ of our (period) graph $\graph$ are (continuous) curves in $\ddom$.
We know that, when our graph consists of straight edges, that tangential gradients are directed along the edges $I_{jk}$ and gradients of perpendicular to the edges.
With the change from straight edges to curved ones, nothing in this geometric understanding changes --- we should still expect tangential gradients to point ``along" the edges $I_{jk}$, for example.
Indeed, the (now curved) edges $I_{jk}$ are still related to the intervals $\interval{l_{jk}}$, it is simply the case now that the vector $e_{jk}$ is no longer constant along $I_{jk}$.
However we also remark that the vector $e_{jk}$ being no longer constant means that the quasi-momentum-related terms $\qm_{jk}$ are now \emph{also} non constant.
As such, we should expect that
\begin{align*}
	\tgrad_{\lambda_{jk}}u(x) = \bracs{ u_{jk}'(y) + \rmi\qm_{jk}(y) u_{jk}(y)}e_{jk}(y),
	\qquad x = r_{jk}(y),
\end{align*}
where $r_{jk}:\interval{l_{jk}}\rightarrow I_{jk}$ is the (measure-preserving) parametrisation of the curve $I_{jk}$, and $e_{jk}(y) = r_{jk}'(y)$ is the direction tangential to the edge $I_{jk}$ at the point $x$, and $\qm_{jk}(y)=\qm\cdot e_{jk}(y)$.
We make these predictions not only because of the geometric interpretations we now have for tangential gradients (and gradients of zero), but because the arguments of section \ref{sec:3DGradSobSpaces} do not depend on our requirement that $I_{jk}$ be straight, \emph{except} when we come to identify the explicit form for gradients of zero and tangential gradients on our edges.
Of course, there are likely to be some technicalities to overcome, for example the smoothness of the curves $I_{jk}$ might affect how one goes about constructing smooth approximating sequences.

Further to these expectations, one would expect to obtain a system of the form
\begin{subequations}
	\begin{align*}
		-\bracs{\diff{}{y} + \rmi\qm_{jk}(y)}^2 u^{(jk)} &= \omega^2 u^{(jk)}, \quad &y\in\interval{l_{jk}}, \ \forall I_{jk}\in\edgeSet, \\
		u \text{ is continuous at } & v_j, \quad &\forall v_j\in\vertSet,  \\
		\sum_{j\con k}\bracs{\pdiff{}{n} + \rmi\qm_{jk}(v_j)}u^{(jk)}\bracs{v_j} &= \omega^2\alpha_j u\bracs{v_j}, \quad &\forall v_j\in\vertSet,
	\end{align*}
\end{subequations}
in place of the system \eqref{eq:SingularWaveEqnQGProblem}).
This new system is still a quantum graph problem, or more precisely \emph{interpretable} as a quantum graph problem, however we now have co-ordinate varying coefficients (through the $\qm_{jk}(y)$) in our edge ODEs.
Operators of this kind have been studied before in the context of differential operators on non-singular structures in $\reals^2$, see for example \cite{shterenberg2007homogenization}. \tstk{more examples? Also check BK - does their book cover coordinate varying coefficients in the edge ODEs? If no examples in graph context, this is something very new that someone could do!}
