\section{Derivation of quantum graph problem} \label{sec:ScalarDerivation}
In this section we provide an overview of a system of the form \eqref{eq:SingularWaveEqnQGProblem} is obtained from \eqref{eq:SingularScalarWaveEqn}, which will setup our discussion revolving around the methods we employ for solving \eqref{eq:SingularWaveEqnQGProblem} in section \ref{sec:ScalarDiscussion}.

Precise definition and analysis of the ``Sobolev spaces" used here can be found in section \ref{sec:3DGradSobSpaces}, although we provide a short intuitive idea of the object $\tgrad_{\ddmes}u$ here.
The central idea behind understanding $\tgrad_{\ddmes}u$ is that the singular measure $\ddmes$ only supports the edges of $\graph$, and so cannot ``see" any changes in the function $u$ ``across" (in the direction perpendicular to) the edge $I_{jk}$.
So at any point $x\in I_{jk}$, the ``gradient" $\tgrad_{\ddmes}u(x)$ encapsulates the rate of change of the function $u$ at $x\in I_{jk}$ \emph{only} in the direction along $I_{jk}$.
As a result, it is not inaccurate to think of $\tgrad_{\ddmes}u(x) = \bracs{u^{(jk)}}'(x)e_{jk}$ (up to an appropriate shift due to the presence of the quasi-momentum $\qm$) for $x\in I_{jk}$, where $\bracs{u^{(jk)}}' = \pdiff{u^{(jk)}}{e_{jk}}$.
This also provides us with an intuitive understanding of how $\tgrad_{\ddmes}u$ behaves on each edge of the graph $\graph$, which is crucial for deriving the set of ``edge ODEs" \eqref{eq:SingularWaveEqnQGProblem} and providing a meaning to the $\diff{}{x}$ operator that appears in those equations.
As discussed in section \ref{ssec:FunctionSpaces}, the coupling constants attached to the vertices of the graph as well as the connectivity of the graph itself then dictate that these ``edge-wise" components $u^{(jk)}$ and $\bracs{u^{(jk)}}'$ adhere to certain matching conditions at the vertices.
The functions $u\in\gradSobQM{\ddom}{\dddmes}$ and their gradients $\tgrad_{\dddmes}u$ can be thought of as possessing the following properties (precise statements can be found in section \ref{sec:3DGradSobSpaces}):
\begin{enumerate}[(a)]
	\item The function $u$ is continuous at all vertices $v_j\in\vertSet$.
	\item On each edge $I_{jk}\in\edgeSet$; $\tgrad_{\dddmes}u = \tgrad_{\lambda_{jk}}u$, and $\tgrad_{\lambda_{jk}}u = \bracs{\bracs{u^{(jk)}}' + \rmi\qm_{jk}u^{(jk)}}e_{jk}$.
	The function $\bracs{u^{(jk)}}'$ being the derivative (in the $H^1$-sense) of the function $u^{(jk)}\circ r_{jk}$.
	\item At each vertex $v_j\in\vertSet$, we have $\tgrad_{\dddmes}u=0$, however $\lim_{x\rightarrow v_j}\tgrad_{\lambda_{jk}}u$ need not be zero.
\end{enumerate}

We can now provide a conceptual argument for how a system of the form \eqref{eq:SingularWaveEqnQGProblem} arises from \eqref{eq:SingularScalarWaveEqn}.
A function $u\in\gradSobQM{\ddom}{\dddmes}$ is said to be a solution to \eqref{eq:SingularScalarWaveEqn} if
\begin{align} \label{eq:SingularWaveEqnWeakForm}
	\integral{\ddom}{\tgrad_{\dddmes}u\cdot\overline{\tgrad_{\dddmes}\phi}}{\dddmes} &= \omega^2\integral{\ddom}{u\overline{\phi}}{\dddmes}, \quad\forall \phi\in\smooth{\ddom}.
\end{align}
We first note that whenever the equality in \eqref{eq:SingularWaveEqnWeakForm} holds for all smooth functions $\phi$, it holds in particular when $\phi$ has support that intersects the interior of a single edge $I_{jk}\in\edgeSet$ and no other parts of $\graph$.
Combined with the fact that $\dddmes$ is a sum of the edge measures and point masses at the vertices, and that $\tgrad_{\dddmes}u=\tgrad_{\lambda_{jk}}u$ on the edge $I_{jk}$, equation \eqref{eq:SingularWaveEqnWeakForm} implies
\begin{align*}
	0 &= \integral{\ddom}{ \bracs{\tgrad_\ddmes u \cdot \overline{\tgrad\phi} - \omega^2 u\overline{\phi}} }{\ddmes}
	= \integral{I_{jk}}{ \bracs{\tgrad_{\lambda_{jk}}u \cdot \overline{\tgrad\phi} - \omega^2 u^{(jk)}\overline{\phi}} }{\lambda_{jk}} \\
	&= \integral{I_{jk}}{ \clbracs{ \bracs{\bracs{u^{(jk)}}' + \rmi\qm_{jk} u^{(jk)}}\bracs{\overline{\phi}' - \rmi\qm_{jk} \overline{\phi} } - \omega^2 u^{(jk)}\overline{\phi} } }{\lambda_{jk}}.
\end{align*}
Now using the change of variables $r_{jk}$ and denoting $\tilde{u}^{(jk)} = u^{(jk)} \circ r_{jk}$ and $\varphi = \phi\circ r_{jk}$, we arrive at
\begin{align*}
	0 &= \int_{0}^{\abs{I_{jk}}} \bracs{\bracs{\tilde{u}^{(jk)}}' + \rmi\qm_{jk} \tilde{u}^{(jk)}}\bracs{\overline{\varphi}' - \rmi\qm_{jk} \overline{\varphi} } - \omega^2 \tilde{u}^{(jk)}\overline{\varphi} \ \md y. \\
	\implies
	\int_{0}^{\abs{I_{jk}}} \bracs{\tilde{u}^{(jk)}}'\overline{\varphi}' \ \md y &=
	\int_{0}^{\abs{I_{jk}}} \clbracs{ \omega^2\tilde{u}^{(jk)} + 2\rmi\qm_{jk}\bracs{\tilde{u}^{(jk)}}' + \bracs{\rmi\qm_{jk}}^2\tilde{u}^{(jk)} } \overline{\varphi} \ \md y.
\end{align*}
This holds for all smooth $\varphi$ with support contained in the interior of $\interval{I_{jk}}$, and as such we can deduce that $\tilde{u}^{(jk)}$ is twice (weakly) differentiable, and obtain the (strong) equation
\begin{align*}
	-\bracs{\diff{}{x} + \rmi\qm_{jk}}^2 \tilde{u}^{(jk)} &= \omega^2 \tilde{u}^{(jk)}, \quad x\in\interval{I_{jk}}.
\end{align*}

Now we turn our attention to the derivation of the vertex conditions.
Fix a vertex $v_j\in \vertSet$, and consider functions $\phi\in\smooth{\ddom}$ whose support intersects $\graph$ in a neighbourhood of $v_j$ that only contains edges which connect to $v_j$ (which can be, for example, a ball of sufficiently small radius centred on $v_j$).
Using the change of variables $r_{jk}$ on each edge and writing $\tilde{u}^{(jk)} = u^{(jk)} \circ r_{jk}$, $\varphi_{jk} = \phi\circ r_{jk}$ for each $k\con j$, we can work from \eqref{eq:SingularWaveEqnWeakForm} to obtain
\begin{align*}
	0 &= \sum_{k: \ k\con j} \integral{I_{jk}}{ \bracs{ \tgrad_\ddmes u \cdot \overline{\tgrad\phi} - \omega^2 u\overline{\phi} } }{\lambda_{jk}} 
	+ \integral{\ddom}{ \bracs{ \tgrad_{\dddmes}u\cdot\overline{\tgrad_{\dddmes}\phi}-\omega^2 u\overline{\phi} } }{\nu} \\
	&= \sum_{k: \ k\con j} \int_{0}^{\abs{I_{jk}}} \clbracs{ \bracs{\bracs{\tilde{u}^{(jk)}}' + \rmi\qm_{jk} \tilde{u}^{(jk)}}\bracs{\overline{\varphi}' - \rmi\qm_{jk} \overline{\varphi} } - \omega^2 \tilde{u}^{(jk)}\overline{\varphi} } \ \md y \\
	&\qquad + \alpha_j\left.\bracs{ \tgrad_{\dddmes}u\cdot\overline{\tgrad_{\dddmes}\phi}-\omega^2 u\overline{\phi} }\right\vert_{v_j} \\
	&= \sum_{k: \ k\con j} \int_{0}^{\abs{I_{jk}}} \clbracs{ \bracs{\bracs{\tilde{u}^{(jk)}}' + \rmi\qm_{jk} \tilde{u}^{(jk)}}\bracs{\overline{\varphi}' - \rmi\qm_{jk} \overline{\varphi} } - \omega^2 \tilde{u}^{(jk)}\overline{\varphi} } \ \md y
	 - \alpha_j \omega^2 u\bracs{v_j}\overline{\phi}\bracs{v_j}.
\end{align*}
Here we have used the fact that $\tgrad_{\dddmes}u\bracs{v_j}=0$ (see section \ref{sec:3DGradSobSpaces}).
Given that (from before) $u$ is twice differentiable on each $I_{jk}$, it follows that
\begin{align*}
	\alpha_j\omega^2 u\bracs{v_j}\overline{\phi}\bracs{v_j} 
	&= - \sum_{k: \ k\con j} \int_{0}^{\abs{I_{jk}}} \bracs{ \bracs{\diff{}{x} + \rmi\qm_{jk}}^2 \tilde{u}^{(jk)} +\omega^2 \tilde{u}^{(jk)} }\overline{\varphi} \ \md x \\
	&\qquad + \sum_{k: \ k\con j}\overline{\varphi}\bracs{v_j}\bracs{\pdiff{}{n} + \rmi\qm_{jk}}\tilde{u}^{(jk)}\bracs{v_j} \\
	&= \overline{\varphi}\bracs{v_j}\sum_{k: \ k\con j}\bracs{\pdiff{}{n} + \rmi\qm_{jk}}\tilde{u}^{(jk)}\bracs{v_j}. \labelthis\label{eq:DerivationVertexConditionWeak}
\end{align*}
Given that \eqref{eq:DerivationVertexConditionWeak} holds for every smooth $\varphi$, and that $\overline{\varphi}\bracs{v_j}=\overline{\phi}\bracs{v_j}$, we arrive at the condition
\begin{align*}
	\alpha_j\omega^2 u\bracs{v_j} &= \sum_{j\con k}\bracs{\pdiff{}{n} + \rmi\qm_{jk}}\tilde{u}^{(jk)}\bracs{v_j}, \quad \forall v_j \in \vertSet.
\end{align*}
Repeating the argument for each $v_j\in \vertSet$ then provides us with a condition of this form at each vertex.
One should note the presence of $\omega^2$ in this equation, so this is not a standard Robin condition on the derivatives of the edge-wise components of $u$, but rather indicates that our problem belongs to the class of problems with generalised resolvents, as mentioned in section \ref{ssec:DiffOpsOnGraphs}.
The result of theorem \ref{thm:dddmesTangGradImplication} tells us that functions $u\in\gradSobQM{\ddom}{\dddmes}$ are also continuous at each vertex $v_j$, and thus the following problem (precisely \eqref{eq:SingularWaveEqnQGProblem}) has been derived:
\begin{subequations}
	\begin{align}
		-\bracs{\diff{}{y} + \rmi\qm_{jk}}^2 u^{(jk)} &= \omega^2 u^{(jk)}, \quad &y\in\interval{I_{jk}}, \ \forall I_{jk}\in\edgeSet, \\
		u \text{ is continuous at } & v_j, \quad &\forall v_j\in\vertSet,  \\
		\sum_{j\con k}\bracs{\pdiff{}{n} + \rmi\qm_{jk}}u^{(jk)}\bracs{v_j} &= \omega^2\alpha_j u\bracs{v_j}, \quad &\forall v_j\in\vertSet,
	\end{align}
\end{subequations}
where we henceforth drop the overhead tilde notation and simply write $u^{(jk)}$ for brevity (appealing to the obvious association between $u^{(jk)}$ and $\tilde{u}^{(jk)}$).
Solving for the eigenvalues $\omega^2$ will net us the eigenvalues of our original problem \eqref{eq:SingularScalarWaveEqn}, and taking the union of the eigenvalues over $\qm$ will provide the spectrum of \eqref{eq:SingularScalarWaveEqnWholeSpace}.
As will be made clear in the discussion that follows, the quantum graph problem \eqref{eq:SingularWaveEqnQGProblem} is much easier to handle both analytically and numerically thanks to the utility of the $M$-matrix.