\subsection{Geometric Interpretation} \label{ssec:3DGradGeometric}
The intuition behind the form of the tangential gradients (and gradients of zero) with respect to the various measures $\lambda_{jk}, \ddmes, \nu$, and $\dddmes$ can be summarised with the colloquial phrase ``tangential gradients only reflect behaviour that the measure can see".
Let us be more precise, and first consider the measure $\lambda_{jk}$ for a fixed $I_{jk}$ and the associated gradients of zero and tangential gradients.
Suppose that we have a (sufficiently smooth) function $u$ defined on $\ddom$ that satisfies $u=0$ (or any constant value) on $I_{jk}$, from the perspective of $\lambda_{jk}$ this $u$ is the zero function,\footnote{more precisely, $u$ is represented by the zero function in $\ltwo{\ddom}{\lambda_{jk}}$.} regardless of whether $u$ is zero on the whole of $\ddom$ or not.
So despite $u=0$ in $\ltwo{\ddom}{\lambda_{jk}}$, it can have any profile in the direction $n_{jk}$ \emph{as it crosses} $I_{jk}$ and thus any kind of (reasonable) behaviour in the rest of $\ddom$ --- this is schematically illustrated in figure \ref{fig:Diagram_GradZeroIllustrations}.
\begin{figure}[b!]
	\centering
	\includegraphics[scale=1.0]{Diagram_GradZeroIllustrations-Scaled.pdf}
	\caption{\label{fig:Diagram_GradZeroIllustrations} Examples illustrating how non-zero gradients of zero can arise, with the region between the red lines representing an edge $I_{jk}$, which has been thickened so one can view the function values along this edge. Despite each of the functions appearing constant along the edge $I_{jk}$, and thus constant to the measure $\lambda_{jk}$, the function is changing in the direction $n_{jk}$ as it crosses $I_{jk}$ and it's behaviour ``off the edge" is unknowable from the perspective of $\lambda_{jk}$. }
\end{figure}
Indeed, the measure $\lambda_{jk}$ is unable to deduce whether (for $x\in I_{jk}$) $u(x+hn_{jk})$ is different from $u(x)$ for $h\neq0$, given the only information it can have about $u$ are its values on $I_{jk}$.
Consequentially, $\lambda_{jk}$ can have no concept of $\pdiff{u}{n_{jk}}$ --- changing the profile of $u$ across the edge $I_{jk}$ does not change $u$ on $I_{jk}$, and consequentially the component of any ``gradient" directed along $n_{jk}$ corresponds to no change in the function $u$ from the perspective of $\lambda_{jk}$, making this component a gradient of zero.
In contrast, the measure $\lambda_{jk}$ can evaluate expressions like $u(x+he_{jk})-u(x)$ --- that is, changes in the function along $I_{jk}$ are detected by the measure $\lambda_{jk}$.
Such changes correspond to $\pdiff{u}{e_{jk}}$ being non-zero, and thus we find that tangential gradients are directed along $e_{jk}$ (and provide a ``derivative" in this direction too).
This also highlights the reason for defining the gradients of zero and tangential gradients as in section \ref{sec:BorelMeasSobSpaces} --- given a tangential gradient $\ktgrad_{\lambda_{jk}}u$, $\lambda_{jk}$ can reconstruct the function $u$ along the edge $I_{jk}$, but cannot determine what $u$ is doing across $I_{jk}$.
Ergo, every function has a gradient that is unique up to a gradient of zero, because there is no way for $\lambda_{jk}$ to determine what $u$ looks like across (and thus outside of) $I_{jk}$.

The above story is similar when considering the tangential gradient of $u$ with respect to the measure $\nu$.
Here, the ``view" of the measure is even more restricted, only being able to view the value of $u$ at the vertices, which are a set of isolated points in $\ddom$.
As such, there is no way for the measure $\nu$ to reconstruct any kind of sensible gradient --- there are no ``nearby" function values $u\bracs{v_j + h x}, x\neq 0$ to compare the value of $u\bracs{v_j}$ to.
The result is corollary \ref{cory:NuTangGradChar}, any gradient must be a gradient of zero, because as far as $\nu$ is concerned, there is no visible change in $u$ in any neighbourhood of $v_j$.
Then, given that $\dddmes$  is just the sum of the measures $\ddmes$ and $\nu$, we find that $\ktgrad_{\dddmes}u$ inherits the behaviours from $\ddmes$ and $\nu$.