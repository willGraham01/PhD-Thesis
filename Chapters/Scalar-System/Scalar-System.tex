\chapter{Scalar Equation}
\tstk{title name of chapter needs redoing once you have an idea for what to change it to!}

This work is motivated by a desire to study wave propagation in a medium that exhibits (periodic) micro-structure, so a natural starting point from which we develop our framework for singular structures is the wave equation,
\begin{align} \label{eq:WaveEquation}
	- A \laplacian u = f^2 \rho u,
\end{align}
where $f$ is the frequency of a propagating wave, whilst $A$ and $\rho$ correspond to material properties.
In the context of elastic waves, $A$ and $\rho$ represent the elastic modulus and mass density respectively, whilst in the electromagnetic context they represent the electric permittivity and magnetic permeability of the medium (the polarisation dictating which of these properties $A$ and $\rho$ represent).
It is also convenient to non-dimensionalise \eqref{eq:WaveEquation} by introducing a dimensionless variable $\hat{x} = \frac{x}{l}$, where $l>0$ is the spatial extent of the medium, obtaining
\begin{align} \label{eq:WaveEquationNonDim}
	- \laplacian_{\hat{x}}\hat{u}\bracs{\hat{x}} = z \hat{u}\bracs{\hat{x}},
	\quad
	z = \frac{\bracs{fl}^2\rho}{A}.
\end{align} 
Here $\hat{u}\bracs{\hat{x}} = u\bracs{l\hat{x}}$, and the parameter $z$ represents the (square of the) ratio of the spatial extent of the medium against the wavelength of propagating waves, up to a constant.
In the context of electromagnetic waves, we have
\begin{align*}
	z = \frac{\bracs{fl}^2\eps_{\mathrm{r}}\mu_{\mathrm{r}}}{c^2},
\end{align*}
where $\eps_{\mathrm{r}}$ and $\mu_{\mathrm{r}}$ are, respectively, the relative (electric) permittivity and (magnetic) permeability of the medium, and $c$ is the speed of light in vacuum. \tstk{jackson cassent refs?}
To avoid notational clutter in equations, we will write $\omega^2=z$ where convenient.
The spectrum (that is, the set of $z=\omega^2$ in \eqref{eq:WaveEquation} that are allowed in some sense)  provides access to the frequencies of waves that can propagate in a physical material (by ``re-dimensionalising" $z$ via \eqref{eq:WaveEquationNonDim}), which determines the use of the structure itself as a waveguide.
The focus of this chapter will be on quantitatively encapsulating the structure of the spectrum for all problems of the form \eqref{eq:WaveEquation} on our singular-structure domains, via analysis of the spaces $\ktgradSob{\ddom}{\dddmes}$ and the derivation of a quantum graph problem.
We will also discuss how objects such as the $M$-matrix \tstk{section ref!} can be used to aid the analysis of the spectrum of such problems, both numerically and analytically.

We now precisely formulate the problem that we wish to consider.
Let $\graph$ be the period graph of a (periodic) metric graph embedded into $\reals^2$, with unit cell $\ddom\subset\reals^2$ (see section \ref{sec:SingularStructures}).
In this chapter, we will concern ourselves with the analysis of the ``wave equation"
\begin{align} \label{eq:SingularScalarWaveEqn}
	-\bracs{\tgrad_{\dddmes}}^2 u = \omega^2 u, \quad\text{in } \ddom,
\end{align}
by which we mean the problem of finding $u\in\ktgradSob{\ddom}{\dddmes}, \omega^2>0$ such that
\begin{align*}
	\integral{\ddom}{ \tgrad_{\dddmes}u\cdot\overline{\tgrad_{\dddmes}\phi} }{\dddmes} 
	&= \omega^2\integral{\ddom}{ u\overline{\phi} }{\dddmes}, \quad\forall\phi\in\smooth{\ddom}.
\end{align*}
We will demonstrate that solutions to \eqref{eq:SingularScalarWaveEqn} satisfy the quantum graph problem
\begin{subequations} \label{eq:SingularWaveEqnQGProblem}
	\begin{align}
		-\bracs{\diff{}{y} + \rmi\qm_{jk}}^2 u^{(jk)} &= \omega^2 u^{(jk)}, \quad &y\in\interval{I_{jk}}, \ \forall I_{jk}\in\edgeSet, \label{eq:SingularWaveEqnQGProblem-1} \\
		u \text{ is continuous at } & v_j, \quad &\forall v_j\in\vertSet, \label{eq:SingularWaveEqnQGProblem-2} \\
		\sum_{j\con k}\bracs{\pdiff{}{n} + \rmi\qm_{jk}}u^{(jk)}\bracs{v_j} &= \omega^2\alpha_j u\bracs{v_j}, \quad &\forall v_j\in\vertSet. \label{eq:SingularWaveEqnQGProblem-3}
	\end{align}
\end{subequations}
The $\qm_{jk}$ are rotations of the quasi-momentum $\qm$, and can be computed given the orientation of the edge $I_{jk}\in\edgeSet$.
It should be noted that problems like \eqref{eq:SingularWaveEqnQGProblem} belong to the class of problems with generalised resolvents, since the spectral parameter $\omega^2$ appears in the boundary condition \eqref{eq:SingularWaveEqnQGProblem-3}, which we will elaborate on later \tstk{be sure to elaborate on this later! Also the refs!}.
Our interest lies in those problems for which $\Re\bracs{\alpha_j}>0$ (physically, this ensures causality in our problem) and $\Im\bracs{\alpha_j} = 0$.
Of course there is nothing that mathematically forbids the coupling constants $\alpha_j$ being complex-valued, but the restrictions ensure we retain a link to physical applications. 
Indeed, taking $\Im\bracs{\alpha_j} > 0$ (respectively $\Im\bracs{\alpha_j} < 0$) would correspond to placing an energy sink (respectively source) at the vertex $v_j$, whist $\Re\bracs{\alpha_j}<0$ would ``reverse causality" in the solution --- physically speaking, the problem would correspond to a system in which the past depends on the future.

The analysis of the Sobolev space $\gradSobQM{\ddom}{\dddmes}$, and in particular understanding the tangential gradient $\tgrad_{\dddmes}u$, is central to the derivation of \eqref{eq:SingularWaveEqnQGProblem}. \tstk{you're also doing this with $\ktgrad$, not $\tgrad$... maybe just refer to a big appendix after CurlCurl chapter?}
However, the analysis involved is lengthy and so can be found at the end of the chapter in section \ref{sec:3DGradSobSpaces}.
We will provide a short overview of this analysis in section \ref{sec:ScalarDerivation}, before going through the derivation of \eqref{eq:SingularWaveEqnQGProblem} from \eqref{eq:SingularScalarWaveEqn}.
Section \ref{sec:ScalarDiscussion} then provides an explicit expression for the $M$-matrix of \eqref{eq:SingularWaveEqnQGProblem}, and opens a discussion into the methodology and considerations for using it to analyse the spectrum of \eqref{eq:SingularWaveEqnQGProblem}.
We then employ this approach with a selection of examples in section \ref{sec:ScalarExamples}, before concluding.
By the end of this chapter we will have a solid basis from which to develop our framework for handling ``derivatives" on singular-structures further. 
This will bring us naturally towards the analysis of the curl-of-the-curl equation \tstk{ref} and an investigation into the first-order Maxwell system, in which we generalise our approach to handle vector fields and their curls with respect to the measure $\dddmes$.

\tstk{this content is copy-pasted from the Paper\_Scalar file, and might need notational adjustments to be bought into line with the rest of the thesis. In particular, all the theory etc is done by first considering edges parallel to the $x_1$-axis, whilst the curl chapter goes for parallel to $x_2$. These will need to be reconciled.}

\section{Derivation of quantum graph problem} \label{sec:ScalarDerivation}
In this section we provide an overview of a system of the form \eqref{eq:SingularWaveEqnQGProblem} is obtained from \eqref{eq:SingularScalarWaveEqn}, which will setup our discussion revolving around the methods we employ for solving \eqref{eq:SingularWaveEqnQGProblem} in section \ref{sec:ScalarDiscussion}.

Precise definition and analysis of the ``Sobolev spaces" used here can be found in section \ref{sec:3DGradSobSpaces}, although we provide a short intuitive idea of the object $\tgrad_{\ddmes}u$ here.
The central idea behind understanding $\tgrad_{\ddmes}u$ is that the singular measure $\ddmes$ only supports the edges of $\graph$, and so cannot ``see" any changes in the function $u$ ``across" (in the direction perpendicular to) the edge $I_{jk}$.
So at any point $x\in I_{jk}$, the ``gradient" $\tgrad_{\ddmes}u(x)$ encapsulates the rate of change of the function $u$ at $x\in I_{jk}$ \emph{only} in the direction along $I_{jk}$.
As a result, it is not inaccurate to think of $\tgrad_{\ddmes}u(x) = \bracs{u^{(jk)}}'(x)e_{jk}$ (up to an appropriate shift due to the presence of the quasi-momentum $\qm$) for $x\in I_{jk}$, where $\bracs{u^{(jk)}}' = \pdiff{u^{(jk)}}{e_{jk}}$.
This also provides us with an intuitive understanding of how $\tgrad_{\ddmes}u$ behaves on each edge of the graph $\graph$, which is crucial for deriving the set of ``edge ODEs" \eqref{eq:SingularWaveEqnQGProblem} and providing a meaning to the $\diff{}{x}$ operator that appears in those equations.
As discussed in section \ref{ssec:FunctionSpaces}, the coupling constants attached to the vertices of the graph as well as the connectivity of the graph itself then dictate that these ``edge-wise" components $u^{(jk)}$ and $\bracs{u^{(jk)}}'$ adhere to certain matching conditions at the vertices.
The functions $u\in\gradSobQM{\ddom}{\dddmes}$ and their gradients $\tgrad_{\dddmes}u$ can be thought of as possessing the following properties (precise statements can be found in section \ref{sec:3DGradSobSpaces}):
\begin{enumerate}[(a)]
	\item The function $u$ is continuous at all vertices $v_j\in\vertSet$.
	\item On each edge $I_{jk}\in\edgeSet$; $\tgrad_{\dddmes}u = \tgrad_{\lambda_{jk}}u$, and $\tgrad_{\lambda_{jk}}u = \bracs{\bracs{u^{(jk)}}' + \rmi\qm_{jk}u^{(jk)}}e_{jk}$.
	The function $\bracs{u^{(jk)}}'$ being the derivative (in the $H^1$-sense) of the function $u^{(jk)}\circ r_{jk}$.
	\item At each vertex $v_j\in\vertSet$, we have $\tgrad_{\dddmes}u=0$, however $\lim_{x\rightarrow v_j}\tgrad_{\lambda_{jk}}u$ need not be zero.
\end{enumerate}

We can now provide a conceptual argument for how a system of the form \eqref{eq:SingularWaveEqnQGProblem} arises from \eqref{eq:SingularScalarWaveEqn}.
A function $u\in\gradSobQM{\ddom}{\dddmes}$ is said to be a solution to \eqref{eq:SingularScalarWaveEqn} if
\begin{align} \label{eq:SingularWaveEqnWeakForm}
	\integral{\ddom}{\tgrad_{\dddmes}u\cdot\overline{\tgrad_{\dddmes}\phi}}{\dddmes} &= \omega^2\integral{\ddom}{u\overline{\phi}}{\dddmes}, \quad\forall \phi\in\smooth{\ddom}.
\end{align}
We first note that whenever the equality in \eqref{eq:SingularWaveEqnWeakForm} holds for all smooth functions $\phi$, it holds in particular when $\phi$ has support that intersects the interior of a single edge $I_{jk}\in\edgeSet$ and no other parts of $\graph$.
Combined with the fact that $\dddmes$ is a sum of the edge measures and point masses at the vertices, and that $\tgrad_{\dddmes}u=\tgrad_{\lambda_{jk}}u$ on the edge $I_{jk}$, equation \eqref{eq:SingularWaveEqnWeakForm} implies
\begin{align*}
	0 &= \integral{\ddom}{ \bracs{\tgrad_\ddmes u \cdot \overline{\tgrad\phi} - \omega^2 u\overline{\phi}} }{\ddmes}
	= \integral{I_{jk}}{ \bracs{\tgrad_{\lambda_{jk}}u \cdot \overline{\tgrad\phi} - \omega^2 u^{(jk)}\overline{\phi}} }{\lambda_{jk}} \\
	&= \integral{I_{jk}}{ \clbracs{ \bracs{\bracs{u^{(jk)}}' + \rmi\qm_{jk} u^{(jk)}}\bracs{\overline{\phi}' - \rmi\qm_{jk} \overline{\phi} } - \omega^2 u^{(jk)}\overline{\phi} } }{\lambda_{jk}}.
\end{align*}
Now using the change of variables $r_{jk}$ and denoting $\tilde{u}^{(jk)} = u^{(jk)} \circ r_{jk}$ and $\varphi = \phi\circ r_{jk}$, we arrive at
\begin{align*}
	0 &= \int_{0}^{\abs{I_{jk}}} \bracs{\bracs{\tilde{u}^{(jk)}}' + \rmi\qm_{jk} \tilde{u}^{(jk)}}\bracs{\overline{\varphi}' - \rmi\qm_{jk} \overline{\varphi} } - \omega^2 \tilde{u}^{(jk)}\overline{\varphi} \ \md y. \\
	\implies
	\int_{0}^{\abs{I_{jk}}} \bracs{\tilde{u}^{(jk)}}'\overline{\varphi}' \ \md y &=
	\int_{0}^{\abs{I_{jk}}} \clbracs{ \omega^2\tilde{u}^{(jk)} + 2\rmi\qm_{jk}\bracs{\tilde{u}^{(jk)}}' + \bracs{\rmi\qm_{jk}}^2\tilde{u}^{(jk)} } \overline{\varphi} \ \md y.
\end{align*}
This holds for all smooth $\varphi$ with support contained in the interior of $\interval{I_{jk}}$, and as such we can deduce that $\tilde{u}^{(jk)}$ is twice (weakly) differentiable, and obtain the (strong) equation
\begin{align*}
	-\bracs{\diff{}{x} + \rmi\qm_{jk}}^2 \tilde{u}^{(jk)} &= \omega^2 \tilde{u}^{(jk)}, \quad x\in\interval{I_{jk}}.
\end{align*}

Now we turn our attention to the derivation of the vertex conditions.
Fix a vertex $v_j\in \vertSet$, and consider functions $\phi\in\smooth{\ddom}$ whose support intersects $\graph$ in a neighbourhood of $v_j$ that only contains edges which connect to $v_j$ (which can be, for example, a ball of sufficiently small radius centred on $v_j$).
Using the change of variables $r_{jk}$ on each edge and writing $\tilde{u}^{(jk)} = u^{(jk)} \circ r_{jk}$, $\varphi_{jk} = \phi\circ r_{jk}$ for each $k\con j$, we can work from \eqref{eq:SingularWaveEqnWeakForm} to obtain
\begin{align*}
	0 &= \sum_{k: \ k\con j} \integral{I_{jk}}{ \bracs{ \tgrad_\ddmes u \cdot \overline{\tgrad\phi} - \omega^2 u\overline{\phi} } }{\lambda_{jk}} 
	+ \integral{\ddom}{ \bracs{ \tgrad_{\dddmes}u\cdot\overline{\tgrad_{\dddmes}\phi}-\omega^2 u\overline{\phi} } }{\nu} \\
	&= \sum_{k: \ k\con j} \int_{0}^{\abs{I_{jk}}} \clbracs{ \bracs{\bracs{\tilde{u}^{(jk)}}' + \rmi\qm_{jk} \tilde{u}^{(jk)}}\bracs{\overline{\varphi}' - \rmi\qm_{jk} \overline{\varphi} } - \omega^2 \tilde{u}^{(jk)}\overline{\varphi} } \ \md y \\
	&\qquad + \alpha_j\left.\bracs{ \tgrad_{\dddmes}u\cdot\overline{\tgrad_{\dddmes}\phi}-\omega^2 u\overline{\phi} }\right\vert_{v_j} \\
	&= \sum_{k: \ k\con j} \int_{0}^{\abs{I_{jk}}} \clbracs{ \bracs{\bracs{\tilde{u}^{(jk)}}' + \rmi\qm_{jk} \tilde{u}^{(jk)}}\bracs{\overline{\varphi}' - \rmi\qm_{jk} \overline{\varphi} } - \omega^2 \tilde{u}^{(jk)}\overline{\varphi} } \ \md y
	 - \alpha_j \omega^2 u\bracs{v_j}\overline{\phi}\bracs{v_j}.
\end{align*}
Here we have used the fact that $\tgrad_{\dddmes}u\bracs{v_j}=0$ (see section \ref{sec:3DGradSobSpaces}).
Given that (from before) $u$ is twice differentiable on each $I_{jk}$, it follows that
\begin{align*}
	\alpha_j\omega^2 u\bracs{v_j}\overline{\phi}\bracs{v_j} 
	&= - \sum_{k: \ k\con j} \int_{0}^{\abs{I_{jk}}} \bracs{ \bracs{\diff{}{x} + \rmi\qm_{jk}}^2 \tilde{u}^{(jk)} +\omega^2 \tilde{u}^{(jk)} }\overline{\varphi} \ \md x \\
	&\qquad + \sum_{k: \ k\con j}\overline{\varphi}\bracs{v_j}\bracs{\pdiff{}{n} + \rmi\qm_{jk}}\tilde{u}^{(jk)}\bracs{v_j} \\
	&= \overline{\varphi}\bracs{v_j}\sum_{k: \ k\con j}\bracs{\pdiff{}{n} + \rmi\qm_{jk}}\tilde{u}^{(jk)}\bracs{v_j}. \labelthis\label{eq:DerivationVertexConditionWeak}
\end{align*}
Given that \eqref{eq:DerivationVertexConditionWeak} holds for every smooth $\varphi$, and that $\overline{\varphi}\bracs{v_j}=\overline{\phi}\bracs{v_j}$, we arrive at the condition
\begin{align*}
	\alpha_j\omega^2 u\bracs{v_j} &= \sum_{j\con k}\bracs{\pdiff{}{n} + \rmi\qm_{jk}}\tilde{u}^{(jk)}\bracs{v_j}, \quad \forall v_j \in \vertSet.
\end{align*}
Repeating the argument for each $v_j\in \vertSet$ then provides us with a condition of this form at each vertex.
One should note the presence of $\omega^2$ in this equation, so this is not a standard Robin condition on the derivatives of the edge-wise components of $u$, but rather indicates that our problem belongs to the class of problems with generalised resolvents, as mentioned in section \ref{ssec:DiffOpsOnGraphs}.
The result of theorem \ref{thm:dddmesTangGradImplication} tells us that functions $u\in\gradSobQM{\ddom}{\dddmes}$ are also continuous at each vertex $v_j$, and thus the following problem (precisely \eqref{eq:SingularWaveEqnQGProblem}) has been derived:
\begin{subequations}
	\begin{align}
		-\bracs{\diff{}{y} + \rmi\qm_{jk}}^2 u^{(jk)} &= \omega^2 u^{(jk)}, \quad &y\in\interval{I_{jk}}, \ \forall I_{jk}\in\edgeSet, \\
		u \text{ is continuous at } & v_j, \quad &\forall v_j\in\vertSet,  \\
		\sum_{j\con k}\bracs{\pdiff{}{n} + \rmi\qm_{jk}}u^{(jk)}\bracs{v_j} &= \omega^2\alpha_j u\bracs{v_j}, \quad &\forall v_j\in\vertSet,
	\end{align}
\end{subequations}
where we henceforth drop the overhead tilde notation and simply write $u^{(jk)}$ for brevity (appealing to the obvious association between $u^{(jk)}$ and $\tilde{u}^{(jk)}$).
Solving for the eigenvalues $\omega^2$ will net us the eigenvalues of our original problem \eqref{eq:SingularScalarWaveEqn}, and taking the union of the eigenvalues over $\qm$ will provide the spectrum of \eqref{eq:SingularScalarWaveEqnWholeSpace}.
As will be made clear in the discussion that follows, the quantum graph problem \eqref{eq:SingularWaveEqnQGProblem} is much easier to handle both analytically and numerically thanks to the utility of the $M$-matrix.

%\section{General formula for the $M$-Matrix of a Finite, Periodic Graph} \label{sec:ScalarDiscussion}
Having obtained the quantum graph problem \eqref{eq:SingularWaveEqnQGProblem}, we turn our attention to determining the eigenvalues $z := \omega^2$.
In this section we contextualise the theory introduced in section \ref{ssec:MMatrix}, showing how it is employed for studying $\sigma\bracs{-\laplacian_{\dddmes}^{\qm}}$.
Our key result will be the provision of an explicit formula for (the entries of) the $M$-matrix in terms of the underlying (period) graph on which \eqref{eq:SingularWaveEqnQGProblem} is posed.
We will follow up on this in section \ref{sec:ScalarExamples}, where we provide some explicit examples of quantum graph problems whose spectra can be determined through the study of the $M$-matrix.

One will note that only $\alpha_j$ appears in the Wentzell condition in \eqref{eq:GraphLaplacianExample}, but $\alpha_j\omega^2$ is present on the right hand side of \eqref{eq:SingularWaveEqnQGProblem}.
As was raised in section \ref{ssec:Intro-ThinStructures}, the problem \eqref{eq:SingularWaveEqnQGProblem} belongs to the class of problems with generalised resolvents.
In section \ref{ssec:MMatrix} we introduced the $M$-matrix in a more familiar setting, with no $\omega^2$-dependence in the vertex conditions.
However the analysis of the spectrum of $-\laplacian_{\dddmes}^{\qm}$ can be carried out by replacing the matrix $B=-\alpha$ (in section \ref{ssec:MMatrix}) with $\omega^2 B$, as will be done in section \ref{ssec:MMatrixConsequences}.
Justification lies in the observation that introducing explicit $\omega^2$-dependence should not affect the (structure of) the arguments in the supporting theory\footnote{See \cite[page 1846]{cherednichenko2018effective} in reference to the results of \cite{ryzhov2009weyl} and references therein.} and consequentially similar conclusions can be drawn with this alteration to $B$.
However we highlight that a formal presentation of these arguments has not been conducted in the literature, and remains open.
If one has reservations about this ``gap" in the available theory, an alternative to analysing a problem with generalised resolvents directly is explored in \cite[Section 6]{cherednichenko2017norm}.
One could look to transform a problem with $\omega^2$-dependent $\delta$-type vertex condition (like \eqref{eq:SingularWaveEqnQGProblem}) into a problem with $\omega^2$-independent $\delta'$-type vertex conditions.
This comes at the cost of having to determine the appropriate (unitary) transform to apply to \eqref{eq:SingularWaveEqnQGProblem}, but the theory of section \ref{ssec:MMatrix} would apply to the transformed problem, could be used to analyse the spectrum, and then the inverse transform applied.

Define the (Dirichlet) map $\dmap$ as in \eqref{eq:GraphDNMapDef}.
The action of the Neumann map $\nmap$ is defined sightly differently to how it appears in \eqref{eq:GraphDNMapDef} (by virtue of our need to take a Gelfand transform), however ultimately conveys the same meaning:
\begin{align} \label{eq:GraphNMapQM}
	\bracs{\nmap u}_j = -\sum_{j\con k} \bracs{\pdiff{}{n} + \rmi\qm_{jk} }u(v_j).
\end{align}
With these definitions, the Green's identity \eqref{eq:GraphGreensIdentity} 
\begin{align*}
	\ip{ -\bracs{\diff{}{y}+\rmi\qm_{jk}}^2 u }{ v }_{L^2\bracs{\graph}} - \ip{ u }{ -\bracs{\diff{}{y}+\rmi\qm_{jk}}^2 v }_{L^2\bracs{\graph}}
	&= \ip{\nmap u}{\dmap v}_{\complex^{\abs{\vertSet}}} - \ip{\dmap u}{\nmap v}_{\complex^{\abs{\vertSet}}},
\end{align*}
holds, and the map $u\mapsto\bracs{\dmap u, \nmap u}$ is clearly surjective onto $\complex^{\abs{\vertSet}}\times\complex^{\abs{\vertSet}}$.
As such, we can define the $M$-matrix for the problem \eqref{eq:SingularWaveEqnQGProblem} in the manner described in section \ref{ssec:MMatrix}, and in particular know that $\omega^2$ is an eigenvalue of $-\laplacian_{\qm}^{\dddmes}$ whenever $M\bracs{\omega^2}-\omega^2 B$ possesses a zero eigenvalue.
In the abstract, this is not particularly helpful for explicitly determining the eigenvalues --- however the nature of the problem \eqref{eq:SingularWaveEqnQGProblem} allows us to compute the entries of $M\bracs{\omega^2}$ explicitly.

\subsection{General formula for the $M$-matrix} \label{ssec:MMatrixResult}
The relatively simple nature of the structure provided by the metric graph $\graph$, and the action on each edge of the graph, allows us to explicitly compute the entries of the $M$-matrix.
The proposition we present allows for $\graph$ to possess looping edges, although this is largely for completeness because we will want to introduce \emph{artificial vertices} (section \ref{ssec:ArtificialVertices}) to break these loops.
The following proposition provides the entries of the $M$-matrix.
\begin{prop}[$M$-matrix entries] \label{prop:M-MatrixEntries}
	Let $\graph=\bracs{\vertSet,\edgeSet}$ be an embedded graph on which the problem \eqref{eq:SingularWaveEqnQGProblem} is posed.
	Suppose that $\dmap u = e_k$ where $e_k$ is the $k$\textsuperscript{th} canonical unit vector in $\complex^{\abs{\vertSet}}$.
	Then the $j$\textsuperscript{th} entry of $\nmap u$, and hence the $jk$\textsuperscript{th} entry in the $M$-matrix, is given by
	\begin{align*}
		\bracs{\nmap u}_j &= 
		\begin{cases}
			0,	
			& j \not\con k, \\[5pt]
			\sum_{j\conLeft k} \omega \e^{\rmi\qm_{jk}l_{jk}} \csc\bracs{l_{jk}\omega} 
			+ \sum_{j\conRight k} \omega \e^{-\rmi\qm_{kj}l_{kj}} \csc\bracs{l_{kj}\omega},
			& j\neq k, \ j\con k, \\[5pt]
			- \sum_{\substack{j\con l \\ j\neq l}} \omega\cot\bracs{l_{jl}\omega}
			- 2\omega\sum_{j\conLeft j} \clbracs{ \cot\bracs{l_{jj}\omega} - \cos\bracs{\qm_{jj}l_{jj}}\csc\bracs{l_{jj}\omega} },
			& j=k.
		\end{cases}
	\end{align*}
\end{prop}
Note the choice of $j\conLeft j$ in the contributions from loops is simply a convention, $j\conRight j$ is equivalent here.
Also recall the convention for summing over $j\con k$:
\begin{align*}
	\sum_{j\con k} \omega\cot\bracs{l_{jk}\omega} &= \sum_{j\conLeft k} \omega\cot\bracs{l_{jk}\omega}	+ \sum_{j\conRight k} \omega\cot\bracs{l_{kj}\omega}
\end{align*}
\begin{proof}
	The proof below is an explicit computation, similar to that in \cite{ershova2014isospectrality} with adjustments for the dependence on $\qm$.
	
	We first write the general form of the edge solution $u^{(jk)}$ from \eqref{eq:SingularWaveEqnQGProblem-1}:
	\begin{align} \label{eq:EdgeEqnGeneralSolution}
		u^{(jk)} &= \e^{-\rmi\qm_{jk}y}\bracs{ C_{+}^{(jk)}\e^{-\rmi\omega y} + C_{-}^{(jk)}\e^{\rmi\omega y} },
		\quad C_{+}^{(jk)}, C_{-}^{(jk)}\in\complex.
	\end{align}
	Since the $M$-matrix maps $\complex^{\abs{\vertSet}}$ to $\complex^{\abs{\vertSet}}$, it is sufficient to determine its action on the canonical basis of $\complex^{\abs{\vertSet}}$.
	So for each fixed $k\in\clbracs{1,...,\abs{\vertSet}}$ we set $\dmap u = e_k$.
	This provides us with sufficient Dirichlet data to solve \eqref{eq:SingularWaveEqnQGProblem-1} on each edge and eliminate the constants $C_{+}^{(jk)}$, $C_{-}^{(jk)}$ in \eqref{eq:EdgeEqnGeneralSolution}, obtaining
	\begin{align*}
		j\not\con k &\implies
		\begin{cases}
			u_{jk}(x) = 0, \\
			u_{kj}(x) = 0,
		\end{cases} \\
		j\neq k, \ j\con k &\implies
		\begin{cases}
			u_{jk}(x) = \e^{-\rmi\qm_{jk}\bracs{x-l_{jk}}}\csc\bracs{\omega l_{jk}}\sin\bracs{\omega x}, \\
			u_{kj}(x) = \e^{-\rmi\qm_{kj}x}\csc\bracs{\omega l_{kj}}\sin\bracs{\omega \bracs{l_{kj}-x}},
		\end{cases} \\
		j = k &\implies 
		\begin{cases}
			u_{jj}(t) = \e^{-\rmi\qm_{jj}x} \bracs{ \e^{-\rmi\omega x} + \sqbracs{\e^{\rmi\qm_{jj}l_{jj}}-\e^{-\rmi\omega l_{jj}}}\csc\bracs{\omega l_{jj}}\sin\bracs{\omega x}  },
		\end{cases}
	\end{align*}
	This in turn enables us to explicitly differentiate the expressions for $u_{jk}$, and read off the values of $\bracs{\pdiff{}{n}+\rmi\qm_{jk}}u_{jk}$ at the vertices.
	In the case $j\not\con k$, we obviously get zero contribution from the edges $I_{jk}$ and $I_{kj}$.
	The case $j\neq k, \ j\con k$, yields the following contributions from the edges $I_{jk}$ and $I_{kj}$:
	\begin{align*}
		\bracs{\pdiff{}{n}+\rmi\qm_{jk}}u^{(jk)}\bracs{v_j} = -\omega \e^{\rmi\qm_{jk}l_{jk}}\csc\bracs{\omega l_{jk}}, 
		&\qquad \bracs{\pdiff{}{n}+\rmi\qm_{jk}}u^{(jk)}\bracs{v_k} = \omega\cot\bracs{\omega l_{jk}}, \\
		\bracs{\pdiff{}{n}+\rmi\qm_{kj}}u^{(kj)}\bracs{v_j} = -\omega \e^{-\rmi\qm_{kj}l_{kj}}\csc\bracs{\omega l_{kj}}, 
		&\qquad \bracs{\pdiff{}{n}+\rmi\qm_{kj}}u^{(kj)}\bracs{v_k} = \omega\cot\bracs{\omega l_{kj}}.
	\end{align*}
	Finally, when considering the case $j=k$, the contribution to $\bracs{\nmap u}_j$ from loops $I_{jj}$ in the graph also requires us to compute
	\begin{align*}
		-\lim_{x\rightarrow0}\bracs{\bracs{u^{(jj)}}'+i\qm_{jj}u^{(jj)}}(x) + \lim_{x\rightarrow l_{jj}} & \bracs{\bracs{u^{(jj)}}'+i\qm_{jj}u^{(jj)}}(x) \\
		&\qquad = 2\omega\bigl( \cot\bracs{\omega l_{jj}} - \cos\bracs{\qm_{jj}l_{jj}}\csc\bracs{\omega l_{jj}} \bigr).	
	\end{align*}
	We then use the formula \eqref{eq:GraphNMapQM} to obtain the desired result for $\bracs{\nmap u}_j$.
	Since $M(\dmap u) = \nmap u$, and the $e_k$ are a basis for $\complex^{\abs{V}}$, we have also deduced the $k^{\text{th}}$ column of the $M$-matrix.
\end{proof}
Proposition \ref{prop:M-MatrixEntries} also demonstrates how the $M$-matrix is parametrised by $\qm$, and so we shall denote it by $M_{\qm}$ henceforth.
The dependence of $M_\qm$ on $\qm$ is due to our decision to specify our singular structure as an embedded, periodic metric graph and then apply the Gelfand transform (see section \ref{ssec:MMatrix}).
In the following section, we continue our analysis of this family of $M$-matrices and how it can be used to recover the eigenvalues of \eqref{eq:SingularWaveEqnQGProblem}.

\subsection{Consequences of proposition \ref{prop:M-MatrixEntries}} \label{ssec:MMatrixConsequences}
Whilst proposition \ref{prop:M-MatrixEntries} provides an explicit form for the entries of the $M$-matrix,  it is not the most convenient when looking for a method for determining the spectrum of \eqref{eq:SingularWaveEqnQGProblem}.
Proposition \ref{prop:M-MatrixEntries} does however show that $M_\qm$ is meromorphic, and thus has the following decomposition:
\begin{cory} \label{cory:M-MatrixEntriesNoPoles}
	Let $G^{(1)}_\qm\bracs{\omega}$ have entries defined by
	\begin{align*}
		\bracs{G^{(1)}_\qm}_{jk} &= 
		\begin{cases}
			\!\begin{aligned}
				&0,
			\end{aligned}			
			& j \not\con k, \\
			\!\begin{aligned}
				&\sum_{j\conLeft k} \bracs{ \e^{\rmi\qm_{jk}l_{jk}} \prod_{v_l\in\vertSet}\prod_{\substack{ l\conLeft m \\ \bracs{l,m} \neq \bracs{j,k} }}\sin\bracs{l_{lm}\omega} }
				\\ &\quad + \sum_{j\conRight k} \bracs{ \e^{-\rmi\qm_{kj}l_{kj}} \prod_{v_l\in\vertSet}\prod_{\substack{l\conLeft m \\ \bracs{l,m} \neq \bracs{k,j} }}\sin\bracs{l_{lm}\omega} },
			\end{aligned}
			& j\neq k, \ j\con k, \\
			\!\begin{aligned}
				&- \sum_{\substack{j\con l \\ j\neq l}} \bracs{ \cos\bracs{l_{jl}\omega}\prod_{v_m\in\vertSet}\prod_{\substack{ m\conLeft n \\ \bracs{m,n}\neq\bracs{j,l} }}\sin\bracs{l_{mn}\omega} }
				\\ &\quad - 2\sum_{j\conLeft j} \bracs{ \sqbracs{ \prod_{v_l\in\vertSet}\prod_{\substack{l\conLeft m \\ \bracs{l,m}\neq\bracs{j,j} }}\sin\bracs{l_{lm}\omega} }\bigl[ \cos\bracs{l_{jj}\omega} - \cos\bracs{\qm_{jj}l_{jj}} \bigr] },
			\end{aligned}
			& j=k,
		\end{cases}
	\end{align*}
	and set
	\begin{align*}
		G^{(2)}\bracs{\omega} &= \prod_{v_j\in\vertSet} \prod_{j\conLeft k}\sin\bracs{l_{jk}\omega}.
	\end{align*}
	Further define
	\begin{align*}
		H^{(1)}_{\qm}(z) &:= 
		\begin{cases} 
			\omega G_\qm^{(1)}(\omega), & \abs{\edgeSet} \text{ is even}, \\
			G_\qm^{(1)}(\omega), & \abs{\edgeSet} \text{ is odd},
		\end{cases} \\
		H^{(2)}(z) &:=
		\begin{cases}
			G^{(2)}(\omega), & \abs{\edgeSet} \text{ is even}, \\
			\omega^{-1} G^{(2)}(\omega), & \abs{\edgeSet} \text{ is odd}.
		\end{cases}
	\end{align*}
	Then the functions $H^{(1)}_{\qm}(z)$ and $H^{(2)}(z)$ are analytic in $z:=\omega^2$ and we have
	\begin{align*}
		M_\qm\bracs{z} &= \bracs{ H^{(2)}\bracs{z} }^{-1} H^{(1)}_\qm\bracs{z}.
	\end{align*}
\end{cory}
The product notation should be understood analogously to the summation notation over $j\con k$ introduced in section \ref{sec:QuantumGraphs}.
The zeros of $H^{(2)}$ exactly coincide with the poles of $M_\qm$, both $H^{(1)}_\qm$ and $H^{(2)}$ are analytic, and the matrix $H^{(1)}_\qm$ even has its entry at position $jk$ bounded (uniformly in $\omega$) by the number of (direct) connections between $v_j$ and $v_k$.

Recall that (section \ref{ssec:MMatrix}) we need to determine those $z$ for which the matrix $M_\qm(z)-B(z)$ has at least one zero eigenvalue, where we have set $B(z) = -z\alpha$.
Now let $\beta_j^{\qm}\bracs{z}, j\in\clbracs{1,...,\abs{\vertSet}}$ denote the eigenvalue branches of the matrix $M_\qm(z)-B(z)$.
Also set 
\begin{align*}
	\mathfrak{M}_\qm(z) = H^{(1)}_\qm(z) - H^{(2)}(z)B(z),
\end{align*}
and let $\widetilde{\beta}_j^{\qm}\bracs{z}, j\in\clbracs{1,...,\abs{\vertSet}}$ denote the eigenvalue branches of $\mathfrak{M}_\qm$.
With the poles removed, and the entries of $\mathfrak{M}_\qm$ being continuous (even smooth) functions of $\omega$ and $\qm$, the $\widetilde{\beta}_j^{\qm}$ are also continuous with respect to $\omega$ and $\qm$.
For each fixed $\qm$, the matrix $\mathfrak{M}_\qm$ is analytic and so has at least one zero eigenvalue at those $z$ for which there exists a $w\in\complex^{\abs{\vertSet}}\setminus\clbracs{0}$ such that
\begin{align} \label{eq:QGGenEvalSolveNoPoles}
	\mathfrak{M}_\qm\bracs{z}w = 0.
\end{align}
We could also chose to determine these $z$ via solution to 
\begin{align} \label{eq:QGDetSolveCondition}
	\det\mathfrak{M}_\qm\bracs{z} &= 0,
\end{align}
the merits of each approach (via \eqref{eq:QGGenEvalSolveNoPoles} or \eqref{eq:QGDetSolveCondition}) we will discuss in section \ref{ssec:ApproachConsiderations}.
If $z_0$ solves \eqref{eq:QGGenEvalSolveNoPoles} (or equivalently \eqref{eq:QGDetSolveCondition}), then $M_{\qm}\bracs{z_0}-B\bracs{z_0}$ has a zero eigenvalue when
\begin{align} \label{eq:EigenvalueBranchLimit}
	\lim_{z\rightarrow z_0} \beta_j^{\qm}\bracs{z} = \lim_{z\rightarrow z_0} \bracs{ H^{(2)}\bracs{z} }^{-1} \widetilde{\beta}_j^{\qm}\bracs{z} = 0
\end{align}
for at least one $j$ with $\widetilde{\beta}_j^{\qm}\bracs{z_0}=0$.
Checking the limit in \eqref{eq:EigenvalueBranchLimit} is not necessary for all solutions $z_0$ to \eqref{eq:QGDetSolveCondition}; indeed it is only necessary when $H^{(2)}\bracs{z_0} = 0$, which by corollary \ref{cory:M-MatrixEntriesNoPoles} occurs when
\begin{align} \label{eq:H2ZerosEqn}
	z_0 = \bracs{ \frac{n\pi}{l_{jk}} }^2, \quad j\conLeft k, \ n\in\naturals_{0},
\end{align}
which is a countable set of isolated points.
Furthermore the limit \eqref{eq:H2ZerosEqn} is essentially a residue, so one also has the possibility of using numerical methods derived from techniques in complex analysis to evaluate the limit.
As we will discuss in section \ref{ssec:ApproachConsiderations}, one may be able to circumvent checking the limit \eqref{eq:EigenvalueBranchLimit} by exploiting \eqref{eq:TP-DenseQMSubsetSuffices}.
Considerations concerning which of the two equations \eqref{eq:QGGenEvalSolveNoPoles} or \eqref{eq:QGDetSolveCondition} should be used for determining the spectrum $\sigma_\qm$ are also discussed in section \ref{ssec:ApproachConsiderations}.
In any event, \eqref{eq:SingularScalarWaveEqn} has now been reduced to a more accessible (family of) matrix-eigenvalue problems for $\mathfrak{M}_\qm$.

\subsection{Artificial vertices and splitting edges} \label{ssec:ArtificialVertices}
As noted in section section \ref{ssec:MMatrix}, it is required that the underlying graph $\graph$ contains no looping edges and has all edge-lengths pairwise irrationally-related\footnote{More precisely, it was noted that the operator we wish to study is required to be simple, which in the context of quantum graphs is equivalent to these conditions \cite{ashurova2014simplicity}.}.
Failure to ensure that this condition is met results in the ``loss" of certain eigenvalues when using the $M$-matrix to determine the spectrum of \eqref{eq:SingularWaveEqnQGProblem} --- these are highlighted explicitly in section \ref{ssec:Example1DLoop}.
Of course, the graphs motivated by physical applications generally do not adhere to these restrictions --- typically being highly symmetric or and potentially consisting of loops.
It is thus necessary to introduce \emph{artificial vertices} (or \emph{dummy vertices}) to split the edges of the original graph, removing any loops and ensuring all (new) edges have irrationally-related lengths.
Here it is relevant to highlight the discussions in \cite[examples 1.4.3 and 2.1.12]{berkolaiko2013introduction}; which consider a star-graph with Neumann vertex conditions, and demonstrate that the edge lengths being pairwise irrationally-related is also sufficient for all the eigenvalues to have multiplicity one.
The result \cite[theorem 3.7.1]{berkolaiko2013introduction} handles more general graph settings, and using techniques surrounding so-called bond scattering matrices, culminates with \cite[theorem 3.7.1]{berkolaiko2013introduction} relating the multiplicity of the eigenvalues to the multiplicity of the roots of the so-called \emph{secular equation}.
A discussion is provided in \cite[section 3.5]{berkolaiko2013introduction} that relates this framework with the to the Dirichlet-to-Neumann map, and hence our approach via \eqref{eq:QGGenEvalSolveNoPoles} and \eqref{eq:QGDetSolveCondition}.

Introducing an artificial vertex to split an edge is as intuitive as it sounds.
Suppose $\graph=\bracs{\vertSet, \edgeSet}$ and one wishes to split the edge $I_{jk}$ (where it may be the case that $j=k$ and we have a loop).
Place a vertex $v_l$ at some point along the edge $I_{jk}$, and replace $I_{jk}$ with the edges $I_{jl}$ and $I_{lk}$, to obtain a new graph $\graph^*$.
The total length of the edges must be preserved, so $l_{jk} = l_{jl} + l_{lk}$, but the lengths of the new edges should be chosen bearing in mind the aforementioned requirements.
Furthermore, a zero coupling constant should be placed at the artificial vertex $v_l$ --- this ensures matching of the solution $u$ and its derivative at the artificial vertex, as would have been the case along the original edge if it had not been split.
The quasi-momentum parameters should also satisfy $\qm_{jl}=\qm_{lk}=\qm_{jk}$ (although this is a by-product of having straight edges, see assumption \ref{ass:MeasTheoryProblemSetup}), and the direction of the original edge should be preserved.
This ensures (via \eqref{eq:SingularWaveEqnQGProblem-2} and \eqref{eq:SingularWaveEqnQGProblem-3}) that any eigenvalues $\omega^2$ of \eqref{eq:SingularWaveEqnQGProblem} on $\graph^*$ are also eigenvalues of \eqref{eq:SingularWaveEqnQGProblem} on $\graph$, with the eigenfunction $u^{(jk)}$ being related to $u^{(jl)}$ and $u^{(lk)}$ in the obvious manner.
This process can be executed iteratively, splitting edges to remove rational-relations between edge lengths and any loops.
Doing so means that any graph representing a singular-structure can now be treated in the manner described in section \ref{ssec:MMatrixConsequences}.

We highlight here that there are a number of considerations to make regarding the design of a \emph{general algorithm} to create a graph with pairwise irrationally-related edge lengths, from a graph without this quality.
Design of such an algorithm lies outside the objectives of this thesis, however we hope to provide readers interested in developing such a proceedure with a helpful starting point for such considerations.
In the event that one begins with a graph with $L$ edges all of which are \emph{all} pairwise-rationally related, then one has the following proceedural method for creating a graph with $2L$ edges of pairwise irrationally-related lengths:
\begin{enumerate}
   \item Split the edge $I_1$ of length $l_1$ into two edges of lengths $\frac{l_1}{\sqrt{2}}$ and $l_1\bracs{1 - \frac{1}{\sqrt{2}}}$.
   These edges now have lengths that are irrationally related to each other, and all other edges in the graph.
   \item Split the edge $I_2$ of length $l_2$ into two edges of lengths $\frac{l_2}{\sqrt{3}}$ and $l_2\bracs{1-\frac{1}{\sqrt{3}}}$.
   These edges now have lengths that are irrationally related to each other, all edges $I_n$, $2<n<L$, and the two new edges created in step 1.
   \item Proceeding iteratively, one splits the edge $I_n$ of length $l_n$ into two edges of length $\frac{l_n}{\sqrt{p_n}}$ and $l_n\bracs{1-\frac{1}{\sqrt{p_n}}}$ where $p_n$ is the $n^{\text{th}}$ prime number.   
\end{enumerate}
Here we have are dependent on the following result:
\begin{lemma} \label{lem:NumTheory-GraphSplitAlgorithm}
    Suppose that $a,b>0$ and there exists $q\in\rationals$ such that $\frac{a}{b}=q$.
    Let $p_1$ and $p_2$ be distinct prime numbers. 
    Then the values
    \begin{align} \label{eq:NumTheory-PairsWithoutAssumption}
        \frac{a}{\sqrt{p_1}}, \quad
        a\bracs{1-\frac{1}{\sqrt{p_1}}}, \quad
        b,    
    \end{align}
    are pairwise-irrationally related.
    If additionally $\sqrt{\frac{p_1}{p_2}}\frac{\sqrt{p_2}-1}{\sqrt{p_1}-1}$ is irrational\footnote{It is not known to the author whether this condition is always satisfied for primes $p_1$ and $p_2$, so the result may be true without the need to add this additional assumption.}, the values
    \begin{align} \label{eq:NumTheory-PairsWithAssumption}
        \frac{a}{\sqrt{p_1}}, \quad
        a\bracs{1-\frac{1}{\sqrt{p_1}}}, \quad
        \frac{b}{\sqrt{p_2}}, \quad
        b\bracs{1-\frac{1}{\sqrt{p_2}}},
    \end{align}
    are pairwise-irrationally related.
\end{lemma}
\begin{proof}
    This proof is simply a case of comparing the quotients of the possible pairs in \eqref{eq:NumTheory-PairsWithoutAssumption} and \eqref{eq:NumTheory-PairsWithAssumption}, and realising they are irrational.
\end{proof}
For the purposes of completing the proceedure listed above, it would be sufficient to demonstrate the weaker conclusion that there exist at least $L$ such pairs of primes --- because after $L$ steps we would be left with $2L$ edges of irrationally-related lengths.
Things are further complicated however if the original graph contains a mixture of rationally and irrationally related edge lengths, as one might now split an edge in the above manner only to find it then rationally relates to another edge that it previously did not.
A stronger version of lemma \ref{lem:NumTheory-GraphSplitAlgorithm} would be needed in this case (dropping the assumption that $a$ and $b$ are rationally related); or one would have to produce a less-n\"{i}ave approach to the order in which edges are split, as it may no longer be sufficient to simply use reciprocal roots of prime numbers to ensure edges are irrationally-related.

\subsection{Considerations for the approach to solving \eqref{eq:QGGenEvalSolveNoPoles} or \eqref{eq:QGDetSolveCondition}} \label{ssec:ApproachConsiderations}
In this section we briefly discuss some considerations for recovering the spectrum $\sigma\bracs{-\laplacian_{\upsilon}}=:\sigma$, through determination the spectra $\sigma_{\qm}:=\sigma\bracs{-\laplacian_{\dddmes}^{\qm}}$ via \eqref{eq:QGGenEvalSolveNoPoles} or \eqref{eq:QGDetSolveCondition}.
%Recall that our use of the Gelfand transform informs us that $\sigma = \bigcup_{\qm}\sigma_{\qm}$.
We take as given that one has to hand an appropriate numerical scheme for handling the generalised eigenvalue problem \eqref{eq:QGGenEvalSolveNoPoles} (a good introduction to which can be found in \cite{guttel2017nonlinear}), and so do not delve into the details of how such an algorithm would operate.
However it is worth highlighting that $\mathfrak{M}_\qm$ is Hermitian\footnote{Our requirement that $\alpha_j>0$ is implicitly used here --- in general $\mathfrak{M}_{\qm}$ is Hermitian provided $\alpha_j\in\reals$. The $M$-matrix $M_{\qm}$ is Hermitian regardless of the values of the coupling constants.}, from which most numerical schemes benefit.
The element-wise derivative with respect to $\omega$ is also easily computed (this derivative is required for generalised eigenvalue solvers based on Newton's method), although the resulting expressions are often cumbersome.
If we are intent on working directly with the determinant, then we can use corollary \ref{cory:M-MatrixEntriesNoPoles} to prove the following result about its form.
\begin{prop} \label{prop:MMatrixDetForm}
	Given a graph $\graph = \bracs{\vertSet, \edgeSet}$ containing no loops and whose edge-lengths are pairwise irrationally related, there exists a function $F\bracs{\qm,\omega}$ that is analytic in both its arguments, such that
	\begin{align} \label{eq:MMatrixDetForm}
		\det\mathfrak{M}_\qm\bracs{\omega^2} = \bracs{ \omega H^{(2)}\bracs{\omega^2} }^{\abs{\vertSet}-2} F\bracs{\qm,\omega}.
	\end{align}
\end{prop}
A proof of this result can be found in section \ref{sec:ProofOfProp}, but is just book-keeping the number of times a given factor can appear in the expression for the determinant.
Note that a graph with only one vertex can \emph{only} have looping edges, which must be broken via an artificial vertex to produce a graph with (at least two) vertices.
We can then use proposition \ref{prop:MMatrixDetForm} to prove the following result.
\begin{cory} \label{cory:ScalarSetInclusions}
	Define the sets
	\begin{align*}
		F_0^{\qm} := \clbracs{\omega \setVert F\bracs{\qm, \omega}=0},
		\qquad
		F_0 := \bigcup_{\qm\in B}F_0^{\qm},
		\qquad
		H_0 := \clbracs{\omega \setVert H^{(2)}\bracs{\omega^2}=0}.
	\end{align*}		
	Then it holds that
	\begin{enumerate}[(i)]
		\item $F_0\setminus H_0 = \sigma\setminus H_0$,
		\item $\overline{F_0\setminus H_0} \subset \sigma$.
	\end{enumerate}
\end{cory}
\begin{proof}
	\begin{enumerate}[(i)]
		\item This follows from the relation $\widetilde{\beta}_j^{\qm} = H^{(2)}\beta_j^\qm$ between the eigenvalue branches $\widetilde{\beta}_j^{\qm}$ of $\mathfrak{M}_{\qm}$ and $\beta_j^{\qm}$ of $M_{\qm}-B$.
		If $\omega_0\in F_0\setminus H_0$, there exists some $\qm\in B$ such that $F\bracs{\qm,\omega_0}=0$, and so proposition \ref{prop:MMatrixDetForm} implies that $\det\mathfrak{M}_\qm\bracs{\omega_0^2}=0$.
		So there is in particular one eigenvalue branch $\widetilde{\beta}_j^{\qm}$ with $\widetilde{\beta}_j^{\qm}\bracs{\omega_0^2}=0$, and since $\omega_0\not\in H_0$ we have $\beta_j^\qm\bracs{\omega_0^2}=0$ too, so $\omega_0\in\sigma$.
		Conversely, if $\omega_0\in \sigma\setminus H_0$, there exists some $\qm\in B$ and an eigenvalue branch with $\beta_j^\qm\bracs{\omega_0^2}=0$.
		Since $\omega_0\not\in H_0$, $\widetilde{\beta}_j^{\qm}\bracs{\omega_0^2}=0$ too, so $\det\mathfrak{M}_{\qm}\bracs{\omega_0^2}=0$, and thus by proposition \ref{prop:MMatrixDetForm} so is $F\bracs{\qm,\omega_0}$.
		\item By (i), $F_0\setminus H_0 \subset \sigma$.
		Therefore, $\overline{F_0\setminus H_0}\subset\overline{\sigma}=\sigma$, since the spectrum $\sigma$ is closed.
	\end{enumerate}
\end{proof}
Corollary \ref{cory:M-MatrixEntriesNoPoles} demonstrates that the majority of the spectrum can thus be found through examination of the function $F$.
The preimage $F^{-1}\bracs{\clbracs{0}}$ coincides with $\sigma$ up to the roots of $H^{(2)}\bracs{\omega^2}$ --- in particular, it dictates the spectral bands.
This also allows us to avoid manually checking the limit \eqref{eq:EigenvalueBranchLimit} at roots of $H^{(2)}\bracs{\omega^2}$ within the spectral bands --- such values are necessarily within the closure of $F_0\setminus H_0$.
All that remains are then those $\omega_0\in F_0\setminus H_0$ that are not within $\overline{F_0\setminus H_0}$ --- a set of isolated points along the positive real line.
Our example in section \ref{ssec:ExampleCrossInPlane} illustrates that the inclusion in (ii) can indeed be strict.

Finding $\sigma$ (up to checking isolated roots of $H^{(2)}$) now becomes a question of obtaining $F_0$ efficiently.
There is always brute force; computing $F_0^{\qm}$ for each $\qm$ (or for each $\qm$ in a suitable mesh if working numerically), and then taking the union over $\qm$ to obtain $F_0$.
Alternatives are available if $F$ admits additional properties, however these will rely on our ability to find an analytic expression for $F$ (one is obtained in the proof of proposition \ref{prop:MMatrixDetForm}, however it is rather cumbersome to use).
The ideal case being when we can write $F\bracs{\qm,\omega} = F_1\bracs{\qm} - F_2\bracs{\omega}$ for continuous $F_1$ and $F_2$, then \eqref{eq:QGDetSolveCondition} implies $F_0$ can be found simply by examining
\begin{align*}
	\min_{\qm}\clbracs{F_1(\qm)} \leq F_2\bracs{\omega} \leq \max_{\qm}\clbracs{F_1(\qm)},
\end{align*} 
although such a separation of $F$ will not generally be possible.
Of course, if an analytic expression for $F$ can be obtained, we can always then employ a root-finding algorithm to determine the set $F_0$.
Further to this, (the proof of) proposition \ref{prop:MMatrixDetForm} informs us that (for each fixed $\qm$) such an expression for $F$ will be solely in terms of $\omega^2$, sine and cosine functions of $\omega$.

We should not ignore the elephant in the room concerning \emph{how practical} it is to obtain, and work with, the matrix $\mathfrak{M}_\qm$ and the function $F$ in the first place.
Corollary \ref{cory:M-MatrixEntriesNoPoles} affords some insight into the sparsity of $\mathfrak{M}_{\qm}$; for every pair of vertices $v_j$ and $v_k$ that are directly connected by (at least one) edge, there are two non-zero off-diagonal entries introduced to $\mathfrak{M}_{\qm}$, whilst the diagonal entries are always not identically zero (for a connected, periodic graph).
For a given number of vertices in a connected periodic graph without loops, we thus minimise the number of non-zero off-diagonal entries when the graph is a chain --- each vertex $v_j$ has precisely two direct connections to $v_{j-1}$ and $v_{j+1}$.
The ratio $\frac{\abs{\edgeSet}}{\abs{\vertSet}}$ provides a good indicator of how ``close" to a chain graph $\graph$ is (and consequentially, how close to a diagonal matrix $\mathfrak{M}_\qm$ is) --- if $\frac{\abs{\edgeSet}}{\abs{\vertSet}} = 1$, the graph is necessarily a chain.
The opposite extreme $\frac{\abs{\edgeSet}}{\abs{\vertSet}} = \recip{2}(\abs{\vertSet}-1)$ corresponds to a fully connected graph and thus dense $\mathfrak{M}_\qm$.
For graphs with $\frac{\abs{\edgeSet}}{\abs{\vertSet}} \approx 1$, $\mathfrak{M}_{\qm}$ is relatively sparse and computing $F$ is more feasible, although still difficult analytically if $\abs{\vertSet}$ is large.
Therefore for denser $\mathfrak{M}_\qm$ and larger (in the sense of number of edges) graphs, one begins to lean toward numerical schemes based on \eqref{eq:QGGenEvalSolveNoPoles}.

Another topical question worth addressing at this point is whether, given the expectation that $\sigma$ has a band-gap structure, an alternative to finding all $\omega\in F_0$ is to only compute the spectral edges and then reconstruct $F_0$ from them.
It is known for (second order) periodic PDE problems that the edges of the spectral bands occur at the symmetry values of the quasi-momentum --- those values of $\qm$ which correspond to the periodic and anti-periodic problems (in each axis direction) on the unit cell.
If the above statement were true for the problem \eqref{eq:SingularWaveEqnQGProblem}, then the dimensionality of the problem of computing $F_0$ (hence $\sigma$) could be reliably reduced to determining $F_0^\qm$ for the aforementioned symmetry values of $\qm$.
However as discussed in \cite[Chapter 4.6]{berkolaiko2013introduction}, whilst this has been experimentally observed to be true for quantum graph problems motivated by physical structures, it is in fact untrue in general.
This extends to \eqref{eq:SingularWaveEqnQGProblem}, however \cite[Chapter 4.6]{berkolaiko2013introduction} remarks that adopting the approach of assuming the spectral edges lie at the symmetry points of the quasi-momentum does not often lead to errors in practice.
It is unclear why this is the case; the current hypothesis is that the underlying (period) graph needs to be made highly asymmetric to move the spectral edges away from the symmetry points of the quasi-momentum, and since most physical structures of interest display symmetries in their unit cells, the assumption ``holds in practice".
In our examples in section \ref{sec:ScalarExamples}, we will also see that preserving some underlying symmetries in our graphs indeed results in the symmetry values of the quasi-momentum being the boundaries of the spectral gaps.

%\section{Dispersion relations for concrete graph topologies} \label{sec:ScalarExamples}
In this section we provide examples to demonstrate how the spectrum $\sigma=\sigma\bracs{-\laplacian_{\upsilon}}$ is obtained via the analysis of the $M$-matrix for the quantum graph problem \eqref{eq:SingularWaveEqnQGProblem}.
The examples are chosen to highlight the methodology and some of the remarks discussed in sections \ref{sec:QuantumGraphs} and \ref{sec:ScalarDiscussion}.
In each of these examples we will encounter a number of terms with similar forms, so to avoid repetition we define these quantities here.
Let $a\in\bracs{0,1}$ and define
\begin{align} \label{eq:SE-CommonTrigFns}
	s_a\bracs{\omega} &= \sin\bracs{a\omega}, \quad 
	c_a\bracs{\omega} = \cos\bracs{a\omega}, \quad
	\tilde{s}_a\bracs{\omega} = \sin\bracs{(1-a)\omega}, \quad 
	\tilde{c}_a\bracs{\omega} = \cos\bracs{(1-a)\omega}.
\end{align}
The quantities in \eqref{eq:SE-CommonTrigFns} are $\omega$-dependent, however throughout this section we will suppress this dependency unless providing important formulae. 

\subsection{One-Dimensional Loop} \label{ssec:Example1DLoop}
We begin with the simplest example, a chain of vertices periodic in one direction, to demonstrate how one takes the period graph of a physical singular-structure and employs proposition \ref{prop:M-MatrixEntries} to construct the $M$-matrix and extract the spectral information.
We also use the simplicity of this example to highlight the necessity of splitting edges via the introduction of artificial vertices to remove loops and edge-lengths that are rationally-related, to compliment section \ref{ssec:ArtificialVertices}.

Consider the graph $\hat{\graph}$ periodic in one direction in $\reals\times\sqbracs{0,1}$, with vertices $v_j = \bracs{j + \recip{2}, 0}^\top$ and edges $I_{j\bracs{j+1}}, \ j\in\integers$.
Since $\hat{\graph}$ is parallel to- and periodic in the $x_1$-direction, the period cell lies in $S^1$ and the quasi-momentum $\qm\in\left[-\pi,\pi\right)$ is scalar\footnote{One can simply set $\qm_2=0$ in \eqref{eq:SingularWaveEqnQGProblem} when constructing the $\qm_{jk}$ to account for the lack of periodicity in $x_2$.}.
Place identical coupling constants at each vertex, with $\alpha_j = \alpha_1>0 \ \forall v_j\in\vertSet$.
The period graph $\graph$ of $\hat{\graph}$ consists of a single vertex $v$ (without loss of generality placed at the origin) with a looping edge $I$ of length 1, with quasi-momentum $\qm_I=\qm$ on $I$ and coupling constant $\alpha_1$ at $v$.
We must introduce an artificial vertex (section \ref{ssec:ArtificialVertices}) to break the loop $I$ into two edges with irrationally-related edge lengths, producing a new metric graph $\graph^*=\bracs{\vertSet, \edgeSet}$ (on which \eqref{eq:SingularWaveEqnQGProblem} is posed) with
\begin{align*}
	\vertSet = \clbracs{ v_1 , v_2 }, \quad \edgeSet = \clbracs{ I_{12}, I_{21} },
	&\qquad \abs{I_{12}} = a, \quad \abs{I_{21}} = 1-a,  \\
	\alpha_1 >0, \quad \alpha_2 = 0,
	&\qquad \qm_{12} = \qm_{21} = \qm,
\end{align*}
where $a$ and $1-a$ are irrationally related --- taking $a = \recip{\sqrt{2}}$ would suffice, for example.
The process of moving from $\hat{\graph}$ through $\graph$ to $\graph^*$ is illustrated in figure \ref{fig:Diagram_1DExample}.
\begin{figure}[t!]
	\centering
	\begin{subfigure}[t]{0.3\textwidth}
		\centering
		\includegraphics[scale=2]{Diagram_1DLineGraph.pdf}
		\caption[]{\label{fig:Diagram_1DLineGraph} The graph $\hat{\graph}$, periodic in one dimension, consisting of integer-spaced vertices.}
	\end{subfigure}
	~
	\begin{subfigure}[t]{0.3\textwidth}
		\centering
		\includegraphics[scale=2]{Diagram_1DLineQuantumGraph.pdf}
		\caption[]{\label{fig:Diagram_1DLineQuantumGraph} The metric graph corresponding to the period graph $\graph$, containing one (looping) edge of length 1}
	\end{subfigure}
	~
	\begin{subfigure}[t]{0.3\textwidth}
		\centering
		\includegraphics[scale=2]{Diagram_1DLineComputationGraph.pdf}	
		\caption[]{\label{fig:Diagram_1DLineComputationGraph} The metric graph $\graph^*$ obtained by the introduction of a dummy vertex $v_2$, which has coupling constant 0.}
	\end{subfigure}
	\caption[The 1D chain graph studied in the example of section \ref{ssec:Example1DLoop}.]{\label{fig:Diagram_1DExample} The graphs $\hat{\graph}$, $\graph$, and $\graph^*$.}
\end{figure} \newline

Using proposition \ref{prop:M-MatrixEntries} and corollary \ref{cory:M-MatrixEntriesNoPoles} we find that
\begin{align*}
	\mathfrak{M}_{\qm} = 
	\begin{pmatrix}[1.75]
		-\omega c_a \tilde{s}_a - \omega s_a \tilde{c}_{a} + \omega^2\alpha_1 s_a \tilde{s}_a &
		\omega e^{\rmi\qm a} \tilde{s}_a + \omega e^{-\rmi\qm(1-a)} s_a \\
		\omega e^{-\rmi\qm a} \tilde{s}_a + \omega e^{\rmi\qm(1-a)} s_a &
		-\omega c_a \tilde{s}_a - \omega s_a \tilde{c}_a
	\end{pmatrix},
	\qquad
	H^{(2)} = s_a \tilde{s}_a,\\
\end{align*}
using the notation \eqref{eq:SE-CommonTrigFns}.
Since $\graph^*$ is a chain graph with only two vertices, it is easy enough to compute $\det\mathfrak{M}_{\qm}$ and solve \eqref{eq:QGDetSolveCondition} analytically, yielding
\begin{align} \label{eq:1DChainDetEqual0}
	0 = 2\omega^2 s_a\bracs{\omega} \tilde{s}_a\bracs{\omega} \bracs{ \cos\omega - \frac{\omega\alpha_1}{2}\sin\omega - \cos\qm }.
\end{align}
Notice that the factor in front of the brackets in \eqref{eq:1DChainDetEqual0} is $2\omega^2 H^{(2)}\bracs{\omega^2}$, so is zero at $\omega=0$ and at the roots of $H^{(2)}$.
Let us define $\Xi\bracs{\qm,\omega} = \cos\omega - \frac{\omega\alpha_1}{2}\sin\omega - \cos\qm$.
Since $\cos\qm$ attains every value in $\sqbracs{-1,1}$ for $\qm\in\left[-\pi,\pi\right)$, the bracketed term in \eqref{eq:1DChainDetEqual0} implies that any $\omega$ satisfying
\begin{align*}
	-1 \leq \cos\omega - \frac{\omega\alpha_1}{2}\sin\omega \leq 1,
\end{align*}
is part of the spectrum of \eqref{eq:SingularWaveEqnQGProblem}.

We now consider solutions of \eqref{eq:1DChainDetEqual0} that are also zeros of $H^{(2)}$ --- let $\omega_0$ denote one of these values, so $\omega_0\in\clbracs{\frac{n\pi}{a}, \frac{n\pi}{1-a} \setVert n\in\naturals }$.
The eigenvalue branches of $\mathfrak{M}_{\qm}$ can be computed,
\begin{align*}
	\widetilde{\beta}_{\pm, \qm}\bracs{\omega^2} &= -\omega\sin\omega + \frac{\omega^2\alpha_1}{2}s_a \tilde{s}_a \pm \omega\sqrt{ \sin^2\omega + \frac{\omega^2\alpha_1^2}{4}s_a^2 \tilde{s}_a^2 - 2s_a \tilde{s}_a\bracs{\cos\omega+\cos\qm} },
\end{align*}
however only $\widetilde{\beta}_{+, \qm}\bracs{\omega_0^2}=0$.
As such, $\omega_0$ is part of the spectrum of \eqref{eq:SingularWaveEqnQGProblem} when
\begin{align*}
	\lim_{\omega\rightarrow\omega_0}\bracs{ H^{(2)}\bracs{\omega^2} }^{-1}\widetilde{\beta}\bracs{\omega^2}_{+} = 0,
\end{align*}
which (after applying L'h\^{o}spital's rule) only occurs when
\begin{align*}
	\exists\qm_0\in\left[-\pi,\pi\right) \text{ s.t. } \cos\omega_0 - \frac{\omega_0\alpha_1}{2}\sin\omega_0 - \cos\qm_0 = 0,
\end{align*}
that is precisely when there is some $\qm_0$ such that $\Xi\bracs{\qm_0,\omega_0}=0$.
Therefore, the spectrum of \eqref{eq:SingularWaveEqnQGProblem} is fully described by those $\omega$ satisfying
\begin{align*}
	\-1 \leq \cos\omega - \frac{\omega\alpha_1}{2}\sin\omega \leq 1.
\end{align*}
We remark here that in the notation of proposition \ref{prop:MMatrixDetForm} and corollary \ref{cory:ScalarSetInclusions}, $F\bracs{\omega, \qm}$ is defined by the right-hand side of equation \eqref{eq:1DChainDetEqual0}, and we have that $\overline{F_0\setminus H_0}=\sigma$ in this case.

Breaking the looping edge and ensuring that the resulting edge-lengths are irrationally-related is necessary to obtain a full description of $\sigma$ --- failure to do so results in the loss of certain Dirichlet eigenvalues.
By way of illustration, if the looping edge is not broken then one obtains
\begin{align*}
	\det\mathfrak{M}_{\qm}\bracs{\omega^2} &= \cos\omega - \frac{\omega\alpha_1}{2}\sin\omega - \cos\qm, \\
	H^{(2)}\bracs{\omega^2} &= \omega\sin\omega, \\
	\widetilde{\beta}_{\qm}\bracs{\omega^2} &= \cos\omega - \frac{\omega\alpha_1}{2}\sin\omega - \cos\qm.
\end{align*}
This means that $\widetilde{\beta}_{0}\bracs{(2k\pi)^2}=0$ and $\widetilde{\beta}_{-\pi}\bracs{(2(k-1)\pi)^2}=0$ for $k\in\naturals$, but 
\begin{align*}
	\lim_{\omega\rightarrow 2k\pi}\bracs{ H^{(2)}\bracs{\omega^2} }^{-1}\widetilde{\beta}_{0}\bracs{\omega^2} &= \alpha_1\bracs{2k\pi}^2 \neq 0, \\
	\lim_{\omega\rightarrow 2(k-1)\pi}\bracs{ H^{(2)}\bracs{\omega^2} }^{-1}\widetilde{\beta}_{-\pi}\bracs{\omega^2} &= \alpha_1\bracs{2(k-1)\pi}^2 \neq 0,
\end{align*}
which leads one to falsely exclude $\omega=n\pi, \ n\in\naturals$ from the spectrum.
These $\omega=2k\pi, \qm=0$ and $\omega=2(k-1)\pi, \qm=-\pi$ are in fact the eigenvalues of the Dirichlet problem
\begin{align*}
	-\bracs{\diff{}{t} + \rmi\qm_{jk}}^2 \tilde{u}^{(jk)} = \omega^2 \tilde{u}^{(jk)}, \quad & y\in\interval{I_{jk}}, \quad \forall I_{jk}\in \edgeSet, \\
	u \text{ is continuous at each } &v_j \in \vertSet, \\
	u\bracs{v_j} = 0, \quad &v_j\in\vertSet,
\end{align*}
which are also solutions to \eqref{eq:SingularWaveEqnQGProblem}.
Similar problems occur when one breaks the loop into two edges of rationally-related edge lengths --- taking $a=\recip{2}$ results in similar ``loss" of the eigenvalues $\omega=2k\pi, \ k\in\naturals$, for example.
Provided that the value of $a$ is correctly (adhering to pairwise-irrational edge lengths) however, the value used does not affect the resulting conclusions, as can be seen in this example.

\subsection{``Decorated" Graph with Dependencies Arising from the Embedding} \label{ssec:EmbeddingDependentExample}
We next provide an explicit example to complement the discussion that concluded section \ref{ssec:MMatrix}, concerning our decision to bestow our graphs with an embedding prior to determination of the $M$-matrix. 
To avoid confusion in this section, the term \emph{quantum graph} will be prefixed with \emph{embedded} when we are discussing a quantum graph that has been equipped with an embedding, and prefixed with \emph{abstract} when referring to a quantum graph that has not been assigned an embedding.

Consider the embedded graph $\hat{\graph}$ in $\reals\times\sqbracs{0,1}$, with vertices
\begin{align*}
	v_1^m = \bracs{m + \recip{2}, \recip{2}}^\top, 
	&\quad v_2^m = \bracs{m + \recip{2}\bracs{1+\cos\beta}, \recip{2}\bracs{1+\sin\beta}}^\top,
\end{align*}
for a fixed angle $\beta\in\bracs{0,\pi}$, and edges $I_{1}^{m} = \sqbracs{v_1^m, v_1^{m+1}}$ and $I_{12}^m=\sqbracs{v_1^m, v_2^m}$ for $m\in\integers$.
Place a coupling constant $\alpha_1$ at each $v_1^m$, and let $v_2^m$ have zero coupling constant, for each $m$.
Then let $\graph$ be the period graph of $\hat{\graph}$ and let $\graph_{\mathcal{Q}}=\bracs{\vertSet_{\mathcal{Q}}, edgeSet_{\mathcal{Q}}}$ be the abstract quantum graph described by
\begin{align*}
	\vertSet_{\mathcal{Q}} = \clbracs{v_1, v_2},
	\qquad
	\edgeSet_{\mathcal{Q}} = \clbracs{ I_1=\sqbracs{v_1,v_1}, \ I_2=\sqbracs{v_1,v_2} },
	\qquad
	l_{11} = 1, \ l_{12} = \recip{2}.
\end{align*}
Note that $\graph_{\mathcal{Q}}$ is the abstract quantum graph that which can be embedded into $\sqbracs{0,1}^2$ to obtain $\graph$.
Further to this, $\graph_{\mathcal{Q}}$ does not contain any reference to the angle $\beta$ at which the edges $I^m_{12}$ are orientated at --- this is entirely an artefact of our decision to embed $\graph_{\mathcal{Q}}$ into $\reals\times\sqbracs{0,1}$ and obtain $\graph$.
Upon introducing an artificial vertex to break the looping edge (see example \ref{ssec:Example1DLoop}), the embedded quantum graph $\graph^*$ which we study is
\begin{align*}
	&\graph^* = \bracs{\vertSet^*, \edgeSet^*}, \quad
	\vertSet^* = \clbracs{ v_1, v_2, v_3 }, \quad
	\edgeSet^* = \clbracs{ I_{12}, I_{13}, I_{31} }, \\
	&v_1 = \bracs{\recip{2},\recip{2}}, \quad
	v_2 = \recip{2}\bracs{\cos\beta, \sin\beta}, \quad
	v_3 = \bracs{\recip{2}+a, \recip{2}},
\end{align*}
where $a$ is chosen so that the lengths of the edges ($a, 1-a,$ and $\recip{2}$) are pairwise irrationally related.
We again have scalar quasi-momentum $\qm\in\left[-\pi,\pi\right)$ with
\begin{align*}
	\qm_{13} = \qm_{31} = \qm, \quad \qm_{12} = \qm\cos\beta,
\end{align*}
and the coupling constant at $v_1$ is $\alpha_1$, whilst the coupling constants at $v_2$ and $v_3$ are zero.
We again illustrate the embedded periodic graph $\hat{\graph}$, as well as the abstract quantum graphs $\graph_{\mathcal{Q}}$ and that corresponding to $\graph^*$ in figure \ref{fig:Diagram_1DAngledEdgeExample}.
\begin{figure}[b!]
	\centering
	\begin{subfigure}[t]{0.3\textwidth}
		\centering
		\includegraphics[scale=1.85]{Diagram_1DAngledEdge-Embedded.pdf}
		\caption[]{\label{fig:Diagram_1DAngledEdge-Embedded} The graph $\hat{\graph}$, periodic in one dimension, consisting of integer-spaced vertices with an edge ``hanging" at an angle $\beta$.}
	\end{subfigure}
	~
	\begin{subfigure}[t]{0.3\textwidth}
		\centering
		\includegraphics[scale=1.85]{Diagram_1DAngledEdge-Quantum.pdf}
		\caption[]{\label{fig:Diagram_1DAngledEdge-Quantum} The abstract quantum graph $\graph_{\mathcal{Q}}$ corresponding to the embedded period graph $\graph$.}
	\end{subfigure}
	~
	\begin{subfigure}[t]{0.3\textwidth}
		\centering
		\includegraphics[scale=1.85]{Diagram_1DAngledEdge-Computation.pdf}	
		\caption[]{\label{fig:Diagram_1DAngledEdge-Computation} The abstract quantum graph corresponding to the embedded, periodic graph $\graph^*$ that we study.}
	\end{subfigure}
	\caption[The graph and period graph studied in the example of section \ref{ssec:EmbeddingDependentExample}.]{\label{fig:Diagram_1DAngledEdgeExample} The graphs $\graph$, $\graph_{\mathcal{Q}}$, and $\graph^*$.}
\end{figure}

Applying proposition \ref{prop:M-MatrixEntries} and corollary \ref{cory:M-MatrixEntriesNoPoles} yields
\begin{align*} 
	\mathfrak{M}_\qm &=
	\begin{pmatrix}[2.5]
		\begin{split}
			&-c_a \tilde{s}_a s_{\recip{2}} 
			- s_a \tilde{c}_a s_{\recip{2}}  \\
			&- s_a \tilde{s}_a c_{\recip{2}}
			+ \omega\alpha_1 s_a \tilde{s}_a s_{\recip{2}}
		\end{split} &
		\exp\bracs{\dfrac{\rmi\qm\cos\beta}{2}}s_a \tilde{s}_a &
		e^{\rmi\qm a}\tilde{s}_a s_{\recip{2}} + e^{-\rmi\qm(1-a)}s_a s_{\recip{2}} \\
		\begin{split}		
			& \exp\bracs{-\dfrac{\rmi\qm\cos\beta}{2}}s_a \tilde{s}_a 
		\end{split} &
		-s_a \tilde{s}_a c_{\recip{2}} &
		0 \\
		\begin{split}
			& e^{-\rmi\qm a}\tilde{s}_a s_{\recip{2}} + e^{\rmi\qm(1-a)}s_a s_{\recip{2}} 
		\end{split} &
		0 &
		-\bracs{c_a \tilde{s}_a s_{\recip{2}} + s_a \tilde{c}_a s_{\recip{2}}}
	\end{pmatrix}, \\
	H^{(2)} &= \omega^{-1} s_a \tilde{s}_a s_{\recip{2}}.
\end{align*}
The angle $\beta$ has entered into the form of the $M$-matrix due to our embedding, however we shall see that the spectrum $\sigma$ is independent of $\beta$, as one would expect from examining its abstract periodic quantum graph in the alternative manner described in section \ref{ssec:MMatrix}.

Solving \eqref{eq:QGDetSolveCondition} yields
\begin{align} \label{eq:EmbeddedGraphDetSolveCondition}
	0 = 2s_a^2\bracs{\omega} \tilde{s}_a^2\bracs{\omega} s_2^2\bracs{\omega} c_2\bracs{\omega}
	\sqbracs{ \cos\qm + \recip{2} - \frac{3}{2}\cos\omega + \frac{\alpha_1\omega}{2}\sin\omega }.
\end{align}
We can identify 
\begin{align*}
	F\bracs{\qm,\omega} = 2 s_a \tilde{s}_a c_2
	\sqbracs{ \cos\qm + \recip{2} - \frac{3}{2}\cos\omega + \frac{\alpha_1\omega}{2}\sin\omega },
\end{align*}
however it will be slightly more convenient for us to work with $\Xi\bracs{\omega} = \frac{3}{2}\cos\omega - \frac{\alpha_1\omega}{2}\sin\omega$.
The rest of the spectrum is determined by the zeros of $F$, which occur at either $\omega=(2k-1)\pi$ for $k\in\naturals$ or when there exists at least one $\qm$ such that $\Xi\bracs{\omega} = \cos\theta + \recip{2}$.
The latter part of the spectrum consists of those $\omega$ such that
\begin{align*}
	\min_{\qm\in\left[-\pi,\pi\right)}\bracs{\cos\theta + \recip{2}} &\leq \Xi\bracs{\omega} 
	\leq \max_{\qm\in\left[-\pi,\pi\right)} \bracs{\cos\theta + \recip{2}}, \\
	\Leftrightarrow & \abs{ 3\cos\omega - \alpha_1\omega\sin\omega + 1 } \leq 2, 
\end{align*}
these points are visualised in figure \ref{fig:1DDecoratedGraph}.
Note that the points $\omega=(2k-1)\pi$ are included here too.
\begin{figure}[b!]
	\centering
	\begin{subfigure}[t]{0.45\textwidth}
		\centering
		\includegraphics[scale=0.525]{1DDecoratedGraph_alpha1.pdf}
		\caption[]{\label{fig:1DDecoratedGraph_alpha1} The values of $\omega$ which solve \eqref{eq:EmbeddedGraphDetSolveCondition} with $\alpha_1=1$. No zeros of $H^{(2)}$ form part of the spectrum in this case.}
	\end{subfigure}
	~
	\begin{subfigure}[t]{0.45\textwidth}
		\centering
		\includegraphics[scale=0.525]{1DDecoratedGraph_alpha0-25.pdf}
		\caption[]{\label{fig:1DDecoratedGraph_alpha0-25} The values of $\omega$ which solve \eqref{eq:EmbeddedGraphDetSolveCondition} with $\alpha_1=\recip{4}$. With $\alpha_1$ this small, some of the zeros of $H^{(2)}$ form part of the spectrum.}
	\end{subfigure}
	\caption[The spectrum of \eqref{eq:SingularScalarWaveEqn} on the geometry of section \ref{ssec:EmbeddingDependentExample}, and the corresponding poles of the determinant of the $M$-matrix.]{\label{fig:1DDecoratedGraph} The values of $\omega$ which solve \eqref{eq:EmbeddedGraphDetSolveCondition}, using $a=\recip{\sqrt{2}}$. Changing the value of $\alpha$ effects how many zeros of $H^{(2)}$ are included in the spectrum.}
\end{figure}

Zeros of $H^{(2)}$ occur at $\omega= 2n\pi, \frac{n\pi}{a}, \frac{n\pi}{1-a}$, but examining the limit \eqref{eq:EigenvalueBranchLimit} reveals that if $H^{(2)}\bracs{\omega_0^2}=0$, $\omega_0$ is part of the spectrum only when there exists a $\qm_0\in\left[-\pi,\pi\right)$ such that $\Xi\bracs{\omega_0}=\cos\theta_0+\recip{2}$.
That is, we once again require $F\bracs{\qm,\omega}=0$ for a root of $H^{(2)}$ to be part of the spectrum.
It is also worth noting that (in figure \ref{fig:1DDecoratedGraphEvalBranches-Thetas}) there are two eigenvalue branches $\widetilde{\beta}_j^{\qm}$ which are zero at $\omega_0$ (the third being non-zero at $\omega_0$).
The limit \eqref{eq:EigenvalueBranchLimit} exists for both branches, however only for one is it zero --- these branches and the corresponding limits in the vicinity of root $\omega_0=\pi\sqrt{2}$ are plotted in figure \ref{fig:1DDecoratedGraphEvalBranches-Thetas}.
\begin{figure}[b!]
	\centering
	\includegraphics[width=\textwidth]{1DDecoratedGraphEvalBranches-Thetas.pdf}
	\caption[Eigenvalue branches of the $M$-matrix near a pole of the determinant, for the geometry of section \ref{ssec:EmbeddingDependentExample}.]{\label{fig:1DDecoratedGraphEvalBranches-Thetas} Eigenvalue branches of the matrix $\mathfrak{M}_{\qm}$ near $\omega_0 = \pi\sqrt{2}$, which is a root of $H^{(2)}$. The value $\qm_0\approx-0.691\pi$ solves $\Xi\bracs{\omega_0}=\cos\qm_0+\recip{2}$, and the limit \eqref{eq:EigenvalueBranchLimit} is zero. For all other values of $\qm$ however, the limit \eqref{eq:EigenvalueBranchLimit} is non-zero.}
\end{figure}

In summary, $\sigma$ consists of those $\omega^2$ such that
\begin{align*}
	\abs{ 3\cos\omega - \alpha_1\omega\sin\omega + 1 } \leq 2,
\end{align*}
which again is precisely those $\omega$ such that there is some $\qm$ such that $\bracs{\qm,\omega}\in F^{-1}\bracs{\clbracs{0}}$.
Furthermore, and as expected, the spectrum does not depend on $\beta$ despite the fact that the $M$-matrix for each operator on $\graph^*$ does.
The spectrum does of course differ from the spectrum of the quantum graph in section \ref{ssec:Example1DLoop} due to the presence of the ``decoration" $I_{12}$, although this difference only depends on the length this additional edge (and were the coupling constant at $v_2$ non-zero, this too).

\subsection{Cross in the plane geometry} \label{ssec:ExampleCrossInPlane}
Our final example is a two-dimensional graph whose period cell represents a lattice-like structure in $\reals^2$.
Consider the embedded, periodic graph $\hat{\graph}$ defined as follows --- for each $\bracs{n,m}\in\integers^2$ define
\begin{align*}
	v^{(n,m)} &= \bracs{n+\recip{2}, m+\recip{2}}, \quad
	I_{\mathrm{left}}^{\bracs{n,m}} = \sqbracs{v^{\bracs{n,m}}, v^{\bracs{n+1,m}}}, \quad
	I_{\mathrm{up}}^{\bracs{n,m}} = \sqbracs{v^{\bracs{n,m}}, v^{\bracs{n,m+1}}}, \\
	\hat{\vertSet} &= \clbracs{v^{\bracs{n,m}} \setVert \bracs{n,m}\in\integers^2}, \quad
	\hat{\edgeSet} = \clbracs{ I_{\mathrm{l}}^{\bracs{n,m}}, I_{\mathrm{u}}^{\bracs{n,m}} \setVert \bracs{n,m}\in\integers^2}, \quad
	\hat{\graph} = \bracs{\hat{\vertSet}, \hat{\edgeSet}}.
\end{align*}
Place a coupling constant $\alpha^{\bracs{n,m}} =: \alpha_3>0$ at each $v^{(n,m)}$.
The period graph $\graph$ occupies $\left[0,1\right)^2$ and consists of a single vertex with two looping edges of length 1.
Breaking the loops by introducing two artificial vertices takes us to the quantum graph
\begin{align*}
	\vertSet^* = \clbracs{v_1, v_2, v_3}, \quad
	\edgeSet^* = \clbracs{I_{13}, I_{31}, I_{23}, I_{32}}, \quad
	\graph^* = \bracs{\vertSet^*, \edgeSet^*},
\end{align*}
with
\begin{align*}
	l_{13} = b, \quad l_{31} = \tilde{b} := 1-b, \quad 
	l_{23} = a, \quad l_{32} = \tilde{a} := 1-a, \qquad
	\qm_{13} = \qm_{31} = \qm_2, \quad \qm_{23} = \qm_{32} = \qm_1,
\end{align*}
and coupling constant $\alpha_3$ at $v_3$ (and zero coupling constants at the dummy vertices $v_1$ and $v_2$).
Using corollary \ref{cory:M-MatrixEntriesNoPoles} we set $H^{(2)}\bracs{\omega^2} = s_a\bracs{\omega} s_b\bracs{\omega} \tilde{s}_a\bracs{\omega} \tilde{s}_b\bracs{\omega}$, and obtain
\begin{align*}
	\mathfrak{M}_{\qm} & \bracs{\omega^2} = \\
	&
	\begin{pmatrix}[2.5]
		-\omega s_a \tilde{s}_a \bracs{ s_b \tilde{c}_b + c_b \tilde{s}_b } &
		0 &
		\begin{split}
			&\omega s_a \tilde{s}_a \bracs{ e^{\rmi\qm_2\tilde{b}}s_b + e^{-\rmi\qm_2 b}\tilde{s}_b }
		\end{split} \\
		0 &
		-\omega s_b \tilde{s}_b \bracs{ s_a \tilde{c}_a + c_a \tilde{s}_a } &
		\begin{split}
			&\omega s_b \tilde{s}_b \bracs{ e^{\rmi\qm_1\tilde{a}}s_a + e^{-\rmi\qm_1 a}\tilde{s}_a } 
		\end{split} \\
		\omega s_a \tilde{s}_a \bracs{ e^{-\rmi\qm_2\tilde{b})}s_b + e^{\rmi\qm_2 b}\tilde{s}_b } &
		\omega s_b \tilde{s}_b \bracs{ e^{-\rmi\qm_1\tilde{a}}s_a + e^{\rmi\qm_1 a}\tilde{s}_a } &
		\begin{split}
			&-\omega ( s_a s_b \tilde{s}_a \tilde{c}_b 
			+ s_a s_b \tilde{c}_a \tilde{s}_b \\ 
			& + s_a c_b \tilde{s}_a \tilde{s}_b
			+ c_a s_b \tilde{s}_a \tilde{s}_b \\
			& - \omega\alpha_3 s_a s_b \tilde{s}_a \tilde{s}_b )
		\end{split}
	\end{pmatrix}.
\end{align*}

Examining \eqref{eq:QGDetSolveCondition} yields
\begin{align} \label{eq:ExampleThickVertexSolution}
	0 = \omega^3 \bracs{H^{(2)}\bracs{\omega^2}}^2 \tilde{s}_b^2\bracs{\omega} \sin\bracs{\omega} 
	\bracs{ 4\cos\bracs{\frac{\qm_1+\qm_2}{2}}\cos\bracs{\frac{\qm_1-\qm_2}{2}} + \omega\alpha_3\sin\omega - 4\cos\omega }
\end{align}
although for ease we also define
\begin{align*}
	\Xi\bracs{\omega} := \cos\omega - \frac{\alpha_3\omega}{4}\sin\omega.
\end{align*}
Note that any $\omega_0$ for which $-1\leq\Xi\bracs{\omega_0}\leq 1$, there is some $\qm_0$ such that the bracketed factor in \eqref{eq:ExampleThickVertexSolution} is zero --- this again encompasses the case when $\sin\omega=0$, as in this case $\Xi\bracs{\omega} = \pm 1$.
Examination of the eigenvalue branches then produces a familiar conclusion; if $H^{(2)}\bracs{\omega_0^2}=0$, $\omega_0$ forms part of the spectrum of \eqref{eq:SingularWaveEqnQGProblem} if and only if there exists a $\qm_0$ such that
\begin{align} \label{eq:ExampleThickVertexSolutionReduced}
	\Xi\bracs{\omega}=\cos\bracs{\frac{\qm_1+\qm_2}{2}}\cos\bracs{\frac{\qm_1-\qm_2}{2}},
\end{align}
that is when $F\bracs{\qm_0,\omega_0}=0$.
As a result, the spectrum consists of exactly those $\omega$ such that
\begin{align*}
	-1 \leq \cos\omega - \frac{\alpha_3\omega}{4}\sin\omega \leq 1.
\end{align*}
If $\alpha_3=0$, we observe that the spectral bands touch and there are no spectral gaps.
By introducing (geometric) contrast through the coupling constants, gaps between the spectral edges open.
For any $\alpha_3>0$, the spectral bands $I_n$ satisfy $I_n\subset\sqbracs{(n-1)\pi, n\pi}$, each with left-endpoint $(n-1)\pi$ and a right-endpoint strictly less than $n\pi$.
This behaviour is reversed for $\alpha_3<0$, the bands having right-endpoint $n\pi$ and left-endpoint strictly greater than $(n-1)\pi$.
For $\alpha\leq-2$ there is even a gap between an isolated eigenvalue at $0$ and the beginning of the band $I_1$ --- this provides us with a case where the inclusion $\overline{F_0\setminus H_0}\subset\sigma$ in corollary \ref{cory:ScalarSetInclusions} is strict.
However as mentioned in section \ref{sec:ScalarEqnChapterIntro}, $\alpha<0$ does not correspond to any physical material.

In addition to recovering the spectrum of \eqref{eq:SingularWaveEqnQGProblem}, we can use \eqref{eq:ExampleThickVertexSolution} and \eqref{eq:QGGenEvalSolveNoPoles} to recover the eigenfunctions too.
For a given $\qm$, equation \eqref{eq:ExampleThickVertexSolutionReduced} can be solved for (a solution) $\omega=\omega_0$ (of course, we could instead choose an eigenvalue $\omega$ and compute the corresponding quasi-momentum for which $\omega\in\sigma_{\qm}$).
This $\omega_0$ corresponds to an eigenvalue $\omega_0^2$ of \eqref{eq:SingularWaveEqnQGProblem}, but also implies that there exists a $w\in\complex^{\abs{\vertSet}}$ such that \eqref{eq:QGGenEvalSolveNoPoles} holds at $\omega=\omega_0$.
We can compute the eigenvector(s) $w\in\complex^{\abs{\vertSet}}$ of $\mathfrak{M}_{\qm}\bracs{\omega_0^2}$ corresponding to its zero eigenvalue.
Identifying $w = \dmap u$ as the Dirichlet data of the eigenfunction $u$, and given \eqref{eq:EdgeEqnGeneralSolution}, the edge functions $u^{(jk)}$ can be obtained.
Some examples of the result of this process are plotted in figure \ref{fig:CrossInPlane-EdgePlot}; one can observe continuity of the eigenfunctions at the central vertex, whilst their incoming derivatives adhere to the Wentzell condition.
\begin{figure}[t!]
	\centering
	\begin{subfigure}[t]{0.45\textwidth}
		\centering
		\includegraphics[width=\textwidth]{CrossInPlane_EdgePlot-R-a1.pdf}
		\caption[]{\label{fig:CrossInPlane_EdgePlot-R-a1} The real part of the eigenfunction corresponding to $\omega_0=0.63936, \alpha=1$.}
	\end{subfigure}
	~
	\begin{subfigure}[t]{0.45\textwidth}
		\centering
		\includegraphics[width=\textwidth]{CrossInPlane_EdgePlot-I-a1.pdf}
		\caption[]{\label{fig:CrossInPlane_EdgePlot-I-a1} The imaginary part of the eigenfunction corresponding to $\omega_0=0.63936, \alpha=1$.}
	\end{subfigure}
	\vskip\baselineskip
	\begin{subfigure}[t]{0.45\textwidth}
		\centering
		\includegraphics[width=\textwidth]{CrossInPlane_EdgePlot-R-a4.pdf}
		\caption[]{\label{fig:CrossInPlane_EdgePlot-R-a4} The real part of the eigenfunction corresponding to $\omega_0=0.44812, \alpha=4$.}
	\end{subfigure}
	~
	\begin{subfigure}[t]{0.45\textwidth}
		\centering
		\includegraphics[width=\textwidth]{CrossInPlane_EdgePlot-I-a4.pdf}
		\caption[]{\label{fig:CrossInPlane_EdgePlot-I-a4} The imaginary part of the eigenfunction corresponding to $\omega_0=0.44812, \alpha=4$.}
	\end{subfigure}	
	\caption[Eigenfunctions of \eqref{eq:SingularScalarWaveEqn} on the cross-in-the-plane geometry.]{\label{fig:CrossInPlane-EdgePlot} Plots of the eigenfunctions corresponding to the eigenvalue $\omega_0=0.63936$  when $\alpha=1$, and $\omega_0=0.44812$ when $\alpha=4$. Both eigenvalues are attained when $\qm=\bracs{\frac{\pi}{4},\frac{\pi}{4}}^\top$, and the edge functions are plotted above the graph $\graph$ in the $\bracs{x_1,x_2}$-plane. We observe the expected continuity at the vertices, adherence to the Wentzell condition at $v_3$, and matching derivatives at the artificial vertices used to split the edges.}
\end{figure}
At the artificial vertices, we have matching of the incoming edge functions \emph{and} their derivatives, consistent with the zero coupling constant placed at dummy vertices.
To round off the analysis, quantities such as the integrated density of states (IDoS) and density of states (DoS) can also be estimated from \eqref{eq:ExampleThickVertexSolution}, as shown in figure \ref{fig:CrossInPlane_ScalarDoS} along with a display of the band-gap structure of the spectrum.
\begin{figure}[t!]
	\begin{subfigure}[t]{0.45\textwidth}
		\centering
		\includegraphics[scale=0.5]{CrossInPlane_ScalarDoS_alpha1-00.pdf}
		\caption[]{\label{fig:CrossInPlane_ScalarDoS_alpha1-00} The (relative) integrated density of states (IDoS), density of states (DoS) and spectrum for the system with $\alpha_3=1$.}
	\end{subfigure}
	~
	\begin{subfigure}[t]{0.45\textwidth}
		\centering
		\includegraphics[scale=0.5]{CrossInPlane_ScalarDoS_alpha4-00.pdf}
		\caption[]{\label{fig:CrossInPlane_ScalarDoS_alpha4-00} The (relative) integrated density of states (IDoS), density of states (DoS) and spectrum for the system with $\alpha_3=4$.}
	\end{subfigure}	
	\caption[The spectrum and density of states for the problem \eqref{eq:SingularScalarWaveEqn} on the cross-in-the-plane geometry.]{\label{fig:CrossInPlane_ScalarDoS} The (relative) IDoS, DoS, and spectrum for the graph topology in section \ref{ssec:ExampleCrossInPlane}.
	The relative IDoS at the value $x$ is defined as the IDoS at the value $x$ minus $\left\lfloor\frac{x}{\pi}\right\rfloor\bracs{2\pi}^2$.}
\end{figure}
We observe that the spectrum concentrates in the centre of each band, as is to be expected from the symmetry of the geometry.

\section{Sobolev functions on the edges of an embedded graph} \label{ssec:ScalarSobSpaces}
\tstk{proper introductory section here? Also, need to change from $x_1$-parallel to $x_2$-parallel throughout! Otherwise things are going to get really annoying}

\tstk{have done some setup to be able to lead with...}

Thematic throughout our analysis of gradients of zero and Sobolev functions will be the following procedure; we will first aim to understand gradients of zero on a single edge $I_{jk}$ that is parallel to (one of) the co-ordinate axes, and then employ rotation ideas to generalise our arguments to edges at any angle to the axes.
Next, we demonstrate that gradients of zero on $\graph$ can be built up from those on the individual edges --- this is unsurprising given that $\ddmes$ is just the sum of the individual singular measures supporting each edge.
Once we understand gradients of zero on edges, we can then understand the tangential gradients on the edges and on $\graph$ by following a similar line of reasoning --- working on the edges first and then looking at the implications for functions on the entire graph.
We will also have to consider (and analyse) the behaviour of gradients of zero and tangential gradients at the vertices, induced by $\nu$.
This finally allows us to understand Sobolev functions and their tangential gradients (and gradients of zero) on $\graph$ with respect to the measure $\dddmes$.
This approach will also guide us when we come to examine the Sobolev spaces of curls in section \tstk{ref!}.

As promised, we begin with an examination of gradients of zero. \tstk{we will come to consider $\kappa$-gradients in a later chapter, so to save time we might as well deal with them here. We can set $\kappa=0$, or ignore the third component in this section without any harm coming to us.}

\subsection{Gradients of Zero} \label{ssec:muGradZero}
In this section we will characterise gradients of zero with respect to the measures $\ddmes, \nu$, and $\dddmes$, in that order.
Throughout, we denote by $\ograd$ the $\ktgrad$ operator with $\kt=\bracs{0,0}$.
Given proposition \ref{prop:ZeroInvariantUnderQM-Wavenumber}, without loss of generality we can always take any approximating sequence $\phi_n$ for a gradient of zero $g$ to be such that $\ograd\phi_n\rightarrow g$, as opposed to $\ktgrad\phi_n\rightarrow g$.
\begin{prop}[Gradients of Zero on a Segment Parallel to the $x_2$-axis] \label{prop:3DGradZeroParallel}
	Suppose that the edge $I_{jk}$ is parallel to the $x_2$-axis.
	Then 
	\begin{align*}
		\gradZero{\ddom}{\lambda_{jk}} &= 
		\clbracs{ \bracs{g,0,0}^\top	\setVert g\in\ltwo{\ddom}{\lambda_{jk}} }.
	\end{align*}
\end{prop}
\begin{proof}
	This is a version of the argument in \cite[Section~3.1]{zhikov2000extension}, given proposition \ref{prop:GradZeroInvarientUnderQM} we can consider (without loss of generality) $\kt=\bracs{0,0}$ throughout.
	This argument is also one particular version of the argument in the proof of proposition \ref{prop:RotationOfEdgeGradients}, which we present in detail below.
\end{proof}

\begin{prop} \label{prop:3DGradZeroRotated}
	Let $I_{jk}$ be an edge of $\graph$.
	Then
	\begin{align*}
		\gradZero{\ddom}{\lambda_{jk}} 
		&= \clbracs{ g_{jk}\hat{n}_{jk} \setVert g_{jk}\in\ltwo{\ddom}{\lambda_{jk}} } \\
		&= \clbracs{ \begin{pmatrix} R_{jk}^{\top} & 0 \\ 0 & 1 \end{pmatrix} \bracs{g_{jk},0,0}^\top \setVert g_{jk}\in\ltwo{\ddom}{\lambda_{jk}} } \\
		&= \clbracs{ g_{jk}\in\ltwo{\ddom}{\ddmes}^2 \setVert g_{jk}\cdot e_{jk} = 0 }.
	\end{align*}
\end{prop}
Note that the three sets on the right hand side are all equal by definition of $n_{jk}, e_{jk}$, and $R_{jk}$.
As such, we will demonstrate the equality on the first line in the proof. 
\begin{proof}
	Clearly, if $g=\bracs{0,0,g_3}^\top\in\gradZero{\ddom}{\lambda_{jk}}$, then \eqref{eq:GradZeroSequenceDef} implies that $g_3=0$.
	
	Next, suppose that $g=g_{jk}\hat{e}_{jk}\in\gradZero{\ddom}{\lambda_{jk}}$, and take an approximating sequence $\phi_n$ for $g$.
	Given proposition \ref{prop:ZeroInvariantUnderQM-Wavenumber}, this implies that
	\begin{align*}
		\phi_n\lconv{\ltwo{\ddom}{\lambda_{jk}}}0, \quad
		\ograd\phi_n \lconv{\ltwo{\ddom}{\lambda_{jk}}^3} g\hat{e}_{jk}.
	\end{align*}
	Therefore, $\ograd\phi_n\cdot\hat{e}_{jk}\rightarrow g$, and thus
	\begin{align*}
		\int_0^{\abs{I_{jk}}} \abs{\diff{\phi_n}{y}\bracs{r_{jk}(y)} - g\bracs{r_{jk}(y)} }^2 \ \md y
		&= \integral{I_{jk}}{ \abs{\ograd\phi_n\cdot\hat{e}_{jk} - g}^2 }{\lambda_{jk}} \toInfty{n}, \\
		\implies \diff{\phi_n}{y}\bracs{r_{jk}(y)} &\lconv{\ltwo{\interval{I_{jk}}}{y}} g\bracs{r_{jk}(y)}.
	\end{align*}
	We also observe that $\phi_n\bracs{r_{jk}(y)}\rightarrow 0$ in $\ltwo{\interval{I_{jk}}}{y}$, and thus $g\bracs{r_{jk}(y)}$ is the derivative (in the classical $\gradSob{\interval{I_{jk}}}{y}$-sense) of the zero function, and thus $g=0$.
	
	Finally, suppose that $g\in\smooth{\ddom}$ and consider the smooth function $\phi(x) = \bracs{\bracs{x-v_j}\cdot n_{jk}}g(x)$.
	Then we have that $\ograd\phi_n = \bracs{\bracs{x-v_j}\cdot n_{jk}}\ograd g + g\hat{n}_{jk}$, and notice that $\bracs{x-v_j}\cdot n_{jk}	=0$ when $x\in I_{jk}$.
	It is now clear that $\phi=0$ and $\ograd\phi = g\hat{n}_{jk}$ on $I_{jk}$, so $g\hat{n}_{jk}\in\gradZero{\ddom}{\lambda_{jk}}$.
	By density of $\smooth{\ddom}$ in $\ltwo{\ddom}{\lambda_{jk}}$, we can conclude that $g\hat{n}_{jk}\in\gradZero{\ddom}{\lambda_{jk}}$ for every $g\in\ltwo{\ddom}{\lambda_{jk}}$.
\end{proof}
Taking $R_{jk}$ as the identity matrix to recover proposition \ref{prop:3DGradZeroParallel}.

We now focus on demonstrating that the set of gradients of zero on the entirety of $\graph$ is formed from gradients of zero on each edge.
That is, we look to prove the following characterisation of $\gradZero{\ddom}{\ddmes}$:
\begin{prop}[Characterisation of $\gradZero{\ddom}{\ddmes}$] \label{prop:3DGradZeroChar}
	For an embedded graph $\graph$ in $\ddom$, we have that
	\begin{align*}
		\gradZero{\ddom}{\ddmes} &= \clbracs{ g\in\ltwo{\ddom}{\ddmes}^3 \setVert g^{(jk)}\in\gradZero{\ddom}{\lambda_{jk}} \ \forall I_{jk}\in\edgeSet }.
	\end{align*}
\end{prop}
We will denote $B := \clbracs{ g\in\ltwo{\ddom}{\ddmes}^3 \setVert g^{(jk)}\in\gradZero{\ddom}{\lambda_{jk}} \ \forall I_{jk}\in\edgeSet }$ for the time being.
In order to prove proposition \ref{prop:3DGradZeroChar} we will need some supporting results, however the argument can be sketched out like so.
Showing that $\gradZero{\ddom}{\ddmes}\subset B$ is straightforward due to the definition of $\ddmes$ and that the norms we are interested in are related:
\begin{align*}
	\norm{\cdot}_{\ltwo{\ddom}{\lambda_{jk}}} &\leq \norm{\cdot}_{\ltwo{\ddom}{\ddmes}}, \\
	\norm{\cdot}_{\ltwo{\ddom}{\ddmes}}^2 &= \sum_{v_j\in\vertSet}\sum_{j\conLeft k}\norm{\cdot}_{\ltwo{\ddom}{\lambda_{jk}}}^2.
\end{align*}
The reverse inclusion is slightly more technical due to the fact that we have to form an approximating sequence (that converges on all of $\graph$) from a set of approximating sequences that each converge on one particular edge.
However we cannot simply extend an approximating sequence on an edge $I_{jk}$ by zero to the whole graph (as it is no longer guaranteed to be smooth), so have to smooth this sequence to zero over some small region around $I_{jk}$.
This ``smoothing" requires us to always have some non-zero distance between the edge $I_{jk}$ and all other edges of $\graph$, which is a non-trivial process if the support of the approximating sequence is close to one of the vertices $v_j$ or $v_k$.
Having overcome this obstacle, one can show that any $g_{jk}\in\gradZero{\ddom}{\lambda_{jk}}$ \emph{can} be extended by zero to obtain a function $g\in\gradZero{\ddom}{\ddmes}$, and then using linearity of the subspace $\gradZero{\ddom}{\ddmes}$, the proof will be complete.

Before continuing, we introduce two families of smooth functions that we shall make use of during the proof of proposition \ref{prop:3DGradZeroChar}.
Let $\eta\in\smooth{\ddom}$ have the properties
\begin{align*}
	0 \leq \eta(x) \leq 1, \quad
	\eta(x) = 0 \text{ when } \abs{x} \leq 1, \quad
	\eta(x) = 1 \text{ when } \abs{x} \geq 2.
\end{align*}
Then define
\begin{align*}
	\eta_j(x) = \eta\bracs{x-v_j}, \quad
	\eta_j^n(x) = \eta_j\bracs{nx},
\end{align*}
which are both smooth functions by composition.
The functions $\eta_j^n$ will enable us to extend functions defined on one edge $I_{jk}$ to the whole of $\graph$ without worrying about proximity to the vertex $v_j$.
Notice that $\eta_j^n\rightarrow 1 \toInfty{n}$ in $\ltwo{\ddom}{\ddmes}$ since
\begin{align*}
	\integral{\ddom}{ \abs{\eta_j^n - 1}^2 }{\ddmes} &\leq
	\integral{ B_{2/n}\bracs{v_j} }{}{\ddmes}
	= \ddmes\bracs{ B_{2/n}\bracs{v_j} } \leq \frac{4\abs{\edgeSet}}{n}.
\end{align*}
Additionally, we have that $\eta_j^n$ also converges in $\ltwo{\ddom}{\dddmes}$ to the function
\begin{align*}
	\tilde{\charFunc{j}} = \begin{cases} 1 & x\neq v_j, \\ 0 & x=v_j, \end{cases}
\end{align*}
since
\begin{align*}
	\norm{\eta_j^n - \tilde{\charFunc{j}}}_{\ltwo{\ddom}{\ddmes}}^2
	&= \norm{\eta_j^n - 1}_{\ltwo{\ddom}{\ddmes}}^2
	+ \integral{\ddom\setminus\clbracs{v_j}}{ \abs{\eta_j^n - 1}^2 }{\nu} \\
	&= \norm{\eta_j^n - 1}_{\ltwo{\ddom}{\ddmes}}^2 \rightarrow 0.
\end{align*}
Unsurprisingly we also need a family of smooth functions to help us ``smooth off" any approximating sequences.
Let $I_{jk}\in\edgeSet$, $\eps>0$, and set
\begin{align} \label{eq:ShortenedEdgeDef}
	I_{jk}^\eps := \clbracs{ x\in I_{jk} \setVert \mathrm{dist}\bracs{x,\partial I_{jk}}\leq\recip{\eps} }.
\end{align}
Let $\chi_{jk}^\eps\in\smooth{\ddom}$ be such that
\begin{align*}
	0 \leq \chi_{jk}^n \leq 1, \quad
	\chi_{jk}^\eps(x) = 1 \text{ when } \mathrm{dist}\bracs{x, I_{jk}^\eps}\leq \recip{3\eps}, \quad
	\chi_{jk}^\eps(x) = 0 \text{ when } \mathrm{dist}\bracs{x, I_{jk}^\eps}\geq \frac{2}{3\eps}.
\end{align*}
As $\graph$ is finite, we can assume without loss of generality that the only edge of $\graph$ that $\supp\bracs{\chi_{jk}^\eps}$ intersect is $I_{jk}$ (otherwise, we just rescale the argument).
We can also assemble $\chi_{jk}^\eps$ such that $\abs{ \ograd\chi_{jk}^{\eps} } \leq c\eps$ for some $c\geq 0$ independent of $\eps$.
We can check the convergence of $\chi_{jk}^\eps \toInfty{\eps}$ in $\ltwo{\ddom}{\ddmes}$ to the characteristic function of the edge $I_{jk}$, denoted by $\charFunc{jk}$;
\begin{align*}
	\integral{\ddom}{ \abs{\chi_{jk}^\eps - \charFunc{jk}}^2 }{\ddmes}
	&= \integral{I_{jk}}{ \abs{\chi_{jk}^\eps - \charFunc{jk}}^2 }{\lambda_{jk}}
	\leq \integral{I_{jk}\cap\clbracs{\chi_{jk}^\eps\leq 1}}{}{\lambda_{jk}}
	= \frac{2}{3\eps} \rightarrow 0 \toInfty{\eps}.
\end{align*}
We also have that $\chi_{jk}^\eps$ converges to the characteristic function $\charFunc{jk}^\circ$ of the interior of $I_{jk}$ in $\ltwo{\ddom}{\dddmes}$, since
\begin{align*}
	\integral{\ddom}{ \abs{\chi_{jk}^\eps - \charFunc{jk}^\circ}^2 }{\dddmes}
	&= \integral{\ddom}{ \abs{\chi_{jk}^\eps - \charFunc{jk}^\circ}^2 }{\ddmes}
	= \integral{\ddom}{ \abs{\chi_{jk}^\eps - \charFunc{jk}}^2 }{\ddmes}.
\end{align*}

We can now prove proposition \ref{prop:3DGradZeroChar} --- the inclusion $\gradZero{\ddom}{\ddmes}\subset B$ follows immediately.
\begin{lemma} \label{lem:3DGradZeroSubsetB}
	\begin{align*}
		\gradZero{\ddom}{\ddmes} \subset B.
	\end{align*}
\end{lemma}
\begin{proof}
	This is a direct consequence of $\lambda_{jk}$ being a restriction of $\ddmes$ to a given edge.
	Indeed, let $g\in\gradZero{\ddom}{\ddmes}$ and let $\phi_n$ be an approximating sequence for $g$.
	Then clearly
	\begin{align*}
		\norm{\phi_n}_{\ltwo{\ddom}{\lambda_{jk}}} &\leq \norm{\phi_n}_{\ltwo{\ddom}{\ddmes}} \rightarrow 0, \\
		\norm{\ograd\phi_n - g^{(jk)}}_{\ltwo{\ddom}{\lambda_{jk}}} &\leq \norm{\ograd\phi_n - g}_{\ltwo{\ddom}{\ddmes}} \rightarrow 0,
	\end{align*}
	thus $g\in\ltwo{\ddom}{\lambda_{jk}}$ for every $I_{jk}$, so $g\in B$.
\end{proof}
Turning our attention to the reverse inclusion, we first demonstrate that so long as a gradient of zero on an edge $I_{jk}$ has support contained within the interior of the $I_{jk}$, we can extend it to a gradient of zero on the whole graph.
\begin{lemma}[Extension lemma for Gradients of Zero] \label{lem:3DExtensionLemmaGrads}
	Let $n\in\naturals$ and $I_{jk}^n$ be as in \eqref{eq:ShortenedEdgeDef}.
	Suppose that $g_{jk}\in\gradZero{\ddom}{\lambda_{jk}}$ with $\supp\bracs{g_{jk}}\subset I_{jk}^n$.
	Define the functions $g\in\ltwo{\ddom}{\ddmes}$ and $\tilde{g}\in\ltwo{\ddom}{\dddmes}$ by
	\begin{align*}
		g =	\begin{cases} g_{jk} & \mathrm{on} \ I_{jk}, \\ 0 & \mathrm{otherwise}, \end{cases} 
		&\quad
		\tilde{g} =	\begin{cases} g_{jk} & \mathrm{on} \ I_{jk}\setminus\clbracs{v_j,v_k}, \\ 0 & \mathrm{otherwise}. \end{cases}
	\end{align*}
	Then $g\in\gradZero{\ddom}{\ddmes}$ and $\tilde{g}\in\gradZero{\ddom}{\dddmes}$.
\end{lemma}
\begin{proof}
	Let $\phi_n$ be an approximating sequence for $g_{jk}$, and consider the sequence (of smooth functions) $\psi_l = \chi_{jk}^n\phi_l$.
	We have that
	\begin{align*}
		\norm{\psi_l}_{\ltwo{\ddom}{\ddmes}} 
		&= \norm{\chi_{jk}^n\phi_l}_{\ltwo{\ddom}{\lambda_{jk}}}
		\leq \norm{\phi_l}_{\ltwo{\ddom}{\lambda_{jk}}} \rightarrow 0 \toInfty{l}.
	\end{align*}
	Furthermore,
	\begin{align*}
		\norm{\ograd\psi_l - g}_{\ltwo{\ddom}{\ddmes}^2}^2
		&= \norm{\chi_{jk}^n\ograd\phi_l + \phi_l\ograd\chi_{jk}^n - g_{jk}}_{\ltwo{\ddom}{\lambda_{jk}}^2}^2 \\
		&\leq 2\norm{\phi_l\ograd\chi_{jk}^n}_{\ltwo{\ddom}{\lambda_{jk}}^2}^2 + 2\norm{\chi_{jk}^n\ograd\phi_l - g_{jk}}_{\ltwo{\ddom}{\lambda_{jk}}^2}^2 \\
		&\leq 2\sup_{I_{jk}}\abs{\ograd\chi_{jk}^n}^2 \norm{\phi_l}_{\ltwo{\ddom}{\lambda_{jk}}^2}^2
		+ 2\norm{\ograd\phi_l - g_{jk}}_{\ltwo{\ddom}{\lambda_{jk}}^2}^2 \\
		&\rightarrow 0 \toInfty{l}.
	\end{align*}
	Therefore, $\psi_l$ is an approximating sequence for $g$, and thus $g\in\gradZero{\ddom}{\ddmes}$, as required.
	
	Next, notice that $\psi_l\bracs{v_j} = \psi_l\bracs{v_k} = 0$, and $\ograd\psi_l\bracs{v_j} = \ograd\psi_l\bracs{v_k} = 0$ for every $l\in\naturals$.
	Therefore,
	\begin{align*}
		\norm{\psi_l}_{\ltwo{\ddom}{\nu}} = 0 = \norm{\ograd\psi_l}_{\ltwo{\ddom}{\nu}^2},
	\end{align*}
	for every $l\in\naturals$, and so
	\begin{align*}
		\norm{\psi_l}_{\ltwo{\ddom}{\dddmes}} &= \norm{\psi_l}_{\ltwo{\ddom}{\ddmes}} \rightarrow 0, \\
		\norm{\ograd\psi_l - \tilde{g}}_{\ltwo{\ddom}{\dddmes}^2} &= \norm{\ograd\psi_l - g}_{\ltwo{\ddom}{\ddmes}^2}^2 \rightarrow 0,
	\end{align*}
	and thus $\tilde{g}\in\gradZero{\ddom}{\dddmes}$.
\end{proof}
The hypothesis that $\supp\bracs{g_{jk}}\subset I_{jk}^n$ is essential for the inequality 
\begin{align*}
	\norm{\ograd\phi_l - g_{jk}}_{\ltwo{\ddom}{\lambda_{jk}}^2} \leq \norm{\ograd\phi_l - g_{jk}}_{\ltwo{\ddom}{\lambda_{jk}}^2}
\end{align*} 
to hold, and so that we can use the function $\chi_{jk}^n$ to ensure that our approximating sequence $\psi_l$ is ``restricted" to the edge $I_{jk}$ only, on which we know that $\phi_l$ has the properties we need.

We can now use the fact that the space of gradients of zero is closed to complete the proof of proposition \ref{prop:3DGradZeroChar}.
\begin{prop} \label{prop:3DBSubsetGradZero}
	We have that
	\begin{align*}
		\gradZero{\ddom}{\ddmes} \supset B.
	\end{align*}
	Furthermore, for any $g\in B$, let $\tilde{g}\in\ltwo{\ddom}{\dddmes}$ be defined by
	\begin{align*}
		\tilde{g}(x) &= \begin{cases} 0 & x\in\vertSet, \\ g & \mathrm{otherwise}. \end{cases}
	\end{align*}
	Then $\tilde{g}\in\gradZero{\ddom}{\dddmes}$.
\end{prop}
\begin{proof}
	Let $g\in B$, and define a family of functions $g_n$ by
	\begin{align*}
		g_n &= \sum_{v_j\in\vertSet}\sum_{j\conLeft k}\eta_j^n \eta_k^n g^{(jk)}.
	\end{align*}
	The graph $\graph$ is finite, so the sum converges.
	For each $j,k$ in the sum, the function $\eta_j^n \eta_k^n g^{(jk)}$ is an element of $\ltwo{\ddom}{\lambda_{jk}}$ with support in $I_{jk}^n$, so satisfies the hypothesis of the Extension lemma \ref{lem:3DExtensionLemmaGrads}.
	Therefore, $\eta_j^n \eta_k^n g^{(jk)}\in\gradZero{\ddom}{\ddmes}$ and since $\gradZero{\ddom}{\ddmes}$ is a linear subspace, $g_n\in\gradZero{\ddom}{\ddmes}$ for every $n\in\naturals$.
	Furthermore, we can see that $g_n\rightarrow g \toInfty{n}$ in $\ltwo{\ddom}{\ddmes}$ by the algebra of limits.
	Since $\gradZero{\ddom}{\ddmes}$ is closed, we conclude that $g\in\gradZero{\ddom}{\ddmes}$ too.
	
	Similarly, we can conclude by the Extension lemma \ref{lem:3DExtensionLemmaGrads} that the functions
	\begin{align*}
		\tilde{g}_n &= \begin{cases} 0 & x\in\vertSet, \\ g_n & \mathrm{otherwise}, \end{cases}
	\end{align*}
	form elements of $\gradZero{\ddom}{\dddmes}$ for each $n\in\naturals$.
	In addition, $\tilde{g}_n$ converges to $\tilde{g}$, and by closure, we have $\tilde{g}\in\gradZero{\ddom}{\dddmes}$.
\end{proof}
Proposition \ref{prop:3DBSubsetGradZero} and lemma \ref{lem:3DGradZeroSubsetB} then constitute the proof of proposition \ref{prop:3DGradZeroChar}.
The fact that we can also extend elements of $\gradZero{\ddom}{\ddmes}$, and hence $\gradZero{\ddom}{\lambda_{jk}}$, to elements of $\gradZero{\ddom}{\dddmes}$ will also contribute to our characterisation of the set $\gradZero{\ddom}{\dddmes}$.

Having dealt with the behaviour of gradients of zero on the edges of the graph $\graph$, we turn our attention to their behaviour at the vertices, induced by the measure $\nu$.
This analysis is far more straightforward than for $\ddmes$, in no small part due to the fact that the vertices of $\graph$ are isolated from each other, and so there are no problems centred around one vertex's proximity to another.
Let us begin by defining some useful functions and constants.
Set $d := \recip{2}\min\clbracs{\abs{I_{jk}} \setVert I_{jk}\in\edgeSet}$, which exists and is strictly greater than 0 since $\graph$ is finite.
For $c\in\complex$, let $\varphi_c:\reals^2\rightarrow\complex$ be a smooth function such that
\begin{align} \label{eq:NuSmoothVertexFunctionDef}
	\varphi_c(0) = 0, \quad
	\grad\varphi_c(0) = c, \quad
	\supp\bracs{\varphi_c}\subset B_d(0),
\end{align}
where $B_d(0)$ is the ball of radius $d$ centred at the origin.
Finally, set $N = \abs{\vertSet}$ and for each $v_j\in\vertSet$ define
\begin{align*}
	g_1^j(x) = \begin{cases} \bracs{1,0,0}^\top, & x=v_j, \\ 0 & x\neq v_j, \end{cases}
	\quad
	g_2^j(x) = \begin{cases} \bracs{0,1,0}^\top, & x=v_j, \\ 0 & x\neq v_j, \end{cases}
	\quad
	g_3^j(x) = \begin{cases} \bracs{0,0,1}^\top, & x=v_j, \\ 0 & x\neq v_j. \end{cases}
\end{align*}
We will now demonstrate that $\ltwo{\ddom}{\nu}^3$ is isomorphic to $\complex^{3N}$. 
\begin{lemma} \label{lem:L2nuIsomCN}
	The space $\ltwo{\ddom}{\nu}^3$ is isomorphic to $\complex^{3N}$.
	Furthermore, the collection 
	\begin{align*}
		B_{\nu} = \clbracs{g_1^j, g_2^j, g_3^j \setVert v_j\in\vertSet}
	\end{align*} forms a basis of $\ltwo{\ddom}{\nu}^3$.
\end{lemma}
\begin{proof}
	It is sufficient to notice that any $f\in\ltwo{\ddom}{\nu}^3$ is entirely determined by the values it takes at the vertices $v_j$.
	As such, we can define the map
	\begin{align*}
		\iota:\ltwo{\ddom}{\nu} \rightarrow \complex^{3N}, \quad
		\iota(f) = \bracs{\frac{f\bracs{v_1}}{\sqrt{\alpha_1}}, \frac{f\bracs{v_2}}{\sqrt{\alpha_2}}, \hdots, \frac{f\bracs{v_N}}{\sqrt{\alpha_N}}}^\top,
	\end{align*}
	where we have vertically concatenated the collection of three-vectors $f\bracs{v_j}, v_j\in\vertSet$ 	(and use the principle square root if $\alpha_j$ has non-zero imaginary part).
	Clearly $\iota$ is a bijection, and additionally for $f,g\in\ltwo{\ddom}{\nu}^3$ we have that
	\begin{align*}
		\ip{f}{g}_{\ltwo{\ddom}{\nu}^3} &= \integral{\ddom}{f\cdot\overline{g}}{\nu}
		= \sum_{v_j\in\vertSet}\alpha_j f\bracs{v_j}\overline{g\bracs{v_j}}
		= \iota(f)\cdot\overline{\iota(g)}
		= \ip{\iota(f)}{\iota(g)}_{\complex^{3N}},
	\end{align*}
	so $\iota$ is an isometry.
	Furthermore, the image of $B_{\nu}$ under $\iota$ is the canonical basis of $\complex^{3N}$, and thus the collection $B_{\nu}$ forms a basis of $\ltwo{\ddom}{\nu}^3$.
\end{proof}
Of course, if any of the $\alpha_j=0$, then we have that $\ltwo{\ddom}{\nu}^2$ is isomorphic to $\complex^{2(N-M)}$, where there are $M$ such $\alpha_j=0$ --- the obvious adjustment can be made to the map $\iota$.

The reason for observing that the collection $B_{\nu}$ is a basis of $\ltwo{\ddom}{\nu}^3$ is so that characterising the space $\gradZero{\ddom}{\nu}$ is now an easy task.
\begin{prop} \label{prop:NuGradZeroChar}
	We have that 
	\begin{align*}
		\gradZero{\ddom}{\nu} &= \mathrm{span}\clbracs{g_1^j, g_2^j \setVert j\in\vertSet }
		= \clbracs{ g\in\ltwo{\ddom}{\nu}^3 \setVert g_3 = 0}.
	\end{align*}
\end{prop}
\begin{proof}
	Notice that for any $g\in\gradZero{\ddom}{\nu}$, any approximating sequence $\phi_n$ is such that
	\begin{align*}
		\phi_n \rightarrow 0, \quad \partial_1\phi_n \rightarrow g_1, 
		\quad \partial_2\phi_n\rightarrow g_2, \quad \rmi\wavenumber\phi_n \rightarrow g_3,
	\end{align*}
	and therefore $g_3 = 0$ since $\rmi\wavenumber\phi_n$ converges to $g_3$ and the zero function.
	
	Now take $c=\bracs{1,0,0}^\top$ and $v_j\in\vertSet$, and let $\phi(x) = \varphi_c\bracs{x-v_j}$ for $\varphi_c$ as in \eqref{eq:NuSmoothVertexFunctionDef}.
	The function $\phi$ is smooth, has support contained in $B_d\bracs{v_j}$, and is such that $\ograd\phi(x) = \ograd\varphi_c\bracs{x-v_j}$.
	Clearly
	\begin{align*}
		\integral{\ddom}{\abs{\phi}^2}{\nu} = 0, \quad
		\integral{\ddom}{\abs{\ograd\phi - g_1^j}^2}{\nu} = 0,
	\end{align*}
	hence $g_1^j\in\gradZero{\ddom}{\nu}$.
	By a similar construction, we can show that $g_2^j\in\gradZero{\ddom}{\nu}$ too, and since $\gradZero{\ddom}{\nu}$ is a closed linear subspace of $\ltwo{\ddom}{\nu}^3$, we have the desired result.
\end{proof}

Given that the measure $\dddmes = \ddmes + \nu$, propositions \ref{prop:3DGradZeroChar} and \ref{prop:NuGradZeroChar} allow us to understand $\gradZero{\ddom}{\dddmes}$.
Intuitively, $\gradZero{\ddom}{\dddmes}$ is made up of linear combinations of gradients of zero with respect to $\ddmes$ and $\nu$, although we will qualify this statement since the functions (or rather, equivalence classes of functions) that live in $\gradZero{\ddom}{\ddmes}$ and $\gradZero{\ddom}{\nu}$ are defined on different parts of $\graph$.
\begin{theorem}[``Characterisation" of $\gradZero{\ddom}{\dddmes}$]
	Let $\tilde{g}\in\ltwo{\ddom}{\dddmes}^3$ and define
	\begin{align*}
		g_\mu(x) = \begin{cases} \tilde{g}(x) & x\neq v_j, \ \forall v_j\in\vertSet, \\ 0 & x=v_j, \ v_j\in\vertSet, \end{cases}
		\qquad
		g_\nu(x) = \begin{cases} 0 & x\neq v_j, \ \forall v_j\in\vertSet, \\ \tilde{g}\bracs{v_j} & x=v_j, \ v_j\in\vertSet. \end{cases}
	\end{align*}
	Then
	\begin{align*}
		\tilde{g}\in\gradZero{\ddom}{\dddmes} \quad\Leftrightarrow\quad
		g_\mu\in\gradZero{\ddom}{\ddmes} \text{ and } g_\nu\in\gradZero{\ddom}{\nu}.
	\end{align*}
\end{theorem}
\begin{proof}
	$\bracs{\Rightarrow}$ For the right-directed implication, it is sufficient to notice that
	\begin{align*}
		\norm{\cdot}_{\ltwo{\ddom}{\dddmes}^3}^2 &= \norm{\cdot}_{\ltwo{\ddom}{\ddmes}^3}^2 + \norm{\cdot}_{\ltwo{\ddom}{\nu}^3}^2,
	\end{align*}
	so any approximating sequence for $\tilde{g}$ also converges to $g_\mu$ in $\ltwo{\ddom}{\ddmes}^3$ and $g_\nu$ in $\ltwo{\ddom}{\nu}^3$.
	
	$\bracs{\Leftarrow}$ For the left-directed implication,
\end{proof}

\subsection{Tangential Gradients} \label{ssec:3DTangGradients}

\subsection{Geometric Interpretation} \label{ssec:3DGradGeometric}
