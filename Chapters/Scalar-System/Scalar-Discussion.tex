\section{General formula for the $M$-matrix of a finite period graph} \label{sec:ScalarDiscussion}
Having obtained the quantum graph problem \eqref{eq:SingularWaveEqnQGProblem}, we turn our attention to determining the eigenvalues $z := \omega^2$.
The advantage of \eqref{eq:SingularWaveEqnQGProblem} over working directly with \eqref{eq:SingularScalarWaveEqn} is that we can now use the $M$-matrix (introduced in section \ref{ssec:MMatrix}) as a tool in our analysis.
In this section we contextualise the theory introduced in section \ref{sec:QuantumGraphs}, and in particular section \ref{ssec:MMatrix}, showing how it is employed for studying the spectrum of \eqref{eq:SingularWaveEqnQGProblem} and thus $\sigma\bracs{-\laplacian_{\dddmes}^{\qm}}$.
In doing so, we provide a general formula for the $M$-matrix in terms of the underlying (period) graph on which \eqref{eq:SingularWaveEqnQGProblem} is posed.
We will follow up on this in section \ref{sec:Examples}, where we provide some explicit examples of quantum graph problems that can be solved by employing the $M$-matrix in the manner discussed below.

\subsection{General formula for the $M$-matrix} \label{ssec:MMatrixResult}
One of the foremost advantages of the $M$-matrix is that we can explicitly (and analytically) compute its entries for any (finite period) metric graph $\graph$ on which the system \eqref{eq:SingularWaveEqnQGProblem} is posed.
We also break from assumption \ref{ass:MeasTheoryProblemSetup}, and allow for the period graph to include looping edges (edges whose endpoints correspond to same vertex), since such loops do not effect the (procedure in the) proof of the result.
The inclusion of looping edges is mostly for completeness, since when we come to use the $M$-matrix to determine the eigenvalues of \eqref{eq:SingularWaveEqnQGProblem}, we will want to introduce \emph{artificial vertices} (section \ref{ssec:ArtificialVertices}) to break these loops.
The following proposition provides the entries of the $M$-matrix.
\begin{prop}[$M$-matrix entries] \label{prop:M-MatrixEntries}
	Let $\graph=\bracs{\vertSet,\edgeSet}$ be an embedded graph on which the problem \eqref{eq:SingularWaveEqnQGProblem} is posed.
	Suppose that $\dmap u = e_k$ where $e_k$ is the $k$\textsuperscript{th} canonical unit vector in $\complex^{\abs{\vertSet}}$.
	Then the $j$\textsuperscript{th} entry of $\nmap u$, and hence the $jk$\textsuperscript{th} entry in the $M$-matrix, is given by
	\begin{align*}
		\bracs{\nmap u}_j &= 
		\begin{cases}
			0,	
			& j \not\con k, \\[5pt]
			\sum_{j\conLeft k} \omega \e^{\rmi\qm_{jk}l_{jk}} \csc\bracs{l_{jk}\omega} 
			+ \sum_{j\conRight k} \omega \e^{-\rmi\qm_{kj}l_{kj}} \csc\bracs{l_{kj}\omega},
			& j\neq k, \ j\con k, \\[5pt]
			- \sum_{\substack{j\con l \\ j\neq l}} \omega\cot\bracs{l_{jl}\omega}
			- 2\omega\sum_{j\conLeft j} \clbracs{ \cot\bracs{l_{jj}\omega} - \cos\bracs{\qm_{jj}l_{jj}}\csc\bracs{l_{jj}\omega} },
			& j=k.
		\end{cases}
	\end{align*}
\end{prop}
Note the choice of $j\conLeft j$ in the contributions from loops is simply a convention, $j\conRight j$ is equivalent here.
Also recall the convention for summing over $j\con k$:
\begin{align*}
	\sum_{j\con k} \omega\cot\bracs{l_{jk}\omega} &= \sum_{j\conLeft k} \omega\cot\bracs{l_{jk}\omega}	+ \sum_{j\conRight k} \omega\cot\bracs{l_{kj}\omega}
\end{align*}
\begin{proof}
	The proof below is an explicit computation, similar to that in \cite{ershova2014isospectrality} with adjustments for the dependence on $\qm$.
	
	We first write the general form of the edge solution $u^{(jk)}$ from \eqref{eq:SingularWaveEqnQGProblem-1}:
	\begin{align} \label{eq:EdgeEqnGeneralSolution}
		u^{(jk)} &= \e^{-\rmi\qm_{jk}t}\bracs{ C_{+}^{(jk)}\e^{-\rmi\omega x} + C_{-}^{(jk)}\e^{\rmi\omega x} },
		\quad C_{+}^{(jk)}, C_{-}^{(jk)}\in\complex.
	\end{align}
	Since the $M$-matrix maps $\complex^{\abs{\vertSet}}$ to $\complex^{\abs{\vertSet}}$, it is sufficient to determine its action on the canonical basis of $\complex^{\abs{\vertSet}}$.
	So for each fixed $k\in\clbracs{1,...,\abs{\vertSet}}$ we set $\dmap u = e_k$.
	This provides us with sufficient Dirichlet data to solve \eqref{eq:SingularWaveEqnQGProblem-1} on each edge and eliminate the constants $C_{+}^{(jk)}$, $C_{-}^{(jk)}$ in \eqref{eq:EdgeEqnGeneralSolution}, obtaining
	\begin{align*}
		j\not\con k &\implies
		\begin{cases}
			u_{jk}(x) = 0, \\
			u_{kj}(x) = 0,
		\end{cases} \\
		j\neq k, \ j\con k &\implies
		\begin{cases}
			u_{jk}(x) = \e^{-\rmi\qm_{jk}\bracs{x-l_{jk}}}\csc\bracs{\omega l_{jk}}\sin\bracs{\omega x}, \\
			u_{kj}(x) = \e^{-\rmi\qm_{kj}x}\csc\bracs{\omega l_{kj}}\sin\bracs{\omega \bracs{l_{kj}-x}},
		\end{cases} \\
		j = k &\implies 
		\begin{cases}
			u_{jj}(t) = \e^{-\rmi\qm_{jj}x} \bracs{ \e^{-\rmi\omega x} + \sqbracs{\e^{\rmi\qm_{jj}l_{jj}}-\e^{-\rmi\omega l_{jj}}}\csc\bracs{\omega l_{jj}}\sin\bracs{\omega x}  },
		\end{cases}
	\end{align*}
	This in turn enables us to explicitly differentiate the expressions for $u_{jk}$, and read off the values of $\bracs{\pdiff{}{n}+\rmi\qm_{jk}}u_{jk}$ at the vertices.
	In the case $j\not\con k$, we obviously get zero contribution from the edges $I_{jk}$ and $I_{kj}$.
	The case $j\neq k, \ j\con k$, yields the following contributions from the edges $I_{jk}$ and $I_{kj}$:
	\begin{align*}
		\bracs{\pdiff{}{n}+\rmi\qm_{jk}}u^{(jk)}\bracs{v_j} = -\omega \e^{\rmi\qm_{jk}l_{jk}}\csc\bracs{\omega l_{jk}}, 
		&\qquad \bracs{\pdiff{}{n}+\rmi\qm_{jk}}u^{(jk)}\bracs{v_k} = \omega\cot\bracs{\omega l_{jk}}, \\
		\bracs{\pdiff{}{n}+\rmi\qm_{kj}}u^{(kj)}\bracs{v_j} = -\omega \e^{-\rmi\qm_{kj}l_{kj}}\csc\bracs{\omega l_{kj}}, 
		&\qquad \bracs{\pdiff{}{n}+\rmi\qm_{kj}}u^{(kj)}\bracs{v_k} = \omega\cot\bracs{\omega l_{kj}}.
	\end{align*}
	Finally, when considering the case $j=k$, the contribution to $\bracs{\nmap u}_j$ from loops $I_{jj}$ in the graph also requires us to compute
	\begin{align*}
		-\lim_{x\rightarrow0}\bracs{\bracs{u^{(jj)}}'+i\qm_{jj}u^{(jj)}}(x) + \lim_{x\rightarrow l_{jj}} & \bracs{\bracs{u^{(jj)}}'+i\qm_{jj}u^{(jj)}}(x) \\
		&\qquad = 2\omega\bigl( \cot\bracs{\omega l_{jj}} - \cos\bracs{\qm_{jj}l_{jj}}\csc\bracs{\omega l_{jj}} \bigr).	
	\end{align*}
	We then use the formula
	\begin{align*}
		\bracs{\nmap u}_j &= -\sum_{j\con l} \bracs{\pdiff{}{n}+\rmi\qm_{jl}}u^{(jl)}\bracs{v_j},
	\end{align*}
	which yields the desired result for $\bracs{\nmap u}_j$.
	Since $M(\dmap u) = \nmap u$, and the $e_k$ are a basis for $\complex^{\abs{V}}$, we have also deduced the $k^{\text{th}}$ column of the $M$-matrix.
\end{proof}
Proposition \ref{prop:M-MatrixEntries} also explicitly demonstrates that the $M$-matrix (in the context of \eqref{eq:SingularWaveEqnQGProblem}) is parametrised by $\qm$, and so we shall denote it by $M_{\qm}$ henceforth.
The dependence of $M_\qm$ on $\qm$ is due to our decision to specify our singular structure as an embedded, periodic metric graph and then apply the Gelfand transform (see section \ref{ssec:MMatrix}).
In the following section, we continue our analysis of the $M$-matrix and outline how it can be used to recover the eigenvalues $z=\omega^2$ of \eqref{eq:SingularWaveEqnQGProblem}.

\subsection{Consequences of Proposition \ref{prop:M-MatrixEntries}} \label{ssec:MMatrixConsequences}
Whilst proposition \ref{prop:M-MatrixEntries} provides an explicit form for the entries of the $M$-matrix,  it is not the most convenient when looking for a method for determining the spectrum of \eqref{eq:SingularWaveEqnQGProblem}.
Proposition \ref{prop:M-MatrixEntries} does however show that $M_\qm$ is meromorphic, and thus has the following decomposition:
\begin{cory} \label{cory:M-MatrixEntriesNoPoles}
	Let $G^{(1)}_\qm\bracs{\omega}$ have entries defined by
	\begin{align*}
		\bracs{G^{(1)}_\qm}_{jk} &= 
		\begin{cases}
			\!\begin{aligned}
				&0,
			\end{aligned}			
			& j \not\con k, \\
			\!\begin{aligned}
				&\sum_{j\conLeft k} \bracs{ \e^{\rmi\qm_{jk}l_{jk}} \prod_{v_l\in\vertSet}\prod_{\substack{ l\conLeft m \\ \bracs{l,m} \neq \bracs{j,k} }}\sin\bracs{l_{lm}\omega} }
				\\ &\quad + \sum_{j\conRight k} \bracs{ \e^{-\rmi\qm_{kj}l_{kj}} \prod_{v_l\in\vertSet}\prod_{\substack{l\conLeft m \\ \bracs{l,m} \neq \bracs{k,j} }}\sin\bracs{l_{lm}\omega} },
			\end{aligned}
			& j\neq k, \ j\con k, \\
			\!\begin{aligned}
				&- \sum_{\substack{j\con l \\ j\neq l}} \bracs{ \cos\bracs{l_{jl}\omega}\prod_{v_m\in\vertSet}\prod_{\substack{ m\conLeft n \\ \bracs{m,n}\neq\bracs{j,l} }}\sin\bracs{l_{mn}\omega} }
				\\ &\quad - 2\sum_{j\conLeft j} \bracs{ \sqbracs{ \prod_{v_l\in\vertSet}\prod_{\substack{l\conLeft m \\ \bracs{l,m}\neq\bracs{j,j} }}\sin\bracs{l_{lm}\omega} }\bigl[ \cos\bracs{l_{jj}\omega} - \cos\bracs{\qm_{jj}l_{jj}} \bigr] },
			\end{aligned}
			& j=k,
		\end{cases}
	\end{align*}
	and set
	\begin{align*}
		G^{(2)}\bracs{\omega} &= \prod_{v_j\in\vertSet} \prod_{j\conLeft k}\sin\bracs{l_{jk}\omega}.
	\end{align*}
	Further define
	\begin{align*}
		H^{(1)}_{\qm}(z) &:= 
		\begin{cases} 
			\omega G_\qm^{(1)}(\omega), & \abs{\edgeSet} \text{ is even}, \\
			G_\qm^{(1)}(\omega), & \abs{\edgeSet} \text{ is odd},
		\end{cases} \\
		H^{(2)}(z) &:=
		\begin{cases}
			G^{(2)}(\omega), & \abs{\edgeSet} \text{ is even}, \\
			\omega^{-1} G^{(2)}(\omega), & \abs{\edgeSet} \text{ is odd}.
		\end{cases}
	\end{align*}
	Then the functions $H^{(1)}_{\qm}(z)$ and $H^{(2)}(z)$ are analytic in $z:=\omega^2$ and we have
	\begin{align*}
		M_\qm\bracs{z} &= \bracs{ H^{(2)}\bracs{z} }^{-1} H^{(1)}_\qm\bracs{z}.
	\end{align*}
\end{cory}
The product notation should be understood analogously to the summation notation over $j\con k$ introduced in section \ref{sec:QuantumGraphs}.
The zeros of $H^{(2)}$ exactly coincide with the poles of $M_\qm$, both $H^{(1)}_\qm$ and $H^{(2)}$ are analytic, and the matrix $H^{(1)}_\qm$ even has its entry at position $jk$ bounded (uniformly in $\omega$) by the number of (direct) connections between $v_j$ and $v_k$.

Recall that (section \ref{ssec:MMatrix}) we need to determine those $z$ for which the matrix $M_\qm(z)-B(z)$ has at least one zero eigenvalue, where we have $B(z) = -z\alpha$.
Note the dependence of $B$ on $z$ --- this was raised in section \ref{ssec:DiffOpsOnGraphs} \tstk{you mean it was put off here - this discussion should go after we derive the scalar system, and should include how the $M$-matrix theory doesn't really change if we allow $B$ to be $z$-varying}.
Now let $\beta_j\bracs{z}, j\in\clbracs{1,...,\abs{\vertSet}}$ denote the eigenvalue branches of the matrix $M_\qm(z)-B(z)$.
Also set $\mathfrak{M}_\qm(z) = H^{(1)}_\qm(z) - H^{(2)}(z)B(z)$, and let $\widetilde{\beta}_j\bracs{z}, j\in\clbracs{1,...,\abs{\vertSet}}$ denote the eigenvalue branches of $\mathfrak{M}_\qm$.
The matrix $\mathfrak{M}_\qm$ is analytic, and so has at least one zero eigenvalue at those $z$ for which there exists a $w\in\complex^{\abs{\vertSet}}\setminus\clbracs{0}$ such that
\begin{align} \label{eq:QGGenEvalSolveNoPoles}
	\mathfrak{M}_\qm\bracs{z}w = 0.
\end{align}
We could also chose to determine these $z$ via solution to 
\begin{align} \label{eq:QGDetSolveCondition}
	\det\mathfrak{M}_\qm\bracs{z} &= 0,
\end{align}
the merits of each approach (via \eqref{eq:QGGenEvalSolveNoPoles} or \eqref{eq:QGDetSolveCondition}) we will discuss in section \ref{ssec:ApproachConsiderations}.
If $z_0$ solves \eqref{eq:QGGenEvalSolveNoPoles} (or equivalently \eqref{eq:QGDetSolveCondition}), then $M_{\qm}\bracs{z_0}-B\bracs{z_0}$ has a zero eigenvalue when
\begin{align} \label{eq:EigenvalueBranchLimit}
	\lim_{z\rightarrow z_0} \beta_j\bracs{z} = \lim_{z\rightarrow z_0} \bracs{ H^{(2)}\bracs{z} }^{-1} \widetilde{\beta}_j\bracs{z} = 0
\end{align}
for at least one $j$ with $\widetilde{\beta}_j\bracs{z_0}=0$.
Checking the limit in \eqref{eq:EigenvalueBranchLimit} is not necessary for all solutions $z_0$ to \eqref{eq:QGDetSolveCondition}; provided that $H^{(2)}\bracs{z_0}\neq 0$, which by corollary \ref{cory:M-MatrixEntriesNoPoles} occurs when
\begin{align*}
	z_0 \neq \bracs{ \frac{n\pi}{l_{jk}} }^2, \quad j\conLeft k, \ n\in\naturals_{0},
\end{align*}
the limit in \eqref{eq:EigenvalueBranchLimit} is clearly zero, and so $z_0$ belongs to the spectrum of \eqref{eq:SingularWaveEqnQGProblem}.
As we will discuss in section \ref{ssec:ApproachConsiderations}, checking the limit \eqref{eq:EigenvalueBranchLimit} may not be necessary at all, given known results about the spectra of periodic quantum graph problems. \tstk{closure result - it suffices to examine spectrum on a dense subset of the Brilliouin zone then take the closure}
Considerations concerning which of the two equations (\eqref{eq:QGGenEvalSolveNoPoles} or \eqref{eq:QGDetSolveCondition}) should be used for determining the spectrum $\sigma_\qm$ are also discussed in section \ref{ssec:ApproachConsiderations}.
In any event, \eqref{eq:SingularScalarWaveEqn} has now been reduced to a more accessible (family of) matrix-eigenvalue problems for $\mathfrak{M}_\qm$.

\subsection{Artificial Vertices and Splitting Edges} \label{ssec:ArtificialVertices}
As noted in section section \ref{ssec:MMatrix}, it is required that the underlying graph $\graph$ contains no looping edges and has all edge-lengths pairwise irrationally-related.
Failure to ensure that this condition is met may result in the ``loss" of certain eigenvalues when using the $M$-matrix to determine the spectrum of \eqref{eq:SingularWaveEqnQGProblem} --- these are highlighted explicitly in section \ref{ssec:Example1DLoop}.
Of course, the graphs motivated by physical applications generally do not adhere to these restrictions, so it is necessary to introduce \emph{artificial} or \emph{dummy} vertices.
These artificial vertices split edges of the original graph, removing any loops and ensuring all (new) edges have irrationally-related lengths.

Introducing an artificial vertex to split an edge is as intuitive as it sounds --- suppose $\graph=\bracs{\vertSet, \edgeSet}$ and one wishes to split the edge $I_{jk}$ (where it may be the case that $j=k$).
Place a vertex $v_l$ at some point along the edge $I_{jk}$, and replace $I_{jk}$ with the edges $I_{jl}$ and $I_{lk}$, to obtain a new graph $\graph^*$.
The total length of the edges must be preserved, so $l_{jk} = l_{jl} + l_{lk}$, but the lengths of the new edges should be chosen in accordance with the requirements above in mind.
Furthermore, a zero coupling constant should be placed at the artificial vertex $v_l$ --- this ensures matching of the solution $u$ and its derivative at the artificial vertex, as would have been the case along the original edge if it had not been split.
The quasi-momentum parameters should also satisfy $\qm_{jl}=\qm_{lk}=\qm_{jk}$ (although this is a by-product of having straight edges, see assumption \ref{ass:MeasTheoryProblemSetup}).
This ensures (via \eqref{eq:SingularWaveEqnQGProblem-2} and \eqref{eq:SingularWaveEqnQGProblem-3}) that any eigenvalues $\omega^2$ of \eqref{eq:SingularWaveEqnQGProblem} on $\graph^*$ are also eigenvalues of \eqref{eq:SingularWaveEqnQGProblem} on $\graph$, with the eigenfunction $u^{(jk)}$ being related to $u^{(jl)}$ and $u^{(lk)}$ in the obvious manner.
This process can be iterated, splitting edges iteratively to remove rational-relations between edge lengths, and any loops themselves.
Doing so means that any graph representing a singular-structure can now be treated in the manner described in section \ref{ssec:MMatrixConsequences}.

\subsection{Considerations for the Approach to Solving \eqref{eq:QGGenEvalSolveNoPoles} or \eqref{eq:QGDetSolveCondition}} \label{ssec:ApproachConsiderations}
In this section we briefly discuss some considerations for recovering the spectrum of $-\laplacian_{\upsilon}$ (which we denote by $\sigma$), and the merits of determining the spectrum of \eqref{eq:SingularWaveEqnQGProblem} (denoted $\sigma_\qm$) via \eqref{eq:QGGenEvalSolveNoPoles} or \eqref{eq:QGDetSolveCondition}.
Recall that our use of the Gelfand transform informs us that $\sigma = \bigcup_{\qm}\sigma_{\qm}$.
We also take as a baseline that one has to hand an appropriate numerical scheme for handling the generalised eigenvalue problem \eqref{eq:QGGenEvalSolveNoPoles} (a good introduction to which can be found in \cite{guttel2017nonlinear}), and so do not delve into the details of how such an algorithm would operate.
It is however worth mentioning that $\mathfrak{M}_\qm$ is Hermitian, from which most numerical schemes benefit.

There are also several known results concerning the spectra of (periodic) quantum graphs, and here we highlight those most relevant to our context. 
A detailed discussion of the spectral structure of (periodic) quantum graphs, including statements (and proofs) of the results cited here, can be found in \cite[Chapter 4]{berkolaiko2013introduction}.
Foremost, it is known that there exist real numbers $a_j, b_j, j\in\naturals$ such that $\sigma = \bigcup_{j\in\naturals}\sqbracs{a_j,b_j}$ --- $\sigma$ is said to have a \emph{band-gap structure} or \emph{representation} \cite[Chapter 4.3]{berkolaiko2013introduction} if this is the case.
The \emph{spectral bands} are the intervals $I_j=\sqbracs{a_j,b_j}, \ j\in\naturals$, with the regions between $b_j$ and $a_{j+1}$ being referred to as \emph{spectral gaps}, where there are no eigenvalues.
For this reason the points $a_j$ and $b_j$ sometimes referred to as the \emph{spectral edges}, and in general $a_j\rightarrow\infty$ as $j\rightarrow\infty$.
Knowing that $\sigma$ has a band-gap representation is particularly useful when attempting to use either \eqref{eq:QGGenEvalSolveNoPoles} or \eqref{eq:QGDetSolveCondition} to determine it, which we will touch on shortly.
Other notable results are that $\sigma$ has no singular continuous part (\cite[Chapter 4.4]{berkolaiko2013introduction}) but may have non-empty pure-point part (\cite[Chapter 4.5]{berkolaiko2013introduction}) --- a non-empty pure-point spectrum implies the existence of a compactly supported eigenfunction for the graphs considered in this work.

Using corollary \ref{cory:M-MatrixEntriesNoPoles}, the following proposition can be proved.
\begin{prop} \label{prop:MMatrixDetForm}
	Given a graph $\graph = \bracs{\vertSet, \edgeSet}$, with the lengths of the edges of $\graph$ pairwise-irrationally related, there exists a function $F\bracs{\qm,\omega}$ such that
	\begin{align} \label{eq:MMatrixDetForm}
		\det\mathfrak{M}_\qm\bracs{\omega^2} = \bracs{ \omega H^{(2)}\bracs{\omega^2} }^{\abs{\vertSet}-2} F\bracs{\qm,\omega}.
	\end{align}
	Furthermore, $F\bracs{\qm,\omega}$ is analytic in both its arguments.
\end{prop}
The proof of this result can be found in section \ref{sec:ProofOfProp}, but essentially involves counting the number of times a given factor can appear in the expression for the determinant.
The prefactor $\bracs{ \omega H^{(2)}\bracs{\omega^2} }^{\abs{\vertSet}-2}$ informs us that checking \eqref{eq:EigenvalueBranchLimit} will be necessary for all the roots of $H^{(2)}$.
The set $F_0 := \clbracs{\omega \setVert \exists\qm \text{ s.t. }F\bracs{\qm, \omega}=0}$ then determines the remainder of $\sigma$ --- so in particular, the function $F$ determines the ``width" of the spectral bands, or the vast majority of the spectrum.
Finding $\sigma$ (up to checking roots of $H^{(2)}$) now becomes a question of obtaining $F_0$ efficiently.
One can always take the ``brute-force" approach: compute $F_0^{\qm} := \clbracs{\omega \setVert F\bracs{\qm, \omega}=0}$ for each $\qm$ (or for each $\qm$ in a suitable mesh if working numerically), and then take the union over $\qm$ to obtain $F_0$.
This is computationally expensive (both numerically and analytically) and ``brute force" should only be a last resort, so we highlight some alternatives.

If we can write $F\bracs{\qm,\omega} = F_1\bracs{\qm} - F_2\bracs{\omega}$ for continuous $F_1$ and $F_2$, then \eqref{eq:QGDetSolveCondition} implies $F_0$ can be found simply by examining
\begin{align*}
	\min_{\qm}\clbracs{F_1(\qm)} \leq F_2\bracs{\omega} \leq \max_{\qm}\clbracs{F_1(\qm)},
\end{align*} 
although such a separation of $F$ will not generally be possible, and this also relies finding an analytic expression for $\det\mathfrak{M}_\qm$.
One can always ask the more general question of whether, given the knowledge that $\sigma$ has a band-gap structure, an alternative to finding all $\omega\in F_0$ is to only compute the spectral edges $a_j, b_j$ and then reconstruct $F_0$ from them.
It is known for (second order) periodic PDE problems that the edges of the spectrum occur at the symmetry values of the quasi-momentum --- those values of $\qm$ which correspond to the periodic and anti-periodic problems (in each axis direction) on the unit cell.
If the above statement were true for (periodic) quantum graphs, then the dimensionality of the problem of computing $F_0$ (hence $\sigma$) could be reliably reduced to determining $F_0^\qm$ for the aforementioned symmetry values of $\qm$.
However as discussed in \cite[Chapter 4.6]{berkolaiko2013introduction}, whilst this has been experimentally observed to be true for most physically motivated quantum graph topologies, but is in fact untrue in general.
Despite this, \cite[Chapter 4.6]{berkolaiko2013introduction} remarks that adopting this approach of assuming the spectral edges lie at the symmetry points of the quasi-momentum doesn't often lead to errors in practice, although it is unclear why.
The ``working hypothesis" or ``rule of thumb" is that the period graph needs to be made (very) asymmetric to move the spectral edges away from the symmetry points of the quasi-momentum, and since most physical structures of interest display symmetries in their unit cells, the statement appears to be ``true in practice".

The discussion in the previous paragraph focused on obtaining $\sigma$ and $\sigma_\qm$ via the solution to \eqref{eq:QGDetSolveCondition}.
Here we discuss some situations where it is more appropriate to use \eqref{eq:QGGenEvalSolveNoPoles} to determine $\sigma_{\qm}$ (and thus $\sigma$), although the discussion concerning the determination of the spectral edges $a_j, b_j$ is still applicable to solution methods that work via \eqref{eq:QGGenEvalSolveNoPoles}.
If one is looking to numerically solve for $\sigma_\qm$, then the default choice should be to solve the generalised eigenvalue problem \eqref{eq:QGGenEvalSolveNoPoles}.
Eigenvalue-finding schemes usually benefit from knowing that $\mathfrak{M}_\qm$ is Hermitian, and \eqref{eq:QGGenEvalSolveNoPoles} avoids having to work with $\det\mathfrak{M}_\qm$ --- which is generally advisable when handling matrices numerically. 
On the other hand, if one is willing to perform some analytic work, solving via \eqref{eq:QGDetSolveCondition} offers the possibility of finding an expression for $\det\mathfrak{M}_\qm$ in terms of rational functions ($\sin, \cos$, etc) on which less expensive numerical schemes can be used, or determining $\sigma_\qm$ outright.
As such, the choice of approach largely depends on how willing the solver is to work analytically with $\mathfrak{M}_\qm$.
With this in mind, corollary \ref{cory:M-MatrixEntriesNoPoles} tells us that the complexity of the entry of $\bracs{\mathfrak{M}_\qm}_{jk}$ depends on the number of edges between the vertices $v_j$ and $v_k$, whilst the complexity of the diagonal entries depends on the degree of the vertex $v_j$.
Furthermore, the number of vertices in the graph determines the dimensions of $\mathfrak{M}_\qm$, and the sparsity of $\mathfrak{M}_\qm$ depends on the number of pairs of vertices that are not (directly) connected by an edge.
This leads to the colloquial rule that solving \eqref{eq:QGDetSolveCondition} (analytically) is generally manageable for graphs with a ``small number" of edges and/or vertices, whilst graphs with a large number of edges and/or vertices often warrant solution via \eqref{eq:QGGenEvalSolveNoPoles}.
Of course, the cut-off for the terms ``small number" and ``large number" will depend on the person (or program) solving the problem themselves!
With this in mind, the ratio $\frac{\abs{\edgeSet}}{\abs{\vertSet}}$ can also be a good indicator the ``sparsity" of $\mathfrak{M}_\qm$, and thus how difficult $\mathfrak{M}_\qm$ will be to work with.
If $\frac{\abs{\edgeSet}}{\abs{\vertSet}}\approx 1$, then the number of non-zero entries in any row (or column) of $\mathfrak{M}_\qm$ should be approximately 2 --- the diagonal entry plus the entry corresponding to the expected single connection of this vertex.
At the extremes, $\frac{\abs{\edgeSet}}{\abs{\vertSet}}\approx 0$ corresponds to an almost-diagonal $\mathfrak{M}_\qm$, whilst $\frac{\abs{\edgeSet}}{\abs{\vertSet}}\approx \abs{\vertSet}$ corresponds to an almost-dense $\mathfrak{M}_\qm$.
A sparse $\mathfrak{M}_\qm$ can be easy to handle analytically and efficient to work with numerically, even if $\abs{\vertSet}$ is relatively large.
For denser $\mathfrak{M}_\qm$, one likely begins to lean toward numerical schemes, as the expressions involved in \eqref{eq:QGDetSolveCondition} becomes increasingly cumbersome.