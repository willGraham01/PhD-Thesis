\section{Chapter Introduction} \label{sec:SingIncChapterIntro}

The focus of the previous chapters \tstk{refs} rested solely on the singular structure itself and how to understand the notion of a derivative (or curl) on such a structure.
Once established, it was shown that a quantum graph problem can be obtained from a more abstract variational formulation, and the spectrum of such a problem analysed via tools like the $M$-matrix.
This choice to ignore or neglect the ``bulk" of our domain ($\ddom\setminus\graph$) reflects a physical situation where we expect there to be no field (or propagation) in this region.
In the context of electromagnetism for example, this would represent some (singular) dielectric material surrounded by conductors --- there would be no field in the conducting regions, and only ``along" the dielectric materials.
Of course, photonic crystals (and the fibres they are manufactured into) do not consist of a (thin, periodic) dielectric encased in a conducting material; rather they are composed of a (thin, periodic) dielectric material surrounded by \emph{another} dielectric material.
This is where the motivation for our final problem stems from --- we will examine a problem on a (two dimensional) domain with \emph{singular} inclusions.
We would like to view such a problem as (some appropriate) limit of a problem on a domain with thin-structure inclusions, as the thickness of such inclusions tends to zero.
There have been studies related to this topic, \tstk{Kirill-Evans, Zhikov-Past. on how what we're doing here is similar to this stuff, but is not it exactly.}

\subsection{Notation and Conventions} \label{ssec:SINotation}
Let us formalise the notation and terminology we will us to describe the domain we will be working with in this section.
As usual, we take $\ddom$ and $\graph$ to be a singular structure domain, with $\graph$ being the period graph of some embedded, metric graph and $\ddom$ the unit cell, which is identified with the torus \tstk{as in section where this was all established}.
The domain $\ddom$ will henceforth be referred to as the ``composite domain".
The graph $\graph$ and the periodic boundaries of $\ddom$ naturally break $\ddom$ into a disjoint union of (open) polygonal regions (or subdomains); we label these $\ddom_i$ for $i\in\Lambda$ for some appropriate (finite) index set $\Lambda$, and we then have that $\ddom = \graph\cup\bigcup_{i\in\Lambda}\ddom_i$.
We will refer to the edges $I_{jk}\in\edgeSet$ as the ``(singular) inclusions", and refer to the $\ddom_i$ as the ``bulk (regions)" or ``(dielectric) composite regions".
Additionally, we denote by $\lambda_2$ the two-dimensional Lebesgue measure on $\ddom$ and write
\begin{align*}
	\compMes := \lambda_2 + \ddmes,
\end{align*}
where we shall refer to $\compMes$ as the ``composite" measure on $\ddom$ with respect to the graph $\graph$.
\tstk{haven't done $\nu$ additions yet, mention this because using the tilde notation is somewhat unfortunate.}
Also, define $\compMes_{jk} = \lambda_2 + \lambda_{jk}$ for each edge $I_{jk}$.

Note that our singular inclusions are not merely interfaces; at an interface one simply has matching conditions between the solutions (to a suitable differential problem) approaching from one side of the interface and from the other.
The interface itself has no size or bulk, and there are no dynamics happening along these interfaces beyond the matching conditions imposed --- the behaviour of the solution is determined in the bulk, and then matched to the expected (or physically relevant) interface conditions.
Singular inclusions on the other hand retain a notion of length and thus will give rise to dynamics along the edges of $\graph$, which will be coupled to the dynamics in the composite regions $\ddom_i$.
That is to say, the behaviour of a solution is not determined by the behaviour in the bulk regions, then matched to the other regions via the inclusions which separate them.
In fact, we will see that it is even possible to reformulate a problem on the composite domain into a problem posed solely on the graph $\graph$, where the interplay between the solution in the bulk and on the edges is en1coded in the non-locality of the resulting problem.