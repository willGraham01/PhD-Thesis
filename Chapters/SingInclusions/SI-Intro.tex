\section{Chapter Introduction} \label{sec:SingIncChapterIntro}

The focus of the previous chapters \tstk{refs} rested solely on the singular structure itself and how to understand the notion of a derivative (or curl) on such a structure.
Once established, it was shown that a quantum graph problem can be obtained from a more abstract variational formulation, and the spectrum of such a problem analysed via tools like the $M$-matrix.
This choice to ignore or neglect the ``bulk" of our domain ($\ddom\setminus\graph$) reflects a physical situation where we expect there to be no field (or propagation) in this region.
In the context of electromagnetism for example, this would represent some (singular) dielectric material surrounded by conductors --- there would be no field in the conducting regions, and only ``along" the dielectric materials.
Of course, photonic crystals (and the fibres they are manufactured into) do not consist of a (thin, periodic) dielectric encased in a conducting material; rather they are composed of a (thin, periodic) dielectric material surrounded by \emph{another} dielectric material.
This is where the motivation for our final problem stems from --- we will examine a problem on a (two dimensional) domain with \emph{singular} inclusions.
We would like to view such a problem as (some appropriate) limit of a problem on a domain with thin-structure inclusions, as the thickness of such inclusions tends to zero.
There have been studies related to this topic, \tstk{Kirill-Evans, Zhikov-Past. on how what we're doing here is similar to this stuff, but is not it exactly. Highlight that this means we are shooting in the dark a bit, as we have nothing to go off of!}

\subsection{Notation and Conventions} \label{ssec:SINotation}
Let us formalise the notation and terminology we will us to describe the domain we will be working with in this section.
As usual, we take $\ddom$ and $\graph$ to be a singular structure domain, with $\graph$ being the period graph of some embedded, metric graph and $\ddom$ the unit cell, which is identified with the torus \tstk{as in section where this was all established}.
The domain $\ddom$ will henceforth be referred to as the ``composite domain".
The graph $\graph$ naturally breaks $\ddom$ into a collection of disjoint union of (open) polygonal regions (or subdomains); we label these $\ddom_i$ for $i\in\Lambda$ for some appropriate (finite) index set $\Lambda$, and we then have that $\ddom = \graph\cup\bigcup_{i\in\Lambda}\ddom_i$.
We will refer to the graph $\graph$ and its constituent edges $I_{jk}\in\edgeSet$ as the ``(singular) skeleton", and refer to the $\ddom_i$ as the ``bulk (regions)" or ``(dielectric) composite regions"\footnote{We neglect to use the term ``inclusion" for either of $\ddom_i$ or $\graph$ since there is an argument to be made both ways about which material is being ``included" in the other.}.
Additionally, we denote by $\lambda_2$ the two-dimensional Lebesgue measure on $\ddom$ and write
\begin{align*}
	\compMes := \lambda_2 + \ddmes,
\end{align*}
where we shall refer to $\compMes$ as the ``composite" measure on $\ddom$ with respect to the graph $\graph$.
\tstk{haven't done $\nu$ additions yet, mention this because using the tilde notation is somewhat unfortunate.}
Also, define $\compMes_{jk} = \lambda_2 + \lambda_{jk}$ for each edge $I_{jk}$.

Note that our singular inclusions are distinct from boundaries or material interfaces; at an interface one simply has matching conditions between the solutions (to a suitable differential problem) approaching from one side of the interface and from the other.
The interface itself has no size or bulk, and there are no dynamics happening along these interfaces beyond the matching conditions imposed --- the behaviour of the solution is determined in the bulk, and then matched to the expected (or physically relevant) interface conditions.
On the other hand, our skeleton is bestowed a notion of length by $\ddmes$, and thus has the potential to (and does) give rise to dynamics along the edges of $\graph$, which will be coupled to the dynamics in the composite regions $\ddom_i$.
That is to say, the behaviour of a solution is no longer determined by the behaviour in the bulk regions, then matched to the other regions via the inclusions which separate them.
In fact, we will see that it is even possible to reformulate a problem on the composite domain into a problem posed solely on the graph $\graph$, where the interplay between the solution in the bulk and on the edges is encoded in the non-locality of the resulting problem.

\subsection{Chapter Overview} \label{ssec:SIOverview}
The focus of this chapter will be on the ``Helmholtz-equation" (thought of as a Fourier transformed ``wave-equation"),
\begin{align} \label{eq:SI-WaveEqn}
	-\bracs{\tgrad_{\compMes}}^2 u = \omega^2 u \qquad\text{in } \ddom,
\end{align}
now posed on our composite domain and respecting our singular skeleton.
Our goal of studying the spectrum of \eqref{eq:SI-WaveEqn} will direct our analysis in this chapter as follows; foremost, \tstk{section ref where appropriate} we establish formal definitions of the appropriate function space for $u$ and the operator for which \eqref{eq:SI-WaveEqn} is the eigenvalue problem of, motivated by our approaches in chapters \tstk{scalar and curl}.
This ensures that there \emph{are} eigenfunctions and eigenvalues of \eqref{eq:SI-WaveEqn} to talk about, and we then move onto considerations for determining these.
In doing so, we discover that \eqref{eq:SI-WaveEqn} possesses several equivalent formulations, each of which will be the basis of a numerical scheme with benefits and hindrances relative to the other formulations.
The first such formulation we consider  \tstk{add subsections!} comes directly from the variational problem for the operator that defines \eqref{eq:SI-WaveEqn}, and the second comes from the corresponding ``strong form" that we can derive using analysis of the function space that $u$ lives in.
These formulations still require us to work with the unfamiliar gradients ($\tgrad_{\compMes}u$) or handle interplay between the solution in the bulk and on the skeleton, which brings us to the third formulation in section \tstk{ref!} where we formulate \eqref{eq:SI-WaveEqn} on the skeleton only.

Our investigation into each of these formulations will highlights several ``trade-offs" that are made as we move between the various formulations or numerical approaches to solving \eqref{eq:SI-WaveEqn}.
For example, moving towards a problem on the skeleton only allows us to avoid handling tangential gradients with respect to $\compMes$ (and other non-classical objects) and theoretically reduces the dimensionality of any numerical scheme we want to employ because the skeleton is 1D.
On the other hand, moving onto the skeleton also results in the introduction of non-local effects into the equations on each $I_{jk}$, to compensate for the effect of the bulk regions, which complicates the solutions process.
We also establish a link between the first and second formulations by means of \tstk{motivated by the use of a Strauss dilations, extended spaces}, and speculate on the affect of introducing non-zero coupling constants at the vertices.