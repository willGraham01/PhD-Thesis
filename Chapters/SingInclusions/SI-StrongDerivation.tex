\section{``Strong Formulation" of \eqref{eq:SI-WaveEqn}} \label{sec:SI-StrongDerivation}
Whilst we can choose to work directly from the min-max principle \eqref{eq:SI-VarProb} in an attempt to solve \eqref{eq:SI-WaveEqn}, we still do not have any explicit insights into how the presence of the skeleton is affecting the (solutions and eigenvalues of) this problem.
It is also desirable for us to move away from working with (approximations to) the tangential gradients themselves --- compared to classical gradients and derivatives, we do not have many tools to handle these objects numerically.
Therefore, in this section we attempt to derive a ``strong" formulation for \eqref{eq:SI-WaveEqn}, with our motivations similar to those of sections \ref{sec:ScalarDerivation} and \ref{sec:3DSystemDerivation} --- we want to be able to analyse a more tractable problem in terms of classical objects.

Before we begin, we should consider what our intuition is telling us about the behaviour we expect from any solutions $u$ to \eqref{eq:SI-WeakWaveEqn}.
A good starting point is to consider how we expect our solutions to behave if we could (n\"{i}avely) interpret \eqref{eq:SI-WeakWaveEqn} in a strong sense.
Away from the skeleton, \eqref{eq:SI-WeakWaveEqn} looks like the usual Helmholtz problem on a bounded domain, and so we expect our solution to possess sufficient regularity to be differentiated twice in the bulk.
We also know that solutions to the acoustic approximation on the inclusions possess two derivatives along the edges of $\graph$ (section \ref{sec:ScalarDerivation}), and are tied together through the vertex conditions.
Finally, we must consider what should happen in the vicinity of the skeleton --- here we have (what we expect to be) a twice differentiable function in a bulk region $\ddom_i$ approaching its boundary, and so there should be ($L^2$) traces of $u$ and its normal derivative onto this boundary.
However, this boundary coincides with (a subregion of) the skeleton, so the function $u$ should ``feel" the affect of these traces as it moves along the skeleton.
A partial converse is also expected; $u$ is twice differentiable along the skeleton, and given that $u$ \emph{also} has a trace onto the skeleton, we expect that these traces should be consistent with the function values from the bulk.
as such, we expect that \eqref{eq:SI-WeakWaveEqn} can be reformulated into a system that consists of the following components:
\begin{enumerate}[(a)]
	\item A (Helmholtz-like) PDE in each of the bulk regions, the solution to which has boundary traces matching the solution to a quantum graph problem on the inclusions.
	\item A 2nd-order quantum graph problem on the singular inclusions, with the edge ODEs involving or being influenced by the traces from the bulk regions.
	\item Conditions at the vertices of the graph to tie the quantum graph problem, and hence the PDE problems, together.
\end{enumerate}

At a glance, solutions $u$ to \eqref{eq:SI-WeakWaveEqn} appear to be less regular than what we expect from our intuitive arguments above, and \eqref{eq:SI-WeakWaveEqn} itself is of course not in the form (a)-(c).
However, much like in sections \ref{sec:ScalarDerivation} and \ref{sec:3DSystemDerivation} we can work from \eqref{eq:SI-WeakWaveEqn} and the definition of $\tgradSob{\ddom}{\ccompMes}$ to obtain a system as described by (a)-(c).
Our starting point is the problem \eqref{eq:SI-WeakWaveEqn}, repeated here for ease of reading: find $\omega^2>0$ and non-zero $u\in\tgradSob{\ddom}{\ccompMes}$ such that
\begin{align*}
	\integral{\ddom}{ \tgrad_{\ccompMes}u\cdot\overline{\tgrad_{\ccompMes}\phi} }{\ccompMes}
	&= \omega^2 \integral{\ddom}{ u\overline{\phi} }{\ccompMes}, \quad\forall\phi\in\psmooth{\ddom}. \tag{\eqref{eq:SI-WeakWaveEqn} restated}
\end{align*}
We will need to make use of several standard integral identities, which we summarise below.
Let $D$ be an open Lipschitz domain, let $u\vert_{\partial D}$ denote the trace (of a suitably regular) function $u$ on $D$ into $\ltwo{\partial D}{S}$, and $n^D$ denote the exterior normal on the boundary of $D$.
\begin{itemize}
	\item For $u,v\in\gradSob{D}{\lambda_2}$ and $j\in\clbracs{1,2}$,
	\begin{align*}
		\integral{D}{ v\partial_j u + u\partial_j v }{\lambda_2}
		&= \integral{\partial D}{ u v n^D_j }{S}.
	\end{align*}
	\item For $u\in H^2\bracs{D,\lambda_2}, v\in\gradSob{D}{\lambda_2}$,
	\begin{align*}
		\integral{D}{ \grad u\cdot \grad v }{\lambda_2} 
		&=  - \integral{D}{ v\laplacian u }{\lambda_2} + \integral{\partial D}{ v\vert_{\partial D}\pdiff{u}{n^D}\vert_{\partial D} }{S}.
	\end{align*}
\end{itemize}
From the above, we can deduce that whenever $u\in H^2\bracs{D,\lambda_2}$ and $v\in\gradSob{D}{\lambda_2}$, we have that
\begin{align*}
	\integral{D}{ \tgrad u\cdot\overline{\tgrad v} }{\lambda_2}
	&= - \integral{D}{ \overline{v}\tgrad\cdot\tgrad u }{\lambda_2} + \integral{\partial D}{ \overline{v}\vert_{\partial D}\bracs{\tgrad u\cdot n^D}\vert_{\partial D} }{S}.
\end{align*}

We now begin the reformulation, throughout let us assume $\omega^2>0$ and $u\in\tgradSob{\ddom}{\ccompMes}$ solve \eqref{eq:SI-WeakWaveEqn}.
Suppose that the test function $\phi$ in \eqref{eq:SI-WeakWaveEqn} has support contained within one of the bulk regions $\ddom_i$.
This implies that \eqref{eq:SI-WeakWaveEqn} becomes
\begin{align*}
	\omega^2\integral{\ddom_i}{u\overline{\phi}}{\lambda_2} 
	&= \integral{\ddom_i}{ \grad u\cdot\overline{\grad\phi} - \rmi\qm\overline{\phi}\cdot\tgrad u + \rmi\qm  u\cdot\overline{\grad\phi} - \rmi^2\qm\cdot\qm u\overline{\phi} }{\lambda_2} \\
	&= \integral{\ddom_i}{ \grad u\cdot\overline{\grad\phi} - 2\rmi\qm\overline{\phi}\cdot\tgrad u - \rmi^2\qm\cdot\qm u\overline{\phi} }{\lambda_2}, \\ 
	&= \integral{\ddom_i}{ \bracs{\omega^2 u + 2\rmi\qm\cdot\tgrad u + \rmi^2\qm\cdot\qm u} \overline{\phi} }{\lambda_2}, 
\end{align*}
which holds for all smooth $\phi$ with compact support in $\ddom_i$.
Given that we also know that $u$ is $\ltwo{\partial\ddom_i}{S}$ (and is even $H^1$ in this space), this implies that $u\in \gradgradSob{\ddom_i}{\lambda_2}$ with
\begin{subequations} \label{eq:SI-StrongForm} % This is so that the Strong System is coherently numbered, this spans most of the derivation so be careful!
\begin{align*}
	\laplacian u &= -\bracs{ \omega^2 u + 2\rmi\qm\cdot\tgrad u + \rmi^2\qm\cdot\qm u } &\qquad\text{in } \ddom_i,
\end{align*}
so we let $\laplacian_{\qm}:= \laplacian u + 2\rmi\qm\cdot\tgrad u + \rmi^2\qm\cdot\qm u$ and have that
\begin{align*}
	\laplacian_\qm u &= -\omega^2 u &\qquad\text{in } \ddom_i. \labelthis\label{eq:SI-BulkEqn}
\end{align*}
The additional regularity of the solution $u$ in the bulk regions provides equation \eqref{eq:SI-BulkEqn}, which matches our expectations in (a) of $u$ satisfying a Helmholtz-like equation in the bulk regions.

Next, we turn to addressing what happens when we lie in the vicinity of an edge $I_{jk}\in\edgeSet$.
For this, we need to introduce a local labelling system for the bulk regions that are adjacent to $I_{jk}$, as follows.
Let $\ddom_{jk}^+$ be the bulk region whose boundary has non-empty intersection with $I_{jk}$ and whose exterior unit normal on $\partial\ddom_{jk}^+\cap I_{jk}$ is equal to $-n_{jk}$.
Similarly let $\ddom_{jk}^-$ be the bulk region whose boundary has non-empty intersection with $I_{jk}$ and whose exterior unit normal on $\partial\ddom_{jk}^-\cap I_{jk}$ is equal to $n_{jk}$.
Note the sign convention; this is chosen because the region $\ddom_{jk}^+$ is ``to the right" of $I_{jk}$ as viewed from the local coordinate system $y_{jk}=\bracs{n_{jk}, e_{jk}}$, and $\ddom_{jk}^-$ is ``on the left" --- see figure \ref{fig:Diagram_SI-AdjacentBulkRegions}.
\begin{figure}[b!]
	\centering
	\includegraphics[scale=1.0]{Diagram_SI-AdjacentBulkRegions.pdf}
	\caption[Labelling convention for bulk regions adjacent to a skeleton edge.]{\label{fig:Diagram_SI-AdjacentBulkRegions} Labelling convention for regions adjacent to an edge $I_{jk}$.}
\end{figure}
Now consider \eqref{eq:SI-WeakWaveEqn} when $\phi$ is taken to have compact support that intersects (the interior of) an edge $I_{jk}$, the adjacent bulk regions $\ddom_{jk}^+$ and $\ddom_{jk}^-$, and no other parts of $\ddom$.
Equation \eqref{eq:SI-WeakWaveEqn} then implies that
\begin{align*}
	\integral{\ddom}{ \omega^2 u\overline{\phi} - \tgrad_{\lambda_{jk}}u\cdot\overline{\tgrad_{\lambda_{jk}}\phi} }{\lambda_{jk}}
	&= \integral{\ddom}{ \tgrad u\cdot\overline{\tgrad\phi} - \omega^2 u\overline{\phi} }{\lambda_2} \\
	&= \integral{\ddom_{jk}^+}{ \tgrad u\cdot\overline{\tgrad\phi} - \omega^2 u\overline{\phi} }{\lambda_2}
	+ \integral{\ddom_{jk}^-}{ \tgrad u\cdot\overline{\tgrad\phi} - \omega^2 u\overline{\phi} }{\lambda_2}.
\end{align*}
Next, we know that $u\in \gradgradSob{\ddom_{jk}^{\pm}}{\lambda_2}$ for both $\ddom_{jk}^+$ and $\ddom_{jk}^-$, and so $u$ and its normal derivative possess an $L^2$-trace onto $I_{jk}$.
Using the notation $\tgrad u\cdot n_{jk} = \pdiff{u}{n_{jk}} + \rmi\qm u\cdot n_{jk}$; and denoting the trace of $u\in\gradgradSob{\ddom_{jk}^{\pm}}{\lambda_2}$ onto the boundary $\partial\ddom^{\pm}$ by $u^{\pm}$, we have that
\begin{align*}
	\integral{\ddom}{ \omega^2 u\overline{\phi} - \tgrad_{\lambda_{jk}}u\cdot\overline{\tgrad_{\lambda_{jk}}\phi} }{\lambda_{jk}}
	&= \integral{\ddom_{jk}^+}{ -\overline{\phi}\bracs{ \laplacian_{\qm} u + \omega^2 u } }{\lambda_2} \\
	&\qquad + \integral{\ddom_{jk}^-}{ -\overline{\phi}\bracs{ \laplacian_{\qm} u + \omega^2 u } }{\lambda_2} \\
	&\qquad + \integral{\partial\ddom_{jk}^+}{ -\overline{\phi}\bracs{\tgrad u\cdot n_{jk}}^{+} }{S} \\
	&\qquad + \integral{\partial\ddom_{jk}^-}{ \overline{\phi}\bracs{\tgrad u\cdot n_{jk}}^{-} }{S},
\end{align*}
since the exterior normal to $\ddom_{jk}^{\pm}$ is $\mp n_{jk}$.
Given \eqref{eq:SI-BulkEqn} and the support of $\phi$, this further implies that
\begin{align*}
	\integral{\ddom}{ \omega^2 u\overline{\phi} - \tgrad_{\lambda_{jk}}u\cdot\overline{\tgrad_{\lambda_{jk}}\phi} }{\lambda_{jk}}
	&= \integral{I_{jk}}{ \overline{\phi}\sqbracs{\bracs{\tgrad u\cdot n_{jk}}^- - \bracs{\tgrad u\cdot n_{jk}}^+} }{S} \\
	&= \int_0^{l_{jk}} \overline{\phi}\sqbracs{\bracs{\tgrad u\cdot n_{jk}}^- - \bracs{\tgrad u\cdot n_{jk}}^+} \ \md y.
\end{align*}
Changing variables via $r_{jk}$ in the integral on the left hand side, substituting the known form for the tangential gradients, and rearranging then provides us with
\begin{align*}
	\int_0^{l_{jk}} \bracs{u^{(jk)}}'\overline{\phi}' \ \md y
	&= \int_0^{l_{jk}} \overline{\phi}\sqbracs{ \bracs{\tgrad u\cdot n_{jk}}^- - \bracs{\tgrad u\cdot n_{jk}}^+ \right. \\
	&\qquad \left. - \omega^2 u^{(jk)} - 2\rmi\qm_{jk}\bracs{u^{(jk)}}' - \bracs{\rmi\qm_{jk}}^2 u^{(jk)} } \ \md y,
\end{align*}
which holds for all smooth $\phi$ with support contained in the interior of $I_{jk}$.
Thus, we can deduce that $u^{(jk)}\in\gradgradSob{\interval{I_{jk}}}{y}$, and that
\begin{align*}
	- \bracs{\diff{}{y} + \rmi\qm_{jk}}^2u^{(jk)} 
	&= \omega^2 u^{(jk)} + \bracs{\tgrad u\cdot n_{jk}}^+ - \bracs{\tgrad u\cdot n_{jk}}^-,
	&\qquad\text{in } \sqbracs{0,l_{jk}}.
\end{align*}
If we additionally recall that the trace of $u$ from the bulk regions $\ddom_{jk}^{\pm}$ is equal to $u^{(jk)}$, we can eliminate part of the trace-terms to obtain
\begin{align} \label{eq:SI-InclusionEqn}
	- \bracs{\diff{}{y} + \rmi\qm_{jk}}^2u^{(jk)} 
	&= \omega^2 u^{(jk)} + \bracs{\grad u\cdot n_{jk}}^+ - \bracs{\grad u\cdot n_{jk}}^-,
	&\qquad\text{in } I_{jk}.
\end{align}
This provides us with part (b) from our intuitive argument --- on the edges of the graph we have the second-order differential equation from chapter \ref{ch:ScalarSystem}, but with the addition of a term capturing the differences in the trace of the normal derivative of $u$ from either side of the inclusion.
It is worth remarking how, if our inclusions were merely interfaces, we would simply obtain an algebraic equation in the difference of the normal derivative traces on the $I_{jk}$.
Giving the edges a notion of length, even though it is 1-dimensional length within a 2-dimensional domain, has resulted in this difference (or ``jump" in the normal derivatives) directly influencing the behaviour of $u$ on the inclusions.
Conversely, the requirement that the traces of $u$ from $\ddom_{jk}^{\pm}$ be equal to $u^{(jk)}$ also means that the behaviour of $u$ on the inclusions will affect the solution in the bulk regions.
This means we have something resembling a ``feedback loop"; the solution in the bulk exerts influence on the edges through the traces of the normal derivatives, and the solution on the inclusions exerts influence on the bulk via the requirement that the traces coincide with the values on the inclusion.

Finally, we consider the solution $u$ to \eqref{eq:SI-WeakWaveEqn} in the vicinity of a vertex, or more precisely when $\phi$ has support containing a vertex $v_j$ (and without loss of generality, no other vertices).
The process is straightforward; we aim to proceed as before and use \eqref{eq:SI-BulkEqn} and \eqref{eq:SI-InclusionEqn} to cancel terms on the inclusions and in the bulk regions, leaving us with a ``vertex condition", however we require one final set of temporary notation to transcribe the argument.
Fix $v_j\in\vertSet$ and consider the junction region $J\bracs{v_j}$.
For each $I_{jk}\in J(v_j)$ let $\beta_{jk}$ be the anticlockwise angle between the segment $I_{jk}$ and the $v_j+\hat{x}_1$ direction.
The $\clbracs{\beta_{jk}}$ can then be ordered by size, and correspondingly we can also order the $I_{jk}\in J(v_j)$, writing
\begin{align*}
	\beta_{jk_1} < \beta_{jk_2} < ... < \beta_{jk_{\deg(v_j)}}, 
	\qquad I_{jk_1} < I_{jk_2} < ... < I_{jk_{\deg(v_j)}}.
\end{align*}
Also adopt a cyclic convention, where $k_0 = k_{\deg(v_j)}$ and $k_{\deg(v_j)+1} = k_1$.
Now, for each $l\in\clbracs{1,...,\deg(v_j)}$ let $\ddom_{jk_l}$ be the bulk region that lies between (in the sense of the angles $\beta_{jk_{l-1}}$ and $\beta_{jk_l}$) $I_{jk_{l-1}}$ and $I_{jk_l}$.
This labelling can be visualised in figure \ref{fig:Diagram_SI-JunctionLabelling}.
\begin{figure}[b!]
	\centering
	\includegraphics[scale=1.5]{Diagram_SI-JunctionLabelling.pdf}
	\caption[Labelling convention for bulk regions surrounding a vertex.]{\label{fig:Diagram_SI-JunctionLabelling} The labelling conventions for the bulk regions and edges surrounding a vertex.}
\end{figure}

Fix $v_j\in\vertSet$ and let $\phi\in\psmooth{\ddom}$ be such that $\supp(\phi)\subset\bigcup_{l}\ddom_{jk_l}$, and have $v_j\in\supp(\phi)$.
With such a $\phi$, the following equalities hold;
\begin{align*}
	\integral{\ddom}{ \tgrad u\cdot\overline{\tgrad\phi} - \omega^2u\overline{\phi} }{\lambda_2}
	&= \sum_{l} \integral{\ddom_l}{ \tgrad u\cdot\overline{\tgrad\phi} - \omega^2u\overline{\phi} }{\lambda_2} \\
	&= \sum_{l} \integral{\ddom_l}{ -\overline{\phi}\bracs{ \laplacian_{\qm} u + \omega^2 u } }{\lambda_2}
	+ \integral{\partial\ddom_l}{ \overline{\phi}\tgrad u\cdot n_{jk_l} }{S} \\
	&= \sum_l \integral{I_{jk_l}}{ \overline{\phi}\tgrad u\cdot n_{jk_l} }{S} + \integral{I_{jk_{l-1}}}{ \phi\tgrad u\cdot n_{jk_{l-1}} }{S} \\
	&= \sum_l \integral{I_{jk_l}}{ \overline{\phi}\bracs{ \bracs{\tgrad u\cdot n_{jk_l}}^- - \bracs{\tgrad u\cdot n_{jk_l}}^+ } }{S}, \\
	\integral{\ddom}{ \tgrad_{\ccompMes} u\cdot\overline{\tgrad_{\ccompMes}\phi} - \omega^2u\overline{\phi} }{\ddmes}
	&= \sum_l \integral{I_{jk_l}}{ -\overline{\phi}\bracs{\bracs{\diff{}{y}+\rmi\qm_{jk_l}}^2 u^{(jk_l)} + \omega^2 u^{(jk_l)}} }{\lambda_{jk_l}} \\
	&\qquad + \sum_l \sqbracs{ \bracs{\pdiff{}{n}+\rmi\qm_{jk_l}} u^{(jk_l)}(v_j)\overline{\phi}(v_j) }.
\end{align*}
Thus, equation \eqref{eq:SI-WeakWaveEqn} becomes
\begin{align*}
	\alpha_j\omega^2 u(v_j)\overline{\phi}(v_j)
	&= \sum_l \clbracs{ \integral{I_{jk_l}}{ \overline{\phi}\bracs{ \bracs{\tgrad u\cdot n_{jk_l}}^- - \bracs{\tgrad u\cdot n_{jk_l}}^+ } }{S} \right. \\
	&\qquad \left.	+ \integral{I_{jk_l}}{ -\overline{\phi}\bracs{\bracs{\diff{}{y}+\rmi\qm_{jk_l}}^2 u^{(jk_l)} + \omega^2 u^{(jk_l)}} }{\lambda_{jk_l}} \right. \\
	&\qquad \left.	+ \sqbracs{ \bracs{\pdiff{}{n}+\rmi\qm_{jk_l}} u^{(jk_l)}(v_j)\overline{\phi}(v_j) } } \\
	&= \sum_l \sqbracs{ \bracs{\pdiff{}{n}+\rmi\qm_{jk_l}} u^{(jk_l)}(v_j)\overline{\phi}(v_j) }
	&\quad\text{using \eqref{eq:SI-InclusionEqn}}, \\
	\implies \alpha_j\omega^2 u(v_j)\overline{\phi}(v_j)
	&= \sum_l \bracs{\pdiff{}{n}+\rmi\qm_{jk_l}} u^{(jk_l)}(v_j), \labelthis\label{eq:SI-VertexCondition}
\end{align*}
\end{subequations} % last equation that we needed in the strong form has been provided, close off the subequations environment
since $\phi(v_j)$ is arbitrary.
In addition to continuity of $u$ at each of the vertices, we have found that $u$ also adheres to a non-classical Kirchoff-like condition at each of the vertices.

The system \eqref{eq:SI-StrongForm}\footnote{We call \eqref{eq:SI-StrongForm} ``strong" since it is no longer understood in a variational sense, and its relation to \eqref{eq:SI-WaveEqn} is akin to the relationship between the strong and weak forms of PDEs.}, 
\begin{subequations}
	\begin{align*}
		-\laplacian_\qm u 
		&= \omega^2 u 
		&\text{in } \ddom_i, \tag{\eqref{eq:SI-BulkEqn} restated} \\
		- \bracs{\diff{}{y} + \rmi\qm_{jk}}^2u^{(jk)}  
		&= \omega^2 u^{(jk)} + \bracs{\bracs{\grad u\cdot n_{jk}}^+ - \bracs{\grad u\cdot n_{jk}}^-}
		&\text{in } I_{jk}, \tag{\eqref{eq:SI-InclusionEqn} restated} \\
		\sum_{j\con k} \bracs{\pdiff{}{n}+\rmi\qm_{jk}} u^{(jk)}(v_j) 
		&= \alpha_j\omega^2 u(v_j)
		&\text{at } v_j\in\vertSet, \tag{\eqref{eq:SI-VertexCondition} restated}
	\end{align*}
\end{subequations}
combined with the knowledge that $u^{(jk)}$ matches the traces of $u$ from the adjacent bulk regions, provides us with a reformulated system reflecting our intuition from (a)-(c).
Indeed, the system is composed of two differential equations (the PDE \eqref{eq:SI-BulkEqn} on the bulk, and the ODE \eqref{eq:SI-InclusionEqn} on the skeleton) coupled through the vertex condition \eqref{eq:SI-VertexCondition} and the requirement that the traces from adjacent regions match the function values on the skeleton.
The objects that we are required to work with are more familiar to us, being a combination of classical gradients and those studied in section \ref{sec:3DGradSobSpaces}. 
We also expect it will be easier for us to work with this system both analytically and numerically, due to the interaction between the bulk regions and skeleton being made explicit.
There are however two ``scales" on which a solution $u$ behaves, being 2D in the bulk and 1D on the skeleton, and we need resolution in both of them.
On the matter of consistency between solutions to the problem \eqref{eq:SI-VarProb} and \eqref{eq:SI-StrongForm}, note that \eqref{eq:SI-StrongForm} is also the result of applying the theorem of Lagrange multipliers to \eqref{eq:SI-VarProb}.

It is worth making some remarks here to reconnect \eqref{eq:SI-StrongForm} with the physical (or ``dimensionfull") material it is supposed to represent.
The system \eqref{eq:SI-StrongForm} represents the acoustic approximation on a composite material whose \emph{material} properties in the bulk regions and on the skeleton are identical, but whose geometric properties are in contrast.
Indeed, one may find it odd to notice that the (dimensionless) parameters $\alpha_j$ only appear in one of the three equations, \eqref{eq:SI-VertexCondition}.
It is precisely this equation which links the behaviour of the solutions at the vertices to that on the edges. 
The two geometric properties of the physical material are the ``thickness" of the edges and that of the vertices, and these are the two properties the $\alpha_j$ represent a relationship between.
This is directly analogous to the situation one ends in when studying composite materials with contrasting material properties; the dimensionless parameters representing the contrast appear in (what is normally a boundary condition) relating the values of the solution across the interface between the inclusion and background material.
No such parameter appears in \eqref{eq:SI-StrongForm} because the physical material we are representing does not have any material contrast , only geometric contrast.
With this understanding it is not difficult to see how the introduction of such a contrast would affect \eqref{eq:SI-StrongForm}.
Let us now suppose there is some contrast in the material properties of the bulk regions and skeleton, represented through the (dimensionless) function
\begin{align*}
	\epsilon_m =
	\begin{cases} \epsilon & x\in\ddom_i, \\ 1 & x\in\graph, \end{cases}
\end{align*}
where $\epsilon$ is the dimensionless parameter representing the contrast between the material properties on the skeleton and the bulk.
One would then obtain a system of the form 
\begin{subequations}
	\begin{align*}
		-\epsilon^{-1}\laplacian_\qm u 
		&= \omega^2 u 
		&\text{in } \ddom_i, \\
		- \bracs{\diff{}{y} + \rmi\qm_{jk}}^2u^{(jk)}  
		&= \omega^2 u^{(jk)} + \epsilon^{-1}\bracs{\bracs{\grad u\cdot n_{jk}}^+ - \bracs{\grad u\cdot n_{jk}}^-}
		&\text{in } I_{jk}, \\
		\sum_{j\con k} \bracs{\pdiff{}{n}+\rmi\qm_{jk}} u^{(jk)}(v_j) 
		&= \alpha_j\omega^2 u(v_j)
		&\text{at } v_j\in\vertSet,
	\end{align*}
\end{subequations}
through studying the problem \eqref{eq:SI-WaveEqn} in the space $\tgradSob{\ddom}{\bracs{\epsilon^{-1}\lambda_2}^{\dddmes}}$, where $\bracs{\epsilon^{-1}\lambda_2}^{\dddmes} = \epsilon^{-1}\lambda_2 + \dddmes$.
The effects of material contrast can thus be introduced in an identical fashion to the effects of geometric contrast --- through ``adding" mass to one part of the medium and not to the others.
For geometric effects, mass $\alpha_j$ is ``added" to the vertices $v_j$.
For material contrast, mass $\epsilon^{-1}$ is ``taken away"\footnote{We could also introduce this affect by ``adding" mass to the skeleton, by considering $\epsilon\dddmes$ in place of $\dddmes$.} from the bulk regions $\ddom_i$. 

Our final comments on the system \eqref{eq:SI-StrongForm} pertain to the form of the ODEs on the skeleton \eqref{eq:SI-InclusionEqn}.
Notice that there is a coupling between the flux across the skeleton and the ``diffusion" (the second-derivative operator) along the skeleton.
Analogous boundary conditions (and equations in the bulk) are observed in the study \cite{cherednichenko2019homogenisation} for a similar material setup, but considering the effective problem for the equations of (linearised) elasticity as the period cell size shrinks to zero.
The effective problem obtained in this study also consisted of equations on the bulk regions, with a coupling between diffusion along and flux across the interfaces between the two components.
It is noted in this study that such conditions, which coupling diffusion along a boundary to diffusion along said boundary, belong to the class of \tstk{Ventcel' conditions, reference}.
The material considered in \cite{cherednichenko2019homogenisation} did not possess geometric contrast (between the vertex and edge volume) however, so there is no analogous equation to \eqref{eq:SI-VertexCondition} appears in the effective problem.
However us having obtained analogous conditions in the ``electromagnetic" setting provides further evidence that our motivation for studying \eqref{eq:SI-SingularWaveEqn} was well-founded; obtaining the system \eqref{eq:SI-StrongForm} and the analogous results in the elasticity context suggest that it is the limiting problem for the acoustic equation on a material with thin-structure inclusions that possess some geometric contrast\footnote{And given the discussion of the previous paragraph, can be easily generalised to a periodic, composite material exhibiting both material and geometric contrast.}.

%Another comment should be made on how the ODEs on the skeleton \eqref{eq:SI-InclusionEqn} depend on the spectral parameter $\omega^2$. \tstk{Venstcel conditions, see Kirill and Evans: that the skeleton no longer acts as a simple boundary between the bulk regions $\ddom_i$.}
%The system \eqref{eq:SingularWaveEqnQGProblem} had $\omega^2$ appearing in the vertex conditions, that is what would normally be the \emph{boundary conditions} for the ODEs on $I_{jk}$.
%The appearance of $\omega^2$ in these conditions was a direct consequence of the presence of the measure $\massMes$ providing the vertices with some ``size".
%Rather than a graph with ``bulky vertices", we now have a set of regions $\ddom_i$ with ``bulky boundaries" (the skeleton).
%And again, we observe that bestowing a notion of length to our singular inclusions (through $\ddmes$) has caused $\omega^2$ to appear in what would normally be our boundary conditions for the regions $\ddom_i$.