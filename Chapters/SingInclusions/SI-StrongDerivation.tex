\section{``Strong Formulation" of the Acoustic Approximation \eqref{eq:SI-WaveEqn}} \label{sec:SI-StrongDerivation}
Whilst we can choose to work directly from the min-max principle \eqref{eq:SI-VarProb} in an attempt to solve \eqref{eq:SI-WaveEqn}, we still do not have any explicit insights into how the presence of the skeleton is affecting the (solutions and eigenvalues of) this problem.
It is also desirable for us to move away from working with (approximations to) the tangential gradients themselves --- compared to classical gradients and derivatives, we do not have many tools to handle these objects numerically.
Therefore, in this section our goal is to derive a ``strong" formulation for \eqref{eq:SI-WaveEqn}, with motivations similar to those of sections \ref{sec:ScalarDerivation} and \ref{sec:3DSystemDerivation} --- we want to be able to analyse a more tractable problem, preferably in terms of objects familiar to us from classical calculus or chapter \ref{ch:ScalarSystem}.

Before we begin, we should consider what our intuition is telling us about the behaviour we expect from any solutions $u$ to \eqref{eq:SI-WeakWaveEqn}.
A good starting point is to consider how we expect our solutions to behave if we could (n\"{i}avely) interpret \eqref{eq:SI-WeakWaveEqn} in a strong sense.
Away from the skeleton, \eqref{eq:SI-WeakWaveEqn} looks like the usual acoustic approximation on a bounded domain, and so we expect our solution to possess sufficient regularity to be differentiated twice in the bulk and to look similar to the classical acoustic approximation here.
Regarding the skeleton, we know that solutions to the acoustic approximation on the singular structure corresponding to the skeleton possess two derivatives along the edges of $\graph$ (section \ref{sec:ScalarDerivation}), and are tied together through the vertex conditions.
So now we ask what should happen in the vicinity of the skeleton --- here we have (what we expect to be) a twice differentiable function in a bulk region $\ddom_i$ approaching its boundary, and so there should be ($L^2$) traces of $u$ and its normal derivative onto this boundary.
However this boundary coincides with (a subregion of) the skeleton, so the function $u$ should ``feel" the affect of these traces as it moves along the skeleton.
A partial converse is also expected; $u$ is twice differentiable along the skeleton, and given that $u$ \emph{also} has a trace onto the skeleton, we expect that these traces should be consistent with the function values from the bulk.
In summary, we should expect that \eqref{eq:SI-WeakWaveEqn} can be reformulated into a system that consists of the following components:
\begin{enumerate}[(a)]
	\item A (Helmholtz-like) PDE in each of the bulk regions, the solution to which has boundary traces matching the solution to a quantum graph problem on the inclusions.
	\item A 2nd-order quantum graph problem on the singular inclusions, with the edge ODEs involving or being influenced by the traces from the bulk regions.
	\item Conditions at the vertices of the graph to tie the quantum graph problem, and hence the PDE problems, together.
\end{enumerate}

Much like in sections \ref{sec:ScalarDerivation} and \ref{sec:3DSystemDerivation} we can work from \eqref{eq:SI-WeakWaveEqn} and the definition of $\tgradSob{\ddom}{\ccompMes}$ to obtain a system as described by (a)-(c).
The arguments we employ and the results we obtain are precisely those one would employ when applying the method of Lagrange multipliers to the problem \eqref{eq:SI-VarProb}.
Our starting point is the problem \eqref{eq:SI-WeakWaveEqn}, repeated here for ease of reading: find $\omega^2>0$ and non-zero $u\in\tgradSob{\ddom}{\ccompMes}$ such that
\begin{align*}
	\integral{\ddom}{ \tgrad_{\ccompMes}u\cdot\overline{\tgrad_{\ccompMes}\phi} }{\ccompMes}
	&= \omega^2 \integral{\ddom}{ u\overline{\phi} }{\ccompMes}, \quad\forall\phi\in\psmooth{\ddom}. \tag{\eqref{eq:SI-WeakWaveEqn} restated}
\end{align*}
We will need to make use of several standard integral identities, which we summarise below.
Let $D$ be an open Lipschitz domain, let $u\vert_{\partial D}$ denote the trace (of a suitably regular) function $u$ on $D$ into $\ltwo{\partial D}{S}$, and $n^D$ denote the exterior normal on the boundary of $D$.
\begin{itemize}
	\item For $u,v\in\gradSob{D}{\lambda_2}$ and $j\in\clbracs{1,2}$,
	\begin{align*}
		\integral{D}{ v\partial_j u + u\partial_j v }{\lambda_2}
		&= \integral{\partial D}{ u v n^D_j }{S}.
	\end{align*}
	\item For $u\in H^2\bracs{D,\lambda_2}, v\in\gradSob{D}{\lambda_2}$,
	\begin{align*}
		\integral{D}{ \grad u\cdot \grad v }{\lambda_2} 
		&=  - \integral{D}{ v\laplacian u }{\lambda_2} + \integral{\partial D}{ v\vert_{\partial D}\pdiff{u}{n^D}\vert_{\partial D} }{S}.
	\end{align*}
\end{itemize}
From the above, we can deduce that whenever $u\in H^2\bracs{D,\lambda_2}$ and $v\in\gradSob{D}{\lambda_2}$, we have that
\begin{align*}
	\integral{D}{ \tgrad u\cdot\overline{\tgrad v} }{\lambda_2}
	&= - \integral{D}{ \overline{v}\tgrad\cdot\tgrad u }{\lambda_2} + \integral{\partial D}{ \overline{v}\vert_{\partial D}\bracs{\tgrad u\cdot n^D}\vert_{\partial D} }{S}.
\end{align*}

We now begin the reformulation, starting by considering what happens when we test against functions supported in the bulk regions.
Suppose that the test function $\phi$ in \eqref{eq:SI-WeakWaveEqn} has support contained within the interior of one of the bulk regions $\ddom_i$, that is $\phi\in\csmooth{\ddom_i}$.
In this case \eqref{eq:SI-WeakWaveEqn} becomes
\begin{align*}
	\omega^2\integral{\ddom_i}{u\overline{\phi}}{\lambda_2} 
	&= \integral{\ddom_i}{ \grad u\cdot\overline{\grad\phi} - \rmi\qm\overline{\phi}\cdot\tgrad u + \rmi\qm  u\cdot\overline{\grad\phi} - \rmi^2\qm\cdot\qm u\overline{\phi} }{\lambda_2} \\
	&= \integral{\ddom_i}{ \grad u\cdot\overline{\grad\phi} - 2\rmi\qm\overline{\phi}\cdot\tgrad u - \rmi^2\qm\cdot\qm u\overline{\phi} }{\lambda_2}, \\
	\implies \integral{\ddom_i}{ \grad u\cdot\overline{\grad\phi} }{\lambda_2}
	&= \integral{\ddom_i}{ \bracs{\omega^2 u + 2\rmi\qm\cdot\tgrad u + \rmi^2\qm\cdot\qm u} \overline{\phi} }{\lambda_2}, 
\end{align*}
which holds for any $\phi\in\csmooth{\ddom_i}$.
Given that we also know that $u$ is $\ltwo{\partial\ddom_i}{S}$ (and is even $H^1$ in this space), this implies that $u\in \gradgradSob{\ddom_i}{\lambda_2}$ with
\begin{align*}
	\laplacian u &= -\bracs{ \omega^2 u + 2\rmi\qm\cdot\tgrad u + \rmi^2\qm\cdot\qm u }
\end{align*}
in $\ddom_i$, so we let $\laplacian^{\qm}:= \laplacian u + 2\rmi\qm\cdot\tgrad u + \rmi^2\qm\cdot\qm u$ and have that
\begin{subequations} \label{eq:SI-StrongForm} % This is so that the Strong System is coherently numbered, this spans most of the derivation so be careful!
\begin{align*}
	\laplacian_\qm u = -\omega^2 u, &\text{in } \ddom_i. \labelthis\label{eq:SI-BulkEqn}
\end{align*}
The additional regularity of the solution $u$ in the bulk regions provides equation \eqref{eq:SI-BulkEqn}, which matches our expectations in (a) of $u$ satisfying the acoustic approximation in the bulk regions, away from the skeleton.

Next we address what happens when we test against functions whose support straddles an edge $I_{jk}\in\edgeSet$.
For this we need to introduce a local labelling system for the bulk regions that are adjacent to $I_{jk}$.
Let $\ddom_{jk}^+$ be the bulk region whose boundary has non-empty intersection with $I_{jk}$ and whose exterior unit normal on $\partial\ddom_{jk}^+\cap I_{jk}$ is equal to $-n_{jk}$.
Similarly let $\ddom_{jk}^-$ be the bulk region whose boundary has non-empty intersection with $I_{jk}$ and whose exterior unit normal on $\partial\ddom_{jk}^-\cap I_{jk}$ is equal to $n_{jk}$.
Note the sign convention; this is chosen because the region $\ddom_{jk}^+$ is ``to the right" of $I_{jk}$ as viewed from the local coordinate system $y_{jk}=\bracs{n_{jk}, e_{jk}}$, and $\ddom_{jk}^-$ is ``on the left" --- see figure \ref{fig:Diagram_SI-AdjacentBulkRegions}.
\begin{figure}[t!]
	\centering
	\includegraphics[scale=1.0]{Diagram_SI-AdjacentBulkRegions.pdf}
	\caption[Labelling convention for bulk regions adjacent to a skeleton edge.]{\label{fig:Diagram_SI-AdjacentBulkRegions} Labelling convention for regions adjacent to an edge $I_{jk}$.}
\end{figure}
Now consider \eqref{eq:SI-WeakWaveEqn} when $\phi$ is taken to have compact support that intersects (the interior of) an edge $I_{jk}$, the adjacent bulk regions $\ddom_{jk}^+$ and $\ddom_{jk}^-$, and no other parts of $\graph$.
Equation \eqref{eq:SI-WeakWaveEqn} then implies that
\begin{align*}
	\integral{\ddom}{ \omega^2 u\overline{\phi} - \tgrad_{\lambda_{jk}}u\cdot\overline{\tgrad_{\lambda_{jk}}\phi} }{\lambda_{jk}}
	&= \integral{\ddom}{ \tgrad u\cdot\overline{\tgrad\phi} - \omega^2 u\overline{\phi} }{\lambda_2} \\
	&= \integral{\ddom_{jk}^+}{ \tgrad u\cdot\overline{\tgrad\phi} - \omega^2 u\overline{\phi} }{\lambda_2}
	+ \integral{\ddom_{jk}^-}{ \tgrad u\cdot\overline{\tgrad\phi} - \omega^2 u\overline{\phi} }{\lambda_2}.
\end{align*}
We know that $u\in \gradgradSob{\ddom_{jk}^{\pm}}{\lambda_2}$ for both $\ddom_{jk}^+$ and $\ddom_{jk}^-$, and so $u$ and its normal derivative possess have $L^2$-traces onto $I_{jk}$.
Using the notation $\tgrad u\cdot n_{jk} = \pdiff{u}{n_{jk}} + \rmi\qm u\cdot n_{jk}$; and denoting the trace of $u\in\gradgradSob{\ddom_{jk}^{\pm}}{\lambda_2}$ onto the boundary $\partial\ddom^{\pm}$ by $u^{\pm}$, we have that
\begin{align*}
	\integral{\ddom}{ & \omega^2 u\overline{\phi} - \tgrad_{\lambda_{jk}}u \cdot\overline{\tgrad_{\lambda_{jk}}\phi} }{\lambda_{jk}} \\
	&= \integral{\ddom_{jk}^+}{ -\overline{\phi}\bracs{ \laplacian_{\qm} u + \omega^2 u } }{\lambda_2}
	+ \integral{\ddom_{jk}^-}{ -\overline{\phi}\bracs{ \laplacian_{\qm} u + \omega^2 u } }{\lambda_2}
	+ \integral{\partial\ddom_{jk}^+}{ -\overline{\phi}\bracs{\tgrad u\cdot n_{jk}}^{+} }{S} \\
	&\qquad + \integral{\partial\ddom_{jk}^-}{ \overline{\phi}\bracs{\tgrad u\cdot n_{jk}}^{-} }{S},
\end{align*}
since the exterior normal to $\ddom_{jk}^{\pm}$ is $\mp n_{jk}$.
Given \eqref{eq:SI-BulkEqn} and the support of $\phi$, this further implies that
\begin{align*}
	\integral{\ddom}{ \omega^2 u\overline{\phi} - \tgrad_{\lambda_{jk}}u\cdot\overline{\tgrad_{\lambda_{jk}}\phi} }{\lambda_{jk}}
	&= \integral{I_{jk}}{ \overline{\phi}\sqbracs{\bracs{\tgrad u\cdot n_{jk}}^- - \bracs{\tgrad u\cdot n_{jk}}^+} }{S} \\
	&= \int_0^{l_{jk}} \overline{\phi}\sqbracs{\bracs{\tgrad u\cdot n_{jk}}^- - \bracs{\tgrad u\cdot n_{jk}}^+} \ \md y,
\end{align*}
where we have used $r_{jk}$ to parametrise\footnote{In the interest of brevity, we have suppressed composition with $r_{jk}$ --- we rely on the domain of integration to imply composition of the integrand with $r_{jk}$.} the boundary $I_{jk}$.
Changing variables via $r_{jk}$ in the integral on the left hand side too, substituting the known form for the tangential gradients, and rearranging then provides us with
\begin{align*}
	\int_0^{l_{jk}} \bracs{u^{(jk)}}'\overline{\phi}' \ \md y
	&= \int_0^{l_{jk}} \overline{\phi}\sqbracs{ \bracs{\tgrad u\cdot n_{jk}}^- - \bracs{\tgrad u\cdot n_{jk}}^+ \right. \\
	&\qquad \left. - \omega^2 u^{(jk)} - 2\rmi\qm_{jk}\bracs{u^{(jk)}}' - \bracs{\rmi\qm_{jk}}^2 u^{(jk)} } \ \md y,
\end{align*}
which holds for all smooth $\phi$ with support contained in the interior of $I_{jk}$.
Since the factor in square brackets is $\ltwo{\sqbracs{0,l_{jk}}}{y}$, we can deduce that $u^{(jk)}\in\gradgradSob{\sqbracs{0,l_{jk}}}{y}$, and
\begin{align*}
	- \bracs{\diff{}{y} + \rmi\qm_{jk}}^2u^{(jk)} 
	&= \omega^2 u^{(jk)} + \bracs{\tgrad u\cdot n_{jk}}^+ - \bracs{\tgrad u\cdot n_{jk}}^-,
	&\qquad\text{in } \sqbracs{0,l_{jk}}.
\end{align*}
If we additionally recall that the trace of $u$ from the bulk regions $\ddom_{jk}^{\pm}$ is equal to $u^{(jk)}$, we can eliminate part of the trace-terms to obtain
\begin{align} \label{eq:SI-InclusionEqn}
	- \bracs{\diff{}{y} + \rmi\qm_{jk}}^2 u^{(jk)} 
	&= \omega^2 u^{(jk)} + \bracs{\grad u\cdot n_{jk}}^+ - \bracs{\grad u\cdot n_{jk}}^-,
	&\qquad\text{in } I_{jk}.
\end{align}
This provides us with part (b) from our intuitive argument --- on the edges of the graph we have the second-order differential equation from chapter \ref{ch:ScalarSystem}, but with the addition of a term capturing the differences in the trace of the normal derivative of $u$ from either side of the inclusion.
It is worth remarking that if our inclusions were merely interfaces, we would simply obtain an algebraic equation in the difference of the normal derivative traces on the $I_{jk}$.
Giving the edges a notion of length, even if it is only one-dimensional length within a two-dimensional domain, has resulted in the difference (or ``jump") in the normal derivatives directly influencing the behaviour of $u$ on the inclusions.
Conversely, the requirement that the traces of $u$ from $\ddom_{jk}^{\pm}$ be equal to $u^{(jk)}$ also means that the behaviour of $u$ on the inclusions will affect the solution in the bulk regions.
This coupling is similar to a ``feedback loop"; the solution in the bulk exerts influence on the edges through the traces of the normal derivatives, and the solution on the inclusions exerts influence on the bulk via the requirement that the traces coincide with the values on the inclusion.

The final question we need to ask concerns the solution $u$ in the vicinity of a vertex, or more precisely when $\phi$ has support containing a vertex $v_j$ (and without loss of generality, no other vertices).
The process is straightforward; we aim to proceed as before and use \eqref{eq:SI-BulkEqn} and \eqref{eq:SI-InclusionEqn} to cancel terms on the inclusions and in the bulk regions, leaving us with a ``vertex condition", however we require one final set of temporary notation to transcribe the argument.
Fix $v_j\in\vertSet$ for each $I_{jk}\in J(v_j)$ let $\beta_{jk}$ be the anticlockwise angle between the segment $I_{jk}$ and the $v_j+\hat{x}_1$ direction.
The $\clbracs{\beta_{jk}}$ can then be ordered by size, and correspondingly we can also order the $I_{jk}\in J(v_j)$, writing
\begin{align*}
	\beta_{jk_1} < \beta_{jk_2} < ... < \beta_{jk_{\deg(v_j)}}, 
	\qquad I_{jk_1} < I_{jk_2} < ... < I_{jk_{\deg(v_j)}}.
\end{align*}
Also adopt a cyclic convention, where $k_0 = k_{\deg(v_j)}$ and $k_{\deg(v_j)+1} = k_1$.
Now, for each $l\in\clbracs{1,...,\deg(v_j)}$ let $\ddom_{jk_l}$ be the bulk region that lies between (in the sense of the angles $\beta_{jk_{l-1}}$ and $\beta_{jk_l}$) $I_{jk_{l-1}}$ and $I_{jk_l}$.
This labelling can be visualised in figure \ref{fig:Diagram_SI-JunctionLabelling}.
\begin{figure}[t!]
	\centering
	\includegraphics[scale=1.5]{Diagram_SI-JunctionLabelling.pdf}
	\caption[Labelling convention for bulk regions surrounding a vertex.]{\label{fig:Diagram_SI-JunctionLabelling} The labelling conventions for the bulk regions and edges surrounding a vertex.}
\end{figure}

Fix $v_j\in\vertSet$ and let $\phi\in\psmooth{\ddom}$ have support that contains $v_j$ and no other points in $\vertSet$.
With such a $\phi$, the following equalities hold;
\begin{align*}
	\integral{\ddom}{ \tgrad u\cdot\overline{\tgrad\phi} - \omega^2u\overline{\phi} }{\lambda_2}
	&= \sum_{l} \integral{\ddom_l}{ \tgrad u\cdot\overline{\tgrad\phi} - \omega^2u\overline{\phi} }{\lambda_2} \\
	&= \sum_{l} \integral{\ddom_l}{ -\overline{\phi}\bracs{ \laplacian_{\qm} u + \omega^2 u } }{\lambda_2}
	+ \integral{\partial\ddom_l}{ \overline{\phi}\tgrad u\cdot n_{jk_l} }{S} \\
	&= \sum_l \integral{I_{jk_l}}{ \overline{\phi}\tgrad u\cdot n_{jk_l} }{S} + \integral{I_{jk_{l-1}}}{ \phi\tgrad u\cdot n_{jk_{l-1}} }{S} \\
	&= \sum_l \integral{I_{jk_l}}{ \overline{\phi}\bracs{ \bracs{\tgrad u\cdot n_{jk_l}}^- - \bracs{\tgrad u\cdot n_{jk_l}}^+ } }{S}, \\
	\integral{\ddom}{ \tgrad_{\ccompMes} u\cdot\overline{\tgrad_{\ccompMes}\phi} - \omega^2u\overline{\phi} }{\ddmes}
	&= \sum_l \integral{I_{jk_l}}{ -\overline{\phi}\bracs{\bracs{\diff{}{y}+\rmi\qm_{jk_l}}^2 u^{(jk_l)} + \omega^2 u^{(jk_l)}} }{\lambda_{jk_l}} \\
	&\qquad + \sum_l \sqbracs{ \bracs{\pdiff{}{n}+\rmi\qm_{jk_l}} u^{(jk_l)}(v_j)\overline{\phi}(v_j) }.
\end{align*}
Thus, equation \eqref{eq:SI-WeakWaveEqn} becomes
\begin{align*}
	\alpha_j\omega^2 u(v_j)\overline{\phi}(v_j)
	&= \sum_l \clbracs{ \integral{I_{jk_l}}{ \overline{\phi}\bracs{ \bracs{\tgrad u\cdot n_{jk_l}}^- - \bracs{\tgrad u\cdot n_{jk_l}}^+ } }{S} \right. \\
	&\qquad \left.	+ \integral{I_{jk_l}}{ -\overline{\phi}\bracs{\bracs{\diff{}{y}+\rmi\qm_{jk_l}}^2 u^{(jk_l)} + \omega^2 u^{(jk_l)}} }{\lambda_{jk_l}} \right. \\
	&\qquad \left.	+ \sqbracs{ \bracs{\pdiff{}{n}+\rmi\qm_{jk_l}} u^{(jk_l)}(v_j)\overline{\phi}(v_j) } } \\
	&= \sum_l \sqbracs{ \bracs{\pdiff{}{n}+\rmi\qm_{jk_l}} u^{(jk_l)}(v_j)\overline{\phi}(v_j) }
	&\quad\text{using \eqref{eq:SI-InclusionEqn}}, \\
	\implies \alpha_j\omega^2 u(v_j)\overline{\phi}(v_j)
	&= \sum_l \bracs{\pdiff{}{n}+\rmi\qm_{jk_l}} u^{(jk_l)}(v_j), \labelthis\label{eq:SI-VertexCondition}
\end{align*}
\end{subequations} % last equation that we needed in the strong form has been provided, close off the subequations environment
since $\phi(v_j)$ is arbitrary.
In addition to continuity of $u$ at each of the vertices, we have found that $u$ also adheres to the same Wentzell condition at each of the vertices as in the case of a singular structure.

The system \eqref{eq:SI-StrongForm}\footnote{We call \eqref{eq:SI-StrongForm} ``strong" since it is no longer understood in a variational sense, and its relation to \eqref{eq:SI-WaveEqn} is akin to the relationship between the strong and weak forms of PDEs.}, 
\begin{subequations}
	\begin{align*}
		-\laplacian_\qm u 
		&= \omega^2 u 
		&\text{in } \ddom_i, \tag{\eqref{eq:SI-BulkEqn} restated} \\
		- \bracs{\diff{}{y} + \rmi\qm_{jk}}^2u^{(jk)}  
		&= \omega^2 u^{(jk)} + \bracs{\bracs{\grad u\cdot n_{jk}}^+ - \bracs{\grad u\cdot n_{jk}}^-}
		&\text{in } I_{jk}, \tag{\eqref{eq:SI-InclusionEqn} restated} \\
		\sum_{j\con k} \bracs{\pdiff{}{n}+\rmi\qm_{jk}} u^{(jk)}(v_j) 
		&= \alpha_j\omega^2 u(v_j)
		&\text{at } v_j\in\vertSet, \tag{\eqref{eq:SI-VertexCondition} restated}
	\end{align*}
\end{subequations}
The system is composed of two differential equations (the PDE \eqref{eq:SI-BulkEqn} on the bulk, and the ODE \eqref{eq:SI-InclusionEqn} on the skeleton) coupled through the vertex condition \eqref{eq:SI-VertexCondition} and the requirement that the traces from adjacent regions match the function values on the skeleton --- matching our expectations from before.
The objects involved are more familiar, being a combination of classical gradients and the essentially one-dimensional tangential derivatives studied in section \ref{sec:3DGradSobSpaces}.
We also expect the system \eqref{eq:SI-StrongForm} to be easier to work with numerically (compared to \eqref{eq:SI-VarProb}), due to the interaction between the bulk regions and skeleton being made explicit.
There are however two ``scales" on which a solution $u$ behaves, being 2D in the bulk and 1D on the skeleton, and we need resolution in both of them.

It is worth making some remarks here to reconnect \eqref{eq:SI-StrongForm} with the physical (or ``dimensionfull") material it represents.
The domain on which the system \eqref{eq:SI-StrongForm} is posed is motivated by the ``visual" limit of the acoustic approximation on a composite material with thin inclusions that are shrinking to zero thickness.
The \emph{material} properties in the background material and on the inclusions\footnote{That is, the permittivities and under the non-magnetic assumption, permeabilities.} in these domains are identical, but there is (geometric) contrast between the edge and vertex volumes of the inclusions.
Indeed, the (dimensionless) parameters $\alpha_j$ only appear in one of the three equations \eqref{eq:SI-VertexCondition}, because this is the only equation in which the vertex and edge regions interact.
%This is directly analogous to the situation one ends in when studying composite materials with contrasting material properties; the dimensionless parameters representing the contrast appear in (what is normally a boundary condition) relating the values of the solution across the interface between the inclusion and background material.
It is natural to ask whether our approach allows for the presence of material contrast, as well as geometric contrast, and how this might affect our problem formulation and the system \eqref{eq:SI-StrongForm}.
With this in mind, let us introduce some material contrast between the bulk regions and the skeleton, represented through the (dimensionless) function
\begin{align*}
	\epsilon_m =
	\begin{cases} 1 & x\in\ddom_i, \\ \epsilon & x\in\graph, \end{cases}
\end{align*}
where $\epsilon$ is the parameter quantifying the contrast between the material properties (the permittivities in this context) on the skeleton and in the bulk.
We also let $m>0$ (whose interpretation will become clear shortly) and modify the system \eqref{eq:SI-WaveEqn} slightly to account for this contrast, considering the problem
\begin{align*}
	-\epsilon_m^{-1}\laplacian_{\ccompMes}^{\qm} u &= \omega^2 u,
	\qquad u\in\tgradSob{\ddom}{\lambda_2^{m\dddmes}},
\end{align*}
understood as the problem of finding $u\in\tgradSob{\ddom}{\lambda_2^{m\dddmes}}$ and $\omega^2>0$ such that
\begin{align} \label{eq:SI-ContrastWeakWaveEqn}
	\integral{\ddom}{ \epsilon_m^{-1}\tgrad_{\ccompMes}u\cdot\overline{\tgrad_{\ccompMes}\phi} }{\lambda_2^{m\dddmes}}
	&= \omega^2 \integral{\ddom}{ u\overline{\phi} }{\lambda_2^{m\dddmes}}, \quad\forall\phi\in\psmooth{\ddom}.
\end{align}
Utilising an identical argument to the above, \eqref{eq:SI-ContrastWeakWaveEqn} admits the ``strong" formulation
\begin{subequations}
	\begin{align*}
		-\laplacian_\qm u 
		&= \omega^2 u 
		&\text{in } \ddom_i, \\
		- \epsilon^{-1}\bracs{\diff{}{y} + \rmi\qm_{jk}}^2u^{(jk)}  
		&= \omega^2 u^{(jk)} + \recip{m}\bracs{\bracs{\grad u\cdot n_{jk}}^+ - \bracs{\grad u\cdot n_{jk}}^-}
		&\text{in } I_{jk}, \\
		m\epsilon^{-1}\sum_{j\con k} \bracs{\pdiff{}{n}+\rmi\qm_{jk}} u^{(jk)}(v_j) 
		&= m\omega^2 \alpha_j u(v_j)
		&\text{at } v_j\in\vertSet.
	\end{align*}
\end{subequations}
Let us consider the case where $\alpha_j=0$ and examine the high-index limit $\epsilon\rightarrow0$ --- we obtain the system
\begin{subequations}
	\begin{align*}
		\laplacian_\qm u 
		&= \omega^2 u 
		&\text{in } \ddom_i, \\
		\omega^2 u^{(jk)}
		&= \bracs{\bracs{\grad u\cdot n_{jk}}^- - \bracs{\grad u\cdot n_{jk}}^+}
		&\text{in } I_{jk},
	\end{align*}
\end{subequations}
which is a realisation of the system \eqref{eq:Intro-KuchFigQGLimit} upon identifying $W=m^{-1}$ and rescaling the spectral parameter $\omega^2=zm^{-1}$ by the constant $m$.
In the study \cite{figotin1998spectral}, the spectrum of \eqref{eq:Intro-KuchFigQGLimit} was shown to coincide with the limit of the spectra of the acoustic equation on the domains illustrated in figure  \ref{fig:Diagram_KF-DoubleLimitStudy} in the limit that the inclusion thickness $\delta\rightarrow0$ simultaneously with $\epsilon\bracs{\delta}\delta\rightarrow m$.
By including the ``mass" $m = \lim_{\delta\rightarrow0}\delta\epsilon\bracs{\delta}$ as a prefactor to the singular measure $\dddmes$ that supports the corresponding singular limit of this region (the skeleton), our variational problem also describes the effective problem.
We also remark that the parameter $m$ fulfils a similar role to the $\alpha_j$; in being the parameter that describes the limiting behaviour of the interactions between the shrinking thickness $\delta$ and increasing permittivity $\epsilon$, and it is introduced in precisely the same manner as $\alpha_j$ --- as a prefactor to the relevant measure.
In light of this, our problem \eqref{eq:SI-WaveEqn} also suggests the effective problem one might obtain if the study \eqref{eq:Intro-KuchFigQGLimit} was extended to \emph{also} encompass possible geometric contrast between the edge and vertex regions of the shrinking inclusions (that is, reintroducing non-zero $\alpha_j$).

Our final comments on the system \eqref{eq:SI-StrongForm} pertain to the form of the ODEs on the skeleton \eqref{eq:SI-InclusionEqn} --- in particular the presence of a coupling between the flux across the skeleton and the ``diffusion" (the second-derivative operator) along the skeleton.
Analogous conditions (and equations in the bulk) are derived in the study \cite{cherednichenko2019homogenisation} in a similar domain setup, but concerning the problem of homogenisation for the equations of (linearised) elasticity.
The effective problem obtained in this study also exhibited coupling between the diffusion along the (equivalent of the) skeleton and flux across the skeleton\footnote{It is noted in \cite{cherednichenko2019homogenisation} that such conditions, in which coupling diffusion along a boundary to diffusion along said boundary, belong to the (very broad) class of \emph{Ventcel}' conditions \cite{venttsel1959boundary}. This is the same class that the \emph{Wentzell} conditions for quantum graphs belong to --- it is likely that Wentzell and Ventcel' are different spellings of the same translated name.}.
Again we highlight that the material considered in \cite{cherednichenko2019homogenisation} did not consider different geometric contrasts (between the vertex and edge regions), so there is no analogous equation to \eqref{eq:SI-VertexCondition} appears in the effective problem, much like in the system obtained from \eqref{eq:Intro-KuchFigQGLimit}.

Having obtained \eqref{eq:SI-StrongForm} from \eqref{eq:SI-WaveEqn} lends further weight to our motivations for using \eqref{eq:SI-WaveEqn} (and the other variational problems we have studied) as a predictive tool, as well as reinforcing the idea that \eqref{eq:SI-WaveEqn} does provide a model of ``wave propagation" on structures with singular components.
Indeed, our construction and setup of the problem \eqref{eq:SI-WaveEqn} enables us to provide the definition of a (candidate) limit operator for the acoustic equation on a material with thin inclusions, that possesses both geometric and material contrasts some geometric contrast.
The reduction under certain regimes to the problem \eqref{eq:Intro-KuchFigQGLimit}; and the similarities with the effective problem derived in \cite{cherednichenko2019homogenisation}, further support the idea that our variational problems can be used to provide the definitions of the limiting operators in such contexts.
Our attempts to compute the spectra of such problems gain additional traction because of this --- being able to do so enables exploration of the parameter space spanned by the ``contrast parameters" ($\alpha_j$, $m$, and potentially others) and thus insights into the behaviours of physical materials under the corresponding asymptotic limits.