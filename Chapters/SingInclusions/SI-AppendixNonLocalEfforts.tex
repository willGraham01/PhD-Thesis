\section{On the Approximation of the Action of the Dirichlet-to-Neumann maps} \label{sec:SIApp-NonLocalSolve}
This section elaborates on one of the ideas suggested at the end of section \ref{ssec:SI-GraphMethod} regarding computation of the action of $\dtn_\omega^i$.
Let us suppose that we can expand $u=\sum_{m=1}^{\infty}u_m\psi_m$ where $\psi_m$ is some suitable basis for $H^2\bracs{\graph}$.
We then look to compute the eigenfunctions $\varphi^{(i)}_n$ and eigenvalues $w^{(i)}_n$ of $\dtn_{\omega}^i$, then express each $\psi_m$ in terms of the $\varphi^{(i)}_n$.
This will enable us to approximate the expression $\dtn^i_\omega\psi_m$ in terms of inner products between the $\varphi^{(i)}$ and $\psi_m$.

To do so we need to assume the operator $\dtn_{\omega}^i$ is self-adjoint on $\ltwo{\ddom_i}{S}$ and has compact resolvent\footnote{Although the inverse is the Neumann to Dirichlet map, so these properties likely hold.}, so that the $\varphi^{(i)}_n$ also form a basis of the space $\ltwo{\partial\ddom_i}{S}$, and can be extended by zero to functions $\hat{\varphi}_n^{(i)}$ in $L^2\bracs{\graph}$.
This means that we can represent each $\psi_m\vert_{\partial\ddom_i}$ as a sum of the $\varphi_n^{(i)}$,
\begin{align*}
	\psi_m = \sum_{n=1}^{\infty} c_{m,n}^i \varphi_n^{(i)}, \quad c_{m,n}^i = \ip{\psi_m}{\varphi_n^{(i)}}_{\ltwo{\partial\ddom_i}{S}},
\end{align*}
and each of the $\hat{\varphi}_n^{(i)}$ as
\begin{align*}
	\hat{\varphi}_n^{(i)} = \sum_{n=1}^{\infty} \hat{c}_{n,m}^i \psi_m, \quad \hat{c}_{n,m}^i = \ip{\varphi_n^i}{\psi_m}_{L^2\bracs{\graph}}.
\end{align*}
Furthermore, extending $\varphi_n^{(i)}$ by zero implies that
\begin{align*}
	\hat{c}_{n,m}^i = \ip{\varphi_n^i}{\psi_m}_{L^2\bracs{\graph}} = \ip{\varphi_n^i}{\psi_m}_{\ltwo{\partial\ddom_i}{S}} = \overline{\ip{\psi_m}{\varphi_n^i}}_{\ltwo{\partial\ddom_i}{S}} = \overline{c}_{m,n}^i,
\end{align*}
which cuts down on the number of computations we need to perform.
Notice then that, for an edge $I_{jk}\subset\partial\ddom_i$,
\begin{align*}
	\dtn^i_\omega \psi_m
	&= \sum_{n=1}^{\infty} c_{m,n}^i w_n^{(i)} \varphi_n^{(i)},
\end{align*}
so we can write expressions such as
\begin{align*}
	\ip{\dtn^i_\omega \psi_m}{\psi_n}_{L^2\bracs{I_{jk}}}
	&= \sum_{p=1}^{\infty}c_{m,p}^i w^{(i)}_p \sum_{q=1}^{\infty} \hat{c}_{p,q}^i \ip{\psi_q}{\psi_n}_{L^2\bracs{I_{jk}}},
\end{align*}
where we have had to utilise two change of bases --- once changing $\psi_m$ to $\varphi^{(i)}_n$ to allow us to compute the action of $\dtn^i_\omega$, then back so that our approximation for $u$ is still in terms of the global basis $\clbracs{\psi_m}_{m\in\naturals}$.
If we had available suitable results concerning the rate of decay of the $c_{m,n}^i$ and $w_n^{(i)}$, we would then choose an appropriate ``truncation points" $N_i,M\in\naturals$ and utilise the approximation
\begin{align*}
	\ip{\dtn^i_\omega \psi_m}{\psi_n}_{L^2\bracs{I_{jk}}}
	&\approx \sum_{p=1}^{N_i}c_{m,p}^i w^{(i)}_p \sum_{q=1}^{M} \hat{c}_{p,q}^i \ip{\psi_q}{\psi_n}_{L^2\bracs{I_{jk}}},
\end{align*}
in some spectral method for approximating $u$ and the corresponding eigenvalue.
Of course, there are a number of questions that the above manipulations perform that require concrete justification.
The primary concern is whether truncating each of the basis expansions will still yield an accurate approximation, and the conditions one needs to adhere to in order to assure this happens.
There is still the issue of how one obtains the (Steklov) eigenfunctions and eigenvalues themselves --- we are still required to solve a differential equation on the bulk regions $\ddom_i$, unless analytic solutions are available.
Aside from the aforementioned questions, there is still the fundamental issue in the above method that the $c_{n,m}^i$ are still dependent on $\omega$ in a non-trivial manner, so it will still be required to recompute these constants if iteratively solving for $\omega$.