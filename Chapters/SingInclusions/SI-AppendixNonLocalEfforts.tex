\section{On Determination of $L_{n,m}$} \label{sec:SIApp-NonLocalSolve}
Section \ref{ssec:SI-GraphMethod} discusses how one might proceed with solving the non-local quantum graph problem \eqref{eq:SI-NonLocalQG}, culminating in the equation \ref{eq:SI-NonLocalQGNumericalDisc} defining the matrix $M(\omega^2)$.
Of the terms in \eqref{eq:SI-NonLocalQGNumericalDisc}, the entries $L_{n,m}$ are the most problematic due to the presence of the Dirichlet-to-Neumann map.
The purpose of deriving the non-local quantum graph problem \eqref{eq:SI-NonLocalQG} was to avoid a return to solving a two-dimensional equations on each of the bulk regions, however this is typically how one would compute the action of the Dirichlet-to-Neumann map $\dtn_\omega^i$ for the region $\ddom_i$.
This section elaborates on one of the ideas suggested at the end of section \ref{ssec:SI-GraphMethod} in attempting to bypass such a computation --- rather than solve a two dimensional problem in each of the bulk regions each time we need to evaluate $\dtn_\omega^i$, we will attempt to work with the eigenvalues and eigenfunctions of $\dtn_\omega^i$ instead.
The idea is then to ``expand" $u\vert_{\partial\ddom_i}$ terms of the eigenfunctions of $\dtn_{\omega}^i$, and then use a change of basis to obtain $L_{n,m}$.
So long as the geometry of the skeleton is not changed, we can then freely change the basis $\psi_m$ that $u$ is expanded in without needing to recompute $\dtn_\omega^i\psi_m$.

The operator $\dtn_{\omega}^i$ is self-adjoint and has compact resolvent (as its inverse is the Neumann to Dirichlet map), and thus possesses a sequence of eigenvalues $\lambda^i_n$ and eigenfunctions $\varphi_n^i$, where we list the $\lambda^i_n$ in ascending order (in $n$, for each $i$).
These eigenfunctions also form a basis of the space $\ltwo{\partial\ddom_i}{S}$, and can be extended by zero to functions $\hat{\varphi}_n^i$ in $L^2\bracs{\graph}$.
This means that we can represent each $\psi_m\vert_{\partial\ddom_i}$ as a sum of the $\varphi_n^i$ as
\begin{align*}
	\psi_m = \sum_{n=1}^{\infty} c_{m,n}^i \varphi_n^i, \quad c_{m,n}^i = \ip{\psi_m}{\varphi_n^i}_{\ltwo{\partial\ddom_i}{S}},
\end{align*}
and each of the $\hat{\varphi}_n^i$ as
\begin{align*}
	\hat{\varphi}_n^i = \sum_{n=1}^{\infty} \hat{c}_{n,m}^i \psi_m, \quad \hat{c}_{n,m}^i = \ip{\varphi_n^i}{\psi_m}_{L^2\bracs{\graph}}.
\end{align*}
Furthermore, extending $\varphi_n^i$ by zero implies that
\begin{align*}
	\hat{c}_{n,m}^i = \ip{\varphi_n^i}{\psi_m}_{L^2\bracs{\graph}} = \ip{\varphi_n^i}{\psi_m}_{\ltwo{\partial\ddom_i}{S}} = \overline{\ip{\psi_m}{\varphi_n^i}}_{\ltwo{\partial\ddom_i}{S}} = \overline{c}_{m,n}^i,
\end{align*}
which cuts down on the number of computations we need to perform.
Choose $N_i\in\naturals$ (which we will call the \emph{truncation point} or \emph{index}) for each $\ddom_i$, and then observe that we can write
\begin{align*}
	L_{n,m} 
	&= \sum_{v_j\in\vertSet}\sum_{j\conLeft k} \ip{\dtn^+_{\omega}\psi_m + \dtn^-_{\omega}\psi_m}{\psi_n}_{L^2\bracs{I_{jk}}} \\
	&\approx \sum_{v_j\in\vertSet}\sum_{j\conLeft k}
	\sqbracs{ \sum_{p=1}^{N_+}c_{m,p}^+\lambda^+_p \sum_{q=1}^M \hat{c}_{p,q}^+ \ip{\psi_q}{\psi_n}_{L^2\bracs{I_{jk}}} + \sum_{p=1}^{N_-}c_{m,p}^-\lambda^-_p \sum_{q=1}^M \hat{c}_{p,q}^- \ip{\psi_q}{\psi_n}_{L^2\bracs{I_{jk}}} },
\end{align*}
where we use our usual $\pm$ notation for the regions $\ddom^{\pm}$ adjacent to an edge $I_{jk}$ (with $N_{\pm}$ being the corresponding truncation point).
Of course, there are a number of questions concerning whether truncating each of the basis expansions at $N_i$ will still yield an accurate approximation (and whether this can be approximated in terms of the $N_i$ themselves).

Aside from the aforementioned questions, there is still the fundamental issue in the above method that the $c_{n,m}^i$ are still dependent on $\omega$ in a non-trivial manner, so it will still be required to recompute these constants if iteratively solving for $\omega$.
In an attempt to bypass this issue, an alternative to look into would be to determine whether it is possible to ``expand" the action of $\dtn_{\omega}^i$ in terms of (functions of) $\omega$ and other, easier to evaluate operators:
\begin{align*}
	\dtn_{\omega}^i u &= \sum_n f_i(\omega)\dtn_n^i u.
\end{align*}
One would then look to truncate such an ``expansion", quantify the error that this truncation would induce, and use the truncated expression to evaluate $L_{m,n}$.
Fundamental issues here revolve around understanding (explicitly) how the value of $\omega$ affects the action of $\dtn_\omega^i$ --- it may be the case that there is no simple way to quantify this in the manner described.