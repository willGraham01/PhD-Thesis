\section{Numerical Approach to the Variational Problem and Strong Formulation} \label{sec:SI-VPandFDM}
\tstk{fix this section's name once you decide what makes the cut. Also make a proper introduction here, or at least a lead-in.}

In this section we turn to numerical schemes for handling the variational problem \eqref{eq:SI-VarProb} and the strong formulation \eqref{eq:SI-BulkEqn}-\eqref{eq:SI-VertexCondition}.
For each formulation, we will outline the numerical scheme we have chosen to pursue and, importantly, the methods in which we can handle the non-standard integrals and gradients with respect to $\compMes$ numerically.
Where appropriate we will use the "cross-in-the-plane" geometry from the example in section \tstk{ref!!}, now equipped with $\compMes$, to illustrate the results of these methods.

\subsection{Variational Problem} \label{ssec:SI-VP}
We begin by examining the variational problem \eqref{eq:SI-VarProb},
\begin{align*} 
	\omega_n^2 &:= \min_{u}\clbracs{ \integral{\ddom}{ \abs{\tgrad_{\compMes}u}^2 }{\compMes} \setVert \norm{u}_{\ltwo{\ddom}{\compMes}}=1, \ u\perp u_l \ \forall 1\leq l\leq n-1 }. \tag{\eqref{eq:SI-VarProb} restated}
\end{align*}
Our interest is in determining the eigenvalues $\omega_n^2$, however we also need to determine the eigenfunctions $u_n$ since we need $u_n$ to be orthogonal to each of $u_l, 1\leq l\leq n-1$.
Given that we can obtain the eigenvalue $\omega_n^2$ from the eigenfunction $u_n$ by evaluating the integral in \eqref{eq:SI-VarProb}, we will focus our discussion on the approximation (and computation) of the eigenfunctions.
We also drop the explicit subscript $n$, and just consider the problem of determining the function $u\in\tgradSob{\ddom}{\compMes}$ which solves the optimisation problem
\begin{subequations} \label{eq:SI-MinProblem}
	\begin{align}
		\text{Minimise} \quad & \quad \integral{\ddom}{ \abs{\tgrad_{\compMes}u}^2 }{\compMes} \\
		\text{Subject to} \quad & \quad \integral{\ddom}{ \abs{u}^2 }{\compMes} = 1, \\
		& \quad \integral{\ddom}{ u\cdot\overline{u}_l }{\compMes} = 0, \ 1\leq l\leq n-1,
	\end{align}
\end{subequations}
where $n\in\naturals$, $u_l, 1\leq l\leq n-1$ are given (pairwise) orthogonal functions.
The traditional idea when attempting to approximate a minimising function is to represent the minimising function $u$ in a basis $\clbracs{\varphi_m}_{m\in\naturals_0}\subset\tgradSob{\ddom}{\compMes}$, truncate the basis expansion at some index $M$,
\begin{align} \label{eq:SI-VPTruncatedBasis}
	u &\approx \sum_{m=0}^M u_m \varphi_m,
\end{align}
and solve the minimisation problem (that arises from substituting \eqref{eq:SI-VPTruncatedBasis} into \eqref{eq:SI-MinProblem}) in the coefficients of the basis expansion that remain --- the choice of $M$ determines the accuracy in the approximate eigenfunction (and hence eigenvalue).
This minimisation problem will be discrete (solving for the $M+1$ independent variables $u_m$), and can be handled using optimisation methods.

None of the steps above are prohibited for the problem \eqref{eq:SI-MinProblem}, and so we can proceed with the aforementioned ideas.
We will illustrate this implementation for the "cross-in-the-plane" geometry first introduced in \tstk{example reference}\footnote{For convenience, we have translated the period cell by $\bracs{\recip{2},\recip{2}}$ with regards to how this geometry was handled in \tstk{ref}. This just allows us to avoid carrying additional constant terms around in our computations, and we would obtain the same results as if we didn't apply any translation.}; we take $\ddom=\left[0,1\right)^2$, and let $\graph$ be the period graph with a single vertex $v_0=\bracs{0,0}^\top$, and two ``looping" edges $I_h = \sqbracs{0,1}\times\clbracs{0}$, $I_v=\clbracs{0}\times\sqbracs{0,1}$.
Our first task is to decide on the basis functions $\varphi_m$ that we want to use to approximate $u$.
From the standpoint of accuracy (and typically speed) of the numerical solution there are several properties that it is desirable for this basis to have; orthonormality between the $\varphi_m$, similar shape to that expected of $u$, and of course periodicity.
This is where problems concerning the unfamiliar nature of our space $\tgradSob{\ddom}{\compMes}$ begin to arise, as the choice of basis is considerably more complex --- as choosing the behaviour of $\varphi_m$ in the bulk regions then restricts what $\varphi_m$ can do on the skeleton, and vice-versa. 
The geometry of the skeleton can also compound this issue, particularly if there are a large number of bulk regions $\ddom_i$, if they have irregular shapes, or if their shapes are significantly different (in terms of size or shape) from each other.
In general, one can choose a basis in similar fashion to how this is done for finite element schemes; mesh $\ddom$ into a union of simplexes (usually triangles) by placing nodes $\tau_i$, ensuring that none of the simplexes straddle any parts of the skeleton (that is, the interior of a simplex never has non empty intersection with part of the skeleton).
Then, use ``tent" or ``hat" functions centred on each node $\tau_i$ for the truncated basis functions $\varphi_m$.
This allows sufficient flexibility in the behaviour of $u$ on the edges and in the bulk regions, at the expense of requiring a new mesh for every new graph geometry.

Fortunately, the geometry of our ``cross-in-the-plane" example is rather simple, since the two edges $I_h, I_v$ are aligned with the coordinate axes.
This makes computing integrals on the skeleton much simpler, and traces from the bulk region can be computed with relative ease, so we can avoid taking the approach of meshing $\ddom$ as described above.
Instead, we can opt to choose a basis in a way more akin to spectral methods --- by choosing ``global" basis functions rather than the ``local" tent-basis functions that meshing $\ddom$ would utilise.
Combined with the fact that we only have one bulk region that spans the entire period cell, the natural candidate for our basis functions would be the 2D Fourier basis $\e^{2\pi\rmi(\alpha x + \beta y}$.
These functions are orthogonal in $\ltwo{\ddom}{\compMes}$, have period cell $\ddom$, and on each of the edges of $\graph$ reduce to a 1D-Fourier series.
However, these functions also possess a continuous (in the sense of matching traces) normal derivative across the skeleton, which functions in $\tgradSob{\ddom}{\compMes}$ are not required to have, and so we cannot use the Fourier basis.
Instead, we will look to use 2D polynomials to approximate our function $u$, by taking $M\in\naturals$ and setting
\begin{align} \label{eq:2DPolyBasisDef}
	\varphi_m(x,y) &= x^{i_m} y^{j_m}, \quad m = j_m + Mi_m, \ i,j\in\clbracs{0,...,M-1}.
\end{align}
These functions are not periodic by definition, so we are required to add the additional constraints
\begin{align*}
	u\bracs{0,y} = u\bracs{1,y}, \ \forall y\in\sqbracs{0,1}, 
	\qquad 
	u\bracs{x,0} = u\bracs{x,1}, \ \forall x\in\sqbracs{0,1},
\end{align*}
to our minimisation problem to account for this.
With this choice of basis, and writing $U = \bracs{u_0,...,u_{M^2-1}}^\top$, we are tasked with solving the following problem.
\begin{problem}[Discrete Variational Problem] \label{prob:DiscVarProb}
	Let $M,n\in\naturals$ and $\varphi_m$ be as in \eqref{eq:2DPolyBasisDef}.
	Given coefficients $U_l=\bracs{u^l_0,...,u^l_{M^2-1}}^\top$ for $1\leq l\neq n-1$, find values $U=\bracs{u_0,...,u_{M^2-1}}^\top$ that:
	\begin{subequations} \label{eq:SI-ExampleMinProb}
		\begin{align}
			\text{Minimise} \quad & \quad J\sqbracs{U} := \sum_{m=0}^{M^2-1}\sum_{n=0}^{M^2-1}u_m\overline{u}_n\ip{\tgrad_{\compMes}\varphi_m}{\tgrad_{\compMes}\varphi_n}_{\ltwo{\ddom}{\compMes}^2} 
			\label{eq:SI-EMPObjectiveFn} \\
			\text{Subject to} \quad & \quad \sum_{m=0}^{M^2-1}\sum_{n=0}^{M^2-1}u_m\overline{u}_n\ip{\varphi_m}{\varphi_n}_{\ltwo{\ddom}{\compMes}} = 1, 
			\label{eq:SI-EMPNormConstraint} \\
			& \quad \sum_{i_m=1}^{M-1}u_{j_m+Mi_m} = 0, \ \forall j_m\in\clbracs{0,...,M-1}, 
			\label{eq:SI-EMPxPeriodicity} \\
			& \quad \sum_{j_m=1}^{M-1}u_{j_m+Mi_m} = 0, \ \forall i_m\in\clbracs{0,...,M-1},
			\label{eq:SI-EMPyPeriodicity} \\
			& \quad \sum_{m=0}^{M^2-1}\sum_{n=0}^{M^2-1}u_m\overline{u}^l_n\ip{\varphi_m}{\varphi_n}_{\ltwo{\ddom}{\compMes}} = 0, \ \forall 1\leq l\leq n-1.
			\label{eq:SI-EMPOrthogonality}
		\end{align}
	\end{subequations}
\end{problem}
The minimiser $U$ of problem \ref{prob:DiscVarProb} then provides our approximation of $u$, and we have that $\omega^2 \approx J[U]$.
Equation \eqref{eq:SI-EMPNormConstraint} is the norm constraint on the eigenfunction $u$, \eqref{eq:SI-EMPxPeriodicity} (respectively \eqref{eq:SI-EMPyPeriodicity}) are the constraints that ensure periodicity of the eigenfunction in the $x$ ($y$) directions, and \eqref{eq:SI-EMPOrthogonality} forces $u$ to be orthogonal to the previously computed eigenfunctions $u_l$.
Due to our truncation, we are only ever able to compute approximations to the lowest $M^2 - \bracs{2M + 1} + 1$ eigenvalues due to the number of constraints in the problem \ref{prob:DiscVarProb}.

\tstk{time to display some nice figures, maybe some comparisons between the two methods etc? Also do a run of this with $\alpha_3\neq0$ just to see what on earth happens!}

\subsection{Finite Difference Scheme} \label{ssec:FDMSingInc}
\tstk{This is the content summary of \texttt{CompositeMedium\_PeriodicFDM.ipynb}}
As an alternative to working directly from the variational problem \ref{prob:DiscVarProb}, we can instead choose to work from our ``strong formulation" \eqref{eq:SI-BulkEqn}-\eqref{eq:SI-VertexCondition}.
Before we do so, it is convenient to notice that we can move the ``trace" terms in \eqref{eq:SI-InclusionEqn} to the left-hand-side to obtain the slightly nicer looking system
\begin{align*}
	-\laplacian_\qm u 
	&= \omega^2 u 
	&\text{in } \ddom_i, \\
	- \bracs{\diff{}{y} + \rmi\qm_{jk}}^2u^{(jk)}  - \bracs{\bracs{\grad u\cdot n_{jk}}^+ - \bracs{\grad u\cdot n_{jk}}^-}
	&= \omega^2 u^{(jk)},
	&\text{in } I_{jk}, \\
	\sum_l \bracs{\pdiff{}{n}+\rmi\qm_{jk_l}} u^{(jk_l)}(v_j) 
	&= 0 
	&\text{at } v_j\in\vertSet.
\end{align*}
We have now placed all differential operators on the left hand side of the equations, and our goal now is to devise a numerical scheme to approximate the action of the ``operators" on the left-hand-side of the above equations, and from that determine the spectrum of the problem.
Since all the objects in the formulation above are classical, we can look to employ a n{\"i}ave finite-difference based numerical scheme to approximate the spectrum of our problem.

\tstk{rewrite from here: it's going to be much easier to first discuss problems with meshing in general. IE placement of nodes, the need to adhere to the coordinate axis in the bulk but the local edge coordinate system on the edges, the need to place a node at every vertex and enough nodes along each edge, and that we can't cross edges when in a bulk region, so have to guarantee we have nearest neighbours contained in each $\overline{\ddom}_i$ whenever we're in a bulk region. All of this makes it hard to come up with a mesh in the first place, and you won't have a constant mesh-width throughout. This also means that centred differences are off the table since we don't have the same inter-nodal difference. But, having accounted for these things, you can still draw up a scheme (and we should translate the more general equations below into such a framework, IE nodes $\tau_i$ and use something like $\tau_i^{++}$ etc for it's neighbour ``up and to the left".}
We first discretise $\ddom$ into a \emph{suitable} uniform mesh consisting of $N\times N$ nodes $\tau_{p,q} = \bracs{(p-1)h,(q-1)h}$ for $p,q\in\clbracs{1,...,N}$, with a mesh width of $h = \recip{N-1}$, and write $u_{p,q} = u\bracs{\tau_{p,q}}$.
Conditions for a ``suitable" mesh will be highlighted as we proceed with the description, and discussed afterwards.
Also note that there is no need to place nodes along both of the periodic edges of $\ddom$, so long as we keep track of which nodes are connected by periodicity, but for notational purposes it is convenient to include such nodes in our description.
Proceeding with the discretisation of the above equations, at points $\tau_{p,q}\in\ddom_i$ in one of the bulk regions, we discretise as
\begin{align*}
	-\laplacian_\qm u\bracs{\tau_{p,q}} &\approx 
	\bracs{\abs{\qm}^2 + 4h^{-2}}u_{p,q}
	-h^{-1}\bracs{h^{-1} + \rmi\qm_1}u_{p+1,q}
	-h^{-1}\bracs{h^{-1} - \rmi\qm_1}u_{p-1,q} \\
	&\qquad -h^{-1}\bracs{h^{-1} + \rmi\qm_2}u_{p,q+1}
	-h^{-1}\bracs{h^{-1} - \rmi\qm_2}u_{p,q-1}, \labelthis\label{eq:SI-FDMBulkDiscretise}
\end{align*}
using centred differences --- of course, using forward (also known as left) or backward (a.k.a right) differences would also be a valid approach.

For nodes $\tau_{p,q}\in I_{jk}$, our finite difference approximations become slightly more complex, as we are forced to consider the nearest nodes to $\tau_{p,q}$ that lie in the directions $e_{jk}$ and $n_{jk}$.
To condense the notation we let;
\begin{itemize}
	\item $\tau_{p,q}^{\pm e_{jk}}$ denote the nearest node to $\tau_{p,q}$ in the direction $\pm e_{jk}$ from $\tau_{p,q}$, and set $h_{p,q}^{\pm e_{jk}} = \abs{ \tau_{p,q} - \tau_{p,q}^{\pm e_{jk}} }$.
	\item $\tau_{p,q}^{\pm n_{jk}}$ denote the nearest node to $\tau_{p,q}$ in the direction $\pm n_{jk}$ from $\tau_{p,q}$, and set $h_{p,q}^{\pm n_{jk}} = \abs{ \tau_{p,q} - \tau_{p,q}^{\pm n_{jk}} }$.
	\item $u_{p,q}^{\pm e_{jk}} = u\bracs{\tau_{p,q}^{\pm e_{jk}}}$ and $u_{p,q}^{\pm n_{jk}} = u\bracs{\tau_{p,q}^{\pm n_{jk}}}$.
\end{itemize}
Note that to be able to perform the above we must require that each of the nodes $\tau_{p,q}^{\pm e_{jk}}$, $\tau_{p,q}^{\pm n_{jk}}$ exist --- an important consideration for a suitable mesh.
Additionally, let us assume that $h_{p,q}^{e_{jk}} := h_{p,q}^{+e_{jk}}=h_{p,q}^{-e_{jk}}$ so that we can use central differences in the direction along $I_{jk}$ --- this is not a necessary property the mesh must have, as we could simply elect to use either forward or backward differences if $h_{p,q}^{+e_{jk}} \neq h_{p,q}^{-e_{jk}}$.
Then we have that 
\begin{align*}
	&- \bracs{\diff{}{y} + \rmi\qm_{jk}}^2u^{(jk)}\bracs{\tau_{p,q}} - \bracs{\bracs{\grad u\bracs{\tau_{p,q}}\cdot n_{jk}}^+ - \bracs{\grad u\bracs{\tau_{p,q}}\cdot n_{jk}}^-} \\
	&\qquad \approx \bracs{\qm_{jk}^2 + 2\bracs{h_{p,q}^{e_{jk}}}^{-1} + 2\bracs{h_{p,q}^{e_{jk}}}^{-2}}u_{p,q} \\
	&\qquad - \bracs{h_{p,q}^{e_{jk}}}^{-1}\bracs{ \bracs{h_{p,q}^{e_{jK}}}^{-1} + \rmi\qm_{jk} }u_{p,q}^{+e_{jk}}
	- \bracs{h_{p,q}^{e_{jk}}}^{-1}\bracs{ \bracs{h_{p,q}^{e_{jk}}}^{-1} - \rmi\qm_{jk} }u_{p,q}^{-e_{jk}} \\
	&\qquad - \bracs{h_{p,q}^{+n_{jk}}}^{-1}u_{p,q}^{+n_{jk}}
	- \bracs{h_{p,q}^{-n_{jk}}}^{-1}u_{p,q}^{-n_{jk}}, \labelthis\label{eq:SI-GeneralEdgeDiscretise}
\end{align*}
which serves as our approximation at $\tau_{p,q}$ --- we have taken ``centred" differences along the edge $I_{jk}$ and one-sided derivatives from the adjacent regions to approximate the traces of the normal derivatives.

For those $\tau_{p,q}$ that are placed at the vertices $v_j\in\vertSet$, our finite difference approximation already assumes continuity of $u$ at these nodes, so instead we must enforce the vertex conditions here.
To this end, we find that
\begin{subequations} \label{eq:SI-FDMGeneralVertex}
	\begin{align}
		\bracs{ \pdiff{}{n} + \rmi\qm_{jk} } u^{(jk)}(v_j)
		&= h_{p,q}^{+e_{jk}}\bracs{ u_{p,q} - u_{p,q}^{+e_{jk}} } - \rmi\qm_{jk}u_{p,q}, \\
		\bracs{ \pdiff{}{n} + \rmi\qm_{jk} } u^{(kj)}(v_j)
		&= h_{p,q}^{-e_{kj}}\bracs{ u_{p,q} - u_{p,q}^{-e_{kj}} } + \rmi\qm_{kj}u_{p,q},	
	\end{align}
\end{subequations}
which will allow us to compute the approximation to \eqref{eq:SI-VertexCondition}.
These approximations then allow us to form a finite difference matrix $\mathcal{F}$\footnote{One final consideration we then have to make is to ensure that we adhere to periodic boundary conditions for any nodes that lie on $\partial\ddom$ when constructing $\mathcal{F}$.}, which acts on the vector $U$ with $U_{i} = u_{p,q}$ where $i = q + Np$.
Then, we solve the (generalised) eigenvalue problem
\begin{align} \label{eq:FDM-MatrixSystem}
	\mathcal{F}U = \beta\bracs{\omega^2}U,
\end{align}
for $\omega^2>0, U\in\complex^{N^2}$, where 
\begin{align*}
	\bracs{\beta\bracs{\omega^2}}_{nm} &= 
	\begin{cases}
 		\omega^2 & n=m, n=q + Np, \text{ and } \tau_{p,q}\not\in\vertSet, \\
 		0 & \text{otherwise. }
	\end{cases}
\end{align*}
The matrix-valued function $\beta$ ensures that the vertex condition is satisfied when solving \eqref{eq:FDM-MatrixSystem}, which can be done with a suitable generalised eigenvalue solver (note that $\beta\bracs{\omega^2}$ is easily computable and positive semi-definite, being a diagonal matrix with non-negative entries).

We must place (enough) nodes on each of the edges $I_{jk}$ to ensure that \eqref{eq:SI-InclusionEqn} is discretised correctly and reflected in the finite difference matrix $\mathcal{F}$.
Thus, the major requirement that we make of our mesh is that the nodes $\tau_{p,q}^{\pm e_{jk}}$ and $\tau_{p,q}^{\pm n_{jk}}$ exist whenever we have $\tau_{p,q}\in I_{jk}$.
Of course, this is not ideal if the edges of $\graph$ are not aligned with the coordinate axes, as using a uniform mesh no longer guarantees that such nodes will exist (nor does it, in general, ensure that at least one node is placed on every edge 
Even if such nodes exist, there is no guarantee that they are ``near" the original node as figure \ref{fig:Diagram_CompMesMeshNearestNeighbours} illustrates.
\begin{figure}
	\centering
	\includegraphics[scale=1.0]{Diagram_CompMesMeshNearestNeighbours.pdf}
	\caption{\label{fig:Diagram_CompMesMeshNearestNeighbours} With a uniform mesh and general skeleton $\graph$, there is no guarantee that the ``nearest neighbours" of a node are close by, which in turn leads to a poor approximation of the gradient at such nodes. Indeed, there isn't even any guarantee that each $I_{jk}$ will have at least one node placed on it.}
\end{figure}
Indeed, it is very possible that the nodes $\tau_{p,q}^{\pm e_{jk}}$ and $\tau_{p,q}^{\pm n_{jk}}$ are significantly further away from $\tau_{p,q}$ than other nodes in the mesh.
However, the requirement that we use a uniform mesh is not a necessity, we can use a non-uniform mesh at the expense of abandoning centred differences in the bulk regions and along the edges $I_{jk}$.
If taking this option, nodes should be placed along the edges $I_{jk}$ first (including placing nodes at each of the vertices), and then placed non-uniformly in the bulk regions.
The equation \eqref{eq:SI-FDMBulkDiscretise} invariably changes (since $\laplacian_{\qm}u_{p,q}$ is no longer discretised with centred differences), but ultimately one can still assemble a system of the form \eqref{eq:FDM-MatrixSystem} that approximates \eqref{eq:SI-BulkEqn}-\eqref{eq:SI-VertexCondition}.
Alternatively, if looking to avoid a complex mesh, one could instead approximate the values like $u_{p,q}^{+e_{jk}}$ using interpolations of nearby nodal values, which would reduce the sparsity of the resulting matrix $\mathcal{F}$, but might avoid the need to use additional nodes, or keep track of nodes in a non-uniform mesh.

To illustrate the above, we perform the steps above using the cross-in-the-plane geometry \tstk{ref}.
Take $N\in\naturals$ to be odd, and place nodes $\tau_{p,q} = \bracs{(p-1)j, (q-1)h}$ for $p,q\in\clbracs{1,...,N}$, giving a mesh-width $h=\recip{N-1}$.
Uniform mesh is suitable for this geometry because our edges are aligned with the coordinate axes, so the nearest-neighbours $\tau_{p,q}^{\pm e_{jk}}, \tau_{p,q}^{\pm n_{jk}}$ always exist when looking at a node $\tau_{p,q}\in I_{jk}$, and are just the 4 adjacent nodes $\tau_{p-1,q},\tau_{p+1,q},\tau_{p,q-1},\tau_{p+1}$.
Encoding the periodic boundary conditions amounts to ``enslaving" the nodal values $u_{p,N}$ to $u_{p,0}$ (and similarly $u_{N,q}$ to $u_{0,q}$) --- as such, we also adopt the convention that $u_{p,-1}=u_{p,N-1}$ and $u_{-1,q}=u_{N-1,q}$.
Furthermore, the nodes $\tau_{\frac{N-1}{2},q}$ and $\tau_{p,\frac{N-1}{2}}$ are precisely the nodes that lie on the singular inclusions, and $\tau_{\frac{N-1}{2},\frac{N-1}{2}}$ is placed at the vertex $v_0$, with all other nodes lying in the (interior of the) bulk regions.
Upon discretising, we retain \eqref{eq:SI-FDMBulkDiscretise} at the nodes $\tau_{p,q}$ that lie in the bulk regions.
For those $\tau_{p,q}$ that lie on the horizontal edge $I_h$ (that is, $\tau_{p,\frac{N-1}{2}}$ for $p\neq\frac{N-1}{2}$) the discretisation \eqref{eq:SI-GeneralEdgeDiscretise} becomes
\begin{align*}
	- \bracs{\diff{}{y} + \rmi\qm_{jk}}^2u^{(jk)}_{p,q} - \bracs{\bracs{\grad u_{p,q}\cdot n_{jk}}^+ - \bracs{\grad u_{p,q}\cdot n_{jk}}^-}
	& \approx \bracs{\qm_1^2 + 2h^{-1} + 2h^{-2}}u_{p,q} \\
	& \quad - h^{-1}\bracs{h^{-1} + \rmi\qm_1}u_{p+1,q} \\
	& \quad - h^{-1}\bracs{h^{-1} - \rmi\qm_1}u_{p-1,q} \\
	& \quad - h^{-1}u_{p,q+1} - h^{-1}u_{p,q-1}. \labelthis\label{eq:SI-FDMHorzEdgeDiscretise}
\end{align*}
Our use of a uniform mesh resulting in \eqref{eq:SI-FDMHorzEdgeDiscretise} being much simpler than \eqref{eq:SI-GeneralEdgeDiscretise}.
Similarly, for nodes $\tau_{p,q}$ on the vertical edge $I_v$ (that is, $\tau_{\frac{N-1}{2},q}$ for $q\neq\frac{N-1}{2}$) we have that
\begin{align*}
	- \bracs{\diff{}{y} + \rmi\qm_{jk}}^2u^{(jk)}_{p,q} - \bracs{\bracs{\grad u_{p,q}\cdot n_{jk}}^+ - \bracs{\grad u_{p,q}\cdot n_{jk}}^-}
	& \approx \bracs{\qm_2^2 + 2h^{-1} + 2h^{-2}}u_{p,q} \\
	& \quad - h^{-1}\bracs{h^{-1} + \rmi\qm_2}u_{p,q+1} \\
	& \quad - h^{-1}\bracs{h^{-1} - \rmi\qm_2}u_{p,q-1} \\
	& \quad - h^{-1}u_{p+1,q} - h^{-1}u_{p-1,q}. \labelthis\label{eq:SI-FDMVertEdgeDiscretise}
\end{align*}
Finally, at the central vertex $v_0 = \tau_{\frac{N-1}{2},\frac{N-1}{2}}$, we have that
\begin{align} \label{eq:SI-FDMVertCond}
	\sum_{j\con k}\bracs{ \pdiff{}{n} + \rmi\qm_{jk} }u^{(jk)}\bracs{v_0}
	&\approx h^{-1} \bracs{ 4u_{p,q} - u_{p+1,q} - u_{p-1,q} - u_{p,q+1} - u_{p,q-1} },
\end{align}
where $p = q = \frac{N-1}{2}$.
The matrix-valued $\beta\bracs{\omega^2}$ is also easily computed as
\begin{align*}
	\beta:\complex\rightarrow\complex^{(N-1)\times(N-1)}, 
	\qquad \bracs{\beta\bracs{\omega^2}}_{jk} = \begin{cases} 1 & j=k\neq\frac{N-1}{2}, \\ 0 & \text{otherwise}. \end{cases}
\end{align*}

\tstk{this gives us the numerical results in the file \texttt{CompositeMedium\_PeriodicFDM.ipynb} - I can export the results we want to pdfs and import them as images here.
Things to note are that the finite-difference matrix $\mathcal{F}$ is not Hermitian, but the eigenvalues appear to all be real.
The eigenfunction plots are also quite nice, but we don't have anything to compare them to (true values, etc).
We could do a crude sweep over the quasi-momentum values to see if the symmetry properties hold, and what the spectrum in general looks like as we vary the quasimomentum?
It might also be worth timing the runs for comparison with the ``spectral" method, or working out the cost in FLOPs, time, etc for the scheme to run?
Also need to mention how we have wasted a lot of our effort in solving in the bulk regions, when in reality we don't actually need the form of the function here...}