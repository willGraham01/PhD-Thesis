\section{Problem Formulation} \label{sec:SI-ProblemFormulation}
Let us begin defining the objects in \eqref{eq:SI-WaveEqn} accurately.
This requires us to first analyse the gradients of functions that live in the space $\tgradSob{\ddom}{\compMes}$ in the same vein as did with $\ktgradSob{\ddom}{\dddmes}$ and $\ktcurlSob{\ddom}{\dddmes}$ before, and is the focus of section \ref{sec:CompSobSpaces}.
As we might expect from the previous chapters, we find that:
\begin{itemize}
	\item The tangential gradient $\tgrad_{\compMes}u$ of $u$ is such that
	\begin{align*}
		\tgrad_{\compMes}u = \begin{cases} \grad u + \rmi\qm u & x\in\ddom\setminus\graph, \\ \tgrad_{\ddmes}u & x\in\graph, \end{cases}
	\end{align*}
	where $\grad u$ denotes the weak gradient of $u\in\gradSob{\ddom}{\lambda_2}$.
	Essentially, in the bulk regions the function $u$ and its tangential gradient coincide with the familiar notion of a weak derivative (with respect to the Lebesgue measure).
	\item The function $u$ lives in $\gradSob{\ddom_i}{\lambda_2}$ for each of the bulk regions, and the traces of $u$ from $\ddom_i$ onto the inclusions $I_{jk}$ coincide with the values of $u^{(jk)}$ on the inclusions.
	This is as close to a condition of ``continuity across the inclusions" as we can get.
	Additionally, the $u^{(jk)}$ are continuous at the vertices of $\graph$, as was the case for functions in $\ktgradSob{\ddom}{\dddmes}$.
\end{itemize}

Now consider (for a fixed $\qm$) the bilinear form $b_{\qm}$ defined on pairs $(u,v)\in\tgradSob{\ddom}{\compMes}\times\tgradSob{\ddom}{\compMes}$ where
\begin{align*}
	b_{\qm}(u,v) &= \integral{\ddom}{ \tgrad_{\compMes}u\cdot\overline{\tgrad_{\compMes}v} }{\compMes}
	= \ip{\tgrad_{\compMes}u}{\tgrad_{\compMes}v}_{\ltwo{\ddom}{\compMes}^2}.
\end{align*}
Clearly $b_{\qm}$ is symmetric and satisfies $b_{\qm}(u,u)\geq 0$ with equality if and only if $u=0$, and thus defines a self-adjoint operator
\begin{align*}
	\mathcal{A}_{\qm} := -\bracs{\tgrad_{\compMes}}^2,
\end{align*}
by
\begin{align*} 
	\dom\bracs{ \mathcal{A}_{\qm} } &= \clbracs{ u\in\tgradSob{\ddom}{\compMes} \setVert \exists f\in\ltwo{\ddom}{\compMes} \text{ s.t. } \right.
	\\
	& \qquad
	\left. \integral{\ddom}{ \tgrad_{\compMes}u\cdot\overline{\tgrad_{\compMes}v} }{\compMes} = \integral{\ddom}{ f\overline{v}}{\compMes}, \quad \forall v\in\tgradSob{\ddom}{\compMes} }, \labelthis\label{eq:CompLaplaceOpDom}
\end{align*}
with action
\begin{align*}
	\mathcal{A}_{\qm}u = -\bracs{\tgrad_{\compMes}}^2 u = f,
\end{align*}
where $u$ and $f$ are related as in \eqref{eq:CompLaplaceOpDom}.
Equation \eqref{eq:SI-WaveEqn} is then the eigenvalue equation for the operator $\mathcal{A}_{\qm}$, interpreted as the problem of finding $\omega^2>0$ and non-zero $u\in\tgradSob{\ddom}{\compMes}$ such that
\begin{align} \label{eq:SI-WeakWaveEqn}
	\integral{\ddom}{ \tgrad_{\compMes}u\cdot\overline{\tgrad_{\compMes}\phi} }{\compMes}
	&= \omega^2 \integral{\ddom}{ u\overline{\phi} }{\compMes}, \quad\forall\phi\in\psmooth{\ddom}.
\end{align}
With $\ddom$ being bounded and $\mathcal{A}_{\qm}$ self-adjoint, the spectrum of $\mathcal{A}_{\qm}$ consists of a discrete set of values $\omega^2\in\reals$ (as we expect from the Gelfand transform and introduction of the quasi-momentum) and taking the union of the spectra of the $\mathcal{A}_{\qm}$ over $\qm$ will provide us with the spectrum of a periodic operator on $\reals^2$ with period cell $\ddom$.
More importantly, we can utilise the min-max principle to write down a variational formulation whose solution determines the eigenvalues (and eigenfunctions) of $\mathcal{A}_{\qm}$, which will form the basis of our numerical approach to solving this problem.
Yet despite these useful analytic properties, \eqref{eq:CompLaplaceOpDom} and \eqref{eq:SI-WeakWaveEqn} do not lend themselves particularly well to explicit analytic solution, nor provide any insight into how to handle objects like $\tgrad_{\compMes}u$ numerically.
This places us in something of a catch-22; the variational formulation \eqref{eq:SI-WeakWaveEqn} is the standard form\footnote{Within the framework of existing theory concerning operators on Hilbert spaces.} of the eigenvalue problem for our operator $\mathcal{A}_{\qm}$ --- ensuring that it exists, is well-defined, and providing qualitative information about the spectrum.
However the tangential gradients and the integrals in \eqref{eq:SI-WeakWaveEqn} with respect to $\compMes$ are unfamiliar both from an analytic and numerical standpoint --- we know some of their properties when restricted to different regions of $\ddom$, but not how to work with them directly to obtain a solution (or approximation thereof) to \eqref{eq:SI-WeakWaveEqn}.
This motivates us to find an alternative formulation to \eqref{eq:SI-WeakWaveEqn}, leading to the formulation obtained in section \ref{sec:SI-StrongDerivation}.
We will take this idea further in section \ref{sec:SI-NonLocalQG} when we attempt to reformulate our problem on the skeleton, and discard the bulk regions.

\tstk{cross in plane setup for future sections can go here, possibly in subsection}
We will use the "cross-in-the-plane" geometry from the example in section \ref{ssec:ExampleCrossInPlane} (now equipped with $\compMes$) to ground our discussion of each of our numerical approaches, and illustrate their implementation.
For convenience, we have translated the period cell by $-\bracs{\recip{2},\recip{2}}$, which allows us to avoid carrying additional constant terms around in what follows.
Recall that this geometry consists of a skeleton $\graph$ within $\ddom=\left[0,1\right)^2$, with a single vertex $v_0=\bracs{0,0}^\top$, and two ``looping" edges $I_h = \sqbracs{0,1}\times\clbracs{0}$, $I_v=\clbracs{0}\times\sqbracs{0,1}$.
The quasi-momentum parameters $\qm_{jk}$ are easily computable as $\qm_h = \qm_1$ and $\qm_v = \qm_2$, and we only have a single bulk region, $\ddom_1 = \ddom^{\circ}=\bracs{0,1}^2$.
There are a number of properties that we expect of our eigenvalues due to the symmetric nature of our geometry, notably the following:
\begin{prop}[Cross in the plane symmetries] \label{prop:CrossInPlaneSymmetries}
	If $\omega^2, u\bracs{x_1,x_2}$ is a solution to \eqref{eq:SI-BulkEqn}-\eqref{eq:SI-VertexCondition} at $\qm = \bracs{\qm_1,\qm_2}\in\left[-\pi,\pi\right)^2$, then:
	\begin{itemize}
		\item $\omega^2, u\bracs{1-x_1,x_2}$ is a solution to \eqref{eq:SI-BulkEqn}-\eqref{eq:SI-VertexCondition} at $\qm = \bracs{-\qm_1, \qm_2}$.
		\item $\omega^2, u\bracs{x_1,1-x_2}$ is a solution to \eqref{eq:SI-BulkEqn}-\eqref{eq:SI-VertexCondition} at $\qm = \bracs{\qm_1, -\qm_2}$.
		\item $\omega^2, u\bracs{x_2,x_1}$ is a solution to \eqref{eq:SI-BulkEqn}-\eqref{eq:SI-VertexCondition} at $\qm = \bracs{\qm_2, \qm_1}$.
	\end{itemize}
\end{prop}
\tstk{proof?}

Additionally for the cross-in-the-plane geometry, we can obtain analytic expressions for a subset of the eigenfunctions and eigenvalues --- namely those inherited from the Dirichlet laplacian on $\bracs{0,1}^2$.
Observe that for $n,m\in\naturals$ the function
\begin{align*}
	u_{n,m}(x) &= \e^{-\rmi\qm\cdot x}\sin\bracs{n\pi x_1}\sin\bracs{m\pi x_2},
\end{align*}
is an eigenfunction of the Dirichlet laplacian $-\laplacian_{\qm}$,
\begin{align*}
	-\laplacian_{\qm}u &= \omega_{n,m}^2 u, &\qquad \text{on } \bracs{0,1}^2, \\
	u\bracs{0,x_2} = u\bracs{1,x_2} &= u\bracs{x_1,0} = u\bracs{x_1,1} = 0,
\end{align*}
where $\omega_{n,m}^2 = \bracs{n^2+m^2}\pi^2$.
It is clear that $u_{n,m}$ also solves \eqref{eq:SI-BulkEqn} with $\omega^2 = \omega_{n,m}^2$ and has $u=0$ on the skeleton, and furthermore is such that $u_{n,m}\in\tgradSob{\ddom}{\compMes}$.
In the event that both
\begin{itemize}
	\item ($n$ is even and $\qm_1=0$) or ($n$ is odd and $\qm_1=-\pi$),
	\item ($m$ is even and $\qm_2=0$) or ($m$ is odd and $\qm_2=-\pi$),
\end{itemize}
hold, then $u_{n,m}$ is also an eigenfunction of \eqref{eq:SI-BulkEqn}-\eqref{eq:SI-VertexCondition} with the same eigenvalue (the quasi-momentum values are required to ensure that the normal derivative traces cancel out in \eqref{eq:SI-InclusionEqn}).
Of course, one can also substitute $\omega^2_{n,m}, u_{n,m}$ and into \eqref{eq:SI-WaveEqn} to arrive at the same conclusion.
Knowing that \eqref{eq:SI-WaveEqn} shares eigenvalues with the Dirichlet laplacian will have consequences for us in section \ref{sec:NonLocalQG}, when we attempt to reduce \eqref{eq:SI-WaveEqn} to a quantum graph problem.
For now however, $u_{n,m}$ and $\omega_{n,m}^2$ provide some suitable test cases for our numerical schemes.