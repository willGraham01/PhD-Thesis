\section{Problem Formulation} \label{sec:SI-ProblemFormulation}
Let us begin defining the objects in \eqref{eq:SI-WaveEqn} accurately.
This requires us to first analyse the tangential gradients of functions that live in the space $\tgradSob{\ddom}{\ccompMes}$ in the same vein as did with $\ktgradSob{\ddom}{\dddmes}$ and $\ktcurlSob{\ddom}{\dddmes}$ before, and is the focus of section \ref{sec:CompSobSpaces}.
As we might expect from the previous chapters, we find that:
\begin{itemize}
	\item The tangential gradient $\tgrad_{\ccompMes}u$ of $u$ is such that
	\begin{align*}
		\tgrad_{\ccompMes}u = 
		\begin{cases} 
			\grad u + \rmi\qm u & x\in\ddom\setminus\graph, \\ 
			\tgrad_{\lambda_{jk}}u & x\in I_{jk}, \ \forall I_{jk}\in\edgeSet, \\
			0 & x\in\vertSet,			
		\end{cases}
	\end{align*}
	where $\grad u$ denotes the weak gradient of $u\in H^1\bracs{\ddom}\cong\gradSob{\ddom}{\lambda_2}$.
	This is to say, in the bulk regions the function $u$ and its tangential gradient coincide with the familiar notion of a weak derivative (with respect to the Lebesgue measure).
	\item The function $u$ lives in $\gradSob{\ddom_i}{\lambda_2}$ for each of the bulk regions, and the traces of $u$ from $\ddom_i$ onto the inclusions $I_{jk}$ coincide with the values of $u^{(jk)}$ on the inclusions.
	This is as close to a condition of ``continuity across the inclusions" as we can get.
	Additionally, the $u^{(jk)}$ are continuous at the vertices of $\graph$, as was the case for functions in $\ktgradSob{\ddom}{\dddmes}$.
\end{itemize}

As with the variational problems of chapters \ref{ch:ScalarSystem} and \ref{ch:CurlCurl}, we understand \eqref{eq:SI-WaveEqn} in the variational sense: the problem of finding $\omega^2>0$ and non-zero $u\in\tgradSob{\ddom}{\ccompMes}$ such that
\begin{align} \label{eq:SI-WeakWaveEqn}
	\integral{\ddom}{ \tgrad_{\ccompMes}u\cdot\overline{\tgrad_{\ccompMes}\phi} }{\ccompMes}
	&= \omega^2 \integral{\ddom}{ u\overline{\phi} }{\ccompMes}, \quad\forall\phi\in\psmooth{\ddom}.
\end{align}
However as before we highlight that we could consider (for a fixed $\qm$) the bilinear map $b_{\qm}$ defined on pairs $(u,v)\in\tgradSob{\ddom}{\ccompMes}\times\tgradSob{\ddom}{\ccompMes}$ where
\begin{align*}
	b_{\qm}(u,v) &= \integral{\ddom}{ \tgrad_{\ccompMes}u\cdot\overline{\tgrad_{\ccompMes}v} }{\compMes}
	= \ip{\tgrad_{\ccompMes}u}{\tgrad_{\ccompMes}v}_{\ltwo{\ddom}{\ccompMes}^2},
\end{align*}
and use $b_{\qm}$ to define the self-adjoint operator $-\laplacian_{\ccompMes}^\qm$ by
\begin{align*} 
	\dom\bracs{ -\laplacian_{\ccompMes}^\qm } &= \clbracs{ u\in\tgradSob{\ddom}{\ccompMes} \setVert \exists f\in\ltwo{\ddom}{\ccompMes} \text{ s.t. } \right.
	\\
	& \qquad
	\left. \integral{\ddom}{ \tgrad_{\ccompMes}u\cdot\overline{\tgrad_{\ccompMes}v} }{\ccompMes} = \integral{\ddom}{ f\overline{v}}{\ccompMes}, \quad \forall v\in\tgradSob{\ddom}{\ccompMes} }, \labelthis\label{eq:CompLaplaceOpDom}
\end{align*}
with action $-\laplacian_{\ccompMes}^\qm = f$, where $u$ and $f$ are related as in \eqref{eq:CompLaplaceOpDom}.
Equation \eqref{eq:SI-WaveEqn} is then the eigenvalue equation for the operator $-\laplacian_{\ccompMes}^\qm$.
With $\ddom$ being bounded and $-\laplacian_{\ccompMes}^\qm$ self-adjoint, the spectrum of each $-\laplacian_{\ccompMes}^\qm$ consists of a discrete set\footnote{As we did in chapter \ref{ch:ScalarSystem}, we reiterate that it is not the purpose of this thesis to investigate the \emph{nature} of the spectrum, which is left to future work. Our interest lies solely in the computation of the values $\omega^2$ that make up the spectrum.} of values $\omega^2\in\reals$.
We can even utilise the min-max principle to write down a variational formulation whose solution determines the eigenvalues (and eigenfunctions) of $-\laplacian_{\ccompMes}^\qm$, which will form the basis of our first approach to numerically solving this problem.
Through our use of the Gelfand transform, taking the union of the spectra over $\qm$ will provide us with the spectrum of a periodic operator on $\reals^2$ with period cell $\ddom$.

Despite their useful analytic properties (as operators), \eqref{eq:CompLaplaceOpDom} and \eqref{eq:SI-WeakWaveEqn} do not lend themselves particularly well to explicit analytic solution, nor provide any insight into how to handle objects like $\tgrad_{\ccompMes}u$ numerically.
Indeed, the tangential gradients and the integrals in \eqref{eq:SI-WeakWaveEqn} with respect to $\compMes$ are unfamiliar both from an analytic and numerical standpoint --- we know some of their properties when restricted to different regions of $\ddom$, but not how to work with them directly to obtain a solution (or approximation thereof) to \eqref{eq:SI-WeakWaveEqn}.
The complications this gives rise to will motivate us to continue our search of an alternative (and more informative) realisation of \eqref{eq:SI-WeakWaveEqn}, leading to the ``strong formulation" obtained in section \ref{sec:SI-StrongDerivation}.
We will take this idea further in section \ref{sec:SI-NonLocalQG} when we attempt to reformulate our problem on the skeleton, and discard the bulk regions.

Our investigation into the acoustic approximation on composite domains will lean on the cross-in-the-plane geometry from the example in section \ref{ssec:ExampleCrossInPlane} (now equipped with $\ccompMes$) to ground our discussion of each of our numerical approaches, and illustrate their implementation.
For convenience, we have translated the period graph by $\bracs{-\recip{2},-\recip{2}}$, which allows us to avoid carrying constant terms around.
The period cell of this geometry now consists of a skeleton $\graph$ within $\ddom=\left[0,1\right)^2$, with a single vertex $v_0=\bracs{0,0}^\top$ with coupling constant $\alpha_0$, and two ``looping" edges $I_h = \sqbracs{0,1}\times\clbracs{0}$, $I_v=\clbracs{0}\times\sqbracs{0,1}$.
The quasi-momentum parameters $\qm_{jk}$ are easily computable as $\qm_h = \qm_1$ and $\qm_v = \qm_2$, and we only have a single bulk region, $\ddom_1 = \ddom^{\circ}=\bracs{0,1}^2$.
There are a number of properties that we expect of our eigenvalues due to the symmetric nature of our geometry, notably the following:
\begin{prop}[Cross in the plane symmetries] \label{prop:CrossInPlaneSymmetries}
	If $\omega^2, u\bracs{x_1,x_2}$ is a solution to \eqref{eq:SI-WaveEqn} at $\qm = \bracs{\qm_1,\qm_2}\in\left[-\pi,\pi\right)^2$, then:
	\begin{itemize}
		\item $\omega^2, u\bracs{1-x_1,x_2}$ is a solution to \eqref{eq:SI-WaveEqn} at $\qm = \bracs{-\qm_1, \qm_2}$.
		\item $\omega^2, u\bracs{x_1,1-x_2}$ is a solution to \eqref{eq:SI-WaveEqn} at $\qm = \bracs{\qm_1, -\qm_2}$.
		\item $\omega^2, u\bracs{x_2,x_1}$ is a solution to \eqref{eq:SI-WaveEqn} at $\qm = \bracs{\qm_2, \qm_1}$.
	\end{itemize}
\end{prop}

Additionally for the cross-in-the-plane geometry, we can obtain analytic expressions for a subset of the eigenfunctions and eigenvalues --- namely those inherited from the Dirichlet Laplacian on $\bracs{0,1}^2$.
Observe that for $n,m\in\naturals$, the function
\begin{align*}
	u_{n,m}(x) &= \e^{-\rmi\qm\cdot x}\sin\bracs{n\pi x_1}\sin\bracs{m\pi x_2},
\end{align*}
is an eigenfunction of the Dirichlet Laplacian $-\laplacian^{\qm}_0$, solving the system
\begin{align*}
	-\laplacian^{\qm}_0 u 
	:= \bracs{\partial_1+\rmi\qm_1}^2 u + \bracs{\partial_2+\rmi\qm_2}^2 u
	&= \omega_{n,m}^2 u, \qquad \text{on } \bracs{0,1}^2, \\
	u\bracs{0,x_2} = u\bracs{1,x_2} = u\bracs{x_1,0} = u\bracs{x_1,1} &= 0,
\end{align*}
where $\omega_{n,m}^2 = \bracs{n^2+m^2}\pi^2$.
It is clear that $u_{n,m}\in\tgradSob{\ddom}{\ccompMes}$ and has $u=0$ on the skeleton.
In the event that both
\begin{itemize}
	\item ($n$ is even and $\qm_1=0$) or ($n$ is odd and $\qm_1=-\pi$),
	\item ($m$ is even and $\qm_2=0$) or ($m$ is odd and $\qm_2=-\pi$),
\end{itemize}
hold, then $u_{n,m}$ solves \eqref{eq:SI-WaveEqn} with $\omega^2 = \omega_{n,m}^2$ --- the phase-factor $\e^{-\rmi\qm\cdot x}$ is required to ensure that certain terms in the variational formulation cancel out\footnote{Specifically, the normal derivative traces onto the skeleton (see \eqref{eq:SI-InclusionEqn}) only cancel under these assumptions.}.
This provides us with some suitable test cases for our numerical schemes.
It will be clear (upon viewing figure \ref{fig:CompositeCross-VP-SpectralBands}) that these eigenvalues of the Dirichlet Laplacian are also not the \emph{only} eigenvalues of the problem \eqref{eq:SI-WaveEqn} --- the two spectra are distinct and neither is contained in the other.
Knowing that \eqref{eq:SI-WaveEqn} shares eigenvalues with the Dirichlet Laplacian will also have consequences for us in section \ref{sec:SI-NonLocalQG}, when we attempt to reduce \eqref{eq:SI-WaveEqn} to a quantum graph problem.