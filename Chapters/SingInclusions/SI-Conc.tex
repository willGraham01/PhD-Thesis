\section{Conclusions and Further Exploration} \label{sec:SI-Conc}
We conclude our investigation into the various formulations of \eqref{eq:SI-WaveEqn}, our attempts to access the spectrum, and the insights we have obtained with a summary and survey of open questions motivated by this work.

Each of the sections \ref{sec:SI-VarProbMethod}, \ref{sec:SI-StrongDerivation}, and \ref{sec:SI-NonLocalQG} has provided us with an alternative formulation of \eqref{eq:SI-WaveEqn}.
We demonstrated in section \ref{sec:SI-VarProbMethod} that we can attempt to solve \eqref{eq:SI-WeakWaveEqn} directly through use of the min-max principle, appealing only to the understanding of $\tgradSob{\ddom}{\ccompMes}$ that we gained from section \ref{sec:CompSobSpaces}.
Whilst approximation of the eigenvalues, eigenfunctions, and dispersion relations was possible using this method, the formulation itself does not provide us with any explicit insights into the interactions between the skeleton and the bulk, nor the effect of the coupling constants $\alpha_j$.
We also highlighted that the lack of a priori knowledge concerning the explicit functions that live in $\tgradSob{\ddom}{\ccompMes}$, and the subset of these that are solutions to \eqref{eq:SI-WaveEqn}, forces us to use inefficient choices for our basis functions.
Indeed, having to hand an explicit orthonormal basis for $\tgradSob{\ddom}{\ccompMes}$, or some a priori information about the form of the eigenfunctions would go a number of ways towards fixing this issue.
We also highlight that if one is intent on solving \eqref{eq:SI-WaveEqn} directly from the variational problem, one might wish to consider a finite element based approach.
There would still be a number of issues surrounding explicitly determining a suitable basis to use in computations, however \eqref{eq:SI-WeakWaveEqn} is already in the form $b_{\qm}(u,\phi)=\ip{u}{\phi}$, so already invites solution via finite elements after checking the bilinear form $b_{\qm}$ is suitably well-behaved\footnote{That is, $b_{\qm}$ is bounded and elliptic on $\tgradSob{\ddom}{\ccompMes}$ so that one has convergence of such a scheme guaranteed.}.

The difficulties we came across when working in $\tgradSob{\ddom}{\ccompMes}$ lead us to the decision to derive a formulation of \eqref{eq:SI-WaveEqn} that does not involve the rather unfamiliar tangential gradients with respect to $\ccompMes$.
This lead us to the ``strong" formulation \eqref{eq:SI-StrongForm}, which clarifies the interactions between the bulk regions, skeleton, and vertices.
We observe that the inclusion of a background material causes a coupling between the solution in the bulk regions and on the skeleton through the \tstk{Ventcel'?} condition \eqref{eq:SI-InclusionEqn}, and bears additional similarities to effective problems derived in the context of elasticity (without geometric contrast between the edges and vertices).
Furthermore, the geometric contrast provided by the coupling constants $\alpha_j$ is again present in a non-classical Kirchoff condition at the vertices.
Whilst the physical material that we have been considering does not possess any material contrast between the bulk and skeleton, we highlight that we may ``add mass" to the measure $\lambda_2$ to capture the effects of such contrast in our strong formulation too.
The observations justify our approach via singular measures and physically motivated variational problems as a useful tool for providing an insight into the effective problems one can expect to obtain from the study of composite materials with geometric and/or material contrast.
However, the question is still open in the literature as to whether the (operator whose spectral problem is the) strong formulation \eqref{eq:SI-StrongForm} is the limit (either in the spectral or norm-resolvent sense) of a sequence of operators on a composite, periodic medium with both geometric and material contrast.
On the numerical side, we can exploit the additional information gained from \eqref{eq:SI-StrongForm} to approximate the action of the operator (that defines the left hand side of \eqref{eq:SI-FDMEquationsToDisc}) on $u$ in each of the bulk regions and on each edge of the skeleton, then couple the various ``regional" approximations via the flux across the edges and the non-classical Kirchoff condition at the vertices.
Such a scheme allows us to avoid the problems surrounding the space $\tgradSob{\ddom}{\ccompMes}$ that we ran into previously, albeit not without introducing a number of other considerations in setting up such a scheme.
Our example concerning the cross-in-the-plane geometry also demonstrates that the introduction of geometric contrast (that is, ensuring that $\alpha_j\neq0$) is sufficient to cause band-gaps to emerge in the spectrum of $-\bracs{\tgrad_{\ccompMes}}^2$.
There also appears to be good agreement with the approximations obtained from solution to \eqref{eq:SI-VarProb}, and convergence to the eigenfunctions and eigenvalues shared with the Dirichlet Laplacian.

The final step in our investigation looks into whether there is anything to be gained by attempting by attempting to ``remove" the bulk regions from our formulation entirely.
In light of chapter \ref{ch:ScalarSystem}, we already have a solid understanding quantum graph problems and how to analyse them, and additionally expect that solving a system of ODEs on the edges will be less complex than a system of PDEs coupled to ODEs (that themselves obey a non standard condition at the vertices).
It is possible for us to make such a reduction, provided that we stay away from those $\omega^2$ that are eigenvalues of the Dirichlet Laplacian on one of the bulk regions $\ddom_i$.
This is not ideal as the cross in the plane geometry demonstrates that such eigenvalues can form part of the spectrum of $-\bracs{\tgrad_{\ccompMes}}^2$.
For the eigenvalues that remain, we can realise \eqref{eq:SI-WaveEqn} as a quantum graph problem.
However this comes at the expense of introducing a non-local term involving Dirichlet-to-Neumann maps into the resulting system of edge ODEs.
Unfortunately, solution to such a problem numerically requires us to have the ability to evaluate (at least approximately) these Dirichlet-to-Neumann maps, the only reliable way being to return to solving PDEs in the bulk regions.
With a deeper understanding of the action of these Dirichlet-to-Neumann maps, a numerical approach to this non-local quantum graph problem might become tractable.
However at present, price we pay for removing the PDEs in the bulk regions from our formulation is not worth the complexity introduced by the non-local effects in the resulting ODEs on the skeleton.

In summary, the strong formulation \eqref{eq:SI-StrongFrom} provides the clearest picture of how the bulk regions, skeleton, and geometric contrast each interact with one another.
We also observe that by adding mass to the relevant measures in our formulation, effects from other forms of contrasts in the material being modelled also appear in the resulting system modelled by \eqref{eq:SI-WaveEqn}.
As such, our ``intuition-motivated" variational problems have provided us with a useful tool for predicting the effective problems for composite materials under contrast.
In pursuit of methods to numerically obtain the eigenvalues $\omega^2$, one can work with either \eqref{eq:SI-VarProb}, or through finite difference approximations in each of the bulk regions coupled along the skeleton.
Each method has a number of considerations to take into account, discussed in the respective sections.
And whilst it is possible to reduce \eqref{eq:SI-WaveEqn} to a problem on the skeleton, it is not possible to do so to recover all values of $\omega^2$ (those that correspond to Dirichlet eigenvalues) nor numerically solve the resulting system without a deeper understanding of the Dirichlet-to-Neumann map.

\subsection{Extensions and Related Open Questions}
Whilst we have been successful in obtaining realisable problems from our original variational problem, and in exploring some solution techniques that can be applied to them, it is important for us not to forget what motivated us to study such systems originally.
Similarly to our work in chapter \ref{ch:CurlCurl}, we have been able to provide a candidate for the limiting problem of the acoustic equation on a (periodic) composite material, with one of the composite materials shrinking to a singular structure.
This candidate also explicitly indicates where any parameters quantifying geometric or material contrast enter into the resulting system, allowing prediction of system behaviours when these parameters fall into certain ranges or adhere to certain (possibly critical) scalings.
A natural next step in the study of such variational problems would be consideration of the curl-of-the-curl equation on a composite domain, akin to how we used the ideas surrounding tangential gradients in chapter \ref{ch:ScalarSystem} to motivate definitions for the tangential curl in chapter \ref{ch:CurlCurl}.
Indeed, the results obtained from and argumentative techniques applied to the study of $\tgradSob{\ddom}{\ccompMes}$ and $\ktcurlSob{\ddom}{\dddmes}$ will be of great use should one choose to pursue such a study.
However, one will recall that the analysis of chapter \ref{ch:CurlCurl} on a singular structure demonstrated that the curl of the curl equation reduced to the acoustic approximation, raising the question as to whether we should expect to find anything different on a composite domain.
In response to this, we highlight that the automatic reduction to the acoustic approximation was due to the two dimensional nature of the domain we were considering --- the presence of the bulk regions (and correspondingly the measure $\lambda_2$) in a composite domain no longer places us in such a situation.
Further to this, we can argue heuristically (as we did at the beginning of section \ref{sec:SI-StrongDerivation}) and predict that the ``strong formulation" of the variational problem we pose --- provided it exists --- will consist of the classical curl-of-the-curl equation in the bulk regions.
It is  much harder to predict what behaviours will emerge from interactions between the bulk and skeleton regions, and in particular what (if anything) the analogue of \eqref{eq:SI-InclusionEqn} will be.

Unlike the current state of the literature surrounding convergence of problems on thin-structures (section \ref{sec:Intro-ThinStructures}), where the question of spectral \tstk{and norm-resolvent??} convergence to quantum graph problems\footnote{Which by the work of chapter \ref{ch:ScalarSystem}, are equivalent to our physically-motivated variational problems on singular structures.} has been well studied, these questions are largely unexplored for periodic composite domains in an electromagnetic setting.
The question of spectral convergence for the acoustic approximation has been explored in \tstk{K-F, graph convergence paper. Also Kuchment's review paper maybe, since that has relevant studies up to 2001?} for the acoustic equation, but this setup lacked any geometric contrast between the vertex and edge regions.
Results on the convergence of the resolvents of such problems are also presently unaddressed and open to investigation.
In light of our approach to the study of variational problems on singular structures, we have provided a tool for predicting the ``limit" operators one might consider\footnote{Including how one can define such operators and the appropriate function spaces.}, and indicated how it can extended to systems more general that the acoustic approximation.
Whilst the establishment of the appropriate convergence results have not been the concern of this work; with the conclusion of our study of variational problems on singular structures, the absence of these convergence results in the literature comes to the forefront of our attention.