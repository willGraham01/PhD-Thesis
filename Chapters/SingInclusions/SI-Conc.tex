\section{Conclusions and Further Exploration} \label{sec:SI-Conc}
Each of the sections \ref{sec:SI-VarProbMethod}, \ref{sec:SI-StrongDerivation}, and \ref{sec:SI-NonLocalQG} has provided us with an alternative formulation of \eqref{eq:SI-WaveEqn}.
We demonstrated in section \ref{sec:SI-VarProbMethod} that we can attempt to solve \eqref{eq:SI-WeakWaveEqn} directly through use of the min-max principle, appealing only to the understanding of $\tgradSob{\ddom}{\ccompMes}$ that we gained from section \ref{sec:CompSobSpaces}.
Whilst approximation of the eigenvalues, eigenfunctions, and dispersion relations was possible using this method, the formulation itself does not provide us with any explicit insights into the interactions between the skeleton and the bulk, nor the effect of the coupling constants $\alpha_j$ representing the geometric contrast in our material.
We also highlighted that the lack of a priori knowledge about the explicit form of the functions that live in $\tgradSob{\ddom}{\ccompMes}$ introduces complications when approximating solutions to (and eigenvalues of) \eqref{eq:SI-VarProb}.
Having an explicit orthonormal basis for $\tgradSob{\ddom}{\ccompMes}$ available, or some a priori information about the form of the eigenfunctions, would aid in addressing this issue.
However use of the min-max principle also requires us to sequentially compute eigenvalues and eigenfunctions, which makes examination of higher spectral bands increasingly troublesome.

The unfamiliarity of the space $\tgradSob{\ddom}{\ccompMes}$ made us seek a formulation of \eqref{eq:SI-WaveEqn} that did not involve tangential gradients with respect to $\ccompMes$, and which made explicit the interaction between the bulk regions, skeleton, and vertices.
This lead us to the ``strong" formulation \eqref{eq:SI-StrongForm}, from which we observed that the presence of a background material causes a coupling between the bulk regions and the skeleton through the condition \eqref{eq:SI-InclusionEqn}.
Similarly to the quantum graph problems we had studied in chapter \ref{ch:ScalarSystem}, this coupling manifests in the form of a Wentzell condition, only this time between the ``flux" from the two-dimensional bulk regions and diffusion ``along" the one-dimensional skeleton.
Further to this point, the geometric contrast provided by the coupling constants $\alpha_j$ is again present in a Wentzell condition at the vertices.
We also pursued a deeper understanding of the effects that the $\alpha_j$ represent, and further generalisations to our problem --- including introducing differences between the material properties of the bulk regions and skeleton.
This lead to the discussion of section \ref{ssec:SI-DoubleLimitReconcile}; we observed coincidence of (a suitable generalisation to) \eqref{eq:SI-StrongForm} with the problem \eqref{eq:Intro-KuchFigQGLimit}, which was derived as the limiting problem for a composite material under critical contrast.
Further to this, we also indicated potential extensions to the study \cite{figotin1998spectral} --- particularly an extension to encompass geometric contrasts --- and how these might be realised through our variational problems.
These observations served to justify our approach via singular measures and physically motivated variational problems as a predictive tool for the effective problems one expects to obtain from the study of composite materials with geometric and/or material contrast.

On the numerical side, we can exploit the additional information gained from \eqref{eq:SI-StrongForm} to approximate the action of the operator (that defines the left hand side of \eqref{eq:SI-FDMEquationsToDisc}) on $u$ in each of the bulk regions and on each edge of the skeleton, then couple these ``regional" approximations via using the aforementioned two Wentzell-type conditions.
Our example concerning the cross-in-the-plane geometry also demonstrates that the introduction of geometric contrast (that is, ensuring that $\alpha_j\neq0$) is sufficient to cause band-gaps to emerge in the spectrum of $-\bracs{\tgrad_{\ccompMes}}^2$.
There also appears to be good agreement with the approximations obtained from solution to \eqref{eq:SI-VarProb}, and convergence to the eigenfunctions and eigenvalues shared with the Dirichlet Laplacian.
We have been able to avoid the problems surrounding the space $\tgradSob{\ddom}{\ccompMes}$ that we ran into previously, however non-rectangular skeleton geometries quickly complicate the implementation of finite difference based schemes.

The approach taken to solving the strong formulation \eqref{eq:SI-StrongForm} suffered from increasing complexity on irregular geometries, due to it being based off finite difference schemes.
However the approach via the min-max principle suffered from complications when selecting and using global approximating functions for the minimisers of \eqref{eq:SI-VarProb}.
In both of these cases, we highlighted that an approach based on finite elements might be considered.
The problem \eqref{eq:SI-WeakWaveEqn} is already in the ``weak" form $b_{\qm}(u,\phi)=\ip{u}{\phi}$, and $b_{\qm}$ is a bilinear form on the Hilbert space $\tgradSob{\ddom}{\ccompMes}$ --- if $b_{\qm}$ is continuous and elliptic one already has the result of Cea's lemma (albeit for the ``resolvent problem" $-\laplacian_{\ccompMes}^\qm u=f$), for example.
The other consideration is again the choice of appropriate basis functions, since such an approach will have to approximate functions in $\tgradSob{\ddom}{\ccompMes}$.
With a finite element scheme one typically employs local basis functions\footnote{The form of the basis functions is a rather general choice and marks the difference between a number of finite difference schemes. Here we are tailoring our discussion to be as general as possible.} as opposed to the global basis functions we employed in section \ref{sec:SI-VarProbMethod}.
Additionally, away from the skeleton functions in $\tgradSob{\ddom}{\ccompMes}$ behave as classical Sobolev functions, and so one can utilise existing finite element basis functions in these regions.
Prescribing the approximating basis functions with support on the skeleton, in such a way that is consistent with the approximations in the neighbouring bulk regions, is the more difficult task.
A sensible starting point would be to extend a one-dimensional finite element basis along each edge ``off" the skeleton.
This approach presents a potential solution to the key issues from sections \ref{sec:SI-VarProbMethod} and \ref{ssec:SI-FDMMethod}.
The use of local basis functions allows us to fall back into the ``classical" setting on the bulk regions, and we now only need to determine appropriate local basis functions within the vicinity of the skeleton.
Simultaneously, using a finite element scheme circumnavigates the difficulties that finite difference schemes encounter with irregular geometries, as one is no longer relying on (somewhat regular) node placement.
We invite readers interested in the numerical solution to the variational problem \eqref{eq:SI-WaveEqn} to pursue this avenue of research\footnote{We also highlight that there are several finite-element software packages seeing active maintenance and development that allow the user to implement custom function spaces (or approximations thereof), such as FEniCS \cite{FENICS2022fenics} or Firedrake \cite{Firedrake2022firedrake}.
Such software could be used for the numerical implementation of finite-element based schemes, should the aforementioned issues surrounding the determination of an appropriate set of basis functions be resolved.}.

The final step in our investigation looks into whether there is anything to be gained by attempting by attempting to ``remove" the bulk regions from our formulation entirely, attempting to work on the skeleton directly.
It is possible for us to make a transition to such a problem, provided that we stay away from those $\omega^2$ that are eigenvalues of the Dirichlet Laplacian on any one of the bulk regions $\ddom_i$ --- this is already not ideal as the cross-in-the-plane geometry demonstrates that such eigenvalues can form part of the spectrum of $-\laplacian_{\ccompMes}^\qm$.
However for the eigenvalues that remain, we can realise \eqref{eq:SI-WaveEqn} as a quantum graph problem.
This comes at the expense of introducing a non-local term involving Dirichlet-to-Neumann maps into the resulting system of edge ODEs.
Unfortunately, solution to such a problem (analytically or numerically) requires us to have the ability to evaluate (at least approximately) these Dirichlet-to-Neumann maps --- without a closed form for such maps, the only reliable way to approximate this action forces us to to return to solving PDEs in the bulk regions.
With a deeper understanding of the action of these Dirichlet-to-Neumann maps, a numerical approach to this non-local quantum graph problem might become tractable, however this is asking for a great deal of information.
At present the price paid for removing the PDEs in the bulk regions is not worth the complexity introduced by the non-local effects in the resulting system \eqref{eq:SI-NonLocalQG} --- particularly when we have the successful alternative methods from sections \ref{sec:SI-VarProbMethod} and \ref{sec:SI-StrongDerivation}, and a potential approach through finite elements available.

In summary, the strong formulation \eqref{eq:SI-StrongForm} provides the clearest picture of how the bulk regions, skeleton, and geometric contrast each interact with one another.
We also observe that by adding mass to the relevant measures in our formulation, we can mimic parameters corresponding to other forms of contrast in the material being modelled.
Further to this point, we also see similarities (and coincidence) of the problem defined by the ``strong formulation" \eqref{eq:SI-StrongForm} and the effective problems obtained in studies of the zero-thickness, high material contrast limits of the acoustic equation on composite materials (see section \ref{sec:SI-StrongDerivation}).
Our variational problems have again provided a useful tool for predicting the effective problems for composite materials under contrast.
In pursuit of methods to numerically obtain the eigenvalues $\omega^2$, we have presented two viable approaches for simple skeletal geometries in sections \ref{sec:SI-VarProbMethod} and \ref{sec:SI-StrongDerivation}.
We suggested future research in this direction focus on the potential for the implementation of a finite element scheme to address the issues that arise from approximating functions in the space $\tgradSob{\ddom}{\ccompMes}$ and their tangential gradients, and from irregular skeletal geometries.
Finally, whilst it is possible to reduce \eqref{eq:SI-WaveEqn} to a problem on the skeleton and thus a quantum graph problem, doing so does not yield any particular benefits.
There are certain eigenvalues(namely those that correspond to Dirichlet eigenvalues) for which we cannot reformulate the problem \eqref{eq:SI-WaveEqn} into a quantum graph problem, and the resulting system when we can is non-local.
Numerically solving the resulting system \eqref{eq:SI-NonLocalQG} without a deeper understanding of the source of this non-locality --- the Dirichlet-to-Neumann map --- requires ultimately accepting that we must return to solving two dimensional PDEs on the bulk regions.

\subsection{Related open questions and further extensions}
Whilst we have been successful in obtaining realisable problems from our original variational problem, and in exploring some solution techniques that can be applied to them, it is important for us not to forget what motivated us to study such systems originally.
Similarly to our work in chapter \ref{ch:CurlCurl}, we have been able to provide a candidate for the limiting problem of the acoustic equation on a (periodic) composite material, with one of the composite materials shrinking to a singular structure.
This candidate also explicitly indicates where any parameters quantifying geometric or material contrast enter into the resulting system, allowing prediction effective problems for the limits of composite materials under contrast.
Unlike the current state of the literature surrounding convergence of problems on thin-structures (section \ref{ssec:Intro-ThinStructures}), where the question of convergence (both spectral and norm-resolvent) to quantum graph problems\footnote{Which by the work of chapter \ref{ch:ScalarSystem}, are equivalent to our physically-motivated variational problems on singular structures.} has been well studied, these questions are largely unexplored for periodic composite domains --- particularly those in an electromagnetic setting.
The question of spectral convergence for the acoustic approximation has been explored in \cite{figotin1998spectral} (see also \cite[section 7.5]{kuchment2001mathematics}), and we explored the coincidence of \eqref{eq:SI-StrongForm} with the effective problem derived in this study, but this setup lacked any geometric contrast between the vertex and edge regions.
Results on the convergence of the resolvents of such problems are also presently unaddressed and open to investigation.
As has been thematic throughout this work, and in keeping with our original motivations, our approach to the study of variational problems on singular structures has provided a tool for predicting the ``limit" operators one might consider\footnote{Including how one can define such operators and the appropriate function spaces.}, and indicated how it can extended to systems more general that the acoustic approximation.
Whilst the establishment of the appropriate convergence results have not been the concern of this work; with the conclusion of our study of variational problems on singular structures, and the discussion of section \ref{ssec:SI-DoubleLimitReconcile}, the absence of these convergence results in the literature comes to the forefront of our attention.

Looking further forward, a natural next step for the study of variational problems on singular structures would be consideration of the curl-of-the-curl equation on a composite domain, akin to how we used the ideas surrounding tangential gradients in chapter \ref{ch:ScalarSystem} to motivate definitions for the tangential curl in chapter \ref{ch:CurlCurl}.
Indeed, the results obtained from and argumentative techniques applied to the study of $\tgradSob{\ddom}{\ccompMes}$ and $\ktcurlSob{\ddom}{\dddmes}$ will be of great use should one choose to pursue such a study.
One will recall that the analysis of chapter \ref{ch:CurlCurl} on a singular structure demonstrated that the curl-of-the-curl equation reduced to the acoustic approximation.
The reduction to the acoustic approximation was due to the two-dimensional nature of the domain we were considering --- the presence of the bulk regions (and correspondingly the measure $\lambda_2$) in a composite domain no longer places us in such a situation, and raises the question as to whether we should expect to find anything different on a composite domain.
Further to this, we can argue heuristically (as we did at the beginning of section \ref{sec:SI-StrongDerivation}) and predict that the ``strong formulation" of the variational problem we pose --- provided it exists --- will consist of the classical curl-of-the-curl equation in the bulk regions.
It is  much harder to predict what behaviours will emerge from interactions between the bulk and skeleton regions, and in particular what (if anything) the analogue of \eqref{eq:SI-InclusionEqn} will be.
This again opens investigation along the lines pursued in this chapter into the solution of such problems, and how the parameters corresponding to contrast manifest themselves in the resulting problems.
Such investigation also extends the scope of the discussion of the previous paragraph, supporting a formal analysis of the limits of the curl-of-the-curl equation on composite materials.