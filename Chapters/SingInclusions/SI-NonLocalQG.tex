\section{Reformulation of \eqref{eq:SI-WaveEqn} as a Non-Local Quantum Graph Problem} \label{sec:SI-NonLocalQG}
Both of the methods employed in sections \ref{sec:SI-VarProbMethod} and \ref{ssec:SI-FDMMethod} force us to handle the interaction between the bulk regions and skeleton in some way; either implicitly through the global approximation we use (if working from \eqref{eq:SI-VarProb}) or explicitly in how we tie the various approximations in the bulk regions and skeleton together (in the approach of section \ref{ssec:SI-FDMMethod}).
We now ask the question as to whether we can go further, and find a formulation for our problem that allows us to remove the equations in the bulk regions, replacing them with suitable terms in the equations along the skeleton.
Our objective is to arrive at a quantum graph problem; we have seen the effectiveness of the techniques involving the $M$-matrix for extracting the spectrum of the problems in chapter \ref{ch:ScalarSystem} (and by extension chapter \ref{ch:CurlCurl}), and hope to find similar success by transforming the problem \eqref{eq:SI-StrongForm}.
From the perspective of solution (either numerically or analytically) we also expect to see a reward in dimension reduction if we can reduce \eqref{eq:SI-StrongForm} to a problem on the one-dimensional skeleton.

\subsection{Formulation of the non-local quantum graph problem} \label{ssec:SI-ToQG}
In order to investigate how to reduce \eqref{eq:SI-WaveEqn} to a quantum graph problem, we need a method of reconstructing the behaviour of a solution $u$ in the bulk regions from its behaviour on the skeleton.
We already know that the eigenfunction itself is continuous across the skeleton, so the function values on the boundary of each bulk region coincide with the function values on the skeleton.
Furthermore, we also observe that the only information that the skeleton needs about the solution in the bulk regions are the traces of the normal derivative.
So the question becomes whether there is a method or function through which we can extract the (traces of the) normal derivatives from the bulk regions from the (boundary values of the) function on the skeleton, given that we know \eqref{eq:SI-BulkEqn} is satisfied in each bulk region.
Our attention is naturally drawn to the Dirichlet-to-Neumann map for the Helmholtz equation on $\ddom_i$.

For each bulk region $\ddom_i$, fix $\omega^2>0$ and a solution $u\in H^2\bracs{\ddom_i}$ to the problem
\begin{align*}
	\bracs{ \laplacian^\qm + \omega^2 }u &= 0, \qquad\text{on } \ddom_i, \\
	u\vert_{\partial\ddom_i} &= g,
\end{align*}
where $g\in\ltwo{\partial\ddom_i}{S}$.
If $\omega^2$ is an eigenvalue of the Dirichlet Laplacian $-\laplacian^\qm_0$ on $\ddom_i$, then notice that the solution $u$ to the above problem is not unique --- we can add any multiple of the eigenfunction $u_f$ of $-\laplacian^\qm_0$ corresponding to the eigenvalue $\omega^2$ to $u$ and obtain another solution.
Consequentially the Neumann data of a solution is not uniquely determined by its Dirichlet data $u\vert_{\partial\ddom_i}$ at such values of $\omega^2$, and thus the Dirichlet to Neumann map $\dtn^i_\omega$ for the operator $\laplacian_{\qm}+\omega^2$ is not well-defined at this value of $\omega^2$.
For those values of $\omega^2$ that are not eigenvalues of $-\laplacian^\qm_0$ however, we can define the Dirichlet to Neumann map $\dtn^i_\omega$ by first letting
\begin{align*}
	\dom\bracs{\dtn^i_\omega} = \{ \bracs{g,h}\in\ltwo{\partial\ddom_i}{S}\times\ltwo{\partial\ddom_i}{S} \ \vert \
	& \exists v\in\gradgradSob{\ddom_i}{\lambda_2} \text{ s.t. } \\
	&\quad \bracs{\laplacian_\qm + \omega^2}v = 0, \\
	&\quad v\vert_{\partial\ddom_i}=g, \ \left.\bracs{\tgrad v\cdot n}\right\vert_{\partial\ddom_i} = h \},
	\labelthis\label{eq:SI-DtNMapRegion}
\end{align*}
and assigning the action $\dtn^i_\omega g = h$ where $g,h$ are related as in \eqref{eq:SI-DtNMapRegion}.

Now suppose we have a solution $u, \omega^2$ to \eqref{eq:SI-StrongForm}, with $\omega^2$ not being an eigenvalue of $-\laplacian^\qm_0$ on any of the bulk regions $\ddom_i$.
The boundary of $\ddom_i$ is a union of a (finite number of) the skeletal edges $I_{jk}$, and since $u\in\tgradSob{\ddom}{\ccompMes}$, we also have that $u\in\tgradSob{\ddom}{\dddmes}$ and consequentially $u\in\gradSob{\partial\ddom_i}{S}$.
Given continuity of $u$ across the skeleton, and $u\in\gradSob{\ddom_i}{\lambda_2}$, we have that $u$ serves the role of $v$ in \eqref{eq:SI-DtNMapRegion}, with $g=u\vert_{\partial\ddom_i}$.
Therefore,
\begin{align*}
	\bracs{\dtn^+_\omega u}\vert_{I_{jk}} = - \bracs{\tgrad u\cdot n_{jk}}^+,
	&\quad
	\bracs{\dtn^-_\omega u}\vert_{I_{jk}} = \bracs{\tgrad u\cdot n_{jk}}^-,
\end{align*}
where we have used $\dtn^\pm_\omega$ to denote the Dirichlet-to-Neumann maps for the regions $\ddom^{\pm}_{jk}$ corresponding to the edge $I_{jk}$.
Substituting into \eqref{eq:SI-InclusionEqn} gives us the new equation
\begin{align*}
	- \bracs{\diff{}{y} + \rmi\qm_{jk}}^2u^{(jk)} 
	&= \omega^2 u^{(jk)} - \bracs{ \dtn^+_\omega u + \dtn^-_\omega u },
\end{align*}
on each $I_{jk}$.

Now consider a solution $\omega^2>0$, $u\in H^2\bracs{\graph}$ of the problem
\begin{subequations} \label{eq:SI-NonLocalQG}
	\begin{align}
		- \bracs{\diff{}{y} + \rmi\qm_{jk}}^2u^{(jk)} 
		&= \omega^2 u^{(jk)} - \bracs{ \dtn^+_\omega u^{(jk)} + \dtn^-_\omega u^{(jk)}},
		&\qquad\text{on } I_{jk}, \label{eq:SI-NonLocalQGEdgeEquation}  \\
		\sum_{j\con k} \bracs{\pdiff{}{n}+\rmi\qm_{jk}} u^{(jk)}(v_j) &= \alpha_j\omega^2 u(v_j),
		&\qquad\text{at every } v_j\in\vertSet, \label{eq:SI-NonLocalQGVertexDeriv} \\
		u \text{ is continuous,} & 
		&\qquad\text{at every } v_j\in\vertSet, \label{eq:SI-NonLocalQGVertexCont}
	\end{align}
\end{subequations}
where again we assume $\omega^2$ is not in the spectrum of $-\laplacian^\qm_0$ on any $\ddom_i$.
Then this eigenpair of \eqref{eq:SI-NonLocalQG} also defines a solution $\omega^2, \tilde{u}$ to \eqref{eq:SI-StrongForm}; take $\tilde{u}=u$ on the skeleton $\graph$, and in the bulk region $\ddom_i$ assign $\tilde{u}$ the values of the function $v$ in \eqref{eq:SI-DtNMapRegion} for $g=u$.
So excluding the possibility for eigenvalues being shared with $-\laplacian^\qm_0$ on one of the bulk regions $\ddom_i$, the system \eqref{eq:SI-NonLocalQG} shares the same eigenvalues as \eqref{eq:SI-StrongForm}.

The problem \eqref{eq:SI-NonLocalQG} is inherently non-local; for each edge $I_{jk}$ the terms $\dtn^{\pm}_\omega u$ require information about the values of the edge functions $u^{(lm)}$ for all edges $I_{jk}\subset\ddom_{jk}^{\pm}$, and these edges are not necessarily directly connected to $I_{jk}$ itself.
Indeed, the appearance of the Dirichlet to Neumann map in the ODEs along the skeleton is the price that we must pay for removing the bulk regions, but still needing to retain the information about how their presence induces interactions between non-adjacent edges.
Our previous approaches that used the $M$-matrix to access the eigenvalues of quantum graph problems (sections \ref{sec:ScalarDiscussion} and \ref{sec:ScalarExamples}) largely depended upon computing an explicit form for the entries of the $M$-matrix.
This is currently unfeasible as there is no closed form for the action of the map $\dtn^i_\omega$ on a general polygonal domain.
This lack of a closed form, and the fact that $\dtn^i_\omega$ depends on our spectral parameter $\omega^2$  also has serious drawbacks when we hypothesise about solving \eqref{eq:SI-NonLocalQG} numerically, which is the subject of section \ref{ssec:SI-GraphMethod}.

We have already mentioned that, when $\omega^2$ is an eigenvalue of $-\laplacian^\qm_0$ on any of the bulk regions $\ddom_i$ we cannot apply the same reasoning to reduce \eqref{eq:SI-StrongForm} to a problem on the skeleton.
It is however worth briefly exploring how much we can do to reduce \eqref{eq:SI-StrongForm} to the skeleton if we happen to land on a value $\omega_i^2$ which \emph{is} an eigenvalue of $-\laplacian^\qm_0$ in one of the bulk regions $\ddom_i$.
Let $\varphi^{(i)}_n$ be the eigenfunction(s) corresponding to this eigenvalue (where the index $n$ ranges from $1$ to the multiplicity $N_i$ of $\omega_i^2$), and write $S_i = \mathrm{span}\clbracs{\varphi^{(i)}_n}$.
As we have already seen, we can add linear combinations of these $\varphi^{(i)}_n$ to a function $u$ that satisfies \eqref{eq:SI-BulkEqn} in $\ddom_i$ and still end up with a function $\tilde{u} := u + \sum_{n}c_n\varphi^{(i)}_n$ (where $c_n\in\complex$) that satisfies \eqref{eq:SI-BulkEqn} with $\tilde{u}\vert_{\partial\ddom_i} = u\vert_{\partial\ddom_i}$.
However the the traces of the normal derivatives of $u$ and $\tilde{u}$ onto $\partial\ddom_i$ no longer match, and we have some additional ``freedom" in \eqref{eq:SI-InclusionEqn} when the edge $I_{jk}$ forms part of the boundary of $\ddom_i$.
Indeed, if there exists a $\phi\in S_i$ such that $u$ satisfies
\begin{subequations}
	\begin{align*}
		-\laplacian_{\qm}u &= \omega_i^2 u, &\text{in } \ddom_i, \\
		-\bracs{\diff{}{y} + \rmi\qm_{jk}}^2u^{(jk)}  
		&= \omega_i^2 u^{(jk)} + \bracs{\bracs{\grad u\cdot n_{jk}}^+ - \bracs{\grad u\cdot n_{jk}}^-} + \mathrm{T}\phi,
		&\text{on } I_{jk}\subset\partial\ddom_i, \\
		- \bracs{\diff{}{y} + \rmi\qm_{jk}}^2u^{(jk)} 
		&= \omega^2 u^{(jk)} - \bracs{ \dtn^+_\omega u^{(jk)} + \dtn^-_\omega u^{(jk)}},
		&\text{on } I_{jk}\cap\partial\ddom_i=\emptyset, \\
		\sum_{k} \bracs{\pdiff{}{n}+\rmi\qm_{jk}} u^{(jk)}(v_j) 
		&= \alpha_j\omega^2 u(v_j),
		&\text{at each } v_j\in\vertSet,
	\end{align*}
\end{subequations}
where
\begin{align*}
	\mathrm{T}\phi &=
	\begin{cases}
		0, & \ddom_i\neq\ddom_{jk}^{\pm}, \\
		\bracs{\grad\phi\cdot n_{jk}}^+, & \ddom_i = \ddom_{jk}^+, \\
		-\bracs{\grad\phi\cdot n_{jk}}^-, & \ddom_i = \ddom_{jk}^-, \\
		\bracs{\grad\phi\cdot n_{jk}}^+ -\bracs{\grad\phi\cdot n_{jk}}^-, & \ddom_i = \ddom_{jk}^+ \text{ and } \ddom_i = \ddom_{jk}^-,
	\end{cases}
\end{align*}
then $u, \omega_i^2$ will be a solution to \eqref{eq:SI-StrongForm}.
The $\mathrm{T}\phi$ term encodes this aforementioned freedom that comes from our ability to add eigenfunctions of $-\laplacian^\qm_0$ to $u$ in $\ddom_i$, and the knock-on affect on the equations on the skeletons.
If one wishes to look for a solution to this problem, then we have to accept that we will be solving for a tuple $\bracs{u, \clbracs{c_n}}\in H^2\bracs{\graph}\times\complex^{N_i}$ where $\phi=\sum_n c_n\varphi^{(i)}_n$, and incorporate this accordingly into our solution method.

\subsection{Considerations for numerical solution on the skeleton} \label{ssec:SI-GraphMethod}
Although we have succeeded in obtaining the problem \eqref{eq:SI-NonLocalQG} which is posed on a (collection of) one-dimensional domains from the two-dimensional problem \eqref{eq:SI-StrongForm} on $\ddom$, the price we have paid is rather severe.
The handling of the Dirichlet-to-Neumann map term is the main complication with any numerical approach to solving \eqref{eq:SI-NonLocalQG}.
The non-locality in \eqref{eq:SI-NonLocalQGEdgeEquation} rules out numerical methods that rely on local approximations to the solution, such as the regional finite differences similar to those employed in \ref{ssec:SI-FDMMethod}, as we have no way of expressing the action of $\dtn^i_\omega$ \emph{solely} in terms of nearby function values.
It also rules out our previous approach involving the $M$-matrix\footnote{There is also the pressing theoretical question as to whether \eqref{eq:SI-NonLocalQG} admits a boundary triple.} since we can no longer construct the entries of the $M$-matrix just from the knowledge of the function values at the vertices.

Given the non-locality that $\dtn^i_\omega$ introduces, we must again turn to ideas from spectral methods to approximate a solution $u$ to \eqref{eq:SI-NonLocalQG}.
So let us suppose that we have some finite dimensional subspace $V$ spanned by functions $\psi_m$, $1\leq m\leq M$, in which we will represent our solution by the formula in \eqref{eq:SI-VPTruncatedBasis}.
The difficulty that we face with any numerical scheme is in how we approximate the action of the maps $\dtn^i_\omega$ on these basis functions $\psi_m$.
This problem is compounded by the fact that $\dtn^i_\omega$ themselves depend on the eigenvalue $\omega^2$, so however we choose to compute $\dtn^i_\omega \psi_m$ has to be general enough to account for the possibility of evaluation at different values of $\omega$.
That is, we need to know the action of $\dtn_{\omega}^i$ for every bulk region $\ddom_i$, on \emph{every} basis function $\psi_m$, or at the least be able to approximate this.
Asking for an analytic expression is a tall order, especially given the wide variety of shapes that the bulk regions $\ddom_i$ can take --- in general, $\ddom_i$ could be any bounded, polygonal domain.
If we only need a numerical approximation, our obvious option for computing $\dtn^i_\omega\psi_m$ is solving the equation \eqref{eq:SI-BulkEqn} subject to the boundary conditions $u=\psi_m$, and reading off the Neumann data.
However this takes us back to numerically solving each of the problems
\begin{align*}
	-\laplacian_{\qm}v &= \omega^2 v &\text{on } \ddom_i, \\
	v\vert_{\partial\ddom_i} &= \psi_m,
\end{align*}
for every $m=1,...,M$ and bulk region $\ddom_i$, and then reading off the Neumann data of the computed solution.
Doing so means that we are no longer only considering a one-dimensional problem: we might as well employ the techniques of section \ref{ssec:SI-FDMMethod} and directly compute approximations over the whole of $\ddom$, rather than solve $M$ second-order equations on each of the bulk regions $\ddom_i$ to extract $\dtn^i_\omega\psi_m$ and then solve a one-dimensional problem to construct $u$ from the $\psi_m$.

Possible ideas for making progress with a numerical scheme might include representing $u$ on each $\partial\ddom_i$ as a sum of the eigenfunctions of the map $\dtn^i_\omega$.
This makes evaluation of the action of $\dtn^i_\omega$ cheap, provided one has access to the eigenvalues $w^{(i)}_n$ and eigenfunctions $\varphi^{(i)}_n$ of $\dtn^i_\omega$.
These are the so-called Steklov eigenvalues and eigenfunctions, that solve the system (Steklov eigenvalue problem)
\begin{align*}
	\bracs{\laplacian_\qm + \omega^2}\varphi^{(i)}_n &= 0, \qquad\text{in } \ddom_i, \\
	\varphi^{(i)}_n\vert_{\partial\ddom_i} &= w^{(i)}_n\pdiff{\varphi^{(i)}_n}{n}\vert_{\partial\ddom_i}. 
\end{align*}
However one then needs to be able to reconcile different representations of $u$ on edges common to two different bulk regions, and still faces the prospect of computing solutions to the Steklov problem in each of the bulk regions.
Further details and the questions that require answers pertaining to these points, from our attempts to implement such a method, are available for the interested reader in the appendix \ref{sec:SIApp-NonLocalSolve}.
Ultimately however, short of obtaining an analytic expression (or an expression that can be easily approximated numerically) for the action of $\dtn^i_{\omega}$ in terms of $\omega$, there is no simple way to deal with the action of $\dtn^i_{\omega}$ that avoids consideration of PDEs in the bulk regions.

%If we are to accept this non-locality we must use the ideas from spectral methods; decide on a finite-dimensional subspace $V$ in which to approximate the solution to $u$, determine a suitable basis for this space, the expand our approximate solution in terms of this basis and choose the coefficients of the basis expansion to satisfy the problem \eqref{eq:SI-NonLocalQG} in $V$.
%For the time being, we will simply let $V\subset\htwo{\graph}$ be a finite-dimensional subspace with dimension $M$ and basis functions $\clbracs{\psi_m}_{m=1}^{M}$.
%For the purposes of our discussion we will also take $\alpha_j=0$ at each of the vertices, since the major problems with a numerical approach to \eqref{eq:SI-NonLocalQG} are present whether or not the coupling constants are non-zero.
%Write the approximate solution $u_V\in V$ to \eqref{eq:SI-NonLocalQG} as
%\begin{align*}
%	u_V &= \sum_{m=1}^M u_m\psi_m, \qquad u_m\in\complex,
%\end{align*}
%for basis coefficients $u_m$ to be determined.
%We then multiply write \eqref{eq:SI-NonLocalQG} in the (``weak") form
%\begin{align*}
%	\sum_{v_j\in\vertSet}\sum_{j\conLeft k} 
%	\clbracs{ 
%	\integral{I_{jk}}{\tgrad_{\lambda_{jk}}u\cdot\overline{\tgrad_{\lambda_{jk}}\phi}}{\lambda_{jk}} 
%	+ \integral{I_{jk}}{\overline{\phi}\dtn^+_{\omega}u + \overline{\phi}\dtn^-_{\omega}u}{\lambda_{jk}} 
%	}
%	&= \omega^2 \sum_{v_j\in\vertSet}\sum_{j\conLeft k}\integral{I_{jk}}{u\overline{\phi}}{\lambda_{jk}},
%\end{align*}
%which holds for each $\phi$ in a suitable space of test functions.
%From here, we replace $u$ with $u_V$, substitute the basis expansion, and choose $\phi=\psi_n$ for each $n\in\clbracs{1,...,M}$ to obtain
%\begin{align*}
%	0 &= 
%	\sum_{m=1}^{M} u_m \bracs{ A_{n,m} + L_{n,m} - \omega^2 B_{n,m} }, \\
%	A_{n,m} &= \sum_{v_j\in\vertSet}\sum_{j\conLeft k} \ip{\tgrad_{\lambda_{jk}}\psi_m}{\tgrad_{\lambda_{jk}}\psi_n}_{L^2\bracs{I_{jk}}}, \\
%	L_{n,m}\bracs{\omega} &= \sum_{v_j\in\vertSet}\sum_{j\conLeft k} \ip{\dtn^+_{\omega}\psi_m + \dtn^-_{\omega}\psi_m}{\psi_n}_{L^2\bracs{I_{jk}}}, \\
%	B_{n,m} &= \sum_{v_j\in\vertSet}\sum_{j\conLeft k} \ip{\psi_m}{\psi_n}_{L^2\bracs{I_{jk}}}.
%\end{align*}
%Letting $M\bracs{\omega^2}$ be the matrix-valued function with entries
%\begin{align} \label{eq:SI-NonLocalQGNumericalDisc}
%	\bracs{M\bracs{\omega^2}}_{n,m} &= A_{n,m} + L_{n,m}\bracs{\omega} - \omega^2 B_{n,m},
%\end{align}
%our coefficients $u_m$ and approximate eigenvalues $\omega^2$ are then determined by the solution to the generalised eigenvalue problem
%\begin{align*}
%	M\bracs{\omega^2}U &= 0,
%	\qquad
%	U = \bracs{u_1, u_2, ... , u_M}^{\top}.
%\end{align*}
%Insofar we have done nothing different to a standard spectral or finite element method, with the only non-standard terms appearing in our matrix $M\bracs{\omega^2}$ being the $L_{n,m}$ terms coming from the Dirichlet to Neumann map.
%These depend in a non-trivial way on the spectral parameter $\omega^2$, and form the major obstacle to the implementation of any numerical scheme.
%In order to evaluate $L_{n,m}$, we need to know the action of $\dtn_{\omega}^i$ (for every bulk region $\ddom_i$) on \emph{every} basis function $\psi_m$, or at the least be able to approximate this.
%Asking for an analytic expression is a tall order, especially given the wide variety of shapes that the bulk regions $\ddom_i$ can take --- in general, $\ddom_i$ could be any bounded, polygonal domain.
%On the numerical approximation side, our obvious option for computing $\dtn^i_\omega\psi_m$ is solving the equation \eqref{eq:SI-BulkEqn} subject to the boundary conditions $u=\psi_m$, and reading off the Neumann data.
%However this forces us back to numerically solving each of the problems
%\begin{align*}
%	-\laplacian_{\qm}v &= \omega^2 v &\text{on } \ddom_i, \\
%	v\vert_{\partial\ddom_i} &= \psi_m,
%\end{align*}
%for every $m=1,...,M$ and bulk region $\ddom_i$, and then reading off the Neumann data of the computed solution.
%Furthermore, we will have to perform these computations every time we want to evaluate $M(\omega^2)$ whilst solving the generalised eigenvalue problem.
%Ultimately, short of obtaining an analytic expression (or an expression that can be easily approximated numerically) for the action of $\dtn^i_{\omega}$ in terms of $\omega$, there is no simple way to deal with
%the inherent influence of \eqref{eq:SI-BulkEqn} on \eqref{eq:SI-NonLocalQGEdgeEquation}.
%Possible ideas for alternatives include expressing $L_{n,m}$ in terms of the eigenvalues and eigenfunctions of the $\dtn^i_\omega$, or exploring whether the Dirichlet-to-Neumann map admits an ``expansion" --- these are briefly elaborated on in section \ref{sec:SIApp-NonLocalSolve}.
%Ultimately however, a deeper understanding of the properties of $\dtn_\omega^i$ are required to make any further progress without falling back to solving a two-dimensional problem in the bulk regions.