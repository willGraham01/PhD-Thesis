\section{Variational Approach to \eqref{eq:SI-WaveEqn}} \label{sec:SI-VarProbMethod}
Now that \eqref{eq:SI-WaveEqn} is well-defined through \eqref{eq:SI-WeakWaveEqn} and the operator $\mathcal{A}_{\qm}$, we can begin to consider our approach to solving it.
As mentioned in section \ref{sec:SI-ProblemFormulation}, the fact that $\mathcal{A}_{\qm}$ is self-adjoint for each $\qm$ combined with the min-max principle implies that the eigenvalues $\omega_{n}^2$ (and eigenfunctions $u_n$) of $\mathcal{A}_{\qm}$ are the minimum values (respectively minimisers) of the following variational problem:
\begin{align} \label{eq:SI-VarProb}
	\omega_n^2 &:= \min_{u}\clbracs{ \integral{\ddom}{ \abs{\tgrad_{\compMes}u}^2 }{\compMes} \setVert \norm{u}_{\ltwo{\ddom}{\compMes}}=1, \ u\perp u^{(l)} \ \forall 1\leq l\leq n-1 },
\end{align} 
where the eigenvalues are numbered in ascending order, and the orthogonality condition is meant with respect to the inner product in $\ltwo{\ddom}{\compMes}$.
Our interest is in determining the eigenvalues $\omega_n^2$, however we also need to determine the eigenfunctions $u_n$ since we need to use $u_{n-1}$ through $u_1$ in the computation of $u_{n}$.
Since we can obtain the eigenvalue $\omega_n^2$ from the eigenfunction $u_n$, by evaluating the integral in \eqref{eq:SI-VarProb}, we will focus our discussion on the approximation (and computation) of the eigenfunctions.
We also drop the explicit subscript $n$, and just consider the problem of determining the function $u\in\tgradSob{\ddom}{\compMes}$ which solves the optimisation problem;
\begin{subequations} \label{eq:SI-MinProblem}
	\begin{align}
		\text{Minimise} \quad & \quad \integral{\ddom}{ \abs{\tgrad_{\compMes}u}^2 }{\compMes} \\
		\text{Subject to} \quad & \quad \integral{\ddom}{ \abs{u}^2 }{\compMes} = 1, \\
		& \quad \integral{\ddom}{ u\cdot\overline{u}^{(l)} }{\compMes} = 0, \ 1\leq l\leq n-1,
	\end{align}
\end{subequations}
where $n\in\naturals$, $u^{(l)}, 1\leq l\leq n-1$ are given (pairwise) orthogonal functions.

The traditional idea when attempting to approximate a minimising function is to represent the minimising function $u$ in terms of a basis $\clbracs{\varphi_m}_{m\in\naturals_0}\subset\tgradSob{\ddom}{\compMes}$, truncate the basis expansion at some index $M$,
\begin{align} \label{eq:SI-VPTruncatedBasis}
	u &\approx \sum_{m=0}^M u_m \varphi_m,
\end{align}
and solve the minimisation problem (that arises from substituting \eqref{eq:SI-VPTruncatedBasis} into \eqref{eq:SI-MinProblem}) in the coefficients of the basis expansion that remain.
The choice of basis and $M$ will affect the accuracy of the approximate eigenfunction (and hence eigenvalue).
This minimisation problem will be discrete (solving for the $M+1$ independent variables $u_m$), and can be handled using optimisation methods.
In principle, we can apply these ideas for the problem \eqref{eq:SI-MinProblem}.
Our first task is to decide on the basis functions $\varphi_m$ that we want to use to approximate $u$.
From the standpoint of accuracy (and typically speed) of the numerical solution there are several properties that it is desirable for this basis to have; orthonormality between the $\varphi_m$, similar shape to that expected of $u$, and of course periodicity.
This is where problems concerning the unfamiliar nature of our space $\tgradSob{\ddom}{\compMes}$ begin to arise, as the choice of basis is considerably more complex --- as choosing the behaviour of $\varphi_m$ in the bulk regions then restricts what $\varphi_m$ can do on the skeleton, and vice-versa. 
The geometry of the skeleton can also compound this issue, particularly if there are a large number of bulk regions $\ddom_i$, if they have irregular shapes, or if their shapes are significantly different (in terms of size or shape) from each other.
In such a setting one could consider choosing a basis in similar fashion to how this is done for finite element schemes; mesh $\ddom$ into a union of simplexes (usually triangles) by placing nodes $\tau_i$, and  ensure that none of the simplexes straddle any parts of the skeleton (that is, the interior of a simplex always has empty intersection with the skeleton).
Then, use ``tent" or ``hat" functions centred on each node $\tau_i$ for the truncated basis functions $\varphi_m$.
This allows sufficient flexibility in the behaviour of $u$ on the edges and in the bulk regions, at the expense of requiring a new mesh for every new graph geometry.

Fortunately, the geometry of our cross-in-the-plane example is rather simple, since the two edges $I_h, I_v$ are aligned with the coordinate axes.
This makes computing integrals on the skeleton much simpler, and traces from the bulk region can be computed with relative ease, so we can avoid taking the approach of meshing $\ddom$ as described above.
Instead, we can opt to choose a basis in a way more akin to spectral methods --- by choosing ``global" basis functions rather than the ``local" tent-basis functions that meshing $\ddom$ would utilise.
Combined with the fact that we only have one bulk region that spans the entire period cell, the natural candidate for our basis functions would be the 2D Fourier basis $\e^{2\pi\rmi(\alpha x + \beta y)}$.
These functions are orthogonal in $\ltwo{\ddom}{\compMes}$, have period cell $\ddom$, and on each of the edges of $\graph$ reduce to a 1D-Fourier series.
However they also possess a continuous (in the sense of matching traces) normal derivative across the skeleton, which functions in $\tgradSob{\ddom}{\compMes}$ are not required to have, which means we cannot reliably use such a basis for our approximation.
Instead, we will look to use 2D polynomials to approximate our function $u$, by taking $M\in\naturals$ and setting
\begin{align} \label{eq:2DPolyBasisDef}
	\varphi_m(x,y) &= x^{i_m} y^{j_m}, \quad m = j_m + Mi_m, \ i,j\in\clbracs{0,...,M-1}.
\end{align}
These functions are not periodic by definition, so we are required to add the additional constraints
\begin{align*}
	u\bracs{0,y} = u\bracs{1,y}, \ \forall y\in\sqbracs{0,1}, 
	\qquad 
	u\bracs{x,0} = u\bracs{x,1}, \ \forall x\in\sqbracs{0,1},
\end{align*}
to our minimisation problem to account for this.
This is a suitable basis for $\tgradSob{\ddom}{\compMes}$; two dimensional (periodic) polynomials are dense in $\tgradSob{\ddom}{\lambda_2}$, and since we have that functions $u\in\tgradSob{\ddom}{\compMes}$ are continuous across the skeleton, the trace of such polynomials also approximates $u$ on the skeleton.
However, such polynomials do not possess matching normal derivative traces across the skeleton.
In the case where one has multiple bulk regions, one can build on this idea by approximating with ``piecewise polynomials" --- approximate via a polynomial basis on each bulk region $\ddom_i$, then impose that the approximations must coincide on any part of the skeleton common to the boundary of two bulk regions.

With our choice of polynomial basis, and writing $U = \bracs{u_0,...,u_{M^2-1}}^\top$, we are tasked with solving the following problem for our cross in the plane geometry.
\begin{problem}[Discrete Variational Problem] \label{prob:DiscVarProb}
	Let $M,N\in\naturals$ and $\varphi_m$ be as in \eqref{eq:2DPolyBasisDef}.
	Given coefficients $U_l=\bracs{u^l_0,...,u^l_{M^2-1}}^\top$ for $1\leq l\neq N-1$, find values $U=\bracs{u_0,...,u_{M^2-1}}^\top$ that:
	\begin{subequations} \label{eq:SI-ExampleMinProb}
		\begin{align}
			\text{Minimise} \quad & \quad J\sqbracs{U} := \sum_{m=0}^{M^2-1}\sum_{n=0}^{M^2-1}u_m\overline{u}_n\ip{\tgrad_{\compMes}\varphi_m}{\tgrad_{\compMes}\varphi_n}_{\ltwo{\ddom}{\compMes}^2} 
			\label{eq:SI-EMPObjectiveFn} \\
			\text{Subject to} \quad & \quad \sum_{m=0}^{M^2-1}\sum_{n=0}^{M^2-1}u_m\overline{u}_n\ip{\varphi_m}{\varphi_n}_{\ltwo{\ddom}{\compMes}} = 1, 
			\label{eq:SI-EMPNormConstraint} \\
			& \quad \sum_{i_m=1}^{M-1}u_{j_m+Mi_m} = 0, \ \forall j_m\in\clbracs{0,...,M-1}, 
			\label{eq:SI-EMPxPeriodicity} \\
			& \quad \sum_{j_m=1}^{M-1}u_{j_m+Mi_m} = 0, \ \forall i_m\in\clbracs{0,...,M-1},
			\label{eq:SI-EMPyPeriodicity} \\
			& \quad \sum_{m=0}^{M^2-1}\sum_{n=0}^{M^2-1}u_m\overline{u}^l_n\ip{\varphi_m}{\varphi_n}_{\ltwo{\ddom}{\compMes}} = 0, \ \forall 1\leq l\leq N-1.
			\label{eq:SI-EMPOrthogonality}
		\end{align}
	\end{subequations}
\end{problem}
The minimiser $U$ of problem \ref{prob:DiscVarProb} then provides our approximation of $u$, and we have that $\omega^2 \approx J[U]$.
Equation \eqref{eq:SI-EMPNormConstraint} is the norm constraint on the eigenfunction $u$, \eqref{eq:SI-EMPxPeriodicity} (respectively \eqref{eq:SI-EMPyPeriodicity}) are the constraints that ensure periodicity of the eigenfunction in the $x$ (respectively $y$) directions, and \eqref{eq:SI-EMPOrthogonality} forces $u$ to be orthogonal to the previously computed eigenfunctions $u^l$.
Due to the finite dimension of our problem (through the truncation at order $M$ for the functions $\varphi_m$), we are only ever able to compute approximations to the lowest $M^2 - \bracs{2M + 1} + 1$ eigenvalues due to the number of constraints in the problem \ref{prob:DiscVarProb}.

\tstk{time to display some nice figures, maybe some comparisons between the two methods etc? Could also do a run of this with $\alpha_3\neq0$ just to see what on earth happens!}