\section{The Space $\tgradSob{\ddom}{\compMes}$} \label{sec:CompSobSpaces}
Our analysis of the relevant Sobolev spaces in this section follows a similar approach to that we performed in chapter \ref{ch:ScalarSystem}.
However, we now have to worry about the behaviour of our approximating sequences on the bulk regions --- previously we did not have to pay particular care to what they were doing in these regions, since $\lambda_2$ was not present in the problem.

Throughout this section, let $\xi:\reals\rightarrow\reals$ be a smooth function such that
\begin{align*}
	\supp\bracs{\xi} = \sqbracs{-1,1}, \quad -1\leq\xi(y)\leq1 \ \forall y\in\sqbracs{-1,1}, \quad \xi(0)=0, \quad \xi'(0)=1.
\end{align*}
Note that we can choose $\xi$ such that there exists a $c>0$ s.t. $\abs{\xi'(y)}\leq c$ for all $y\in\sqbracs{-1,1}$ (and given the support of $\xi$, all $y\in\reals$).
For some of the estimates we want to make below, we also need the following notation.
\begin{definition}[Width in direction]
	Let $\ddom\subset\reals^2$ and $\clbracs{n, e}$ be a pair of orthonormal vectors in $\reals^2$.
	For each $\alpha\in\reals$, let $L_{\alpha} = \clbracs{x\in\ddom \setVert x = \alpha n + \beta e, \beta\in\reals}$ be the segment parallel to $e$ a ``signed distance" $\alpha$ from the diagonal $e$.
	Let $\lambda_{\alpha}$ be the singular measure that supports $L_{\alpha}$.
	Then the \emph{width of $\ddom$ in the direction $e$}, denoted $\ddom_e$, is defined to be $\ddom_e := \sup_{\alpha}\clbracs{\lambda_{\alpha}(L_{\alpha})}$.
\end{definition}
Note that $\ddom$ is always assumed bounded so the supremum exists, and if $\ddom$ is closed then is is a maximum and is attained for some $\alpha'$.
Furthermore, for $\alpha\in\reals$ let $\mathbf{I}^{\alpha} = \cup_{\abs{\alpha'}\leq\abs{\alpha}}L_{\alpha'}$.

\begin{lemma} \label{lem:SI-SmoothFunctionsResults}
	Let $I_{jk}\in\edgeSet$, and for $n\in\naturals$ define the function 
		\begin{align*}
			\xi_n:\ddom\rightarrow\reals, \qquad \xi_n(x) = \recip{n}\xi\bracs{n\bracs{x-v_j}\cdot n_{jk}}
		\end{align*}
		Then $\xi_n\in\psmooth{\ddom}\cap\csmooth{\ddom}$ for any\footnote{It may be required that we discard all  $n$ less than or equal to some $N\in\naturals$ for this to hold, depending on the geometry of the edge $I_{jk}$. Regardless, the result concerning the limit holds, and we can just use the subsequence $\bracs{\xi_n}_{n>N}$.} $n\in\naturals$.
	\begin{enumerate}[(i)]
		\item  We have that
		\begin{align*}
			\xi_n \lconv{\ltwo{\ddom}{\lcompMes}} 0, \qquad
			\grad\xi_n \lconv{\ltwo{\ddom}{\lcompMes}^2} n_{jk}\charFunc{jk},
		\end{align*}
		$\toInfty{n}$.
		\item For every $v_j\in\vertSet$, the functions $\eta_n^j$ defined in \eqref{eq:SmoothEtaDef} are such that
		\begin{align*}
			\eta_n^j \lconv{\ltwo{\ddom}{\compMes}} 1,
			&\qquad
			\eta_n^j \lconv{\ltwo{\ddom}{\ccompMes}} \charFunc{\ddom\setminus\clbracs{v_j}} := \begin{cases} 1 & x\neq v_j, \\ 0 & x=v_j. \end{cases}
		\end{align*}
	\end{enumerate}
\end{lemma}
\begin{proof}
	\begin{enumerate}[(i)]
		\item The $\xi_n$ are clearly smooth by construction, so we move onto establishing the convergences.
		Note that for any $x\in I_{jk}$, we have that $x = v_j + \beta e_{jk}$ for some $\beta\geq0$, and thus $\bracs{x-v_j}\cdot n_{jk} = 0$.
		Therefore, $\xi_n(x) = \recip{n}\xi(0) = 0$ whenever $x\in I_{jk}$.
		Additionally for $x\in I_{jk}$, we have that$\grad\xi_n(x) = n_{jk}\xi'\bracs{n\bracs{x-v_j}\cdot n_{jk}}$ so $\grad\xi_n(x) = n_{jk}$.
		Now,
		\begin{align*}
			\integral{\ddom}{ \abs{\xi_n(x)}^2 }{\lcompMes}
			&= \integral{\ddom}{ \recip{n^2}\abs{\xi\bracs{n\bracs{x-v_j}\cdot n_{jk}}}^2 }{\lambda_2}
			+ \integral{\ddom}{ \abs{\xi_n(0)}^2 }{\lambda_{jk}} \\
			&= \recip{n^2}\integral{\mathbf{I}^{1/n}}{ \abs{\xi\bracs{n\bracs{x-v_j}\cdot n_{jk}}}^2 }{\lambda_2}
			\leq \recip{n^2}\integral{\mathbf{I}^{1/n}}{ }{\lambda_2} \\
			&= \frac{2\ddom_{e_{jk}}}{n^3} \rightarrow 0,
		\end{align*}
		so we have that the $\xi_n$ converge to 0 in $\ltwo{\ddom}{\lcompMes}$.
		Since $n_{jk}\charFunc{jk}=0$ $\lambda_2$-almost-everywhere, we also have the estimate
		\begin{align*}
			\integral{\ddom}{ \abs{ \grad\xi_n(x) - n_{jk}\charFunc{jk} }^2 }{\lambda_2}
			&= \integral{\ddom}{ \abs{ \grad\xi_n(x) }^2 }{\lambda_2}
			= \integral{\ddom}{ \abs{n_{jk}}^2\abs{\xi'\bracs{n\bracs{x-v_j}\cdot n_{jk}}}^2 }{\lambda_2} \\
			&\leq \integral{\mathbf{I}^{1/n}}{ c^2 }{\lambda_2} 
			= \frac{2c^2\ddom_{e_{jk}}}{n},
		\end{align*}
		since $\abs{\xi'(y)}\leq c$ for all $y\in\reals$.
		We also have that
		\begin{align*}
			\integral{\ddom}{ \abs{ \grad\xi_n(x) - n_{jk}\charFunc{jk} }^2 }{\lambda_{jk}}
			&= \integral{I_{jk}}{ \abs{n_{jk}}^2\abs{ \xi'(0)-1 }^2 }{\lambda_{jk}} = 0,
		\end{align*}
		and so
		\begin{align*}
			\integral{\ddom}{ \abs{ \grad\xi_n(x) - n_{jk}\charFunc{jk} }^2 }{\lcompMes}
			&\leq \frac{2c^2\ddom_{e_{jk}}}{n} \rightarrow 0 \toInfty{n},
		\end{align*}
		and we are done.
		\item For these results we can observe that
		\begin{align*}
			\integral{\ddom}{\abs{ \eta_n^j - 1 }^2}{\lambda_2}
			&\leq \integral{B_{1/n}(v_j)}{}{\lambda_2} = \frac{\pi}{n^2}, \\
			\integral{\ddom}{\abs{ \eta_n^j - 1 }^2}{\lambda_{jk}}
			&\leq \integral{I_{jk} \cap B_{1/n}(v_j)}{}{\lambda_{jk}} = \frac{1}{n}, \\
			\integral{\ddom}{\abs{ \eta_n^j - 1 }^2}{\ddmes}
			&= \sum_{j\con k} \integral{\ddom}{\abs{ \eta_n^j - 1 }^2}{\lambda_{jk}} \leq \frac{\abs{\edgeSet}}{n},
		\end{align*}
		which demonstrate that $\eta_n^j \lconv{\ltwo{\ddom}{\compMes}} 1$.
		For the other convergence, observe that for $n \geq 2d$ where $d$ is as in \eqref{eq:HalfMinDistBetweenVertsDef}, we have that $\eta_n^j(v_k) = 1$ when $k\neq j$.
		Thus for $n \geq 2d$ we have
		\begin{align*}
			\integral{\ddom}{\abs{ \eta_n^j - \charFunc{\ddom\setminus\clbracs{v_j}} }^2}{\nu}
			&= \sum_{\substack{v_k\in\vertSet, \\ k\neq j}}\alpha_k\abs{\eta_n^j(v_k) - 1}^2
			+ \alpha_j\abs{ \eta_n^j(v_j) - 0 }^2
			= 0,
		\end{align*}
		and hence
		\begin{align*}
			\integral{\ddom}{\abs{ \eta_n^j - \charFunc{\ddom\setminus\clbracs{v_j}} }^2}{\ccompMes}
			&= \integral{\ddom}{\abs{ \eta_n^j - 1 }^2}{\lambda_2}
			+ \integral{\ddom}{\abs{ \eta_n^j - 1 }^2}{\ddmes}
			+ \integral{\ddom}{\abs{ \eta_n^j - \charFunc{\ddom\setminus\clbracs{v_j}} }^2}{\nu} \\
			&\leq \frac{\pi}{n^2} + \frac{1}{n} + 0 \qquad\text{when } n\geq 2d, \\
			&\rightarrow 0 \toInfty{n},
		\end{align*}
		providing the other convergence result.
	\end{enumerate}
\end{proof}

Lemma \ref{lem:SI-SmoothFunctionsResults} demonstrates to us that, similarly to the case in chapter \ref{ch:ScalarSystem}, the measure $\compMes$ cannot see changes ``across" the edges $I_{jk}$.
\begin{lemma} \label{lem:SI-SmoothGradZero}
	Suppose that $\phi\in\psmooth{\ddom}$, and let $I_{jk}\in\edgeSet$.
	Then the function
	\begin{align*}
		\tilde{g} &= \begin{cases} 0 & x\in\ddom\setminus I_{jk}, \\ \phi n_{jk} & x\in I_{jk}, \end{cases}
		\in\gradZero{\ddom}{\lcompMes}.
	\end{align*}
\end{lemma}
\begin{proof}
	Take $\xi_n$ to be the sequence in lemma \ref{lem:SI-SmoothFunctionsResults}, and set $\phi_n(x) = \phi(x)\xi_n(x)$ for each $n\in\naturals$.
	Note that
	\begin{align*}
		\grad\phi_n(x) &= \xi_n(x)\grad\phi(x) + \phi(x)\xi'\bracs{n\bracs{x-v_j}\cdot n_{jk}}n_{jk},
	\end{align*}
	so $\phi_n(x) = 0$ and $\grad\phi_n(x) = \phi(x)n_{jk}$ when $x\in I_{jk}$, due to the properties of $\xi_n$.
	Then $\phi_n\rightarrow0$ and $\grad\phi_n\rightarrow\tilde{g}$ by lemma \ref{lem:SI-SmoothFunctionsResults}(i).
\end{proof}

Our next result demonstrates that we gradients of zero with respect to $\lambda_{jk}$ can be elevated to be gradients of zero with respect to $\lcompMes$.
\begin{lemma} \label{lem:SI-GradZeroEdgeToComposite}
	Let $I_{jk}\in\edgeSet$, and $g\in\gradZero{\ddom}{\lambda_{jk}}$.
	Then the function
	\begin{align*}
		\tilde{g} = \begin{cases} 0 & x\in\ddom\setminus I_{jk}, \\ g & x\in I_{jk}, \end{cases}
	\end{align*}
	is an element of $\gradZero{\ddom}{\lcompMes}$.
\end{lemma}
\begin{proof}
	By the characterisation proposition \ref{prop:3DGradZeroRotated} we can write $g = g_{jk}n_{jk}$ for some $g_{jk}\in\ltwo{\ddom}{\lambda_{jk}}$.
	Let $\phi_n$ be an approximating sequence for $g$, and notice that since $\grad\phi_n\rightarrow g$, we have that
	\begin{align*}
		\grad\phi_n\cdot n_{jk} \rightarrow g\cdot n_{jk} = g_{jk}, 
		&\implies
		\bracs{ \grad\phi_n\cdot n_{jk} }n_{jk} \rightarrow g_{jk}n_{jk} = g,
	\end{align*}
	in $\ltwo{\ddom}{\lambda_{jk}}^2$.
	Each of the functions $\grad\phi_n\cdot n_{jk}$ are smooth, and thus by lemma \ref{lem:SI-SmoothGradZero} we have that the functions
	\begin{align*}
		\tilde{g}_n &= \begin{cases} 0 & x\in\ddom\setminus I_{jk}, \\ \bracs{ \grad\phi_n\cdot n_{jk} }n_{jk} & x\in I_{jk}, \end{cases}
	\end{align*}
	are all elements of $\gradZero{\ddom}{\lcompMes}$.
	Furthermore, we can observe that
	\begin{align*}
		\integral{\ddom}{\abs{ \tilde{g}_n - \tilde{g} }^2}{\lcompMes}
		&= \integral{\ddom}{\abs{ \tilde{g}_n - \tilde{g} }^2}{\lambda_{jk}}
		= \integral{I_{jk}}{\abs{ \bracs{ \grad\phi_n\cdot n_{jk} }n_{jk} - g }^2}{\lambda_{jk}} \\
		&\rightarrow 0 \toInfty{n},
	\end{align*}
	and so $\tilde{g}_n\lconv{\ltwo{\ddom}{\lcompMes}^2}\tilde{g}$.
	Since $\gradZero{\ddom}{\lcompMes}$ is closed, and $\tilde{g}_n\in\gradZero{\ddom}{\lcompMes}$ for every $n\in\naturals$, we thus conclude that the limit $\tilde{g}\in\gradZero{\ddom}{\lcompMes}$, and we are done.
\end{proof}

Now that we can ``lift" (via lemma \ref{lem:SI-GradZeroEdgeToComposite}) gradients of zero on an edge up to a gradients of zero with respect to the composite measure $\lcompMes$, and that all gradients of zero with respect to $\lambda_2$ are zero, this effectively gives us our characterisation of $\gradZero{\ddom}{\lcompMes}$.
\begin{cory}[Characterisation of $\gradZero{\ddom}{\lcompMes}$.] \label{cory:SI-GradZeroEdgeChar}
	The following characterisations hold:
	\begin{enumerate}[(i)]
		\item Let $I_{jk}\in\edgeSet$, suppose that $g\in\ltwo{\ddom}{\lambda_{jk}}$ and $\tilde{g}\in\ltwo{\ddom}{\lcompMes}$ where
		\begin{align*}
			\tilde{g} = \begin{cases} 0 & x\in\ddom\setminus I_{jk}, \\ g & x\in I_{jk}. \end{cases}
		\end{align*}
		Then
		\begin{align*}
			\tilde{g}\in\gradZero{\ddom}{\lcompMes} \quad\Leftrightarrow\quad 
			& g\in\gradZero{\ddom}{\lambda_{jk}}.
		\end{align*}
		\item
		\begin{align*}
			\gradZero{\ddom}{\lcompMes} 
			&= \clbracs{ \tilde{g}\in\ltwo{\ddom}{\lcompMes} \setVert \exists g\in\gradZero{\ddom}{\lambda_{jk}} \text{ s.t. } \tilde{g} = \begin{cases} 0 & x\in\ddom\setminus I_{jk}, \\ g & x\in I_{jk}. \end{cases}}
		\end{align*}
	\end{enumerate}
\end{cory}
\begin{proof}
	\begin{enumerate}[(i)]
		\item ($\Rightarrow$) The right-directed implication holds since any approximating sequence for $\tilde{g}$ in $\ltwo{\ddom}{\lcompMes}$ also serves as an appropriate approximating sequence for $g$ in $\ltwo{\ddom}{\lambda_{jk}}$. \newline
		($\Leftarrow$) The left-directed implication is simply the result of lemma \ref{lem:SI-GradZeroEdgeToComposite}.
		\item The ``$\supset$" inclusion holds by part (i) above.
		For the ``$\subset$" inclusion, take $\tilde{g}\in\gradZero{\ddom}{\lcompMes}$.
		Then $\tilde{g}\in\gradZero{\ddom}{\lambda_2}$ and $\tilde{g}\in\gradZero{\ddom}{\lambda_{jk}}$ since any approximating sequence for $\tilde{g}$ in $\ltwo{\ddom}{\lcompMes}$ also serves as the approximating sequence in the other two spaces.
		However, $\tilde{g}\in\gradZero{\ddom}{\lambda_2}$ implies that $\tilde{g}=0$ $\lambda_2$-almost-everywhere,	and thus $\tilde{g}$ has the form
		\begin{align*}
			\tilde{g} = \begin{cases} 0 & x\in\ddom\setminus I_{jk}, \\ g & x\in I_{jk}, \end{cases}
		\end{align*}
		where $g = \tilde{g}$ $\lambda_{jk}$-almost-everywhere.
		Since $g = \tilde{g}\in\gradZero{\ddom}{\lambda_{jk}}$, have shown the desired inclusion.
	\end{enumerate}
\end{proof}

Now we look to demonstrate that the set $\gradZero{\ddom}{\compMes}$ is constructed from sums of edge functions belonging to $\gradZero{\ddom}{\lcompMes}$.
\begin{prop}[Relation between $\gradZero{\ddom}{\lcompMes}$ and $\gradZero{\ddom}{\compMes}$.] \label{prop:SI-GradZeroExtensionAndChar}
	Fix $I_{jk}\in\edgeSet$.
	\begin{enumerate}[(i)]
		\item Suppose that $g\in\gradZero{\ddom}{\lcompMes}$ with
		\begin{align*}
			g = \begin{cases} 0 & x\in\ddom\setminus I_{jk}, \\ g_{jk} & x\in I_{jk}, \end{cases}
		\end{align*}
		and $\supp(g_{jk})\subset I_{jk}^n$ (where $I_{jk}^n$ is as in \eqref{eq:ShortenedEdgeDef}).
		Then the function
		\begin{align*}
			\tilde{g}(x) &:= \begin{cases} 0 & x\in\ddom\setminus\graph, \\ 0 & x\in\graph\setminus I_{jk}, \\ g_{jk}(x) & x\in I_{jk}, \end{cases}
		\end{align*}
		is an element of $\gradZero{\ddom}{\compMes}$.
		\item Now only assume that $g\in\gradZero{\ddom}{\lcompMes}$ with
		\begin{align*}
			g = \begin{cases} 0 & x\in\ddom\setminus I_{jk}, \\ g_{jk} & x\in I_{jk}. \end{cases}
		\end{align*}
		Then the function
		\begin{align*}
			\tilde{g}(x) &:= \begin{cases} 0 & x\in\ddom\setminus\graph, \\ 0 & x\in\graph\setminus I_{jk}, \\ g_{jk}(x) & x\in I_{jk}, \end{cases}
		\end{align*}
		is an element of $\gradZero{\ddom}{\compMes}$.
		\item We have that
		\begin{align*}
			\gradZero{\ddom}{\compMes} &= \clbracs{ \tilde{g}\in\ltwo{\ddom}{\compMes} \setVert \tilde{g}^{(jk)}\in\gradZero{\ddom}{\lcompMes} \ \forall I_{jk}\in\edgeSet }.
		\end{align*}
		\item Let $\tilde{g}\in\ltwo{\ddom}{\compMes}^2$, where
	\begin{align*}
		\tilde{g} &= \begin{cases} g_{\lambda_2} & x\in\ddom\setminus\graph, \\ g_{\ddmes} & x\in\graph. \end{cases}
	\end{align*}
	Then
	\begin{align*}
		\tilde{g}\in\gradZero{\ddom}{\compMes}
		\quad\Leftrightarrow\quad &
		g_{\lambda_2} = 0 \text{ and } g_{\ddmes}\in\gradZero{\ddom}{\ddmes}.
	\end{align*}
	\end{enumerate}
\end{prop}
\begin{proof}
	\begin{enumerate}[(i)]
		\item Take an approximating sequence $\phi_l$ for $g$, and set $\psi_l = \chi_{jk}^n\phi_l\in\psmooth{\ddom}$ for each $l\in\naturals$, where $\chi_{jk}^n$ is as in \eqref{eq:SmoothChiDef}.
		We then have that $\grad\psi_l = \chi_{jk}^n\grad\phi_l + \phi_l\grad\chi_{jk}^n$, and immediately see that $\psi_l\lconv{\ltwo{\ddom}{\compMes}}0$ since
		\begin{align*}
			\integral{\ddom}{\abs{ \psi_l }^2}{\compMes}
			&\leq \integral{\ddom}{\abs{ \phi_l }^2}{\compMes}.
		\end{align*}
		Additionally, we have the estimates
		\begin{align*}
			\recip{2}\integral{\ddom}{\abs{ \grad\psi_l - \tilde{g} }^2}{\ddmes}
			&= \recip{2}\integral{I_{jk}}{\abs{ \chi_{jk}^n\grad\phi_l - g  + \phi_l\grad\chi_{jk}^n }^2}{\lambda_{jk}} \\
			&\leq \integral{I_{jk}}{\abs{ \chi_{jk}^n\grad\phi_l - g }^2}{\lambda_{jk}}
			+ \integral{I_{jk}}{\abs{ \phi_l\grad\chi_{jk}^n }^2}{\lambda_{jk}} \\
			&\leq \integral{I_{jk}}{\abs{ \grad\phi_l - g }^2}{\lambda_{jk}}
			+ \sup_{\ddom}\abs{\grad\chi_{jk}^n}^2 \integral{I_{jk}}{\abs{ \phi_l }^2}{\lambda_{jk}}, \\
			\recip{2}\integral{\ddom}{\abs{ \grad\psi_l - \tilde{g} }^2}{\lambda_2}
			&\leq \integral{\ddom}{\abs{ \chi_{jk}^n\grad\phi_l - \tilde{g} }^2}{\lambda_2}
			+ \sup_{\ddom}\abs{\grad\chi_{jk}^n}^2 \integral{\ddom}{\abs{ \phi_l }^2}{\lambda_2} \\
			&\leq \integral{\ddom}{\abs{ \grad\phi_l }^2}{\lambda_2}
			+ \sup_{\ddom}\abs{\grad\chi_{jk}^n}^2 \integral{\ddom}{\abs{ \phi_l }^2}{\lambda_2}.
		\end{align*}
		Therefore,
		\begin{align*}
			\recip{2}\integral{\ddom}{\abs{ \grad\psi_l - \tilde{g} }^2}{\compMes}
			&\leq \integral{\ddom}{\abs{ \grad\phi_l }^2}{\lambda_2}
			+ \integral{I_{jk}}{\abs{ \grad\phi_l - g }^2}{\lambda_{jk}} \\
			&\quad + \sup_{\ddom}\abs{\grad\chi_{jk}^n}^2 
			\bracs{ \integral{\ddom}{\abs{ \phi_l }^2}{\lambda_2} 	+ \integral{I_{jk}}{\abs{ \phi_l }^2}{\lambda_{jk}} } \\
			&= \integral{\ddom}{\abs{ \grad\phi_l - g }^2}{\lcompMes}
			+ \norm{\phi_l}_{\ltwo{\ddom}{\lcompMes}} \sup_{\ddom}\abs{\grad\chi_{jk}^n}^2 \\
			&\rightarrow 0 \toInfty{l},
		\end{align*}
		and we conclude that $\tilde{g}\in\gradZero{\ddom}{\compMes}$.
		\item The function $g_{jk}\in\gradZero{\ddom}{\lambda_{jk}}$ by corollary \ref{cory:SI-GradZeroEdgeChar}, and thus for any $n\in\naturals$ the function $\eta_n^j\eta_n^k g_{jk}\in\gradZero{\ddom}{\lambda_{jk}}$.
		By lemma \ref{lem:SI-GradZeroEdgeToComposite}, the functions
		\begin{align*}
			g_n := \begin{cases} 0 & x\in\ddom\setminus I_{jk}, \\ \eta_n^j\eta_n^k g_{jk} & x\in I_{jk} \end{cases}
		\end{align*}
		are elements of $\gradZero{\ddom}{\lcompMes}$ too.
		By part (i), we then have that
		\begin{align*}
			\tilde{g}_n := 	\begin{cases} 0 & x\in\ddom\setminus\graph, \\ 0 & x\in\graph\setminus I_{jk}, \\ \eta_n^j\eta_n^k g_{jk} & x\in I_{jk}, \end{cases}
		\end{align*}
		is an element of $\gradZero{\ddom}{\compMes}$, $\forall n\in\naturals$.
		As such, by closure of $\gradZero{\ddom}{\compMes}$, it suffices to show that the sequence $\tilde{g}_n$ converges in $\ltwo{\ddom}{\compMes}$ to $\tilde{g}$.
		Given the result of lemma \ref{lem:SI-SmoothFunctionsResults}(ii), we indeed have that $\tilde{g}_n\rightarrow\tilde{g}$ and thus $\tilde{g}\in\gradZero{\ddom}{\compMes}$.
		\item The ``$\subset$" inclusion is simply a result of any approximating sequence for $\tilde{g}$ also serving as an approximating sequence in $\ltwo{\ddom}{\lcompMes}$ for each edge $I_{jk}$. \newline
		The ``$\supset$" inclusion follows from part (ii) and linearity of the space $\gradZero{\ddom}{\compMes}$; each $\tilde{g}^{(jk)}\in\gradZero{\ddom}{\lcompMes}$ can be extended by zero (using part (ii)) to a function in $\gradZero{\ddom}{\compMes}$.
		Summing up these extended functions then yields the function $\tilde{g}$ again, which is an element of $\gradZero{\ddom}{\compMes}$ by linearity of the subspace.
		\item ($\Leftarrow$) The left-directed implication follows from proposition \ref{prop:3DGradZeroChar}, corollary \ref{cory:SI-GradZeroEdgeChar}(ii), and part (iii). \newline
	($\Rightarrow$) The right-directed implication is simply a result of any approximating sequence for $\tilde{g}$ also serving as an approximating sequence for $g_{\lambda_2}$ and $g_{\ddmes}$.
	This also implies that $g_{\lambda_2} = 0$ (being a gradient of zero with respect to the Lebesgue measure), which completes the proof.
	\end{enumerate}
\end{proof}

Having understood the set of gradients of zero, we are now able to deduce some important properties of the functions that live in $\tgradSob{\ddom}{\compMes}$.
The most direct result that follows from proposition \ref{prop:SI-GradZeroExtensionAndChar}(iv) is that the tangential gradient $\tgrad_{\compMes}$ is the usual weak derivative (with respect to the Lebesgue measure) in the bulk regions, and equal to the tangential derivative with respect to $\ddmes$ on $\graph$.
\begin{cory}
	Suppose that $u\in\tgradSob{\ddom}{\compMes}$
	Then 
	\begin{align*}
		\tgrad_{\compMes}u = \begin{cases} \grad u + \rmi\qm u & x\in\ddom\setminus\graph, \\ \tgrad_{\ddmes}u & x\in\graph, \end{cases}
	\end{align*}
	where $\grad u$ denotes the weak derivative of $u\in\gradSob{\ddom}{\lambda_2}$.
\end{cory}
\begin{proof}
	The requirement that $\tgrad_{\compMes}u$ be orthogonal to $\gradZero{\ddom}{\compMes}$ and theorem \ref{thm:SI-GradZeroChar} implies that $\tgrad_{\compMes}u = \tgrad_{\ddmes}u$ on $\graph$.
	For the bulk regions, we again notice that any approximating sequence $\phi_n$ for $u\in\tgradSob{\ddom}{\compMes}$ also serves as an approximating sequence in for $u\in\tgradSob{\ddom}{\lambda_2}$, and $\tgrad_{\lambda_2}u = \grad u + \rmi\qm u$.
	This implies that $\tgrad_{\compMes}u = \grad u + \rmi\qm u$ ($\lambda_2$-almost-everywhere) in $\ddom\setminus\graph$, completing the proof.
\end{proof}

Functions in $\tgrad{\ddom}{\compMes}$ also exhibit additional regularity when approaching the edges of $\graph$ from the bulk regions $\ddom_i$, analogously to how functions in $\ktgradSob{\ddom}{\ddmes}$ exhibit continuity on approach the the vertices.
This is not immediately obvious from the definition (by approximation of smooth functions) of $\tgradSob{\ddom}{\compMes}$; however now that we understand that our tangential gradients are essentially the familiar (weak) gradients with respect to $\lambda_2$ in the bulk and are equal to tangential gradients with respect to $\ddmes$ on $\graph$, we can establish this connection between the bulk and inclusions.
\begin{theorem} \label{thm:SI-SobFuncEdgeContinuity}
	Let $u\in\tgradSob{\ddom}{\compMes}$, and fix a bulk region $\ddom_i$.
	Take $I_{jk}\in\edgeSet$ such that $\partial\ddom_i\cap I_{jk}\neq\emptyset$.
	Denote the trace operator (that extends the classical trace of functions in $\smooth{\overline{\ddom}_i}\cap\gradSob{\ddom_i}{\lambda_2}$) on $\gradSob{\ddom_i}{\lambda_2}$ into $\ltwo{\partial\ddom_i}{S}$ by $\mathrm{Tr}_i$.
	Then $\mathrm{Tr}_i(u)$ exists and $\mathrm{Tr}_i(u) = u^{(jk)}$ on $I_{jk}$.
\end{theorem}
\begin{proof}
	Take an approximating sequence $\phi_n$ for $u\in\tgradSob{\ddom}{\compMes}$; clearly this sequence is also such that
	\begin{align*}
		\phi_n \lconv{\ltwo{\ddom_i}{\lambda_2}} u, &\qquad \grad\phi_n\lconv{\ddom_i}{\lambda_2}\grad u, \\
		\phi_n \lconv{\ltwo{\ddom}{\lambda_{jk}}} u, &\qquad \tgrad\phi_n\lconv{\ddom}{\lambda_{jk}}\ktgrad_{\lambda_{jk}} u.
	\end{align*}
	Furthermore, since $\phi_n\in\psmooth{\ddom}$ and $\overline{\ddom}_i\subset\ddom$ (since the boundary of $\ddom_i$ consists of a union of edges of $\graph\subset\ddom$), we have $\phi_n\in\smooth{\overline{\ddom}_i}$.
	The regions $\ddom_i$ are also Lipschitz so the trace operator
	\begin{align*}
		\mathrm{Tr}_i: \gradSob{\ddom_i}{\lambda_2} \rightarrow \ltwo{\partial\ddom_i}{S},
	\end{align*}
	is linear and continuous.
	Because $\phi_n\rightarrow u$ in $\gradSob{\ddom_i}{\lambda_2}$ and $\phi_n\in\smooth{\overline{\ddom}_i}\cap\gradSob{\ddom_i}{\lambda_2}$, we have that
	\begin{align*}
		u\vert_{\partial\ddom_i} := \mathrm{Tr}_i(u) = \lim_{n\rightarrow\infty}\phi_n\vert_{\partial\ddom_i},
	\end{align*}
	where the limit is taken in $\ltwo{\partial\ddom_i}{S}$.
	In particular, we have that $\phi_n\lconv{\ltwo{I_{jk}}{S}}u\vert_{\partial\ddom_i}$ since $I_{jk}\subset\partial\ddom_i$.
	However,
	\begin{align*}
		\norm{ \phi_n - u\vert_{\partial\ddom_i} }_{\ltwo{I_{jk}}{S}}
		&= \integral{I_{jk}}{\abs{ \phi_n - u\vert_{\partial\ddom_i} }^2}{S}
		= \int_0^{l_{jk}} \abs{ \phi_n - u\vert_{\partial\ddom_i} }^2 \ \md y \\
		&= \integral{\ddom}{\abs{ \phi_n - u\vert_{\partial\ddom_i} }^2}{\lambda_{jk}}
		= \norm{ \phi_n - u\vert_{\partial\ddom_i} }_{\ltwo{\ddom}{\lambda_{jk}}},
	\end{align*}
	and so $\phi_n\rightarrow u\vert_{\partial\ddom_i}$ in $\ltwo{\ddom}{\lambda_{jk}}$.
	But we know that $\phi_n\rightarrow u^{(jk)}$ in $\ltwo{\ddom}{\lambda_{jk}}$ too, so it must be the case that $u^{(jk)} = u\vert_{\partial\ddom_i}$, and we are done.
\end{proof}
Clearly theorem \ref{thm:SI-SobFuncEdgeContinuity} also implies that traces from any two bulk regions $\ddom_1$ and $\ddom_2$ onto a common inclusion (that is, a shared boundary) must coincide with each other, as they both coincide with the values of $u^{(jk)}$ on the shared edges.
Also worth noting is that $u$ (restricted to the graph $\graph$) is continuous at the vertices (see theorem \ref{thm:3DTangGradGraph}), and thus combined with the previous remark will be continuous in the vicinity of any graph edges too.
Much like how theorem \ref{thm:3DTangGradGraph} established continuity of a function at the vertices of a graph, theorem \ref{thm:SI-SobFuncEdgeContinuity} will play an important part in our reformulation of the problem \eqref{eq:SI-WeakWaveEqn}.