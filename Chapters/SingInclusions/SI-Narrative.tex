\section{The Narrative (placeholder section)}

To reiterate our overarching goal, we wish to study wave propagation on these composite domains, and so we will be studying the ``wave equation"
\begin{align} \label{eq:SI-WaveEqn}
	-\bracs{\tgrad_{\compMes}}^2 u = \omega^2 u \qquad\text{in } \ddom,
\end{align}
for some solution $u$.
At this point we should consider what our intuition is telling us about the behaviour we expect from any solutions $u$, and accordingly the appropriate function space to pose \eqref{eq:SI-WaveEqn} in.
A good starting point is to consider how we expect our solutions to behave if we could (n\"{i}avely) interpret \eqref{eq:SI-WaveEqn} in a strong sense.
Away from any singular inclusions, \eqref{eq:SI-WaveEqn} would look like the usual Helmholtz problem on a bounded domain in $\reals^2$, and so we expect our solution to possess sufficient regularity to be differentiated twice in the bulk.
We also know that solutions to the ``wave equation" on the inclusions possess two derivatives along the edges of $\graph$ (\tstk{chapter ref!}), and are tied together through the vertex conditions.
Finally, we must consider what should happen in the vicinity of the singular inclusions --- here we have (what we expect to be) a twice differentiable function in a bulk region $\ddom_i$ approaching its boundary, and so there should be ($L^2$) traces of $u$ and its normal derivative onto this boundary.
However, this boundary coincides with a selection of the singular inclusions, so the function $u$ on these inclusions should ``feel" the affect of these traces.
A partial converse is also expected; on the singular inclusions $u$ is twice differentiable along the inclusion, and given that $u$ \emph{also} has a trace onto the inclusions, we expect that these traces should match the function values.
That is, we might expect that \eqref{eq:SI-WaveEqn} can be reformulated into a system that consists of the following components:
\begin{enumerate}[(a)]
	\item A (Helmholtz-like) PDE in each of the bulk regions, the solution to which has boundary traces matching the solution to a quantum graph problem on the inclusions.
	\item A 2nd-order quantum graph problem on the singular inclusions, with the edge ODEs involving or being influenced by the traces from the bulk regions.
	\item Conditions at the vertices of the graph to tie the quantum graph problem, and hence the PDE problems, together.
\end{enumerate}

With our expectations laid out, we can move towards formulating \eqref{eq:SI-WaveEqn} accurately.
Consider the form $b$ defined on functions $(u,v)\in\tgradSob{\ddom}{\compMes}\times\tgradSob{\ddom}{\compMes}$ where\footnote{Of course, we can be less stringent with this definition and define $b$ on smooth functions, which are (by construction) dense in $\tgradSob{\ddom}{\compMes}$.}
\begin{align*}
	b(u,v) &= \integral{\ddom}{ \tgrad_{\compMes}u\cdot\overline{\tgrad_{\compMes}v} }{\compMes}
	= \ip{\tgrad_{\compMes}u}{\tgrad_{\compMes}v}_{\ltwo{\ddom}{\compMes}^2}.
\end{align*}
Clearly $b$ is symmetric and satisfies $b(u,u)\geq 0$ with equality if and only if $u=0$, and thus defines a self-adjoint operator $\mathcal{A} := -\bracs{\tgrad_{\compMes}}^2$ by
\begin{align*} 
	\dom\bracs{ \mathcal{A} } &= \clbracs{ u\in\tgradSob{\ddom}{\compMes} \setVert \exists f\in\ltwo{\ddom}{\compMes} \text{ s.t. } \right.
	\\
	& \qquad
	\left. \integral{\ddom}{ \tgrad_{\compMes}u\cdot\overline{\tgrad_{\compMes}v} }{\compMes} = \integral{\ddom}{ f\overline{v}}{\compMes}, \quad \forall v\in\tgradSob{\ddom}{\compMes} }, \labelthis\label{eq:CompLaplaceOpDom}
\end{align*}
with action
\begin{align*}
	\mathcal{A}u = -\bracs{\tgrad_{\compMes}}^2 u = f,
\end{align*}
where $u$ and $f$ are related as in \eqref{eq:CompLaplaceOpDom}.
Equation \eqref{eq:SI-WaveEqn} is then the eigenvalue equation for the operator $\mathcal{A}$, interpreted as the problem of finding $\omega^2>0$ and non-zero $u\in\tgradSob{\ddom}{\compMes}$ such that
\begin{align} \label{eq:SI-WeakWaveEqn}
	\integral{\ddom}{ \tgrad_{\compMes}u\cdot\overline{\tgrad_{\compMes}\phi} }{\compMes}
	&= \omega^2 \integral{\ddom}{ u\overline{\phi} }{\compMes}, \quad\forall\phi\in\smooth{\ddom}.
\end{align}
\tstk{worth remarking here how, again, the weak formulation of a familiar PDE is being used as the starting point for our analysis, and actually ends up being what we want to look at too!}

At a glance, solutions $u$ to \eqref{eq:SI-WeakWaveEqn} appear to be far less regular than what we expect from our intuitive arguments above, and \eqref{eq:SI-WeakWaveEqn} itself is not in the form (a)-(c).
However, much like in sections \ref{sec:ScalarDerivation} and \ref{sec:VectorDerivation} we can work from \eqref{eq:SI-WeakWaveEqn} and the definition of $\tgradSob{\ddom}{\compMes}$ to obtain a system as described by (a)-(c).
This requires us to first analyse the set $\tgradSob{\ddom}{\compMes}$, much in the same vein as did with $\ktgradSob{\ddom}{\dddmes}$ and $\ktcurlSob{\ddom}{\dddmes}$ before (\tstk{section ref}), and is the focus of section \ref{sec:CompSobSpaces}.
However, we will summarise the key results for a function $u\in\tgradSob{\ddom}{\compMes}$ here:
\begin{itemize}
	\item The tangential gradient $\tgrad_{\compMes}u$ of $u$ is such that
	\begin{align*}
		\tgrad_{\compMes}u = \begin{cases} \grad u + \rmi\qm u & x\in\ddom\setminus\graph, \\ \tgrad_{\ddmes}u & x\in\graph, \end{cases}
	\end{align*}
	where $\grad u$ denotes the weak derivative of $u\in\gradSob{\ddom}{\lambda_2}$.
	Essentially, in the bulk regions the function $u$ and its tangential gradient coincide with the familiar notion of a weak derivative (with respect to the Lebesgue measure).
	One should note the parallels that can be drawn with the intuitive reasoning for the parts (a) and (b) above.
	\item The function $u$ lives in $\gradSob{\ddom_i}{\lambda_2}$ for each of the bulk regions, and the traces of $u$ from $\ddom_i$ onto the inclusions $I_{jk}$ coincide with the values of $u_{jk}$ on the inclusions.
	This is as close to a condition of ``continuity across the inclusions" as we can get.
	Additionally, the $u_{jk}$ are is continuous at the vertices of $\graph$, as was the case for functions in $\ktgradSob{\ddom}{\dddmes}$.
\end{itemize}
This is all the information we need to begin reformulating \eqref{eq:SI-WeakWaveEqn} into the form (a)-(c), and begin to search for methods of solving the resulting problem.