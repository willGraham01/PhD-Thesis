\section{Introduction to the Domains and Problems we will Consider} \label{sec:TP-DomainSetup}
We can now discuss the domains and variational problems we will be working on throughout chapters \ref{ch:ScalarSystem}-\ref{ch:SingInc}, and establish the notation that we shall use going forward.
Each chapter will contain a short recap of the relevant notation, but unless otherwise stated uses that established here.

Let $\hat{\graph}$ be a periodic graph embedded into $\reals^2$, and let $\hat{\ddmes}$, $\hat{\massMes}$, and $\upsilon$ be (respectively) the measures defined in \eqref{eq:Intro-MeasureDefs} for the graph $\hat{\graph}$.
Suppose that $\hat{\graph}$ has unit cell $\ddom=[0,1)^2$ and unit graph $\graph=\bracs{\vertSet, \edgeSet}$, and adopt assumption \ref{ass:MeasTheoryProblemSetup} for $\graph$.
Define the measures $\lambda_{jk}$, $\ddmes$, $\massMes$, $\dddmes$, and $\ccompMes$ as they appear in section \ref{sec:SingularMeasures} for $\graph$.
We reiterate that we always associate $\ddom$ with the surface of the two dimensional torus; by identifying the ``left and right" boundaries $\clbracs{x_1=0}$ and $\clbracs{x_1=1}$ with each other, and the ``top and bottom" boundaries $\clbracs{x_2=0}$ and $\clbracs{x_2=1}$ similarly.
Furthermore, we will also take the coupling constants $\alpha_j$ to be such that $\Re\bracs{\alpha_j}>0$ and $\Im\bracs{\alpha_j} = 0$.
This decision ensures causality is adhered to in our problems --- the solutions are allowed to depend on the past, but not the future.
Taking $\Im\bracs{\alpha_j} > 0$ (respectively $\Im\bracs{\alpha_j} < 0$) would correspond to placing an energy sink (respectively source) at the vertex $v_j$.
Of course there is nothing that mathematically forbids the coupling constants $\alpha_j$ being complex-valued, nor does it drastically effect our analysis, but such restrictions are sensible in light of the physical applications. 

In chapter \ref{ch:ScalarSystem}, we will be only be concerned with the 2D domain $\ddom$.
We will define the family of operators $-\laplacian_{\dddmes}^{\qm}$ through the (bilinear) form 
\begin{align*}
	b(u,v) = \integral{\ddom}{ \tgrad_{\dddmes} u\cdot\overline{\tgrad_{\dddmes} v} }{\dddmes},
	\qquad u,v\in\tgradSob{\ddom}{\dddmes},
\end{align*}
which will provide us with the (spectral, variational) problem of finding eigenpairs $\omega^2>0$, $u\in\tgradSob{\ddom}{\dddmes}$ such that
\begin{align*}
	\integral{\ddom}{ \tgrad_{\dddmes} u\cdot\overline{\tgrad_{\dddmes}\phi} }{\dddmes}
	&= \omega^2\integral{\ddom}{ u\overline{\phi} }{\dddmes},
	\qquad\forall\phi\in\psmooth{\ddom}.
\end{align*}
As discussed in section \ref{ssec:SobSpacesAndGelfand}, the union of the spectra $\sigma\bracs{-\laplacian_{\dddmes}^{\qm}}$ over the quasi-momentum $\qm$ provides us with the spectrum of the periodic operator $-\laplacian_{\upsilon}$ on $\reals^2$, defined through the form
\begin{align} \label{eq:TP-2DWaveEqnWholeSpace}
	\integral{\reals^2}{ \grad_{\upsilon} u\cdot\overline{\grad_{\upsilon} v} }{\upsilon},
	\qquad u,v\in\gradSob{\reals^2}{\upsilon}.
\end{align}
Aside from replacement of the Lebesgue measure with the singular measure $\dddmes$ (and the corresponding replacement of gradients), \eqref{eq:TP-2DWaveEqnWholeSpace} is markedly similar to the weak formulation for the acoustic equation.
Our analysis of the operators $-\laplacian_{\dddmes}^{\qm}$ will lead us to a quantum graph problem on $\graph$, and analysis of the spectrum via the $M$-matrix.
We derive an explicit formula for the $M$-matrix in terms of the geometry of $\graph$, and discuss the numerical and analytic techniques that this makes available to us, complimenting this with examples.

We then look to extend the work in chapter \ref{ch:ScalarSystem} to encompass the (analogue of the) curl-of-the-curl equation, which will require us to consider the domain $\dddom = \ddom\times[0,\infty)$.
This geometry corresponds to a collection of planes parallel to the $x_3$-axis, and it is not difficult to show that the product measure $\dddmes\times\lambda_1$ corresponds to a singular measure that supports these induced planes (along with a contribution from the measure $\massMes$ at the extruded vertices).
Note that we will not be considering a graph embedded into three dimensions, since such a problem has rather trivial nature, as discussed in section \ref{sec:CC-Geometric}.
A Fourier transform will bring us back onto the domain $\ddom$, and we will analyse the space $\ktcurl{\ddom}{\dddmes}$ to derive a quantum graph problem for the curl-of-the-curl equation on singular structures.

Finally, in chapter \ref{ch:SingInc} we will return to the acoustic equation on the two-dimensional domain $\ddom$, however equip it with the composite measure $\ccompMes$ rather than the singular measure $\dddmes$.
In this case the graph $\graph$ naturally breaks the domain $\ddom$ into disjoint (open) sets $\ddom_i$, for $i$ in some index set, where:
\begin{itemize}
	\item For every $i$, the domain $\ddom_i$ is an $n$-gon, whose boundary $\partial\ddom_i$ is equal to the union of $n$ (distinct) edges of $\graph$.
	Note that this implies that $\ddom_i$ is a Lipschitz domain.
	\item We have that
	\begin{align*}
		\ddom = \graph \cup \bigcup_{i}\ddom_i,
		\qquad
		\emptyset = \graph \cap \bigcup_{i}\ddom_i.
	\end{align*}
\end{itemize}
In this chapter we will refer to the regions $\ddom_i$ as the \emph{bulk regions} and the graph $\graph$ (and its constituent edges) as the \emph{skeleton}, to sidestep a philosophical debate about whether $\graph$ or the $\ddom_i$ are the ``inclusions".