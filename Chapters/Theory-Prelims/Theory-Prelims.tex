\chapter{Preliminaries and Theory} \label{ch:TheoryPrelims}
Throughout this work there will be a number of common themes and conventions whose content is not specific to one particular chapter or research problem.
To avoid redirecting the reader to a number of different points in this work, and in the interest of preventing chapters \ref{ch:ScalarSystem}, \ref{ch:CurlCurl}, and \ref{ch:SingInc} becoming bloated with further introductions and notational conventions, we provide an introduction to the key concepts that will be prevalent throughout.

In section \ref{sec:TP-GelfandTransform}, we introduce the Gelfand transform and the resulting ``shifted gradient" operators $\tgrad, \ktgrad$.
Section \ref{sec:QuantumGraphs} contains the notation we will use, and theory that we will require, from the field of quantum graphs.
This covers how we will choose to represent our ``singular structures" (section \ref{ssec:EmbeddedGraphs}), how one defines differential operators on (metric) graphs (sections \ref{ssec:QG-FunctionSpaces}-\ref{ssec:DiffOpsOnGraphs}), and a valuable tool for the analysis of the spectrum of such operators --- the $M$-matrix (section \ref{ssec:MMatrix}).
The introduction of embedded graphs will then allow us to define the singular measures of interest to us in section \ref{sec:SingularMeasures}, before moving on to the definition of the ``non-classical" Sobolev spaces (section \ref{sec:BorelMeasSobSpaces}) that will be the backbone of the variational problems we consider in later chapters.

Before we begin, we set out some notation for frequently occurring function spaces.
Let $\ddom\subset\reals^n$, and denote by $\smooth{\ddom}$ the set of infinitely differentiable (\emph{smooth}) functions on $\ddom$, $\csmooth{\ddom}$ the set of smooth functions with compact support in the interior of $\ddom$, and $\psmooth{\ddom}$ the set of infinitely differentiable functions on $\reals^2$ that are $\ddom$-periodic.
For a measure $\rho$ be a Borel measure on $\ddom$, and denote by $\ltwo{\ddom}{\rho}$ the space of $\rho$-square-integrable functions on $\ddom$.
If the measure is omitted, it is taken to be the $n$-dimensional Lebesgue measure on $\ddom$.
An exception to this will be cases where we are considering an interval $I\subset\reals$ equipped with the Lebesgue measure; here we will write $\ltwo{I}{y}$ for the space of square integrable functions on this interval, and similarly write $\gradSob{I}{y}$ for the (Sobolev) space of $u\in\ltwo{I}{y}$ that possess a weak derivative $u'\in\ltwo{I}{y}$.

\section{The Gelfand Transform} \label{sec:TP-GelfandTransform}
We will begin with a more detailed overview of the Gelfand transform, for our study of (the spectrum of) variational problems on periodic singular structures.
For a more complete introduction to the Floquet-Bloch theory of periodic differential operators (under which the Gelfand transform falls), one should consult a source such as \cite[section 7.3]{kuchment2001mathematics} or \cite{reed1978iv} and the references provided therein.

Let $\ddom=[0,1)^d$  and construct the \emph{dual cell} or \emph{Brillouin zone} to $\ddom$, $B=[-\pi,\pi)^d$.
The \emph{Gelfand transform} $\gelfand u$ of a function $u\in\csmooth{\reals^d}$ is defined as
\begin{align*}
	\gelfand u\bracs{x,\qm} := \sum_{m\in\integers^d}u(x+m)\e^{\rmi\qm\cdot(x+m)} \in\psmooth{\ddom}.
\end{align*}
The parameter $\qm\in B$ is called the \emph{quasi-momentum} --- it is the analogue of the dual variable $\kappa$ introduced when taking the Fourier transform of a function.
The function $\gelfand u$ is $\ddom$-periodic in $x$, and satisfies the equality
\begin{align} \label{eq:GelfandTransform}
	\gelfand u\bracs{x,\qm+2\pi m'} &= \e^{2\pi m'\rmi\cdot x}\gelfand u\bracs{x,\qm},
\end{align}
for any $m'\in\integers^d$, so provided we know $\gelfand u$ on $\ddom\times B$, we can reconstruct $\gelfand u$ for any $x\in\reals^d$.
In addition, for each $u\in\csmooth{\reals^d}$ and $\qm\in B$, the function $\gelfand u\bracs{\cdot,\qm}$ belongs to the space of smooth functions on the torus --- this new domain is compact, which is central to the utility of this transform.
The formula for the inverse transform of $\gelfand$ is
\begin{align*}
	\gelfand^{-1}v (x) := \integral{B}{\e^{-\rmi\qm\cdot x}v\bracs{x,\qm}}{\qm},
\end{align*}
where for each $\qm$, $v\bracs{\cdot,\qm}\in\psmooth{\ddom}$.
One can demonstrate (using density arguments) that $\gelfand$ can be extended to an isometry from $L^2\bracs{\reals^d}$ to $L^2\bracs{\ddom}$-valued functions on $B$ \cite[theorem 7.3]{kuchment2001mathematics}.

Let us now examine how the Gelfand transform interacts with differential operators.
Notice that, if $\grad_x$ denotes the usual gradient operator in $\reals^d$ and $u(x)$ is a differentiable function, we have that
\begin{align*}
	\grad_x\gelfand u\bracs{x,\qm} 
	&= \sum_{m\in\integers^d} \e^{\rmi\qm\cdot\bracs{x+m}}\grad_x u(x+m) + \rmi\qm\e^{\rmi\qm\cdot\bracs{x+m}}u(x+m) \\
	&= \sum_{m\in\integers^d} \bracs{\grad_x u + \rmi\qm u}(x+m)\e^{\rmi\qm\cdot\bracs{x+m}}
	= \sqbracs{\gelfand\bracs{\grad_x+\rmi\qm}u}(x,\qm).
\end{align*}
That is, the Gelfand transform induces a shift in the gradient operator --- we have used the subscript $x$ to emphasise that the gradient is taken with respect to the variable $x$ (however will drop this additional notation upon completion of this section).
For a general differential operator $\mathcal{A} = \mathcal{A}\bracs{\grad_x}$ with $\ddom$-periodic coefficients acting on functions defined in $\reals^d$, we can deduce that
\begin{align*}
	\sqbracs{\gelfand\bracs{\mathcal{A}\bracs{\grad_x}u}}(x,\qm) 
	&= \mathcal{A}\bracs{\grad_x + \rmi\qm}\gelfand u\bracs{x,\qm}.
\end{align*}
For each $\qm\in B$ denote $\mathcal{A}_{\qm}:=\mathcal{A}\bracs{\grad+\rmi\qm}$ for each $\qm\in B$ --- the domain of (each of) these operators is $L^2\bracs{\ddom}$, identifiable with the space of square-integrable functions defined on the torus.
The Gelfand transform has effectively ``expanded" the operator $\mathcal{A}$ which acted on $L^2\bracs{\reals^d}$ into the family of operators $\mathcal{A}_{\qm}$ parametrised by $\qm$.
The operators $\mathcal{A}_{\qm}$ are referred to as the \emph{fibres} of $\mathcal{A}$, and one typically  sees $\mathcal{A}$ written as a \emph{direct integral} of operators
\begin{align*}
	\mathcal{A} &= \int_{B}^{\bigoplus}\mathcal{A}_{\qm} \ \md\qm,
\end{align*}
the notation for which is discussed in \cite{reed1978iv}.
In the interest of avoiding technical details that are tangential to our work, but still providing the reader with the ideas behind this transform, let us draw analogy with the use of the Fourier transform in the case of constant coefficients.
If $\mathcal{L}\bracs{\grad}$ is an operator with constant coefficients, that defines a system of differential equations, the natural course of action when studying the problem $\mathcal{L}\bracs{\grad}u=\lambda u$ is to apply a Fourier transform.
This turns the action of $\mathcal{L}\bracs{\grad}$ into multiplication by a (self-adjoint) matrix $L(\wavenumber)$ in the dual variable $\wavenumber$, resulting in the equation $L\bracs{\wavenumber}\hat{u}\bracs{\wavenumber}=\lambda\hat{u}\bracs{\wavenumber}$ (with hats denoting the Fourier transform).
The eigenvalues $\lambda$ of our problem in real space are then the union of the eigenvalues $\lambda_i\bracs{\wavenumber}$ of the matrix $L\bracs{\wavenumber}$, over the range of $\wavenumber$.
The direct integral expansion is analogous; except we will be looking for the eigenvalues of a family of operators $\mathcal{A}_{\qm}$ parametrised by the quasi-momentum, rather than a family of eigenvalues $\lambda_i(\wavenumber)$ of matrices $L(\wavenumber)$ parametrised by $\wavenumber$. 
This brings us to a key result; the spectrum of $\mathcal{A}$ is formed from the union of the spectra of the fibres $\mathcal{A}_{\qm}$ over $\qm$,
\begin{align*}
	\sigma\bracs{\mathcal{A}} = \bigcup_{\qm\in B}\sigma\bracs{\mathcal{A}_{\qm}}.
\end{align*}
The domain of the fibres consists of functions defined on a compact domain (the torus), and so provided they also satisfy suitable ellipticity conditions, each of these operators will have compact resolvent, and thus a discrete spectrum of eigenvalues \cite[section 7.3]{kuchment2001mathematics}.
This allows us to order the eigenvalues $\lambda_i\bracs{\qm}$ of $\mathcal{A}_{\qm}$ in ascending order, and when viewed as functions of $\qm$, these $\lambda_i$ are called \emph{dispersion branches}, or individual branches of the \emph{dispersion relations}.
They are also continuous in $\qm$, and allow the spectrum of the original operator $\mathcal{A}$ to be written as
\begin{align*}
	\sigma\bracs{\mathcal{A}} &= \bigcup_{i\in\naturals}\sqbracs{ \min_{\qm}\lambda_i(\qm), \max_{\qm}\lambda_i(\qm) },
\end{align*}
where the interval $\sqbracs{ \min_{\qm}\lambda_i, \max_{\qm}\lambda_i }$ is labelled as the $i^{\text{th}}$ spectral band of $\mathcal{A}$.
With the dispersion branches being continuous functions of $\qm$, there is also the useful result that if $B^*$ is a dense subset of $B$,
\begin{align} \label{eq:TP-DenseQMSubsetSuffices}
	\sigma\bracs{\mathcal{A}} &= \bigcup_{i\in\naturals}\overline{\lambda_i\bracs{B^*}}
	= \overline{\bigcup_{\qm\in B^*}\sigma\bracs{\mathcal{A}_{\qm}}}.
\end{align}
This is to say, not all values of the quasi-momentum have to be considered when computing the spectrum of the original operator $\mathcal{A}$ \cite[section 7.4]{kuchment2001mathematics}, which is something that we can exploit in our discussion in section \ref{ssec:ApproachConsiderations}.

The Gelfand transform will be an incredibly useful tool for us in our analysis of variational problems on periodic, singular structures.
To this end we introduce some shorthand for the ``$\qm$-shifted" gradient operator $\tgrad$ that the transform introduces,
\begin{align*}
	\tgrad u = \grad u + \rmi\qm u 
	= \begin{pmatrix} \bracs{\partial_1 + \rmi\qm_1}u \\ \bracs{\partial_2 + \rmi\qm_2}u \end{pmatrix},
\end{align*}
for all differentiable functions $u$.
In chapter \ref{ch:CurlCurl}, we will be considering a singular structure that has been extruded into three dimensions, and so will also need to consider the operator $\ktgrad$ which acts as
\begin{align*}
	\ktgrad u = 
	\begin{pmatrix} \bracs{\partial_1 + \rmi\qm_1}u \\ \bracs{\partial_2 + \rmi\qm_2}u \\ \rmi\wavenumber u \end{pmatrix},
\end{align*}
which is the result of a Gelfand transform in the $\bracs{x_1,x_2}$-plane and a Fourier transform in the $x_3$ (extruded) direction.
At times, we may also use the operator $\kgrad = \bracs{\partial_1, \partial_2, \rmi\wavenumber}^\top$ (which acts analogously to the above).
We also denote by $\ktcurl{}$ the ``curl" operator obtained after applying the Gelfand transform, so formally
\begin{align*}
	\ktcurl{} u &= \ktgrad\wedge u
	= 
	\begin{pmatrix}
		\bracs{\partial_1+\rmi\qm_1} u_3 - \rmi\wavenumber u_1 \\
		\rmi\wavenumber u_2 - \bracs{\partial_2+\rmi\qm_2} u_3 \\
		\bracs{\partial_2+\rmi\qm_2} u_1 - \bracs{\partial_1+\rmi\qm_1} u_2
	\end{pmatrix},
\end{align*}
where $\wedge$ denotes the vector-cross product.

Before we continue, it is worth mentioning here that an alternative to the Gelfand transform, the \emph{Floquet transform} is also widely used in the study of periodic differential equations.
Whilst these transforms serve essentially the same purpose in aiding the analysis of differential operators with periodic coefficients, there is a subtle difference in the manner in which they do this.
Under the Floquet transform, one obtains a similar ``direct integral" representation of the original operator $\mathcal{A}$ parametrised by $\qm$, however under this representation the ``decomposed" operators $\mathcal{A}_{\qm}$ have the same differential action, but act on different function spaces.
By contrast, the Gelfand transform elects to preserve the underlying function spaces at the expense of an alteration to the action of the differential operators.
Which of the two transforms is used typically comes down to preference, and there are even places in the literature where the two names are used interchangeably.
Throughout this work we will continue to use the Gelfand transform as introduced in this section through equation \eqref{eq:GelfandTransform} --- although if one is interested in the setup of the non-classical Sobolev spaces in the Floquet setting, see \cite{cherednichenko2018operator, cherednichenko2022operator}.

\section{Quantum Graphs}

some stuff about quantum graphs :)

\section{Singular Measures} \label{sec:SingularMeasures}
Now that we have established how we can represent singular structures through embedded graphs, we move on to the introduction of the singular measures which will allow us to pose meaningful variational problems on these structures.
The introduction of these measures also necessitates a change in the way we view gradients, curls, and divergences, which is the subject of section \ref{sec:BorelMeasSobSpaces}, and is our solution to the problems discussed in section \ref{sec:Intro-ProblemIntroduction}.

Let $\ddom\subset\reals^d$, and let $\mathcal{B}_{\ddom}$ denote the Borel sigma algebra for $\ddom$ (with the topology inherited from $\reals^d$).
A (Borel) measure $\rho$ on the measurable space $\bracs{\ddom,\mathcal{B}_{\ddom}}$ is called \emph{singular} if there exist two disjoint sets $S_1,S_2\in \mathcal{B}_{\ddom}$ such that $\rho$ is zero on all (measurable) subsets of $S_1$ and the $d$-dimensional Lebesgue measure $\lambda_d$ is zero on all subsets of $S_2$.
The textbook example of a singular measure is the point-mass (or ``$\delta$-function") measure placed at a point $x_0\in\ddom$,
\begin{align*}
	\delta_{x_0}\bracs{B} &= \begin{cases} 1 & x_0\in B, \\ 0 & x_0\not\in B, \end{cases}
	\qquad \forall B\in\mathcal{B}_{\ddom},
\end{align*}
for which we can take $S_1=\ddom\setminus\clbracs{x_0}$ and $S_2=\clbracs{x_0}$.
In addition to point masses, we will also want to consider singular measures which support our singular structures.
Let $\graph=\bracs{\vertSet, \edgeSet}$ be a metric graph embedded into $\reals^d$.
For each $I_{jk}\in\edgeSet$ define the (Borel) measure $\lambda_{jk}$ as the measure which supports the one-dimensional measure $\lambda_1$ on (or ``along") $I_{jk}$,
\begin{align*}
	\lambda_{jk}\bracs{B} = \lambda_{1}\bracs{r_{jk}^{-1}\bracs{B \cap I_{jk}}},
	&\quad\text{for all Borel } B.
\end{align*}
Here, $r_{jk}$ is the parametrisation of the edge $I_{jk}$ as per assumption \ref{ass:MeasTheoryProblemSetup}.
The measure $\lambda_{jk}$ will be referred to as the \emph{singular measure that supports $I_{jk}$}, or just the \emph{singular measure on $I_{jk}$}.
Since the sum of two singular measures is another singular measure, we can then define the following singular measures;
\begin{align*}
	\ddmes\bracs{B} = \sum_{v_j\in \vertSet}\sum_{j\conLeft k} \lambda_{jk}\bracs{B},
	\quad
	\nu\bracs{B} = \sum_{v_j\in\vertSet}\alpha_j\delta_j\bracs{B},
	\quad
	\dddmes\bracs{B} = \ddmes\bracs{B} + \nu\bracs{B},
\end{align*}
where we refer to $\ddmes$ as the ``\emph{singular measure} that supports $\graph$", and $\nu$ is the singular measure supporting the masses $\alpha_j$ at the vertices of $\graph$.
For graphs embedded into two and three dimensions, figure \ref{fig:Diagram_SingularMeasure2D} illustrates the $\dddmes$ ``measure" of Borel sets.
\begin{figure}[b!]
	\centering
	\begin{subfigure}[t]{0.45\textwidth}
		\centering
		\includegraphics[scale=0.85]{Diagram_SingularMeasure2D.pdf}
		\caption{\label{fig:Diagram_SingularMeasure2D} For a graph embedded in $\reals^2$, the $\ddmes$-measure of any Borel set $B$ is obtained from summing the contributions of each $\lambda_{jk}$, as indicated by the thickened lines. Sets that do not intersect $\graph$ have zero measure.}
	\end{subfigure}
	~
	\begin{subfigure}[t]{0.45\textwidth}
		\centering
		\includegraphics[scale=0.5]{Diagram_SingularMeasure3D.pdf}
		\caption{\label{fig:Diagram_SingularMeasure3D} An extension of the concept of a singular measure to 3 dimensions. The red portions of each edge indicate the contributions from each edge to the measure of the cube $B$.}
	\end{subfigure}
\end{figure}
In chapter \ref{ch:SingInc}, we will reintroduce the background material into which our singular structure is embedded.
This will require us to consider the sum of singular measures and the ``background" Lebesgue measure $\lambda_d$, so for a singular measure $\rho$ we shall write \tstk{change overhead tilde notation on $\dddmes$ to distinguish!!!} $\tilde{\rho} = \rho + \lambda_d$.

\section{Sobolev Spaces with Respect to Borel Measures}

some more words on these wonderful monstrosities!

\section{Introduction to the Domains and Problems we will Consider} \label{sec:TP-DomainSetup}
In this section we setup the domains on which we will be working throughout chapters \ref{ch:ScalarSystem}-\ref{ch:SingInc}, and establish the notation that we shall use going forward.
Each chapter will contain a short recap of the relevant notation, but unless otherwise stated uses the setup and notation established here.

Let $\hat{\graph}$ be a periodic graph embedded into $\reals^2$ and $\upsilon$ be the singular measure supporting $\hat{\graph}$.
Suppose that $\hat{\graph}$ has unit cell $\ddom=[0,1)^2$ and unit graph $\graph=\bracs{\vertSet, \edgeSet}$, and adopt assumption \ref{ass:MeasTheoryProblemSetup} for $\graph$.
Define the measures $\lambda_{jk}$, $\ddmes$, $\nu$, $\dddmes$, and $\compMes$ as they appear in section \ref{sec:SingularMeasures} for $\graph$.
Throughout, we always associate $\ddom$ with the (surface of a) torus, by identifying the ``left and right" boundaries $\clbracs{x_0=0}$ and $\clbracs{x_0=1}$ with each other, and the ``top and bottom" boundaries $\clbracs{x_1=1}$ and $\clbracs{x_1=1}$ similarly.

In chapter \ref{ch:ScalarSystem}, we will be only be concerned with the 2D domain $\ddom$.
We will define the family of operators $-\laplacian_{\qm}$ through the (bilinear) form 
\begin{align*}
	b(u,v) = \integral{\ddom}{ \tgrad_{\dddmes} u\cdot\overline{\tgrad_{\dddmes} v} }{\dddmes},
	\qquad u,v\in\tgradSob{\ddom}{\dddmes},
\end{align*}
which will provide us with the (spectral, variational) problem of finding $\omega^2>0, u\in\tgradSob{\ddom}{\dddmes}$ such that
\begin{align*}
	\integral{\ddom}{ \tgrad_{\dddmes} u\cdot\overline{\tgrad_{\dddmes}\phi} }{\dddmes}
	&= \omega^2\integral{\ddom}{ u\overline{\phi} }{\dddmes},
	\qquad\forall\phi\in\psmooth{\ddom}.
\end{align*}
As discussed in section \ref{ssec:SobSpacesAndGelfand}, the union of the spectra $\sigma\bracs{-\laplacian_{\qm}}$ over the quasi-momentum $\qm$ provides us with the spectrum of the periodic operator $-\laplacian$ on $\reals^2$, defined through the form
\begin{align} \label{eq:TP-2DWaveEqnWholeSpace}
	\integral{\reals^2}{ \grad u\cdot\overline{\grad v} }{\upsilon}
	\qquad u,v\in\gradSob{\reals^2}{\upsilon}.
\end{align}
Aside from replacement of the Lebesgue measure with the singular measure $\dddmes$ (and the corresponding replacement of gradients), \eqref{eq:TP-2DWaveEqnWholeSpace} is markedly similar to the weak formulation for the acoustic approximation.
Our analysis of the operators $-\laplacian_{\qm}$ will lead us to a quantum graph problem on $\graph$, and analysis of the spectrum via the $M$-matrix.
We derive an explicit formula for the $M$-matrix (for given $\kt$) in terms of the geometry of $\graph$, and discuss the numerical and analytic techniques that this makes available to us, complimenting this with examples.

We then look to extend the work in chapter \ref{ch:ScalarSystem} to encompass the (analogue of the) curl-of-the-curl equation, which will require us to consider the domain $\dddom = \ddom\times[0,\infty)$.
This geometry corresponds to a collection of planes parallel to the $x_3$-axis, and it is not difficult to show that the product measure $\dddmes\times\lambda_1$ corresponds to a singular measure that supports these induced planes (along with a contribution from the measure $\nu$ at the extruded vertices).
Note that we will not be considering a graph embedded into three dimensions, since such a problem has rather trivial nature, as discussed in section \tstk{section, curls of zero are everything on 3D graphs}.

Finally, in chapter \ref{ch:SingInc} we will return to the two-dimensional domain $\ddom$, however equip it with the composite measure $\compMes$ rather than the singular measure $\dddmes$.
In this case the graph $\graph$ naturally breaks the domain $\ddom$ into disjoint (open) sets $\ddom_i$, for $i$ in some index set, where
\begin{itemize}
	\item For every $i$, the domain $\ddom_i$ is an $n$-gon, whose boundary $\partial\ddom_i$ is equal to the union of $n$ (distinct) edges of $\graph$.
	Note that this implies that $\ddom_i$ is a Lipschitz domain.
	\item We have that
	\begin{align*}
		\ddom = \graph \cup \bigcup_{i}\ddom_i,
		\qquad
		\emptyset = \graph \cap \bigcup_{i}\ddom_i.
	\end{align*}
\end{itemize}
In this chapter we will refer to the regions $\ddom_i$ as the \emph{bulk regions} and the graph $\graph$ (and its constituent edges) as the \emph{skeleton}, to sidestep a philosophical debate about whether $\graph$ or the $\ddom_i$ are the ``inclusions".