\chapter{Preliminaries and Theory} \label{ch:TheoryPrelims}
Throughout this work there will be a number of common themes and conventions whose content is not specific to one particular chapter or research problem.
To avoid redirecting the reader to a number of different points in this work, and in the interest of preventing chapters \ref{ch:ScalarSystem}, \ref{ch:CurlCurl}, and \ref{ch:SingInc} becoming bloated with further introductions and notational conventions, we provide an introduction to the key concepts that will be prevalent throughout.

In section \ref{sec:TP-GelfandTransform}, we introduce the Gelfand transform and the resulting ``shifted gradient" operators $\tgrad, \ktgrad$.
Section \ref{sec:QuantumGraphs} contains the notation we will use, and theory that we will require, from the field of quantum graphs.
This covers how we will choose to represent our ``singular structures" (section \ref{ssec:EmbeddedGraphs}), how one defines differential operators on (metric) graphs (sections \ref{ssec:QG-FunctionSpaces}-\ref{ssec:DiffOpsOnGraphs}), and a valuable tool for the analysis of the spectrum of such operators --- the $M$-matrix (section \ref{ssec:MMatrix}).
The introduction of embedded graphs will then allow us to define the singular measures of interest to us in section \ref{sec:SingularMeasures}, before moving on to the definition of the ``non-classical" Sobolev spaces (section \ref{sec:BorelMeasSobSpaces}) that will be the backbone of the variational problems we consider in later chapters.

\section{The Gelfand Transform} \label{sec:TP-GelfandTransform}
We will begin with a more detailed overview of the Gelfand transform, for our study of (the spectrum of) differential operators.
Let $\ddom=[0,1)^d$  and construct the ``dual cell" or Brillouin zone to $\ddom$, $B=[-\pi,\pi)^d$.
The \emph{Gelfand transform} of a function $u\in\csmooth{\reals^d}$ can be defined as
\begin{align*}
	\gelfand u\bracs{x,\qm} := \sum_{m\in\integers^d}u(x+m)\e^{-\rmi\qm\cdot(x+m)}.
\end{align*}
The parameter $\qm\in B$ is called the \emph{quasi-momentum} --- it is the analogue of the dual variable $\kappa$ introduced when taking the Fourier transform of a function.
The function $\gelfand u$ is $\ddom$-periodic in $x$, and satisfies the equality
\begin{align} \label{eq:GelfandTransform}
	\gelfand u\bracs{x,\qm+2\pi m'} &= \e^{-2\pi m'\rmi\cdot x}\gelfand u\bracs{x,\qm},
\end{align}
for any $m'\in\integers^d$, so provided we know $\gelfand u$ on $\ddom\times B$, we can reconstruct $\gelfand u$ for any $x\in\reals^d$.
In addition, for each $u\in\csmooth{\reals^d}$ and $\qm\in B$, the function $\gelfand u\bracs{\cdot,\qm}$ belongs to the space of smooth functions on the torus --- this new domain is compact, which is central to the utility of this transform.
It is simple to see that $\gelfand u\bracs{\cdot,\qm}$ can also be thought of as an element of $\psmooth{\ddom}$, the space of smooth, $\ddom$-periodic functions on $\reals^d$.
The formula for the inverse transform of $\gelfand$ is
\begin{align*}
	\gelfand^{-1}v (x) := \integral{B}{\e^{\rmi\qm\cdot x}v\bracs{x,\qm}}{\qm},
\end{align*}
where for each $\qm$, $v\bracs{\cdot,\qm}\in\psmooth{\ddom}$.
One can demonstrate (using density arguments) that $\gelfand$ can be extended to an isometry from $L^2\bracs{\reals^d}$ to $L^2\bracs{\ddom}$-valued functions on $B$. \tstk{get refs for this?}.

Now consider a differential operator $\mathcal{A} = \mathcal{A}\bracs{\grad}$ in $\reals^d$ with $\ddom$-periodic coefficients, for example the acoustic approximation operator $-\grad\cdot a(x)\grad u$ for some $\ddom$-periodic $a(x)$.
Through a short computation, we can deduce that
\begin{align*}
	\sqbracs{\gelfand\bracs{\mathcal{A}\bracs{\grad_x}u}}(x,\qm) 
	&= \mathcal{A}\bracs{\grad_x + \rmi\qm}\gelfand u\bracs{x,\qm},
\end{align*}
where we have used the subscript $x$ to indicate that the gradient operator $\grad$ still acts on the ``spatial" variable $x$.
Denote by $\mathcal{A}_{\qm}$ the operator $\mathcal{A}\bracs{\grad+\rmi\qm}$ for each $\qm\in B$ --- each of these operators now acts on the space of $\ddom$-periodic functions, or rather the space of functions defined on the torus, which we denote by $D_{\mathcal{A}}$.
The Gelfand transform has effectively ``expanded" the operator $\mathcal{A}$ which acted on (functions on) $\reals^d$ into the family of operators $\mathcal{A}_{\qm}$ parametrised by $\qm$.
Typically one sees $\mathcal{A}$ written as a ``direct integral" of operators
\begin{align*}
	\mathcal{A} &= \int_{B}^{\bigoplus}\mathcal{A}_{\qm} \ \md\qm,
\end{align*}
the notation for which is discussed in \cite{reed1978iv}.
The operators $\mathcal{A}_{\qm}$ are referred to as the \emph{fibres} of $\mathcal{A}$.
We can think of the direct integral as a generalisation of a direct sum over a countable index to a continuous index ($\qm$).
Perhaps more helpful is the interpretation that the action of the original operator $\mathcal{A}$ can be thought of as acting independently on an infinite number of copies of $D_{\mathcal{A}}$ indexed by $\qm$, the action on each of these spaces being $\mathcal{A}_{\qm}$.
%To draw a complete analogy, consider a Hilbert space $H$ and an operator $T$ on $H$ which acts independently on two (closed) subspaces $H_1, H_2$, such that $H = H_1 \oplus H_2$.
%We could denote by $T_i$ the restriction of $T$ to $H_i$ ($i=1,2$), and then write the action of $T$ as
%\begin{align*}
%	Tu &= \begin{pmatrix} T_1 & 0 \\ 0 & T_2 \end{pmatrix}
%	\begin{pmatrix} u_1 \\ u_2	\end{pmatrix},
%\end{align*}
%for any element $u = u_1 + u_2$ for $u_i\in H_i$ (an example of this would be the decoupling of Maxwell operator $\mathcal{M}$ into the TE and TM modes, see the discussion in section \ref{sec:Intro-Maxwell} around equation \eqref{eq:Intro-AcousticApprox}).
%By studying the operators $T_1$ and $T_2$, we can deduce the behaviour of the operator $T$, and in particular deduce its spectrum from the union of the spectra of the $T_i$.
%The direct integral notation extends this to when an operator $\mathcal{A}$ cannot be broken down into a countable (or finite) direct sum, and analogous conclusions apply: 
The action of $\mathcal{A}$ on a function $u$ can then be recovered from the ``sum" of actions of the $\mathcal{A}_{\qm}$ on $\gelfand u(\cdot,\qm)$,
\begin{align} \label{eq:GelfandDirectIntegral}
	\mathcal{A}u(x) = \gelfand^{-1}\int_{B}^{\bigoplus}\mathcal{A}_{\qm}\gelfand u\bracs{x,\qm} \ \md\qm.
\end{align}
The consequence that we are concerned with is that the spectrum of $\mathcal{A}$ is formed from the union of the spectra of $\mathcal{A}_{\qm}$ over $\qm$ \tstk{ref for this? It's a standard result...}, 
\begin{align*}
	\sigma\bracs{\mathcal{A}} = \bigcup_{\qm\in B}\sigma\bracs{\mathcal{A}_{\qm}}.
\end{align*}
As mentioned in section \ref{sec:Intro-Maxwell}, the space $D_{\mathcal{A}}$ consists of functions defined on a compact domain, and so provided that the $\mathcal{A}_{\qm}$ satisfy suitable ellipticity conditions, each of these operators will have compact resolvent, and thus a discrete spectrum of eigenvalues.
This allows us to order the eigenvalues $\lambda_i\bracs{\qm}$ of $\mathcal{A}_{\qm}$ in ascending order, and when viewed as functions of $\qm$, these $\lambda_i$ are called \emph{dispersion branches}, or individual branches of the \emph{dispersion relations}.
They are also continuous in $\qm$, and thus allow the spectrum of the original operator $\mathcal{A}$ to be written as
\begin{align*}
	\sigma\bracs{\mathcal{A}} &= \bigcup_{i\in\naturals}\sqbracs{ \min_{\qm}\lambda_i, \max_{\qm}\lambda_i },
\end{align*}
where the interval $\sqbracs{ \min_{\qm}\lambda_i, \max_{\qm}\lambda_i }$ is labelled as the $i^{\text{th}}$ spectral band of $\mathcal{A}$.
With the dispersion branches being continuous functions of $\qm$, we also have the important result that if $B^*$ is a dense subset of $B$,
\begin{align} \label{eq:TP-DenseQMSubsetSuffices}
	\sigma\bracs{\mathcal{A}} &= \bigcup_{i\in\naturals}\overline{\lambda_i\bracs{B^*}}
	= \overline{\bigcup_{\qm\in B^*}\sigma\bracs{\mathcal{A}_{\qm}}}.
\end{align}
This is to say, not all values of the quasi-momentum have to be considered when computing the spectrum of the original operator $\mathcal{A}$ \tstk{kuchment review paper, page 223}, which is something that we shall exploit in our discussion in section \ref{ssec:ApproachConsiderations}.

The Gelfand transform will be an incredibly useful tool for us in our analysis of variational problems on periodic, singular structures.
As such we introduce some shorthand for the ``$\qm$-shifted" gradient operator that the transform introduces, in $\reals^2$ we denote
\begin{align*}
	\tgrad = \grad + \rmi\qm 
	= \begin{pmatrix} \partial_1 + \rmi\qm_1 \\ \partial_2 + \rmi\qm_2 \end{pmatrix},
\end{align*}
which acts analogously to the usual gradient operator $\grad$.
In chapter \ref{ch:CurlCurl}, we will be considering a singular structure that has been extruded into three dimensions, and so will also need to consider the operator
\begin{align*}
	\ktgrad = 
	\begin{pmatrix} \partial_1 + \rmi\qm_1 \\ \partial_2 + \rmi\qm_2 \\ \rmi\wavenumber \end{pmatrix},
\end{align*}
which is the result of a Gelfand transform in the $\bracs{x_1,x_2}$-plane and a Fourier transform in the $x_3$ (extruded) direction.
At times, we may also use the operator $\kgrad = \bracs{\partial_1, \partial_2, \rmi\wavenumber}^\top$.
We also denote by $\ktcurl{}$ the ``curl" operator obtained after applying the Gelfand transform, so formally
\begin{align*}
	\ktcurl{}\phi &= \ktgrad\wedge\phi
	= 
	\begin{pmatrix}
		\bracs{\partial_1+\rmi\qm_1}\phi_3 - \rmi\wavenumber\phi_1 \\
		\rmi\wavenumber\phi_2 - \bracs{\partial_2+\rmi\qm_2}\phi_3 \\
		\bracs{\partial_2+\rmi\qm_2}\phi_1 - \bracs{\partial_1+\rmi\qm_1}\phi_2
	\end{pmatrix}
\end{align*}
for a suitably differentiable vector-valued function $\phi$, and $\wedge$ denotes the vector-cross product.

Before we continue, it is worth mentioning here that an alternative to the Gelfand transform, the \emph{Floquet transform} is also widely used in the study of periodic differential equations.
Whilst these transforms serve essentially the same purpose in aiding the analysis of differential operators with periodic coefficients, there is a subtle difference in the manner in which they do this.
Under the Floquet transform, one obtains a similar ``direct integral" representation of the original operator $\mathcal{A}$ parametrised by $\qm$, however under this representation the ``decomposed" operators $\mathcal{A}_{\qm}$ have the same differential action, but act on different function spaces.
By contrast, the Gelfand transform elects to preserve the underlying function spaces at the expense of an alteration to the action of the differential operators.
Which of the two transforms is used typically comes down to personal preference in most cases, and there are even places in the literature where the two names are used interchangeably.
Throughout this work, we will continue to use the Gelfand transform as introduced in this section through equation \eqref{eq:GelfandTransform}.

\section{Concepts from the Theory of Metric Graphs} \label{sec:QuantumGraphs}
\tstk{introduction needs contextifying (was just copy-pasted from curl-curl paper.... Here we introduce the objects appearing in the system \eqref{eq:3DQGFullSystem} and measure-theoretic formulation \eqref{eq:PeriodCellLaplaceStrongForm}.
To this end, we outline some elements of the theory of metric (and quantum) graphs that we borrow in what follows, and setup the notation we use throughout.
This includes a discussion of what is meant by an embedded graph, in order to provide a foundation for the term ``periodic graph" which represents the geometry of our (physically-motivated) singular structure.
We then define the function spaces we use when dealing with the problem \eqref{eq:3DQGFullSystem}, as well as an overview of the $M$-matrix, a tool that will be key to our analysis of its spectrum.
The concept of a singular measure is needed to formally link \eqref{eq:PeriodCellCurlCurlStrongForm} and \eqref{eq:3DQGFullSystem}, and these details are left to the appendix (\ref{app:SMandSGO}).}

\subsection{Periodic graphs embedded in the plane} \label{ssec:EmbeddedGraphs}
We begin with the definition of a metric graph.
Let $\graph=\bracs{\vertSet,\edgeSet}$ be a directed graph with vertices $v_j\in \vertSet$ and edges $I_{jk}=\bracs{v_j, v_k}\in \edgeSet$, where $I_{jk}$ is directed from vertex $v_j$ (``on the left") to vertex $v_k$ (``on the right").
Strictly, we should use the notation $I_{jk}^l$ for the edges, where the superscript $l$ parametrises the edges sharing the endpoints $v_j$ and $v_k$, however in what follows we drop the superscript $l$ to simplify the notation.
Each edge $I_{jk}$ is assigned a length $l_{jk}>0$, and we use the same notation $I_{jk}$ for the interval $\sqbracs{0,l_{jk}}$.
We also assign each vertex $v_j$ a ``coupling constant" $\alpha_j\in\complex$ (although in our examples we will only be considering $\alpha_j>0$, to be consistent with our physical motivation in section \ref{ssec:PhysMot}), and denote by $\alpha := \mathrm{diag}\bracs{\alpha_1, \alpha_2, ..., \alpha_{\abs{\vertSet}}}$ the diagonal matrix with these coupling constants along the diagonal.
A graph $\graph$ with the metric structure described above is called a \emph{metric graph}.
Metric graphs convey a sense of size or bulk to the structures they describe --- the edges between vertices now represent physical connections as opposed to combinatorial links as in graph theory.
One can form a quantum graph by equipping a metric graph with a suitable differential operator (and associated vertex- or ``boundary"-conditions) as discussed in section \ref{ssec:FunctionSpaces} and in greater detail in \cite{berkolaiko2013introduction}.
Before proceeding to a discussion of operators on metric graphs, we first outline some notation and conventions that we adopt.

We say that a (directed) graph $\graph = \bracs{\vertSet,\edgeSet}$ is embedded into $\reals^2$ if each vertex $v_j$ is associated to a point, which we also label $v_j\in\reals^2$, and each edge $I_{jk}$ associated to a curve $\gamma_{jk}\subset\reals^2$ with arclength $l_{jk}$, so that there is a differentiable \tstk{smooth? doesn't really matter} map
\begin{align} \label{eq:GeneralCurveParam}
	r: \sqbracs{0,l_{jk}} \rightarrow \gamma_{jk}, \quad r(0) = v_j, \quad r\bracs{l_{jk}} = v_k.
\end{align}
We will drop the distinction between $I_{jk}$ and $\gamma_{jk}$, simply using $I_{jk}$ for both.
Clearly, an embedded graph gives rise to a metric graph, and a metric graph can be assigned an embedding to construct an embedded graph (although there is by no means only one choice of embedding to do this, as we will touch on at the end of section \ref{sec:QuantumGraphs}).
Our choice to embed (metric) graphs into $\reals^2$ is a convenient way to represent the singular structures we want to examine (see section \ref{ssec:SingularStructures}), however there are more general definitions depending on the choice of space to ``embed" the graph into.
Prescribing an embedding also allows us to define what it means for a graph to be ``periodic" rather intuitively.
For a unit vector $x\in\reals^2$, an embedded graph is said to be $T$-periodic in the direction $x$ if it is invariant under the translation $Tx$ applied to its vertices and edges.
In what follows, we always refer to the minimal such $T>0$ (``period") for a given $x\in\reals^2$.
If $\graph$ is periodic in the (orthogonal) axial directions $e_1, e_2$ (with periods $T_1, T_2$ respectively), then we can define the ``period cell" or ``unit cell" $\mathcal{P}$ of $\graph$ in the obvious manner: take the intersection of the graph $\graph$ with the region $\mathcal{P} = \sqbracs{0,T_1}\times\sqbracs{0,T_2}$ and match the left boundary to the right, and the top boundary to the bottom.
That is, view the part $\graph_{\mathcal{P}}$ of the graph $\graph$ contained in the region $\mathcal{P}$ as a set on a torus, see figure \ref{fig:PeriodCellIllustration}.
\begin{figure}[b!]
	\centering
	\begin{subfigure}[t]{0.45\textwidth}
		\centering
		\includegraphics[height=4.5cm]{Diagram_PeriodCellFullLattice.pdf}
		\caption{\label{fig:Diagram_PeriodCellFullLattice} A periodic graph embedded into $\reals^2$, with the period cell marked.}
	\end{subfigure}
	~
	\begin{subfigure}[t]{0.45\textwidth}
		\centering
		\includegraphics[height=4.5cm]{Diagram_PeriodCellEdgeAssociation.pdf}
		\caption{\label{fig:Diagram_PeriodCellEdgeAssociation} The period cell of the graph in \ref{fig:Diagram_PeriodCellFullLattice}. Notice how the edges of the period cell are associated.}
	\end{subfigure}
	\\
	\begin{subfigure}[b]{0.75\textwidth}
		\centering
		\includegraphics[scale=	1.0]{Diagram_PeriodCellOnTorus.pdf}
		\caption{\label{fig:Diagram_PeriodCellOnTorus} An illustration of the period cell as a subset of a torus.}
	\end{subfigure}
	\caption{\label{fig:PeriodCellIllustration} Illustrating a periodic cell of a periodic embedded graph.}
\end{figure} 
Note that so long as the graph $\graph$ is periodic in two linearly independent directions, a linear transform can be applied to transform the period cell of $\graph$ into a rectangle with sides parallel to the co-ordinate axes, and so without loss of generality we only consider rectangular period cells henceforth.
We will refer to $\graph_{\mathcal{P}} = \graph \cap \mathcal{P}$ as the ``period graph", or ``unit graph" of $\graph$.
\tstk{possible to define periodicity without prescribing an embedding --- this discussion appears in the scalar paper's discussion chapter, but could be placed here. There's also the related discussion after introducing the M-matrix...
It should be noted that it is possible to define a notion of ``periodicity" for a metric graph \emph{without} prescribing the graph with an embedding into a Euclidean space, see for example \cite[Chapter~4]{berkolaiko2013introduction}.
However, our physical motivation \tstk{section} behind the singular-structure problems that we wish to consider means it is }

Having equipped our graphs with a metric structure, we are ready to define function spaces on (the edges of) these graphs.
Before doing so however, we outline some further standing assumptions and notational conventions that we adopt throughout this work.
\begin{assumption} \label{ass:MeasTheoryProblemSetup}
	Let $\graph=\bracs{\vertSet,\edgeSet}$ be the period graph of an embedded graph in $\reals^2$ with period cell $\ddom$, so $\graph\subset\ddom$.
	We consider straight edges between vertices, so each $I_{jk}\in \edgeSet$ is the line segment joining the vertices at either end, with lengths $l_{jk} = \abs{I_{jk}} = \abs{v_j-v_k}$, where $\abs{\cdot}$ denotes Euclidean distance.
	Let $e_{jk} = \bracs{e_1^{(jk)}, e_2^{(jk)}}^\top$ be the unit vector parallel to $I_{jk}$ and directed from $v_j$ to $v_k$ (that is, directed ``left to right", or in the direction of the edge $I_{jk}$).
	Set
	\begin{align} \label{eq:EdgeParameterisation}
		r_{jk}:\sqbracs{0, l_{jk}} \ni y \mapsto v_j + ye_{jk} \in I_{jk},
	\end{align}
	and note that $r_{jk}'(y) = e_{jk}, \abs{r_{jk}'(y)}=1$ for all $y\in\sqbracs{0, l_{jk}}$.
	Let $n_{jk} = \bracs{n_1^{(jk)}, n_2^{(jk)}}^\top$ be the unit normal to $I_{jk}$ so the frame $y_{jk} := \bracs{n_{jk}, e_{jk}}$ can be obtained by an orthogonal rotation $R_{jk}\in\mathrm{SO}(2)$ of the (canonical) axis vectors $x = \bracs{x_1, x_2}$, formally by $x = R_{jk}y_{jk}$ (where no summation over $j,k$ is implied).
	Note that, under this setup, we have that $e_1^{(jk)} = - n_2^{(jk)}$ and $e_2^{(jk)} = n_1^{(jk)}$.
	Finally, write $\widehat{n}_{jk} = \bracs{ n_{jk}, 0 }^\top\in\reals^3$ and $\widehat{e}_{jk} = \bracs{ e_{jk}, 0 }^\top\in\reals^3$, and $\widehat{x}_3 = \bracs{0,0,1}^\top$.
\end{assumption}
\tstk{check assumptions are consistent throughout the work! In particular, you define the $l_{jk}$ here, does this clash with other notation in the paper????}

\subsection{Function spaces} \label{ssec:FunctionSpaces}
Consider the following function spaces:
\begin{subequations} \label{eq:GraphFuncSpaces}
	\begin{align}
		L^2\bracs{\graph} := \bigoplus_{I_{jk}\in \edgeSet} \ltwo{I_{jk}}{y},
		&\quad H^1\bracs{\graph} := \bigoplus_{I_{jk}\in \edgeSet} \gradSob{I_{jk}}{y}, \\
		H^2\bracs{\graph} := \bigoplus_{I_{jk}\in \edgeSet} H^2_\mathrm{grad}\bracs{I_{jk}, \md y}. &
	\end{align}
\end{subequations}
We define $u^{(jk)} = u\vert_{I_{jk}}$ to be the restriction of a function $u$ defined on $\graph$ to the edge $I_{jk}$ (extended by zero to the other edges of the graph when evaluated at points outside $I_{jk}$).
If $u$ is a vector field, then the subscript $u_{i}$ will be used to denote the $i$\textsuperscript{th} component of $u$.
The shorthand $u^{(jk)}\bracs{v_j}$ and $u^{(jk)}\bracs{v_k}$ will be used for the values $u^{(jk)}\bracs{0}$ and $u^{(jk)}\bracs{l_{jk}}$, respectively.
If additionally $u$ is continuous at a vertex $v_j$, we will use the notation $u\bracs{v_j}$ for this value.
We also use a notion of ``signed derivative" at the endpoints of an edge:
\begin{subequations} \label{eq:SignedDerivConvention}
	\begin{align}
		\pdiff{}{n}u^{(jk)}\bracs{v_j} &= -\diff{u^{(jk)}}{y}\bracs{v_j} = -\lim_{y\rightarrow0}\diff{u^{(jk)}}{y}(y), \\
		\pdiff{}{n}u^{(jk)}\bracs{v_k} &= \diff{u^{(jk)}}{y}\bracs{v_k} = \lim_{y\rightarrow l_{jk}}\diff{u^{(jk)}}{y}(y),
	\end{align}
\end{subequations}
for a differentiable function $u$ on $\graph$ --- we will occasionally use prime notation to denote differentiation by $y$ in the interest of brevity.
We use similar notation for the ``shifted, signed derivative",
\begin{subequations} \label{eq:SignedDerivConventionShifted}
	\begin{align*}
		\bracs{\pdiff{}{n}+ \rmi\qm}u_{jk}\bracs{v_j} &= -\bracs{u_{jk}' + \rmi\qm u_{jk}}\bracs{v_j} = -\lim_{x\rightarrow0} \bracs{ u_{jk}'(x)+\rmi\qm u_{jk}(x) }, \\
		\bracs{\pdiff{}{n}+ \rmi\qm}u_{jk}\bracs{v_k} &= \bracs{u_{jk}' + \rmi\qm u_{jk}}\bracs{v_k} = \lim_{x\rightarrow l_{jk}} \bracs{ u_{jk}'(x)+\rmi\qm u_{jk}(x) },
	\end{align*}
\end{subequations}
as is appropriate for use after application of a Gelfand transform.
One can think of the formulae \eqref{eq:SignedDerivConvention}-\eqref{eq:SignedDerivConventionShifted} as defining an ``exterior facing" derivative at each endpoint, bearing in mind that the edge $I_{jk}$ is directed from $v_j$ to $v_k$.

Before discussing differential operators on metric graphs, we introduce some additional notation for convenience: $j\conLeft k$ stands for ``an edge with endpoints $v_j$ and $v_k$, with $v_j$ on the left", and $j\con k$ is used to mean ``an edge with endpoints $v_j$ and $v_k$", whenever the direction of the edge is not important.
We also use $j \conRight k$ alongside $k\conLeft j$.
In other words, we have
\begin{align*}
	j\conLeft k \Leftrightarrow I_{jk}\in \edgeSet, &\qquad
	j\con k \Leftrightarrow I_{jk}\in \edgeSet \text{ or } I_{kj}\in \edgeSet.
\end{align*}
We then utilise this notation as shorthand in summations involving restrictions of functions to edges:
\begin{align*}
	\sum_{j\conLeft k} = \sum_{\substack{k \\ I_{jk}\in \edgeSet}}, 
	\qquad 	\sum_{j\conRight k} = \sum_{\substack{k \\ I_{kj}\in \edgeSet}},
	\qquad \sum_{j\con k} = \sum_{j\conLeft k} + \sum_{j\conRight k},
\end{align*}
so for example
\begin{align*}
	\sum_{j\con k}u^{(jk)}\bracs{v_j} &= \sum_{j\conLeft k}u^{(jk)}\bracs{v_j} + \sum_{j\conRight k}u^{(kj)}\bracs{v_j}.
\end{align*}
If there are multiple edges $I_{jk}^l$ connecting two vertices, these sums are interpreted as running over all such edges.

\subsection{Differential operators on graphs} \label{ssec:DiffOpsOnGraphs}
Differential operators on metric graphs are defined by specifying a domain and an action on each edge of the graph $\graph$.
A well-posed problem on a graph requires specifying boundary conditions at the vertices (also referred to as ``vertex conditions"), which are identified with the end-points of the corresponding edges $I_{jk}$ as discussed in section \ref{ssec:FunctionSpaces}.
Specifying these vertex conditions also reflects the connectivity of the graph, which the function spaces \eqref{eq:GraphFuncSpaces} do not convey on their own.
A \emph{quantum graph} is then simply a metric graph equipped with such a differential operator.
There are a variety of choices one can make for the vertex conditions, and we will be interested in imposing continuity of the solution at each vertex and a Robin-like condition on the derivatives of the solution, also known as a ``$\delta-$type" condition \cite{albeverio2012solvable, berkolaiko2013introduction}.
By way of example, let us demonstrate how we define a graph Laplacian $\laplacian$ on the period graph $\graph_{\mathcal{P}}$ for a given set of coupling constants $\alpha_j$.
The domain of $\laplacian$ is defined as
\begin{align} \label{eq:GraphLaplacianExample}
	\mathrm{dom}\bracs{\laplacian} &= \clbracs{ u\in H^2\bracs{\graph_{\mathcal{P}}} \ \vert \ \forall j, \ u \text{ is continuous at } v_j, \sum_{j\con k}\pdiff{u^{(jk)}}{n}\bracs{v_j} = \alpha_j u\bracs{v_j} },
\end{align}
and the differential expression (or action) on each edge is $-\ddiff{}{y}$.
For a function $f\in L^2\bracs{\graph_{\mathcal{P}}}$ we can then pose the resolvent problem of finding $u\in\mathrm{dom}\bracs{\laplacian}$ such that
\begin{align*}
	\laplacian u &= f,
\end{align*}
or alternatively can consider the spectral problem of finding eigenpairs $\bracs{z,u}\in\complex\times\mathrm{dom}\bracs{\laplacian}\setminus\clbracs{0}$ such that
\begin{align*}
	\laplacian u &= z u.
\end{align*}
Each of these problems can be rewritten as a system of ODEs on the edge intervals coupled through the vertex conditions:
\begin{align*}
	\laplacian u = f \quad\Leftrightarrow\quad &
	\begin{cases}
		-\bracs{u^{(jk)}}'' = f^{(jk)} \ \forall j\conLeft k, \\
		\forall j, \ u \text{ is continuous at } v_j, \\
		\forall j, \sum_{j\con k}\pdiff{u^{(jk)}}{n}\bracs{v_j} = \alpha_j u\bracs{v_j}.
	\end{cases} \\
	\laplacian u = z u \quad\Leftrightarrow\quad &
	\begin{cases}
		-\bracs{u^{(jk)}}'' = z u^{(jk)} \ \forall j\conLeft k, \\
		\forall j, \ u \text{ is continuous at } v_j, \\
		\forall j, \sum_{j\con k}\pdiff{u^{(jk)}}{n}\bracs{v_j} = \alpha_j u\bracs{v_j}.
	\end{cases}
\end{align*}
\tstk{spectral param in BCs discussion comes here --- these sections might need to be re-referenced. Also, Kuchment-Zeng and Exner-Post touch on this association somewhat}
One will note that only $\alpha_j$ appears in the Robin-like condition in \eqref{eq:GraphLaplacianExample}, but $\alpha_j\bracs{\omega^2-\wavenumber^2}$ is present on the right hand side of \eqref{eq:3DQGDerivCondition}.
\tstk{here we have to talk about Gen. Resolvents --- what to say?}
This means the problem \eqref{eq:3DQGFullSystem} belongs to the class of problems with generalised resolvents.
In section \ref{ssec:MMatrix} we will introduce the $M$-matrix in a more familiar setting (with no $\omega^2$-dependence in the vertex conditions) and remarked that the analysis of the spectrum of \eqref{eq:QGFullSystem} can be carried by replacing the matrix $B$ in section \ref{ssec:MMatrix} with $\omega^2 B$ in section \ref{ssec:MMatrixConsequences}.
Justification for doing so lies in observing that introducing explicit $\omega^2$-dependence will not affect the (structure of) the arguments in the supporting theory \tstk{refs to Ryzhov here?}, and consequentially is expected to give rise to the aforementioned alteration to $B$.
However, a formal argument to justify methodology based on the $M$-matrix and boundary triples (in the context of generalised resolvents) has not been carried out in the literature.
As such it remains open to investigation, but would closely follow and resemble arguments that are already available (and form the basis for the theory presented in section \ref{ssec:MMatrix}).
This observation (and justification) is made in other works that analyse problems with generalised resolvents via use of the theory of boundary triples and the $M$-matrix --- see for example \cite[page 1846]{cherednichenko2018effective} concerning the results of \tstk{The paper \cite[page 1846]{cherednichenko2018effective} specifically refers to those arguments found in Ryzhov, V.: Weyl-Titchmarsh function of an abstract boundary value problem, operator colligations, and linear systems with boundary control. Complex Anal. Oper. Theory 3(1), 289–322 (2009)}.
If one has reservations about this ``gap" in the available theory, an alternative to analysing a problem with generalised resolvents directly is explored in \cite[Section 6]{cherednichenko2017norm}.
One could look to transform a problem with $\omega^2$-dependent $\delta$-type vertex condition (like \eqref{eq:3DQGFullSystem}) into a problem with $\omega^2$-independent $\delta'$-type vertex conditions.
This comes at the cost of having to determine the appropriate (unitary) transform to apply to \eqref{eq:3DQGFullSystem}; but the theory of section \ref{ssec:MMatrix} would apply to the transformed problem, could be used to analyse the spectrum, and then the inverse transform applied.

\subsection{The $M$-Matrix} \label{ssec:MMatrix}
We next provide a brief overview of the $M$-matrix, a tool for characterising the spectrum of a quantum graph, \tstk{which we implement in section \ref{sec:SystemDerivation}.}
The $M$-matrix is a generalisation of the classical Weyl-Titchmarsh $m$-function, and a particular case of the abstract $M$-operator, as it appears in the theory of boundary triples (more information on the $M$-operator can be found in \cite{kochubei1975extensions, kochubei1980characteristic, gorbachuk1991boundary, brown2008boundary, brown2020functional, cherednichenko2020scattering, cherednichenko2018functional}, whilst information for the $m$-function can be found in \cite{titchmarsh1962eigenfunction, atkinson1964discrete}).
Here we restrict ourselves to a short review the specific theory relevant to the present work.
Let $\mathcal{A}$ be a differential operator on a graph $\graph$, and consider the maps
\begin{align*}
	\dmap, \nmap: \mathrm{dom}\bracs{\mathcal{A}} \rightarrow \complex^{\abs{\vertSet}},
\end{align*}
sending a function $u\in\mathrm{dom}\bracs{\mathcal{A}}$ to its Dirichlet and Neumann data at each of the vertices, respectively:
\begin{align*}
	\bracs{\dmap u}_j &= u\bracs{v_j}, \quad &j=1,...,\abs{\vertSet}, \\
	\bracs{\nmap u}_j &= -\sum_{j\con k}\pdiff{u^{(jk)}}{n}\bracs{v_j}, \quad &j=1,...,\abs{\vertSet}. 
\end{align*}
The $M$-matrix at $z\in\complex$ is then defined by
\begin{align*}
	M(z): \ \complex^{\abs{\vertSet}} \ni \dmap u \mapsto \nmap u \in \complex^{\abs{\vertSet}},
	 &\quad \forall u\in\mathrm{ker}\bracs{\mathcal{A}-z}.
\end{align*}
(More general statements concerning the $M$-operator can be found in \cite{derkach1991generalized, derkach2014boundary}).
One detail that should be noted here is that $\mathcal{A}$ is required to be an extension of a \emph{simple} operator \tstk{(see \cite[Section 2.2]{ershova2014isospectrality} for precise definitions, but basically it means that there's no reducing subspace in which the operator is self-adjoint)}.
This can be guaranteed by ensuring $\graph$ contains no looping edges, and has all edge-lengths pairwise-irrationally related \cite{ashurova2014simplicity}.
In section \ref{ssec:ArtificialVertices} we will see that this any graph can be manipulated to adhere to these specifications, through the use of ``artificial" vertices.

Our interest in the $M$-matrix as a tool for analysing the spectrum of a quantum graph problem stems from the following property; it can be shown that $z_0\in\complex$ belongs to the spectrum of $\mathcal{A}$ if and only if $\bracs{M\bracs{z} - B}^{-1}$ does \emph{not} admit analytic continuation into $z_0$ (see \cite[Theorem 2.1]{ershova2014isospectrality}), where $B=-\alpha$.
Equivalently, zero is an eigenvalue of $M\bracs{z_0}-B$ if and only if $z_0$ is an eigenvalue of $\mathcal{A}$ (as presented in \cite[Proposition 1]{derkach1991generalized}, \cite[page 698]{cherednichenko2019time}).
The eigenvalues of $\mathcal{A}$ can thus be found through analysis of the $M$-matrix, by solving for pairs $u\in\mathrm{dom}\bracs{\mathcal{A}}\setminus\clbracs{0}$, $z_0\in\complex$ such that
\begin{align*}
	\bigl( M\bracs{z_0} - B \bigr) \dmap u = 0.
\end{align*}
In section \ref{sec:Discussion} we will expand on how this result provides access to the spectrum of $\mathcal{A}$, and utilise these ideas in the examples of section \ref{sec:Examples}.

Before continuing, we address a detail concerning how we formulate our quantum graph problems and the definition of the $M$-matrix.
In this work we will prescribe an embedded, periodic graph $\graph$ (usually in $\reals^2$) to represent our singular-structure \tstk{you've mentioned this then?}.
As a consequence, upon taking a Gelfand transform the family of operators on the period graph $\graph_{\mathcal{P}}$ contain dependencies on the choice of geometry provided by the embedding, as well as on the quasi-momentum $\qm$.
For example, there may be dependencies on the angle between the edges and the co-ordinate axes (which is not unexpected given how the constants $\qm_{jk}$ in \eqref{eq:3DQGFullSystem} are derived, see section \ref{sec:3DSystemDerivation} or appendix \ref{app:3dMuAnalysis} for details).
For a family of operators on $\graph_{\mathcal{P}}$ containing these dependencies, the resulting family of  $M$-matrices will also reflect these dependencies.
That is, the $M$-matrix for each member of the family will depend on the spectral parameter $z$, the quasi-momentum $\qm$, and a number of geometric quantities arising from the embedding.
As the $M$-matrix contains all the spectral information about the operator, this means that the spectra of each individual operator on $\graph_{\mathcal{P}}$ will inherit these dependencies from the embedding. 
The union of these spectra over the quasi-momentum $\qm$ will yield the spectrum of the original operator on $\graph$.
However, the notion of a periodic quantum (or metric) graph $\tilde{\graph}$ can be defined without prescribing an embedding, as is done in \cite[Chapter~4]{berkolaiko2013introduction}.
The graph $\tilde{\graph}$ prescribes a notion of length and derivative, and so its $M$-matrix can then also be defined and computed, and analysed to obtain the spectrum --- potentially in a manner involving an abstract version of the Gelfand transform.
Proceeding this way means that, since no embedding of $\tilde{\graph}$ is made into a realisable (or physical) space, the resulting analysis is free of the additional geometric dependencies described above.
When it is possible to prescribe an embedded ``realisation" $\graph$ of $\tilde{\graph}$, the two approaches produce the same spectrum.
In our work, we elect to prescribe an embedding because our examples are motivated by physical structures (see section \ref{sec:Intro}), and it is therefore more convenient, and not to mention much more intuitive, to define a graph along with its embedding into physical space.
However, it should be held in mind that the embedded, periodic quantum graph $\graph$ we chose to represent a given singular structure may not be the only possible realisation of an abstract periodic quantum graph $\tilde{\graph}$.
This means that the spectrum obtained for any given quantum graph (or singular-structure) may coincide with that of multiple others, so long as they also represent an embedding of $\tilde{\graph}$, despite the $M$-matrices appearing to be different.
\tstk{In section \ref{ssec:EmbeddingDependentExample} we provide a concrete example to compliment this discussion.}

\section{Singular Measures} \label{sec:SingularMeasures}
\tstk{introduction, in which we should probably define what we mean by a singular structure, or refer back to where it has already been introduced and properly defined.}

Let 
\begin{align*}
	\ddom:= \left[0,T_1\right)\times\left[0,T_2\right), 
\end{align*}
and $\dddom = \ddom \times \bracs{0,\infty}$.
Let $\graph = \bracs{\vertSet,\edgeSet}$ be an embedded graph in $\ddom$ \tstk{this needs to come after QGs then}, and for each $I_{jk}\in \edgeSet$, define the (Borel) measure $\lambda_{jk}$ as the measure that supports the one-dimensional Lebesgue measure on (or ``along") $I_{jk}$:
\begin{align*}
	\lambda_{jk}\bracs{B} = \lambda_{1}\bracs{r_{jk}^{-1}\bracs{B \cap I_{jk}}},
	&\quad\text{for all Borel } B.
\end{align*}
Here, $\lambda_1$ is the Lebesgue measure on $\reals$, and $r_{jk}$ is the parametrisation of the edge $I_{jk}$ (\tstk{general notation and assumptions}).
Then set $\ddmes$ to be the (Borel) measure defined by
\begin{align*}
	\ddmes\bracs{B} = \sum_{v_j\in \vertSet}\sum_{j\conLeft k} \lambda_{jk}\bracs{B}.
\end{align*}
We refer to $\ddmes$ as the ``singular measure that supports $\graph$"; or alternatively the ``singular measure on $\graph$", or the ``(singular) measure that supports the edges of $\graph$".
For a graph embedded into a 2D domain, the singular measure $\ddmes$ is illustrated in figure \ref{fig:Diagram_SingularMeasure2D}.
\begin{figure}[b!]
	\centering
	\includegraphics[scale=0.85]{Diagram_SingularMeasure2D.pdf}
	\caption{\label{fig:Diagram_SingularMeasure2D} For a graph embedded in $\reals^2$, the $\ddmes$-measure of any Borel set $B$ is obtained from summing the contributions of each $\lambda_{jk}$, as indicated by the thickened lines.
	Sets that do not intersect $\graph$ have zero measure.}
\end{figure}

Now for each $v_j\in\vertSet$ let $\delta_j$ be a point-mass measure centred on $v_j$, namely
\begin{align*}
	\delta_j\bracs{B} &= \begin{cases} 1, & v_j\in B, \\ 0, & v_j\not\in B, \end{cases}
\end{align*}
and let
\begin{align*}
	\nu\bracs{B} = \sum_{v_j\in\vertSet}\alpha_j\delta_j\bracs{B},
	\quad\text{for all Borel } B,
\end{align*}
where each $\alpha_j$ is the coupling constant at the vertex $v_j$, and set $\dddmes = \ddmes + \nu$.
\tstk{here, can mention how ``extruding" the graph into the $x_3$-direction would provide another singular measure on the induced planes --- this is the opposite to what we are doing, using a Fourier transform in the $x_3$-direction so that we can FORGET about the $x_3$-dependence!}
The measures $\dddmes$, $\ddmes$, and $\nu$ will be key to our measure-theoretic formulations, enabling us to establish notions of derivatives on a domains with no interior (or no ``area" in the Lebesgue-sense).

\tstk{again, we are assuming we have established some notion of singular structure layout!}
When we come to consider variational problems on our singular structure, it is helpful to take both a Fourier and Gelfand transform due to the features our singular structure possesses.
The result of these transforms being that the usual gradient operator is (effectively) replaced by the operator $\ktgrad$ in \eqref{eq:PeriodCellCurlCurlStrongForm}.
Clearly, the Fourier transform in the $x_3$ direction introduces the Fourier variable (physically, the ``wavenumber") $\wavenumber$ and replaces derivatives with respect to $x_3$ by multiplication by a constant $\rmi\wavenumber$.
The Gelfand transform then affects the components of the gradient operator which act in the $\bracs{x_1,x_2}$-plane, where we have a periodic structure.
Recall that we pose \eqref{eq:WholeSpaceCurlCurl} on a periodic, embedded graph $\graph$ in $\reals^2$, and the Gelfand transform to move us to a family of problems on the period cell of $\graph$.
As a result, we introduce the quasi-momentum
\begin{align*}
	\qm=\bracs{\qm_1,\qm_2}\in\left[-\frac{\pi}{T_1},\frac{\pi}{T_1}\right)\times\left[-\frac{\pi}{T_2},\frac{\pi}{T_2}\right)
\end{align*}
define the ``shifted" gradient operator $\ktgrad$ on smooth functions $\phi\in\smooth{\ddom}$ by
\begin{align*}
	\ktgrad\phi &= \begin{pmatrix} \partial_1\phi + \rmi\qm_1\phi \\ \partial_2\phi + \rmi\qm_2\phi \\ \rmi\wavenumber\phi \end{pmatrix},
\end{align*}
and the ``shifted" curl operator $\ktcurl{}$ which acts on smooth functions $\Phi=\bracs{\phi_1, \phi_2, \phi_3}^{\top}\in\smooth{\ddom}^3$ as
\begin{align*}
	\ktcurl{}\Phi &= \begin{pmatrix} \bracs{\partial_2 + \rmi\qm_2}\phi_3 - i\wavenumber\phi_2 \\ i\wavenumber\phi_1 - \bracs{\partial_1 + \rmi\qm_1}\phi_3 \\ \bracs{\partial_1 + \rmi\qm_1}\phi_2 - \bracs{\partial_2 + \rmi\qm_2}\phi_1 \end{pmatrix}.
\end{align*}
Note that $\phi$ and $\Phi$ are functions of two variables $\bracs{x_1,x_2}\in\ddom$ --- a Fourier transform removes the dependence on the variable $x_3$, allowing us to work on an embedded graph in the $\bracs{x_1,x_2}$-plane rather than on planes in 3 dimensions corresponding to the ``extrusion" of the graph.

\section{Sobolev Spaces with Respect to Borel Measures} \label{sec:BorelMeasSobSpaces}
In this section we introduce the ``non-classical" Sobolev spaces with respect to Borel measures, the final ingredient that we require to pose variational problems on our singular structures.
Since we will be working with a number of different measures (section \ref{sec:SingularMeasures}) in two  (and occasionally three) dimensions, the content of this section is presented with respect to a general Borel measure $\rho$ defined on a bounded $\ddom\subset\reals^2$, which should be thought of as the period cell of some periodic structure.
Our overarching goal is to construct an analogy to a Sobolev space for the measure $\rho$, and hence obtain a concept of (weak) derivative.
We will predominantly work ``post-Gelfand transform", hence our definitions will involve the shifted gradient ($\ktgrad$) and curl operators ($\ktcurl{}$) operators introduced in section \ref{sec:TP-GelfandTransform}.
Section \ref{ssec:SobSpacesAndGelfand} clarifies the relationship between the spaces we introduce here and the non-classical Sobolev spaces one would consider when studying a periodic problem in $\reals^2$.

Let us begin by defining
\begin{align*}
	W^{\kt}_{\rho,\mathrm{grad}} &:= \overline{\clbracs{ \bracs{\phi, \ktgrad\phi} \setVert \phi\in\psmooth{\ddom} }},
\end{align*}
where the closure is taken in $\ltwo{\ddom}{\rho}\times\ltwo{\ddom}{\rho}^3$, and 
\begin{align*}
	W^{\kt}_{\rho,\mathrm{curl}} &:= \overline{\clbracs{ \bracs{\Phi, \ktcurl{}\Phi} \setVert \Phi\in\bracs{\psmooth{\ddom}}^3 }},
\end{align*}
where the closure is taken in $\ltwo{\ddom}{\rho}^3\times\ltwo{\ddom}{\rho}^3$.
This mimics the ``$H$"-definition of classical Sobolev spaces (when $\rho$ is the Lebesgue measure), however for general measures $\rho$ the Meyers-Serrin ``$H=W$" theorem (see \cite[Theorem 3.17]{adams2003sobolev}) doesn't hold due to the lack of the integration by parts technique. \tstk{but there is a partial integration by parts technique for our S-spaces on graphs, could include in an appendix if we wanted. In fact, it's very useful for our discussion about Zhikov's result!}
We would like to call the second element of a pair $\bracs{u,g}\in W^{\kt}_{\rho,\mathrm{grad}}$ ``the gradient" of the function $u$, however notice that if $\bracs{u,g_1}, \bracs{0, g_2}\in W^{\kt}_{\rho,\mathrm{grad}}$ then we also have that $\bracs{u, g_1+g_2}\in W^{\kt}_{\rho,\mathrm{grad}}$.
As such, the term ``gradient" being used in reference to the second element of a pair in $W^{\kt}_{\rho,\mathrm{grad}}$ does not indicate a unique function --- both $g_1$ and $g_1+g_2$ are ``gradients" of $u$ in this sense.
An analogous deduction can be made for ``curl" and the second member of elements of $W^{\kt}_{\rho,\mathrm{curl}}$.
The non-uniqueness of ``gradients" (in the aforementioned sense) is not a major hindrance \tstk{Zhikov, Kirill \& Serena's homogenisation papers deal with this fine)}, but it requires us to define and study the set of ``($\rho$)-gradients of zero" as
\begin{align*}
	\gradZero{\ddom}{\rho} &= \clbracs{ g\in \ltwo{\ddom}{\rho}^3 \setVert \bracs{0,g}\in W^{\kt}_{\rho,\mathrm{grad}}}, \\
	&= \clbracs{ g\in\ltwo{\ddom}{\rho}^3 \setVert \exists\phi_n\in\psmooth{\ddom} \text{ s.t. } \phi_n \lconv{\ltwo{\ddom}{\rho}}0, \ktgrad\phi_n\lconv{\ltwo{\ddom}{\rho}^3} g }, \labelthis\label{eq:GradZeroSequenceDef}
\end{align*}
and analogously define the set of ``($\rho$)-curls of zero" as
\begin{align*}
	\curlZero{\ddom}{\rho} &= \clbracs{ c\in \ltwo{\ddom}{\rho}^3 \setVert \bracs{0,c}\in W^{\kt}_{\rho,\mathrm{curl}}}, \\
	&= \clbracs{ c\in\ltwo{\ddom}{\rho}^3 \setVert \exists\Phi^n\in\psmooth{\ddom}^3 \text{ s.t. } \Phi^n\lconv{\ltwo{\ddom}{\rho}^3}0, \ \ktcurl{}\Phi\lconv{\ltwo{\ddom}{\rho}^3} c}. \labelthis\label{eq:CurlZeroSequenceDef}
\end{align*}
Given an element $g\in\gradZero{\ddom}{\rho}$, we will refer to a sequence $\phi_n$ as in \eqref{eq:GradZeroSequenceDef} as an ``approximating sequence" for $g$, and will make use of the phrase ``take an approximating sequence $\phi_n$ for $g$" (or similar) to mean ``let $\phi_n$ be a sequence as in \eqref{eq:GradZeroSequenceDef} for the element $g$ of $\gradZero{\ddom}{\rho}$".
We also adopt a similar convention for elements $c\in\curlZero{\ddom}{\rho}$ and ``approximating sequences" as in \eqref{eq:CurlZeroSequenceDef}.
The functions that belong to these sets can be thought of as the functions that are changing (or vector fields that are rotating) in such a way that the measure $\rho$ cannot see these changes, whilst the Lebesgue measure can --- concrete interpretations for the singular measures we study in depth can be found in sections \ref{ssec:3DGradGeometric} and \ref{sec:CC-Geometric}.

One may notice the absence of any $\kt$ labelling on the sets $\gradZero{\ddom}{\rho}$ and $\curlZero{\ddom}{\rho}$, which is due to their independence of these quantities, as demonstrated in proposition \ref{prop:ZeroInvariantUnderQM-Wavenumber} below.
This is not unexpected --- given that $\phi_n$ converging to zero, adding $\phi_n$ multiplied by a combination of the quasi-momentum $\qm$ and Fourier variable $\wavenumber$ is not going to change any behaviour.
\begin{prop} \label{prop:ZeroInvariantUnderQM-Wavenumber}
	Fix a wavenumber $\wavenumber$ and a quasi-momentum $\qm$, and let
	\begin{align*}
		\mathcal{G}_{\ddom,\md\rho}^{\kt}(0) &:= \clbracs{ g\in \ltwo{\ddom}{\rho}^3 \setVert \bracs{0,g}\in W^{\kt}_\mathrm{grad} }, \\
		\gradZero{\ddom}{\rho} &:= \clbracs{ g\in \ltwo{\ddom}{\rho}^3 \setVert \bracs{0,g}\in W^{\bracs{0, 0}}_\mathrm{grad} }, \\
		\mathcal{C}_{\ddom,\md\rho}^{\kt}(0) &:= \clbracs{ c\in \ltwo{\ddom}{\rho}^3 \setVert \bracs{0,c}\in W^{\kt}_\mathrm{curl} }, \\
		\curlZero{\ddom}{\rho} &:= \clbracs{ g\in \ltwo{\ddom}{\rho}^3 \setVert \bracs{0,g}\in W^{\bracs{0, 0}}_\mathrm{curl} }.
	\end{align*}
	Then the following sets are equal:
	\begin{align*}
		\mathcal{G}_{\ddom,\md\rho}^{\kt}(0) &= \gradZero{\ddom}{\rho}, \\
		\mathcal{C}_{\ddom,\md\rho}^{\kt}(0) &= \curlZero{\ddom}{\rho}.
	\end{align*}
\end{prop}
\begin{proof}
	This is seen by observing that for $\phi\in\psmooth{\ddom}$ and $\Phi\in\psmooth{\ddom}^3$,
	\begin{align*}
		\grad^{\kt}\phi &= \grad^{\bracs{0, 0}}\phi + \rmi\wavenumber\begin{pmatrix} 0 \\ 0 \\ \phi \end{pmatrix} + \rmi\begin{pmatrix} \qm_1 \\ \qm_2 \\ 0 \end{pmatrix}\phi, \\
		\grad^{\kt}\wedge\Phi &= \grad^{\bracs{0, 0}}\wedge\Phi + \rmi\wavenumber\begin{pmatrix} 0 \\ 0 \\ 1 \end{pmatrix}\wedge\Phi + \rmi\begin{pmatrix} \qm_1 \\ \qm_2 \\ 0 \end{pmatrix} \wedge \Phi.
	\end{align*}
	Thus, if $g\in \mathcal{G}_{\ddom,\md\rho}^{\kt}(0)$ there exists a sequence $\phi_n\in\smooth{\ddom}$ such that
	\begin{align*}
		\phi_n \lconv{\ltwo{\ddom}{\rho}} 0, &\quad \ktgrad\phi_n \lconv{ \ltwo{\ddom}{\rho}^3 } g,
	\end{align*}
	(as in \eqref{eq:GradZeroSequenceDef}).
	But $\rmi\bracs{\qm_1, \qm_2, 0}^{\top}\phi_n\rightarrow 0$, as does $\rmi\wavenumber\phi_n$.
	Given the formulae above, we must also have that $\grad^{\bracs{0, 0}}\phi_n\rightarrow g$, so $g\in \gradZero{\ddom}{\rho}$.
	The reverse implication, and the proof for $\rho$-curls of zero, is similar.
\end{proof}

Another useful property of the sets $\gradZero{\ddom}{\rho}$ and $\curlZero{\ddom}{\rho}$ is that they are closed, linear subspaces of $\ltwo{\ddom}{\rho}^3$.
\begin{prop}
	The sets $\gradZero{\ddom}{\rho}$ and $\curlZero{\ddom}{\rho}$ are closed, linear subspaces of $\ltwo{\ddom}{\rho}^3$.
\end{prop}
\begin{proof}
	We only present the proof for $\gradZero{\ddom}{\rho}$, since the proof for $\curlZero{\ddom}{\rho}$ is analogous.
	
	The fact that $\gradZero{\ddom}{\rho}$ is a subspace follows from its definition; if $g, h\in\gradZero{\ddom}{\rho}$ take approximating sequences $\phi_n$, $\psi_n$ for $g,h$ respectively.
	Then for any $\alpha\in\complex$, we have that $\alpha\phi_n + \psi_n\in\psmooth{\ddom}$ too, and clearly
	\begin{align*}
		\alpha\phi_n + \psi_n &\lconv{\ltwo{\ddom}{\rho}} \alpha\times 0 + 0 = 0, \\
		\ograd\bracs{\alpha\phi_n + \psi_n} &= \alpha\ograd\phi_n + \ograd\psi_n
		\lconv{\ltwo{\ddom}{\rho}^3} \alpha g + h,
	\end{align*}
	so $\alpha g + h\in\gradZero{\ddom}{\rho}$.
	
	Next we prove that $\gradZero{\ddom}{\rho}$ is closed, through a diagonal argument.
	To this end, let $g_n$ be a sequence in $\gradZero{\ddom}{\rho}$ that converges (in $\ltwo{\ddom}{\rho}^3$) to some function $g$.
	For each $n\in\naturals$, there exists an approximating sequence of (periodic) smooth functions $\phi_n^l$ for $g_n$ since each $g_n\in\gradZero{\ddom}{\rho}$.
	Now, let $m\in\naturals$.
	Since $g_n\rightarrow g$, there exists $N_m\in\naturals$ such that $\norm{g_n-g}_{\ltwo{\ddom}{\rho}^3}<\recip{2m}$ for all $n\geq N_m$ --- in particular, $g_{N_m}$ satisfies this inequality.
	Then, since $\phi_{N_m}^l$ is an approximating sequence for $g_{N_m}$, we have that there exist $L_{N_m}^{(1)}, L_{N_m}^{(2)}\in\naturals$ such that
	\begin{align*}
		\norm{\phi_{N_m}^l}_{\ltwo{\ddom}{\rho}} < \recip{m}, &\quad\forall l\geq L_{N_m}^{(1)}, \\
		\norm{\ograd\phi_{N_m}^l - g_{N_m}}_{\ltwo{\ddom}{\rho}^3} < \recip{2m}, &\quad\forall l\geq L_{N_m}^{(2)}.
	\end{align*}	 
	Set $L_{N_m} = \max\clbracs{ L_{N_m}^{(1)}, L_{N_m}^{(2)} }$, and define $\psi_m = \phi^{L_{N_m}}_{N_m}$ for each $m$.
	Then we have that
	\begin{align*}
		\norm{\psi_m}_{\ltwo{\ddom}{\rho}} &= \norm{\phi_{N_m}^{L_{N_m}}}_{\ltwo{\ddom}{\rho}} < \recip{m}, \\
		\norm{\ograd\psi_m - g}_{\ltwo{\ddom}{\rho}^3} &= \norm{\ograd\phi_{N_m}^{L_{N_m}} - g}_{\ltwo{\ddom}{\rho}^3} \\
		&\leq \norm{\ograd\phi_{N_m}^{L_{N_m}} - g_{N_m}} + \norm{g_{N_m} - g}_{\ltwo{\ddom}{\rho}^3} \\
		&< \recip{2m} + \recip{2m} = \recip{m}.
	\end{align*}
	Therefore,
	\begin{align*}
		\psi_m \lconv{\ltwo{\ddom}{\rho}} 0, \quad \ograd\psi_m \lconv{\ltwo{\ddom}{\rho}^3} g,
	\end{align*}
	and each $\psi_m\in\psmooth{\ddom}$, and so $g\in\gradZero{\ddom}{\rho}$.
\end{proof}

Since $\gradZero{\ddom}{\rho}$ is a closed, linear subspace of $\ltwo{\ddom}{\rho}^3$ we have the direct sum decomposition
\begin{align*}
	\ltwo{\ddom}{\rho}^3 = \gradZero{\ddom}{\rho}^{\perp} \oplus \gradZero{\ddom}{\rho}.
\end{align*}
Now suppose that we have $\bracs{u,g_1}, \bracs{u,g_2}\in W^{\kt}_{\rho,\mathrm{grad}}$ with $g_1, g_2\in\gradZero{\ddom}{\rho}^\perp$.
This implies there exist approximating sequences $\phi_n, \psi_n$ such that
\begin{align*}
	\phi_n \lconv{\ltwo{\ddom}{\rho}} u, &\quad \psi_n \lconv{\ltwo{\ddom}{\rho}} u, \\
	\ktgrad\phi_n \lconv{\ltwo{\ddom}{\rho}^3} g_1, &\quad \ktgrad\psi_n \lconv{\ltwo{\ddom}{\rho}^3} g_2.
\end{align*}
Since $g_1, g_2\in \gradZero{\ddom}{\rho}^{\perp}$, we have that $g_1-g_2\in \gradZero{\ddom}{\rho}^{\perp}$ too.
However, 
\begin{align*}
	\phi_n - \psi_n \lconv{\ltwo{\ddom}{\rho}} 0, &\quad \ktgrad\bracs{\phi_n-\psi_n} \lconv{\ltwo{\ddom}{\rho}^3} g_1 - g_2,
\end{align*}
so $g_1 - g_2\in \gradZero{\ddom}{\rho}$ too --- therefore we must have that $g_1=g_2$.
With this in mind, we see that each $u$ possesses a unique gradient \emph{that is orthogonal to} $\gradZero{\ddom}{\rho}$.
This allows construction of the \emph{non-classical} Sobolev space of gradients with respect to the measure $\rho$:
\begin{definition}[Sobolev Space of $\kt$-Gradients] \label{def:3DGradSobSpace}
	The set
	\begin{align*}
		\ktgradSob{\ddom}{\rho} &= \clbracs{ \bracs{u, \ktgrad_{\rho}u}\in W^{\kt}_{\rho,\mathrm{grad}} \setVert \ktgrad_{\rho}u \perp \gradZero{\ddom}{\rho} },
	\end{align*}
	is called the \emph{Sobolev space of $\kt$-gradients with respect to the measure $\rho$}.
	We call the member $\ktgrad_{\rho}u$ of the pair $\bracs{u, \ktgrad_{\rho}u}\in\ktgradSob{\ddom}{\rho}$ the \emph{$\kt$-tangential gradient of $u$ with respect to $\rho$}.
	When the context is clear, we will simply refer to $\ktgrad_{\rho}u$ as the \emph{tangential gradient} or \emph{$\kt$-tangential gradient}.
	Additionally, it is enough to specify the first member $u$ of the pair $\bracs{u, \ktgrad_{\rho}u}$ due to the uniqueness of $\kt$-tangential gradients, so we will often write $u\in\ktgradSob{\ddom}{\rho}$ as shorthand.
	\tstk{We can equip $\ktgradSob{\ddom}{\rho}$ with the inner product and norm
	\begin{align*}
		\ip{u}{v}_{\ktgradSob{\ddom}{\rho}} 
		&= \integral{\ddom}{u\overline{v}}{\rho}
		+ \integral{\ddom}{\ktgrad_{\rho}u \cdot \overline{\ktgrad_{\rho}v}}{\rho}, \\
		\norm{u}_{\ktgradSob{\ddom}{\rho}}^2 &= \ip{u}{u}_{\ktgradSob{\ddom}{\rho}}
		= \norm{u}_{\ltwo{\ddom}{\rho}}^2 + \norm{\ktgrad_{\rho}u}_{\ltwo{\ddom}{\rho}^3}^2,
	\end{align*}		
	to make it a Hilbert space.}
\end{definition}

We can also perform analogous steps for $\curlZero{\ddom}{\rho}$, which allows us to define the \emph{Sobolev space of curls} with respect to the measure $\rho$:
\begin{definition}[Sobolev Space of $\kt$-Curls] \label{def:CurlSobSpace}
	The set
	\begin{align*}
		\ktcurlSob{\ddom}{\rho} &= \clbracs{ \bracs{u, \ktcurl{\rho}u}\in W^{\kt}_{\rho,\mathrm{curl}} \setVert \ktcurl{\rho}u \perp \curlZero{\ddom}{\rho} },
	\end{align*}
	is called the \emph{Sobolev space of $\kt$-curls with respect to the measure $\rho$}.
	The member $\ktcurl{\rho}u$ of the pair $\bracs{u, \ktcurl{\rho}u}$ is the \emph{$\kt$-tangential curl of $u$ with respect to $\rho$}.
\end{definition}
Like with tangential gradients; we will use the shorthand $u\in\ktcurlSob{\ddom}{\rho}$ to refer to the element $\bracs{u, \ktcurl{\rho}u}\in\ktcurlSob{\ddom}{\rho}$, since each $u$ has a unique $\ktcurl{\rho}u$ that is orthogonal to $\curlZero{\ddom}{\rho}$, and when the context is clear will refer to $\ktcurl{\rho}u$ as the \emph{tangential curl} or \emph{$\kt$-tangential curl}.
\tstk{norm as well?}

%Having now assigned a meaning to (tangential) gradients and curls for the measure $\rho$, we can also define what it means for a vector field $u$ to be $\kt$-divergence-free with respect to $\rho$.
%\begin{definition}[$\kt$-divergence-free] \label{def:ktDivFree}
%	A vector field $u\in\ltwo{\ddom}{\rho}^3$ is said to be \emph{$\kt$-divergence-free} (with respect to $\rho$) if
%	\begin{align*}
%		0 &= \integral{\ddom}{ u\cdot\overline{g} }{\rho}, \qquad\forall\bracs{v,g}\in W^{\kt}_{\rho,\mathrm{grad}}.
%	\end{align*}
%\end{definition}
%We have setup divergence-free vector fields to be those that are orthogonal (in $\ltwo{\ddom}{\rho}^3$) to all gradients of $\ltwo{\ddom}{\rho}$-functions, including gradients of zero.
%\tstk{Helmholtz decomp stuff, L2\_Decomposition.pdf. Might go at end with the other div-free result? Also might want to save this for the Maxwell chapter discussion to flesh out that chapter, as it's a focal point there (even though we can prove the statements in general for $\rho$).}

Having provided the definition of the Sobolev spaces that we will later be considering, we provide some simple results which will prove useful throughout our analysis.
By construction, functions with tangential gradients or curls can also be approximated by smooth functions.
\begin{cory} \label{cory:SobSpaceApproxSequences}
	\begin{align*}
		\ktgradSob{\ddom}{\rho} &= \clbracs{ \bracs{u, \ktgrad_{\rho}u} \setVert \exists\phi_n\in\psmooth{\ddom} \text{ s.t. } \right. \\ 
		&\hspace{0.25\textwidth} \left. \phi_n\lconv{\ltwo{\ddom}{\rho}}u, \ \ktgrad\phi_n\lconv{\ltwo{\ddom}{\rho}^3}\ktgrad_{\rho}u }, \\
		\ktcurlSob{\ddom}{\rho} &= \clbracs{ \bracs{u, \ktcurl{\rho}u} \setVert \exists\Phi^n\in\psmooth{\ddom}^3 \text{ s.t. } \right. \\
		&\hspace{0.25\textwidth} \left. \Phi^n\lconv{\ltwo{\ddom}{\rho}^3}u, \ \ktcurl{}\Phi^n\lconv{\ltwo{\ddom}{\rho}^3}\ktgrad_{\rho}u }.
	\end{align*}
\end{cory}
\begin{proof}
	This is a direct consequence of the construction of $W^{\kt}_{\rho,\mathrm{grad}}$ and $W^{\kt}_{\rho,\mathrm{curl}}$.
\end{proof}
We will adopt analogous terminology and phrasing when using sequences $\phi_n$ and $\Phi^n$ as they appear in corollary \ref{cory:SobSpaceApproxSequences} to that used for sequences in \eqref{eq:GradZeroSequenceDef}-\eqref{eq:CurlZeroSequenceDef} regarding gradients and curls of zero --- referring to them as \emph{approximating sequences} for elements of $\ktgradSob{\ddom}{\rho}$ or $\ktcurlSob{\ddom}{\rho}$.

Given our choice to construct (tangential) gradients and curls via approximation, we retain the rule that the curl of a gradient is 0 --- once translated into its analogue for gradients and curls with respect to the measure $\rho$.
\begin{lemma} \label{lem:CurlOfGradSmoothFunctions}
	For $\phi\in\psmooth{\ddom}$, we have that
	\begin{align*}
		\ktcurl{}\bracs{\ktgrad\phi} = 0.
	\end{align*}
\end{lemma}
\begin{proof}
	Let $\phi\in\psmooth{\ddom}$, $\Phi = \ktgrad\phi$, and set $\widetilde{\qm} = \bracs{\qm_1, \qm_2, 0}^\top$ throughout.
	Notice that
	\begin{align*}
		\Phi = \ktgrad\phi &= \e^{-\rmi\qm\cdot x}\kgrad\bracs{ \e^{\rmi\qm\cdot x}\phi },
	\end{align*}
	so with $\psi = \e^{\rmi\qm\cdot x}\phi$, we have that
	\begin{align*}
		\ktcurl{}\Phi 
		&= \kgrad\wedge\bracs{ \Phi } + \rmi\widetilde{\qm}\wedge\Phi
		= -\rmi\widetilde{\qm}\wedge\Phi + \e^{-\rmi\qm\cdot x}\kgrad\wedge\psi + \rmi\widetilde{\qm}\wedge\Phi \\
		&= \e^{-\rmi\qm\cdot x}\kgrad\wedge\psi
		= \e^{-\rmi\qm\cdot x} 
		\begin{pmatrix}
			\partial_2\bracs{ \rmi\wavenumber \e^{\rmi\qm\cdot x}\phi } - \rmi\wavenumber\partial_2\bracs{ \e^{\rmi\qm\cdot x}\phi } \\
			\rmi\wavenumber\partial_1\bracs{ \e^{\rmi\qm\cdot x}\phi } - \partial_1\bracs{ \rmi\wavenumber \e^{\rmi\qm\cdot x}\phi } \\
			\partial_1\bracs{ \partial_2\bracs{\rmi\wavenumber\e^{\rmi\qm\cdot x}\phi} } - \partial_2\bracs{ \partial_1\bracs{\rmi\wavenumber\e^{\rmi\qm\cdot x}\phi} }
		\end{pmatrix} \\
		&= 0.
	\end{align*}
\end{proof}

Since our definitions of $\gradZero{\ddom}{\rho}, \curlZero{\ddom}{\rho}$ and tangential gradients and curls all depend on approximations by smooth functions, lemma \ref{lem:CurlOfGradSmoothFunctions} informs us that adding a gradient to an existing vector field will not change the curl of said vector field.
\begin{prop} \label{prop:CurlIgnoresGradients}
	Suppose that $u\in\ktcurlSob{\ddom}{\rho}$, $g\in\gradZero{\ddom}{\rho}$, $w\in\ktgradSob{\ddom}{\rho}$ and set $v = u + g + w\in\ltwo{\ddom}{\rho}^3$. 
	Then $\bracs{v, \ktcurl{\rho}u}\in\ktcurlSob{\ddom}{\rho}$ --- that is, $v$ has a tangential $\kt$-curl equal to that of $u$.
\end{prop}
\begin{proof}
	Take an approximating sequence $\Phi^n$ for $u$ as in corollary \ref{cory:SobSpaceApproxSequences}, $\phi_n$ for $w$ as in  corollary \ref{cory:SobSpaceApproxSequences}, and $\psi_n$ for $g$ as in \eqref{eq:GradZeroSequenceDef}.
	Define $\varphi^n := \Phi^n + \ktgrad\phi_n + \ktgrad\psi_n\in\psmooth{\ddom}^3$ for each $n\in\naturals$.
	Given lemma \ref{lem:CurlOfGradSmoothFunctions}, we have that $\ktcurl{}\varphi^n = \ktcurl{}\Phi^n$ for every $n\in\naturals$.
	Therefore, we have that
	\begin{align*}
		\varphi^n &\lconv{\ltwo{\ddom}{\rho}^3} u + w + g = v, \\
		\ktcurl{}\varphi^n &= \ktcurl{}\Phi^n 
		\lconv{\ltwo{\ddom}{\rho}^3} \ktcurl{\rho}u.
	\end{align*}
	Thus, we conclude that $\bracs{v, \ktcurl{\rho}u}\in W^{\kt}_{\rho, \mathrm{curl}}$.
	Since $\ktcurl{\rho}u\in\curlZero{\ddom}{\rho}^\perp$, we must conclude that $\bracs{ v, \ktcurl{\rho}u }\in\ktcurlSob{\ddom}{\rho}$ too.
\end{proof}
In particular, proposition \ref{prop:CurlIgnoresGradients} informs us that the $\kt$-tangential curl of either a $\kt$-tangential gradient, or a gradient of zero, is zero.
%
%Another remark that deserves to be made is that given the construction of $W^{\kt}_{\rho,\mathrm{grad}}$, it is sufficient to only test orthogonality of $u$ against smooth gradients when checking if a field is divergence free.
%\begin{cory} \label{cory:DivFreeSufficient}
%	If
%	\begin{align*}
%		0 &= \integral{\ddom}{ u\cdot\overline{\ktgrad\phi} }{\rho}, \qquad\forall\phi\in\psmooth{\ddom},
%	\end{align*}
%	then $u$ is $\kt$-divergence free with respect to $\rho$.
%\end{cory}
%\begin{proof}
%	If $\bracs{v,g}\in W^{\kt}_{\rho,\mathrm{grad}}$ take an approximating sequence as in \eqref{eq:GradZeroSequenceDef}.
%	Then clearly
%	\begin{align*}
%		\integral{\ddom}{ u\cdot\overline{g} }{\rho} 
%		&= \lim_{n\rightarrow\infty}\integral{\ddom}{ u\cdot\overline{\ktgrad\phi_n} }{\rho}
%		= \lim_{n\rightarrow\infty} 0 = 0.
%	\end{align*}
%\end{proof}
%Corollary \ref{cory:DivFreeSufficient} does mean that definition \ref{def:ktDivFree} could be weakened, however we chose to present the definition of divergence free in this manner to maintain parallels to the usual intuition of a divergence free field being orthogonal to \emph{all} gradients.

\subsection{Other Sobolev Spaces, and the image under the Gelfand Transform} \label{ssec:SobSpacesAndGelfand}
One can define analogous ``non-classical" Sobolev spaces on $\reals^d$ (or subsets thereof) through an analogous procedure to the above; let $\upsilon$ be a Borel measure on $\reals^d$ and define
\begin{align*}
	W_{\upsilon,\mathrm{grad}} := \overline{ \clbracs{ \bracs{\phi, \grad\phi} \setVert \phi\in\csmooth{\reals^d} } },
\end{align*}
with the closure taken in $\ltwo{\reals^d}{\upsilon}\times\ltwo{\reals^d}{\upsilon}^d$.
One can again define a set of gradients of zero, demonstrate that it forms a closed linear subspace of $\ltwo{\reals^d}{\upsilon}^d$, and construct the Sobolev space $\gradSob{\reals^d}{\upsilon}$ consisting of functions $u$ and their tangential gradients $\grad_{\upsilon}u$ --- for further information, the interested reader can see \tstk{Zhikov, with section ref}.
We can also define the Sobolev space $\tgradSob{\ddom}{\rho}$ via the same procedure by first considering the closure of pairs $\bracs{\phi, \tgrad\phi}, \phi\in\psmooth{\ddom}$ in $\ltwo{\ddom}{\rho}\times\ltwo{\ddom}{\rho}^2$.

Upon recalling section \ref{sec:TP-GelfandTransform}, the reader may notice that our notation and construction of the Sobolev spaces $\ktgradSob{\ddom}{\rho}$ and $\ktcurlSob{\ddom}{\rho}$ is heavily suggestive of being the result of the combination of a Gelfand transform in the first two coordinate directions and a Fourier transform in the third.
Indeed, our choice to use periodic smooth functions and the shifted gradient operator to construct these spaces was not a coincidence.
Now suppose that $\hat{\graph}$ is a periodic graph embedded into $\reals^2$ with period cell $\ddom$, and let $\upsilon$ be one of the singular measures detailed in section \ref{sec:SingularMeasures} with respect to the graph $\hat{\graph}$.
Let $\graph$ be the period graph of $\hat{\graph}$ and let $\rho$ be the ``restriction" of the measure $\upsilon$ to $\ddom$.
Finally, consider the extrusion $\dddom:=\hat{\graph}\times[0,\infty)\subset\reals^3$ equipped with the product measure $\upsilon\times\lambda_1$, and the image of the set
\begin{align*}
	\clbracs{ \bracs{\phi, \grad\phi} \setVert \phi\in\csmooth{\dddom} }
\end{align*}
under a Gelfand transform in the $\bracs{x_1,x_2}$-plane and a Fourier transform in the $x_3$ plane.
For given quasi-momentum $\qm$ and Fourier variable (wavenumber) $\kappa$, the resulting functions belong to the set
\begin{align*}
	\clbracs{ \bracs{\phi, \ktgrad\phi} \setVert \phi\in\psmooth{\ddom} }.
\end{align*}
By virtue of how we chose to construct the space $\gradSob{\dddom}{\bracs{\upsilon\times\lambda_1}}$, we have that the image of a function $u\in\gradSob{\dddom}{\bracs{\upsilon\times\lambda_1}}$ will be some $\hat{u}\in\ktgradSob{\ddom}{\rho}$.
Similar conclusions can be drawn for the spaces $\curlSob{\dddom}{\bracs{\upsilon\times\lambda_1}}$ and $\ktcurlSob{\ddom}{\rho}$ and the spaces $\gradSob{\reals^2}{\upsilon}$ and $\tgradSob{\ddom}{\rho}$.