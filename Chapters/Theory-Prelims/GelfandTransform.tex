\section{The Gelfand Transform} \label{sec:TP-GelfandTransform}
We will begin with a more detailed overview of the Gelfand transform, for our study of (the spectrum of) variational problems on periodic singular structures.
Our review is restricted to the key ideas and results that we will require in the later chapters, and as such should not be considered a complete introduction to the subject.
For a more complete introduction to the Floquet-Bloch theory of periodic differential operators (under which the Gelfand transform falls), one should consult a source such as \cite[section 7.3]{kuchment2001mathematics} and the references provided therein.

Let $\ddom=[0,1)^d$  and construct the \emph{dual cell} or \emph{Brillouin zone} to $\ddom$, $B=[-\pi,\pi)^d$.
The \emph{Gelfand transform} of a function $u\in\csmooth{\reals^d}$ can be defined as
\begin{align*}
	\gelfand u\bracs{x,\qm} := \sum_{m\in\integers^d}u(x+m)\e^{\rmi\qm\cdot(x+m)}.
\end{align*}
The parameter $\qm\in B$ is called the \emph{quasi-momentum} --- it is the analogue of the dual variable $\kappa$ introduced when taking the Fourier transform of a function.
The function $\gelfand u$ is $\ddom$-periodic in $x$, and satisfies the equality
\begin{align} \label{eq:GelfandTransform}
	\gelfand u\bracs{x,\qm+2\pi m'} &= \e^{2\pi m'\rmi\cdot x}\gelfand u\bracs{x,\qm},
\end{align}
for any $m'\in\integers^d$, so provided we know $\gelfand u$ on $\ddom\times B$, we can reconstruct $\gelfand u$ for any $x\in\reals^d$.
In addition, for each $u\in\csmooth{\reals^d}$ and $\qm\in B$, the function $\gelfand u\bracs{\cdot,\qm}$ belongs to the space of smooth functions on the torus --- this new domain is compact, which is central to the utility of this transform.
It is simple to see that $\gelfand u\bracs{\cdot,\qm}$ can also be thought of as an element of $\psmooth{\ddom}$, the space of smooth, $\ddom$-periodic functions on $\reals^d$.
The formula for the inverse transform of $\gelfand$ is
\begin{align*}
	\gelfand^{-1}v (x) := \integral{B}{\e^{-\rmi\qm\cdot x}v\bracs{x,\qm}}{\qm},
\end{align*}
where for each $\qm$, $v\bracs{\cdot,\qm}\in\psmooth{\ddom}$.
One can demonstrate (using density arguments) that $\gelfand$ can be extended to an isometry from $L^2\bracs{\reals^d}$ to $L^2\bracs{\ddom}$-valued functions on $B$ \cite[theorem 7.3]{kuchment2001mathematics}.

Let us now examine how the Gelfand transform interacts with differential operators.
Notice that, if $\grad_x$ denotes the usual gradient operator in $\reals^d$ and $u(x)$ is a differentiable function, we have that
\begin{align*}
	\grad_x\gelfand u\bracs{x,\qm} 
	&= \sum_{m\in\integers^d} \e^{\rmi\qm\cdot\bracs{x+m}}\grad_x u(x+m) + \rmi\qm\e^{\rmi\qm\cdot\bracs{x+m}}u(x+m) \\
	&= \sum_{m\in\integers^d} \bracs{\grad_x u + \rmi\qm u}(x+m)\e^{\rmi\qm\cdot\bracs{x+m}}
	= \sqbracs{\gelfand\bracs{\grad_x+\rmi\qm}u}(x,\qm).
\end{align*}
That is, the Gelfand transform induces a shift in the gradient operator.
We have used the subscript $x$ to emphasise that the gradient is taken with respect to the variable $x$, however will drop this additional notation upon completion of this section.
For a general differential operator $\mathcal{A} = \mathcal{A}\bracs{\grad_x}$ with $\ddom$-periodic coefficients acting on functions defined in $\reals^d$, we can deduce that
\begin{align*}
	\sqbracs{\gelfand\bracs{\mathcal{A}\bracs{\grad_x}u}}(x,\qm) 
	&= \mathcal{A}\bracs{\grad_x + \rmi\qm}\gelfand u\bracs{x,\qm}.
\end{align*}
Denote by $\mathcal{A}_{\qm}$ the operator $\mathcal{A}\bracs{\grad+\rmi\qm}$ for each $\qm\in B$ --- the domain $D_{\mathcal{A}}$ of these operators is the space of $\ddom$-periodic functions, identifiable with the space of functions defined on the torus.
The Gelfand transform has effectively ``expanded" the operator $\mathcal{A}$ which acted on (functions on) $\reals^d$ into the family of operators $\mathcal{A}_{\qm}$ parametrised by $\qm$.
Typically one sees $\mathcal{A}$ written as a ``direct integral" of operators
\begin{align*}
	\mathcal{A} &= \int_{B}^{\bigoplus}\mathcal{A}_{\qm} \ \md\qm,
\end{align*}
the notation for which is discussed in \cite{reed1978iv}, and the operators $\mathcal{A}_{\qm}$ are referred to as the \emph{fibres} of $\mathcal{A}$.
In the interest of avoiding technical details that are tangential to our work, but still providing the reader with an idea of the purpose of this transform, let us draw analogy with the use of the Fourier transform in the case of constant coefficients.
If $\mathcal{L}\bracs{\grad}$ is an operator with constant coefficients, that defines a system of differential equations, the natural course of action when studying the problem $\mathcal{L}\bracs{\grad}u=\lambda u$ is to apply a Fourier transform.
This turns the action of $\mathcal{L}\bracs{\grad}$ into multiplication by a (self-adjoint) matrix $L(\wavenumber)$ in the dual variable $\wavenumber$, resulting in the equation $L\bracs{\wavenumber}\hat{u}\bracs{\wavenumber}=\lambda\hat{u}\bracs{\wavenumber}$ (with hats denoting the Fourier transform).
The eigenvalues $\lambda$ of our problem in real space are then the union of the eigenvalues $\lambda_i\bracs{\wavenumber}$ of the matrix $L\bracs{\wavenumber}$, over the range of $\wavenumber$.
The direct integral expansion is analogous; except we will be looking for the eigenvalues $\lambda_i(\qm)$ of a family of operators $\mathcal{A}_{\qm}$ parametrised by the quasi-momentum, rather than a family of eigenvalues $\lambda_i(\wavenumber)$ of matrices $L(\wavenumber)$ parametrised by $\wavenumber$. 
Consequentially, the key result of the Gelfand transform is that the spectrum of $\mathcal{A}$ is formed from the union of the spectra of $\mathcal{A}_{\qm}$ over $\qm$,
\begin{align*}
	\sigma\bracs{\mathcal{A}} = \bigcup_{\qm\in B}\sigma\bracs{\mathcal{A}_{\qm}}.
\end{align*}
The space $D_{\mathcal{A}}$ consists of functions defined on a compact domain (the torus), and so provided that the $\mathcal{A}_{\qm}$ satisfy suitable ellipticity conditions, each of these operators will have compact resolvent, and thus a discrete spectrum of eigenvalues \cite[section 7.3]{kuchment2001mathematics}.
This allows us to order the eigenvalues $\lambda_i\bracs{\qm}$ of $\mathcal{A}_{\qm}$ in ascending order, and when viewed as functions of $\qm$, these $\lambda_i$ are called \emph{dispersion branches}, or individual branches of the \emph{dispersion relations}.
They are also continuous in $\qm$, and thus allow the spectrum of the original operator $\mathcal{A}$ to be written as
\begin{align*}
	\sigma\bracs{\mathcal{A}} &= \bigcup_{i\in\naturals}\sqbracs{ \min_{\qm}\lambda_i, \max_{\qm}\lambda_i },
\end{align*}
where the interval $\sqbracs{ \min_{\qm}\lambda_i, \max_{\qm}\lambda_i }$ is labelled as the $i^{\text{th}}$ spectral band of $\mathcal{A}$.
With the dispersion branches being continuous functions of $\qm$, there is also the useful result that if $B^*$ is a dense subset of $B$,
\begin{align} \label{eq:TP-DenseQMSubsetSuffices}
	\sigma\bracs{\mathcal{A}} &= \bigcup_{i\in\naturals}\overline{\lambda_i\bracs{B^*}}
	= \overline{\bigcup_{\qm\in B^*}\sigma\bracs{\mathcal{A}_{\qm}}}.
\end{align}
This is to say, not all values of the quasi-momentum have to be considered when computing the spectrum of the original operator $\mathcal{A}$ \cite[section 7.4]{kuchment2001mathematics}, which is something that we can exploit in our discussion in section \ref{ssec:ApproachConsiderations}.

The Gelfand transform will be an incredibly useful tool for us in our analysis of variational problems on periodic, singular structures.
To this end, we introduce some shorthand for the ``$\qm$-shifted" gradient operator that the transform introduces, in (subregions of) $\reals^2$ we denote
\begin{align*}
	\tgrad = \grad + \rmi\qm 
	= \begin{pmatrix} \partial_1 + \rmi\qm_1 \\ \partial_2 + \rmi\qm_2 \end{pmatrix},
\end{align*}
which acts analogously to the usual gradient operator $\grad$.
In chapter \ref{ch:CurlCurl}, we will be considering a singular structure that has been extruded into three dimensions, and so will also need to consider the operator
\begin{align*}
	\ktgrad = 
	\begin{pmatrix} \partial_1 + \rmi\qm_1 \\ \partial_2 + \rmi\qm_2 \\ \rmi\wavenumber \end{pmatrix},
\end{align*}
which is the result of a Gelfand transform in the $\bracs{x_1,x_2}$-plane and a Fourier transform in the $x_3$ (extruded) direction.
At times, we may also use the operator $\kgrad = \bracs{\partial_1, \partial_2, \rmi\wavenumber}^\top$.
We also denote by $\ktcurl{}$ the ``curl" operator obtained after applying the Gelfand transform, so formally
\begin{align*}
	\ktcurl{}\phi &= \ktgrad\wedge\phi
	= 
	\begin{pmatrix}
		\bracs{\partial_1+\rmi\qm_1}\phi_3 - \rmi\wavenumber\phi_1 \\
		\rmi\wavenumber\phi_2 - \bracs{\partial_2+\rmi\qm_2}\phi_3 \\
		\bracs{\partial_2+\rmi\qm_2}\phi_1 - \bracs{\partial_1+\rmi\qm_1}\phi_2
	\end{pmatrix}
\end{align*}
for a suitably differentiable vector-valued function $\phi$, and $\wedge$ denotes the vector-cross product.

Before we continue, it is worth mentioning here that an alternative to the Gelfand transform, the \emph{Floquet transform} is also widely used in the study of periodic differential equations.
Whilst these transforms serve essentially the same purpose in aiding the analysis of differential operators with periodic coefficients, there is a subtle difference in the manner in which they do this.
Under the Floquet transform, one obtains a similar ``direct integral" representation of the original operator $\mathcal{A}$ parametrised by $\qm$, however under this representation the ``decomposed" operators $\mathcal{A}_{\qm}$ have the same differential action, but act on different function spaces.
By contrast, the Gelfand transform elects to preserve the underlying function spaces at the expense of an alteration to the action of the differential operators.
Which of the two transforms is used typically comes down to personal preference in most cases, and there are even places in the literature where the two names are used interchangeably.
Throughout this work, we will continue to use the Gelfand transform as introduced in this section through equation \eqref{eq:GelfandTransform}.