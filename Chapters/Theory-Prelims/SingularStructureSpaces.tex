\section{Sobolev Spaces with Respect to Borel Measures} \label{sec:BorelMeasSobSpaces}
\tstk{introduction fix, from curl paper: \newline
In this section we address the question of how one understands the equations \eqref{eq:WholeSpaceLaplaceEqn} and \eqref{eq:PeriodCellLaplaceStrongForm}, by introducing the relevant differential operators and function spaces.
Attempting to pose a boundary-value problem on a singular-structure, by drawing analogy to the ``ingredients" of a boundary-value problem on a thin structure, runs into problems.
These are due to the singular structure lacking a domain interior from the perspective of the space it is embedded in, so the notion of boundary values ceases to make sense.
This issue is resolved by not abandoning what we believe are the ingredients of a boundary-value problem, but rather by reworking our concepts of integration and differentiation so that they respect the fact that we are looking at a problem on the singular structure itself.
As we will be working with each of the measures $\dddmes, \ddmes$, and $\nu$ above individually before combining our knowledge of each, what we present here is stated in terms of a generic measure $\rho$.
}

The overarching idea behind this procedure is for us to construct an analogy to a Sobolev space for the measure $\rho$, and hence obtain a concept of (weak) derivative.
Define
\begin{align*}
	W^{\kt}_{\rho,\mathrm{grad}} &:= \overline{\clbracs{ \bracs{\phi, \ktgrad\phi} \setVert \phi\in\smooth{\ddom} }},
\end{align*}
where the closure is taken in $\ltwo{\ddom}{\rho}\times\ltwo{\ddom}{\rho}^3$, and 
\begin{align*}
	W^{\kt}_{\rho,\mathrm{curl}} &:= \overline{\clbracs{ \bracs{\Phi, \ktcurl{}\Phi} \setVert \Phi\in\bracs{\smooth{\ddom}}^3 }},
\end{align*}
where the closure is taken in $\ltwo{\ddom}{\rho}^3\times\ltwo{\ddom}{\rho}^3$.
This mimics the ``$H$"-definition of classical Sobolev spaces (when $\rho$ is the Lebesgue measure), however for general measures $\rho$ the Meyers-Serrin ``$H=W$" theorem doesn't hold due to the lack of the integration by parts technique.
We would like to call the second element of a pair $\bracs{u,g}\in W^{\kt}_{\rho,\mathrm{grad}}$ ``the gradient" of the function $u$, however notice that if $\bracs{u,g_1}, \bracs{0, g_2}\in W^{\kt}_{\rho,\mathrm{grad}}$ then we also have that $\bracs{u, g_1+g_2}\in W^{\kt}_{\rho,\mathrm{grad}}$.
As such, the term ``gradient" being used in reference to the second element of a pair in $W^{\kt}_{\rho,\mathrm{grad}}$ does not indicate a \emph{unique} function --- both $g_1$ and $g_1+g_2$ are ``gradients" of $u$ in this sense.
An analogous deduction can be made for ``curl" and the second member of elements of $W^{\kt}_{\rho,\mathrm{curl}}$.
Whilst non-uniqueness of ``gradients" (in the aforementioned sense) is not an active hindrance \tstk{Zhikov, Kirill \& Serena deal with this fine)}, for our purposes it is convenient for us to define the set of ``($\rho$)-gradients of zero" as
\begin{align*}
	\gradZero{\ddom}{\rho} &= \clbracs{ g\in \ltwo{\ddom}{\rho}^3 \setVert \bracs{0,g}\in W^{\kt}_{\rho,\mathrm{grad}}}, \\
	&= \clbracs{ g\in\ltwo{\ddom}{\rho}^3 \setVert \exists\phi_n\in\smooth{\ddom} \text{ s.t. } \phi_n \lconv{\ltwo{\ddom}{\rho}}0, \ktgrad\phi_n\lconv{\ltwo{\ddom}{\rho}^3} g }, \labelthis\label{eq:GradZeroSequenceDef}
\end{align*}
and analogously define the set of ``($\rho$)-curls of zero" as
\begin{align*}
	\curlZero{\ddom}{\rho} &= \clbracs{ c\in \ltwo{\ddom}{\rho}^3 \setVert \bracs{0,c}\in W^{\kt}_{\rho,\mathrm{curl}}}, \\
	&= \clbracs{ c\in\ltwo{\ddom}{\rho}^3 \setVert \exists\Phi^n\in\smooth{\ddom}^3 \text{ s.t. } \Phi^n\lconv{\ltwo{\ddom}{\rho}^3}0, \ \ktcurl{}\Phi\lconv{\ltwo{\ddom}{\rho}^3} c}. \labelthis\label{eq:CurlZeroSequenceDef}
\end{align*}
Given an element $g\in\gradZero{\ddom}{\rho}$, we will refer to a sequence $\phi_n$ as in \eqref{eq:GradZeroSequenceDef} as an ``approximating sequence" for $g$, and will make use of the phrase ``take an approximating sequence $\phi_n$ for $g$" (or similar) to mean ``let $\phi_n$ be a sequence as in \eqref{eq:GradZeroSequenceDef} for the element $g$ of $\gradZero{\ddom}{\rho}$".
We also adopt a similar convention for elements $c\in\curlZero{\ddom}{\rho}$ and ``approximating sequences" as in \eqref{eq:CurlZeroSequenceDef}.

The sets $\gradZero{\ddom}{\rho}$ and $\curlZero{\ddom}{\rho}$ do not depend on $\kt$, as the following proposition demonstrates.
\begin{prop} \label{prop:ZeroInvariantUnderQM-Wavenumber}
	Fix a wavenumber $\wavenumber$ and a quasi-momentum $\qm$, and let
	\begin{align*}
		\gradZero{\ddom}{\rho}^{\kt} &:= \clbracs{ g\in \ltwo{\ddom}{\rho}^3 \setVert \bracs{0,g}\in W^{\kt}_\mathrm{grad} }, \\
		\gradZero{\ddom}{\rho} &:= \clbracs{ g\in \ltwo{\ddom}{\rho}^3 \setVert \bracs{0,g}\in W^{\bracs{0, 0}}_\mathrm{grad} }, \\
		\curlZero{\ddom}{\rho}^{\kt} &:= \clbracs{ c\in \ltwo{\ddom}{\rho}^3 \setVert \bracs{0,c}\in W^{\kt}_\mathrm{curl} }, \\
		\curlZero{\ddom}{\rho} &:= \clbracs{ g\in \ltwo{\ddom}{\rho}^3 \setVert \bracs{0,g}\in W^{\bracs{0, 0}}_\mathrm{curl} }.
	\end{align*}
	Then the following sets are equal:
	\begin{align*}
		\gradZero{\ddom}{\rho}^{\kt} &= \gradZero{\ddom}{\rho}, \\
		\curlZero{\ddom}{\rho}^{\kt} &= \curlZero{\ddom}{\rho}.
	\end{align*}
\end{prop}
\begin{proof}
	This is seen by observing that for $\phi\in\smooth{\ddom}$ and $\Phi\in\smooth{\ddom}^3$,
	\begin{align*}
		\grad^{\kt}\phi &= \grad^{\bracs{0, 0}}\phi + \rmi\wavenumber\begin{pmatrix} 0 \\ 0 \\ \phi \end{pmatrix} + \rmi\begin{pmatrix} \qm_1 \\ \qm_2 \\ 0 \end{pmatrix}\phi, \\
		\grad^{\kt}\wedge\Phi &= \grad^{\bracs{0, 0}}\wedge\Phi + \rmi\wavenumber\begin{pmatrix} 0 \\ 0 \\ 1 \end{pmatrix}\wedge\Phi + \rmi\begin{pmatrix} \qm_1 \\ \qm_2 \\ 0 \end{pmatrix} \wedge \Phi.
	\end{align*}
	Thus, if $g\in \gradZero{\ddom}{\rho}^{\kt}$ there exists a sequence $\phi_n\in\smooth{\ddom}$ such that
	\begin{align*}
		\phi_n \lconv{\ltwo{\ddom}{\rho}} 0, &\quad \ktgrad\phi_n \lconv{ \ltwo{\ddom}{\rho}^3 } g,
	\end{align*}
	(as in \eqref{eq:GradZeroSequenceDef}).
	But $\rmi\bracs{\qm_1, \qm_2, 0}^{\top}\phi_n\rightarrow 0$, as does $\rmi\wavenumber\phi_n$.
	Given the formulae above, we must also have that $\grad^{\bracs{0, 0}}\phi_n\rightarrow g$, so $g\in \gradZero{\ddom}{\rho}^{\bracs{0, 0}}$.
	The reverse implication, and the proof for $\rho$-curls of zero, is similar.
\end{proof}
Proposition \ref{prop:ZeroInvariantUnderQM-Wavenumber} will be helpful later (section \ref{sec:3DGradSobSpaces} \tstk{section refs}) when we want to understand the behaviour of gradients (and curls) of zero for the measures $\dddmes, \ddmes$, and $\nu$.
Another useful (although expected) property of the sets $\gradZero{\ddom}{\rho}$ and $\curlZero{\ddom}{\rho}$ is that they are closed, linear subspaces of $\ltwo{\ddom}{\rho}^3$.
\begin{prop}
	The sets $\gradZero{\ddom}{\rho}$ and $\curlZero{\ddom}{\rho}$ are closed, linear subspaces of $\ltwo{\ddom}{\rho}^3$.
\end{prop}
\begin{proof}
	We only present the proof for $\gradZero{\ddom}{\rho}$, since the proof for $\curlZero{\ddom}{\rho}$ is analogous.
	
	The fact that $\gradZero{\ddom}{\rho}$ is a subspace follows from its definition; if $g, h\in\gradZero{\ddom}{\rho}$ take approximating sequences $\phi_n$, $\psi_n$ for $g,h$ respectively.
	Then for any $\alpha\in\complex$, we have that $\alpha\phi_n + \psi_n\in\smooth{\ddom}$ too, and clearly
	\begin{align*}
		\alpha\phi_n + \psi_n &\lconv{\ltwo{\ddom}{\rho}} \alpha\times 0 + 0 = 0, \\
		\ograd\bracs{\alpha\phi_n + \psi_n} &= \alpha\ograd\phi_n + \ograd\psi_n
		\lconv{\ltwo{\ddom}{\rho}^3} \alpha g + h,
	\end{align*}
	so $\alpha g + h\in\gradZero{\ddom}{\rho}$.
	
	Next we prove that $\gradZero{\ddom}{\rho}$ is closed.
	To this end, let $g_n$ be a sequence in $\gradZero{\ddom}{\rho}$ that converges (in $\ltwo{\ddom}{\rho}^3$) to some function $g$.
	For each $n\in\naturals$, there exists an approximating sequence of smooth functions $\phi_n^l$ for $g_n$ since each $g_n\in\gradZero{\ddom}{\rho}$.
	Now, let $m\in\naturals$.
	Since $g_n\rightarrow g$, there exists $N_m\in\naturals$ such that $\norm{g_n-g}_{\ltwo{\ddom}{\rho}^3}<\recip{2m}$ for all $n\geq N_m$ --- in particular, $g_{N_m}$ satisfies this inequality.
	Then, since $\phi_{N_m}^l$ is an approximating sequence for $g_{N_m}$, we have that there exist $L_{N_m}^{(1)}, L_{N_m}^{(2)}\in\naturals$ such that
	\begin{align*}
		\norm{\phi_{N_m}^l}_{\ltwo{\ddom}{\rho}} < \recip{m}, &\quad\forall l\geq L_{N_m}^{(1)}, \\
		\norm{\ograd\phi_{N_m}^l - g_{N_m}}_{\ltwo{\ddom}{\rho}^3} < \recip{2m}, &\quad\forall l\geq L_{N_m}^{(2)}.
	\end{align*}	 
	Set $L_{N_m} = \max\clbracs{ L_{N_m}^{(1)}, L_{N_m}^{(2)} }$, and define $\psi_m = \phi^{L_{N_m}}_{N_m}$ for each $m$.
	Then we have that
	\begin{align*}
		\norm{\psi_m}_{\ltwo{\ddom}{\rho}} &= \norm{\phi_{N_m}^{L_{N_m}}}_{\ltwo{\ddom}{\rho}} < \recip{m}, \\
		\norm{\ograd\psi_m - g}_{\ltwo{\ddom}{\rho}^3} &= \norm{\ograd\phi_{N_m}^{L_{N_m}} - g}_{\ltwo{\ddom}{\rho}^3} \\
		&\leq \norm{\ograd\phi_{N_m}^{L_{N_m}} - g_{N_m}} + \norm{g_{N_m} - g}_{\ltwo{\ddom}{\rho}^3} \\
		&< \recip{2m} + \recip{2m} = \recip{m}.
	\end{align*}
	Therefore,
	\begin{align*}
		\psi_m \lconv{\ltwo{\ddom}{\rho}} 0, \quad \ograd\psi_m \lconv{\ltwo{\ddom}{\rho}^3} g,
	\end{align*}
	and each $\psi_m$ is smooth, and so $g\in\gradZero{\ddom}{\rho}$.
\end{proof}

Since $\gradZero{\ddom}{\rho}$ is a closed, linear subspace of $\ltwo{\ddom}{\rho}^3$ we can decompose 
\begin{align*}
	\ltwo{\ddom}{\rho}^3 = \gradZero{\ddom}{\rho}^{\perp} \oplus \gradZero{\ddom}{\rho}.
\end{align*}
Now suppose that we have $\bracs{u,g_1}, \bracs{u,g_2}\in W^{\kt}_{\rho,\mathrm{grad}}$ with $g_1, g_2\in\gradZero{\ddom}{\rho}^\perp$.
This implies there exist smooth functions $\phi_n, \psi_n$ such that
\begin{align*}
	\phi_n \lconv{\ltwo{\ddom}{\rho}} u, &\quad \psi_n \lconv{\ltwo{\ddom}{\rho}} u, \\
	\ktgrad\phi_n \lconv{\ltwo{\ddom}{\rho}^3} g_1, &\quad \ktgrad\psi_n \lconv{\ltwo{\ddom}{\rho}^3} g_2.
\end{align*}
Since $g_1, g_2\in \gradZero{\ddom}{\rho}^{\perp}$, we have that $g_1-g_2\in \gradZero{\ddom}{\rho}^{\perp}$ too.
However, 
\begin{align*}
	\phi_n - \psi_n \lconv{\ltwo{\ddom}{\rho}} 0, &\quad \ktgrad\bracs{\phi_n-\psi_n} \lconv{\ltwo{\ddom}{\rho}^3} g_1 - g_2,
\end{align*}
so $g_1 - g_2\in \gradZero{\ddom}{\rho}$ too --- therefore we must have that $g_1=g_2$.
With this in mind, we see that each $u$ possesses a unique gradient \emph{that is orthogonal to} $\gradZero{\ddom}{\rho}$.
This allows construction of the ``Sobolev space" of gradients with respect to the measure $\rho$:
\begin{definition}[Sobolev Space of Gradients] \label{def:3DGradSobSpace}
	The set
	\begin{align*}
		\ktgradSob{\ddom}{\rho} &= \clbracs{ \bracs{u, \ktgrad_{\rho}u}\in W^{\kt}_{\rho,\mathrm{grad}} \setVert \ktgrad_{\rho}u \perp \gradZero{\ddom}{\rho} },
	\end{align*}
	is called the ``Sobolev space of $\kt$-gradients with respect to the measure $\rho$".
	We call the member $\ktgrad_{\rho}u$ of the pair $\bracs{u, \ktgrad_{\rho}u}\in\ktgradSob{\ddom}{\rho}$ the ``$\kt$-tangential gradient of $u$ with respect to $\rho$".
\end{definition}
When the context is clear, we will simply refer to $\ktgrad_{\rho}u$ as the ``tangential gradient" or ``$\kt$-tangential gradient".
Additionally, it is enough to specify the first member $u$ of the pair $\bracs{u, \ktgrad_{\rho}u}$ due to the uniqueness of $\kt$-tangential gradients, so we will often write $u\in\ktgradSob{\ddom}{\rho}$ as shorthand.

We can also perform analogous steps for $\curlZero{\ddom}{\rho}$, which allows us to define the ``Sobolev space" of curls with respect to the measure $\rho$:
\begin{definition}[Sobolev Space of Curls] \label{def:CurlSobSpace}
	The set
	\begin{align*}
		\ktcurlSob{\ddom}{\rho} &= \clbracs{ \bracs{u, \ktcurl{\rho}u}\in W^{\kt}_{\rho,\mathrm{curl}} \setVert \ktcurl{\rho}u \perp \curlZero{\ddom}{\rho} },
	\end{align*}
	is called the ``Sobolev space of $\kt$-curls with respect to the measure $\rho$".
	The member $\ktcurl{\rho}u$ of the pair $\bracs{u, \ktcurl{\rho}u}$ is the ``$\kt$-tangential curl of $u$ with respect to $\rho$".
\end{definition}
Like with tangential gradients; we will use the shorthand $u\in\ktcurlSob{\ddom}{\rho}$ to refer to the element $\bracs{u, \ktcurl{\rho}u}\in\ktcurlSob{\ddom}{\rho}$, since each $u$ has a unique $\ktcurl{\rho}u$ that is orthogonal to $\curlZero{\ddom}{\rho}$, and when the context is clear will refer to $\ktcurl{\rho}u$ as the ``tangential curl" or ``$\kt$-tangential curl".

Having now assigned a meaning to (tangential) gradients and curls for the measure $\rho$, we can also define what it means for a vector field $u$ to be $\kt$-divergence-free with respect to $\rho$.
\begin{definition}[$\kt$-divergence-free] \label{def:ktDivFree}
	A vector field $u\in\ltwo{\ddom}{\rho}^3$ is said to be $\kt$-divergence-free with respect to $\rho$ if
	\begin{align*}
		0 &= \integral{\ddom}{ u\cdot\overline{g} }{\rho}, \qquad\forall\bracs{v,g}\in W^{\kt}_{\rho,\mathrm{grad}}.
	\end{align*}
	When the context is clear, we simply shorten ``$\kt$-divergence-free with respect to $\rho$" to ``$\kt$-divergence-free".
\end{definition}
Intuitively, we have setup ``divergence-free" vector fields to be those that are orthogonal (in $\ltwo{\ddom}{\rho}^3$) to all gradients of $\ltwo{\ddom}{\rho}$-functions, including gradients of zero.
\tstk{Helmholtz decomp stuff, L2\_Decomposition.pdf. Might go at end with the other div-free result? Also might want to save this for the Maxwell chapter discussion to flesh out that chapter, as it's a focal point there (even though we can prove the statements in general for $\rho$).}

To conclude this section, we provide some simple results which will prove useful throughout our analysis.
First, it is useful to note that (by construction) functions with tangential gradients or curls can also be approximated by smooth functions.
\begin{cory} \label{cory:SobSpaceApproxSequences}
	\begin{align*}
		\ktgradSob{\ddom}{\rho} &= \clbracs{ \bracs{u, \ktgrad_{\rho}u} \setVert \exists\phi_n\in\smooth{\ddom} \text{ s.t. } \right. \\ 
		&\hspace{0.25\textwidth} \left. \phi_n\lconv{\ltwo{\ddom}{\rho}}u, \ \ktgrad\phi_n\lconv{\ltwo{\ddom}{\rho}^3}\ktgrad_{\rho}u }, \\
		\ktcurlSob{\ddom}{\rho} &= \clbracs{ \bracs{u, \ktcurl{\rho}u} \setVert \exists\Phi^n\in\smooth{\ddom}^3 \text{ s.t. } \right. \\
		&\hspace{0.25\textwidth} \left. \Phi^n\lconv{\ltwo{\ddom}{\rho}^3}u, \ \ktcurl{}\Phi^n\lconv{\ltwo{\ddom}{\rho}^3}\ktgrad_{\rho}u }.
	\end{align*}
\end{cory}
\begin{proof}
	This is a direct consequence of the construction of $W^{\kt}_{\rho,\mathrm{grad}}$ and $W^{\kt}_{\rho,\mathrm{curl}}$.
\end{proof}
We will adopt analogous terminology and phrasing when using sequences $\phi_n$ and $\Phi^n$ as they appear in corollary \ref{cory:SobSpaceApproxSequences} to that used for sequences in \eqref{eq:GradZeroSequenceDef}-\eqref{eq:CurlZeroSequenceDef} regarding gradients and curls of zero --- referring to them as ``approximating sequences" for elements of $\ktgradSob{\ddom}{\rho}$ or $\ktcurlSob{\ddom}{\rho}$.

Given our choice to construct (tangential) gradients and curls via approximation, we retain the rule that ``the curl of a gradient is 0" --- once translated into its analogue for gradients and curls with respect to the measure $\rho$.
\begin{lemma} \label{lem:CurlOfGradSmoothFunctions}
	For $\phi\in\smooth{\ddom}$, we have that
	\begin{align*}
		\ktcurl{}\bracs{\ktgrad\phi} = 0.
	\end{align*}
\end{lemma}
\begin{proof}
	Let $\phi\in\smooth{\ddom}$, $\Phi = \ktgrad\phi$, and set $\widetilde{\qm} = \bracs{\qm_1, \qm_2, 0}^\top$ throughout.
	Notice that
	\begin{align*}
		\Phi = \ktgrad\phi &= \e^{-\rmi\qm\cdot x}\kgrad\bracs{ \e^{\rmi\qm\cdot x}\phi },
	\end{align*}
	so with $\psi = \e^{\rmi\qm\cdot x}\phi$, we have that
	\begin{align*}
		\ktcurl{}\Phi 
		&= \kgrad\wedge\bracs{ \Phi } + \rmi\widetilde{\qm}\wedge\Phi
		= -\rmi\widetilde{\qm}\wedge\Phi + \e^{-\rmi\qm\cdot x}\kgrad\wedge\psi + \rmi\widetilde{\qm}\wedge\Phi \\
		&= \e^{-\rmi\qm\cdot x}\kgrad\wedge\psi
		= \e^{-\rmi\qm\cdot x} 
		\begin{pmatrix}
			\partial_2\bracs{ \rmi\wavenumber \e^{\rmi\qm\cdot x}\phi } - \rmi\wavenumber\partial_2\bracs{ \e^{\rmi\qm\cdot x}\phi } \\
			\rmi\wavenumber\partial_1\bracs{ \e^{\rmi\qm\cdot x}\phi } - \partial_1\bracs{ \rmi\wavenumber \e^{\rmi\qm\cdot x}\phi } \\
			\partial_1\bracs{ \partial_2\bracs{\rmi\wavenumber\e^{\rmi\qm\cdot x}\phi} } - \partial_2\bracs{ \partial_1\bracs{\rmi\wavenumber\e^{\rmi\qm\cdot x}\phi} }
		\end{pmatrix} \\
		&= 0.
	\end{align*}
\end{proof}
Since our definitions of $\gradZero{\ddom}{\rho}, \curlZero{\ddom}{\rho}$ and tangential gradients and curls all depend on approximations by smooth functions, lemma \ref{lem:CurlOfGradSmoothFunctions} informs us that adding a ``gradient" to an existing vector field will not change the curl of said vector field.
\begin{prop} \label{prop:CurlIgnoresGradients}
	Suppose that $u\in\ktcurlSob{\ddom}{\rho}$, $g\in\gradZero{\ddom}{\rho}$, $w\in\ktgradSob{\ddom}{\rho}$ and set $v = u + g + w\in\ltwo{\ddom}{\rho}^3$. 
	Then $\bracs{v, \ktcurl{\rho}u}\in\ktcurlSob{\ddom}{\rho}$ --- that is, $v$ has a tangential $\kt$-curl equal to that of $u$.
\end{prop}
\begin{proof}
	Take an approximating sequence $\Phi^n$ for $u$ as in corollary \ref{cory:SobSpaceApproxSequences}, $\phi_n$ for $w$ as in  corollary \ref{cory:SobSpaceApproxSequences}, and $\psi_n$ for $g$ as in \eqref{eq:GradZeroSequenceDef}.
	Define $\varphi^n := \Phi^n + \ktgrad\phi_n + \ktgrad\psi_n\in\smooth{\ddom}^3$ for each $n\in\naturals$.
	Given lemma \ref{lem:CurlOfGradSmoothFunctions}, we have that $\ktcurl{}\varphi^n = \ktcurl{}\Phi^n$ for every $n\in\naturals$.
	Therefore, we have that
	\begin{align*}
		\varphi^n &\lconv{\ltwo{\ddom}{\rho}^3} u + w + g = v, \\
		\ktcurl{}\varphi^n &= \ktcurl{}\Phi^n 
		\lconv{\ltwo{\ddom}{\rho}^3} \ktcurl{\rho}u.
	\end{align*}
	Thus, we conclude that $\bracs{v, \ktcurl{\rho}u}\in W^{\kt}_{\rho, \mathrm{curl}}$.
	Since $\ktcurl{\rho}u\in\curlZero{\ddom}{\rho}^\perp$, we must conclude that $\bracs{ v, \ktcurl{\rho}u }\in\ktcurlSob{\ddom}{\rho}$ too.
\end{proof}
In particular, proposition \ref{prop:CurlIgnoresGradients} informs us that the $\kt$-tangential curl of either a $\kt$-tangential gradient, or a gradient of zero, is zero.

Another remark that deserves to be made is that given the construction of $W^{\kt}_{\rho,\mathrm{grad}}$, it is sufficient to only test orthogonality of $u$ against smooth gradients when checking if a field is divergence free.
\begin{cory} \label{cory:DivFreeSufficient}
	If
	\begin{align*}
		0 &= \integral{\ddom}{ u\cdot\overline{\ktgrad\phi} }{\rho}, \qquad\forall\phi\in\smooth{\ddom},
	\end{align*}
	then $u$ is $\kt$-divergence free with respect to $\rho$.
\end{cory}
\begin{proof}
	If $\bracs{v,g}\in W^{\kt}_{\rho,\mathrm{grad}}$ take an approximating sequence as in \eqref{eq:GradZeroSequenceDef}.
	Then clearly
	\begin{align*}
		\integral{\ddom}{ u\cdot\overline{g} }{\rho} 
		&= \lim_{n\rightarrow\infty}\integral{\ddom}{ u\cdot\overline{\ktgrad\phi_n} }{\rho}
		= \lim_{n\rightarrow\infty} 0 = 0.
	\end{align*}
\end{proof}
Corollary \ref{cory:DivFreeSufficient} does mean that definition \ref{def:ktDivFree} can be weakened, however we chose to present the definition of ``divergence free" in this manner to maintain parallels to the usual intuition of a divergence free field being orthogonal to \emph{all} gradients.