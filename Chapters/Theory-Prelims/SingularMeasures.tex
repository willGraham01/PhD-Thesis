\section{Singular Measures} \label{sec:SingularMeasures}
Now that we have established how we can represent singular structures through embedded graphs, we move on to the introduction of the singular measures which will allow us to pose meaningful variational problems on these structures.
The introduction of these measures also necessitates a change in the way we view gradients, curls, and divergences, which is the subject of section \ref{sec:BorelMeasSobSpaces}, and is our solution to the problems discussed in section \ref{sec:Intro-ProblemIntroduction}.

Let $\ddom\subset\reals^d$, and let $\mathcal{B}_{\ddom}$ denote the Borel sigma algebra for $\ddom$ (with the topology inherited from $\reals^d$).
A (Borel) measure $\rho$ on the measurable space $\bracs{\ddom,\mathcal{B}_{\ddom}}$ is called \emph{singular} if there exist two disjoint sets $S_1,S_2\in \mathcal{B}_{\ddom}$ such that $\rho$ is zero on all (measurable) subsets of $S_1$ and the $d$-dimensional Lebesgue measure $\lambda_d$ is zero on all subsets of $S_2$.
The textbook example of a singular measure is the point-mass measure (or \emph{$\delta$-function}) placed at a point $x_0\in\ddom$,
\begin{align*}
	\delta_{x_0}\bracs{B} &= \begin{cases} 1 & x_0\in B, \\ 0 & x_0\not\in B, \end{cases}
	\qquad \forall B\in\mathcal{B}_{\ddom},
\end{align*}
for which we can take $S_1=\ddom\setminus\clbracs{x_0}$ and $S_2=\clbracs{x_0}$.
In addition to point masses, we will also want to consider singular measures which support our singular structures.
Let $\graph=\bracs{\vertSet, \edgeSet}$ be a metric graph embedded into $\reals^d$.
For each $I_{jk}\in\edgeSet$ define the (Borel) measure $\lambda_{jk}$ as the measure which supports the one-dimensional measure $\lambda_1$ on (or ``along") $I_{jk}$,
\begin{align*}
	\lambda_{jk}\bracs{B} = \lambda_{1}\bracs{r_{jk}^{-1}\bracs{B \cap I_{jk}}},
	&\quad\text{for all Borel } B.
\end{align*}
Here, $r_{jk}$ is the parametrisation of the edge $I_{jk}$ as per assumption \ref{ass:MeasTheoryProblemSetup}.
The measure $\lambda_{jk}$ will be referred to as the \emph{singular measure that supports $I_{jk}$}, or just the \emph{singular measure on $I_{jk}$}.
Since the sum of two singular measures is another singular measure, we can then define the following singular measures;
\begin{align*}
	\ddmes\bracs{B} = \sum_{v_j\in \vertSet}\sum_{j\conLeft k} \lambda_{jk}\bracs{B},
	\quad
	\nu\bracs{B} = \sum_{v_j\in\vertSet}\alpha_j\delta_j\bracs{B},
	\quad
	\dddmes\bracs{B} = \ddmes\bracs{B} + \nu\bracs{B},
\end{align*}
where we refer to $\ddmes$ as the ``\emph{singular measure} that supports $\graph$", and $\nu$ is the singular measure supporting the masses $\alpha_j$ at the vertices of $\graph$.
For graphs embedded into two and three dimensions, figure \ref{fig:Diagram_SingularMeasure2D} illustrates the $\dddmes$ notion of size.
\begin{figure}[b!]
	\centering
	\begin{subfigure}[t]{0.45\textwidth}
		\centering
		\includegraphics[scale=0.85]{Diagram_SingularMeasure2D.pdf}
		\caption{\label{fig:Diagram_SingularMeasure2D} For a graph embedded in $\reals^2$, the $\ddmes$-measure of any Borel set $B$ is obtained from summing the contributions of each $\lambda_{jk}$, as indicated by the thickened lines. Sets that do not intersect $\graph$ have zero measure.}
	\end{subfigure}
	~
	\begin{subfigure}[t]{0.45\textwidth}
		\centering
		\includegraphics[scale=0.5]{Diagram_SingularMeasure3D.pdf}
		\caption{\label{fig:Diagram_SingularMeasure3D} An extension of the concept of a singular measure to 3 dimensions. The red portions of each edge indicate the contributions from each edge to the measure of the cube $B$.}
	\end{subfigure}
\end{figure}
In chapter \ref{ch:SingInc}, we will reintroduce the background material into which our singular structure is embedded.
This will require us to consider the sum of singular measures and the ``background" Lebesgue measure $\lambda_2$, so for a singular measure $\rho$ we shall write \tstk{change overhead tilde notation to distinguish -  you're using this for both $\dddmes$, the sum of vertex and edge singular measures, and the composite measure in chapter 5} $\tilde{\rho} = \rho + \lambda_2$.