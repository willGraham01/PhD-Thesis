\section{Physical Motivation} \label{sec:PhysMot}
Optical fibres are the \textit{de facto} industry standard for large telecommunications systems, thanks to their ability to transmit information quickly and with far less signal loss than other methods (such as metal cables).
The technology has rapidly developed since the first optical fibres were fabricated in the 1970s \cite{knight2003photonic} and optical fibres in use today present a balance between several competing factors to deliver a reliable performance.
Some of these factors such as (optical) loss are inherent, brought about by the materials needed to build the fibres,
Other factors can be influenced by the fibres design (group-velocity dispersion) or fabrication process, which can lead to imperfections and polarisation effects.
The common design of an optical fibre will consist of a ``core" made of a dielectric (non-conducting) material with a given refractive index, surrounded by ``cladding", another dielectric material of a lower refractive index.
In practice this is normally achieved by choosing a material for the core and then using a doped version of said material for the cladding, with silica being a common choice, which leads to typical differences in refractive indices of the core and cladding of around $0.001$.
By ensuring the cladding material has a lower refractive index than the core material, modes of light\footnote{A mode of light is a mono-frequency solution to the governing equations of electromagnetism in the fibre.} can be confined to the core of an optical fibre via the phenomenon of Total Internal Reflection (TIR), illustrated in figure \ref{fig:Diagram_OpticalFibre}.
\begin{figure}[h]
	\centering
	\includegraphics[scale=1.0]{Diagram_OpticalFibre.pdf}
	\caption{\label{fig:Diagram_OpticalFibre} A schematic diagram of a weakly-guiding optical fibre. A core is surrounded by a cladding, giving a cross section composed of concentric circles. Light propagates along the fibre axis, confined to the core by the process of TIR. Note: this is illustrated with a ray picture of light.}
\end{figure} 
One can imagine light entering the core of the fibre at one end, and (totally internally) reflecting off the boundaries of core and cladding as it moves along the fibre (although this intuition uses the ray description of light rather than the wave description).
Wave guidance in fibres using TIR is known as weak guiding, and typically allows light to be propagated over tens of kilometres before a signal boost is required.
However despite the majority of modern optical fibres utilising this method of guidance, all improvements to the technology have been incremental and largely centre around the manufacturing process. \newline

The first significant departure from this traditional setup came with the realisation that the ability to 
structure a material on the same scale as the wavelength of optical light will drastically alter its optical properties.
With the advancement of fabrication techniques to allow for these structures to be physically manufactured, the development and study of materials known as ``photonic crystals" begun in earnest.
These photonic crystals are materials composed of a periodic microstructure (on the aforementioned length scale), typically a periodic arrangement (``matrix") of one dielectric material within another.
They can be used as cladding for a traditional optical fibre in the same manner as described above, by encasing a core material of a higher refractive index within the photonic crystal --- such fibres are referred to as ``photonic crystal fibres" (PCFs).
However more radical designs for PCFs utilise the fact that, depending on their frequency (or equivalently wavelength), certain modes of light cannot propagate through a given photonic crystal.
Due to the inhomogeneity of the photonic crystal, waves propagating (or attempting to propagate) within it undergo multiple scattering and are subject to interference.
At certain frequencies, a wave attempting to propagate through the crystal may be subject to cumulative destructive inference off the inclusions, effectively forbidding wave propagation within the crystal at that frequency.
These intervals or ranges of frequencies at which light cannot propagate within the crystal are referred to as \emph{photonic} or \emph{frequency band gaps}, and their compliment as \emph{bands}.
These photonic band gaps offer an alternative method of ``guidance", rather than having light undergo TIR within a core material, instead these modes can be confined to the core by virtue of having a frequency within one of these band gaps.
Since such a mode of light cannot propagate in the surrounding crystal, it is confined, and can propagate along the core (provided its propagation \emph{is} supported by the core).
From these ideas came several new designs for PCFs, the first being the design and manufacture of ``hollow-core"\footnote{The name coming from the fact that the first such fibres were typically made from a photonic crystal of glass periodically punctured with air holes, surrounding a larger air cavity that played the role of the core. The core consisting of air rather than a solid dielectric material gave rise to the term ``hollow".} fibres schematically illustrated in figure \ref{fig:Diagram_PCF-AirHollowConfine.pdf}.
\begin{figure}[b!]
	\centering
	\begin{subfigure}[t]{0.45\textwidth}
		\centering
		\includegraphics[scale=0.5]{Diagram_PCF-AirHollowConfine.pdf}
		\caption{\label{fig:Diagram_PCF-AirHollowConfine.pdf} A hollow-core fibre formed from a material (typically a glass) punctured with air holes.}
	\end{subfigure}
	~
	\begin{subfigure}[t]{0.45\textwidth}
		\centering
		\includegraphics[scale=0.5]{Diagram_PCF-LowIndexConfine.pdf}
		\caption{\label{fig:Diagram_PCF-LowIndexConfine.pdf} An all-solid PCF, whose core is a defect in the periodic structure.}
	\end{subfigure}
	\caption{\label{fig:Diagram_PCF} Schematic illustration of the cross-section of a PCF. A photonic crystal surrounds a core, whose bandgaps confine certain modes of light to the cores, and support their propagation down the fibre. Darker colours represent higher refractive indices.}
\end{figure}
These PCFs confine light of a given frequency to this core at the centre, using the surrounding periodic arrangement of low-index inclusions.
However, this band-gap confinement can also be used to confine light to a low refractive index core via the placement of higher-index inclusions in the crystal (in figure \ref{fig:Diagram_PCF-LowIndexConfine.pdf}), in ``solid-core" PCFs --- see for example the study \cite{luan2004allsolid}.
Further investigations into PCF designs have even shown confinement and propagation of light using metallic reflection \cite{hou2008metallic}.
Importantly, all of these PCF designs do not utilise TIR to ensure that light is confined to the core, breaking tradition with established optical fibre technology \cite{knight2003photonic, russell2003photonic}.

The physical theory underlying the process by which they operate means that PCFs have the potential to replace traditional weakly guiding optical fibres as the industry standard.
PCFs have an advantage over conventional optical fibres in that their applications are not limited to telecommunications, with alternative applications including non-linear optics (where they offer high optical intensities per unit power, making them highly efficient) and particle guidance (dielectric particles can be guided by the dipole forces exerted by light).
Knowing how a photonic crystals band gaps depend on its geometry and microstructure is of great importance in the design process, as these gaps determine the range of frequencies at which the fibre can operate as a waveguide.
Hence there has been much to motivate study of the optical properties of PCFs; and understanding this interplay serves as the motivation for the study of the systems considered in this work, albeit these systems represent an approximation to these 2D photonic crystals.

\subsection{Exisiting Models for PCFs} \label{ssec:ExistingPCFModels}
Before moving on to introduce the systems to be studied in this work, we briefly outline some of the existing modelling techniques for PCFs.
The most conceptually straightforward of these is the use of numerical techniques to solve Maxwell's equations to determine modes that are supported by the PCF.
These modes can then be used to construct an approximate band-gap plot for the fibre, with the level of precision in the computational results coming at the cost of increasing the computing time.
Whilst this can provide detailed information about a fibre, it does not provide any insight into how the structure of the photonic crystal has contributed or affected the resulting band-gap plot.
Furthermore numerical schemes (particularly those based off finite elements) are known to require special treatment when being used to solve Maxwell's equations, but this in itself has been studied in detail (see for example \cite{monk2003finite}).
In our work we will largely stay clear of these kinds of solvers, primarily because they will not be directly applicable to the problems that we are looking to consider \tstk{(section \ref{sec:OurPhysicalSetup})} and the aforementioned lack of insight into the dependence of the band-gap structure on the underlying fibre geometry. \newline

There has been some success in developing approximate models for wave guidance in PCFs that retain key features of experimental band-gap plots.
Typically one reduces (by means of an appropriate transform or use of Bloch's theorem) the analysis of the governing equations for wave guidance in the photonic crystal to that on a period cell, and reassembles the band-gap structure through the analysis of the family of problems on the unit cell.
The ARROW model \tstk{Litchinitser, N., Dunn, S., Steinvurzel, P., Eggleton, B., White, T., McPhedran,
R., and De Sterke, C. Application of an arrow model for designing tunable photonic
devices. Opt. Express 12, (8) (2004), 15401550. Also references within \cite{birks2006approximate}} has been highly effective at explaining confinement of light to the core in certain solid-core fibres through anti-resonant reflections off the higher-index inclusions.
Models such as \cite{birks2006approximate}, simplify the geometry of the period cell and solve the (scalar) wave equation in the resulting structure, reporting good agreement with numerical solvers for the types of PCF the approximate model considers.
Other works such as \cite{birks2006approximate} look to understand how band-gaps for hollow-core fibres scale with the contrast in refractive index.
Such approaches useful when speed is more important than high accuracy, or when an intuitive picture of the band-structure is desired.
\tstk{now Cooper's pre-print too? not entirely sure how to transition here}

 by treating the physical origins of band-gaps in fibres as arising from the resonant properties of the cores.
In particular the spectral bands of a PCF are taken to correspond to modes that couple between the cores.
If the cores are assumed identical in shape then only information about the modes of a single core, plus the relative core spacing and size, are needed to build an approximate band-gap plot.
These methods have made several insights into the origins of band-gap structures, but are limited by their approximations in the information they provide about the fibres themselves.
Intermediate approaches have also been proposed, for example in \cite{birks2006approximate} which proposes a model that considers the sizes of the regions separating the cores in the PCF and provides alternative boundary conditions for when the unit cell of the fibre cross-section is hexagonal or circular.
This model is still based on the physical origins of the band-gaps being due to resonances within the cores of the fibre however; and eventually relies on a numerical scheme.
But the numerical scheme is only needed to the extent of root-finding for analytic expressions, and produces band-gap plots that are in good agreement with the more fully-fledged numerical models, provided certain scaling between parameters is adhered to.
This makes such approaches useful when speed is more important than high accuracy, or when an intuitive picture of the band-structure is required.