\section{Physical Motivation} \label{sec:PhysMot}
Optical fibres are the \textit{de facto} industry standard for large telecommunications systems, thanks to their ability to transmit information quickly and with far less signal loss than other methods (such as metal cables).
The technology has rapidly developed since the first optical fibres were fabricated in the 1970s \cite{knight2003photonic} and optical fibres in use today present a balance between several competing factors to deliver a reliable performance.
The common design of an optical fibre will consist of a \emph{core} made of a dielectric (non-conducting) material with a given refractive index, surrounded by \emph{cladding}, another dielectric material of a lower refractive index.
In practice this is normally achieved by choosing a material (typically silica) for the core and then using a doped version of said material for the cladding, resulting in typical differences of around $0.001$ between the refractive indices.
By ensuring the cladding material has a lower refractive index than the core material, modes of light\footnote{A mode of light is a mono-frequency solution to the governing equations of electromagnetism in the fibre.} can be confined to the core of an optical fibre via the phenomenon of Total Internal Reflection (TIR), illustrated in figure \ref{fig:Diagram_OpticalFibre}.
\begin{figure}[t!]
	\centering
	\includegraphics[scale=1.0]{Diagram_OpticalFibre.pdf}
	\caption[Illustration of a weakly-guiding optical fibre.]{\label{fig:Diagram_OpticalFibre} A schematic diagram of a weakly-guiding optical fibre. A core is surrounded by a cladding, giving a cross section composed of concentric circles. Light propagates along the fibre axis, confined to the core by the process of TIR. Note: this is illustrated with a ray picture of light.}
\end{figure} 
One can imagine light entering the core at one end of the fibre and (totally internally) reflecting off the boundaries of core and cladding as it moves along the fibre, although this picture uses the ray description of light rather than the wave description.
Wave guidance in fibres using TIR is known as weak guiding, and typically allows light to be propagated over tens of kilometres before a signal boost is required.
However despite the majority of modern optical fibres utilising this method of guidance, all improvements to the technology have been incremental and largely centre around the manufacturing process.

As fabrication techniques advanced it became possible to design a material with internal structure on the same length scale as the wavelength of optical light --- and it was quickly realised that structuring a material in this drastically alters its optical properties.
Thus lead to the development and study of \emph{photonic crystals} (PCs); materials composed of a periodic microstructure (on the aforementioned optical wavelength scale), typically a periodic arrangement of one dielectric material within another.
An optical fibre utilising a PC in its design is referred to as a \emph{photonic crystal fibre} (PCF).
PCs can be used as the cladding materials to form weakly guiding PCFs, however more radical designs utilise the fact that certain frequencies of light incident onto a PC do not support any modes of light.
These frequency intervals or ranges of frequencies that do not support modes are referred to as (\emph{photonic-} or \emph{frequency-}) \emph{band gaps}, and their compliment as \emph{bands}.
This offers an alternative method of wave-guidance, rather than having light undergo TIR within a core material, modes can be confined to a core material by virtue of having a frequency within one of these band gaps.
From these ideas came several new designs for PCFs, the first being \emph{hollow-core}\footnote{The name coming from the fact that the first such fibres were typically made from a photonic crystal of glass periodically punctured with air holes, surrounding a larger air cavity that played the role of the core. The core consisting of air rather than a solid dielectric material gave rise to the term ``hollow".} fibres schematically illustrated in figure \ref{fig:Diagram_PCF-AirHollowConfine.pdf}.
\begin{figure}[b!]
	\centering
	\begin{subfigure}[t]{0.45\textwidth}
		\centering
		\includegraphics[scale=0.5]{Diagram_PCF-AirHollowConfine.pdf}
		\caption[]{\label{fig:Diagram_PCF-AirHollowConfine.pdf} A hollow-core fibre formed from a material (typically a glass) punctured with air holes.}
	\end{subfigure}
	~
	\begin{subfigure}[t]{0.45\textwidth}
		\centering
		\includegraphics[scale=0.5]{Diagram_PCF-LowIndexConfine.pdf}
		\caption[]{\label{fig:Diagram_PCF-LowIndexConfine.pdf} An all-solid PCF, whose core is a defect in the periodic structure.}
	\end{subfigure}
	\caption[Schematic illustration of the cross-section of a photonic crystal fibre.]{\label{fig:Diagram_PCF} Schematic illustration of the cross-section of a PCF. A photonic crystal surrounds a core, whose bandgaps confine certain modes of light to the cores, and support their propagation down the fibre. Darker colours represent higher refractive indices.}
\end{figure}
These PCFs confine light of a given frequency to a core at the centre of the fibre, which is typically a defect in a PC formed from a periodic arrangement of low-index inclusions.
However, band-gaps can also be used to confine light to a low refractive index core via the placement of higher-index inclusions in the crystal (in figure \ref{fig:Diagram_PCF-LowIndexConfine.pdf}), in \emph{solid-core} PCFs --- see for example the study \cite{luan2004allsolid}.
Further investigations into PCF designs have even shown confinement and propagation of light using metallic reflection \cite{hou2008metallic}.
Significantly, none of these new PCF designs utilise TIR to guide light, breaking tradition with established optical fibre technology \cite{knight2003photonic, russell2003photonic}.

The physical theory underlying the process by which PCFs operate gives them the potential to replace traditional weakly guiding optical fibres as the industry standard.
PCFs have an advantage over conventional optical fibres in that their applications are not limited to telecommunications, with alternative applications including non-linear optics (where they offer high optical intensities per unit power, making them highly efficient) and particle guidance (dielectric particles can be guided by the dipole forces exerted by light).
Knowing how the band gaps of a PC depend on its geometry and material properties is of great importance in the design process, as these gaps determine the range of frequencies at which the fibre can operate as a waveguide.
Hence there has been a drive to study of the optical properties of PCFs, and understand the interplay between the crystal design and the structure of its band gaps.

\subsection{Physical models for photonic crystal fibres} \label{ssec:ExistingPCFModels}
Before moving onto a detailed description of the governing equations and domains that are used to model PCs (and PCFs); we outline some of the more application-focused modelling techniques, looking to provide a brief survey of some of the methods used in the explicit determination of band-gaps or other physical properties.
The most conceptually straightforward of these is the use of numerical techniques to solve Maxwell's equations to determine modes that are supported by the PCF.
Whilst this can provide detailed information about a given PCF, it does not provide any insight into how the structure of the PC has contributed or affected the resulting band-gaps.
Additionally, numerical schemes (particularly those based off finite elements) are known to require special treatment when being used to solve Maxwell's equations, although this in itself has been studied in detail (see for example \cite{monk2003finite}).
The closet we will come to employing such solvers in this work will be when we come to consider (``singular") composite media in chapter \ref{ch:SingInc}, and even then there are a number of considerations to be made before such methods can be applied to our problems.

There has been some success in developing approximate models for wave guidance in PCFs that retain key features of experimental band-gap plots.
The ARROW model (see \cite{laegsgaard2004gap, litchinitser2004application} and references therein) suggests that the confinement of modes to the (low index) core in solid-core fibres can be attributed to anti-resonant reflections off the (higher-index) inclusions --- the allows such fibres to be studied through the properties of guided modes in conventional, weakly guiding fibres.
Models such as \cite{birks2006approximate} simplify the geometry of the period cell and solve the (scalar) wave equation in the resulting structure, reporting good agreement with numerical solvers for the original (unsimplified) problems.
Other works such as \cite{birks2004scaling} look to understand how band-gaps for hollow-core fibres scale with the contrast in refractive index.
Such approaches are useful when speed is more important than high accuracy, or when an intuitive picture of the band-structure is desired.
The problems we shall consider are motivated by similar considerations about the geometry of a PC, and the desire to obtain accurate approximate models to them.
Before we introduce such problems however, we should provide concrete details on the governing equations and domain setups one uses when studying PCFs.