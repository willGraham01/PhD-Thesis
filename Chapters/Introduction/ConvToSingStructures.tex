\section{Singular Structures as a limit of Thin Structures} \label{sec:ConvToSS}
\tstk{Rename when you get the chance, to something more appropriate}

We now move on towards the study

\subsection{An aside on Homogenisation Theory} \label{ssec:Intro-HomoTheory}
\tstk{Kirill's book is a good traditional reference here, and we can then throw the more recent ``two-scale" stuff in papers as a follow up. We also have the more measure-theoretic stuff to talk about too.}

Before we review studies into the existence of band-gaps for periodic media, we provide a short overview of the techniques (and supporting theory) from homogenisation, which these studies typically rely on.
In its most general terms, the process of homogenisation looks to take the problem of studying an inhomogeneous material, and derive an \emph{effective} (or limit or approximate) problem under some particular asymptotic regime.
The functions encoding the material parameters in the differential equations governing wave propagation (or other physical phenomena of interest) in the inhomogeneous material are rapidly varying and so difficult to compute or handle.
By contrast, the effective problem looks to replace these material parameters with easier-to-handle effective material parameters --- from which the notion of an \emph{effective material} comes --- and establish some correspondence between the effective problem and original material.
In order to be more precise, let us illustrate this ethos using the (non-dimensional) wave equation 
\begin{align} \label{eq:Intro-WaveEqnHomo}
	-\grad\cdot A(x)\grad u(x) &= \omega^2 u(x),
\end{align}
in the domain with period cell $Q=[0,1)^d$ described in section \ref{ssec:Intro-NonDimMax}.
In the standard notation within homogenisation theory the material properties are represented by a (matrix-valued) $Q$-periodic function $A(x)$, which in this case is $A(x)\eps_{\mathrm{r}}^{-1}(x)$.
The process of homogenisation would look to replace the wave equation \eqref{eq:Intro-WaveEqnHomo} with an equation of the form
\begin{align}
	-\grad\cdot A^{0}\grad u
\end{align}
\tstk{but where does the length scaling come in? I think I've set myself up in the FFreq case, rather than the HFreq case as is reported in most texts...}

\tstk{here, I also want to insert the discussion about how metamaterial behaviour comes about (and other dissipative effects), and hopefully touch on generalised resolvents if there is an appropriate moment. All the while, we will be talking about systems with two independent parameters, ready for us to pull the rug out in the next section. }

\subsection{The $\delta\rightarrow0$ limit}

Graphically, the ``limit" $\delta\rightarrow0$ in these domain setups is rather intuitive --- one can visualise the region $Q_1$ becoming increasingly fine, getting closer to a collection of connected line segments as $\delta$ decreases to 0.
This raises the natural question as to whether it is reasonable (or even possible) to approximate a photonic crystal through a \emph{quantum graph} problem; for now it is sufficient for us to imagine such problems as being composed of a graph $\graph$ whose edges $e$ are associated to intervals $\sqbracs{0,l_{e}}, l_{e}>0$, and an operator $\mathcal{G}$ whose action can be described through a differential expression on (the interval associated to) each edge, complimented by boundary or matching conditions at the vertices --- a more complete description and definition of quantum graph problems is provided in \tstk{chapter or section reference}.
The use of quantum graphs as approximations to physical processes and the study of the spectra of the associated operators has a rather rich history thanks \tstk{get refs for this, Kuch-Zheng is a good place to start} to interests from mesoscopic physics\footnote{Broadly speaking, this is a branch of condensed matter physics focusing on materials whose size ranges from a few molecules or atoms to micrometres.}.
\tstk{Although we do not delve into the details, such quantum graph problems are known to describe thin superconducting structures called ``quantum wires", ``molecular wires", and free-electron theory of conjugated molecules.}
In these contexts one is typically looking at a domain that is akin to a ``thickened graph" $G_{\delta}$ of width $\delta$; this is as it sounds with the graph edges being inflated into tubes or radius $\eps$ and the vertices expanded into junctions connecting the ends of these tubes.
In the interest of avoiding addressing several technicalities and detracting from the focus of our review, one can think of $G_{\delta}$ as being the surface of the resulting ``inflated" structure --- the illustration in figure \ref{fig:Diagram_ThickenedGraph} provides a sufficient visualisation.
\begin{figure}[b]
	\centering
	\includegraphics[scale=1.0]{Diagram_ThickenedGraph.pdf}
	\caption{\label{fig:Diagram_ThickenedGraph} A schematic illustration of a thickened graph $G_{\delta}$, consisting of tubes whose central axes are the edges of the underlying graph $\graph$, meeting at inflated vertex regions. The graph $\graph$ consists of the dashed edges and connecting vertex.}
\end{figure}
We can then define some (differential) operator $\mathcal{G}_{\delta}$ on $G_{\delta}$, and ask \emph{what quantum graph problem} $\mathcal{G}$ (if anything) describes $\mathcal{G}_{\delta}$ in the limit as $\delta\rightarrow0$ --- the main difficulty coming from the behaviour near the vertices of $\graph$, which in turn (effectively) determines the ``correct" boundary conditions at the vertices.
The study of \tstk{Kuchment-Zheng, themselves extending J. Rubinstein and M. Schatzman} established the convergence of the spectrum of the Neumann laplacian on $G_{\delta}$ to the spectrum of a quantum graph problem $\mathcal{G}$.
In particular, the vertex conditions associated with the ``limiting" spectral problem $\mathcal{G}u = \lambda u$ were continuity of $u$ at the vertices of $\graph$, and a Kirchoff condition at each vertex --- the sum of derivatives from the edges into each vertex had to total zero\footnote{This is a simplification in the interest of providing the reader with an overall idea of the key concepts. The study \tstk{Kuchment and Zheng} results in a weighted sum in the Kirchoff condition, with the weights coming from the aforementioned technicalities in the definition of $G_{\delta}$.}.
These conditions validated long-standing physical intuition and heuristic arguments that the aforementioned boundary conditions were the correct conditions to impose in the approximate model.

However, the studies \tstk{Exner-Post and Exner standalone} later demonstrated that this was not the complete story, and that the ``approximate" quantum graph problem ad a more general dependence on the original domain.
Specifically, the relative scaling of the volume of the inflated edges $V_{\mathrm{edge}}$ with $\delta$ to the volume of the expanded vertices (or junction regions) $V_{\mathrm{vertex}}$ determined the resulting boundary conditions in the limit quantum graph problem;
\begin{itemize}
	\item If $V_{\mathrm{edge}}\ll V_{\mathrm{vertex}}\rightarrow0$ as $\delta\rightarrow0$, then the resulting ``limit" problem is just a quantum graph problem with (homogeneous) Dirichlet conditions at each of the vertices.
	This case is referred to as Dirichlet decoupling; the resulting limit problem is just a collection of independent ODEs along the edges of the graph $\graph$, intuitively the result of the vertices being ``too big" when compared to the edges and thus preventing any interaction between the edges.
	\item If $V_{\mathrm{vertex}}\ll V_{\mathrm{edge}}\rightarrow0$ as $\delta\rightarrow0$, the edges dominate in the limit problem and one obtains the same boundary conditions as \tstk{Kuch-Zheng} found in their study.
	Intuitively, since the vertices ``disappear" before the edges as $\delta\rightarrow0$, the resulting problem (and it's solutions) must have strict matching conditions where the edges meet --- resulting in continuity and a net flux of 0 (Kirchoff condition) at each vertex.
	\item Of particular interest though is when $\frac{V_{\mathrm{edge}}}{V_{\mathrm{vertex}}}\rightarrow c>0$ as $\delta\rightarrow0$, that is when the volume of the tubes and junctions are of the same order of $\eps$.
	This is an intermediary for the other two cases, the effect of decoupling the edges (large junctions) is balanced by the need for consistency between edges (large tubes).
	The resulting vertex conditions still require continuity at each of the vertices, but one obtains a non-standard Kirchoff condition where the sum of fluxes into each vertex equals a ``coupling constant"\footnote{These constants are determined from the ratio $\frac{V_{\mathrm{edge}}}{V_{\mathrm{vertex}}}$.} multiplying the function value at that vertex (rather than 0).
\end{itemize}
This is an important illustration of why the non-dimensionalisation process should not be overlooked, since 
\begin{figure}[h]
	\centering
	\includegraphics[scale=1.0]{Diagram_InclusionAsTubes.pdf}
	\caption{\label{fig:Diagram_InclusionAsTubes}}
\end{figure}

Of particular interest is the final case, where a non-standard Kirchoff condition is obtained.
In such cases, one finds themselves facing a problem of the generalised resolvent type when this coupling constant depends on the spectral parameter, which \tstk{as we have seen?!?!?!} can result in band gaps in the spectrum (and possible other effects).

Let us return to the photonic crystal context, where the region $Q_1$ is represented by (the volume enclosed by) a thickened graph $\mathcal{G}_\delta$ and the material itself is under critical contrast.
It makes sense to analyse the asymptotic limit $\eps\rightarrow\infty, \delta\rightarrow0$ to attempt to understand the effective behaviour of the crystal.
The studies \tstk{Fig and Kuchment; papers Band-gap structure of the spectrum of periodic and acoustic media. I. Scalar model, SIAM J. Appl. Math., Band-gap structure of the spectrum of periodic
and acoustic media. II. 2D photonic crystals, SIAM J. Appl. Math., 56 (1996)} focused on a square geometry, with the region $Q_0$ being a cube of side $l=1-\delta$.

And as we have seen, \tstk{if you've already mentioned https://epubs.siam.org/doi/epdf/10.1137/1.9780898717594.ch7, this does a thorough study of the D-t-N map for a connected periodic graph in the plane, although check the definition. I don't think Kuchment actually writes down a QG problem for photonic fibres in the sense that we are looking for here.}  similar problems arise in the study of high contrast photonic crystals.
