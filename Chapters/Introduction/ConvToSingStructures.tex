\section{Singular Structures as a limit of Thin Structures} \label{sec:ConvToSS}

We have reviewed a number of studies into the existence of band-gaps for periodic media, and have also seen how a material under critical contrast can be shown --- using techniques from homogenisation theory --- to also exhibit dissipative properties.
Directly studying a periodic medium filling $\reals^d$ required each of $\delta:=1-l$, $a\delta^{-1}$, and $a^{-1}\delta^2$ to be small (simultaneously) to open up band gaps in the spectrum.
Additionally, ensuring that a material is under critical contrast allows us to be certain that its effective material properties allow for dissipative effects, including metamaterial behaviour and band-gaps.
\tstk{if we mentioned generalised resolvents in the previous section, they should also get a callback here.}
\tstk{linking sentence, also if we talk about $\delta\rightarrow0$, then it makes sense to introduce singular structures here, leading on from that.}

Graphically, the ``limit" $\delta\rightarrow0$ in these domain setups is rather intuitive --- one can visualise the region $Q_1$ becoming increasingly fine, getting closer to a collection of connected line segments as $\delta$ decreases to 0.
This raises the natural question as to whether it is reasonable (or even possible) to approximate a photonic crystal through a \emph{quantum graph} problem; for now it is sufficient for us to imagine such problems as being composed of a graph $\graph$ whose edges $e$ are associated to intervals $\sqbracs{0,l_{e}}, l_{e}>0$, and an operator $\mathcal{G}$ whose action can be described through a differential operator on (the interval associated to) each edge, complimented by boundary or matching conditions at the vertices --- a more complete description and definition of quantum graph problems is provided in \tstk{chapter or section reference}.
The use of quantum graphs as approximations to physical processes and the study of the spectra of the associated operators has a rather rich history thanks \tstk{get refs for this, Kuch-Zheng is a good place to start} to interests from mesoscopic physics\footnote{Broadly speaking, this is a branch of condensed matter physics focusing on materials whose size ranges from a few molecules or atoms to micrometres.}.
\tstk{Although we do not delve into the details, such quantum graph problems are known to describe thin superconducting structures called ``quantum wires", ``molecular wires", and free-electron theory of conjugated molecules.}
In these contexts one is typically looking at a domain that is akin to a ``thickened graph" $G_{\eps}$ of width $\eps$; in the interest of avoiding introducing cumbersome notation, one can think of $G_{\eps}$ as being the surface of a collection of tubes of radius $\eps$ whose central axis corresponds to the edges of a graph $\graph$, and which meet at ``inflated vertices" or vertex ``regions" --- the illustration in figure \ref{fig:Diagram_ThickenedGraph} provides a visualisation that will be sufficient for our review here.
\begin{figure}[b]
	\centering
	\includegraphics[scale=1.0]{Diagram_ThickenedGraph.pdf}
	\caption{\label{fig:Diagram_ThickenedGraph} A schematic illustration of a thickened graph $G_{\eps}$, consisting of tubes whose central axes are the edges of the underlying graph $\graph$, meeting at inflated vertex regions. The graph $\graph$ consists of the dashed edges and connecting vertex.}
\end{figure}
We can then define some (differential) operator $\mathcal{G}_{\eps}$ on $G_{\eps}$, and ask \emph{what quantum graph problem} $\mathcal{G}$ (if anything) describes $\mathcal{G}_{\eps}$ in the limit as $\eps\rightarrow0$ --- the main difficulty coming from the behaviour near the vertices of $\graph$, which in turn (effectively) determines the ``correct" boundary conditions at the vertices.
The study of \tstk{Kuchment-Zheng, themselves extending J. Rubinstein and M. Schatzman} established the convergence of the spectrum of the Neumann laplacian on $G_{\eps}$ to the spectrum of a quantum graph problem $\mathcal{G}$.
In particular, the vertex conditions associated with the ``limiting" spectral problem $\mathcal{G}u = \lambda u$ were continuity of $u$ at the vertices of $\graph$, and a Kirchoff condition at each vertex --- the sum of derivatives from the edges into each vertex had to total zero\footnote{This is a simplification in the interest of providing the reader with an overall idea of the key concepts. The study \tstk{Kuchment and Zheng} results in a weighted sum in the Kirchoff condition, with the weights coming from the aforementioned technicalities in the definition of $G_{\eps}$.}.
These conditions validated long-standing physical intuition and heuristic arguments that the aforementioned boundary conditions were the correct conditions to impose in the approximate model.
However, the studies \tstk{Exner-Post and Exner standalone} later demonstrated that this was not the complete story, and that the correct ``approximate" quantum graph problem depended upon the geometry of the domain $G_{\eps}$.
Specifically, the relative scaling of the volume of the thickened edges $V_{\mathrm{edge}}$ with $\varepsilon$ to that of the thickened vertices $V_{\mathrm{vertex}}$ dramatically effects the resulting boundary conditions in the limit quantum graph problem. \tstk{GOT TO HERE}
Rather than go into details, the crux of this argument
 rather than a domain with a thickened graph as an \emph{inclusion}, as is the case for a photonic crystal.

And as we have seen, \tstk{if you've already mentioned https://epubs.siam.org/doi/epdf/10.1137/1.9780898717594.ch7, this does a thorough study of the D-t-N map for a connected periodic graph in the plane, although check the definition. I don't think Kuchment actually writes down a QG problem for photonic fibres in the sense that we are looking for here.}  similar problems arise in the study of high contrast photonic crystals.
