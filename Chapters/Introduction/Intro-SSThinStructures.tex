\subsection{Thin structures in the zero-thickness limit} \label{ssec:Intro-ThinStructures}
Graphically, the ``limit" $\delta\rightarrow0$ (with the other parameters fixed) in the domain setups of section \ref{sec:Intro-Maxwell} is rather intuitive; one can visualise the region $Q_1$ becoming increasingly fine, getting closer to a collection of connected line segments as $\delta$ decreases to 0.
The resulting (singular) structure is a region embedded into a space in which it has no volume (or area in two dimensions) --- figure \ref{fig:Diagram_ShrinkToSingularSquareCell} illustrates this process for a 2-dimensional geometry that is extruded into 3 dimensions like a PCF.
\begin{figure}[t]
	\centering
	\includegraphics[scale=0.5]{Diagram_ShrinkToSingularSquareCell.pdf}
	\caption[Visual illustration of the singular limit of a thin-structure geometry.]{\label{fig:Diagram_ShrinkToSingularSquareCell} A visual illustration of the $\delta\rightarrow0$ limit in a ``fibre-like" geometry. The cross sectional geometry (dark cross) shrinks to a collection of line segments, resulting in a union of planes whose cross-section is a graph-like structure.}
\end{figure}
The remainder of $\reals^d$ upon taking the ``limit" $\delta\rightarrow0$ is filled by the inclusion $Q_0$.
This raises the natural question as to whether it is reasonable (or even possible) to appeal to this visual intuition and approximate a PC by a \emph{singular structure} embedded into a 2 or 3 dimensional background material.
Furthermore, we can ask what kinds of behaviours emerge as a result of part of our structure shrinking to zero-thickness, and (rather importantly) how these ``limits" are realised in the first place.

Relevant to both of these questions and our studies in chapters \ref{ch:ScalarSystem} and \ref{ch:CurlCurl} will be the study of so called \emph{thin-structures} in the zero-thickness limit.
Such thin-structures $G_{\delta}$ consist only of the region (formed by translations of) $Q_1$, that is $G_{\delta} = \bigcup_{l\in\integers^d}(l+Q_1)$.
In almost all contexts one is typically looking at a thin-structure domain that is akin to a ``thickened graph"; one takes a graph $\graph$, embeds it into $\reals^d$, and inflates the edges into tubes of radius $\sim\delta$ and the vertices into junction regions (whose size also depends on $\delta$) connecting the ends of these tubes.
In the interest of avoiding addressing several technicalities and detracting from the focus of our review, one can think of $G_{\delta}$ as being the volume enclosed by the resulting ``inflated" structure --- the illustration in figure \ref{fig:Diagram_ThickenedGraph} provides a sufficient visualisation.
\begin{figure}[b]
	\centering
	\includegraphics[scale=1.0]{Diagram_ThickenedGraph.pdf}
	\caption[Illustration of a ``thickened graph" thin-structure domain.]{\label{fig:Diagram_ThickenedGraph} A schematic illustration of a thickened graph $G_{\delta}$, consisting of tubes whose central axes are the edges of the underlying graph $\graph$, meeting at inflated vertex regions. The graph $\graph$ consists of the dashed edges and connecting vertex. Diagram adapted from \cite[Figure 1]{exner2005convergence}.}
\end{figure}

By design, the thin-structure $G_{\delta}$ converges in the eyeball norm as $\delta\rightarrow0$ to the (metric) graph $\graph$, whose edges are correspond to the central axes of the tubes and whose vertices correspond to the centre of the collapsing junction regions.
Since $\graph$ is embedded into real space, each edge can be assigned a length and consequentially functions on such graphs can be defined through their behaviour on each edge and their values at the vertices which connect the edges.
This also means that $\graph$ can be equipped with a differential operator on each of the edges, complimented by boundary conditions at each of the vertices.
Standard Neumann or Dirichlet conditions on a function $u$ at each vertex can be used for these boundary conditions, however these tend to neglect (or to an extent, suppress) the underlying graph structure, which allows for more adventurous conditions to be used.
Conditions that are more commonly used in the approximation of physical systems are are continuity of the function $u$ though the vertices\footnote{That is whenever two edges share a common vertex, the traces into the vertex from each of these edges must coincide.} and the Kirchoff condition,
\begin{align*}
	\sum_{\text{vertices } v_k \text{ that connect to } v_j} 
	\pdiff{u^{(jk)}}{n}\bracs{v_j} = 0, %\alpha_j u(v_j),
\end{align*}
where $\pdiff{u^{(jk)}}{n}\bracs{v_j}$ denotes the \emph{incoming} derivative to the vertex $v_j$. 
The Kirchoff condition corresponds relates the flux at each vertex to the function value; when the coupling constant (or function value) at the vertex is zero, it's form and interpretation is analogous to Kirchoff's first law of ``net zero current" through junctions in electric circuits.
A (metric) graph equipped with such an operator is called a \emph{quantum graph}, and the system of equations and ``boundary conditions" defined by this operator a \emph{quantum graph problem} --- we provide a more precise description and introduction in section \ref{sec:QuantumGraphs}, and a more complete overview of the field can be found in \cite{berkolaiko2013introduction}.
It is sufficient for this introduction to think of such a quantum graph problem as consisting of an ODE on each edge of $\graph$, coupled through matching conditions at the vertices.

The use of quantum graphs as approximations to physical processes on thin (graph-like) structures has a rather rich history thanks to interests from molecular chemistry, semi-conductor physics, and the design of nano-structures.
In each of these situations the domain of the physical system is akin to a thickened graph and are governed by Laplacian-like equations.
Given that the thickness $\delta$ is significantly smaller than the lengths of each of the inflated edges, it is natural to ask in such situations whether the ``one-dimensional" graph model is a good approximation to the higher-dimensional physical model on a thin-structure.
The early work of \cite{ruedenberg1953free} put forward a heuristic argument for approximation of such systems through a quantum graph problem with Kirchoff conditions at the vertices --- although their assumptions would later turn out to be false, they still achieved reasonably accurate results.
Interest in the convergence properties of Laplacians on thickened graphs to quantum graph problems has continued regardless, due to the potential to learn useful information about a physical system from ``cheap" one-dimensional models.
The studies \cite{rubinstein2001variational, kuchment2001convergence} established convergence of the spectrum of the Neumann Laplacian on $G_{\delta}$ (which we will denote by $\laplacian_{\delta}$) to the spectrum of a quantum graph problem (defined by an operator we denote by $\laplacian_0$) with Kirchoff boundary conditions at the vertices\footnote{This is a simplification in the interest of providing the reader with an overall idea of the key concepts. The study \cite{kuchment2001convergence} allowed for a slightly more general, weighted domain $G_{\delta}$ which results in a weighted sum in the Kirchoff condition, and also allows the presence of an external field.}.
Precisely, the result
\begin{align*}
	\lim_{\delta\rightarrow0} \lambda_n\bracs{\laplacian_{\delta}} = \lambda_n\bracs{\laplacian_0}
\end{align*}
was proved for every $n\in\naturals$, where $\lambda_n\bracs{\laplacian_{\delta}}$ and $\lambda_n\bracs{\laplacian_0}$ are the eigenvalues of the respective operators ordered in ascending order.

The study \cite{kuchment2003asymptotics} later demonstrated that this was not the only possible behaviour in the zero-thickness limit.
Starting from the domain $G_{\delta}$, one can define the volume of the inflated edges $V_{\mathrm{edge}}$ and volume of the junction regions $V_{\mathrm{vertex}}$ as functions of the thickness $\delta$.
In the limit $\delta\rightarrow0$, the operator $\laplacian_0$ (whose spectrum coincides with the limit of the spectrum of $\laplacian_\delta$) depends on the relative scaling between $V_{\mathrm{vertex}}$ and $V_{\mathrm{edge}}$.
The later study \cite{exner2005convergence} would extend\footnote{The study \cite{kuchment2003asymptotics} elects to work with a scaling parameter rather than directly with the ratio of volumes, and it is the study \cite{exner2005convergence} that first works with ratios of volumes. However in the interest of providing a simplified review, we have rephrased the key ideas into this common language.} these results to Neumann Laplacians on graph-like manifolds (domains more general than the thickened graphs we have introduced here).
The following limiting behaviours are possible for the Neumann Laplacian:
\begin{itemize}
	\item (``Thick vertex" setup) If $V_{\mathrm{edge}}\ll V_{\mathrm{vertex}}\rightarrow0$ as $\delta\rightarrow0$, then $\laplacian_0$ defines a quantum graph problem with (homogeneous) Dirichlet conditions at each of the vertices.
	This case is referred to as Dirichlet decoupling; the resulting limit problem is just a collection of independent ODEs along the edges of the graph $\graph$, intuitively the result of the vertices being ``too big" when compared to the edges and thus preventing any interaction between the edges.
	\item (``Thick edge" setup) If $V_{\mathrm{vertex}}\ll V_{\mathrm{edge}}\rightarrow0$ as $\delta\rightarrow0$, the edges dominate in the limit problem and one obtains an operator $\laplacian_0$ on the graph with Kirchoff boundary conditions, as in the study \cite{kuchment2001convergence}.
	Intuitively, since the vertices ``disappear" before the edges as $\delta\rightarrow0$, the resulting problem (and it's solutions) must have strict matching conditions where the edges meet --- resulting in continuity and a net flux of 0 at each vertex.
	\item (``Borderline/critical" case) Of particular interest is when $\frac{V_{\mathrm{edge}}}{V_{\mathrm{vertex}}}\rightarrow c>0$ as $\delta\rightarrow0$, that is when the volume of the tubes and junctions are of the same order of $\delta$.
	This is an intermediary for the other two cases, the effect of decoupling the edges (thick vertices) is balanced by the need for consistency (of the solution) between the incoming edges (thick edges).
	The spectral problem for the resulting operator $\laplacian_0$ can realised as a quantum graph problem whose vertex conditions are continuity at each of the vertices, but along with a ``non-standard" Kirchoff condition (also called a \emph{Wentzell} condition) at the vertices,
	\begin{align*}
	\sum_{v_k \text{ connects to } v_j} 
	\pdiff{u^{(jk)}}{n}\bracs{v_j} = \lambda\alpha_j u(v_j).
	\end{align*}
	Here, $\alpha_j$ is a ``coupling constant" at the vertex $v_j$ (related to the edge and vertex volumes) and $\lambda$ the spectral parameter --- in physical terms, one can say that the strength of any coupling at a vertex is proportional to the system's energy.
\end{itemize}

All of these results concern convergence of the spectrum of the Neumann Laplacian on a thin structure to a quantum graph problem.
The question of resolvent convergence is addressed in the monograph \cite{post2012spectral}; since the limiting operator $\laplacian_0$ acts on a different space of functions to the operators $\laplacian_{\delta}$ in the borderline case, resolvent convergence must be defined through the use of identification operators (\cite[section 4]{post2012spectral}), and the notion of convergence rate similarly tied to this definition.
This study then goes on to establish resolvent convergence for the Neumann Laplacian on graph-like manifolds, and providing rates of convergence for each of the cases listed above \cite[theorem 1.3.2]{post2012spectral}.
More recently, convergence rates for the resolvent in the borderline case have been investigated in \cite{cherednichenko2022norm}, again through the use of identification operators.
Convergence rates of $O(\delta\ln\delta)$ with the thickness $\delta$ (for two-dimensional planar structures) and $O(\delta)$ for thickened-graph structures in three dimensions have been demonstrated.
%It is also worth mentioning the earlier study \cite{freidlin1993diffusion} who considered a related problem on a thickened graph within the context of probability, and proved pointwise convergence of the resolvent and \cite{saito2000limiting} who proved weak convergence of the resolvent when the underlying graph possessing no loops or cycles (``tree-like").
Analysis concerning the Dirichlet Laplacian is undertaken in parallel to the Neumann Laplacian, also presented in \cite{post2012spectral}; in this case the analysis is more complicated and either a rescaling of the operator, or an asymptotic expansion of the eigenvalues in terms of $\delta$, is necessary --- we refer the reader to the aforementioned study for further information.

We should elaborate further on the operator $\laplacian_0$ in the borderline case; it is not an operator on a quantum graph per say, but its spectral problem can be interpreted as a quantum graph problem with the spectral parameter appearing in the boundary (or vertex) conditions.
The operator $\laplacian_0$ does not act in a Hilbert space $\mathcal{H}$ of functions on metric graphs (see \eqref{eq:GraphFuncSpaces}), but rather in an \emph{extended space} $\mathcal{H}^+ = \mathcal{H}\oplus\complex^N$ (where $N$ is typically the number of vertices in the underlying graph).
The quantum graph problem with Wentzell conditions at the vertices is a realisation of the spectral problem for $\laplacian_0$ as a problem on a quantum graph, and belongs to the class of problems with generalised resolvents, with $\laplacian_0$ the so-called \emph{Strauss extension} of this realisation.
The solution $u\in\mathcal{H}$ to the quantum graph problem with Wentzell conditions (defined through $\laplacian_0 u = f$) can be expressed in terms of the resolvent of $\laplacian_0$ sandwiched by projections from $\mathcal{H}^+$ onto $\mathcal{H}$ and back.
Further details and information on generalised resolvents and extensions of symmetric operators can be found in \cite{strauss1999function} --- we will not delve further into the available theory on this topic, although we shall see in section \ref{ssec:ExtendedSpaces} our approach via singular structures provides a natural construction of these extended spaces.

The takeaway message is that the resulting ``limit" problem is not as simple to determine as the ``limit of the geometry", even determining the extension from the quantum graph problem is also not a simple task.
One typically needs to know \emph{a priori} what the limiting problem will be, and then attempt to prove convergence using this guess.
Our approach using singular structures and variational problems will provide a natural bridge to this gap --- providing both an intuitive starting point which coincides with the extended spaces one wants to work with, and enabling us to obtain the resulting limiting quantum graph problems with relative ease.

To round off our discussion, we highlight that at present there are no results which extend the studies \cite{kuchment2001convergence, kuchment2003asymptotics, exner2005convergence} into the Maxwell setting (that is, to either the curl-of-the-curl equation or the full Maxwell system).
By contrast, there has been considerable interest in the study of the equations of elasticity on thin-structures and the resulting ``singular" limits.
The work \cite{zhikov2002homogenization} studied the problem of homogenisation for periodic thin-structure of thickness $\delta$, looking to derive the resulting limiting (or ``effective") problem in the long-wavelength ($\tilde{L}$ small) regime.
In this work determination of the effective problem in the so-called critical scaling regime $\delta=O(\tilde{L})$ of the thickness parameter $\delta$ with the period cell size $\tilde{L}$ was noted as being an open problem, with the resulting effective problem being different to that in the non-critical scaling regimes (similar to the borderline case detailed above for Laplacians).
This work was followed up in \cite{zhikov2003homogenization}, in which these differences were identified and the effective problem in the critical setting presented.
A method for passing to the limit in the equations of elasticity on a bounded thin-structure domain as the thickness $\delta\rightarrow0$ was also developed in \cite{zhikov2006derivation}, the resulting system being a quantum graph problem with Kirchoff matching conditions at each of the vertices.
We highlight these results because these studies do not account for the possible ``borderline" scaling between the thickened edges and vertices --- only the edge thickness $\delta$ is present in the studies, and the vertex volume as a function of $\delta$ is not considered.
For this reason it is likely that results similar to those presented by \cite{exner2005convergence} can be derived within the context of elasticity too, however this remains unexplored in the literature.

The work of chapters \ref{ch:ScalarSystem} and \ref{ch:CurlCurl} will be particularly relevant to the aforementioned studies and the ``limiting" quantum graph problems.
In these chapters we will take the ``limiting" singular structure as our starting point, and pose a variational problem upon this singular structure.
This will require us to setup and study appropriate function spaces (section \ref{sec:BorelMeasSobSpaces}), before then demonstrating the relationship between the abstract variational problem and the various ``limiting" quantum graph problems discussed above.
We will also go into detail about how one can explicitly solve the resulting quantum graph problems, before looking to extend our results into the Maxwell setting, and indicating some of the challenges that this brings.
The theory we establish in these chapters will also be valuable to us in chapter \ref{ch:SingInc}, when we come to study a ``PC with singular inclusions".