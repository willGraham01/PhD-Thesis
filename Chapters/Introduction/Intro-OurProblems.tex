\section{Research Overview} \label{sec:Intro-ProblemIntroduction}

\tstk{This section will serve as the link between our literature review and the work that is carried out in the thesis. It is a conclusion of sorts to the introduction, as it should set the reader up for any of the following chapters, and the ``information" chapter.}

As we have already mentioned in our review \tstk{previous sections}, our interest lies in studying problems involving so-called singular structures, akin to the visual ``limits" discussed in the aforementioned sections.
Whilst we postpone the more involved and formal definitions to chapter \tstk{theory chapter}; here we provide the reader with an overview of the various problems that are considered, some further motivation and ideas behind why we consider such problems, and then direct them to the details in the subsequent sections.
Before we begin, let us establish some notation and concepts that will be integral to our discussion, foremost amongst them the notion of a singular structure.
A singular structure is a subset $S\subset\reals^d$ with $d$-dimensional Lebesgue measure is zero, perhaps more useful is its description as a region or object which has no area from the perspective of the space into which is is embedded.
We will always choose to represent singular structures through a graph $\graph=\bracs{\vertSet,\edgeSet}$, where each vertex is associated with a point in $\reals^d$ and each edge to the line segment joining the (points associated to the) vertices at each endpoint.
In addition, we will place a ``coupling constant" $\alpha_j$ at each $v_j\in\vertSet$, which we think of as playing the role of the ratio $V_{\mathrm{edge}} / V_{\mathrm{vertex}}$ as described in section \ref{ssec:Intro-ThinStructures}.

Now that we have a singular structure, we then have to ask how we can define differential equations (or more precisely, operators) on them.
To motivate this thought process, let us take a singular structure (represented by a graph) $\graph$ and consider the more familiar setup of a differential equation complimented by boundary conditions on the corresponding thickened graph $G_{\delta}$.
For illustrative purposes, we will consider the system
\begin{subequations} \label{eq:Intro-ThinToSingularSys}
	\begin{align}
		-\grad\cdot\grad u &= f, \qquad x\in G_{\delta}, \label{eq:Intro-ThinToSingularDE} \\
		\pdiff{u}{n}\big\vert_{\partial G_{\delta}} &= 0. \label{eq:Intro-ThinToSingularBC}
	\end{align}
\end{subequations}
Now consider which pieces of this formulation either change or cease to have clear meaning as $\delta$ shrinks to zero:
\begin{enumerate}
	\item The most notable effect is the loss of a distinction between the boundary $\partial G_{\delta}$ and the interior of $G_{\delta}$ when $\delta$ ``reaches" zero and becomes the singular structure $\graph$ --- the edges of this graph correspond to the overlap between what was previously two parts of the boundary of $G_{\delta}$.
	\item Furthermore, the graph $\graph$ has no interior to speak of and so the ``differential equation" \eqref{eq:Intro-ThinToSingularDE} becomes meaningless.
	Where or how do we pose this if our new domain doesn't have any interior?
\end{enumerate}
We can address our first observation by realising that it is only our explicit separation of the boundary condition \eqref{eq:Intro-ThinToSingularBC} and differential equation \eqref{eq:Intro-ThinToSingularDE} that leads to this problem.
Instead we can consider the variational formulation for \eqref{eq:Intro-ThinToSingularSys}, to find $u$ such that
\begin{align} \label{eq:Intro-ThinToSingularVar}
	\integral{G_{\delta}}{ \grad u\cdot \overline{\grad\phi} }{x} 
	&= \integral{G_{\delta}}{ f\overline{\phi} }{x}, \qquad\forall\phi\in V_{\mathrm{test}},
\end{align}
for some suitable set of test functions $V_{\mathrm{test}}$ (typically the set of smooth functions on the domain $G_{\delta}$).
Notably the equation \eqref{eq:Intro-ThinToSingularVar} ``combines" the boundary condition and differential equation in \eqref{eq:Intro-ThinToSingularSys}, in the sense that one can obtain \eqref{eq:Intro-ThinToSingularDE} by considering the test functions whose support does not intersect the boundary of $G_{\delta}$, then \eqref{eq:Intro-ThinToSingularBC} from those that do.
This would potentially allow us to circumnavigate the coincidence of the interior of our singular structure with what was previously the boundary of our thin structure, however we are not out of the woods yet.
If we simply replace the domain of integration $G_{\delta}$ with $\graph$ in \eqref{eq:Intro-ThinToSingularVar}, both sides of the equation are identically zero in lieu of $\graph$s singular nature --- which is another incarnation of the second observation we made.
To address this we have to stop viewing our singular structure from the perspective of the Lebesgue measure, and instead properly view our singular structure as an embedded object in a higher dimensional space.
This leads us to introduce the notion of singular measures that support graphs embedded into $\reals^d$, and pose variational problems of the form \eqref{eq:Intro-ThinToSingularVar} with respect to these measures, and on the singular domain $\graph$. \tstk{section ref}
Changing measure in this way forces us to change the notion of area, and thus integration and correspondingly differentiation too.
The classical gradient is no longer appropriate for studying problems on singular structures, and again there is difficulty in defining this object in the classical sense involving limits.
Similar problems emerge for the interpretation of the curl and divergence of vector fields.
However, if we can overcome these final difficulties, the resulting variational problems we are left with have a striking resemblance to their classical analogues --- in fact, the only differences being the measure in the variational problem and the ``Sobolev space" that our solution $u$ belongs to.
This motivates our study of the so-called ``non-classical" Sobolev spaces \tstk{section?} and later generalisation to curls and divergences in chapter \tstk{ref!} and ``composite materials" in chapter \tstk{ref!}.

We now outline the arrangement of the content of this work. In chapter....
