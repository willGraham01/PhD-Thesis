\section{Domains and Singular Structures} \label{sec:SingularStructures}

\tstk{making this section just so that I can definitively have a place where whenever I talk about ``one of our usual domains", I can point the reader back to here.}

The domains that we will come to consider are motivated by applications in photonics, and from photonic crystals specifically.
As such, the domains we choose to consider will reflect the structure of such a material, however possess a ``singular component" in place of the fine or `thin-structure" that a physical photonic crystal exhibits.
Precisely, in this work we consider a \emph{photonic crystal} to be a 2 dimensional, periodic material in $\reals^2$ that is then extruded into $\reals^3$, see figure \ref{fig:Diagram_ShrinkToSingularSquareCell}.
The axis along which the extrusion takes place is the \emph{fibre axis}.
\begin{figure}
	\centering
	\includegraphics[scale=0.5]{Diagram_ShrinkToSingularSquareCell.pdf}
	\caption{\label{fig:Diagram_ShrinkToSingularSquareCell} The ``singular structures" that form the domains (or pieces of the domains) of the problems that we will consider. On the left, the period cell of a 2 dimensional periodic material extruded into 3 dimensions. On the right, the corresponding singular-structure that comes about as a ``limit" of the physical structure.}
\end{figure}
Typically when faced with (a differential equation on a) domain like that described, one uses a Fourier transform along the fibre axis and a Gelfand transform (or alternatively a Floquet transform) in the periodic directions to reduce the complexity in analysing the problem.
Such a procedure would bring us back to a problem on the unit/period cell (in $\reals^2$) of the periodic structure, with artefacts from the transforms we applied.
However, we are interested in studying wave-propagation problems on so called \emph{singular structures} --- regions which possess no interior (or have no area) from the perspective of the space they are embedded into.
The geometry of our singular structures will reflect that of the aforementioned thin-structure materials studied in photonics, \tstk{limit argument mention here}.
The natural way for us to do this is as follows: take a (periodic) metric graph in $\reals^2$ (see section \ref{ssec:EmbeddedGraphs}), and extrude the edges of this graph into the $x_3$ direction to create a 3-dimensional structure which is a union of planes induced by the graph edges.
This creates a region that mimics the structure of a photonic crystal (or a photonic fibre yet to have a core drilled out), with an ``infinitely thin", periodic cross-sectional pattern.
Of course, we then apply a Fourier transform along the fibre axis, and a Gelfand transform (or Floquet \tstk{serena and kirill use this one}) in the periodic cross-section to reduce the problem to a family of more manageable problems on the unit cell.

As such, the domains that will be considered will consist of a metric graph $\graph$ embedded into a region $\ddom\subset\reals^2$, which should also be thought of as the ``unit cell" of the periodic singular structure.
The graph $\graph$ itself represents the singular structure we are interested in working on (or more precisely, the cross-section of the singular structure included by the extrusion of $\graph$).
We will write ``a domain $\ddom$" (or similar) when we wish to refer to such a region $\ddom$ containing an embedded metric graph as just described, and ``a singular structure" to refer to the graph contained in such a domain.
Although $\ddom$ is only a subset of $\reals^2$ thanks to the use of a Fourier transform, it is important to remember that the graph $\graph$ represents a union of planes in 3 dimensions, particularly when we come to define (and interpret) gradient-like operators on functions defined on these singular structures.
Of course, wanting to pose ``differential equations" on a domain with no interior raises several questions as to how one can understand the notion of a derivative --- clearly the ``usual" 2-dimensional definition of a gradient will not suffice for a singular structure!
This leads us to define and study Sobolev spaces of functions with gradients (and later curls) with respect to a Borel measure, as in section \tstk{section ref!}
