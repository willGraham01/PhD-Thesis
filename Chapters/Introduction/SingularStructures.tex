\section{Domains and Singular Structures} \label{sec:SingularStructures}

\tstk{making this section just so that I can definitively have a place where whenever I talk about ``one of our usual domains", I can point the reader back to here.}

The domains that we will come to consider are motivated by applications in photonics, and from photonic crystals specifically.
As such, the domains we choose to consider will reflect the structure of such a material, however possess a ``singular component" in place of the fine or `thin-structure" that a photonic crystal typically exhibits.
Precisely, in this work we consider a \emph{photonic crystal} to be a 2 dimensional, periodic material in $\reals^2$ that is then extruded into $\reals^3$, see figure \ref{fig:Diagram_ShrinkToSingularSquareCell}.
The axis along which the extrusion takes place is the \emph{fibre axis}.
\begin{figure}
	\centering
	\includegraphics[scale=0.5]{Diagram_ShrinkToSingularSquareCell.pdf}
	\caption{\label{fig:Diagram_ShrinkToSingularSquareCell} The ``singular structures" that form the domains (or pieces of the domains) of the problems that we will consider. On the left, the period cell of a 2 dimensional periodic material extruded into 3 dimensions. On the right, the corresponding singular-structure that comes about as a ``limit" of the physical structure.}
\end{figure}
Typically when faced with (a differential equation on a) domain like that described, one uses a Fourier transform along the fibre axis and a Floquet or Gelfand transform in the periodic directions to reduce the complexity in analysing the problem.
In this instance, such a procedure would bring us back to a problem on the unit/period cell (in $\reals^2$) of the periodic structure, with artefacts from the transforms we applied.
However, we are interested in studying wave-propagation problems on so called \emph{singular structures}.
These are domains which contain no interior (or have no area) from the perspective of the space they are embedded into.
We will chose to represent singular structure by metric graphs embedded into $\reals^2$ (see section \ref{ssec:EmbeddedGraphs}), which again we can extrude into $\reals^3$ to provide a structure formed of a union of planes, and mimicking the structure of a photonic crystal (or a photonic fibre yet to have a core drilled out).
Of course, we then apply a Fourier transform along the fibre axis and a Gelfand transform (or Floquet \tstk{serena and kirill use this one}) in the periodic cross-section to reduce the problem to a family of more manageable problems on the unit cell, as described above.

As such, the domains we will consider in this work will be some set $\graph$ with empty interior, represented as an embedded graph, contained in a (bounded) subset of $\reals^2$ (the unit cell of a periodic domain).
Wanting to pose ``differential equations" on a domain with no interior raises several questions as to how one can understand the notation of a derivative (and other derivative-like operations like the curl) on such a structure, and \tstk{boundary conditions, how to start???}