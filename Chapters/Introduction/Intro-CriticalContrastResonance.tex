\subsection{Critical, or High Contrast Composites} \label{ssec:Intro-CritContrast}
We have seen that there are a variety of effective problems that one can obtain from the $\delta\rightarrow0$ limit of a thin structure in section \ref{ssec:Intro-ThinStructures}.
Now we put the shoe on the other foot, and focus on the various behaviours that can emerge when considering asymptotic limits of the material properties (or rather, limits of the dimensionless quantities that are derived from them).
The studies discussed here typically concern periodic, composite materials whose domains are described in the manner introduced in section \ref{ssec:Intro-NonDimMax}, with the size of the inclusion $\delta$ fixed.
Such domains are more reflective of physical photonic crystals than the thin domains considered previously in section \ref{ssec:Intro-ThinStructures}.
Despite our interest being primarily in singular structures and the resulting spectra of operators on them, it is nonetheless important for us to highlight how asymptotic models for materials at ``critical-contrast" can also give rise to dispersive effects.
Indeed, we shall see that the dispersive effects that arise from these models mirror those that emerge from limits of thin structures.

Studies of such materials typically rely on techniques from homogenisation theory to determine ``effective" material properties in the asymptotic limit of interest.
For example, consider the operator $\mathcal{A}=-\grad\cdot A\grad$ on such a composite domain --- the (possibly matrix-valued) function $A$ describes the ``non-dimensional" material parameters, and throughout we will assume it is $Q$-periodic.
One regime of interest is the ``long wave" (or low frequency) regime, where one attempts to garner information about the lowest spectral band of a composite material.
Accordingly, the (small) parameter $\eps\ll 1$ is introduced and we consider the (sequence of) operators $\mathcal{A}^{\eps} = -\grad\cdot A\bracs{\eps^{-1}x}\grad$ in the limit $\eps\rightarrow0$.
We then derive an ``effective" operator $\mathcal{A}^{\mathrm{hom}} = -\grad\cdot A^{\mathrm{hom}}\grad$ that describes the resulting limit, and look to formalise the relationship between $\mathcal{A}^{\mathrm{hom}}$ and the $\mathcal{A}^{\eps}$.
The matrix $A^{\mathrm{hom}}$ is sometimes labelled or interpreted as the ``effective" material properties.
Typically one desires convergence of $\mathcal{A}^{\eps}$ to $\mathcal{A}^{\mathrm{hom}}$ as $\eps\rightarrow0$ in the norm-resolvent sense, namely that the family of solutions $u_{\eps}$ to the problems $\mathcal{A}^{\eps}u_{\eps}=f_{\eps}$ converges to the solution $u$ of the effective problem $\mathcal{A}^{\mathrm{hom}}u = f$, for any suitable initial data $f_{\eps}\rightarrow f$.
However one may also establish weaker results, for example coincidence of the spectrum of $\mathcal{A}^{\mathrm{hom}}$ with the limit of $\sigma\bracs{\mathcal{A}^{\eps}}$ (akin to the results for thin structures discussed in section \ref{ssec:Intro-ThinStructures}) which would otherwise be implied by norm-resolvent convergence.
This convergence in turn ensures that the effective problem faithfully describes the behaviour of the composite material within this regime, and one typically hopes that the effective problem is more mailable to analysis than its counterpart.
These studies can also reveal interesting material effects within such regimes that might otherwise be lost or go unnoticed.
The design of materials to open spectral band gaps is one such effect that is desired, however other dispersive effects such as metamaterial behaviour --- where the effective material properties appear to take unphysical values --- can also emerge from these limits.
In these settings, one is typically expecting to obtain an effective problem where the effective material properties $A^{\mathrm{hom}}$ depend non-trivially on the frequency variable $\omega$.

The asymptotic treatment of composite materials has seen considerable interest due to the aforementioned motivations.
However, it has been known to both the mathematical and physical community for some time that the toolbox of standard homogenisation theory is unable to capture the correct asymptotic behaviour for composite materials.
It has been reported (\cite{cherednichenko2019unified}, in reference to the results of \cite{birman2004second} and \tstk{V. V. Zhikov, Spectral approach to asymptotic diffusion problems. (Russian), Differentsial’nye uravneniya 25 (1989), no. 1, 44–50. }) that if the matrix $\mathcal{A}(\eps^{-1}x)$ and its inverse are uniformly bounded (``uniformly elliptic" setting), that the resulting effective problem has $A^{\mathrm{hom}}$ being a constant matrix, which leaves no room for such effects to emerge.
If one drops the assumption of uniform ellipticity, the resulting analysis becomes more complicated and the standard toolbox of asymptotic study is no longer sufficient for describing the effective problem.
Indeed the inadequacy of standard techniques in such contexts has been reported in works such as \cite{nicorovici1995photonic}, which highlighted ``non-commuting limits" when considering wave scattering off a periodic composite material.
In such a problem there are two scales to consider; the wavenumber of the incident wave $\wavenumber$ and the refractive index of the inclusions within the composite, $N$ (which is proportional to the square root of the relative electric permittivity of the inclusions, to maintain a connection to the dimensionless quantities introduced in section \ref{sec:Intro-Maxwell}).
In order to study the first spectral band\footnote{This is also referred to as the lowest spectral band, or the ``acoustic band" in the literature.} in the case when the inclusions are highly conducting; we can either consider the case when the inclusions are perfectly conducting ($N\rightarrow\infty$) for fixed $\wavenumber$ and then take the ``static-limit" $\wavenumber\rightarrow0$, \emph{or} consider the static limit ($\wavenumber\rightarrow0$) at a high but finite refractive index and then take $N\rightarrow\infty$.
However the two approaches produce different effective materials --- the explanation provided in \cite{movchan2001noncommuting} is that the trajectory in $\bracs{N^{-1},\wavenumber}$-space that one takes as these values approach the origin must be of the form $N^{-1}=\wavenumber^p$, with $p<1$ for homogenisation to be satisfactory, otherwise the problem being considered is ``singularly perturbed" and boundary layer effects emerge near the inclusion-matrix interface which are not handled by the homogenisation process.
However, there have been a number of developments aimed at expanding the tools of homogenisation theory to handle contexts such as the above, where the material parameters scale in particular relation to the size of an incident wave or the size of the period cell.

Let us begin with the study of \cite{zhikov2000extension}, that was concerned with the homogenisation problem for the operator $\mathcal{A}^{\eps}$,
\begin{align*}
	\mathcal{A}^{\eps} := -\grad\cdot A^{\eps}(\eps^{-1}x)\grad,
\end{align*}
in the limit as $\eps\rightarrow0$, where $A^{\eps}(y)=\eps^2$ on $Q_0$ and $1$ on $Q_1$\footnote{This setup corresponds to a periodic structure with low-index inclusions, as opposed to the high-index inclusions considered in \cite{movchan2001noncommuting}.}, see the domain on the left in figure \ref{fig:Diagram_HL-Zhikov}.
\begin{figure}[b]
	\centering
	\includegraphics[scale=1.0]{Diagram_HL-Zhikov.pdf}
	\caption[Relation between materials under critical contrast and at resonance.]{\label{fig:Diagram_HL-Zhikov} A material under critical contrast (left) as studied in \cite{zhikov2000extension}, and a material at resonance studied in \cite{hempel2000spectral} (right).}
\end{figure}
Materials such as this are said to be under ``critical contrast"; the (dimensionless) material properties scale in certain proportion to the size of the period cell $\eps$, analogous to (although not exactly) the relation between refractive index and incident wavenumber mentioned earlier.
A limiting operator $\mathcal{A}$, and the convergence of the solutions to $\mathcal{A}^{\eps}u^{\eps}=f^{\eps}$ to the solution $u$ of $\mathcal{A}u=f$ is obtained --- however this convergence is in the so-called two-scale sense rather than the standard norm-resolvent sense of classical homogenisation.
The spectrum of the limiting operator $\mathcal{A}$ is also shown to be the Hausdorff limit of the spectra of the operators $\mathcal{A}^{\eps}$ as $\eps\rightarrow0$, and demonstrates dissipative properties.
It is argued (and demonstrated) that two-scale convergence ensures that not only do the solutions $u^{\eps}$ converge to $u$, but the energy of the system also converges (in this two-scale sense) to the energy of the effective problem --- something which is not always guaranteed by classical homogenisation in contexts such as this.
Furthermore, it is also demonstrated that if $A^{\eps}(y)\propto\eps^p$ on $Q_0$ for $p<2$, the effective problem coincides with that obtained through classical homogenisation, whilst for $p>2$ the effective problem is again the ``two-scale limit" of the original problem.
This allows us to obtain an effective problem for materials under critical contrast, where previous homogenisation techniques fell short --- however the convergence obtained is not the standard norm-resolvent convergence we are used to from classical homogenisation.
\tstk{also, this is done with respect to MEASURES, which we need to talk about!}

In a related development, the study \cite{hempel2000spectral} proved the band-gap structure of the resulting spectrum, although starting from a different domain, illustrated in right of figure \ref{fig:Diagram_HL-Zhikov}.
The period cell and inclusion are of a fixed size in this study, and the operator considered is $\widetilde{\mathcal{A}}^\sigma:=-\grad\cdot\tilde{A}\grad$ is studied\footnote{The study \cite{hempel2000spectral} actually concerns a slightly more general version of this operator.} where $\widetilde{A}=1$ on $Q_0$ and $\widetilde{A}=\sigma\gg 1$ on $Q_1$.
The asymptotic limit $\kappa\rightarrow\infty$ was then explored, with the deduction that the endpoints of the spectral bands correspond to the Dirchlet and (the limits of the) Neumann eigenvalues of the Laplacian on the inclusions $Q_0$.
Additional results about the density of states were also proven, the main takeaway being that the spectrum concentrates near the right-endpoints of the bands as $\sigma$ increases. 
These results are also proved in \cite{friedlander2002density} under slightly stricter assumptions concerning smoothness of the boundary of the inclusion $Q_0$, but providing an asymptotic form for the integrated density of states. 
This asymptotic form for the integrated density of states is further refined in \cite{selden2005periodic}, in the setup of \cite{friedlander2002density}. 
Upon identifying $\sigma=\eps^{-2}$ (so $\sigma\rightarrow\infty \Leftrightarrow \eps\rightarrow0$) it is clear why the effective problems for $\mathcal{A}^{\eps}$ and $\widetilde{\mathcal{A}}^{\sigma}$ (under the respective limits) possess the same spectra, although it should be noted that this is one of the only common features of the two problems --- in most other respects they behave differently.
This association between $\mathcal{A}^{\eps}$ and $\widetilde{\mathcal{A}}^{\sigma}$ also proves a good time for us to introduce another term (or concept) that is often thrown around in the literature and hard to nail down --- that of resonance.
We can observe that, with the size of the period cell being unity in the setup of \cite{hempel2000spectral}, the material considered cannot be under critical contrast.
However, this material is under (or at) \emph{resonance} --- exactly one of the non-dimensional parameters describing this material ($\sigma$) is large, whilst the others are of order unity.
The term ``resonance" typically brings with it connotations from more physically minded contexts, one things of resonance as a phenomenon where the frequency of a periodic driving force is close to the natural frequency of the system on which it acts, causing a large spike in the amplitude of the response of the system.
This intuition still holds; observe that in terms of the dimensionless parameters \eqref{eq:Intro-NonDimLengthScales} we have $\delta\in\bracs{0,1}$, $\hat{l}^{\mathrm{inc}}=1$, and $\hat{l}^{\mathrm{bulk}}=\sigma\gg 1$, the wavelength within the bulk $\lambda^{\mathrm{bulk}}$ has been ``driven" in response to the periodic structure of the material.
Through these two studies one can see a correspondence between materials ``under resonance" and those at ``critical contrast", both phenomena can result in the emergence of band-gaps or other dissipative effects, despite seeming to model different material setups.

The studies discussed above, whilst illustrative and helpful in demonstrating how it is possible for dissipative effects to emerge, still fall short of definitively establishing these effects as a direct consequence of being under critical contrast (or at resonance).
The work of \cite{cherednichenko2019unified, cherednichenko2019time} attempts to rectify this, looking to provide a more general theory analogous to that which already exists for the moderate (non-critical) contrast setting.
An key observation made in these works is that models with frequency dependent boundary conditions can be extended to conservative systems, and these ``extended" systems are precisely the asymptotic limits of models of high (or critical) contrast media. \tstk{examples include Figotin-Schenker, and the aforementioned quantum graph limits of thin structures?}
We have seen (section \ref{ssec:Intro-ThinStructures}) that such frequency-dependent problems and operators acting in extended spaces also emerge in the asymptotic limit of thin structure problems as the thickness of the structure tends to zero.
This would seem to imply that the same asymptotic properties, and thus dispersive effects such as metamaterial behaviour or existence of band gaps, can be obtained from either geometric contrast (manifesting in the correct vertex-to-edge volume scaling, section \ref{ssec:Intro-ThinStructures}) or from high contrast materials (that is, contrast between the material properties).
From a physical design perspective, having the ability to obtain such effects through two different means is helpful in contexts where one such contrast cannot be guaranteed, or is difficult to manufacture.
The relevance of our singular structure problems now extends beyond applications to thin structures --- if geometric contrast can mirror the effects of material contrast, then insights from our singular structure problems will also be informative and useful in the critical contrast context.