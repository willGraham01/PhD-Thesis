\section{Asymptotic Studies and Dispersive Effects} 

% include the discussion on thin structures
\subsection{Thin Structures in the $\delta\rightarrow0$ limit} \label{ssec:Intro-ThinStructures}

Graphically, the ``limit" $\delta\rightarrow0$ in these domain setups is rather intuitive --- one can visualise the region $Q_1$ becoming increasingly fine, getting closer to a collection of connected line segments as $\delta$ decreases to 0.
This raises the natural question as to whether it is reasonable (or even possible) to appeal to this visual intuition and approximate a photonic crystal as a \emph{singular structure} embedded into a 2 or 3 dimensional matrix.
Figure \ref{fig:Diagram_ShrinkToSingularSquareCell} illustrates this process for a 2-dimensional geometry that is extruded into 3 dimensions.
\begin{figure}[h]
	\centering
	\includegraphics[scale=0.5]{Diagram_ShrinkToSingularSquareCell.pdf}
	\caption{\label{fig:Diagram_ShrinkToSingularSquareCell} A visual illustration of the $\delta\rightarrow0$ limit in a ``fibre-like" geometry. The cross sectional geometry (dark cross) shrinks to a collection of line segments, resulting in a union of planes whose cross-section is a graph-like structure.}
\end{figure}
The resulting ``singular structure" is a region embedded into a space in which it has no volume (or area in two dimensions); in the illustration the resulting planes have no volume from the perspective of $\reals^3$.
If we remain true to our photonic crystal setup, the rest of $\reals^d$ upon taking the ``limit" $\delta\rightarrow0$ is taken up by the ``inclusion" $Q_0$.
However, relevant to our studies in \tstk{sections} will be the study of so called \emph{thin-structures}.
Such thin-structures $G_{\delta}$ consist only of the region (formed by translations of) $Q_1$, that is $G_{\delta} = \bigcup_{l\in\integers^d}(l+Q_1)$.
In almost all contexts one is typically looking at a thin-structure domain that is akin to a ``thickened graph"; one takes a graph $\graph$, embeds it into $\reals^d$, and inflates the edges into tubes of radius $\sim\delta$ and the vertices into junctions regions connecting the ends of these tubes.
In the interest of avoiding addressing several technicalities and detracting from the focus of our review, one can think of $G_{\delta}$ as being the surface of (or volume enclosed by) the resulting ``inflated" structure --- the illustration in figure \ref{fig:Diagram_ThickenedGraph} provides a sufficient visualisation.
\begin{figure}[t]
	\centering
	\includegraphics[scale=1.0]{Diagram_ThickenedGraph.pdf}
	\caption{\label{fig:Diagram_ThickenedGraph} A schematic illustration of a thickened graph $G_{\delta}$, consisting of tubes whose central axes are the edges of the underlying graph $\graph$, meeting at inflated vertex regions. The graph $\graph$ consists of the dashed edges and connecting vertex.}
\end{figure}

By design, the thin-structure $G_{\delta}$ converges in the eyeball norm as $\delta\rightarrow0$ to the (metric) graph $\graph$, whose edges are correspond to the central axes of the tubes and whose vertices correspond to the centre of the collapsing junction regions.
Since $\graph$ is embedded into real space, each edge $I_{jk}$ which connects the vertex $v_j$ to $v_k$ can be bestowed a length $l_{jk}>0$ and a corresponding interval $\sqbracs{0,l_{jk}}$ where we associate $0\in\sqbracs{0,l_{jk}}$ to $v_j$ and $l_{jk}\in\sqbracs{0,l_{jk}}$ to $v_k$.
A function defined on a quantum graph $u$ is then specified by its restriction to each of the $I_{jk}$, denoted by $u^{(jk)}$, and the values $u$ takes at the vertices of the graph.
With this notion of length, one can now define a derivative for $u$ along the edges, and thus can equip $\graph$ with (the analogue of) a differential operator.
Since the intervals associated to the edges $I_{jk}$ do not convey the connectivity of $\graph$, the ``boundary conditions" for these differential operators come in the form of matching conditions at the vertices.
A graph equipped with such an operator is called a \emph{quantum graph}, and the system of equations and ``boundary conditions" defined by this operator a \emph{quantum graph problem}.
A more precise description and introduction is provided in section \tstk{section ref}, and a more complete overview of the field can be found in \cite{berkolaiko2013introduction}, however it is sufficient for this introduction to think of such a quantum graph problem as consisting of an ODE on each of the intervals associated to $I_{jk}$, coupled through matching conditions at the vertices.
Standard Neumann or Dirichlet conditions on $u$ at each vertex can be used as the vertex conditions, however these tend to neglect (or to an extent, suppress) the underlying graph structure, which allows for more adventurous conditions to be used.
More common choices of vertex conditions are continuity of the function $u$ though the vertices\footnote{That is whenever two edges $I_{jk}$ and $I_{kl}$ share a common vertex, the restrictions $u^{(jk)}$ and $u^{(kl)}$ must take the same value at the shared vertex $v_k$.} and the Kirchoff condition,
\begin{align*}
	\sum_{v_k \text{ connects to } v_j} 
	\pdiff{u^{(jk)}}{n}\bracs{v_j} = \alpha_j u(v_j),
\end{align*}
where $\pdiff{u^{(jk)}}{n}\bracs{v_j}$ denotes the \emph{incoming} derivative to $v_j$, and $\alpha_j$ is a (``coupling") constant. 
The Kirchoff condition corresponds relates the flux at each vertex to the function value, when the coupling constant (or function value) at $v_j$ is zero, it's form and interpretation is analogous to Kirchoff's first law of ``net zero current" through junctions in electric circuits.

The use of quantum graphs as approximations to physical processes and the study of the spectra of the associated operators has a rather rich history thanks \tstk{get refs for this, Kuch-Zheng is a good place to start} to interests from mesoscopic physics\footnote{Broadly speaking, this is a branch of condensed matter physics focusing on materials whose size ranges from a few molecules or atoms to micrometres.}.
\tstk{Although we do not delve into the details, such quantum graph problems are known to describe thin superconducting structures called ``quantum wires", ``molecular wires", and free-electron theory of conjugated molecules.}
In each of these contexts there is difficulty in identifying the \emph{correct} quantum graph problem. Whilst the graph itself is easily identifiable, the resulting vertex conditions are not due to complications in coming from the behaviour of the solutions within the junction regions of $G_{\delta}$ as $\delta$ decreases.
Heuristic arguments and physical intuition led to the standard practice of imposing continuity and Kirchoff conditions as the vertex conditions in the ``approximating" quantum graphs for these applications.
These long-standing assumptions seemed justified with the study of \tstk{\cite{kuchment2001convergence} themselves extending J. Rubinstein and M. Schatzman}, which established convergence of the spectrum of the Neumann laplacian on $G_{\delta}$ (which we will denote by $\mathcal{G}_{\delta}$) to the spectrum of a quantum graph problem (defined by an operator we denote by $\mathcal{G}$).
Precisely, the result
\begin{align*}
	\lim_{\delta\rightarrow0} \lambda_n\bracs{\mathcal{G}_{\delta}} = \lambda_n\bracs{\mathcal{G}}
\end{align*}
we proved for every $n\in\naturals$, where $\lambda_n\bracs{\mathcal{G}_{\delta}}$ and $\lambda_n\bracs{\mathcal{G}}$ are the eigenvalues of the respective operators ordered in ascending order.
The argument that was utilised revolved around representing the eigenvalues of $\mathcal{G}_{\delta}$ and $\mathcal{G}$ via the minimax principle, and a method of translating functions on $G_{\delta}$ to $\graph$ and back so that the increase in the Rayleigh quotient could be controlled.
Convergence of the resolvent was proved in \tstk{Satio - in \cite{exner2005convergence}, [27]th reference} when the underlying graph had no loops or cycles.
In particular, the vertex conditions associated with $\mathcal{G}$ were continuity of $u$ at the vertices of $\graph$, and a Kirchoff condition at each vertex\footnote{This is a simplification in the interest of providing the reader with an overall idea of the key concepts. The study \cite{kuchment2001convergence} allowed for a slightly more general, weighted domain $G_{\delta}$ which results in a weighted sum in the Kirchoff condition, and also allows the presence of an external field.}, justifying the heuristic arguments that had come before.

However, the study \cite{exner2005convergence} later demonstrated that this was not the complete story.
Starting from the domain\footnote{The study \cite{exner2005convergence} actually considers the more general setup where one is working with manifolds, and differential operators on these manifolds. Here, we present the results in a manner contextualized to our review.} $G_{\delta}$, one can define the volume of the inflated edges $V_{\mathrm{edge}}$ as a function of the thickness $\delta$ and volume of the junction regions $V_{\mathrm{vertex}}$, which also scales with $\delta$.
In the limit $\delta\rightarrow0$, the spectrum of $\mathcal{G}_{\delta}$ coincides with the spectrum of an operator $\tilde{\mathcal{G}}$ which depends on the relative scaling between $V_{\mathrm{vertex}}$ and $V_{\mathrm{edge}}$.
\begin{itemize}
	\item (``Thick vertex" setup) If $V_{\mathrm{edge}}\ll V_{\mathrm{vertex}}\rightarrow0$ as $\delta\rightarrow0$, then $\tilde{\mathcal{G}}$ defines a quantum graph problem with (homogeneous) Dirichlet conditions at each of the vertices.
	This case is referred to as Dirichlet decoupling; the resulting limit problem is just a collection of independent ODEs along the edges of the graph $\graph$, intuitively the result of the vertices being ``too big" when compared to the edges and thus preventing any interaction between the edges.
	\item (``Thick edge" setup) If $V_{\mathrm{vertex}}\ll V_{\mathrm{edge}}\rightarrow0$ as $\delta\rightarrow0$, the edges dominate in the limit problem and one obtains $\tilde{\mathcal{G}} = \mathcal{G}$ from the study \cite{kuchment2001convergence}.
	Intuitively, since the vertices ``disappear" before the edges as $\delta\rightarrow0$, the resulting problem (and it's solutions) must have strict matching conditions where the edges meet --- resulting in continuity and a net flux of 0 (Kirchoff condition) at each vertex.
	\item (``Borderline/critical" case) Of particular interest is when $\frac{V_{\mathrm{edge}}}{V_{\mathrm{vertex}}}\rightarrow c>0$ as $\delta\rightarrow0$, that is when the volume of the tubes and junctions are of the same order of $\delta$.
	This is an intermediary for the other two cases, the effect of decoupling the edges (large junctions) is balanced by the need for consistency between edges (large tubes).
	The spectral problem for the resulting operator $\tilde{\mathcal{G}}$ can be shown to define a quantum graph problem whose vertex conditions are continuity at each of the vertices, but along with a ``non-standard" Kirchoff condition at the vertices,
	\begin{align*}
	\sum_{v_k \text{ connects to } v_j} 
	\pdiff{u^{(jk)}}{n}\bracs{v_j} = \lambda\alpha_j u(v_j).
	\end{align*}
	Here, $\alpha_j$ is the aforementioned coupling constant and $\lambda$ is the \emph{spectral parameter} for $\tilde{\mathcal{G}}$ --- in physical terms, one can say that the strength of any coupling at a vertex is proportional to the system's energy.
	Mathematically, the operator $\tilde{\mathcal{G}}$ is said to have a generalised resolvent --- we elaborate on this in what follows.
\end{itemize}
Of particular interest is the operator $\tilde{\mathcal{G}}$ in the borderline case; it is not an operator on a quantum graph per say, but its spectral problem $\tilde{\mathcal{G}}\lambda = \lambda u$ can be interpreted as a quantum graph problem with the spectral parameter appearing in the boundary (or vertex) conditions.
More precisely, the operator $\tilde{\mathcal{G}}$ acts in an ``extended" (or ``larger") function space, rather than the standard function spaces for functions on metric graphs, \tstk{and is said to possess a ``generalised resolvent" - define this, what is it?}
The quantum graph problem with the non-standard Kirchoff condition belongs to the class of problems with generalised resolvents, \tstk{the refs for this?}, and the operator $\tilde{\mathcal{G}}$ is the so-called ``Strauss extension" or ``dilation" of this problem.
\tstk{as we will see, our starting point addresses multiple issues here: we can ``start" from an intuitive operator and problem to arrive at this non-standard QG problem. Also, this makes the job of ``guessing" the limiting problem much easier!}

To conclude our discussion of the literature, we highlight that at present there are no results which extend the studies \cite{kuchment2001convergence, exner2005convergence} into the Maxwell setting (that is, to either the curl-of-the-curl equation or the full Maxwell system).
By contrast, there has been considerable interest in the study of the equations of elasticity on thin-structures and the resulting ``singular" limits.
The work \cite{zhikov2002homogenization} studied the problem of homogenisation for periodic thin-structure of thickness $\delta$, looking to derive the resulting effective problem in the ``long-wavelength" regime.
In this work determination of the effective problem in the so-called critical scaling regime $\delta\sim\eps$ of the thickness $\delta$ with the period $\eps$ was noted as being an open problem, with the resulting homogenised operator being different to that in the non-critical scaling regimes.
This work was followed up in \cite{zhikov2003homogenization}, in which these differences were identified and the effective problem in the critical setting presented.
A method for passing to the limit in the equations of elasticity on a bounded thin-structure domain as the thickness $\delta\rightarrow0$ was also developed in \cite{zhikov2006derivation}, the resulting system being a quantum graph problem with Kirchoff matching conditions at each of the vertices.
We highlight these results because, akin to \cite{kuchment2001convergence}, these studies do not account for the possible ``borderline" scaling case --- only the edge thickness $\delta$ is present in the studies.
For this reason it is likely that results similar to those presented by \cite{exner2005convergence} can be derived within the context of elasticity too, however this remains unexplored in the literature.

The work of chapters \tstk{1 and 2} will be particularly relevant to the aforementioned studies and the ``limiting" quantum graph problems.
In these chapters we will take the ``limiting" singular structure as our starting point, and pose a variational problem upon this singular structure.
This will require us to setup and study appropriate function spaces \tstk{section ref}, before then demonstrating the relationship between the abstract variational problem and the various ``limiting" quantum graph problems discussed above --- including the generalised resolvent problem.
We will also go into detail about how one can explicitly solve the resulting quantum graph problems, before looking to extend our results into the Maxwell setting, and indicating some of the challenges that this brings.
The theory we establish in these chapters will also be valuable to us in chapter \tstk{composite measure}, when we come to study a ``photonic crystal with singular inclusions".

% now begin the discussion of critical contrast
\subsection{Critical, or High Contrast Composites} \label{ssec:Intro-CritContrast}
We have seen that there are a variety of effective problems that one can obtain from the $\delta\rightarrow0$ limit of a thin structure in section \ref{ssec:Intro-ThinStructures}.
Now we put the shoe on the other foot, and focus on the various behaviours that can emerge when considering asymptotic limits of the material properties (or rather, limits of the dimensionless quantities that are derived from them).
The studies discussed here typically concern periodic, composite materials whose domains are described in the manner introduced in section \ref{ssec:Intro-NonDimMax}, with the size of the inclusion $\delta$ fixed.
Such domains are more reflective of physical photonic crystals than the thin domains considered previously in section \ref{ssec:Intro-ThinStructures}.
Despite our interest being primarily in singular structures and the resulting spectra of operators on them, it is nonetheless important for us to highlight how asymptotic models for materials at ``critical-contrast" can also give rise to dispersive effects.
Indeed, we shall see that the dispersive effects that arise from these models mirror those that emerge from limits of thin structures.

Studies of such materials typically rely on techniques from homogenisation theory to determine ``effective" material properties in the asymptotic limit of interest.
For example, consider the operator $\mathcal{A}=-\grad\cdot A\grad$ on such a composite domain --- the (possibly matrix-valued) function $A$ describes the ``non-dimensional" material parameters, and throughout we will assume it is $Q$-periodic.
One regime of interest is the ``long wave" (or low frequency) regime, where one attempts to garner information about the lowest spectral band of a composite material.
Accordingly, the (small) parameter $\eps\ll 1$ is introduced and we consider the (sequence of) operators $\mathcal{A}^{\eps} = -\grad\cdot A\bracs{\eps^{-1}x}\grad$ in the limit $\eps\rightarrow0$.
We then derive an ``effective" operator $\mathcal{A}^{\mathrm{hom}} = -\grad\cdot A^{\mathrm{hom}}\grad$ that describes the resulting limit, and look to formalise the relationship between $\mathcal{A}^{\mathrm{hom}}$ and the $\mathcal{A}^{\eps}$.
The matrix $A^{\mathrm{hom}}$ is sometimes labelled or interpreted as the ``effective" material properties.
Typically one desires convergence of $\mathcal{A}^{\eps}$ to $\mathcal{A}^{\mathrm{hom}}$ as $\eps\rightarrow0$ in the norm-resolvent sense, namely that the family of solutions $u_{\eps}$ to the problems $\mathcal{A}^{\eps}u_{\eps}=f_{\eps}$ converges to the solution $u$ of the effective problem $\mathcal{A}^{\mathrm{hom}}u = f$, for any suitable initial data $f_{\eps}\rightarrow f$.
However one may also establish weaker results, for example coincidence of the spectrum of $\mathcal{A}^{\mathrm{hom}}$ with the limit of $\sigma\bracs{\mathcal{A}^{\eps}}$ (akin to the results for thin structures discussed in section \ref{ssec:Intro-ThinStructures}) which would otherwise be implied by norm-resolvent convergence.
This convergence in turn ensures that the effective problem faithfully describes the behaviour of the composite material within this regime, and one typically hopes that the effective problem is more mailable to analysis than its counterpart.
These studies can also reveal interesting material effects within such regimes that might otherwise be lost or go unnoticed.
The design of materials to open spectral band gaps is one such effect that is desired, however other dispersive effects such as metamaterial behaviour \tstk{ref?} --- where the effective material properties appear to take unphysical values --- can also emerge from these limits.
In these settings, one is typically expecting to obtain an effective problem where the effective material properties $A^{\mathrm{hom}}$ depend non-trivially on the frequency variable $\omega$.

The asymptotic treatment of composite materials has seen considerable interest due to the aforementioned motivations.
However, it has been known to both the mathematical and physical community for some time that the toolbox of standard homogenisation theory is unable to capture the correct asymptotic behaviour for composite materials.
It has been reported \tstk{proved, no? Reported in K-E-K-Nab Unified Approach, pointing to the references: M. Sh. Birman and T. A. Suslina, Second order periodic differential operators. Threshold properties and homogenization, St. Petersburg Math. J. 15 (2004), no. 5, 639–714 and V. V. Zhikov, Spectral approach to asymptotic diffusion problems. (Russian), Differentsial’nye uravneniya 25 (1989), no. 1, 44–50. } in the uniformly elliptic setting, that is where the matrix $\mathcal{A}(\eps^{-1}x)$ and its inverse are uniformly bounded, that the resulting effective problem has $A^{\mathrm{hom}}$ being a constant matrix, which leaves no room for time dispersive effects.
If one drops the assumption of uniform ellipticity, the resulting analysis becomes more complicated and the standard toolbox of asymptotic study is no longer sufficient for describing the effective problem.
Indeed the inadequacy of standard techniques in such contexts has been reported in works such as \cite{nicorovici1995photonic} \tstk{later paper implies this is part of a series of works though}, which highlighted ``non-commuting limits" when considering wave scattering off a periodic composite material.
In such a problem there are two scales to consider; the wavenumber of the incident wave $\wavenumber$ and the refractive index of the inclusions within the composite, $N$ (which is proportional to the square root of the relative electric permittivity of the inclusions, to maintain a connection to the dimensionless quantities introduced in section \ref{sec:Intro-Maxwell}).
In order to study the first spectral band\footnote{This is also referred to as the lowest spectral band, or the ``acoustic band" in the literature.} in the case when the inclusions are highly conducting; we can either consider the case when the inclusions are perfectly conducting ($N\rightarrow\infty$) for fixed $\wavenumber$ and then take the ``static-limit" $\wavenumber\rightarrow0$, \emph{or} consider the static limit ($\wavenumber\rightarrow0$) at a high but finite refractive index and then take $N\rightarrow\infty$.
However the two approaches produce different effective materials --- the explanation provided in \cite{movchan2001noncommuting} is that the trajectory in $\bracs{N^{-1},\wavenumber}$-space that one takes as these values approach the origin must be of the form $N^{-1}=\wavenumber^p$, with $p<1$ for homogenisation to be satisfactory, otherwise the problem being considered is ``singularly perturbed" and boundary layer effects emerge near the inclusion-matrix interface which are not handled by the homogenisation process.
However, there have been a number of developments aimed at expanding the tools of homogenisation theory to handle contexts such as the above, where the material parameters scale in particular relation to the size of an incident wave or the size of the period cell.

Let us begin with the study of \cite{zhikov2000extension}, that was concerned with the homogenisation problem for the operator $\mathcal{A}^{\eps}$,
\begin{align*}
	\mathcal{A}^{\eps} := -\grad\cdot A^{\eps}(\eps^{-1}x)\grad,
\end{align*}
in the limit as $\eps\rightarrow0$, where $A^{\eps}(y)=\eps^2$ on $Q_0$ and $1$ on $Q_1$\footnote{This setup corresponds to a periodic structure with low-index inclusions, as opposed to the high-index inclusions considered in \cite{movchan2001noncommuting} \tstk{and other McPhedran papers?}.}, see the domain on the left in figure \ref{fig:Diagram_HL-Zhikov}.
\begin{figure}[b]
	\centering
	\includegraphics[scale=1.0]{Diagram_HL-Zhikov.pdf}
	\caption[Relation between materials under critical contrast and at resonance.]{\label{fig:Diagram_HL-Zhikov} A material under critical contrast (left) as studied in \cite{zhikov2000extension}, and a material at resonance studied in \cite{hempel2000spectral} (right).}
\end{figure}
Materials such as this are said to be under ``critical contrast"; the (dimensionless) material properties scale in certain proportion to the size of the period cell $\eps$, analogous to (although not exactly) the relation between refractive index and incident wavenumber mentioned earlier.
A limiting operator $\mathcal{A}$, and the convergence of the solutions to $\mathcal{A}^{\eps}u^{\eps}=f^{\eps}$ to the solution $u$ of $\mathcal{A}u=f$ is obtained --- however this convergence is in the so-called two-scale sense rather than the standard ``norm-resolvent" sense of classical homogenisation.
The spectrum of the limiting operator $\mathcal{A}$ is also shown to be the Hausdorff limit of the spectra of the operators $\mathcal{A}^{\eps}$ as $\eps\rightarrow0$, and demonstrates dissipative properties.
It is argued (and demonstrated) that two-scale convergence ensures that not only do the solutions $u^{\eps}$ converge to $u$, but the energy of the system also converges (in this two-scale sense) to the energy of the effective problem --- something which is not always guaranteed by classical homogenisation in contexts such as this. \tstk{isn't there also a study that does this in the scattering context - would be good to have this since McPhedran was doing scattering too? G. Bouchitté and D. Felbacq, Homogenization near resonances and artificial magnetism from dielectrics, C. R. Math. Acad. Sci. Paris 339 (2004), no. 5, 377–382.}
Furthermore, it is also demonstrated that if $A^{\eps}(y)\propto\eps^p$ on $Q_0$ for $p<2$, the effective problem coincides with that obtained through classical homogenisation, whilst for $p>2$ the effective problem is again the ``two-scale limit" of the original problem.
This allows us to obtain an effective problem for materials under critical contrast, where previous homogenisation techniques fell short --- however the convergence obtained is not the standard norm-resolvent convergence we are used to from classical homogenisation.

In a related development, the study \cite{hempel2000spectral} proved the band-gap structure of the resulting spectrum, although starting from a different domain, illustrated in right of figure \ref{fig:Diagram_HL-Zhikov}.
The period cell and inclusion are of a fixed size in this study, and the operator considered is $\widetilde{\mathcal{A}}^\sigma:=-\grad\cdot\tilde{A}\grad$ is studied\footnote{The study \cite{hempel2000spectral} actually concerns a slightly more general version of this operator.} where $\widetilde{A}=1$ on $Q_0$ and $\widetilde{A}=\sigma\gg 1$ on $Q_1$.
The asymptotic limit $\kappa\rightarrow\infty$ was then explored, with the deduction that the endpoints of the spectral bands correspond to the Dirchlet and (the limits of the) Neumann eigenvalues of the Laplacian on the inclusions $Q_0$.
Additional results about the density of states were also proven, the main takeaway being that the spectrum concentrates near the right-endpoints of the bands as $\sigma$ increases. 
These results are also proved in \cite{friedlander2002density} under slightly stricter assumptions concerning smoothness of the boundary of the inclusion $Q_0$, but providing an asymptotic form for the integrated density of states. 
This asymptotic form for the integrated density of states is further refined in \cite{selden2005periodic}, in the setup of \cite{friedlander2002density}. 
Upon identifying $\sigma=\eps^{-2}$ (so $\sigma\rightarrow\infty \Leftrightarrow \eps\rightarrow0$) it is clear why the effective problems for $\mathcal{A}^{\eps}$ and $\widetilde{\mathcal{A}}^{\sigma}$ (under the respective limits) possess the same spectra, although it should be noted that this is one of the only common features of the two problems --- in most other respects they behave differently.
This association between $\mathcal{A}^{\eps}$ and $\widetilde{\mathcal{A}}^{\sigma}$ also proves a good time for us to introduce another term (or concept) that is often thrown around in the literature and hard to nail down --- that of resonance.
We can observe that, with the size of the period cell being unity in the setup of \cite{hempel2000spectral}, the material considered cannot be under critical contrast.
However, this material is under (or at) \emph{resonance} --- exactly one of the non-dimensional parameters describing this material ($\sigma$) is large, whilst the others are of order unity.
The term ``resonance" typically brings with it connotations from more physically minded contexts, one things of resonance as a phenomenon where the frequency of a periodic driving force is close to the natural frequency of the system on which it acts, causing a large spike in the amplitude of the response of the system.
This intuition still holds; observe that in terms of the dimensionless parameters \eqref{eq:Intro-NonDimLengthScales} we have $\delta\in\bracs{0,1}$, $\hat{l}^{\mathrm{inc}}=1$, and $\hat{l}^{\mathrm{bulk}}=\sigma\gg 1$, the wavelength within the bulk $\lambda^{\mathrm{bulk}}$ has been ``driven" in response to the periodic structure of the material.
Through these two studies one can see a correspondence between materials ``under resonance" and those at ``critical contrast", both phenomena can result in the emergence of band-gaps or other dissipative effects, despite seeming to model different material setups.

The studies discussed above, whilst illustrative and helpful in demonstrating how it is possible for dissipative effects to emerge, still fall short of definitively establishing these effects as a direct consequence of being under critical contrast (or at resonance).
The work of \cite{cherednichenko2019unified, cherednichenko2019time} attempts to rectify this, looking to provide a more general theory analogous to that which already exists for the moderate (non-critical) contrast setting.
An key observation made in these works is that models with frequency dependent boundary conditions can be extended to conservative systems \tstk{via the analysis of generalised resolvents --- link to where this was first mentioned}) , and these ``extended" systems are precisely the asymptotic limits of models of high (or critical) contrast media.
We have seen (section \ref{ssec:Intro-ThinStructures}) that such frequency-dependent problems also emerge in the asymptotic limit of thin structure problems as the thickness of the structure tends to zero \tstk{Kirill-Sasha's Kronig-Penney diople paper even highlights this link right?}.
This would seem to imply that the same asymptotic properties, and thus dispersive effects such as metamaterial behaviour or existence of band gaps, can be obtained from either geometric contrast (manifesting in the correct vertex-to-edge volume scaling, section \ref{ssec:Intro-ThinStructures}) or from high contrast materials (that is, contrast between the material properties).
From a physical design perspective, having the ability to obtain such effects through two different means is helpful in contexts where one such contrast cannot be guaranteed, or is difficult to manufacture.
This correspondence also firmly establishes the utility of quantum graph problems as effective models for a variety of media.
Furthermore, the relevance of our singular structure problems now extends beyond applications to thin structures --- if geometric contrast can mirror the effects of material contrast, then insights from our singular structure problems will also be informative and useful in the critical contrast context.

% finally, talk about when both $\delta$ and $\eps$ are present, and lead into Research Overview
\subsection{Double limits in $\eps$ and $\delta$.} \label{ssec:Intro-DoubleLimits}
So far we have considered the result of a shrinking domain thickness given fixed material properties, and then various asymptotic limits of high contrast materials with fixed inclusion sizes. 
It is natural to ask whether any consideration has been put into considering the effects that might emerge when both the (relative) size $\delta$ of the region $Q_1$ shrinks to zero \emph{and} the contrast in the material properties increases to infinity.
The series of papers \cite{figotin1996band-scalar, figotin1996band-maxwell, figotin1998spectral} was dedicated to investigating the spectral properties that emerge in the aforementioned limits, for a variety of operators relevant to models of wave propagation.
This series begins \cite{figotin1996band-scalar} by studying the acoustic equation on a non-magnetic composite material whose period cell $Q$ is the unit square in $\reals^2$, and whose inclusion $Q_0$ is a smaller square of side length $1-\delta$.
The region $Q_1$ is therefore shrinking as $\delta\rightarrow0$ (visually, to a singular structure) and possesses (``dimensionless") dielectric constant $\eps_m$ which is equal to $\eps$ on $Q_1$ and $1$ on $Q_0$.
For sufficiently small values of $\delta$, $\bracs{\delta\eps}^{-1}$, and $\eps\delta^2$, it is demonstrated that spectral gaps open, that the spectrum concentrates near the eigenvalues of the Neumann Laplacian on (the union of all periodic translations of) $Q_0$, and an estimate for the width of the spectral bands is also provided.
This was followed up \cite{figotin1996band-maxwell} by a consideration of the curl-of-the-curl equation in $\reals^3$, on a domain that was the extrusion into 3 dimensions of the 2D domain described previously.
In the asymptotic regime $\eps\delta^{\frac{3}{2}}\ll 1$, $\eps\delta\gg1$, gaps in the spectrum once again begin to open --- details on the structure of the spectrum and estimates regarding the width of the spectral bands are again provided.
Finally, analysis of the acoustic equation when the region $Q_0$ was a polygon \cite{figotin1998spectral} was considered, in the limit as $\delta\rightarrow0$, and $\bracs{\delta\eps}^{-1}\rightarrow W<\infty$.
Note that since the region $Q_0$ is a polygon, the region $Q_1$ can be thought of as a thickening of some graph $\graph$ embedded into $\reals^2$.
Under these assumptions, it is shown that the spectrum of the acoustic equation on such a medium converges to the spectrum of the problem
\begin{align} \label{eq:Intro-KuchFigQGLimit}
	-\laplacian u &= \lambda\bracs{\delta_{\graph} + W}u,
\end{align}
where $\lambda$ is the spectral parameter and $\delta_{\graph}$ is the delta function supporting the graph $\graph$.
Note that this convergence is not in the norm-resolvent sense, and only demonstrates coincidence of the limit of the acoustic equation as $\delta\rightarrow0$, $\bracs{\delta\eps}^{-1}\rightarrow W<\infty$ with the spectrum of the problem \eqref{eq:Intro-KuchFigQGLimit}.
This is of particular interest to our work in chapter \ref{ch:SingInc}, where we will consider a composite medium with one of the components being a singular structure, and look to obtain (and solve) a non-local quantum graph problem from our singular structure formulation.
However we note that our approach via singular structures is motivated by geometric contrasts, rather than the combination of geometric and material contrasts that these studies consider.
\tstk{anything that goes beyond these studies regarding the double limit? I don't think there is.}