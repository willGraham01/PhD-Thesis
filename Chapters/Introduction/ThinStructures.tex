\section{Mathematical Treatment of Composite, Periodic Media} \label{sec:MathHomLR}
\tstk{this is the section where I plan to talk about how band-gaps, 2-composite media, etc, have been treated in the mathematical community. Also need to talk about work on thin structures, and setup the stage for convergence to QGs.}

%At their heart, photonic crystals are simply periodic mediums whose period cell consists of a set of (dielectric material) inclusions embedded into another (background, or ``matrix") material.
%With developing interest in designing photonic crystals, there have been a number of studies dedicated to understanding the factors which contribute to the opening of spectral band gaps in periodic, composite media (the wider class of materials into which photonic crystals fall).
Before discussing the relevant results in the literature, we provide a toy setup for the study of wave propagation in a composite, periodic medium, to aid in contextualising what follows.
A suitable governing equation for wave propagation is the wave equation;
\begin{align*}
	-\grad\cdot A(x)\grad u(x,t) &= \pdiff{u}{t}(x,t),
\end{align*}
where $A(x)$ represents the material properties relevant to wave propagation, and is spatially varying if the material is non-homogeneous.
For example, in the acoustic setting $A$ would represent the ratio of the elastic modulus and mass density, whilst in the electromagnetic setting would represent the (inverse) product of the electric permittivity and magnetic permeability.
Although a good starting point, we note that the wave equation is not capable of describing all the dynamics of wave propagation in electromagnetic and acoustic settings --- where one would need to consider the Maxwell system or equations of elasticity, which are vectorial.
Indeed the scalar wave equation typically emerges as a special case of these vectorial systems; in the context of electromagnetism in absence of wave sources each component of the electric field vector satisfies the scalar wave equation, for example.
We will continue to use the wave equation for illustrative purposes in this section, but our discussion of the literature (and the systems we later come to study) will cover these more general governing equations.

Seeking a Bloch wave solution $u(x,t) = \hat{u}(x)\e^{\rmi\omega t}$ and de-dimensionalising the wave equation via $y=lx, l>0$\footnote{In the event we are looking at a finite medium, $l$ is typically the spatial extent of the medium. For studying periodic materials, $l$ can typically be taken to be the period of the medium so that the period cell has unit non-dimensional length.} and $a(y) = \hat{A}A(x/l)$, one obtains the eigenvalue problem
\begin{align*}
	-\grad_{y}\cdot a\bracs{y}\grad_{y} U(y) = z U(y),
	&\qquad 
	z = \frac{\omega^2 l^2}{\hat{A}},
\end{align*}
where $U(y)=\hat{u}(x/l)$.
In the event that $a$ is constant, the parameter $z$ represents (up to a constant) the square of the ratio of the spatial extent of the medium against the wavelength of propagating waves.
In the electromagnetic setting for example, one has
\begin{align*}
	z &= \frac{l^2 \omega^2 \eps_\mathrm{r}\mu_{\mathrm{r}}}{c^2},
\end{align*}
where $\eps_\mathrm{r},\mu_{\mathrm{r}}$ are the relative permittivity and permeability of the medium.
As such, throughout this work we will consider the (de-dimensionalised) ``wave equation" (more precisely Helmholtz equation) to be the equation
\begin{align*}
	-\grad\cdot a(x)\grad u(x) &= \omega^2 u(x), \qquad x\in\ddom,
\end{align*}
on some suitable domain $\ddom$, and we will also illustrate some of the concepts from the relevant literature using this equation as an example.
Working with non-dimensional units also allows to qualitatively describe the structure of the spectrum of these problems, and the physically-inclined reader can then obtain the physical wave-frequencies by reversing the process described above.
In a slight abuse of notation, we will continue to use $\omega^2 = z$ for the spectral parameter as it will be convenient for us to use the square-root of $z$ at certain points in what follows.

At their heart, photonic crystals are simply periodic materials whose period cell consists of a set of (dielectric material) inclusions embedded into another (background, or ``matrix") material.
With developing interest in designing photonic crystals, there have been a number of studies dedicated to understanding the factors which contribute to the opening of spectral band gaps in periodic, composite media (the wider class of materials into which photonic crystals fall).
A periodic composite medium can be modelled by filling the space $\reals^d$ with translations of a period cell $Q=[0,1)^d$ that is further divided into two regions, the ``inclusion" $Q_0$ and ``background" $Q_1$ with $Q_0\subset Q$ and $Q_1=Q\setminus\overline{Q_0}$.
One also typically specifies the relative size of the inclusion to the size of the period cell, which we will denote by $0<l<1$ (so if the side of the period cell is of unit length, the inclusion $Q_0$ has volume of the order $l^d$).
The material properties are then represented by a $Q$-periodic function $a(x)$;
\begin{align*}
	a(x) &= \begin{cases} a & x\in Q_i, \\ 1 & x\in Q_j, \end{cases}
	\qquad i,j\in\clbracs{0,1}, \ i\neq j.
\end{align*}
with the regions $Q_0$ and $Q_1$ typically labelled the ``soft" (where $a(x)=a$) and ``stiff" ($a(x)=1$), and the constant $a$ representing the contrast between the material parameters, see figure \ref{fig:Diagram_HomoPeriod1} for an illustration.
\begin{figure}[b!]
	\centering
	\begin{subfigure}[t]{0.3\textwidth}
		\centering
		\includegraphics[scale=1.0]{Diagram_HomoPeriod1.pdf}
		\caption{\label{fig:Diagram_HomoPeriod1} Schematic illustration for the period cell of a composite, periodic medium.}
	\end{subfigure}
	~
	\begin{subfigure}[t]{0.3\textwidth}
		\centering
		\includegraphics[scale=1.0]{Diagram_HomoHighFreq.pdf}
		\caption{\label{fig:Diagram_HomoHighFreq} The domain $\ddom_h$ for the high-frequency approximation.}
	\end{subfigure}
	~
	\begin{subfigure}[t]{0.3\textwidth}
		\centering
		\includegraphics[scale=1.0]{Diagram_HomoFiniteFreq.pdf}
		\caption{\label{fig:Diagram_HomoFiniteFreq} The domain $\ddom_f$ for the finite-frequency approximation.}
	\end{subfigure}
	\caption{\label{fig:Diagram_HomoFreqRanges} The period cell of a periodic, composite medium and the corresponding domains of the high and finite frequency approximations.}	
\end{figure}
With this domain setup and a suitable governing equation, one can then look to determine whether the material exhibits spectral band gaps, dispersion effects, or explore the behaviour of the material under certain limits such as $l\rightarrow0$ or $a\rightarrow\infty$.
The high-contrast limit $a\rightarrow\infty$ was explored in \tstk{Kirill-Sasha-Luis Op Norm Resolvant Asymptotic Analysis} on (a domain with similar structure to) $Q$, who proved convergence to a limiting problem and in particular established an asymptotic approximation for the limiting spectrum.
Although this study only worked on a finite domain $Q$ rather than a periodic medium filling $\reals^d$ with period cell $Q$, the analysis is nonetheless relevant, particularly if one uses a Gelfand or Floquet transform (\tstk{section ref}) to study the periodic problem.
\tstk{Fig. Kuchment: Band-Gap structure of periodic dielectric and acoustic media (scalar); focused on the spectrum of the operator $-\grad\cdot a\bracs{y}\grad$ for a composite, periodic medium with $Q=[0,1)^3$, and stiff inclusions $Q_0$ being cubes of side length $1-\delta$.
Provided that $\delta, a\delta^{-1}$ and $a^{-1}\delta^2$ are sufficiently small, the existence of spectral gaps was found to be guaranteed, and could even be opened up in any finite part of the spectrum.
F.Kuch: Band-gap structure of the spectrum of periodic Maxwell operators established similar results on existence of spectral band gaps for the $\curl{}\bracs{a(x)\curl{}}$ in a similar material.}
\tstk{more studies?!?! Cooper-K-S relevant here?}

The periodic nature of the material properties (encoded in the function $a(x)$) and the desire to understand the effective properties of the medium naturally lends itself to study through the process of homogenisation.
Broadly speaking, homogenisation theory looks to take a problem on an inhomogeneous material and derive (in some limit) an ``effective" problem for that material in terms of some ``homogenised" or ``effective" material properties.
Then by establishing convergence results for the solution of the original problem to those of the effective problem, one ensures that the approximation is valid under certain parameter regimes and can replace the potentially complex original problem with the effective problem, which is usually more tractable to analysis.
We will attempt to illustrate some of these frameworks; suppose our interest lies in studying the eigenvalues of the wave equation
\begin{align*} 
	\mathcal{A}u &:= -\grad\cdot a(x)\grad u = \omega^2 u, \qquad x\in\reals^d,
\end{align*}
where $\reals^d$ is filled with a periodic, composite medium with unit cell $Q=Q_0\cup Q_1$ as described above.
To approximate the spectrum of $\mathcal{A}$ at high frequency, we introduce the parameter $\eps\ll 1$ and consider the problem
\begin{align} \label{eq:Homo-HighFreqApprox}
	\mathcal{A}_{\eps}u_{\eps}(x) &:= -\grad\cdot a\bracs{\frac{x}{\eps}}\grad u_{\eps}(x) = \omega^2 u_{\eps}(x), \qquad x\in\ddom_{h}:=[0,1)^d,
\end{align}
with periodic boundary conditions on $\ddom_h$, in the limit as $\eps\rightarrow0$ (see figure \ref{fig:Diagram_HomoHighFreq}).
Using the theory and techniques from homogenisation theory, we would look to derive an effective problem of the form
\begin{align} \label{eq:Homo-HighFreqApproxEffective}
	\mathcal{A}_0 u_0(x) &:= -\grad\cdot a_{\mathrm{hom}}\grad u_0(x) = \omega^2 u_0(x), \qquad x\in\ddom_h,
\end{align} 
where $a_{\mathrm{hom}}$ represents the homogenised or effective material properties.
We would then look to study this effective problem, and prove convergence (in an appropriate norm) of the spectrum of the operator $\mathcal{A}_{\eps}$ to that of $\mathcal{A}_0$.
The convergence results ensure that $\mathcal{A}_0$ serves as a faithful approximation to our original problem, and thus its behaviour is reflective of the original problem we set out to model.
Alternative frequency approximations, such as the finite frequency approximation, are also realisable and adhere to a similar ethos.
In the finite frequency regime, we introduce the length scale $L>0$ and form the domain
\begin{align*}
	\ddom_f &= \bigcup_{T\in\clbracs{0,...,L}^d}\bracs{Q + T}\subset[0,L)^d,
\end{align*}
again imposing periodic boundary conditions, and study the ``limit" of the problem
\begin{align*}
	\mathcal{A}_L u_L(x) &:= -\grad\cdot a(x)\grad u_L(x) = \omega^2 u_L(x), \qquad x\in\ddom_f
\end{align*}
as $L\rightarrow\infty$ (figure \ref{fig:Diagram_HomoFreqRanges}).
The two regimes are related through appropriate length and frequency scaling, indeed if we make the domain-length-scale substitution $y = \frac{x}{\eps}$ into the high frequency problem we observe that
\begin{align*}
	-\grad\cdot a\bracs{\frac{x}{\eps}}\grad u_{\eps}(x) &= \omega^2 u_{\eps}(x), &\qquad x\in\ddom_h, \\
	\Leftrightarrow -\eps\grad_{y}\cdot \eps a(y)\grad_{y}u_{L}(y) &= \omega^2 u_{L}(y), &\qquad y\in\ddom_f,
\end{align*}
where $L = \recip{\eps}$.
Thus, we can transition between the high and finite frequency approximations using the length and frequency scale transforms $x \mapsto \eps x$ and $\omega \mapsto \frac{\omega}{\eps}$.

Having decided on a frequency regime to work in, the interest now lies in studying the effective problem and determining under what conditions the effective material exhibits time (or spatial) dispersion, which can result in interesting phenomena such as band-gaps, or metamaterial behaviour.
Such media are described by a governing equation in the form
\begin{align} \label{eq:DispersiveEqn}
	-\grad\cdot A\grad u &= \beta(\omega)u,
\end{align}
for some frequency-dependent $\beta(\omega)$.
If $\beta(\omega) = omega^2 b(\omega)I$ then rearranging \eqref{eq:DispersiveEqn} provides the form
\begin{align} \label{eq:DispersiveEqn-FreqDepA}
	-\grad\cdot A(\omega)\grad u &= \omega^2 u,
\end{align}
where it is clear from the physical interpretation of $A$ that the material parameters exhibit non-trivial dependence on the frequency.
When considering the wave equation in a strong elliptic setting (that is, the function $a(x/\eps)$ in \eqref{eq:Homo-HighFreqApprox} and its inverse are uniformly bounded), homogenisation theory establishes \tstk{as reported in Kirill-Yulia-Sasha crit contrast} that the effective medium is described by an equation of the form \eqref{eq:Homo-HighFreqApproxEffective} with $a_{\mathrm{hom}}$ a constant, leaving no room for time-dispersion.
This result is also reported to carry over into the vector setting, including the Maxwell system for example.

Frequency dispersion leads to effects such as the opening of spectral band gaps and metamaterial behaviour --- where the (effective) material parameters $a_{\mathrm{hom}}$ acquire non-trivial dependence on the frequency $\omega$, and take negative values within certain frequency ranges.
As such, there have been a number of studies exploring the conditions under which an effective material can be made to exhibit such behaviour. \tstk{maybe here say that you're looking for an effective (spectral) problem of the form $-\grad a(\omega)\grad u = -\omega^2 u$, or are considering $-\grad a\grad u + \beta(\omega)u = 0$. See Kirill-Sasha-Yulia's critical contrast papers.}
This resulted in the study of materials under \emph{critical contrast}, also referred to as \emph{high-contrast} problems/materials.
Being under critical contrast means that the components of the medium $Q_0$ and $Q_1$ must have specific relative material properties, governed by the size of the period cell -- in the high frequency context above, this means that the parameters $a$ and $l$ must be made to scale in a particular fashion with $\eps$. \tstk{should mention that two-scale convergence needed to be developed to properly handle these homogenised problems? As the usual assumptions on $a$ (uniformly elliptic etc) and }
Materials under critical contrast were found to exhibit frequency dispersion and band-gap spectra in their associated effective problems.
\tstk{dump the studies here, providing context for each.}
