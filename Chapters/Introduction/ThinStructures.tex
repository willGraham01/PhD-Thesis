\section{Mathematical Treatment of Composite, Periodic Media} \label{sec:MathHomLR}
\tstk{this is the section where I plan to talk about how band-gaps, 2-composite media, etc, have been treated in the mathematical community. Also need to talk about work on thin structures, and setup the stage for convergence to QGs.}

At their heart, photonic crystals are simply periodic mediums whose period cell consists of a set of (dielectric material) inclusions embedded into another (background, or ``matrix") material.
With developing interest in designing photonic crystals, there have been a number of studies dedicated to understanding the factors which contribute to the opening of spectral band gaps in periodic, composite media (the wider class of materials into which photonic crystals fall).
Before discussing the relevant results in the literature, we provide a toy setup for the study of wave propagation in a composite, periodic medium, to aid in contextualising what follows.
A suitable governing equation for wave propagation is the wave equation;
\begin{align*}
	-\grad\cdot A(x)\grad u(x,t) &= \pdiff{u}{t}(x,t),
\end{align*}
where $A(x)$ represents the material properties relevant to wave propagation, and may be spatially varying if the material is non-homogeneous.
For example, in the acoustic setting $A$ would represent the ratio of the elastic modulus and mass density, whilst in the electromagnetic setting would represent the (inverse) product of the electric permittivity and magnetic permeability.
More precise models would use the system of Maxwell equations or the equations of elasticity to model wave propagation in place of the wave equation, at the cost of increased complexity that comes with handling vector-valued systems.
Seeking a Bloch wave solution $u(x,t) = \hat{u}(x)\e^{\rmi\omega t}$ and de-dimensionalising the equation via $y=lx, l>0$\footnote{In the event we are looking at a finite medium, $l$ is typically the spatial extent of the medium. For studying periodic materials, $l$ can typically be taken to be the period of the medium so that the period cell has unit non-dimensional length.} and $a(y) = \hat{A}A(x/l)$, one obtains the eigenvalue problem
\begin{align*}
	-\grad\cdot a\bracs{y}\grad U(y) = z U(y),
	&\qquad 
	z = \frac{\omega^2 l^2}{\hat{A}},
\end{align*}
where $U(y)=\hat{u}(x/l)$.
In the event that $a$ is constant, the parameter $z$ represents (up to a constant) the square of the ratio of the spatial extent of the medium against the wavelength of propagating waves.
In the electromagnetic setting for example, one has
\begin{align*}
	z &= \frac{l^2 \omega^2 \eps_\mathrm{r}\mu_{\mathrm{r}}}{c^2},
\end{align*}
where $\eps_\mathrm{r},\mu_{\mathrm{r}}$ are the relative permittivity and permeability of the medium.
Along with a governing equation, we also need to specify our domain that represents our composite medium.
Typically a periodic composite medium is represented as filling the space $\reals^d$, with period cell $Q\subset\reals^d$ that is further divided into two regions, the ``inclusion" $Q_0$ and ``background" $Q_1$ with $\overline{Q_0}\subset Q$ and $Q_1=Q\setminus\overline{Q_0}$.
The function $a$ is then the $Q$-periodic function
\begin{align*}
	a(y) &= \begin{cases} a & y\in Q_i, \\ 1 & y\in Q_j, \end{cases}
	\qquad i,j\in\clbracs{0,1}, \ i\neq j.
\end{align*}
with the regions $Q_0$ and $Q_1$ typically labelled the ``soft" (where $a(y)=a$) and ``stiff" ($a(y)=1$), and the constant $a$ representing the contrast between the material parameters.
One then looks to determine or examine those pairs $z, U$ that satisfy the governing equation, with any ``gaps" in $z$ representing spectral band gaps that the composite material possesses, and relate the emergence of these gaps back to relations between quantities such as the material contrast $a$ or inclusion-volume fraction $\abs{Q_0}/\abs{Q}$.

The existence of spectral gaps, and the extent of control one can exert on them through the material parameters, has been the subject of several studies.
The study in \tstk{Fig. Kuchment: Band-Gap structure of periodic dielectric and acoustic media (scalar)} focused on the spectrum of the operator $-\grad\cdot a\bracs{y}\grad$ for a composite, periodic medium with $Q=[0,1)^3$, and stiff inclusions $Q_0$ being cubes of side length $1-\delta$.
Provided that $\delta, a\delta^{-1}$ and $a^{-1}\delta^2$ are sufficiently small, the existence of spectral gaps was found to be guaranteed, and could even be opened up in any finite part of the spectrum.
A follow-up study \tstk{F.Kuch: Band-gap structure of the spectrum of periodic Maxwell operators} established similar results on existence of spectral band gaps for the \tstk{defn? F and K considered $\curl{}\bracs{a(x)\curl{}}$ in that study, can't really call it Maxwell operator though} in a similar material.
\tstk{now onto critical contrast, and homogenisation. Why is homogenisation relevant (need to establish finite frequency approximation setup too)? What is critical contrast (scaling of the inclusion volume ratio to the material properties)? How does it relate to the study of photonic crystals and/or metamaterials? The critical contrast and volume ratio of the inclusions shrinking to zero then brings us nicely on towards singular inclusions and Exner-Post, K-Zheng in the next section too.}