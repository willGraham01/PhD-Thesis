\section{The Maxwell Equations and Derived Systems} \label{sec:Intro-Maxwell}
Wave propagation in electromagnetic contexts is governed by the system of Maxwell equations;
\begin{align*}
	\grad\cdot \mathbf{D} = \rho_f, &\qquad
	\curl{}\mathbf{E} = -\pdiff{\mathbf{B}}{t}, \\
	\grad\cdot \mathbf{B} = 0, &\qquad
	 \curl{}\mathbf{H} = J_f + \pdiff{\mathbf{D}}{t},
\end{align*}
where the vector fields $\mathbf{E}$, $\mathbf{D}$, $\mathbf{H}$, and $\mathbf{B}$ represent (respectively) the electric field, electric displacement field, magnetic field, and magnetic induction field, and the functions $\rho_f$ and $J_f$ are the (free) electric charge density and (free) electric current density respectively \cite{jackson1999classical, cessenat1996mathematical}.
This system is incomplete without constitutive relations informing us how $\mathbf{D}$ and $\mathbf{B}$ depend on $\mathbf{E}$ and $\mathbf{H}$, and we will concern ourselves with the linear approximations
\begin{align*}
	\mathbf{D} = \epsilon_m \mathbf{E}, \qquad \mathbf{B} = \mu_{m}\mathbf{H},
\end{align*}
where $\epsilon_m$ (respectively $\mu_m$) is the electric permittivity (magnetic permeability) of the material.
Further to our consideration of photonic crystals, we also treat $\epsilon_m$ and $\mu_m$ as time-independent, scalar-valued functions of position $x$, which is suitable for studying inhomogeneous, isotropic media such as photonic crystals\footnote{For a more general material the constitutive relations can be much more complex, potentially being non-linear and spatially varying. 
Other effects such as hysteresis in ferromagnets can introduce time dependencies, whilst Lorentz materials have material parameters that depend on the frequency of incident electromagnetic radiation.}.
Under these constitutive relations and in the absence of free charges and currents, the Maxwell system reduces to
\begin{align*}
	\grad\cdot \epsilon_m \mathbf{E} = 0, &\qquad
	\curl{}\mathbf{E} = -\mu_m\pdiff{\mathbf{H}}{t}, \\
	\grad\cdot \mu_m \mathbf{H} = 0, &\qquad
	\curl{}\mathbf{H} = \epsilon_m \pdiff{\mathbf{E}}{t}.
\end{align*}
One can then seek time-harmonic solutions (or more precisely, take a Fourier transform in time), 
\begin{align*}
	\mathbf{E}\bracs{x,t} = E\bracs{x}e^{-\rmi\omega t},
	&\qquad \mathbf{H}\bracs{x,t} = H\bracs{x}e^{-\rmi\omega t},
\end{align*}
and write the resulting system
\begin{align*}
	\grad\cdot \epsilon_m \mathbf{E} = 0,
	&\qquad \recip{\rmi\mu_m}\curl{}E = \omega H, \\ 
	\grad\cdot \mu_m \mathbf{H} = 0,
	&\qquad -\recip{\rmi\epsilon_m}\curl{}H = \omega E,.
\end{align*}
The equations involving the curl can be written in matrix form, and correspond to the spectral problem for the ``operator" 
\begin{align*}
	\mathcal{M} &:=
	\begin{pmatrix}
		0 & -\recip{\rmi\epsilon_m}\curl{} \\
		\recip{\rmi\mu_m}\curl{} & 0
	\end{pmatrix}.
\end{align*}
We will come to call $\mathcal{M}$ the Maxwell operator, however we need to be slightly careful in defining it so that we respect the divergence-free conditions, and end up with a self adjoint operator.
Typically this not difficult if the domain we are considering is smooth (or the whole of $\reals^d$), even the regularity of $\epsilon_m$ and $\mu_m$ tends not to be an issue in these contexts.
However there is significantly more work that needs to be done for domains with (possibly non-smooth) metallic inclusions or boundaries, with the work of \cite{birman1987l2, birman1989selfadjoint} providing a detailed study of the Maxwell operator in these contexts.
For the purposes of this review, we can define the Maxwell operator $\mathcal{M}$ by assigning it the domain
\begin{align*}
	\dom{\mathcal{M}} = \left\{ \bracs{E,H} \ \middle\vert \right. 
	&
	\left. E\in L^2\bracs{\ddom, \epsilon_m\md x}^3, \ H\in L^2\bracs{\ddom, \mu_m\md x}^3, \right. \\
	&
	\left. \curl{}E, \ \curl{}H\in\ltwo{\ddom}{x}^3, \right. \\
	&
	\left. \grad\cdot \epsilon_m E = 0, \ \grad\cdot \mu_m H = 0 \right\},
\end{align*}
where the derivatives are understood in the weak (or distributional) sense, and $\ddom\subset\reals^3$ is our domain.
The weighted spaces $L^2\bracs{\ddom, w\md x}^3$ consist of the square-integrable functions $f:\reals^3\rightarrow\complex^3$ equipped with the norm
\begin{align*}
	\norm{ f }_{L^2\bracs{\ddom, w\md x}^3} &= \integral{\ddom}{\abs{f(x)}^2 w(x)}{x}.
\end{align*}
Under this setup, the Maxwell operator $\mathcal{M}$ is self-adjoint and the ``Maxwell system" is the spectral problem for this operator.
The spectrum of $\mathcal{M}$ determines the frequencies of light that can propagate in the medium, with any intervals in $\omega$ that are absent from the spectrum of $\mathcal{M}$ corresponding to band gaps.

Even with the Maxwell operator being self-adjoint, it is still not elliptic and has eigenvalues that extend across the whole real line (that is, to both $\pm\infty$), making direct analysis of it difficult.
To tackle this, $\mathcal{M}$ is applied to itself (also referred to as squaring the operator) to produce a positive-definite operator whose action decouples the $E$ and $H$ fields, resulting in the ``curl-of-the-curl" spectral problems
\begin{subequations} \label{eq:Intro-CurlCurlEqns}
	\begin{align}
		\epsilon_m^{-1}\curl{}\bracs{\mu_m^{-1}\curl{}E} &= \omega^2 E, \\
		\mu_m^{-1}\curl{}\bracs{\epsilon_m^{-1}\curl{}H} &= \omega^2 H,
	\end{align}
\end{subequations}
the eigenvalues $\omega^2$ \emph{of either problem} then determining the eigenvalues of $\mathcal{M}$.
If the mediums properties are independent of one of the coordinates (say $x_3$), so $\epsilon_m(x)=\epsilon_m(x_1,x_2)$ and $\mu_m(x)=\mu_m(x_1,x_2)$ only, then one can also consider the waves propagating in the $\bracs{x_1,x_2}$-plane.
In this case, the action of $\mathcal{M}$ decouples into separate actions on the transverse electric (TM) field $\bracs{E_1,E_2,0,0,0,H}^\top$ and transverse magnetic (TM) field $\bracs{0,0,E_3,H_1,H_2,0}^\top$, yielding the equations
\begin{subequations} \label{eq:Intro-AcousticApprox}
	\begin{align}
		-\mu_m^{-1}\grad\cdot\epsilon_m^{-1}\grad E_3(x_1,x_2) &= \omega^2 E_3(x_1,x_2), \\
		-\epsilon_m^{-1}\grad\cdot\mu_m^{-1}\grad H_3(x_1,x_2) &= \omega^2 H_3(x_1,x_2).
	\end{align}
\end{subequations}
These equations may also be referred to as ``acoustic approximations", as they also appear when studying acoustic waves in periodic media.

Having provided an outline of the important governing equations and operators that describe wave propagation in photonic crystals, we move on to a brief discussion of the spectra of periodic media.
Here, we have an operator $\mathcal{A}$ with\footnote{Think of $\mathcal{A}$ as being one of \eqref{eq:Intro-CurlCurlEqns} or \eqref{eq:Intro-AcousticApprox} with $\epsilon_m$ and $\mu_m$ periodic.} periodic coefficients with respect to the period cell $Q=[0,1)^d$.
The natural transform to apply is the Gelfand (or Floquet) transform, which first requires us to introduce the dual cell (also termed the Brillouin zone in solid state physics) $B=[-\pi,\pi)^d$ to $Q$.
We then define the Gelfand transform $\gelfand u$ of the function $u$ as 
\begin{align*}
	\gelfand u(x, \qm) &= \sum_{n\in\integers^d} u(x+n)\e^{\rmi\qm(x+n)},
\end{align*}
where the variable $\qm$ is called the \emph{quasi-momentum}, the analogue of the dual variable in the Fourier transform.
We will go into more details in section \ref{sec:TP-GelfandTransform}, restricting ourselves to a short summary of the key affects of this transform.
The foremost effect of this transform is that we can replace the spectral problem for $\mathcal{A}$ with the spectral problem for each member of a family of operators $\mathcal{A}_{\qm}$ parametrised by the quasi-momentum.
Importantly, since each $\mathcal{A}_{\qm}$ now acts on a compact domain, under ellipticity assumptions these operators (have compact resolvents and thus) possess discrete spectra.
This allows us to order the eigenvalues $\lambda_j\bracs{\qm}$, $j\in\naturals$, of $\mathcal{A}_{\qm}$ in ascending order in $j$.
The (continuous) functions $\lambda_j$ of $\qm$ are called the \emph{dispersion branches}, or alternatively the \emph{dispersion relations} or \emph{(spectral) band functions}.
It holds that the spectrum of $\mathcal{A}$ is equal to the union of the spectra of the $\mathcal{A}_\qm$ --- that is, the projection of the graphs $y=\lambda_j(\qm)$ onto the $y$ axis provides the spectrum of $\mathcal{A}$.
Explicitly, we have that
\begin{align*}
	\sigma\bracs{\mathcal{A}} &= \bigcup_{\qm\in B} \sigma\bracs{\mathcal{A}_{\qm}}
	= \bigcup_{j\in\naturals} \lambda_j\bracs{B}
	= \bigcup_{j\in\naturals} \sqbracs{ \min_{\qm\in B}\lambda_j, \max_{\qm\in B}\lambda_j },
\end{align*}
the final equality coming from the fact that the branches $\lambda_j$ are continuous functions of $\qm$, and the interval $\sqbracs{ \min_{\qm}\lambda_j, \max_{\qm}\lambda_j }$ being labelled the $j^{\text{th}}$ spectral band of $\mathcal{A}$.
This highlights where the band-gap structure of these composite materials comes from, whenever 
\begin{align} \label{eq:Intro-DispersionBranchGapIneq}
	\max_{\qm}\lambda_{j-1} < \min_{\qm}\lambda_{j},
\end{align}
there is a corresponding gap between the end of the $(j-1)^{\text{th}}$ band and the beginning of the $j^{\text{th}}$.
It is known that for ordinary differential equations of second order the strict version of the inequality in \eqref{eq:Intro-DispersionBranchGapIneq} holds, so the spectral bands cannot overlap but may touch \cite[chapter XIII]{reed1978iv}.
As such, it is conceivable that alterations to the material may open up gaps between adjacent bands.
In two dimensions (or higher) the spectral bands can and usually do overlap, which makes the task of designing a material with band gaps harder though not impossible.

\subsection{Non-dimensionalisation of the Maxwell System} \label{ssec:Intro-NonDimMax}
We now turn specifically towards the setup of domains that model photonic crystals, and the process of non-dimensionalising the governing equations and consideration of the various free parameters that emerge from this process.
Regrettably, this process is often skipped over in the mathematical literature, with most works electing to start from a non-dimensionalised system.
Whilst there is nothing wrong with choosing to start here, there are several sets of non-dimensional parameters that one can choose to express the resulting system in, and it is important to keep track of what each set represents when interpreting physical behaviour.
This can also lead to some rather confusing (and seemingly conflicting) language in the literature, and as such we provide an explicit account of this process.

A periodic medium can be modelled by specifying a period cell $Q=[0,L)^d$, $L>0$, and then taking a union of translated copies of this period cell
\begin{align*}
	\bigcup_{n\in\integers^d} (Q + nL),
\end{align*}
to fill the whole of $\reals^d$.
Note that the requirement that $Q$ be (hyper-) cubic in shape is not restrictive, as long as the material is periodic in $d$ linearly independent directions one can apply a linear transform to produce a (hyper-) cubic period cell.
For a composite such as a photonic crystal, this period cell $Q$ is then further divided into two regions, the \emph{inclusions} $Q_0$ with $\overline{Q_0}\subset Q$ and the \emph{bulk} (also called \emph{matrix} or \emph{background}) $Q_1:=Q\setminus \overline{Q_0}$.
The electric permittivity and magnetic permeability of the medium are then defined as the $Q$-periodic functions with
\begin{align*}
	\epsilon_m(x) = \begin{cases} \epsilon^{\mathrm{inc}} & x\in Q_0, \\ \epsilon^{\mathrm{bulk}} & x\in Q_1, \end{cases}
	\qquad
	\mu_m(x) = \begin{cases} \mu^{\mathrm{inc}} & x\in Q_0, \\ \mu^{\mathrm{bulk}} & x\in Q_1, \end{cases}
\end{align*}
and we also suppose that the inclusions have a characteristic size $L-l$, $0<l<L$, giving the period cell illustrated in figure \ref{fig:Diagram_ScalingDimensionfull}.
\begin{figure}[t]
	\centering
	\begin{subfigure}[t]{0.45\textwidth}
		\centering
		\includegraphics[scale=1.0]{Diagram_ScalingDimensionfull.pdf}
		\caption{\label{fig:Diagram_ScalingDimensionfull} Schematic illustration of the period cell of a composite, periodic medium, prior to non-dimensionalisation.}
	\end{subfigure}
	~
	\begin{subfigure}[t]{0.45\textwidth}
		\centering
		\includegraphics[scale=1.0]{Diagram_ScalingDimensionless.pdf}
		\caption[Illustration of the relationship between physical properties and dimensionless parameters for composite media.]{\label{fig:Diagram_ScalingDimensionless} The period cell of the domain of the non-dimensionalised problem.}
	\end{subfigure}
\end{figure}

We now wish to treat the system of Maxwell equations mathematically, and so proceed to write the equations with respect to dimensionless variables.
Introduce the dimensionless ``electric" and ``magnetic" fields $\tilde{E}$ and $\tilde{H}$ and corresponding dimensionless ``spatial" variable $\tilde{x}$ by
\begin{align*}
	H = \hat{h}\tilde{H}, 
	\qquad E = \hat{h}\bracs{\frac{\mu^{\mathrm{bulk}}}{\epsilon^{\mathrm{bulk}}}}^{\recip{2}}\tilde{E},
	\qquad x = L\tilde{x},
\end{align*}
where $\hat{h}$ is a constant with the units of magnetic field (Amperes per metre, $\mathrm{A}\mathrm{m}^{-1}$), and $\epsFS$ and $\muFS$ are (respectively) the permittivity and permeability of vacuum.
The Maxwell system
\begin{align*}
	\mu_m^{-1}\curl{}E &= -\rmi\omega H, \\
	\epsilon_m^{-1}\curl{}H &= \rmi\omega E,
\end{align*}
then becomes
\begin{align*}
	\mu_m^{-1}(x) \ \curl{\tilde{x}}\tilde{E} 
	&= -\rmi\omega L\bracs{\frac{\epsilon^{\mathrm{bulk}}}{\mu^{\mathrm{bulk}}}}^{\recip{2}} \tilde{H}, \\
	\epsilon_m^{-1}(x) \ \curl{\tilde{x}}\tilde{H} 
	&= \rmi\omega L\bracs{\frac{\mu^{\mathrm{bulk}}}{\epsilon^{\mathrm{bulk}}}}^{\recip{2}} \tilde{E},
\end{align*}
Multiplying by $\mu^{\mathrm{bulk}}$ in the first equation, and $\epsilon^{\mathrm{bulk}}$ in the second then gives us
\begin{align*}
	\bracs{\frac{\mu_m}{\mu^{\mathrm{bulk}}}}^{-1}(x) \ \curl{\tilde{x}}\tilde{E} 
	&= -\rmi\omega L\bracs{\epsilon^{\mathrm{bulk}}\mu^{\mathrm{bulk}}}^{\recip{2}} \tilde{H}, \\
	\bracs{\frac{\epsilon_m}{\epsilon^{\mathrm{bulk}}}}^{-1}(x) \ \curl{\tilde{x}}\tilde{H} 
	&= \rmi\omega L\bracs{\epsilon^{\mathrm{bulk}}\mu^{\mathrm{bulk}}}^{\recip{2}} \tilde{E}.
\end{align*}
Defining the relative permittivity and permeabilities
\begin{align*}
	\eps_r\bracs{\tilde{x}} = \frac{\epsilon_m(x)}{\epsilon^{\mathrm{bulk}}},
	\qquad \mu_r\bracs{\tilde{x}} = \frac{\mu_m(x)}{\mu^{\mathrm{bulk}}},
\end{align*}
and the ``bulk frequency"
\begin{align*}
	\omega_{\mathrm{bulk}}
	&= L^{-1}\bracs{\epsilon^{\mathrm{bulk}}\mu^{\mathrm{bulk}}}^{-\recip{2}},
\end{align*}
we finally obtain
\begin{align*}
	\mu_r^{-1} \ \curl{\tilde{x}}\tilde{E} 
	&= -\rmi\frac{\omega}{\omega_{\mathrm{bulk}}} \tilde{H}, \\
	\eps_r^{-1} \ \curl{\tilde{x}}\tilde{H} 
	&= \rmi\frac{\omega}{\omega_{\mathrm{bulk}}} \tilde{E}.
\end{align*}
This results in the domain setup illustrated in figure \ref{fig:Diagram_ScalingDimensionless}: we are now studying the (dimensionless) equations
\begin{subequations}
	\begin{align} \label{eq:Intro-NonDimMaxwell}
		\mu_r^{-1} \ \curl{}\tilde{E} 
		&= -\rmi z \tilde{H}, \\
		\eps_r^{-1} \ \curl{}\tilde{H} 
		&= \rmi z \tilde{E},
	\end{align}
\end{subequations}
for $\tilde{x}\in\reals^d$, with a period cell $\widetilde{Q} = \recip{L}Q = [0,1)^d$, a bulk region $\widetilde{Q}_1 = \recip{L}Q_1$ of size $0<\delta:=\frac{l}{L}<1$, an inclusion $\widetilde{Q}_0 = \recip{L} Q_0$ of size $1-\delta$, and (dimensionless) relative permittivity and permeability
\begin{align*}
	\eps_r\bracs{\tilde{x}} = 
	\begin{cases} 
		1 & \tilde{x}\in\widetilde{Q}_1, \\ 
		\frac{\epsilon^{\mathrm{inc}}}{\epsilon^{\mathrm{bulk}}} & \tilde{x}\in\widetilde{Q}_0, 
	\end{cases}
	&\qquad
	\mu_r\bracs{\tilde{x}} = 
	\begin{cases} 
		1 & \tilde{x}\in\widetilde{Q}_1, \\ 
		\frac{\mu^{\mathrm{inc}}}{\mu^{\mathrm{bulk}}} & \tilde{x}\in\widetilde{Q}_0.
	\end{cases}
\end{align*}
From here, one can derive the dimensionless ``curl-of-the-curl" equations and acoustic approximations.

Now we highlight that our choice of non-dimensionalisation is far from unique, and there are several alternatives one could choose from.
For example, we could non-dimensionalise $E$ as
\begin{align*}
	E = \hat{h}\bracs{\frac{\mu^{\mathrm{inc}}}{\epsilon^{\mathrm{inc}}}}^{\recip{2}}\tilde{E},
\end{align*}
in which case one would still obtain a dimensionless system with three free parameters akin to figure \ref{fig:Diagram_ScalingDimensionless}, only with the values of $\epsilon_r$ and $\mu_r$ being unity on the inclusion and the ratio of bulk to inclusion properties on $\tilde{Q}_1$.
Furthermore, it is common to adopt the non-magnetic approximation when modelling photonic crystals, assuming that the magnetic permeabilities of the inclusion and bulk materials are equal (without loss of generality, to $\muFS$).
In this case, we can define the quantities
\begin{align*}
	\omega_0 = L^{-1}\bracs{\epsFS\muFS}^{-\recip{2}}, \quad
	\lambda^{\mathrm{inc}} = \omega_0^{-1}\bracs{\epsilon^{\mathrm{inc}}\muFS}^{-\recip{2}}, \quad
	\lambda^{\mathrm{bulk}} = \omega_0^{-1}\bracs{\epsilon^{\mathrm{bulk}}\muFS}^{-\recip{2}},
\end{align*}
with $\omega_0$ representing the frequency at which a wave in free space of wavelength $L$ propagates with, and $\lambda^{\mathrm{inc}}$ (respectively $\lambda^{\mathrm{bulk}}$) the corresponding wavelength of a wave propagating with frequency $\omega_0$ in the inclusion (bulk).
The system \eqref{eq:Intro-NonDimMaxwell} then becomes
\begin{subequations} \label{eq:Intro-NonDimMaxwellLengths}
	\begin{align}
		\curl{}\tilde{E} &= -\rmi z\tilde{H}, \\
		\lambda_{r}^2\curl{}\tilde{H} &= \rmi z\tilde{E},
	\end{align}
\end{subequations}
where
\begin{align*}
	\lambda_r\bracs{\tilde{x}} = 
	\begin{cases} 
		\frac{\lambda^{\mathrm{bulk}}}{L} & \tilde{x}\in\widetilde{Q}_1, \\ 
		\frac{\lambda^{\mathrm{inc}}}{L} & \tilde{x}\in\widetilde{Q}_0,
	\end{cases}
	\qquad
	z = \frac{\omega}{\omega_0}.
\end{align*}
Here we have three independent, dimensionless parameters
\begin{align} \label{eq:Intro-NonDimLengthScales}
	\delta = \frac{l}{L}, \quad
	\tilde{l}^{\mathrm{inc}} = \frac{\lambda^{\mathrm{inc}}}{L}, \quad
	\tilde{l}^{\mathrm{bulk}} = \frac{\lambda^{\mathrm{bulk}}}{L},
\end{align}
which are all ratios of lengths.
Having all the dimensionless parameters represent ratios of lengths is typically done to make the description of the phenomenons of \emph{resonance} and \emph{critical contrast} more intuitive, which will be done in section \ref{ssec:Intro-CritContrast}.
In fact, in the mathematical literature the non-dimensional Maxwell system (under the non-magnetic assumption) \eqref{eq:Intro-NonDimMaxwellLengths} is simply introduced as \emph{the} Maxwell system, and one typically sees the notation ``$\eps^{-1}$" for $\lambda_r^2$ in \eqref{eq:Intro-NonDimMaxwellLengths}.
This can lead to some confusion on the part of a new researcher in the field, as it is common to see such an ``$\eps$" used implicitly as both the (reciprocal square) ratio of lengths $\lambda_r^2$ (as it appears in \eqref{eq:Intro-NonDimMaxwellLengths}) and as the relative permittivity $\eps_r$ (as it appears in \eqref{eq:Intro-NonDimMaxwell} in the literature).
As such, publications may talk about $\eps$ as if it were a ratio of length scales, whilst others may refer to seemingly the same quantity as a contrast between permittivites.

The benefits of non-dimensionalising are that all possible behaviours of the Maxwell system are characterised by the values these independent parameters take, and the (non-dimensional) spectrum is qualitatively representative of the physical frequencies, and can be obtained through ``re-dimensionalising" the spectral parameter $z$.
Given that the various regimes of these parameters govern the behaviour of the system, it is natural to consider the various asymptotic limits of these free parameters and the resulting asymptotic problems.
Typically these asymptotic (or effective) problems are easier to handle analytically and numerically than the original system --- which is particularly pronounced for the Maxwell system --- and can reveal effects that are otherwise hard to notice.
A discussion of these asymptotic limits and their relation to homogenisation of so called \emph{critical contrast} materials is the focal point of section \ref{ssec:Intro-CritContrast}.
Upon completion of this section, we now drop the overhead tilde notation on our non-dimensional fields and independent variables for brevity, and adopt $\omega$ as the symbol for the non-dimensional spectral parameter rather than $z$.