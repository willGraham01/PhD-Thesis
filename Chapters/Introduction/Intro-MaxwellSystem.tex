\section{The Maxwell Equations and Derived Systems} \label{sec:Intro-Maxwell}
Wave propagation in electromagnetic contexts is governed by the system of Maxwell equations; \tstk{Jackson, Cassent for some flavour?}
\begin{align*}
	\grad\cdot \mathbf{D} = \rho_f, &\qquad
	\curl{}\mathbf{E} = -\pdiff{\mathbf{B}}{t}, \\
	\grad\cdot \mathbf{B} = 0, &\qquad
	 \curl{}\mathbf{H} = J_f + \pdiff{\mathbf{D}}{t},
\end{align*}
where the vector fields $\mathbf{E}$, $\mathbf{D}$, $\mathbf{H}$, and $\mathbf{B}$ represent (respectively) the electric field, electric displacement field, magnetic field, and magnetic induction field, and the functions $\rho_f$ and $J_f$ are the (free) electric charge density and (free) electric current density respectively.
This system is incomplete without constitutive relations informing us how $\mathbf{D}$ and $\mathbf{B}$ depend on $\mathbf{E}$ and $\mathbf{H}$, and we will concern ourselves with the linear approximations
\begin{align*}
	\mathbf{D} = \epsilon_m \mathbf{E}, \qquad \mathbf{B} = \mu_{m}\mathbf{H},
\end{align*}
where $\epsilon_m$ (respectively $\mu_m$) is the electric permittivity (magnetic permeability) of the material.
Further to our consideration of photonic crystals, we also treat $\epsilon_m$ and $\mu_m$ as time-independent, scalar-valued functions of position $x$, which is suitable for studying inhomogeneous, isotropic media such as photonic crystals\footnote{For a more general material the constitutive relations can be much more complex, potentially being non-linear and spatially varying. 
Other effects such as hysteresis in ferromagnets can introduce time dependencies, whilst Lorentz materials have material parameters that depend on the frequency of incident electromagnetic radiation.}.
Under these constitutive relations and in the absence of free charges and currents, the Maxwell system reduces to
\begin{align*}
	\grad\cdot \epsilon_m \mathbf{E} = 0, &\qquad
	\curl{}\mathbf{E} = -\mu_m\pdiff{\mathbf{H}}{t}, \\
	\grad\cdot \mu_m \mathbf{H} = 0, &\qquad
	\curl{}\mathbf{H} = \epsilon_m \pdiff{\mathbf{E}}{t}.
\end{align*}
One can then seek time-harmonic solutions (or more precisely, take a Fourier transform in time), 
\begin{align*}
	\mathbf{E}\bracs{x,t} = E\bracs{x}e^{-\rmi\omega t},
	&\qquad \mathbf{H}\bracs{x,t} = H\bracs{x}e^{-\rmi\omega t},
\end{align*}
and write the resulting system
\begin{align*}
	\grad\cdot \epsilon_m \mathbf{E} = 0,
	&\qquad \recip{\rmi\mu_m}\curl{}E = \omega H, \\ 
	\grad\cdot \mu_m \mathbf{H} = 0,
	&\qquad -\recip{\rmi\epsilon_m}\curl{}H = \omega E,.
\end{align*}
The equations involving the curl can be written in matrix form, and correspond to the spectral problem for the ``operator" 
\begin{align*}
	\mathcal{M} &:=
	\begin{pmatrix}
		0 & -\recip{\rmi\epsilon_m}\curl{} \\
		\recip{\rmi\mu_m}\curl{} & 0
	\end{pmatrix}.
\end{align*}
We will come to call $\mathcal{M}$ the Maxwell operator, however we need to be slightly careful in defining it so that we respect the divergence-free conditions, and end up with a self adjoint operator.
Typically this not difficult if the domain we are considering is smooth (or the whole of $\reals^d$), even the regularity of $\epsilon_m$ and $\mu_m$ tends not to be an issue in these contexts.
However there is significantly more work that needs to be done for domains with (possibly non-smooth) metallic inclusions or boundaries, with the work of \cite{birman1987l2, birman1989selfadjoint} providing a detailed study of the Maxwell operator in these contexts.
For the purposes of this review, we can define the Maxwell operator $\mathcal{M}$ by assigning it the domain
\begin{align*}
	\dom{\mathcal{M}} = \left\{ \bracs{E,H} \ \middle\vert \right. 
	&
	\left. E\in L^2\bracs{\ddom, \epsilon_m\md x}^3, \ H\in L^2\bracs{\ddom, \mu_m\md x}^3, \right. \\
	&
	\left. \curl{}E, \ \curl{}H\in\ltwo{\ddom}{x}^3, \right. \\
	&
	\left. \grad\cdot \epsilon_m E = 0, \ \grad\cdot \mu_m H = 0 \right\},
\end{align*}
where the derivatives are understood in the weak (or distributional) sense, and $\ddom\subset\reals^3$ is our domain.
The weighted spaces $L^2\bracs{\ddom, w\md x}^3$ consist of the square-integrable functions $f:\reals^3\rightarrow\complex^3$ equipped with the norm
\begin{align*}
	\norm{ f }_{L^2\bracs{\ddom, w\md x}^3} &= \integral{\ddom}{\abs{f(x)}^2 w(x)}{x}.
\end{align*}
Under this setup, the Maxwell operator $\mathcal{M}$ is self-adjoint and the ``Maxwell system" is the spectral problem for this operator.
The spectrum of $\mathcal{M}$ determines the frequencies of light that can propagate in the medium, with any intervals in $\omega$ that are absent from the spectrum of $\mathcal{M}$ corresponding to band gaps.

Even with the Maxwell operator being self-adjoint, it is still not elliptic and has eigenvalues that extend across the whole real line (that is, to both $\pm\infty$), making direct analysis of it difficult.
To tackle this, $\mathcal{M}$ is applied to itself (also referred to as squaring the operator) to produce a positive-definite operator whose action decouples the $E$ and $H$ fields, resulting in the ``curl-of-the-curl" spectral problems
\begin{subequations} \label{eq:Intro-CurlCurlEqns}
	\begin{align}
		\epsilon_m^{-1}\curl{}\bracs{\mu_m^{-1}\curl{}E} &= \omega^2 E, \\
		\mu_m^{-1}\curl{}\bracs{\epsilon_m^{-1}\curl{}H} &= \omega^2 H,
	\end{align}
\end{subequations}
the eigenvalues $\omega^2$ \emph{of either problem} then determining the eigenvalues of $\mathcal{M}$.
If the mediums properties are independent of one of the coordinates (say $x_3$), so $\epsilon_m(x)=\epsilon_m(x_1,x_2)$ and $\mu_m(x)=\mu_m(x_1,x_2)$ only, then one can also consider the waves propagating in the $\bracs{x_1,x_2}$-plane.
In this case, the action of $\mathcal{M}$ decouples into separate actions on the transverse electric (TM) field $\bracs{E_1,E_2,0,0,0,H}^\top$ and transverse magnetic (TM) field $\bracs{0,0,E_3,H_1,H_2,0}^\top$, yielding the equations
\begin{subequations} \label{eq:Intro-AcousticApprox}
	\begin{align}
		-\mu_m^{-1}\grad\cdot\epsilon_m^{-1}\grad E_3(x_1,x_2) &= \omega^2 E_3(x_1,x_2), \\
		-\epsilon_m^{-1}\grad\cdot\mu_m^{-1}\grad H_3(x_1,x_2) &= \omega^2 H_3(x_1,x_2).
	\end{align}
\end{subequations}
These equations may also be referred to as ``acoustic approximations", as they also appear when studying acoustic waves in periodic media.

\tstk{possibly move this discussion to AFTER scaling discussions so $Q$ being unit size isn't too restrictive?}
Having provided an outline of the important governing equations and operators that describe wave propagation in photonic crystals, we move on to a brief discussion of the spectra of periodic media.
Since we are primarily concerned with spectral problems, working directly in $\reals^d$ with our chosen operator is often inconvenient \tstk{WHY? doesn't need to be precise right now} and so one often looks to apply a transform to aid in the analysis of such problems.
When faced with a differential operator with constant coefficients (or coefficients constant in one particular coordinate direction), one typically utilises a Fourier transform to turn the action of differentiation into the action of multiplication by (a function of) the dual variable.
Here, we have an operator $\mathcal{A}$ with\footnote{Think of $\mathcal{A}$ as being one of \eqref{eq:Intro-CurlCurlEqns} or \eqref{eq:Intro-AcousticApprox} with $\epsilon_m$ and $\mu_m$ periodic.} periodic (with respect to some period cell $Q$), rather than constant, coefficients.
The natural transform to apply is the Gelfand (or Floquet) transform, which first requires us to introduce the dual cell (also termed the Brillouin zone in solid state physics) $B=[-\pi,\pi)^d$ to $Q$.
We then define the Gelfand transform $\hat{u}$ of the function $u$ as 
\begin{align*}
	\hat{u}(x, \qm) &= \sum_{n\in\integers^d} u(x+n)\e^{\rmi\qm(x+n)},
\end{align*}
where the variable $\qm$ is called the \emph{quasi-momentum}, the analogue of the dual variable in the Fourier transform.
We will go into more details in section \ref{sec:TP-GelfandTransform}, restricting ourselves to a short summary of the key affects of this transform.
The foremost effect of this transform is that we can replace the spectral problem for $\mathcal{A}$ with the spectral problem for each member of a family of operators $\mathcal{A}_{\qm}$ parametrised by the quasi-momentum.
Importantly, since each $\mathcal{A}_{\qm}$ now acts on a compact domain, under ellipticity assumptions these operators (have compact resolvents and thus) possess discrete spectra.
This allows us to order the eigenvalues $\lambda_j\bracs{\qm}$, $j\in\naturals$, of $\mathcal{A}_{\qm}$ in ascending order in $j$.
The (continuous) functions $\lambda_j$ of $\qm$ are called the dispersion branches\footnote{Alternative names include dispersion relations, or (spectral) band functions.}, and it holds that the spectrum of $\mathcal{A}$ is equal to the union of the spectra of the $\mathcal{A}_\qm$ --- that is, the projection of the graphs $y=\lambda_j(\qm)$ onto the $y$ axis provides the spectrum of $\mathcal{A}$.
Explicitly, we have that
\begin{align*}
	\sigma\bracs{\mathcal{A}} &= \bigcup_{\qm\in B} \sigma\bracs{\mathcal{A}_{\qm}}
	= \bigcup_{j\in\naturals} \mathrm{Im}\bracs{\lambda_j\bracs{\qm}}
	= \bigcup_{j\in\naturals} \sqbracs{ \min_{\qm}\lambda_j, \max_{\qm}\lambda_j },
\end{align*}
the final equality coming from the fact that the branches $\lambda_j$ are continuous functions of $\qm$, and the interval $\sqbracs{ \min_{\qm}\lambda_j, \max_{\qm}\lambda_j }$ being labelled the $j^{\text{th}}$ spectral band of $\mathcal{A}$.
This highlights where the band-gap structure of these composite materials comes from, whenever 
\begin{align} \label{eq:Intro-DispersionBranchGapIneq}
	\max_{\qm}\lambda_{j-1} < \min_{\qm}\lambda_{j},
\end{align}
there is a corresponding gap between the end of the $(j-1)^{\text{th}}$ band and the beginning of the $j^{\text{th}}$.
It is known that for ordinary differential equations of second order the strict version of the inequality in \eqref{eq:Intro-DispersionBranchGapIneq} holds, so the spectral bands cannot overlap but may touch \tstk{refs? Pointed to a book by a couple of the references that say this: M. Reed and B. Simon, Methods of Modern Mathematical Physics, Vol. IV:Analysis of Operators}.
As such, it is conceivable that alterations to the material may open up gaps between adjacent bands.
In two dimensions (or higher) the spectral bands can and usually do overlap, which makes the task of designing a material with band gaps harder though not impossible. \tstk{BK book also has some notes on this for QG systems too that can go here.}

\subsection{Non-dimensionalisation of the Maxwell System} \label{ssec:Intro-NonDimMax}
We now turn specifically towards the setup of domains that model photonic crystals, and the process of non-dimensionalising the governing equations and consideration of the various free parameters that emerge from this process.
Regrettably, this process is often skipped over in the mathematical literature, with most works electing to start from a non-dimensionalised system.
Whilst there is nothing wrong with choosing to start here, it can lead to one making implicit (or unnoticed) assumptions about the physical system (or domain) that is being modelled (see \tstk{part of lit review where this comes in}), and confusing to a newcomer to the field.
As such, we will provide an explicit account of this process, and highlight where the aforementioned assumptions on the physical system might arise if one is not careful.

A periodic medium can be modelled by specifying a period cell $Q=[0,L)^d$, $L>0$, and then taking a union of translated copies of this period cell
\begin{align*}
	\bigcup_{n\in\integers^d} (Q + nL),
\end{align*}
to fill the whole of $\reals^d$.
Note that the requirement that $Q$ be (hyper-) cubic in shape is not restrictive, as long as the material is periodic in $d$ linearly independent directions one can apply a linear transform to produce a (hyper-) cubic period cell.
For a composite such as a photonic crystal, this period cell $Q$ is then further divided into two regions, the \emph{inclusions} $Q_0$ with $\overline{Q_0}\subset Q$ and the \emph{bulk} (also called \emph{matrix} or \emph{background}) $Q_1:=Q\setminus \overline{Q_0}$.
The electric permittivity and magnetic permeability of the medium are then defined as the $Q$-periodic functions with
\begin{align*}
	\epsilon_m(x) = \begin{cases} \epsilon^{\mathrm{inc}} & x\in Q_0, \\ \epsilon^{\mathrm{bulk}} & x\in Q_1, \end{cases}
	\qquad
	\mu_m(x) = \begin{cases} \mu^{\mathrm{inc}} & x\in Q_0, \\ \mu^{\mathrm{bulk}} & x\in Q_1, \end{cases}
\end{align*}
and we also suppose that the inclusions have a characteristic size $L-l$, $0<l<L$, giving the period cell illustrated in figure \ref{fig:Diagram_ScalingDimensionfull}.
\begin{figure}[t]
	\centering
	\begin{subfigure}[t]{0.45\textwidth}
		\centering
		\includegraphics[scale=1.0]{Diagram_ScalingDimensionfull.pdf}
		\caption{\label{fig:Diagram_ScalingDimensionfull} Schematic illustration of the period cell of a composite, periodic medium, prior to non-dimensionalisation.}
	\end{subfigure}
	~
	\begin{subfigure}[t]{0.45\textwidth}
		\centering
		\includegraphics[scale=1.0]{Diagram_ScalingDimensionless.pdf}
		\caption{\label{fig:Diagram_ScalingDimensionless} The period cell of the domain of the non-dimensionalised problem.}
	\end{subfigure}
\end{figure}

We now wish to treat the system of Maxwell equations mathematically, and so proceed to write the equations with respect to dimensionless variables. \tstk{following the example in K-Serena, or perhaps their reference?}.
Introduce the dimensionless ``electric" and ``magnetic" fields $\tilde{E}$ and $\tilde{H}$ and corresponding dimensionless ``spatial" variable $\tilde{x}$ by
\begin{align*}
	H = \hat{h}\tilde{H}, 
	\qquad E = \hat{h}\bracs{\frac{\mu^{\mathrm{bulk}}}{\epsilon^{\mathrm{bulk}}}}^{\recip{2}}\tilde{E},
	\qquad x = L\tilde{x},
\end{align*}
where $\hat{h}$ is a constant with the units of magnetic field (Amperes per metre, $\mathrm{A}\mathrm{m}^{-1}$), and $\epsFS$ and $\muFS$ are (respectively) the permittivity and permeability of vacuum.
The Maxwell system
\begin{align*}
	\mu_m^{-1}\curl{}E &= -\rmi\omega H, \\
	\epsilon_m^{-1}\curl{}H &= \rmi\omega E,
\end{align*}
then becomes
\begin{align*}
	\mu_m^{-1}(x) \ \curl{\tilde{x}}\tilde{E} 
	&= -\rmi\omega L\bracs{\frac{\epsilon^{\mathrm{bulk}}}{\mu^{\mathrm{bulk}}}}^{\recip{2}} \tilde{H}, \\
	\epsilon_m^{-1}(x) \ \curl{\tilde{x}}\tilde{H} 
	&= \rmi\omega L\bracs{\frac{\mu^{\mathrm{bulk}}}{\epsilon^{\mathrm{bulk}}}}^{\recip{2}} \tilde{E},
\end{align*}
Multiplying by $\mu^{\mathrm{bulk}}$ in the first equation, and $\epsilon^{\mathrm{bulk}}$ in the second then gives us
\begin{align*}
	\bracs{\frac{\mu_m}{\mu^{\mathrm{bulk}}}}^{-1}(x) \ \curl{\tilde{x}}\tilde{E} 
	&= -\rmi\omega L\bracs{\epsilon^{\mathrm{bulk}}\mu^{\mathrm{bulk}}}^{\recip{2}} \tilde{H}, \\
	\bracs{\frac{\epsilon_m}{\epsilon^{\mathrm{bulk}}}}^{-1}(x) \ \curl{\tilde{x}}\tilde{H} 
	&= \rmi\omega L\bracs{\epsilon^{\mathrm{bulk}}\mu^{\mathrm{bulk}}}^{\recip{2}} \tilde{E}.
\end{align*}
Defining the relative permittivity and permeabilities
\begin{align*}
	\eps_r\bracs{\tilde{x}} = \frac{\epsilon_m(x)}{\epsilon^{\mathrm{bulk}}},
	\qquad \mu_r\bracs{\tilde{x}} = \frac{\mu_m(x)}{\mu^{\mathrm{bulk}}},
\end{align*}
then defining
\begin{align*}
	\omega_{\mathrm{bulk}}
	&= L^{-1}\bracs{\epsilon^{\mathrm{bulk}}\mu^{\mathrm{bulk}}}^{-\recip{2}},
\end{align*}
we finally obtain
\begin{align*}
	\mu_r^{-1} \ \curl{\tilde{x}}\tilde{E} 
	&= -\rmi\frac{\omega}{\omega_{\mathrm{bulk}}} \tilde{H}, \\
	\eps_r^{-1} \ \curl{\tilde{x}}\tilde{H} 
	&= \rmi\frac{\omega}{\omega_{\mathrm{bulk}}} \tilde{E}.
\end{align*}
This results in the domain setup illustrated in figure \ref{fig:Diagram_ScalingDimensionless}: we are now studying the (dimensionless) equations
\begin{subequations}
	\begin{align} \label{eq:Intro-NonDimMaxwell}
		\mu_r^{-1} \ \curl{}\tilde{E} 
		&= -\rmi z \tilde{H}, \\
		\eps_r^{-1} \ \curl{}\tilde{H} 
		&= \rmi z \tilde{E},
	\end{align}
\end{subequations}
for $\tilde{x}\in\reals^d$, with a period cell $\widetilde{Q} = \recip{L}Q = [0,1)^d$, a bulk region $\widetilde{Q}_1 = \recip{L}Q_1$ of size $0<\delta:=\frac{l}{L}<1$, an inclusion $\widetilde{Q}_0 = \recip{L} Q_0$ of size $1-\delta$, and (dimensionless) relative permittivity and permeability
\begin{align*}
	\eps_r\bracs{\tilde{x}} = 
	\begin{cases} 
		1 & \tilde{x}\in\widetilde{Q}_1, \\ 
		\frac{\epsilon^{\mathrm{inc}}}{\epsilon^{\mathrm{bulk}}} & \tilde{x}\in\widetilde{Q}_0, 
	\end{cases}
	&\qquad
	\mu_r\bracs{\tilde{x}} = 
	\begin{cases} 
		1 & \tilde{x}\in\widetilde{Q}_1, \\ 
		\frac{\mu^{\mathrm{inc}}}{\mu^{\mathrm{bulk}}} & \tilde{x}\in\widetilde{Q}_0.
	\end{cases}
\end{align*}
From here, one can derive the dimensionless ``curl-of-the-curl" equations and acoustic approximations.

It is worth noting that our choice of non-dimensionalisation is not the only possible choice, one could choose to non-dimensionalise $E$ as
\begin{align*}
	E = \hat{h}\bracs{\frac{\mu^{\mathrm{inc}}}{\epsilon^{\mathrm{inc}}}}^{\recip{2}}\tilde{E},
\end{align*}
in which case one would still obtain a dimensionless system with three free parameters akin to figure \ref{fig:Diagram_ScalingDimensionless}, only with the values of $\epsilon_r$ and $\mu_r$ being unity on the inclusion and the ratio of bulk to inclusion properties on $\tilde{Q}_1$.
Furthermore, it is common to adopt the non-magnetic approximation when modelling photonic crystals, assuming that the magnetic permeabilities of the inclusion and bulk materials are equal (without loss of generality, to $\muFS$).
In this case, we can define the quantities
\begin{align*}
	\omega_0 = L^{-1}\bracs{\epsFS\muFS}^{-\recip{2}}, \quad
	\lambda^{\mathrm{inc}} = \omega_0^{-1}\bracs{\epsilon^{\mathrm{inc}}\muFS}^{-\recip{2}}, \quad
	\lambda^{\mathrm{bulk}} = \omega_0^{-1}\bracs{\epsilon^{\mathrm{bulk}}\muFS}^{-\recip{2}},
\end{align*}
with $\omega_0$ representing the frequency at which a wave in free space of wavelength $L$ propagates with, and $\lambda^{\mathrm{inc}}$ (respectively $\lambda^{\mathrm{bulk}}$) the corresponding wavelength of a wave propagating with frequency $\omega_0$ in the inclusion (bulk).
The system \eqref{eq:Intro-NonDimMaxwell} then becomes
\begin{subequations} \label{eq:Intro-NonDimMaxwellLengths}
	\begin{align}
		\curl{}\tilde{E} &= -\rmi z\tilde{H}, \\
		\lambda_{r}^2\curl{}\tilde{H} &= \rmi z\tilde{E},
	\end{align}
\end{subequations}
where
\begin{align*}
	\lambda_r\bracs{\tilde{x}} = 
	\begin{cases} 
		\frac{\lambda^{\mathrm{bulk}}}{L} & \tilde{x}\in\widetilde{Q}_1, \\ 
		\frac{\lambda^{\mathrm{inc}}}{L} & \tilde{x}\in\widetilde{Q}_0,
	\end{cases}
	\qquad
	z = \frac{\omega}{\omega_0}.
\end{align*}
Here we have three independent, dimensionless parameters
\begin{align} \label{eq:Intro-NonDimLengthScales}
	\delta = \frac{l}{L}, \quad
	\tilde{l}^{\mathrm{inc}} = \frac{\lambda^{\mathrm{inc}}}{L}, \quad
	\tilde{l}^{\mathrm{bulk}} = \frac{\lambda^{\mathrm{bulk}}}{L},
\end{align}
which are all ratios of lengths.
Having all the dimensionless parameters represent ratios of lengths is typically done to make the description of the phenomenons of \emph{resonance} and \emph{critical contrast} more intuitive, which will be done in section \ref{ssec:Intro-CritContrast}.
In fact, in the mathematical literature the non-dimensional Maxwell system (under the non-magnetic assumption) \eqref{eq:Intro-NonDimMaxwellLengths} is simply introduced as \emph{the} Maxwell system, and one typically sees the notation ``$\eps$" for $\lambda_r^2$ in \eqref{eq:Intro-NonDimMaxwellLengths}.
This can lead to some confusion on the part of a new researcher in the field, as it is common to see such an ``$\eps$" used implicitly as both the (square) ratio of lengths $\lambda_r^2$ (as it appears in \eqref{eq:Intro-NonDimMaxwellLengths}) and as the relative permittivity $\eps_r$ (as it appears in \eqref{eq:Intro-NonDimMaxwell} in the literature.

The benefits of non-dimensionalising are that all possible behaviours of the Maxwell system are characterised by the values these independent parameters take, and the (non-dimensional) spectrum is qualitatively representative of the physical frequencies, and can be obtained through ``re-dimensionalising" the spectral parameter $z$.
Given that the various regimes of these parameters govern the behaviour of the system, it is natural to consider the various asymptotic limits of these free parameters and the resulting asymptotic problems.
Typically these asymptotic (or effective) problems are easier to handle analytically and numerically than the original system --- which is particularly pronounced for the Maxwell system --- and can reveal effects that are otherwise hard to notice.
A discussion of these asymptotic limits and their relation to homogenisation of so called \emph{critical contrast} materials is the focal point of the next section (\tstk{ref}).
Upon completion of this section, we now drop the overhead tilde notation on our non-dimensional fields and independent variables for brevity, and adopt $\omega$ as the symbol for the non-dimensional spectral parameter rather than $z$.

%The high-contrast limit $a\rightarrow\infty$ for \eqref{eq:Intro-GeneralWaveEqn} was explored in \tstk{Kirill-Sasha-Luis Op Norm Resolvant Asymptotic Analysis} on (a domain with similar structure to) $Q$, who proved convergence to a limiting problem and in particular established an asymptotic approximation for the limiting spectrum.
%Although this study only worked on a finite domain $Q$ rather than a periodic medium filling $\reals^d$ with period cell $Q$, the analysis is highly relevant if one uses a Gelfand or Floquet transform (\tstk{section ref}) to study the periodic problem.
%The study \tstk{Fig. Kuchment: Band-Gap structure of periodic dielectric and acoustic media (scalar)} focused on the spectrum of the operator $-\grad\cdot a\bracs{y}\grad$ for a composite, periodic medium with $Q=[0,1)^3$, and stiff inclusions $Q_0$ being cubes of side length $l = 1-\delta$.
%Here it was proved that provided that $\delta, a\delta^{-1}$ and $a^{-1}\delta^2$ are sufficiently small, the existence of spectral gaps is guaranteed, and can even be opened up in any finite part of the spectrum.
%The follow-up study \tstk{F.Kuch: Band-gap structure of the spectrum of periodic Maxwell operators} established similar results on existence of spectral band gaps for the operator $\curl{}\bracs{a(x)\curl{}\cdot}$, an operator closely related to the Maxwell system \eqref{eq:MaxwellSystem}.
%\tstk{more studies?!?! Cooper-K-S relevant here?}
%
%\tstk{now we talk about homogenisation theory, first we should outline what it's all about.}
%The periodic nature of the material properties (encoded in the function $a(x)$) and the desire to understand the effective properties of the medium naturally lends itself to study through the process of homogenisation.
%Broadly speaking, homogenisation theory looks to take a problem on an inhomogeneous material and derive (in some limit) an ``effective" problem for that material in terms of some ``homogenised" or ``effective" material properties.
%Then by establishing convergence results for the solution of the original problem to those of the effective problem, one ensures that the approximation is valid under certain parameter regimes and can replace the potentially complex original problem with the effective problem, which is usually more tractable to analysis.
%We will attempt to illustrate some of these frameworks; suppose our interest lies in studying the eigenvalues of the wave equation \tstk{perhaps it would be better here to write the resolvent form first, IE with $\omega^2 u + f$ on the RHS. Then we can talk about proving convergence of the solutions $u_\eps$ to $u_0$ as well as convergence of the spectra.}
%\begin{align*} 
%	\mathcal{A}u &:= -\grad\cdot a(x)\grad u = \omega^2 u, \qquad x\in\reals^d,
%\end{align*}
%where $\reals^d$ is filled with a periodic, composite medium with unit cell $Q=Q_0\cup Q_1$ as described above.
%To approximate the spectrum of $\mathcal{A}$ at high frequency, we introduce the parameter $\eps\ll 1$ and consider the problem
%\begin{align} \label{eq:Homo-HighFreqApprox}
%	\mathcal{A}_{\eps}u_{\eps}(x) &:= -\grad\cdot a\bracs{\frac{x}{\eps}}\grad u_{\eps}(x) = \omega^2 u_{\eps}(x), \qquad x\in\ddom_{h}:=[0,1)^d,
%\end{align}
%with periodic boundary conditions on $\ddom_h$, in the limit as $\eps\rightarrow0$ (see figure \ref{fig:Diagram_HomoHighFreq}).
%Using the theory and techniques from homogenisation theory, we would look to derive an effective problem of the form
%\begin{align} \label{eq:Homo-HighFreqApproxEffective}
%	\mathcal{A}_0 u_0(x) &:= -\grad\cdot a_{\mathrm{hom}}\grad u_0(x) = \omega^2 u_0(x), \qquad x\in\ddom_h,
%\end{align} 
%where $a_{\mathrm{hom}}$ represents the homogenised or effective material properties, which in general may be spatially or frequency dependent (as we will discuss shortly).
%We would then look to study this effective problem, and prove convergence (in an appropriate norm) of the spectrum of the operator $\mathcal{A}_{\eps}$ to that of $\mathcal{A}_0$.
%The convergence results ensure that $\mathcal{A}_0$ serves as a faithful approximation to our original problem (at high frequencies), and thus its behaviour is reflective of the original problem we set out to model.
%Alternative frequency approximations, such as the finite frequency approximation, are also realisable and adhere to a similar ethos.
%In the finite frequency regime, we introduce the length scale $L>0$ and form the domain
%\begin{align*}
%	\ddom_f &= \bigcup_{T\in\clbracs{0,...,L}^d}\bracs{Q + T}\subset[0,L)^d,
%\end{align*}
%again imposing periodic boundary conditions, and study the ``limit" of the problem
%\begin{align*}
%	\mathcal{A}_L u_L(x) &:= -\grad\cdot a(x)\grad u_L(x) = \omega^2 u_L(x), \qquad x\in\ddom_f
%\end{align*}
%as $L\rightarrow\infty$ (figure \ref{fig:Diagram_HomoFreqRanges}).
%The two regimes are related through appropriate length and frequency scaling, indeed if we make the domain-length-scale substitution $y = \frac{x}{\eps}$ into the high frequency problem we observe that
%\begin{align*}
%	-\grad\cdot a\bracs{\frac{x}{\eps}}\grad u_{\eps}(x) &= \omega^2 u_{\eps}(x), &\qquad x\in\ddom_h, \\
%	\Leftrightarrow -\eps\grad_{y}\cdot \eps a(y)\grad_{y}u_{L}(y) &= \omega^2 u_{L}(y), &\qquad y\in\ddom_f,
%\end{align*}
%where $L = \recip{\eps}$.
%Thus, we can transition between the high and finite frequency approximations using the length and frequency scale transforms $x \mapsto \eps x$ and $\omega \mapsto \frac{\omega}{\eps}$.
%
%\tstk{this is where we will talk about relevant studies, in particular critical contrast leading to time dissipative systems and band-gaps. We will need to link this back to our work; the link is that critical contrast provides a means of inducing band-gaps in the effective medium. Our work will introduce this ``contrast" effect in an alternative manner. Also, we touch on generalised resolvents, which we will need to mention in the next section too}
%Having decided on a frequency regime to work in, the interest now lies in studying the effective problem and determining under what conditions the effective material exhibits time (or spatial) dispersion, which can result in interesting phenomena such as band-gaps, or metamaterial behaviour\footnote{Metamaterial behaviour is when the parameters describing the physical properties of the material appear to take negative or other unphysical values within certain frequency (or spatial) ranges.}.
%When handling such media, we typically face an (effective) problem in the frequency domain of the form
%\begin{align} \label{eq:DispersiveEqn}
%	-\grad\cdot A\grad u &= \beta(\omega)u,
%\end{align}
%for some frequency-dependent $\beta(\omega)$ and constant $A$.
%If $\beta(\omega) = \omega^2 b(\omega)I$ then rearranging \eqref{eq:DispersiveEqn} provides the form
%\begin{align} \label{eq:DispersiveEqn-FreqDepA}
%	-\grad\cdot A(\omega)\grad u &= \omega^2 u,
%\end{align}
%where the physical interpretation of $A$ implies that the material parameters exhibit non-trivial dependence on the frequency, and the relation to \eqref{eq:Homo-HighFreqApproxEffective} ($A(\omega)=a_{\mathrm{hom}}$) is clear.
%When considering the wave equation in a strong elliptic setting (that is, the function $a(x/\eps)$ in \eqref{eq:Homo-HighFreqApprox} and its inverse are uniformly bounded), homogenisation theory establishes \tstk{as reported in Kirill-Yulia-Sasha crit contrast, pointing to the references M. Sh. Birman and T. A. Suslina, Second order periodic differential operators. Threshold properties
%and homogenization and V. V. Zhikov, Spectral approach to asymptotic diffusion problems. (Russian)} that the effective medium is described by an equation of the form \eqref{eq:Homo-HighFreqApproxEffective} with $a_{\mathrm{hom}}$ a constant matrix, leaving no room for time-dispersion.
%This result is also reported to carry over into the vector setting, including the Maxwell system for example.
%By dropping the assumption of uniform ellipticity on $a$ the analysis becomes much more complicated, but by using the technique of two scale convergence \tstk{do I need to elaborate on what this is (and why things are more complicated)?} the study \tstk{Zhikov, Extension method two scale convergence} was able to obtain an effective problem of the form \eqref{eq:DispersiveEqn-FreqDepA}.
%This technique allows for the derivation of effective problems of materials under \emph{critical contrast}\footnote{Also referred to as \emph{high-contrast} problems/materials.}, meaning that the components of the medium $Q_0$ and $Q_1$ must have specific relative material properties governed by the size of the period cell.
%In the high frequency context above, this means that the parameters $a$ and $l$ must be made to scale in a particular fashion with $\eps$.
%Further developments \tstk{Kirill-Sasha-Luis, both papers} have demonstrated that a material being under critical contrast is sufficient for the resulting effective medium to be time-dispersive, although there is no definitive answer as to whether it is necessary. \tstk{further studies? A nice study to end on, that would link nicely to our singular structure considerations, is here: https://epubs.siam.org/doi/epdf/10.1137/1.9780898717594.ch7 by Kuchment, talking about both asymptotic limits and the resulting graph problems.}