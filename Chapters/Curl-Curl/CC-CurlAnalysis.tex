\section{Sobolev Spaces of Functions with Curls} \label{sec:CC-CurlAnalysis}
In similar vein to section \ref{sec:3DGradSobSpaces}, our focus for this section will be the analysis of curls of zero and tangential curls with respect to the measures $\ddmes$, $\massMes$ and $\dddmes$.
We will again utilise the thematic approach of the aforementioned section; first looking to understand curls of zero and tangential curls on (planes induced by an) edge $I_{jk}$, then show that we can ``add" such edge functions together to form functions on the whole graph $\graph$.

\subsection{Curls of Zero} \label{ssec:CurlsOfZero}
Our analysis begins, analogously to how it did in section \ref{ssec:GradZero}, with an examination of $\curlZero{\ddom}{\lambda_{jk}}$.
\begin{prop} \label{prop:CurlZero-Parallel}
	Let $I_{jk}$ be an edge in $\ddom$ with $e_{jk} = \widehat{x}_2$.
	Then
	\begin{align*}
		\curlZero{\ddom}{\lambda_{jk}} &= \clbracs{ \bracs{0,c_2,c_3}^\top \setVert c_2,c_3\in\pltwo{\ddom}{\lambda_{jk}} }.
	\end{align*}
\end{prop}
The proof of this result is one particular case of the more general proof that we present for the following corollary, when we no longer assume that the edge $I_{jk}$ is parallel to the $x_2$-axis.
However we note that for any $c\in\pltwo{\ddom}{\lambda_{jk}}^3$, the function $\bracs{0,c_2,c_3}^{\top}$ is simply the function $c$ with its $\widehat{x}_1$ component removed, and in the setup of proposition \ref{prop:CurlZero-Parallel} $n_{jk} = \widehat{x}_1$.
\begin{cory}[Curls of Zero on an Edge] \label{cory:CurlZero-Rotated}
	Let $I_{jk}$ be an edge of $\graph$.
	Then we have that
	\begin{align*}
		\curlZero{\ddom}{\lambda_{jk}} &= \clbracs{ c_{jk}\widehat{e}_{jk} + c_3\widehat{x}_3 \setVert c_{jk},c_3\in\pltwo{\ddom}{\lambda_{jk}} }.
	\end{align*}
\end{cory}
\begin{proof}
	First, suppose that $c\widehat{n}_{jk}\in\curlZero{\ddom}{\lambda_{jk}}$ and take an approximating sequence $\Phi^n$ for it.
	This means that we have the following convergences in $\ltwo{\ddom}{\lambda_{jk}}$:
	\begin{align*}
		\Phi^n_1 \rightarrow 0, \quad
		\Phi^n_2 &\rightarrow 0, \quad
		\Phi^n_3 \rightarrow 0, \\
		\partial_2\Phi^n_3 - \rmi\wavenumber\Phi^n_2 \rightarrow c n_1^{(jk)} = c e_2^{(jk)}, \quad
		\partial_1\Phi^n_3 - \rmi\wavenumber\Phi^n_1 &\rightarrow -c n_2^{(jk)} = c e_1^{(jk)}, \quad
		\partial_1\Phi^n_2 - \partial_2\Phi^n_1 \rightarrow 0.
	\end{align*}
	Therefore, we have that
	\begin{align*}
		\grad^{(0)}\Phi^n_3 \lconv{\ltwo{\ddom}{\lambda_{jk}}^2} c \widehat{e}_{jk},
	\end{align*}
	and $\Phi^n_3\rightarrow 0$ in $\ltwo{\ddom}{\lambda_{jk}}$.
	As such, $c\widehat{e}_{jk}\in\gradZero{\ddom}{\lambda_{jk}}$, and then proposition \ref{prop:3DGradZeroRotated} implies that $c=0$.
	
	Now, let $c\in\psmooth{\ddom}$.
	Take $\Phi, \Psi\in\csmooth{\ddom}$ which, within some open neighbourhood of $I_{jk}$, have the form
	\begin{align*}
		\Phi = \bracs{v_j - x}\cdot n_{jk} c\widehat{x}_3, &\quad
		\Psi = \bracs{v_j - x}\cdot n_{jk} c\widehat{e}_{jk}.
	\end{align*}
	This ensures that both $\Phi$ and $\Psi$ are zero on $I_{jk}$, as for $x\in I_{jk}$ we have that $v_j-x$ is some scalar multiple of $e_{jk}$.
	Furthermore, we also have that
	\begin{align*}
		\curl{}\Phi &= \begin{pmatrix} \bracs{v_j-x}\cdot n_{jk} \partial_2 c - c n_2^{(jk)} \\ -\bracs{v_j-x}\cdot n_{jk} \partial_1 c + c n_1^{(jk)} \\ 0 \end{pmatrix}
		= \begin{pmatrix} \bracs{v_j-x}\cdot n_{jk} \partial_2 c + c e_1^{(jk)} \\ -\bracs{v_j-x}\cdot n_{jk} \partial_1 c + c e_2^{(jk)} \\ 0 \end{pmatrix}, \\
		\curl{}\Psi &= \begin{pmatrix} 0 \\ 0 \\ \bracs{\partial_1 c + \partial_2 c}\bracs{x-v_j}\cdot n_{jk} + c\norm{n_{jk}}_2 \end{pmatrix}
		= \begin{pmatrix} 0 \\ 0 \\ \bracs{\partial_1 c + \partial_2 c}\bracs{x-v_j}\cdot n_{jk} + c \end{pmatrix},
	\end{align*}
	so again for $x\in I_{jk}$, we have that
	\begin{align*}
		\curl{}\phi = c\widehat{e}_{jk}, \qquad \curl{}\psi = c\widehat{x}_3,
	\end{align*}
	whenever $x\in I_{jk}$, and thus $c\widehat{e}_{jk}, c\widehat{x}_3\in\curlZero{\ddom}{\lambda_{jk}}$.
	Density of $\psmooth{\ddom}$ in $\pltwo{\ddom}{\lambda_{jk}}$ then completes the proof.
\end{proof}

Corollary \ref{cory:CurlZero-Rotated} provides the backbone for our geometric interpretation of curls of zero and tangential curls, which was given in section \ref{sec:CC-Geometric}.
The curls of zero always lie in the plane induced by $I_{jk}$, as vectors within the plane only serve to induce rotations perpendicular to said plane.
In three dimensions (that is, with the usual Lebesgue measure) we would be able to observe such rotations, however with our limited view (through $\lambda_{jk}$), we cannot.
As one might expect, the behaviour of curls of zero on $\graph$ can be characterised by the behaviour of in the individual edges, as was the case for gradients of zero and tangential gradients.
\begin{theorem}[Characterisation of $\curlZero{\ddom}{\ddmes}$] \label{thm:CurlZeroChar}
	We have that
	\begin{align*}
		\curlZero{\ddom}{\ddmes} = \clbracs{ c\in\pltwo{\ddom}{\ddmes}^3 \setVert c\in\curlZero{\ddom}{\lambda_{jk}}, \ \forall I_{jk}\in\edgeSet }.
	\end{align*}
\end{theorem}
For ease, let us denote $C = \clbracs{ c\in\pltwo{\ddom}{\ddmes}^3 \setVert c\in\curlZero{\ddom}{\lambda_{jk}}, \ \forall I_{jk}\in\edgeSet }$, and we note that thanks to corollary \ref{cory:CurlZero-Rotated} we have that
\begin{align*}
	C &= \clbracs{ c\in\pltwo{\ddom}{\ddmes} \setVert c\cdot\widehat{n}_{jk}\vert_{I_{jk}}=0, \ \forall I_{jk}\in\edgeSet }.
\end{align*}

The method of proof proceeds in much the same way as that of proposition \ref{prop:3DGradZeroChar}, the inclusion $\curlZero{\ddom}{\ddmes}\subset C$ is immediate, the reverse direction requires us to deal with the issue of reconciling the different edge-wise curls of zero as we approach the vertices.
Our first target is the aforementioned easier inclusion:
\begin{lemma} \label{lem:CurlZeroInC}
	\begin{align*}
		\curlZero{\ddom}{\ddmes} \subset C.
	\end{align*}
\end{lemma}
\begin{proof}
	Let $c\in\curlZero{\ddom}{\ddmes}$, and take an approximating sequence $\Phi^n$.
	Clearly we have that
	\begin{align*}
		\norm{ \Phi^n }_{\ltwo{\ddom}{\lambda_{jk}}^3} &\leq \norm{ \Phi^n }_{\ltwo{\ddom}{\ddmes}^3} \rightarrow 0, \\
		\norm{ \grad^{(0)}\wedge\Phi^n - c }_{\ltwo{\ddom}{\lambda_{jk}}^3} & \leq \norm{ \grad^{(0)}\wedge\Phi^n - c }_{\ltwo{\ddom}{\ddmes}^3} \rightarrow 0,
	\end{align*}
	and so $c\in C$.
\end{proof}

Next, we require an extension lemma for curls of zero (compare this result with the role of lemma \ref{lem:3DExtensionLemmaGrads} for gradients of zero), and from this we will use closure of $\curlZero{\ddom}{\ddmes}$ to show the inclusion $\curlZero{\ddom}{\ddmes}\supset C$.
\begin{lemma}[Extension lemma for $\curlZero{\ddom}{\lambda_{jk}}$] \label{lem:CurlZeroExtensionLemma}
	Let $n\in\naturals$ and $I_{jk}^n$ be as in \eqref{eq:ShortenedEdgeDef}.
	Suppose that $c\in\curlZero{\ddom}{\lambda_{jk}}$ with $c=0$ on $I_{jk}\setminus I_{jk}^n$.
	Then
	\begin{align*}
		c\in\curlZero{\ddom}{\ddmes}.
	\end{align*}
	Furthermore, define $\tilde{c}$ via
	\begin{align*}
		\tilde{c} &= \begin{cases} c & x\not\in\vertSet, \\ 0 & x\in\vertSet. \end{cases}
	\end{align*}
	Then we also have that $\tilde{c}\in\curlZero{\ddom}{\dddmes}$.
\end{lemma}
\begin{proof}
	Let $\Phi^l$ be an approximating sequence for $c$ as in \eqref{eq:CurlZeroSequenceDef}, and $\chi_{jk}^n$ be the smooth function defined in \eqref{eq:SmoothChiDef}.
	Recall that we can choose $\chi_{jk}^n$ such that $\sup\abs{\grad\chi_{jk}^n}\leq kn$ for some constant $k$.
	Consider the sequence $\Psi^l := \chi_{jk}^n \Phi^l$, by construction we have that
	\begin{align*}
		\integral{\ddom}{ \abs{\Psi^l}^2 }{\ddmes} \leq \integral{I_{jk}}{ \abs{\Phi^l}^2 }{\lambda_{jk}} \rightarrow 0 \toInfty{l}.
	\end{align*}
	Since $\chi_{jk}^n$ and $\Phi^l$ are smooth, we can use the identity
	\begin{align*}
		\curl{}\Psi^l &= \grad\chi_{jk}^n\wedge\Phi^l + \chi_{jk}^n\curl{}\Phi^l,
	\end{align*}
	to deduce that
	\begin{align*}
		\integral{\ddom}{ \abs{ \curl{}\Psi^l - c }^2 }{\ddmes}
		&\leq 2\integral{I_{jk}}{ \abs{ \chi_{jk}^n\curl{}\Phi^l - c }^2 }{\lambda_{jk}}
		+ 2\sup\abs{\grad\chi_{jk}^n}^2 \integral{I_{jk}}{ \abs{ \Phi^l }^2 }{\lambda_{jk}} \\
		&\leq2\integral{I_{jk}}{ \abs{ \curl{}\Phi^l - c }^2 }{\lambda_{jk}}
		+ 2(kn)^2 \integral{I_{jk}}{ \abs{ \Phi^l }^2 }{\lambda_{jk}} \\
		&\rightarrow 0 \toInfty{l}.
	\end{align*}
	The sequence $\Psi^l$ now serves as the required approximating sequence, giving us that $c\in\curlZero{\ddom}{\ddmes}$.
	Furthermore, we also notice that $\Psi^l\bracs{v_j}=0$ and $\curl{}\Psi^l\bracs{v_j}=0$ for every $v_j\in\vertSet$, so we additionally have that
	\begin{align*}
		\Psi^l \lconv{\ltwo{\ddom}{\dddmes}^3} 0, 
		\qquad
		\curl{}\Psi^l \lconv{\ltwo{\ddom}{\dddmes}^3} \tilde{c},
	\end{align*}
	which demonstrates that $\tilde{c}\in\curlZero{\ddom}{\dddmes}$.
\end{proof}

\begin{lemma} \label{lem:CInCurlZero}
	\begin{align*}
		C \subset \curlZero{\ddom}{\ddmes}.
	\end{align*}
	Furthermore, the function
	\begin{align*}
		\tilde{c} &= \begin{cases} c & x\not\in\vertSet, \\ 0 & x\in\vertSet, \end{cases}
	\end{align*}
	is an element of $\curlZero{\ddom}{\dddmes}$.
\end{lemma}
\begin{proof}
	Let $\eta_j^n$ be as in \eqref{eq:SmoothEtaDef}, take $c\in C$, and define two families of functions $c_n$, $\tilde{c}_n$ by
	\begin{align*}
		c_n = \sum_{v_j\in\vertSet}\sum_{j\conLeft k} \eta_j^n \eta_k^n c_{jk},
		\qquad
		\tilde{c}_n = \begin{cases} c_n & x\not\in\vertSet, \\ 0 & x\in\vertSet. \end{cases}
	\end{align*}
	For each $n\in\naturals$ and $j\conLeft k$, the function $\eta_j^n \eta_k^n c_{jk}$ satisfies the hypothesis of the extension lemma \ref{lem:CurlZeroExtensionLemma}, and so is an element of $\curlZero{\ddom}{\ddmes}$, and its extension by zero the the vertices of $\graph$ is an element of $\curlZero{\ddom}{\dddmes}$.
	As both $\curlZero{\ddom}{\ddmes}$ and $\curlZero{\ddom}{\dddmes}$ are linear subspaces, $c_n\in\curlZero{\ddom}{\ddmes}$ and $\tilde{c}_n\in\curlZero{\ddom}{\dddmes}$ for every $n\in\naturals$.
	Since $\eta_j^n\rightarrow 1$ in $\ltwo{\ddom}{\ddmes}$, we conclude that $c_n\rightarrow c$ in $\ltwo{\ddom}{\ddmes}^3$, and since $\curlZero{\ddom}{\ddmes}$ is closed, we have that $c\in\curlZero{\ddom}{\ddmes}$.
	Similarly we can conclude that $\tilde{c}$ is the limit of $\tilde{c}_n$, and thus is an element of $\curlZero{\ddom}{\dddmes}$.
\end{proof}
Lemmas \ref{lem:CurlZeroInC}, \ref{lem:CurlZeroExtensionLemma}, and \ref{lem:CInCurlZero} then form the proof of theorem \ref{thm:CurlZeroChar}.

Our next brief stop will be an examination of the set $\curlZero{\ddom}{\massMes}$, before we then look to fully characterise the set $\curlZero{\ddom}{\dddmes}$.
Much of the same intuition that we had for gradients also holds for curls in the context of the measure $\massMes$ --- given that $\massMes$ only respects a finite number of points in $\ddom$, there is no way to measure rotations at these points, and so all curls will be curls of zero.
\begin{prop} \label{prop:VertexCurlZero}
	\begin{align*}
		\curlZero{\ddom}{\massMes} &= \pltwo{\ddom}{\massMes}^3.
	\end{align*}
\end{prop}
\begin{proof}
	For each $j\in\vertSet$ and $k\in\clbracs{1,2,3}$, let $c^k_1\in\pltwo{\ddom}{\massMes}^3$ be the function
	\begin{align*}
		c^j_k\bracs{x} &= \begin{cases} e_k & x=v_j, \\ 0 & x\neq v_j, \end{cases}
	\end{align*}
	where $e_k$ is the $k$\textsuperscript{th} canonical unit vector in $\complex^3$.
	Note that the collection 
	\begin{align*}
		\mathcal{C} = \clbracs{c^j_k \ \setVert v_j\in\vertSet, k\in\clbracs{1,2,3}}
	\end{align*}
	is a basis for $\pltwo{\ddom}{\massMes}^3$, so it is sufficient to show that $\mathcal{C}\subset\curlZero{\ddom}{\massMes}$, since $\curlZero{\ddom}{\massMes}$ is a closed linear subspace of $\ltwo{\ddom}{\massMes}^3$.
	To this end, let $d=\min_\edgeSet\abs{I_{jk}}$ and $\psi:\reals\rightarrow\reals$ be a smooth function with 
	\begin{align*}
		\psi(0) = 0, \quad \psi'(0) = 1, \quad \supp\bracs{\psi} \subset \bracs{-d,d}.
	\end{align*}
	Then for each $v_j\in\vertSet$, the function $\Phi_j\in\csmooth{\ddom}^3$ via 
	\begin{itemize}
		\item $\Phi_j(x) = \bracs{0, 0, \psi\bracs{x_2 - v_j^{(2)}}}^\top$ is such that $\Phi_j = 0, \ \grad^{(0)}\wedge\Phi_j = c^j_1$ in $\ltwo{\ddom}{\massMes}^3$,
		\item $\Phi_j(x) = \bracs{0, 0, \psi\bracs{v_j^{(1)} - x_1}}^\top$ is such that $\Phi_j = 0, \ \grad^{(0)}\wedge\Phi_j = c^j_2$ in $\ltwo{\ddom}{\massMes}^3$,
		\item $\Phi_j(x) = \bracs{0, \psi\bracs{x_1 - v_j^{(1)}}, 0}^\top$ is such that $\Phi_j = 0, \ \grad^{(0)}\wedge\Phi_j = c^j_3$ in $\ltwo{\ddom}{\massMes}^3$,
	\end{itemize}
	and thus, after multiplying by a smooth cut-off function equal to unity on some open neighbourhood of $v_j$, we see that $\mathcal{C}\subset\curlZero{\ddom}{\massMes}$ and the result follows.
\end{proof}

Although we usually conduct a study of tangential curls after completing our study of the curls of zero (akin to our methodology for studying gradients of zero and tangential gradients), there is little point in delaying the obvious characterisation for $\ktcurl{\ddom}{\massMes}$.
\begin{cory} \label{cory:VertexCurlSob}
	\begin{align*}
		\ktcurlSob{\ddom}{\massMes} &= \clbracs{ \bracs{u,0} \setVert u\in\pltwo{\ddom}{\massMes}^3 }.
	\end{align*}
\end{cory}
\begin{proof}
	($\subset$) If $u\in\ktcurlSob{\ddom}{\massMes}$ with $\ktcurl{\massMes}u=\bracs{v_1,v_2,v_3}^\top$ then the requirement that $\ktcurl{\massMes}u \perp \curlZero{\ddom}{\massMes}$ and proposition \ref{prop:VertexCurlZero} imply that $v_1=v_2=v_3=0$.
	
	($\supset$) Conversely if $u\in\ltwo{\ddom}{\massMes}^3$, for each $v_j\in\vertSet$ let $\psi_j$ denote the smooth ``bump" function centred on $v_j$ with
	\begin{align*}
		\psi_j\bracs{v_j} = 1, \quad
		\psi_j'\bracs{v_j} = 0, \quad
		\supp\bracs{\psi_j} \subset B_d\bracs{v_j},
	\end{align*}
	where $d=\min_{\edgeSet}\abs{I_{jk}}$.
	Setting $\Phi(x) = \sum_{v_j\in\vertSet}u\bracs{v_j}\psi_j\bracs{x}$, we have that $\Phi=u$ in $\ltwo{\ddom}{\massMes}^3$ so $\bracs{u,\ktcurl{}\Phi}\in W^{\kt}_{\mathrm{curl}}$.
	Therefore, there exists some $c\in\curlZero{\ddom}{\massMes}^{\perp}$ such that $\bracs{u,c}\in\ktcurlSob{\ddom}{\massMes}$, however from the ``$\subset$" inclusion we can then conclude that $c = 0$.
\end{proof}

We can now combine our understanding of $\curlZero{\ddom}{\ddmes}$ and $\curlZero{\ddom}{\massMes}$ to show that linear combinations of these functions make up the set $\curlZero{\ddom}{\dddmes}$.
\begin{prop} \label{prop:ThickVertexCurlZeroCharacterisation}
	Let $\tilde{c}\in\ltwo{\ddom}{\dddmes}^3$ where
	\begin{align*}
		\tilde{c} &= \begin{cases} c_{\ddmes} & x\not\in\vertSet, \\ c_{\massMes} & x\in\vertSet, \end{cases}
	\end{align*}
	for $c_{\ddmes}\in\pltwo{\ddom}{\ddmes}^3$ and $c_{\massMes}\in\pltwo{\ddom}{\massMes}^3$.
	Then we have that
	\begin{align*}
		c\in\curlZero{\ddom}{\dddmes} \quad\Leftrightarrow\quad 
		& c_{\ddmes}\in\curlZero{\ddom}{\ddmes} \text{ and } c_{\massMes}\in\curlZero{\ddom}{\massMes}.
	\end{align*}
\end{prop}
\begin{proof}
	($\Rightarrow$) For the right-directed implication it is sufficient to notice that 
	\begin{align*}
		\norm{\cdot}_{\ltwo{\ddom}{\dddmes}}^2 &= \norm{\cdot}_{\ltwo{\ddom}{\ddmes}}^2 + \norm{\cdot}_{\ltwo{\ddom}{\massMes}}^2,
	\end{align*}
	so any approximating sequence for $c$ that converges in $\ltwo{\ddom}{\dddmes}^3$ also converges in $\ltwo{\ddom}{\ddmes}^3$ to $c_{\ddmes}$ and in $\ltwo{\ddom}{\massMes}^3$ to $c_{\massMes}$.
	
	($\Leftarrow$) For the left-directed implication, it is sufficient for us to demonstrate the implication holds for the case when $c_{\massMes}=0$, and the case that $c_{\ddmes}=0$ with $c_{\massMes}\neq0$ at precisely one vertex $v$.
	Having shown the implication in these cases, linearity of $\curlZero{\ddom}{\dddmes}$ will then complete the proof.
	
	The case when $c_{\massMes}=0$ is handled by the extension lemma \ref{lem:CurlZeroExtensionLemma} and lemma \ref{lem:CInCurlZero}.
	
	Next, consider the case when $c_{\ddmes}=0$, and when $c_{\massMes}=0$ at all vertices except $v\in\vertSet$, with $v=\bracs{v_1, v_2}\in\ddom$ and with $c_{\massMes}(v) = \bracs{c_1, c_2, c_3}^\top$.
	For each $n\in\naturals$ take a smooth function $\phi_n:\reals\rightarrow\sqbracs{-1,1}$ with the properties
	\begin{align*}
		\phi_n(0) = 0,
		&\quad	\phi'_n(0) = 1, \\
		\phi_n(t) = 0, &\quad t\not\in B_{\frac{2}{n}}(0), \\
		\abs{\phi_n(t)} \leq \recip{n} &\quad t\in B_{\frac{2}{n}}(0).
	\end{align*}
	Since $\abs{\phi_n(t)} \leq \recip{n}$ when $\abs{t}\leq\recip{n}$, $\phi_n$ can be chosen so that exists a constant $K$ independent of $n$ such that $\abs{\grad\phi_n} \leq K$ when $\recip{n} \leq \abs{x-v} \leq \frac{2}{n}$.
	Define the functions $\Phi^n\in\csmooth{\ddom}$ (and compute their curls) as follows;
	\begin{align*}
		\Phi^n(x) = 
		\begin{pmatrix} 
			0 \\ 
			c_3\phi_n\bracs{x_1 - v_1} \\ 
			c_1\phi_n\bracs{x_2-v_2} + c_2\phi_n\bracs{v_1-x_1} 
		\end{pmatrix},
		&\qquad
		\curl{}\Phi_n(x) =
		\begin{pmatrix}
			c_1\phi'_n\bracs{x_2-v_2} \\
			c_2\phi'_n\bracs{v_1-x_1} \\
			c_3\phi'_n\bracs{x_1-v_1}
		\end{pmatrix}.
	\end{align*}
	Then we have the following:
	\begin{align*}
		\integral{\ddom}{ \abs{ \Phi^n }^2 }{\dddmes}
		&\leq \bracs{c_3^2 + \bracs{c_1 + c_2}^2} \bracs{ \integral{B_{\frac{2}{n}}(v)}{ \recip{n^2} }{\ddmes}
		+ \alpha_v\abs{\phi_n(0)}^2 } \\
		&= \frac{2\mathrm{deg}(v)}{n^3}\bracs{\abs{c_3}^2 + \abs{c_1 + c_2}^2} \rightarrow 0 \toInfty{n}, \\
		\integral{\ddom}{ \abs{ \curl{}\Phi^n - c }^2 }{\dddmes}
		&= \integral{\ddom}{ \abs{ \curl{}\Phi^n}^2 }{\ddmes}
		+ \integral{\ddom}{ \abs{ \curl{}\Phi^n - c_{\massMes} }^2 }{\massMes} \\
		&= \abs{c_1}^2\integral{\ddom}{ \abs{\phi'_n\bracs{x_2-v_2}}^2 }{\ddmes}
		+ \abs{c_2}^2\integral{\ddom}{ \abs{\phi'_n\bracs{v_1-x_1}}^2 }{\ddmes} \\
		&\quad + \abs{c_3}^2\integral{\ddom}{ \abs{\phi'_n\bracs{x_1-v_1}}^2 }{\ddmes}
		+ \alpha_v\abs{c(v)}^2\abs{\phi'_n(0)-1}^2 \\
		&\leq \abs{c(v)}^2 \bracs{ K^2\integral{B_{\frac{2}{n}}(v)}{ }{\ddmes}
		+ \alpha_v\abs{\phi'_n(0)-1}^2 } \\
		&= \frac{2\mathrm{deg}(v)}{n}\abs{c(v)}^2 K^2 \rightarrow 0 \toInfty{n},
	\end{align*}
	where $\mathrm{\deg}(v)$ is the degree of the vertex $v$.
	We thus conclude that $c\in\curlZero{\ddom}{\dddmes}$, and given the linearity of $\curlZero{\ddom}{\dddmes}$, the proof is complete.
\end{proof}

\subsection{Tangential Curls} \label{ssec:TangCurls}
With theorem \ref{thm:CurlZeroChar} in hand, we are ready to explore the form of tangential curls with respect to the measures $\ddmes$ and $\dddmes$.
Note that we have already identified the curls of zero with respect to $\massMes$, through corollary \ref{cory:VertexCurlSob}.
We begin, as has become standard, with an edge parallel to one of the coordinate axes.
\begin{prop} \label{prop:TangCurlEdgeParallel}
	Let $I_{jk}$ be parallel to the $x_2$ axis, and write $\gradSob{\sqbracs{0,l_{jk}}}{y}$ for the classical Sobolev space of functions on $\sqbracs{0,l_{jk}}$ with respect to the Lebesgue measure.
	Suppose that $u\in\ktgradSob{\ddom}{\lambda_{jk}}$ and set $\tilde{u}=u\circ r_{jk}$ on $I_{jk}$.
	Then $\tilde{u}_3\in\gradSob{\sqbracs{0,l_{jk}}}{y}$ and 
	\begin{align*}
		\ktcurl{\lambda_{jk}}u &= \bracs{ u_3' + \rmi\qm_2 u_3 - \rmi\wavenumber u_2 }\widehat{x}_1,
	\end{align*}
	where $u_3'=\tilde{u}_3'\circ r_{jk}^{-1}$.
\end{prop}

The proof follows from the general argument presented in corollary \ref{cory:TangCurlEdgeRotated}, however proposition \ref{prop:TangCurlEdgeParallel} is useful for validating our expectations that the tangential curl is directed ``out of" the plane induced by $I_{jk}$, and only concerns ``rates of change" that the measure $\lambda_{jk}$ can see.
Indeed we can observe that we essentially have the operation $\partial_2+\rmi\qm_2$ acting on $u_3$; and $\rmi\wavenumber$ acting on $u_2$, which is essentially a rate of change along $x_3$ due to our previous taking of a  Fourier transform in $\widehat{x}_3$.
\begin{cory} \label{cory:TangCurlEdgeRotated}
	Let $I_{jk}\in\edgeSet$, and let $u\in\ktcurlSob{\ddom}{\lambda_{jk}}$.
	Define
	\begin{align*}
		U(x) = R_{jk}\begin{pmatrix} u_1(x) \\ u_2(x) \end{pmatrix},
		\qquad
		\tilde{u}_3 &= u_3\circ r_{jk}, 
		\qquad
		\qm_{jk} = \qm\cdot e_{jk} = \bracs{R_{jk}\qm}_2.
	\end{align*}
	Then $\tilde{u}_3\in\gradSob{\sqbracs{0,l_{jk}}}{y}$ and
	\begin{align*}
		\ktcurl{\lambda_{jk}}u &= \bracs{ u_3' + \rmi\qm_{jk}u_3 - \rmi\wavenumber U_2 }\widehat{n}_{jk},
	\end{align*}
	where $u_3' := \tilde{u}_3'\circ r_{jk}^{-1}$.
\end{cory}
It is worth remarking that $U_2 = \bracs{u_1,u_2}^{\top}e_{jk}$ on $I_{jk}$, so that we can see that $U_2$ retains precisely the component of $U$ in the direction along $I_{jk}$.
\begin{proof}
	Let us write $c = \ktcurl{\lambda_{jk}}u$.
	The requirement that $c\perp\curlZero{\ddom}{\lambda_{jk}}$, combined with theorem \ref{thm:CurlZeroChar}, implies that
	\begin{align*}
		c\cdot\widehat{e}_{jk} = 0 = c\cdot\widehat{x}_3,
	\end{align*}
	so we can write $c = c_{jk}\widehat{n}_{jk}$, for some $c_{jk}\in\pltwo{\ddom}{\lambda_{jk}}$ to be determined.
	To this end, let us now take an approximating sequence $\Phi^n$ for $u$, which provides us with the following convergences in $\ltwo{\ddom}{\lambda_{jk}}$:
	\begin{align*}
		\Phi^n_1 \rightarrow u_1, \quad 
		\Phi^n_2 &\rightarrow u_2, \quad
		\Phi^n_3 \rightarrow u_3, \\
		\bracs{\partial_2+\rmi\qm_2}\Phi^n_3 - \rmi\wavenumber\Phi^n_2 &\rightarrow c_{jk}n_1^{(jk)} = c_{jk}e_2^{(jk)}, \\
		\bracs{\partial_1+\rmi\qm_1}\Phi^n_3 - \rmi\wavenumber\Phi^n_1 &\rightarrow -c_{jk}n_2^{(jk)} = c_{jk}e_1^{(jk)}, \\
		\bracs{\partial_1 + \rmi\qm_1}\Phi^n_2 - \bracs{\partial_2+\rmi\qm_2}\Phi^n_1 &\rightarrow 0.
	\end{align*}
	In particular we notice that
	\begin{align*}
		\grad\Phi^n_3\cdot e_{jk} \rightarrow c_{jk} + \rmi\wavenumber U\cdot e_{jk} - \rmi u_3\qm\cdot e_{jk},
	\end{align*}
	and thus the sequence $\varphi_n := \Phi^n_3\circ r_{jk}$ is such that
	\begin{align*}
		\varphi_n\rightarrow\tilde{u}_3, \quad
		\varphi_n'\rightarrow c_{jk}\circ r_{jk} + \rmi\wavenumber\bracs{U\circ r_{jk}}\cdot e_{jk} - \rmi\tilde{u}_3\qm\cdot e_{jk}, \quad
		\text{in } \ltwo{\sqbracs{0,l_{jk}}}{y}.
	\end{align*}
	As such, we have that $\tilde{u}_3\in\gradSob{\sqbracs{0,l_{jk}}}{y}$, and upon rearrangement and undoing the change of variables $r_{jk}$, we obtain
	\begin{align*}
		c_{jk}(x) &= u_3'(x) + \rmi\qm_{jk}u_3(x) - \rmi\wavenumber U(x)\cdot e_{jk}
		= u_3'(x) + \rmi\qm_{jk}u_3(x) - \rmi\wavenumber U_2(x),
	\end{align*}
	providing the result.
\end{proof}

Our goal now is to characterise tangential curls belonging to functions in the spaces $\ktcurlSob{\ddom}{\ddmes}$ and $\ktcurlSob{\ddom}{\dddmes}$ using our edge-wise understanding from corollary \ref{cory:TangCurlEdgeRotated} and vertex-characterisation from corollary \ref{cory:VertexCurlSob}.
A simple comparison of norms shows us that the edge-wise behaviour must be inherited:
\begin{prop} \label{prop:TC-dddmesImpliesOthers}
	Suppose that $\bracs{u,\ktcurl{\dddmes}u}\in\ktcurlSob{\ddom}{\dddmes}$.
	Then we have that $\bracs{u,\ktcurl{\dddmes}u}\in\ktcurlSob{\ddom}{\ddmes}$ and $\bracs{u,\ktcurl{\dddmes}u}\in\ktcurlSob{\ddom}{\massMes}$ too.
\end{prop}
\begin{proof}
	Simply take an approximating sequence $\Phi^n$ for $u$, and notice that
	\begin{align*}
		\norm{\cdot}_{\ltwo{\ddom}{\dddmes}^3} &= \norm{\cdot}_{\ltwo{\ddom}{\ddmes}^3} + \norm{\cdot}_{\ltwo{\ddom}{\massMes}^3}.
	\end{align*}
	This implies that $\Phi^n$ and $\ktcurl{}\Phi^n$ also converge in these norms to (functions almost everywhere equal to) $u$ and $\ktcurl{\dddmes}u$ respectively.
	We are also assured that $\ktcurl{\dddmes}u$ is orthogonal to $\curlZero{\ddom}{\ddmes}$ and $\curlZero{\ddom}{\massMes}$ by theorem \ref{thm:CurlZeroChar}, which completes the proof.
\end{proof}

The proof of corollary \ref{cory:TangCurlEdgeRotated} hints at a connection between the third component $u_3$ of a function $u\in\ktcurlSob{\ddom}{\ddmes}$ and the space $\ktgradSob{\ddom}{\ddmes}$.
Further to this, we know from theorem \ref{thm:CharOfSobSpaces} that functions in this space are precisely those that can be constructed from edge functions, provided these edge functions are continuous at the vertices.
\begin{prop} \label{prop:TC-3rdComponentIFF}
	Let $u_3^{(jk)}\in\pltwo{\ddom}{\lambda_{jk}}$, $u_3\in\pltwo{\ddom}{\ddmes}$, and $\tilde{u}_3\in\pltwo{\ddom}{\dddmes}$.
	Then the following equivalences hold.
	\begin{align*}
		\mathrm{(i)} \ &
		u^{(jk)}:= \bracs{0,0,u_3^{(jk)}}^\top\in\ktcurlSob{\ddom}{\lambda_{jk}}
		\quad &\Leftrightarrow &\quad
		u_3^{(jk)}\in\ktgradSob{\ddom}{\lambda_{jk}}, \\
		\mathrm{(ii)} \ &
		u:= \bracs{0,0,u_3}^\top\in\ktcurlSob{\ddom}{\ddmes}
		\quad &\Leftrightarrow &\quad
		u_3\in\ktgradSob{\ddom}{\ddmes}, \\
		\mathrm{(iii)} \ &
		\tilde{u} := \bracs{0,0,\tilde{u}_3}^\top\in\ktcurlSob{\ddom}{\dddmes}
		\quad &\Leftrightarrow &\quad
		\tilde{u}_3\in\ktgradSob{\ddom}{\dddmes}.
	\end{align*}
\end{prop}
\begin{proof}
	We will explicitly prove the statement (ii), since the other statements (i) and (iii) follow essentially the same argument --- taking an approximating sequence, observing the convergences in $L^2$ that we obtain, and appealing the the characterisation of gradients or curls of zero depending on which implication is being shown.
	
	$(\Leftarrow)$ For the left-directed implication, let $\phi_n$ be an approximating sequence for $u_3$ and write $g = \ktgrad_{\ddmes}u$.
	Define the sequence $\Phi^n = \bracs{0,0,\phi_n}^\top\in\psmooth{\ddom}^3$, and notice that
	\begin{align*}
		\Phi^n\lconv{\ltwo{\ddom}{\ddmes}^3}u, \qquad
		\ktcurl{}\Phi^n = 
		\begin{pmatrix} 
			\bracs{\partial_2+\rmi\qm_2}\phi_n \\ -\bracs{\partial_1 + \rmi\qm_1}\phi_n \\ 0 
		\end{pmatrix}
		\lconv{\ltwo{\ddom}{\ddmes}^3}
		\begin{pmatrix}
			g_2 \\ -g_1 \\ 0
		\end{pmatrix}
		=: c.
	\end{align*}
	Note that, since $g\perp\gradZero{\ddom}{\ddmes}$, we have $g\perp\widehat{e}_{jk}$ on every edge $I_{jk}$, implying that $c\perp\widehat{n}_{jk}$ on each $I_{jk}$ and thus $c\perp\curlZero{\ddom}{\ddmes}$ by theorem \ref{thm:CurlZeroChar}.
	Therefore, $\Phi^n$ serves as an approximating sequence for $\bracs{u,c}\in\ktcurlSob{\ddom}{\ddmes}$.
	
	$(\Rightarrow)$ For the right-directed implication, let $\Phi^n$ be an approximating sequence for $u$ and write $c=\ktcurl{}u$.
	Define the sequence $\phi_n = \Phi^n_3$; similarly to the above, it can be shown that
	\begin{align*}
		\phi_n \lconv{\ltwo{\ddom}{\ddmes}} u_3, \qquad
		\ktgrad\phi_n \lconv{\ltwo{\ddom}{\ddmes}^3} \begin{pmatrix} -c_2 \\ c_1 \\ \rmi\wavenumber u_3 \end{pmatrix}
		=: g.
	\end{align*}
	Again by recognising $c\perp\curlZero{\ddom}{\ddmes}$ implies that $g\perp\gradZero{\ddom}{\ddmes}$, we can conclude that $\phi_n$ serves as an approximating sequence for $\bracs{u,g}\in\ktgradSob{\ddom}{\ddmes}$.
\end{proof}

Proposition \ref{prop:TC-3rdComponentIFF} also allows us to invoke theorem \ref{thm:CharOfSobSpaces} to conclude that the component $u_3$ is continuous at the vertices of $\graph$.
This just leaves questions concerning the remaining two components $u_1$ and $u_2$; we have already seen from proposition \ref{prop:TC-dddmesImpliesOthers} that a function on the whole graph breaks down edge-wise.
However we are lacking continuity of $u_1$ and $u_2$ at the vertices, (and even weak differentiability along the edges), so cannot sum edge-functions to obtain a function on the whole graph as we could with scalar functions possessing gradients.
The best we can do (utilising only knowledge of curls) comes when we have a vector field whose first and second components have support contained in the interior of an edge, analogous to lemma \ref{lem:ExtensionLemmaEdgeFunctions}.
\begin{prop}
	Suppose that $=\bracs{u_1,u_2,0}^\top\in\ktcurlSob{\ddom}{\lambda_{jk}}$ with $u=0$ outside $I_{jk}^\eps$ for some $\eps>0$.
	Then $u\in\ktcurlSob{\ddom}{\ddmes}$ and $u\in\ktcurlSob{\ddom}{\dddmes}$ with
	\begin{align*}
		\ktcurl{\ddmes}u =
		\begin{cases} 
			\ktcurl{\lambda_{jk}}u & x\in I_{jk}, \\
			0 & x\not\in I_{jk}, 
		\end{cases}
		\qquad
		\ktcurl{\dddmes}u = 
		\begin{cases} 
			\ktcurl{\lambda_{jk}}u & x\in I_{jk}, \\
			0 & x\not\in I_{jk}, \\
			0 & x\in\vertSet.
		\end{cases}
	\end{align*}
\end{prop}
\begin{proof}
	Take an approximating sequence $\Phi^n$ for $u$, and consider the smooth cut-off function $\chi_{jk}^{\eps}$ in \eqref{eq:SmoothChiDef}.
	Construct the sequence $\Psi^n = \chi_{jk}^{\eps}\Phi^n$, and note that for each $n\in\naturals$, 
	\begin{align*}
		\Psi^n\bracs{v_j}=0, \qquad
		\ktcurl{}\Psi_n = \chi_{jk}^{\eps}\ktcurl{}\Phi^n + \grad\chi_{jk}^{\eps}\wedge\Phi^n, \qquad
		\ktcurl{}\Psi_n\bracs{v_j} = 0.
	\end{align*}
	We then observe that
	\begin{align*}
		\integral{\ddom}{ \abs{\Psi^n-u}^2 }{\ddmes}
		&= \integral{\ddom}{ \abs{\chi_{jk}^\eps\Phi^n-u}^2 }{\lambda_{jk}} \\
		&\leq \integral{\ddom}{ \abs{\Phi^n-u}^2 }{\lambda_{jk}} \rightarrow 0 \toInfty{n}.
	\end{align*}
	Furthermore, since $\grad\chi_{jk}^\eps$ is only non-zero outside the support of $u$, we have that
	\begin{align*}
		\recip{2}\integral{\ddom}{ & \abs{\ktcurl{}\Psi^n - \ktcurl{\ddmes}u}^2 }{\ddmes} \\
		&\leq \integral{\ddom}{ \abs{\chi_{jk}^{\eps}\ktcurl{}\Phi^n - \ktcurl{\ddmes}u}^2 }{\ddmes}
		+ \integral{\ddom}{ \abs{\grad\chi_{jk}^{\eps}\wedge\Phi^n}^2 }{\lambda_{jk}} \\
		&\leq \integral{\ddom}{ \abs{\ktcurl{}\Phi^n - \ktcurl{\ddmes}u}^2 }{\lambda_{jk}} %\\
%		&\qquad 
		+ \integral{\ddom}{ \abs{\grad\chi_{jk}^\eps}^2\abs{\Phi^n - u}^2 }{\lambda_{jk}} \\
		&\leq \norm{\ktcurl{}\Phi^n - \ktcurl{\ddmes}u}_{\ltwo{\ddom}{\lambda_{jk}}^3} \\
		&\qquad + \sup\abs{\grad\chi_{jk}^\eps}^2\norm{\Phi^n - u}_{\ltwo{\ddom}{\lambda_{jk}}^3} \\
		&\rightarrow0 \toInfty{n}.
	\end{align*}
	This gives us the required convergences for $u\in\ktcurlSob{\ddom}{\ddmes}$, we also observe that $u\in\ktcurlSob{\ddom}{\dddmes}$ through use of the same approximating sequence as
	\begin{align*}
		\norm{\Psi^n-u}_{\ltwo{\ddom}{\massMes}^3} = 0 = \norm{\ktcurl{}\Psi^n}_{\ltwo{\ddom}{\massMes}^3}
	\end{align*}
	for every $n\in\naturals$.
\end{proof}

The issue preventing us from establishing the converse implication to that in proposition \ref{prop:TC-dddmesImpliesOthers} is the aforementioned lack of control over the components $u_1$ and $u_2$ near the vertices of a graph when we attempt to ``lift" edge-functions.
In the case of tangential gradients, the we had to depend on continuity of the incoming edge functions at the vertex to circumnavigate this issue, and we also depend on this fact when handling the component $u_3$ of vector fields with tangential curls.
This continuity does not hold for the incoming components $u_1$ and $u_2$ since knowing the field $u$ possesses a tangential curl does not establish whether these components have gradients of their own, or even directional derivatives along the edges (which would also suffice).
Consideration of what it means for a field to be divergence-free with respect to $\dddmes$ (section \ref{sec:DivFreeCondition}) goes some of the way towards to remedying this.