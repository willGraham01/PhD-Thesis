\subsection{Geometric Interpretation of Tangential Curls} \label{ssec:CC-Geometric}
Unlike the gradient, the curl of a vector field does not have an intuitive one-dimensional analogue that we can appeal to when looking to work on singular structure domains.
This makes the analysis of \tstk{appendix section} crucial to our understanding of the curl of the curl equation \tstk{ref!}.
In the interests of providing the reader an intuitive idea of what the tangential curl is for a singular measure, we provide a geometric interpretation in this section prior to beginning the process of deriving a quantum graph problem from our variational formulation.
We will also address a remark made in section \tstk{our systems section}, concerning why we do not consider a singular structure modelled by a (periodic) graph embedded into $\reals^3$, and instead consider a domain formed from the extrusion into 3 dimensions of a periodic graph embedded in $\reals^2$.

\subsubsection{Curls on Graphs Embedded into $\reals^3$} \label{sssec:3DGraphCurlsZero}
Let us begin by demonstrating why curls of zero on a singular structure embedded into $\reals^3$ do not give rise to any interesting variational problems.
Consider a segment $I_{jk}\subset\reals^3$, with $I_{jk} = \clbracs{0}\times\sqbracs{0,l_{jk}}\times\clbracs{0}$, which we can view as one edge of a graph embedded into $\reals^3$.
We can then prove the following result:
\begin{prop} \label{prop:3DGraph-CurlsAreZero}
	Every vector field is a curl of zero, that is
	\begin{align*}
		\curlZero{\reals^3}{\lambda_{jk}} = \ltwo{\reals^3}{\lambda_{jk}}^3,
	\end{align*}
	and consequentially every vector field $u\in\ltwo{\reals^3}{\lambda_{jk}}^3$ is also an element of $\ktcurlSob{\reals^3}{\lambda_{jk}}$ with tangential curl equal to zero.
\end{prop}
The assumption on the position and orientation of $I_{jk}$ is not restrictive, in the sense that the result we prove will generalise to any line segment (and thus edge) in $\reals^3$.
This result also holds when we replace $\reals^3$ with a finite domain $U\subset\reals^3$ (with non-zero $\lambda_3$-measure) and consider $I_{jk}\subset U$, and consequentially if we consider $U$ the period cell of some 3-dimensional singular structure $\graph$ and consider curls on the torus.
\begin{proof}
	We construct explicit approximating sequences for any function in $\ltwo{\reals^3}{\lambda_{jk}}^3$, by first considering $\phi\in\smooth{\reals^3}$.
	Consider the vector field $\varphi^{(2)} = -x_1\phi\hat{x}_3\in\smooth{\reals^3}^3$ --- note that we can always multiply $\varphi^{(2)}$ by a smooth cut-off function which is equal to unity on $I_{jk}$ to obtain a compactly supported smooth function if we are so inclined.
	Then we can observe that $\varphi^{(2)}=0$ on $I_{jk}$ (since $x_1=0$ here) and 
	\begin{align*}
		\curl{}\varphi^{(2)} = -x_1\partial_2\phi\hat{x}_1 + \bracs{\phi + x_1\partial_1\phi}\hat{x}_2,
	\end{align*}
	which is equal to $\phi\hat{x}_2$ on $I_{jk}$.
	This implies that $\varphi^{(2)}$ serves as an approximating ``sequence" with
	\begin{align*}
		\varphi^{(2)} \lconv{\ltwo{\reals^3}{\lambda_{jk}}^3} 0, 
		\qquad
		\curl{}\varphi^{(2)} \lconv{\ltwo{\reals^3}{\lambda_{jk}}^3} \phi\hat{x}_2,
	\end{align*}
	so $\phi\hat{x}_2\in\curlZero{\reals^3}{\lambda_{jk}}$.
	Consideration of the functions $\varphi^{(1)} = -x_3\phi\hat{x}_2$ and $\varphi^{(3)}=x_1\phi\hat{x}_2$ demonstrates that $\phi\hat{x}_1$ and $\phi\hat{x}_3$ are also elements of $\curlZero{\reals^3}{\lambda_{jk}}$.
	By linearity, we have that $\smooth{\reals^3}^3\subset\curlZero{\reals^3}{\lambda_{jk}}$ and then by the density of smooth functions in $\ltwo{\reals^3}{\lambda_{jk}}$, we obtain
	\begin{align*}
		\curlZero{\reals^3}{\lambda_{jk}} = \ltwo{\reals^3}{\lambda_{jk}}^3.
	\end{align*}
	We can also infer that any smooth vector field $\Phi\in\smooth{\reals^3}{\lambda_{jk}}^3$ is an element of $\ktcurlSob{\reals^3}{\lambda_{jk}}$ with $\ktcurl{\lambda_{jk}}\Phi = 0$ since $\ltwo{\reals^3}{\lambda_{jk}}^{\perp} = \clbracs{0}$.
	Density of smooth functions in $\ltwo{\reals^3}{\lambda_{jk}}^3$ then demonstrates that any $u\in\ltwo{\reals^3}{\lambda_{jk}}^3$ is also an element of $\ktcurlSob{\reals^3}{\lambda_{jk}}$ with tangential curl equal to zero.
\end{proof}

That is to say, a singular structure in 3 dimensions represented by a graph does not give rise to any non-trivial notion of the curl of a vector field.
Needless to say, this makes exploration of the curl of the curl equation trivial in this context --- it is clear from section \tstk{curl appendix} that tangential curls with respect to the measure $\ddmes$ will inherit the behaviour of tangential curls with respect to each $I_{jk}$ on the respective edges, and thus only produce trivial variational problems.

\subsubsection{Tangential Curls on the domain $\dddom$} \label{sssec:ktCurlsGeometric}
Rather than examining a graph embedded into $\reals^3$, let us instead return to the domain $\dddom$.
Section \tstk{appendix} contains the arguments (and precise statements) that we aim to provide a review of here.
First, recall that $\dddom$ consists of a periodic graph embedded into $\reals^2$ with unit cell $\ddom$, and then extruded into $\dddom = \ddom\times[0,\infty)$ to form a domain that consists of a union of planes parallel to the $x_3$-axis.
Our use of a Fourier transform in $x_3$ then brings us back onto the 2 dimensional domain $\ddom$, however it is important for us to remember that each edge of $\graph$ in $\ddom$ actually represents a \emph{plane} in $\reals^3$.
In what follows, we will let $P_{jk}$ be the plane induced by the edge $I_{jk}$ of the graph $\graph$, so $P_{jk} = I_{jk}\times[0,\infty)\subset\ddom\times[0,\infty)$.

Classically, one can interpret the curl $c$ of a vector field $u$ by identifying $c(x)$ as the axis of rotation that an (infinitesimally small) spherical body would undergo if placed in the field $u$ at position $x$, with the angular speed of the rotation equal to half the magnitude of $c(x)$.
Our intuition and geometric interpretation for tangential gradients (section \ref{ssec:3DGradGeometric}) was largely based on the idea that the singular measure $\lambda_{jk}$ cannot ``see" changes in functions \emph{across} the edges $I_{jk}$.
The interpretation we have for tangential curls (and curls of zero) also appeals to this idea; the measure $\lambda_{jk}$ does not have any concept of normal derivative across $I_{jk}$, and correspondingly the product measure $\lambda_{jk}\times\lambda_1$ does not have any concept of a derivative \emph{outward} from the plane $P_{jk}$.
However, when we are restricted to only observing the values of $u$ in the plane $P_{jk}$, any changes in $u$ in the direction of the outward normal $\hat{n}_{jk}$ to $P_{jk}$ cannot be seen, or observed.
Such changes in $u$ only affect the components of $c(x)$ in the directions orthogonal to $\hat{n}_{jk}$, but if we do not know how we are rotating in one direction, we do not know how to orient \emph{the axis} about which we rotate.
In this case, the directions $\hat{e}_{jk}$ and $\hat{x}_3$ are those which are orthogonal to $\hat{n}_{jk}$, landing us with the result \tstk{curls of zero}.
On the other hand, the only component of $c(x)$ that isn't derived from changes in $u$ in the direction $\hat{n}_{jk}$ is the component along the direction $\hat{n}_{jk}$ itself, which is the only component in tangential curls that is non-zero. \tstk{result reference}

This interpretation carries over into our previous consideration of singular structures (and conclusion in proposition \ref{prop:3DGraph-CurlsAreZero}) in three dimensions too ---  $\lambda_{jk}$ still has no concept of derivative across the edge $I_{jk}$, and in three dimensions this means there are two linearly independent directions in which rates of change cannot be seen.
Consequentially, we can never accurately reconstruct the axis about which we are rotating, as all three components of the curl would need to know the rate of change in $u$ in at least one of the directions normal to $I_{jk}$.

One can identify tangential curls in the following manner; place a small sphere of radius $r\ll 1$\footnote{Or alternatively, think of $r$ as being infinitesimally small.} centred at a point $x\in P_{jk}$.
Draw a closed line $\gamma$ around the ``equator" of the sphere, that is along the surface of the sphere which intersects $P_{jk}$, and then allow the sphere to begin rotating under the effects of the vector field $u$.
Let us suppose that the sphere has an angular frequency $f$, so completes one full ``rotation" under $u$ in $f^{-1}$ units of time.
Under the axis of rotation provided by a tangential curl, $\gamma$ will remain in the plane $P_{jk}$ at all times, enclosing the same circular area in $P_{jk}$.
Under the axis of rotation of a curl of zero, $\gamma$ will not remain in the plane $P_{jk}$ --- in fact, there will only be at most two points of $\gamma$ in $P_{jk}$ at times $t\neq \recip{2}kf^{-1}$, for $k\in\naturals$.
That is to say, curls of zero induce rotations ``out of the plane" $P_{jk}$, which the measure $\lambda_{jk}$ cannot observe, whilst tangential curls ensure that all rotation happens ``within" the plane $P_{jk}$.