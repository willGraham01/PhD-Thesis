\section{Geometric Interpretation of Tangential Curls} \label{sec:CC-Geometric}
Unlike the gradient, the curl of a vector field does not have an intuitive one-dimensional analogue that we can appeal to when looking to work on singular structure domains.
This makes the analysis of section \ref{sec:CC-CurlAnalysis} crucial to our understanding of the curl of the curl equation \eqref{eq:SingularCurlEquation}.
In the interests of providing the reader an intuitive idea of what the tangential curl is for a singular measure, we provide a geometric interpretation in this section prior to beginning the process of deriving a quantum graph problem from our variational formulation.
We will also address a remark made in section \ref{sec:TP-DomainSetup}, concerning why we do not consider a singular structure modelled by a (periodic) graph embedded into $\reals^3$, and instead consider a domain formed from the extrusion into 3 dimensions of a periodic graph embedded in $\reals^2$.

Let us begin by demonstrating why curls of zero on a singular structure embedded into $\reals^3$ do not give rise to any interesting variational problems.
Consider a segment $I_{jk}\subset\reals^3$, with $I_{jk} = \clbracs{0}\times\sqbracs{0,l_{jk}}\times\clbracs{0}$, which we can view as one edge of a graph embedded into $\reals^3$.
We can then prove the following result:
\begin{prop} \label{prop:3DGraph-CurlsAreZero}
	Every vector field is a curl of zero, that is
	\begin{align*}
		\curlZero{\reals^3}{\lambda_{jk}} = \ltwo{\reals^3}{\lambda_{jk}}^3,
	\end{align*}
	and consequentially every vector field $u\in\ltwo{\reals^3}{\lambda_{jk}}^3$ is also an element of $\ktcurlSob{\reals^3}{\lambda_{jk}}$ with tangential curl equal to zero.
\end{prop}
The assumption on the position and orientation of $I_{jk}$ is not restrictive, in the sense that the result we prove will generalise to any line segment (and thus edge) in $\reals^3$.
This result also holds when we replace $\reals^3$ with a finite domain $U\subset\reals^3$ (with non-zero $\lambda_3$-measure) and consider $I_{jk}\subset U$, and consequentially if we consider $U$ the period cell of some 3-dimensional singular structure $\graph$ and consider curls on the torus.
\begin{proof}
	We construct explicit approximating sequences for any function in $\ltwo{\reals^3}{\lambda_{jk}}^3$, by first considering $\phi\in\smooth{\reals^3}$.
	Consider the vector field $\varphi^{(2)} = -x_1\phi\widehat{x}_3\in\smooth{\reals^3}^3$ --- note that we can always multiply $\varphi^{(2)}$ by a smooth cut-off function which is equal to unity on $I_{jk}$ to obtain a compactly supported smooth function if we are so inclined.
	Then we can observe that $\varphi^{(2)}=0$ on $I_{jk}$ (since $x_1=0$ here) and 
	\begin{align*}
		\curl{}\varphi^{(2)} = -x_1\partial_2\phi\widehat{x}_1 + \bracs{\phi + x_1\partial_1\phi}\widehat{x}_2,
	\end{align*}
	which is equal to $\phi\widehat{x}_2$ on $I_{jk}$.
	This implies that $\varphi^{(2)}$ serves as an approximating ``sequence" with
	\begin{align*}
		\varphi^{(2)} \lconv{\ltwo{\reals^3}{\lambda_{jk}}^3} 0, 
		\qquad
		\curl{}\varphi^{(2)} \lconv{\ltwo{\reals^3}{\lambda_{jk}}^3} \phi\widehat{x}_2,
	\end{align*}
	so $\phi\widehat{x}_2\in\curlZero{\reals^3}{\lambda_{jk}}$.
	Consideration of the functions $\varphi^{(1)} = -x_3\phi\widehat{x}_2$ and $\varphi^{(3)}=x_1\phi\widehat{x}_2$ demonstrates that $\phi\widehat{x}_1$ and $\phi\widehat{x}_3$ are also elements of $\curlZero{\reals^3}{\lambda_{jk}}$.
	By linearity, we have that $\smooth{\reals^3}^3\subset\curlZero{\reals^3}{\lambda_{jk}}$ and then by the density of smooth functions in $\ltwo{\reals^3}{\lambda_{jk}}$, we obtain
	\begin{align*}
		\curlZero{\reals^3}{\lambda_{jk}} = \ltwo{\reals^3}{\lambda_{jk}}^3.
	\end{align*}
	We can also infer that any smooth vector field $\Phi\in\smooth{\reals^3}{\lambda_{jk}}^3$ is an element of $\ktcurlSob{\reals^3}{\lambda_{jk}}$ with $\ktcurl{\lambda_{jk}}\Phi = 0$ since $\ltwo{\reals^3}{\lambda_{jk}}^{\perp} = \clbracs{0}$.
	Density of smooth functions in $\ltwo{\reals^3}{\lambda_{jk}}^3$ then demonstrates that any $u\in\ltwo{\reals^3}{\lambda_{jk}}^3$ is also an element of $\ktcurlSob{\reals^3}{\lambda_{jk}}$ with tangential curl equal to zero.
\end{proof}
That is to say, a singular structure in 3 dimensions represented by a graph does not give rise to any non-trivial notion of the curl of a vector field.
Needless to say, this makes exploration of the curl of the curl equation trivial in this context --- it is clear from section \ref{sec:CC-CurlAnalysis} that tangential curls with respect to the measure $\ddmes$ will inherit the behaviour of tangential curls with respect to each $I_{jk}$ on the respective edges, and thus only produce trivial variational problems.

Rather than examining a graph embedded into $\reals^3$, let us instead return to the domain $\dddom$.
First, recall that $\dddom$ consists of a periodic graph embedded into $\reals^2$ with unit cell $\ddom$, and then extruded into $\dddom = \ddom\times[0,\infty)$ to form a domain that consists of a union of planes parallel to the $x_3$-axis.
Our use of a Fourier transform in $x_3$ then brings us back onto the 2 dimensional domain $\ddom$, however it is important for us to remember that each edge of $\graph$ in $\ddom$ actually represents a \emph{plane} in $\reals^3$.
In what follows, we will let $P_{jk}$ be the plane induced by the edge $I_{jk}$ of the graph $\graph$, so $P_{jk} = I_{jk}\times[0,\infty)\subset\ddom\times[0,\infty)$.

Classically, one can interpret the curl $c$ of a vector field $u$ by identifying $c(x)$ as the axis of rotation that an (infinitesimally small) spherical body would undergo if placed in the field $u$ at position $x$, with the angular speed of the rotation equal to half the magnitude of $c(x)$.
Our intuition and geometric interpretation for tangential gradients (section \ref{ssec:3DGradGeometric}) was largely based on the idea that the singular measure $\lambda_{jk}$ cannot ``see" changes in functions \emph{across} the edges $I_{jk}$.
Formally, the interpretation we have for tangential curls (and curls of zero) also appeals to this idea; the measure $\lambda_{jk}$ does not have any concept of normal derivative across $I_{jk}$, and correspondingly the product measure $\lambda_{jk}\times\lambda_1$ does not have any concept of a derivative \emph{outward} from the plane $P_{jk}$.
However, when we are restricted to only observing the values of $u$ in the plane $P_{jk}$, any changes in $u$ in the direction of the outward normal $\widehat{n}_{jk}$ to $P_{jk}$ cannot be seen, or observed.
Such changes in $u$ only affect the components of $c(x)$ in the directions orthogonal to $\widehat{n}_{jk}$, but if we do not know how we are rotating in one direction, we do not know how to orient \emph{the axis} about which we rotate.
In this case, the directions $\widehat{e}_{jk}$ and $\widehat{x}_3$ are those which are orthogonal to $\widehat{n}_{jk}$ (corollary \ref{cory:CurlZero-Rotated}).
On the other hand, the only component of $c(x)$ that isn't derived from changes in $u$ in the direction $\widehat{n}_{jk}$ is the component along the direction $\widehat{n}_{jk}$ itself, which is the only component in tangential curls that is non-zero (corollary \ref{cory:TangCurlEdgeRotated}).
This idea carries over into our previous consideration of singular structures (and conclusion in proposition \ref{prop:3DGraph-CurlsAreZero}) in three dimensions too ---  $\lambda_{jk}$ still has no concept of derivative across the edge $I_{jk}$, and in three dimensions this means there are two linearly independent directions in which rates of change cannot be seen.
Consequentially, we can never accurately reconstruct the axis about which we are rotating, as all three components of the curl would need to know the rate of change in $u$ in at least one of the directions normal to $I_{jk}$.
The measure $\massMes$ tells a similar story --- except in this case we have thrown away the ability to detect change in \emph{every} axial direction!

For a visual interpretation of tangential curls, one can imagine the following picture (illustrated in figure \ref{fig:Diagram_CurlGeometric} with the commentary that follows).
Place a sphere of radius $\ll 1$\footnote{Or alternatively, think of the sphere as being infinitesimally small.} centred at a point $x\in P_{jk}$.
Place a point $p$ on the intersection between the surface of the sphere and the plane $P_{jk}$, and consider the (closed) path $\gamma_p$ that $p$ traces out under the rotation induced by the vector field $u$.
Under the axis of rotation provided by $\ktcurl{\lambda_{jk}}u$, the curve $\gamma_p$ will be entirely contained in the plane $P_{jk}$, and thus the rotation is ``visible" to\footnote{Or for want of a better phrase, the rotation \emph{traces out a path that can be followed by}.} the measure $\lambda_{jk}\times\lambda_1$.
\begin{figure}[b!]
	\centering
	\includegraphics[scale=1.0]{./Diagram_CurlGeometric.pdf}
	\caption{\label{fig:Diagram_CurlGeometric} An illustration of the notation of tangential curl and curls of zero on $P_{jk}$. Tangential curls cause points on an (infinitesimally small) sphere to rotate within the plane $P_{jk}$, whilst curls of zero induce rotations that move points on the sphere out of the plane.}
\end{figure}
However under the axis of rotation provided by a curl of zero, the curve $\gamma_p$ is not contained in the plane $P_{jk}$ --- in fact, it's intersection with the plane will be at most a finite number of points.
If instead $P_{jk}\cap\gamma_p\supset\clbracs{p,p^*}$ for $p^*\neq p$, the measure $\lambda_{jk}\times\lambda_1$ still has no way of knowing \emph{how} $p$ travelled to $p^*$, since the path $\gamma_p$ is not visible (and, from the 3-dimensional perspective, there are infinitely many curls of varying magnitudes that could rotate $p$ to $p^*$ and back).
In the event that $P_{jk}\cap\gamma_p=\clbracs{p}$, then it appears that the sphere is not rotating at all.
That is to say, curls of zero induce rotations ``out of the plane" $P_{jk}$, which the measure $\lambda_{jk}$ cannot observe, whilst tangential curls ensure that all rotation happens ``within" the plane $P_{jk}$.

Similarly to chapter \ref{ch:ScalarSystem}, having an edge-wise behaviour for the tangential curl and curls of zero (that is, with respect to $\lambda_{jk}$) will allow us to construct those with respect to $\ddmes$ and $\dddmes$.
These arguments are the focus of section \ref{sec:CC-CurlAnalysis}, however we again provide the reader with a summary of the key properties of the tangential curl so that the manipulations in section \ref{sec:3DSystemDerivation} can be followed.
To this end, a function $u\in\ktcurlSob{\ddom}{\dddmes}$ has the following properties:
\begin{enumerate}[(i)]
	\item On a given edge $I_{jk}$, we have that $\ktcurl{\dddmes}u = \ktcurl{\lambda_{jk}}u$.
	\item Along the edge $I_{jk}$, only the ``in-plane" components of $u$ are relevant.
	To this end we set $U^{(jk)} = R_{jk}\begin{pmatrix} u^{(jk)}_1 \\ u^{(jk)}_2 \end{pmatrix}$.
	Then we have that $\ktcurl{\lambda_{jk}}u = \bracs{u_3' + \rmi\qm_{jk}u_3 - \rmi\wavenumber U^{(jk)}_2}\widehat{n}_{jk}$, where $\qm_{jk}$ are the same constants from section \ref{sec:ScalarDerivation}, and $u_3'$ is the derivative in the $H^1$ sense of $u^{(jk)}_3\circ r_{jk}$.
	The term $\rmi\wavenumber U^{(jk)}_2$ resembles the information about any rotation induced by changes in $u$ along the $x_3$-axis that the measure $\lambda_{jk}\times\lambda_1$ can see.
	\item We have that $u_3\in\ktgradSob{\ddom}{\dddmes}$, so in particular it is continuous at each vertex $v_j$.
	\item At each $v_j$, we have that $\ktcurl{\dddmes}u = 0$.
\end{enumerate}