\section{Geometric Interpretation of Tangential Curls} \label{sec:CC-Geometric}
Unlike the gradient, the curl of a vector field does not have a natural one-dimensional analogue that we can appeal to when looking to work on singular structure domains.
This makes the analysis of section \ref{sec:CC-CurlAnalysis}, which takes the procedure laid out in section \ref{sec:3DGradSobSpaces} and extends it to curls, crucial to our understanding of the problem \eqref{eq:SingularCurlEquation} and the tangential curls themselves.
In the interests of providing the reader an intuitive idea of what the tangential curl with respect to a singular measure is, we provide a geometric interpretation in this section prior along with a summary of the key behaviours $\ktcurl{\dddmes}u$ exhibits.
We will also address a remark made in section \ref{sec:TP-DomainSetup}, concerning why we do not consider a singular structure modelled by a (periodic) graph embedded into $\reals^3$, and instead only consider a domain formed from the extrusion into 3 dimensions of a periodic graph embedded in $\reals^2$.

Let us begin by demonstrating why curls of zero on a singular structure embedded into $\reals^3$ do not give rise to any interesting variational problems.
Consider a segment $I_{jk}\subset\reals^3$, with $I_{jk} = \clbracs{0}\times\sqbracs{0,l_{jk}}\times\clbracs{0}$, which we can view as one edge of a graph embedded into $\reals^3$ --- the illustration in figure \ref{fig:Diagram_SingularMeasure3D} is representative of the situation.
We can then prove the following result:
\begin{prop} \label{prop:3DGraph-CurlsAreZero}
	Every vector field is a curl of zero, that is
	\begin{align*}
		\curlZero{\reals^3}{\lambda_{jk}} = \ltwo{\reals^3}{\lambda_{jk}}^3,
	\end{align*}
	and consequentially every vector field $u\in\ltwo{\reals^3}{\lambda_{jk}}^3$ is also an element of $\ktcurlSob{\reals^3}{\lambda_{jk}}$ with tangential curl equal to zero.
\end{prop}
\begin{proof}
	We construct explicit approximating sequences for any function in $\ltwo{\reals^3}{\lambda_{jk}}^3$, by first considering $\phi\in\csmooth{\reals^3}$, and considering the vector field $\varphi^{(2)} = -x_1\phi\widehat{x}_3\in\csmooth{\reals^3}^3$.
	We observe that $\varphi^{(2)}=0$ on $I_{jk}$ (since $x_1=0$ here) and 
	\begin{align*}
		\curl{}\varphi^{(2)} = -x_1\partial_2\phi\widehat{x}_1 + \bracs{\phi + x_1\partial_1\phi}\widehat{x}_2,
	\end{align*}
	which is equal to $\phi\widehat{x}_2$ on $I_{jk}$.
	This implies that $\varphi^{(2)}$ serves as an approximating ``sequence" with
	\begin{align*}
		\varphi^{(2)} \lconv{\ltwo{\reals^3}{\lambda_{jk}}^3} 0, 
		\qquad
		\curl{}\varphi^{(2)} \lconv{\ltwo{\reals^3}{\lambda_{jk}}^3} \phi\widehat{x}_2,
	\end{align*}
	so $\phi\widehat{x}_2\in\curlZero{\reals^3}{\lambda_{jk}}$.
	Consideration of the functions $\varphi^{(1)} = -x_3\phi\widehat{x}_2$ and $\varphi^{(3)}=x_1\phi\widehat{x}_2$ demonstrates that $\phi\widehat{x}_1$ and $\phi\widehat{x}_3$ are also elements of $\curlZero{\reals^3}{\lambda_{jk}}$.
	By linearity we have that $\csmooth{\reals^3}^3\subset\curlZero{\reals^3}{\lambda_{jk}}$, and then by a density argument we obtain
	\begin{align*}
		\curlZero{\reals^3}{\lambda_{jk}} = \ltwo{\reals^3}{\lambda_{jk}}^3.
	\end{align*}
	We can also infer that any smooth vector field $\Phi\in\csmooth{\reals^3}{\lambda_{jk}}^3$ is an element of $\ktcurlSob{\reals^3}{\lambda_{jk}}$ with $\ktcurl{\lambda_{jk}}\Phi = 0$ since $\ltwo{\reals^3}{\lambda_{jk}}^{\perp} = \clbracs{0}$.
	Density of smooth functions in $\ltwo{\reals^3}{\lambda_{jk}}^3$ then demonstrates that any $u\in\ltwo{\reals^3}{\lambda_{jk}}^3$ is also an element of $\ktcurlSob{\reals^3}{\lambda_{jk}}$ with tangential curl equal to zero.
\end{proof}
The assumption on the position and orientation of $I_{jk}$ is not restrictive, in the sense that the result we prove will generalise to any line segment (and thus edge) in $\reals^3$.
This result also holds when we replace $\reals^3$ with a finite domain $U\subset\reals^3$ (with non-zero $\lambda_3$-measure) and consider $I_{jk}\subset U$.
We can also obtain a similar conclusion when $U$ is considered to be the period cell of some 3-dimensional singular structure, and $I_{jk}$ a piece of this structure.
Arguments similar to those in section \ref{sec:CC-CurlAnalysis} can then be used to deduce that tangential curls with respect to $\ddmes$ will inherit the behaviour of tangential curls with respect to each $I_{jk}$ on the respective edges.
This culminates in the realisation that a singular structure in 3 dimensions represented by a graph does not give rise to a non-trivial notion of the (tangential) curl of a vector field, and consequentially there are no non-trivial solutions to \eqref{eq:SingularCurlEquation} on such a structure.

This trivial behaviour is consistent with the interpretation of tangential curls that we obtain from our analysis in section \ref{sec:CC-CurlAnalysis} of the space $\ktcurlSob{\ddom}{\dddmes}$.
Classically, one can interpret the curl $c$ of a vector field $u$ by identifying $c(x)$ as the axis of rotation that an (infinitesimally small) spherical body would undergo if placed in the field $u$ at position $x$ (with the angular speed of the rotation proportional to the magnitude of $c(x)$).
Our intuition and geometric interpretation for tangential gradients (section \ref{ssec:3DGradGeometric}) was largely based on the idea that the singular measure $\lambda_{jk}$ cannot perceive changes in functions \emph{across} the edges $I_{jk}$.
The interpretation we have for tangential curls (and curls of zero) also appeals to this idea; the measure $\lambda_{jk}$ does not have any concept of normal derivative across $I_{jk}$, and correspondingly the product measure $\lambda_{jk}\times\lambda_1$ does not have any concept of a derivative \emph{outward} from the plane $P_{jk}$.
When we are restricted to only observing the values of $u$ in the plane $P_{jk}$, any changes in $u$ in the direction of the outward normal $\widehat{n}_{jk}$ to $P_{jk}$ cannot be observed.
Such changes in $u$ only affect the components of $c(x)$ in the directions orthogonal to $\widehat{n}_{jk}$ --- but if we do not know how we are rotating in these directions, we do not know how to orient \emph{the axis} about which we rotate.
As a result, any ``rotations" with axes parallel to the $\widehat{x}_3$ and $\widehat{e}_{jk}$ directions must correspond to curls of zero (corollary \ref{cory:CurlZero-Rotated}).
On the other hand, the only component of $c(x)$ that isn't derived from changes in $u$ in the direction $\widehat{n}_{jk}$ is the component along the direction $\widehat{n}_{jk}$ itself, which is the only component in tangential curls that is non-zero (corollary \ref{cory:TangCurlEdgeRotated}).
We can apply these ideas to our previous consideration of singular structures represented by graphs embedded into $\reals^3$ (proposition \ref{prop:3DGraph-CurlsAreZero}) ---  $\lambda_{jk}$ has no concept of derivative across the edge $I_{jk}$, and in three dimensions this means there are two linearly independent directions in which rates of change cannot be seen.
Consequentially, we can never accurately reconstruct the axis about which we are rotating, as all three components of any ``curl" require information about the rate of change in $u$ in at least one of the directions normal to $I_{jk}$.
The measure $\massMes$ tells a similar story: in this case we have thrown away the ability to perceive change in \emph{every} axial direction, and like the case for tangential gradients we find that $\ktcurl{\massMes}u=0$ for any vector field.

For a visual interpretation of tangential curls one can imagine the following scenario, which is illustrated in figure \ref{fig:Diagram_CurlGeometric} using the language of the commentary that follows.
\begin{figure}[t!]
	\centering
	\includegraphics[scale=1.0]{./Diagram_CurlGeometric.pdf}
	\caption[Geometric interpretation of tangential curls with respect to $\dddmes$.]{\label{fig:Diagram_CurlGeometric} An illustration of the notation of tangential curl and curls of zero on $P_{jk}$. Tangential curls cause points on an (infinitesimally small) sphere to rotate within the plane $P_{jk}$, whilst curls of zero induce rotations that move points on the sphere out of the plane.}
\end{figure}
Imagine an infinitesimally small sphere whose centre is at a point $x\in P_{jk}$, and then place another point $p$ on the intersection between the surface of the sphere and the plane $P_{jk}$.
Then consider the (closed) path $\gamma_p$ that $p$ traces out under the rotation induced by the vector field $u$.
Under the axis of rotation provided by $\ktcurl{\lambda_{jk}}u$, the curve $\gamma_p$ will be entirely contained in the plane $P_{jk}$, and thus the rotation is visible to\footnote{Or for want of a better phrase, the rotation \emph{traces out a path that can be followed by}.} the measure $\lambda_{jk}\times\lambda_1$.
However under the axis of rotation provided by a curl of zero, the curve $\gamma_p$ is not contained in the plane $P_{jk}$ --- in fact, it's intersection with the plane will be at most a finite number of points.
If instead $P_{jk}\cap\gamma_p\supset\clbracs{p,p^*}$ for $p^*\neq p$, the measure $\lambda_{jk}\times\lambda_1$ still has no way of knowing \emph{how} $p$ travelled to $p^*$, since the path $\gamma_p$ is not visible (and, from the 3-dimensional perspective, there are infinitely many curls of varying magnitudes that could rotate $p$ to $p^*$ and back).
In the event that $P_{jk}\cap\gamma_p=\clbracs{p}$, then it appears that the sphere is not rotating at all.
That is to say, curls of zero induce rotations ``out of the plane" $P_{jk}$, which the measure $\lambda_{jk}\times\lambda_1$ cannot observe, whilst tangential curls ensure that all rotation happens ``within" the plane $P_{jk}$.

Similarly to chapter \ref{ch:ScalarSystem}, it is the case that the behaviours of tangential curls and curls of zero on the edges of our singular structure (that is, with respect to $\lambda_{jk}$) are inherited by vector fields with tangential curls with respect to $\ddmes$ and $\dddmes$.
These arguments are the focus of section \ref{sec:CC-CurlAnalysis}, however we again provide the reader with a summary of the key properties of the tangential curl so that the manipulations in section \ref{sec:3DSystemDerivation} can be followed.
To this end, a function $u\in\ktcurlSob{\ddom}{\dddmes}$ has the following properties:
\begin{enumerate}[(i)]
	\item On a given edge $I_{jk}$, we have that $\ktcurl{\dddmes}u = \ktcurl{\lambda_{jk}}u$.
	\item Along the edge $I_{jk}$, only the ``in-plane" components of $u$ are relevant.
	To this end we set $U^{(jk)} = R_{jk}\begin{pmatrix} u^{(jk)}_1 \\ u^{(jk)}_2 \end{pmatrix}$, so that $U_2^{(jk)} = \bracs{u_1^{(jk)},u_2^{(jk)}}^\top e_{jk}$, and we have that
	\begin{align*}
		\ktcurl{\lambda_{jk}}u = \bracs{\bracs{u_3^{(jk)}}' + \rmi\qm_{jk}u_3^{(jk)} - \rmi\wavenumber U^{(jk)}_2}\widehat{n}_{jk}.
	\end{align*}
	The constants $\qm_{jk}=\qm\cdot e_{jk}$ are the same constants from section \ref{sec:ScalarDerivation}, and $u_3'$ is the derivative in the $H^1$ sense of $u^{(jk)}_3\circ r_{jk}$.
	The term $\rmi\wavenumber U^{(jk)}_2$ resembles the information about any rotation induced by changes in $u$ along the $x_3$-axis that the measure $\lambda_{jk}\times\lambda_1$ can see.
	\item We have that $u_3\in\ktgradSob{\ddom}{\dddmes}$, so in particular it is continuous at each vertex $v_j$.
	\item At each $v_j$, we have that $\ktcurl{\dddmes}u = 0$.
	However much like the case with gradients, the incoming curls to a vertex from the connecting edges do not have to have a limit of zero into the vertex!
\end{enumerate}