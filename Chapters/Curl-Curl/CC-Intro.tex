\section{Chapter Introduction} \label{sec:Curl-Intro}
The work undertaken in this chapter is a natural extension and continuation on the motivation and ideas of of chapter 3, pushing us towards the equations of electromagnetism on singular structures.
Here we look to study a variational problem motivated by the (weak form of the) curl-of-the-curl equation, akin to how we motivated the problem \eqref{eq:SingularScalarWaveEqn} as an analogue to the acoustic approximation in the ``zero-thickness" limit.
This takes us outside the realms in which we can rely on intuition to guide us --- the curl and divergence do not have (obvious) one-dimensional analogues in the same way that the gradient does (in the derivative).
As such, we are heavily dependent on our analysis of the spaces of functions with curls (with respect to our singular measure) in order to turn our variational problem into a more tractable problem.
Furthermore, in choosing to study vectorial systems we move outside the reach of the known results concerning the zero-thickness limits of thin-structure problems, as analysis akin to the studies \cite{kuchment2001convergence, kuchment2003asymptotics, exner2005convergence, post2012spectral} (see \ref{ssec:Intro-ThinStructures}) --- an analogous analysis of the curl-of-the-curl equation (and broader Maxwell system) on a thin structure as the thickness tends to zero is presently an open problem in the literature.
Furthermore, the arguments available in the context of the acoustic approximation rely on knowing \emph{a priori} the resulting limiting problem one should obtain.
Our physically motivated approach serves as a predictive tool in this regard; we use our variational problems with respect to singular structures to provide candidates for such problems (or even operators) to support such analysis.
This builds off the success of our variational approach and singular structures in the previous chapter; the system \eqref{eq:SingularWaveEqnQGProblem} matches that of the limiting problems one obtains for the acoustic equation on a thin structure, and the variational problem studied is the direct analogue its thin-structure counterpart.
%
%The focus of this chapter is extending the work of chapter \ref{ch:ScalarSystem} in understanding variational problems on singular structures, the spaces and objects involved, and their explicit solution to the ``curl of the curl" equation.
%Using the ideas and concepts developed in chapter \ref{ch:ScalarSystem}, we can begin to define analogues of the curl and divergence of vector fields with respect to our singular measures.
%This takes us outside the realms in which we can rely on intuition to guide us --- the curl and divergence do not have (obvious) one-dimensional analogues in the same way that the gradient possesses (with the derivative).
%However it in turn allows us to consider variational problems motivated by the Maxwell system that go beyond the acoustic approximation, examine candidates for ``limiting" problems for the Maxwell system on thin structures, and provide insights into the effects that might be lost when considering the acoustic approximation over the vectorial equations.
%Unlike with the acoustic approximation, an analysis of the curl of the curl equation (and broader Maxwell system) on a thin structure as the thickness tends to zero is presently an open problem in the literature.
%Furthermore, the closest analogue available in the context of the acoustic approximation --- the studies \cite{kuchment2001convergence, kuchment2003asymptotics, exner2005convergence} --- rely on knowing \emph{a priori} the resulting limiting problem one should obtain.
%Here we appeal to the success of our variational approach and singular structures in the previous chapter; the system \eqref{eq:SingularWaveEqnQGProblem} matches that of the limiting problems one obtains for the acoustic equation on a thin structure, and the variational problem studied is the direct analogue its thin-structure counterpart.
%By starting from a variational problem on a singular structure that reflects the curl of the curl equation, we aim to provide a candidate for the thin structure limit of said equation, and to examine some of the behaviours one can expect from the spectrum and/or solutions.

We now formulate the variational problem for the curl-of-the-curl equation on a singular structure.
Let $\ktcurlSob{\ddom}{\dddmes}$ be the space of functions with $\dddmes$-tangential curls as introduced in section \ref{sec:BorelMeasSobSpaces}, and let $\graph=\bracs{\vertSet,\edgeSet}$ be the period graph of a (periodic) metric graph $\hat{\graph}$ embedded into $\reals^2$, with unit cell $\ddom=[0,1)^2$.
We now extrude $\hat{\graph}$ into three dimensions, forming the singular structure $\hat{\graph}\times[0,\infty)\subset\reals^3$ which consists of a union of planes parallel to the $x_3$ axis, and contained within the region $\dddom=\ddom\times[0,\infty)$.
This provides us with a fibre-like geometry --- $\hat{\graph}$ forms a the periodic, two dimensional cross section as in a two-dimensional PC\footnote{However, we are still only considering a singular structure without a material filling the space between the singular structure.} or ``coreless" PCF, looking to guide light along the $x_3$-axis along which the material properties are invariant.
The invariance along the fibre axis, and periodic structure in the $\bracs{x_1,x_2}$-plane means it is natural for us to take a Fourier transform in $\widehat{x}_3$ and a Gelfand transform in the cross-sectional plane, taking us to the problem
\begin{align} \label{eq:SingularCurlEquation}
	\ktcurl{\dddmes}\bracs{ \ktcurl{\dddmes}u } &= \omega^2 u,
	\qquad \text{in } \ddom,
\end{align}
for \emph{divergence-free} $u\in\ktcurlSob{\ddom}{\dddmes}$, with $\wavenumber$ the Fourier variable (and $\qm$ the quasi-momentum).
We again need to study the spaces $\curlZero{\ddom}{\dddmes}$ and $\ktcurlSob{\ddom}{\dddmes}$ in order to understand what the tangential curl of a vector field \emph{is} and how it behaves, which is the subject of analysis in section \ref{sec:CC-CurlAnalysis}.
Additionally, we need an analogue of the divergence-free condition that compliments the curl-of-the-curl equation, which is the topic of discussion in section \ref{sec:DivFreeCondition}.
It is also important that we bear in mind that each edge $I_{jk}$ induces a plane $P_{jk} = I_{jk}\times[0,\infty)$ parallel to $\widehat{x}_3$, and multiplication by $\rmi\wavenumber$ in Fourier space represents differentiation with respect to $x_3$ --- the interpretation of tangential curls in section \ref{sec:CC-Geometric} relies on this.

Of course, we should again clarify how we interpret \eqref{eq:SingularCurlEquation}: as the problem of finding non-zero $u\in\ktcurlSob{\ddom}{\dddmes}$ and $\omega>0$ such that 
\begin{align} \label{eq:SingularCurlEquation-VariationalForm}
	\integral{\ddom}{ \ktcurl{\dddmes}u\cdot\overline{\ktcurl{\dddmes}\phi} }{\dddmes}
	&= \omega^2 \integral{\ddom}{ u\cdot\overline{\phi} }{\dddmes}, 
	\qquad \forall \phi\in\psmooth{\ddom}.
\end{align}
Once again however, we remark that one can define an operator whose spectral problem corresponds to \eqref{eq:SingularCurlEquation-VariationalForm}; first by defining the bilinear form
\begin{align*}
	\dom \bracs{c_{\qm}} &= \ktcurlSob{\ddom}{\dddmes} \times \ktcurlSob{\ddom}{\dddmes}, \\
	c_{\qm}\bracs{u,v} &= \integral{\ddom}{ \ktcurl{\dddmes}u\cdot\overline{\ktcurl{\dddmes}v} }{\dddmes},
\end{align*}
and then the operator $\ktcurl{\dddmes}\bracs{ \ktcurl{\dddmes} \cdot }$  via
\begin{align*}
	\dom\bracs{ \ktcurl{\dddmes}\bracs{ \ktcurl{\dddmes} \cdot } }
	&= \clbracs{ u\in\ktcurlSob{\ddom}{\dddmes} \setVert \exists f\in\pltwo{\ddom}{\dddmes} \text{ s.t. } \right. \\
	&\qquad \left. c_{\qm}\bracs{u,v}=\ip{f}{v}_{\ktcurlSob{\ddom}{\dddmes}}, \quad\forall v\in\ktcurlSob{\ddom}{\dddmes} }, \labelthis\label{eq:CurlCurlOperatorDefinition}
\end{align*}
with action
\begin{align*}
	\ktcurl{\dddmes}\bracs{ \ktcurl{\dddmes} u } &= f,
\end{align*}
where $f$ and $u$ are related as in \eqref{eq:CurlCurlOperatorDefinition}.
As in the case of the acoustic approximation, we can also define an operator $\kcurl{\upsilon}\bracs{ \kcurl{\upsilon} \cdot }$ acting on functions in $\ktcurlSob{\ddom}{\upsilon}$ where $\upsilon$ is the singular measure supporting $\hat{\graph}$, and have that
\begin{align*}
	\sigma\bracs{ \kcurl{\upsilon}\bracs{ \kcurl{\upsilon} \cdot } }
	&= \bigcup_{\qm\in[-\pi,\pi)^2} \sigma\bracs{ \ktcurl{\dddmes}\bracs{ \ktcurl{\dddmes} \cdot } }.
\end{align*}

We will again look to turn the problem \eqref{eq:SingularCurlEquation} into a more tractable problem on the underlying graph $\graph$, which is the objective of section \ref{sec:3DSystemDerivation}.
Once again we will be left with a system that is realisable as a quantum graph problem (see \eqref{eq:QGRawSystem}), however the vectorial nature of the resulting equations will be lost by virtue due to consequences of the divergence-free conditions.
Although this is unexpected given the starting point \eqref{eq:SingularCurlEquation}, in section \ref{sec:CC-Discussion} we will explore some of the underlying reasons for this reduction to a scalar system, further parallels with the classical setting that indicate this is the correct behaviour to expect, and some of the implications for the first-order Maxwell system if this is the case.
We will conclude by highlighting the open problems in the literature that have been bought to our attention in light of this analysis, and whose investigation may provide further insight.