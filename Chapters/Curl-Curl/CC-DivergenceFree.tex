\section{The Divergence-Free Condition} \label{sec:DivFreeCondition}
As was mentioned in section \ref{sec:Intro-Maxwell}, the definition of the Maxwell operator requires us to pay respect to the divergence-free nature of the electric and magnetic fields; in particular the vector fields that make up the space on which the curl of the curl equation is posed on are required to be divergence-free.
This divergence-free condition adds extra constraints to the fields, and the analysis of section \ref{sec:CC-CurlAnalysis} indicates that we require an analogue of this divergence-free condition to constrain the behaviour of vector fields with $\dddmes$-tangential curls near the vertices.
Given the motivation for our consideration of the curl of the curl equation on a singular structure, it is natural for us to consider what the analogue of a divergence free field would be.
Using the classical notion of a divergence-free vector field as our starting point, we find ourselves presented with several reasonable definitions for the notion of a divergence free vector field with respect to our singular measures, which we will discuss in turn below.
However, we will see that a ``strong" definition through approximating sequences fails to provide a rather underwhelming\footnote{If not entirely unexpected.} description (see lemma \ref{lem:DivZero-Everything}), whilst the ``weaker" definitions \ref{def:DivFree-AllGradients} and \ref{def:DivFree-TangGradients} do provide some additional constraints on our vector fields, but section \ref{sec:3DSystemDerivation} will show that these constraints are also directly obtainable from \tstk{curl of curl equation}.

Given how we have chosen to define gradients and curls with respect to our singular measures, it is natural to ask whether it would be possible to define the tangential divergence through an approximation via smooth functions in a similar manner.
A short investigation proves that this is not be the case.
If we follow our motivations for gradients and curls, for a (Borel) measure $\rho$ we could define
\begin{align*}
	W_{\rho,\mathrm{div}}^{\kt} = \overline{\clbracs{ \bracs{\phi,\ktgrad\cdot\phi} \setVert \phi\in\psmooth{\ddom}^3 }},
\end{align*}
with the closure taken in $\ltwo{\ddom}{\rho}^3\times\ltwo{\ddom}{\rho}$, and define the space of divergences of zero (with respect to $\rho$) as
\begin{align*}
	\mathcal{D}_{\ddom,\md\rho}\bracs{0} &= \clbracs{d \setVert \bracs{0,d}\in W_{\rho,\mathrm{div}}^{\kt}}.
\end{align*}
Similar arguments to those in section \ref{sec:BorelMeasSobSpaces} establish that $\mathcal{D}_{\ddom,\md\rho}\bracs{0}$ does not depend on the value of $\kt$, and is a closed, linear subspace of $\pltwo{\ddom}{\rho}$.
Hence we can decompose $\pltwo{\ddom}{\rho}$ as the direct sum of $\mathcal{D}_{\ddom,\md\rho}\bracs{0}$ and its orthogonal complement.
In turn, we would find that a function has at most one ``tangential divergence", and thus could define the Sobolev space of functions possessing divergences as
\begin{align*}
	H^1_{\qm,\wavenumber,\mathrm{div}}\bracs{\ddom,\md\rho}
	&= \clbracs{ \bracs{u,d}\in W_{\rho,\mathrm{div}}^{\kt} \setVert d\perp\mathcal{D}_{\ddom,\md\rho}\bracs{0} }.
\end{align*}
So far there have been no problems.
However once we consider the divergences of zero on a single edge of our singular structure, we will quickly see an issue:
\begin{lemma} \label{lem:DivZero-Everything}
	For any edge $I_{jk}$, we have that $\mathcal{D}_{\ddom,\md\lambda_{jk}}\bracs{0}=\pltwo{\ddom}{\lambda_{jk}}$.
\end{lemma}
\begin{proof}
	Let $\phi\in\psmooth{\ddom}$, and let $\varphi\in\csmooth{\ddom}$ be a smooth function with $\varphi(x)=\bracs{x-v_j}\cdot n_{jk}\phi(x)$ in some neighbourhood of $I_{jk}$.
	Then the smooth function $\varphi\widehat{n}_{jk}\in\csmooth{\ddom}^3$ is such that $\varphi=0$ and $\grad\cdot\varphi\widehat{n}_{jk}=\phi$ on $I_{jk}$, so $\phi\in\mathcal{D}_{\ddom,\md\lambda_{jk}}\bracs{0}$.
	A density argument then completes the proof.
\end{proof}
Further to this, it is not difficult to prove an analogy of the extension results (proposition \ref{prop:3DGradZeroChar} and theorem \ref{thm:CurlZeroChar}) for these divergences of zero.
Neither is it difficult to show that $\mathcal{D}_{\ddom,\md\massMes}\bracs{0}=\pltwo{\ddom}{\massMes}$, and consequentially that an analogue of theorem \ref{thm:3DdddmesCharGradZero} and proposition \ref{prop:ThickVertexCurlZeroCharacterisation} holds, so every tangential divergence is zero!
We should not be surprised at this result --- the divergence is a scalar representing the net flow of a vector field out of a given point $x$.
However our singular measure cannot see any such flux across the edges $I_{jk}$ (or out of the planes $P_{jk}$), thus any divergence induced by such a flux must be a divergence of zero.
However it is then impossible to know ``how much" of the scalar divergence would be due to these unobservable fluxes across $I_{jk}$, and how much was from the observable portions of the vector field along $I_{jk}$.

Naturally, one might find it dissatisfying that all vector fields are divergence free in lieu of the above computation, and instead look for other possible definitions or interpretations.
A characterising feature of (classical) divergence-free fields is that they are orthogonal to all potential vector fields (that is, any vector field equal to the gradient of a scalar function).
Utilising this analogue with the classical divergence, we could adopt the following definition for our analogue of divergence-free fields:
\begin{definition}[$\dddmes$-Divergence Free] \label{def:DivFree-AllGradients}
	A field $u\in\pltwo{\ddom}{\dddmes}^3$ is divergence free (with respect to $\dddmes$) if (and only if)
	\begin{align*}
		\ip{u}{g}_{\ltwo{\ddom}{\dddmes}^3} = \integral{\ddom}{ u\cdot\overline{g} }{\dddmes} = 0,
	\end{align*}
	for every $g\in W_{\mathrm{grad}}^{\kt}\bracs{\ddom,\md\dddmes}$.
\end{definition}
That is, a field is divergence free only if it is orthogonal in $\ltwo{\ddom}{\dddmes}$ to any gradient field, including the gradients of zero.
Whilst this ``weak" definition breaks the tradition of our approach via smooth approximations, it will nonetheless be valuable for us to examine the implications of definition \ref{def:DivFree-AllGradients} (and the later definition \ref{def:DivFree-TangGradients}) when we come to the analysis of \tstk{curl-curl equation}.
Definition \ref{def:DivFree-AllGradients} admits the following characterisation.
\begin{prop} \label{prop:DivFree-AllGradsConditions}
	Let $u\in\pltwo{\ddom}{\dddmes}^3$ and define (for each edge $I_{jk}$)
	\begin{align*}
		U^{(jk)} = R_{jk}\begin{pmatrix} u_1^{(jk)} \\ u_2^{(jk)} \end{pmatrix}, 
		\qquad \qm_{jk} = \qm\cdot e_{jk}, 
		\qquad \widetilde{U}^{(jk)} = U^{(jk)}\circ r_{jk}.
	\end{align*}		
	Then $u$ is divergence-free if and only if
	\begin{align*}
		\mathrm{(i)} \ & U_1^{(jk)} = 0, \quad \forall I_{jk}\in\edgeSet \\
		\mathrm{(ii)} \ & \widetilde{U}_2^{(jk)}\in\gradSob{\sqbracs{0,l_{jk}}}{y}, \quad \forall I_{jk}\in\edgeSet, \\
		\mathrm{(iii)} \ & \bracs{U_2^{(jk)}}' + \rmi\qm_{jk}U_2^{(jk)} + \rmi\wavenumber u_3^{(jk)} = 0 \text{ on } I_{jk}, \quad \forall I_{jk}\in\edgeSet, \\
		\mathrm{(iv)} \ & u_1\bracs{v_j} = u_2\bracs{v_j} = 0, \quad \forall v_j\in\vertSet, \\
		\mathrm{(v)} \ & \sum_{j\conRight k}U_2^{(kj)}\bracs{v_j} - \sum_{j\conLeft k}U_2^{(jk)}\bracs{v_j} = \rmi\wavenumber\alpha_j u_3\bracs{v_j}, \quad \forall v_j\in\vertSet,
	\end{align*}
	where $\bracs{U_2^{(jk)}}' = \bracs{\widetilde{U}_2^{(jk)}}'\circ r_{jk}^{-1}$.
\end{prop}
\begin{proof}
	First assume that the field $u$ is divergence-free (in the sense of definition \ref{def:DivFree-AllGradients}).
	\begin{enumerate}[(i)]
		\item We first utilise theorem \ref{thm:3DdddmesCharGradZero} to deduce that $\ip{u}{g}_{\ltwo{\ddom}{\lambda_{jk}}}=0$ for every $g\in\gradZero{\ddom}{\lambda_{jk}}$.
		Then since we know that $\gradZero{\ddom}{\lambda_{jk}}$ consists of all functions of the form $g_{jk}\hat{n}_{jk}$ (proposition \ref{prop:3DGradZeroChar}), we must conclude that
		\begin{align*}
			0 &= \integral{\ddom}{ \overline{g}_{jk}\begin{pmatrix} u_1^{(jk)} \\ u_2^{(jk)} \end{pmatrix}\cdot n_{jk} }{\lambda_{jk}}, \qquad \forall g_{jk}\in\pltwo{\ddom}{\lambda_{jk}},
		\end{align*}
		so $U_1^{(jk)}=0$ on $I_{jk}$.
		\item Let $\phi\in\csmooth{\sqbracs{0,l_{jk}}}$.
		Utilising suitable cut-off functions, we can then construct a function $\varphi\in\csmooth{\ddom}\cap\psmooth{\ddom}$ such that $\varphi(x)=\phi\bracs{r_{jk}^{-1}(x)}$ when $x\in I_{jk}$, and with $\varphi=0$ on all other parts of $\graph$.
		Since such $\varphi$ are smooth, they are elements of $\ktgradSob{\ddom}{\lambda_{jk}}$ and by lemma \ref{lem:ExtensionLemmaEdgeFunctions} also elements of $\ktgradSob{\ddom}{\dddmes}$.
		As such, we have that
		\begin{align*}
			0 &= \integral{\ddom}{ u\cdot\overline{\ktgrad_{\dddmes}\varphi} }{\dddmes}
			= \integral{I_{jk}}{ u\cdot\bracs{ \bracs{\overline{\varphi}' - \rmi\qm_{jk}\overline{\varphi}}\widehat{e}_{jk} - \rmi\wavenumber\overline{\varphi}\widehat{x}_3} }{\lambda_{jk}} \\
			&= \integral{I_{jk}}{ \bracs{\overline{\varphi}' - \rmi\qm_{jk}\overline{\varphi}}U_2^{(jk)} - \rmi\wavenumber\overline{\varphi}u_3^{(jk)} }{\lambda_{jk}} \\
			&= \int_0^{l_{jk}} \bracs{\overline{\phi}' - \rmi\qm_{jk}\overline{\phi}}\widetilde{U}_2^{(jk)} - \rmi\wavenumber\overline{\phi}\widetilde{u}_3^{(jk)} \ \md y, \\
			\implies
			\int_0^{l_{jk}} \overline{\phi}'\widetilde{U}_2^{(jk)} \ \md y
			&= \int_0^{l_{jk}} \overline{\phi}\bracs{ \rmi\qm_{jk}\widetilde{U}_2^{(jk)} + \rmi\wavenumber\widetilde{u}_3^{(jk)} } \ \md y.
		\end{align*}
		This holds for every $\phi\in\csmooth{\sqbracs{0,l_{jk}}}$, and so we can conclude that $\widetilde{U}_2^{(jk)}\in\gradSob{\sqbracs{0,l_{jk}}}{y}$.
		\item Following immediately on from the above, we have that
		\begin{align*}
			\bracs{U_2^{(jk)}}' = - \bracs{ \rmi\qm_{jk}U_2^{(jk)} + \rmi\wavenumber u_3^{(jk)} },
		\end{align*}
		which upon rearrangement gives the result.
		\item We again notice through theorem \ref{thm:3DdddmesCharGradZero} we have that $\ip{u}{g}_{\ltwo{\ddom}{\massMes}}=0$ for every $g\in\gradZero{\ddom}{\massMes}$.
		Fix $v_j\in\vertSet$, and take $g=\bracs{g_1,g_2,0}^\top$ at $v_j$ and zero elsewhere.
		Given the characterisation in proposition \ref{prop:NuGradZeroChar}, we can conclude that
		\begin{align*}
			0 &= \integral{\ddom}{ u_\cdot\overline{g} }{\massMes}
			= \alpha_j\bracs{\overline{g_1}u_1 + \overline{g_2}u_2},
		\end{align*}
		for all $g_1,g_2\in\complex$.
		Therefore, we must have that $u_1\bracs{v_j}=u_2\bracs{v_j}=0$.
		\item Finally, take $\varphi\in\csmooth{\ddom}$ with support
		\begin{align*}
			\supp\bracs{\varphi} \subset \mathcal{J}\bracs{v_j}\setminus\clbracs{v_k\in\vertSet \setVert v_k\neq v_j}.
		\end{align*}
		Since this $\varphi$ is smooth, it is an element of $\ktgradSob{\ddom}{\dddmes}$ by definition, and we have that
		\begin{align*}
		0 &= \integral{\ddom}{ u\cdot\overline{\ktgrad_{\dddmes}\varphi} }{\dddmes} \\
		&= \sum_{j\con k}\integral{I_{jk}}{ U_2^{(jk)}\bracs{\overline{\varphi}' - \rmi\qm_{jk}\overline{\varphi}} - \rmi\wavenumber u_3^{(jk)}\overline{\varphi} }{\lambda_{jk}} 
		+ \integral{\ddom}{ u\cdot\overline{\ktgrad_{\dddmes}\varphi} }{\massMes} \\
		&= \sum_{j\con k}\int_0^{l_{jk}} \widetilde{U}_2^{(jk)}\bracs{\overline{\widetilde{\varphi}}' - \rmi\qm_{jk}\overline{\widetilde{\varphi}}} - \rmi\wavenumber u_3^{(jk)}\overline{\widetilde{\varphi}} \ \md y
		-\rmi\wavenumber\alpha_j u_3\bracs{v_j}\overline{\varphi}\bracs{v_j}.
		\end{align*}
		Rearranging and using (ii) and (iii), we find that
		\begin{align*}
			\rmi\wavenumber\alpha_j u_3\bracs{v_j}\overline{\varphi}\bracs{v_j}
			&= - \sum_{j\con k}\int_0^{l_{jk}} \overline{\widetilde{\varphi}}\bracs{ \widetilde{U}_2^{(jk)} + \rmi\qm_{jk}\widetilde{U}_2^{(jk)} + \rmi\wavenumber\widetilde{u}_3^{(jk)} } \ \md y \\
			&\quad + \overline{\varphi}\bracs{v_j}\bracs{ \sum_{j\conRight k} U_2^{(kj)}\bracs{v_j} - \sum_{j\conLeft k} U_2^{(jk)}\bracs{v_j} } \\
			&= \overline{\varphi}\bracs{v_j}\bracs{ \sum_{j\conRight k} U_2^{(kj)}\bracs{v_j} - \sum_{j\conLeft k} U_2^{(jk)}\bracs{v_j} }.
		\end{align*}
		This holds for every such smooth $\varphi$, and so we must have that
		\begin{align*}
			\rmi\wavenumber\alpha_j u_3\bracs{v_j} &= \sum_{j\conRight k} U_2^{(kj)}\bracs{v_j} - \sum_{j\conLeft k} U_2^{(jk)}\bracs{v_j}.
		\end{align*}
	\end{enumerate}
	
	Now let us suppose that (i)-(v) hold.
	Conditions (i) and (iv) ensure that $u$ is orthogonal to $\gradZero{\ddom}{\dddmes}$ via theorem \ref{thm:3DdddmesCharGradZero}.
	Now let $v\in\ktgradSob{\ddom}{\dddmes}$, and consider an approximating sequence $\phi_n$ for $v$.
	Clearly,
	\begin{align*}
		\ip{u}{\ktgrad_{\dddmes}v}_{\ltwo{\ddom}{\dddmes}^3}
		&= \lim_{n\rightarrow\infty}\ip{u}{\ktgrad\phi_n}_{\ltwo{\ddom}{\dddmes}^3},
	\end{align*}
	from which we can use (ii), (iii), and (v) to deduce that $\ip{u}{\ktgrad\phi_n}_{\ltwo{\ddom}{\dddmes}^3}=0$ for every $n\in\naturals$, providing orthogonality to tangential gradients.
\end{proof}
We highlight that a field $u$ being divergence-free provides us with some additional information about the regularity of the components of $u$, however still falls short of establishing any kind of continuity of the in-plane components $u_1$ and $u_2$ near the vertices.
The condition (ii) provides us with regularity of (some linear combination of) these components along each edge, but we do not obtain any information about their traces, other than the condition (v) --- which does not imply the traces must match even when $u_3(v_j)=0$.

We can make some remarks about each of the conditions (i)-(v) in proposition \ref{prop:DivFree-AllGradsConditions} to draw parallels with the usual notion of divergence free.
The left hand side of condition (iii) bears close resemblance to the expression for the three dimensional divergence; recalling our use of a Fourier transform in the $\widehat{x}_3$ direction, the $\rmi\wavenumber u_3$ term is the result of the presence of a $\partial_3 u_3$ term in non-Fourier space.
Furthermore, we can also associate
\begin{align*}
	\bracs{U_2^{(jk)}}' + \rmi\qm_{jk}U_2^{(jk)} &= \bracs{\pdiff{}{e_{jk}} + \rmi\qm\cdot e_{jk}}\bracs{U^{(jk)}\cdot e_{jk}},
\end{align*}
which is the (shifted) derivative along the edge $I_{jk}$ (which exists thanks to (ii)).
Given that the vectors $\widehat{e}_{jk}$ and $\widehat{x}_3$ span the plane $P_{jk}$ induced by the edge $I_{jk}$, the left hand side of (iii) amounts to an ``in-plane" divergence (using the local coordinate frame $y_{jk}$ over the axial reference frame in the $\bracs{x_1,x_2}$-plane) which, for a divergence free field is required to be zero as one might expect.
It is not surprising that the condition (iii) does not involve any rates of change in the $\hat{n}_{jk}$ direction (``out of the plane" $P_{jk}$), since our singular measure cannot see such changes as they occur across the edges $I_{jk}$.
The presence of the conditions (i) and (iv) is slightly more perplexing.
Condition (i) informs us that our vector field $u$ is \emph{not} the result of changes across the edges $I_{jk}$, that our singular measure cannot observe.
We can take this as the only assurance $\lambda_{jk}$ can provide that our vector field $u$ is not inducing any flux or flow in the direction normal to $I_{jk}$ --- whilst we cannot see changes across the edges, we can at least observe the direction of the field \emph{on} the edge, and make the ``out-of-plane" component zero.
The condition (iv) fills the same role at the vertices --- recall that $\massMes$ cannot observe the function $u$ outside of the vertices, so $\massMes\times\lambda_1$ can only observe changes along the line induced by $v_j$ and thus the ``in-plane" components $u_1$ and $u_2$ must be zero at each vertex.
The condition (v) is then the result of the interaction of the incoming fields at the vertices\footnote{The condition (v) even bears resemblance to a $\delta'$-type vertex condition in a quantum graph problem --- the ``derivative" $\rmi\wavenumber u_3$ at $v_j$ must equal a multiple of the incoming edge functions.}.
To be divergence free at the vertex $v_j$; the sum of the ``flux" seen by the measure $\ddmes$, represented by the trace values of the $U_2^{(jk)}$ (the sign depending on whether the corresponding edge is directed into or out of the vertex $v_j$), and the ``flux" observed by $\massMes\times\lambda_1$ along the line induced by $v_j$, must balance.

This interpretation does raise the question as to whether we can define some kind of \emph{weak divergence} of a vector field with respect to our singular measures.
However we can note from proposition \ref{prop:DivFree-AllGradsConditions} that only conditions (iii) and (v) actually affect what such a ``weak divergence" would look like --- the requirement that a field be orthogonal to gradients of zero to be divergence-free directly affects the field itself.
Classically of course, there are no ``gradients of zero" and so there is no distinction between tangential gradients and gradients of zero.
Consequentially, divergence free functions being orthogonal to gradients is functionally the same as them being orthogonal to all tangential gradients.
In pursuit of a ``weak divergence", we first put forward a weaker definition of ``divergence-free", which for ease of language we will refer to as \emph{tangentially divergence-free}.
\begin{definition}[Tangentially $\dddmes$-Divergence Free] \label{def:DivFree-TangGradients}
	A field $u\in\pltwo{\ddom}{\dddmes}^3$ is tangentially divergence free with respect to $\dddmes$ if (and only if)
	\begin{align*}
		\ip{u}{\ktgrad_{\dddmes}v}_{\ltwo{\ddom}{\dddmes}^3} = \integral{\ddom}{ u\cdot\overline{\ktgrad_{\dddmes}v} }{\dddmes} = 0,
	\end{align*}
	for every $v\in\ktgradSob{\ddom}{\dddmes}$.
\end{definition}

As one might expect from the differences between the definitions \ref{def:DivFree-AllGradients} and \ref{def:DivFree-TangGradients}, a characterisation of tangentially divergence free can be obtained by recycling the proof of proposition \ref{prop:DivFree-AllGradsConditions}.
Indeed, this characterisation retains precisely the conditions (ii), (iii), and (v):
\begin{cory} \label{cory:DivFree-TangGradsConditions}
	Let $u\in\pltwo{\ddom}{\dddmes}^3$ and define (for each edge $I_{jk}$)
	\begin{align*}
		U^{(jk)} = R_{jk}\begin{pmatrix} u_1^{(jk)} \\ u_2^{(jk)} \end{pmatrix}, 
		\qquad \qm_{jk} = \qm\cdot e_{jk}, 
		\qquad \widetilde{U}^{(jk)} = U^{(jk)}\circ r_{jk}.
	\end{align*}		
	Then $u$ is tangentially divergence-free if and only if
	\begin{align*}
		\mathrm{(ii)} \ & \widetilde{U}_2^{(jk)}\in\gradSob{\sqbracs{0,l_{jk}}}{y}, \quad \forall I_{jk}\in\edgeSet, \\
		\mathrm{(iii)} \ & \bracs{U_2^{(jk)}}' + \rmi\qm_{jk}U_2^{(jk)} + \rmi\wavenumber u_3^{(jk)} = 0 \text{ on } I_{jk}, \quad \forall I_{jk}\in\edgeSet, \\
		\mathrm{(v)} \ & \sum_{j\conRight k}U_2^{(kj)}\bracs{v_j} - \sum_{j\conLeft k}U_2^{(jk)}\bracs{v_j} = \rmi\wavenumber\alpha_j u_3\bracs{v_j}, \quad \forall v_j\in\vertSet,
	\end{align*}
	where $\bracs{U_2^{(jk)}}' = \bracs{\widetilde{U}_2^{(jk)}}'\circ r_{jk}^{-1}$.
\end{cory}
We purposefully retain the same labels as proposition \ref{prop:DivFree-AllGradsConditions}.
Similarly to how tangential gradients and curls only encode information about changes of functions or fields in the plane, a field being tangentially divergence free only imposes balance between the changes (in their respective directions) of the in-plane components of a vector field $u$.

This brings us closer to the idea of defining a ``weak divergence" in a similar manner to that of gradients and curls in section \ref{sec:BorelMeasSobSpaces}.
If we bring each of the terms in (v) over to the left-hand side, we now have a function $F\in\pltwo{\ddom}{\dddmes}$ with
\begin{align} \label{eq:ktDivergence-Guess}
	F(x) &= 
	\begin{cases} 
		\bracs{U_2^{(jk)}}'(x) + \rmi\qm_{jk}U_2^{(jk)}(x) + \rmi\wavenumber u_3(x) &
		x\in I_{jk}\setminus\clbracs{v_j,v_k}, \\
		\sum_{j\conRight k}U_2^{(kj)}\bracs{v_j} - \sum_{j\conLeft k}U_2^{(jk)}\bracs{v_j} -\rmi\wavenumber\alpha_j u_3\bracs{v_j} &
		x=v_j\in\vertSet,
	\end{cases}
\end{align}
and that suggestively satisfies $u$ is tangentially divergence free if and only if $F=0$.
Indeed, we could elect to define the $\kt$-divergence with respect to $\dddmes$ of a vector field $u$ in the following manner:
\begin{definition}[$\kt$-Divergence (with respect to $\dddmes$)] \label{def:dddmesDivergence}
	Let $u\in\pltwo{\ddom}{\dddmes}^3$.
	If there exists a function $d\in\pltwo{\ddom}{\dddmes}$ such that
	\begin{align*}
		\integral{\ddom}{ u\cdot\overline{\ktgrad_{\dddmes}\phi} }{\dddmes}
		&= -\integral{\ddom}{ d\overline{\phi} }{\dddmes}, \qquad\forall\phi\in\psmooth{\ddom},
	\end{align*}
	we say that $u$ admits (or possesses) a weak $\kt$-divergence with respect to $\dddmes$.
	Then we call $d:=\ktgrad_{\dddmes}\cdot u$ the $\kt$-divergence with respect to $\dddmes$ of the vector field $u$.
	If $\ktgrad_{\dddmes}\cdot u=0$, then $u$ is weakly divergence free.
\end{definition}
Clearly we could produce a similar definition for any Borel measure $\rho$ on $\ddom$.
The notions of $\kt$-tangentially divergence free (definition \ref{def:DivFree-TangGradients}) and weakly $\kt$-divergence free as in definition \ref{def:dddmesDivergence} coincide.
Furthermore, the weak $\kt$-divergence of a field $u$ is unique --- if there are two such divergences, the difference integrated against all $\phi\in\psmooth{\ddom}$ must be zero, from which we can deduce the difference is zero almost everywhere.
And finally, we can conclude that the weak $\kt$-divergence has the same form as the function $F$ in \eqref{eq:ktDivergence-Guess}, up to the position of the constant $\alpha_j$.
\begin{cory}
	Let $u\in\pltwo{\ddom}{\dddmes}^3$, and define (for each edge $I_{jk}$)
	\begin{align*}
		U^{(jk)} = R_{jk}\begin{pmatrix} u_1^{(jk)} \\ u_2^{(jk)} \end{pmatrix}, 
		\qquad \qm_{jk} = \qm\cdot e_{jk}, 
		\qquad \widetilde{U}^{(jk)} = U^{(jk)}\circ r_{jk}.
	\end{align*}	
	Then
	\begin{enumerate}[(i)]
		\item If $u$ admits a $\kt$-divergence, then $\widetilde{U}_2^{(jk)}\in\ktgradSob{\sqbracs{0,l_{jk}}}{y}$ for each $I_{jk}$ and 
		\begin{align} \label{eq:dddmesDivergence-Form}
			\ktgrad_{\dddmes}\cdot u &= 
			\begin{cases} 
				\bracs{U_2^{(jk)}}' + \rmi\qm_{jk}U_2^{(jk)} + \rmi\wavenumber u_3 &
				x\in I_{jk}\setminus\clbracs{v_j,v_k}, \\
				-\recip{\alpha_j}\bracs{\sum_{j\conRight k}U_2^{(kj)}\bracs{v_j} - \sum_{j\conLeft k}U_2^{(jk)}\bracs{v_j}} + \rmi\wavenumber u_3\bracs{v_j} &
				x=v_j\in\vertSet,
			\end{cases}
		\end{align}
		where $\bracs{U_2^{(jk)}}' = \bracs{\widetilde{U}_2^{(jk)}}'\circ r_{jk}^{-1}$.
		\item If $\widetilde{U}_2^{(jk)}\in\ktgradSob{\sqbracs{0,l_{jk}}}{y}$ for each $I_{jk}$, then $u$ admits a $\kt$-divergence $\ktgrad_{\dddmes}\cdot u$ defined as in \eqref{eq:dddmesDivergence-Form}.
	\end{enumerate}
\end{cory}
The proof of this corollary follows the same arguments as those for the conditions (ii), (iii), and (v) in proposition \ref{prop:DivFree-AllGradsConditions}.

\tstk{now do the de Rham stuff. Also, the curl-curl equation automatically gives us the other, weak definitions of divergence free to play with, so de Rham is a kind of tie-breaker in this respect.}