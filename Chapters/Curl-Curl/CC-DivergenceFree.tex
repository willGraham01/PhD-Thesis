\section{The Divergence-Free Condition} \label{sec:DivFreeCondition}
\tstk{introductory paragraph}

As was mentioned in \tstk{the introduction}, the definition of the Maxwell operator requires us to pay respect to the divergence-free nature of the electric and magnetic fields.
The fields that solve the curl of the curl equation are also required to be divergence free, and indeed one usually finds the curl of the curl equation posed on a space consisting of divergence-free fields.
Given the motivation for our consideration of the curl of the curl equation on a singular structure, it is natural for us to consider what the analogue of a divergence free field would be.
We find ourselves presented with several possibilities for the definition of a divergence free field, however none of these \tstk{complete train of thought - also, check whether we put a definition of divergence free in the introduction, because we might want to ret-con that}.

\subsection{Orthogonality to Gradients} \label{ssec:DivFree-OrthToAllGradients}
A characterising feature of divergence-free fields is that they are orthogonal to all potential vector fields; that is any vector field equal to the gradient of a scalar function.
Through this analogue, we can provide the following definition for our analogue of divergence-free fields:
\begin{definition}[Divergence Free (all Gradients)] \label{def:DivFree-AllGradients}
	A field $u\in\pltwo{\ddom}{\dddmes}^3$ is divergence free (with respect to $\dddmes$) if (and only if)
	\begin{align*}
		\ip{u}{g}_{\ltwo{\ddom}{\dddmes}^3} = \integral{\ddom}{ u\cdot\overline{g} }{\dddmes} = 0,
	\end{align*}
	for every $g\in W_{\mathrm{grad}}^{\kt}\bracs{\ddom,\md\dddmes}$.
\end{definition}
That is, a field is divergence free only if it is orthogonal in $L^2$ to any gradient field, including the gradients of zero.

This allows for the rather quick characterisation of divergence-free vector fields.
\begin{prop}
	Let $u\in\pltwo{\ddom}{\dddmes}^3$ and define (for each edge $I_{jk}$)
	\begin{align*}
		U^{(jk)} = R_{jk}\begin{pmatrix} u_1^{(jk)} \\ u_2^{(jk)} \end{pmatrix}, 
		\qquad \qm_{jk} = \qm\cdot e_{jk}, 
		\qquad \widetilde{U}^{(jk)} = U^{(jk)}\circ r_{jk}.
	\end{align*}		
	Then $u$ is divergence-free (in the sense of definition \ref{def:DivFree-AllGradients}) if and only if
	\begin{align*}
		\mathrm{(i)} \ & U_1^{(jk)} = 0, \quad \forall I_{jk}\in\edgeSet \\
		\mathrm{(ii)} \ & \widetilde{U}_2^{(jk)}\in\gradSob{\sqbracs{0,l_{jk}}}{y}, \quad \forall I_{jk}\in\edgeSet, \\
		\mathrm{(iii)} \ & \bracs{U_2^{(jk)}}' + \rmi\qm_{jk}U_2^{(jk)} + \rmi\wavenumber u_3^{(jk)} = 0 \text{ on } I_{jk}, \quad \forall I_{jk}\in\edgeSet, \\
		\mathrm{(iv)} \ & u_1\bracs{v_j} = u_2\bracs{v_j} = 0, \quad \forall v_j\in\vertSet, \\
		\mathrm{(v)} \ & \sum_{j\conRight k}U_2^{(jk)}\bracs{v_j} - \sum_{j\conLeft k}U_2^{(jk)}\bracs{v_j} = \rmi\wavenumber\alpha_j u_3\bracs{v_j}, \quad \forall v_j\in\vertSet,
	\end{align*}
	where $\bracs{U_2^{(jk)}}' = \bracs{\widetilde{U}_2^{(jk)}}'\circ r_{jk}^{-1}$.
\end{prop}
\begin{proof}
	First assume that the field $u$ is divergence-free (in the sense of definition \ref{def:DivFree-AllGradients}).
	\begin{enumerate}[(i)]
		\item We first utilise theorem \ref{thm:3DdddmesCharGradZero} to deduce that $\ip{u}{g}_{\ltwo{\ddom}{\lambda_{jk}}}=0$ for every $g\in\gradZero{\ddom}{\lambda_{jk}}$.
		Then since we know that $\gradZero{\ddom}{\lambda_{jk}}$ consists of all functions of the form $g_{jk}\hat{n}_{jk}$ (proposition \ref{prop:3DGradZeroChar}), we must conclude that
		\begin{align*}
			0 &= \integral{\ddom}{ \overline{g}_{jk}\begin{pmatrix} u_1^{(jk)} \\ u_2^{(jk)} \end{pmatrix}\cdot n_{jk} }{\lambda_{jk}}, \qquad \forall g_{jk}\in\pltwo{\ddom}{\lambda_{jk}},
		\end{align*}
		so $U_1^{(jk)}=0$ on $I_{jk}$.
		\item Let $\phi\in\csmooth{\sqbracs{0,l_{jk}}}$.
		Utilising suitable cut-off functions, we can then construct a function $\varphi\in\csmooth{\ddom}\cap\psmooth{\ddom}$ such that $\varphi(x)=\phi\bracs{r_{jk}^{-1}(x)}$ when $x\in I_{jk}$, and with $\varphi=0$ on all other parts of $\graph$.
		Since such $\varphi$ are smooth, they are elements of $\ktgradSob{\ddom}{\lambda_{jk}}$ and by lemma \ref{lem:ExtensionLemmaEdgeFunctions} also elements of $\ktgradSob{\ddom}{\dddmes}$.
		As such, we have that
		\begin{align*}
			0 &= \integral{\ddom}{ u\cdot\overline{\ktgrad_{\dddmes}\varphi} }{\dddmes}
			= \integral{I_{jk}}{ u\cdot\bracs{ \bracs{\overline{\varphi}' - \rmi\qm_{jk}\overline{\varphi}}\widehat{e}_{jk} - \rmi\wavenumber\overline{\varphi}\widehat{x}_3} }{\lambda_{jk}} \\
			&= \integral{I_{jk}}{ \bracs{\overline{\varphi}' - \rmi\qm_{jk}\overline{\varphi}}U_2^{(jk)} - \rmi\wavenumber\overline{\varphi}u_3^{(jk)} }{\lambda_{jk}} \\
			&= \int_0^{l_{jk}} \bracs{\overline{\phi}' - \rmi\qm_{jk}\overline{\phi}}\widetilde{U}_2^{(jk)} - \rmi\wavenumber\overline{\phi}\widetilde{u}_3^{(jk)} \ \md y, \\
			\implies
			\int_0^{l_{jk}} \overline{\phi}'\widetilde{U}_2^{(jk)} \ \md y
			&= \int_0^{l_{jk}} \overline{\phi}\bracs{ \rmi\qm_{jk}\widetilde{U}_2^{(jk)} + \rmi\wavenumber\widetilde{u}_3^{(jk)} } \ \md y.
		\end{align*}
		This holds for every $\phi\in\csmooth{\sqbracs{0,l_{jk}}}$, and so we can conclude that $\widetilde{U}_2^{(jk)}\in\gradSob{\sqbracs{0,l_{jk}}}{y}$.
		\item Following immediately on from the above, we have that
		\begin{align*}
			\bracs{U_2^{(jk)}}' = - \bracs{ \rmi\qm_{jk}U_2^{(jk)} + \rmi\wavenumber u_3^{(jk)} },
		\end{align*}
		which upon rearrangement gives the result.
		\item We again notice through theorem \ref{thm:3DdddmesCharGradZero} we have that $\ip{u}{g}_{\ltwo{\ddom}{\massMes}}=0$ for every $g\in\gradZero{\ddom}{\massMes}$.
		Fix $v_j\in\vertSet$, and take $g=\bracs{g_1,g_2,0}^\top$ at $v_j$ and zero elsewhere.
		Given the characterisation in proposition \ref{prop:NuGradZeroChar}, we can conclude that
		\begin{align*}
			0 &= \integral{\ddom}{ u_\cdot\overline{g} }{\massMes}
			= \alpha_j\bracs{\overline{g_1}u_1 + \overline{g_2}u_2},
		\end{align*}
		for all $g_1,g_2\in\complex$.
		Therefore, we must have that $u_1\bracs{v_j}=u_2\bracs{v_j}=0$.
		\item Finally, take $\varphi\in\csmooth{\ddom}$ with support
		\begin{align*}
			\supp\bracs{\varphi} \subset \mathcal{J}\bracs{v_j}\setminus\clbracs{v_k\in\vertSet \setVert v_k\neq v_j}.
		\end{align*}
		Since this $\varphi$ is smooth, it is an element of $\ktgradSob{\ddom}{\dddmes}$ by definition, and we have that
		\begin{align*}
		0 &= \integral{\ddom}{ u\cdot\overline{\ktgrad_{\dddmes}\varphi} }{\dddmes} \\
		&= \sum_{j\con k}\integral{I_{jk}}{ U_2^{(jk)}\bracs{\overline{\varphi}' - \rmi\qm_{jk}\overline{\varphi}} - \rmi\wavenumber u_3^{(jk)}\overline{\varphi} }{\lambda_{jk}} 
		+ \integral{\ddom}{ u\cdot\overline{\ktgrad_{\dddmes}\varphi} }{\massMes} \\
		&= \sum_{j\con k}\int_0^{l_{jk}} \widetilde{U}_2^{(jk)}\bracs{\overline{\widetilde{\varphi}}' - \rmi\qm_{jk}\overline{\widetilde{\varphi}}} - \rmi\wavenumber u_3^{(jk)}\overline{\widetilde{\varphi}} \ \md y
		-\rmi\wavenumber\alpha_j u_3\bracs{v_j}\overline{\varphi}\bracs{v_j}.
		\end{align*}
		Rearranging and using (ii) and (iii), we find that
		\begin{align*}
			\rmi\wavenumber\alpha_j u_3\bracs{v_j}\overline{\varphi}\bracs{v_j}
			&= - \sum_{j\con k}\int_0^{l_{jk}} \overline{\widetilde{\varphi}}\bracs{ \widetilde{U}_2^{(jk)} + \rmi\qm_{jk}\widetilde{U}_2^{(jk)} + \rmi\wavenumber\widetilde{u}_3^{(jk)} } \ \md y \\
			&\quad + \overline{\varphi}\bracs{v_j}\bracs{ \sum_{j\conRight k} U_2^{(jk)}\bracs{v_j} - \sum_{j\conLeft k} U_2^{(jk)}\bracs{v_j} } \\
			&= \overline{\varphi}\bracs{v_j}\bracs{ \sum_{j\conRight k} U_2^{(kj)}\bracs{v_j} - \sum_{j\conLeft k} U_2^{(jk)}\bracs{v_j} }.
		\end{align*}
		This holds for every such smooth $\varphi$, and so we must have that
		\begin{align*}
			\rmi\wavenumber\alpha_j u_3\bracs{v_j} &= \sum_{j\conRight k} U_2^{(kj)}\bracs{v_j} - \sum_{j\conLeft k} U_2^{(jk)}\bracs{v_j}.
		\end{align*}
	\end{enumerate}
	
	Now let us suppose that (i)-(v) hold.
	Conditions (i) and (iv) ensure that $u$ is orthogonal to $\gradZero{\ddom}{\dddmes}$ via theorem \ref{thm:3DdddmesCharGradZero}.
	Now let $v\in\ktgradSob{\ddom}{\dddmes}$, and consider an approximating sequence $\phi_n$ for $v$.
	Clearly,
	\begin{align*}
		\ip{u}{\ktgrad_{\dddmes}v}_{\ltwo{\ddom}{\dddmes}^3}
		&= \lim_{n\rightarrow\infty}\ip{u}{\ktgrad\phi_n}_{\ltwo{\ddom}{\dddmes}^3},
	\end{align*}
	from which we can use (ii), (iii), and (v) to deduce that $\ip{u}{\ktgrad\phi_n}_{\ltwo{\ddom}{\dddmes}^3}=0$ for every $n\in\naturals$, providing orthogonality to tangential gradients.
\end{proof}