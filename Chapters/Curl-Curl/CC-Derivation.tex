\section{Derivation of quantum graph problem} \label{sec:3DSystemDerivation}
We now look to find an alternative formulation to \eqref{eq:SingularCurlEquation}, which is more amenable to analysis.
Similarly to our working in section \ref{sec:ScalarDerivation}, the result will be a problem consisting of ODEs on the edges of $\graph$ coupled through the behaviour of incoming functions at the vertices.

Let $u=\bracs{u_1,u_2,u_3}^\top\in\ktcurlSob{\ddom}{\dddmes}$ and let $\Phi=\bracs{\phi_1,\phi_2,\phi_3}^\top\in\psmooth{\ddom}^3$.
Also define for each $I_{jk}\in\edgeSet$,
\begin{align*}
	U^{(jk)} := R_{jk} \begin{pmatrix} u_1^{(jk)} \\ u_2^{(jk)} \end{pmatrix},
	\qquad
	\Psi^{(jk)} := R_{jk} \begin{pmatrix} \phi_1^{(jk)} \\ \phi_2^{(jk)} \end{pmatrix}.
\end{align*}
We use an overhead tilde to denote composition with $r_{jk}$, and for a function $v$ with $\widetilde{v}^{(jk)}\in\ktgradSob{\sqbracs{0,l_{jk}}}{y}$ write $\bracs{v^{(jk)}}' := \bracs{\widetilde{v}^{(jk)}}' \circ r_{jk}^{-1}$.
Finally, for a given quasi-momentum $\qm$, set $\qm_{jk} = \bracs{ R_{jk}\qm }_2 = \qm\cdot e_{jk}$.
Immediately, we can notice that $u$ is tangentially-divergence free (definition \ref{def:DivFree-TangGradients}), since for any $v\in\ktgradSob{\ddom}{\dddmes}$ we can take an approximating sequence $\Phi^n$ for $v$ and obtain
\begin{align*}
	\omega^2\integral{\ddom}{ u\cdot\overline{\ktgrad_{\dddmes}v} }{\dddmes}
	&= \omega^2 \lim_{n\rightarrow\infty} \integral{\ddom}{ u\cdot\overline{\ktgrad\Phi^n} }{\dddmes} \\
	&= \lim_{n\rightarrow\infty} \integral{\ddom}{ \ktcurl{\dddmes}u\cdot\overline{\ktcurl{\dddmes}\bracs{\ktgrad\Phi^n}} }{\dddmes}
	= 0,
\end{align*}
through using lemma \ref{lem:CurlOfGradSmoothFunctions}.
Note that we cannot employ the same technique show that $u$ is orthogonal to all gradients of zero,, however we will see that the additional information (the conditions (i) and (iv) in proposition \ref{prop:DivFree-AllGradsConditions}) are obtained from the argument that follows.

Now let us fix an edge $I_{jk}$ and take $\Phi$ such that $\supp\Phi\cap\graph\subset I_{jk}^{\circ}$.
The equation \eqref{eq:SingularCurlEquation-VariationalForm} reduces to
\begin{align*}
	\omega^2 \integral{I_{jk}}{ u\cdot\overline{\Phi} }{\lambda_{jk}}
	&= \integral{I_{jk}}{ \ktcurl{\ddmes}u\cdot\overline{\ktcurl{\ddmes}\Phi} }{\lambda_{jk}} \\
	&= \integral{I_{jk}}{ \bracs{ \bracs{u_3^{(jk)}}' + \rmi\qm_{jk} u_3^{(jk)} - \rmi\wavenumber U_2^{(jk)} }\overline{\bracs{ \bracs{\phi_3^{(jk)}}' + \rmi\qm_{jk} \phi_3^{(jk)} - \rmi\wavenumber \Psi_2^{(jk)} }} }{\lambda_{jk}}.
\end{align*}
Using the change of variables $r_{jk}$ we obtain
\begin{align*}
	\int_0^{l_{jk}} & \bracs{ \bracs{\widetilde{u}_3^{(jk)}}' + \rmi\qm_{jk} \widetilde{u}_3^{(jk)} - \rmi\wavenumber \widetilde{U}_2^{(jk)} } \bracs{ \bracs{\overline{\widetilde{\phi}}_3^{(jk)}}' - \rmi\qm_{jk} \overline{\widetilde{\phi}}_3^{(jk)} + \rmi\wavenumber \overline{\widetilde{\Psi}}_2^{(jk)} } \ \md y \\
	&= \omega^2 \int_0^{l_{jk}} \widetilde{U}_1^{(jk)}\overline{\widetilde{\Psi}}_1^{(jk)} + \widetilde{U}_2^{(jk)}\overline{\widetilde{\Psi}}_2^{(jk)} + \widetilde{u}_3^{(jk)}\overline{\widetilde{\phi}}_3^{(jk)} \ \md y.
\end{align*}
Now suppose we have $\varphi_1,\varphi_2\in\csmooth{\sqbracs{0,l_{jk}}}$.
Since $R_{jk}\in\mathrm{SO}(2)$, the system
\begin{align*}
	\begin{pmatrix} \phi_1 \\ \phi_2 \end{pmatrix} &= R_{jk}^\top \begin{pmatrix} \varphi_1 \\ \varphi_2 \end{pmatrix}
\end{align*}
has a unique solution for $\bracs{\phi_1,\phi_2}^\top$, and we can construct a $\Phi\in\csmooth{\ddom}$ with $\supp\Phi\cap\graph\subset I_{jk}^{\circ}$, $\widetilde{\Psi}_1^{(jk)}=\varphi_1$, and $\widetilde{\Psi}_2^{(jk)}=\varphi_2$.
We thus observe that
\begin{align*}
	\int_0^{l_{jk}} & \bracs{ \bracs{\widetilde{u}_3^{(jk)}}' + \rmi\qm_{jk} \widetilde{u}_3^{(jk)} - \rmi\wavenumber \widetilde{U}_2^{(jk)} } \bracs{ \overline{\varphi_3}' - \rmi\qm_{jk} \overline{\varphi_3} + \rmi\wavenumber \overline{\varphi_2} } \ \md y \\
	&= \omega^2 \int_0^{l_{jk}} \widetilde{U}_1^{(jk)}\overline{\varphi_1} + \widetilde{U}_2^{(jk)}\overline{\varphi_2} + \widetilde{u}_3^{(jk)}\overline{\varphi_3} \ \md y
\end{align*}
holds for any $\varphi_1,\varphi_2,\varphi_3\in\csmooth{\sqbracs{0,l_{jk}}}$.
Therefore it must hold that for any $\varphi\in\csmooth{\sqbracs{0,l_{jk}}}$,
\begin{subequations}
	\begin{align*}
		0 &= \int_0^{l_{jk}} \overline{\varphi} \widetilde{U}_1^{(jk)} \ \md y, \labelthis\label{eq:CurlCurlWeakFormPhi1} \\
		0 &= \int_0^{l_{jk}} \overline{\varphi} \bracs{ \rmi\wavenumber\bracs{\widetilde{u}_3^{(jk)}}' + \bracs{\wavenumber^2 - \omega^2}\widetilde{U}_2^{(jk)} - \wavenumber\qm_{jk}\widetilde{u}_3^{(jk)}  } \ \md y, \labelthis\label{eq:CurlCurlWeakFormPhi2} \\
		0 &= \int_0^{l_{jk}} \overline{\varphi}' \bracs{ \bracs{\widetilde{u}_3^{(jk)}}'
		- \rmi\wavenumber\widetilde{U}_2^{(jk)} + \rmi\qm_{jk}\widetilde{u}_3^{(jk)} } \\
		&\qquad -\rmi\qm_{jk}\overline{\varphi}\bracs{ \bracs{\widetilde{u}_3^{(jk)}}' - \rmi\wavenumber\bracs{\widetilde{U}_2^{(jk)}}' + \rmi\qm_{jk}\widetilde{u}_3^{(jk)} }
		- \omega^2 \widetilde{u}_3^{(jk)}\overline{\varphi} \ \md y. \labelthis\label{eq:CurlCurlWeakFormPhi3}
	\end{align*}
\end{subequations}
The equation \eqref{eq:CurlCurlWeakFormPhi1} implies that $U_1^{(jk)}=0$ (almost everywhere) on $I_{jk}$, whilst \eqref{eq:CurlCurlWeakFormPhi2} implies that
\begin{align*}
	\rmi\wavenumber\bracs{\diff{}{y} + \rmi\qm_{jk}}\widetilde{u}_3^{(jk)} + \wavenumber^2\widetilde{U}_2^{(jk)} &= \omega^2\widetilde{U}_2^{(jk)},
\end{align*}
on $I_{jk}$.
Given that $u$ is divergence-free, we know that $\widetilde{U}_2^{(jk)}$ is weakly differentiable in the $\gradSob{\sqbracs{0,l_{jk}}}{y}$-sense and can manipulate \eqref{eq:CurlCurlWeakFormPhi3} to obtain
\begin{align*}
	\int_0^{l_{jk}} \overline{\varphi}' \widetilde{u}'_{3,jk} \ \md y
	&= -\int_0^{l_{jk}} \overline{\psi} \bracs{ \rmi\wavenumber\widetilde{U}'_{2,jk} - \wavenumber\qm_{jk}\widetilde{U}_{2,jk} - 2\rmi\qm_{jk}\widetilde{u}'_{3,jk} + \qm_{jk}^2\widetilde{u}_{3,jk} - \omega^2\widetilde{u}_{3,jk} } \ \md y.,
\end{align*}
for all $\varphi\in\csmooth{\sqbracs{0,l_{jk}}}$.
Thus we can conclude that $\widetilde{u}_3\in\gradgradSob{\sqbracs{0,l_{jk}}}{y}$, from which we deduce that
\begin{align*}
	-\bracs{ \diff{}{y} + \rmi\qm_{jk} }^2\widetilde{u}_3^{(jk)} + \rmi\wavenumber\bracs{ \diff{}{y} + \rmi\qm_{jk} }\widetilde{U}_2^{(jk)} &= \omega^2 \widetilde{u}_3^{(jk)}
\end{align*}
on $I_{jk}$.
This provides us with a system of coupled ODEs on each of the edges $I_{jk}$ (and the information that the component $\widetilde{U}_1^{(jk)}=0$ on each edge).

Now we look at how these edge ODEs are coupled at the vertices.
Fix a vertex $v_j$ and consider a $\Phi\in\csmooth{\ddom}$ whose support contains $v_j$ in its interior, and no other vertices of $\graph$.
Testing against such $\Phi$ in \eqref{eq:SingularCurlEquation-VariationalForm} implies that
\begin{align*}
	\alpha_j\omega^2 u(v_j)\cdot\overline{\Phi}(v_j)
	&= \integral{\ddom}{ \ktcurl{\ddmes}u\cdot\overline{\ktcurl{\ddmes}\Phi} - \omega^2 u\cdot\overline{\Phi} }{\ddmes} \\
	&= \sum_{j\con k}\integral{I_{jk}}{ \bracs{ \bracs{u_3^{(jk)}}' + \rmi\qm_{jk}u_3^{(jk)} - \rmi\wavenumber U_2^{(jk)}} \bracs{ \bracs{\overline{\phi}_3^{(jk)}}' - \rmi\qm_{jk}\overline{\phi}_3^{(jk)} + \rmi\wavenumber \overline{\Psi}_2^{(jk)}} \\
	&\qquad - \omega^2 U_1^{(jk)}\overline{\Psi}_1 - \omega^2 U_2^{(jk)}\overline{\Psi}_2 - \omega^2 u_3^{(jk)}\overline{\phi}_3 }{\lambda_{jk}} \\
	&= \sum_{j\con k}\int_0^{l_{jk}} \bracs{ \bracs{\widetilde{u}_3^{(jk)}}' + \rmi\qm_{jk}\widetilde{u}_3^{(jk)} - \rmi\wavenumber \widetilde{U}_2^{(jk)}} \bracs{ \bracs{\overline{\widetilde{\phi}}_3^{(jk)}}' - \rmi\qm_{jk}\overline{\widetilde{\phi}}_3^{(jk)} + \rmi\wavenumber \overline{\widetilde{\Psi}}_2^{(jk)}} \\
	&\qquad - \bracs{\bracs{\widetilde{u}_3^{(jk)}}' + \rmi\qm_{jk}\widetilde{u}_3^{(jk)} - \rmi\wavenumber \widetilde{U}_2^{(jk)}}\rmi\wavenumber\overline{\widetilde{\Psi}_2^{(jk)}} - \omega^2 \widetilde{u}_3^{(jk)}\overline{\widetilde{\phi}}_3 \ \md y \\
	&= -\sum_{j\con k}\int_0^{l_{jk}} \overline{\widetilde{\phi}}_3
	\sqbracs{ \bracs{\widetilde{u}_3^{(jk)}}'' + 2\rmi\qm_{jk}\bracs{\widetilde{u}_3^{(jk)}}' + \bracs{\rmi\qm_{jk}}^2\widetilde{u}_3^{(jk)} + \omega^2\widetilde{u}_3^{(jk)} } \\
	&\qquad - \overline{\widetilde{\phi}}_3\sqbracs{\rmi\wavenumber\bracs{ \bracs{\widetilde{U}_2^{(jk)}}' + \rmi\qm_{jk}\widetilde{U}_2^{(jk)} } } \ \md y \\
	&\qquad + \overline{\phi}_3(v_j)\sqbracs{ \sum_{j\con k}\bracs{\pdiff{}{n}+\rmi\qm_{jk}}u_3^{(jk)}(v_j) + \sum_{j\conRight k}U_2^{(kj)}(v_j) - \sum_{j\conLeft k}U_2^{(jk)}(v_j) } \\
	&= \overline{\phi}_3(v_j)\sqbracs{ \sum_{j\con k}\bracs{\pdiff{}{n}+\rmi\qm_{jk}}u_3^{(jk)}(v_j) + \sum_{j\conRight k}U_2^{(kj)}(v_j) - \sum_{j\conLeft k}U_2^{(jk)}(v_j) },
\end{align*}
where $U_2^{(jk)}(v_j)$ is the trace of $U_2^{(jk)}$ to the vertex $v_j$.
Identifying that this holds for all $\phi_1(v_j), \phi_2(v_j), \phi_3(v_j)\in\complex$, we must conclude that
\begin{align*}
	u_1(v_j) &= 0, \\
	u_2(v_j) &= 0, \\
	\alpha_j\omega^2u_3(v_j) &= \sum_{j\con k}\bracs{\pdiff{}{n}+\rmi\qm_{jk}}u_3^{(jk)}(v_j) + \sum_{j\conRight k}U_2^{(kj)}(v_j) - \sum_{j\conLeft k}U_2^{(jk)}(v_j).
\end{align*}
We are thus presented with the following set of equations on each edge $I_{jk}$;
\begin{subequations} \label{eq:QGRawSystem}
	\begin{align*}
		\widetilde{U}_1^{(jk)} &= 0 \\
		\rmi\wavenumber \bracs{ \diff{}{y} + \rmi\qm_{jk} }\widetilde{u}_3^{(jk)} + \wavenumber^2\widetilde{U}_2^{(jk)} &= \omega^2\widetilde{U}_2^{(jk)}, \labelthis\label{eq:QGPhi2} \\
		-\bracs{ \diff{}{y} + \rmi\qm_{jk} }^2\widetilde{u}_3^{(jk)} + \rmi\wavenumber\bracs{ \diff{}{y} + \rmi\qm_{jk} }\widetilde{U}_2^{(jk)} &= \omega^2 \widetilde{u}_3^{(jk)}, \labelthis\label{eq:QGPhi3} 
	\end{align*}
	complemented by the vertex conditions
	\begin{align*}
		\widetilde{u}_3 \text{ is continuous at } v_j &\quad\forall v_j\in\vertSet, \labelthis\label{eq:QGContinuity} \\
		u_1(v_j) &= 0, \\
		u_2(v_j) &= 0, \\
		\alpha_j \omega^2 u_3\bracs{v_j}
		&= \sum_{j\con k} \bracs{ \pdiff{}{n} + \rmi\qm_{jk} }u_3^{(jk)}\bracs{v_j} - \rmi\wavenumber\bracs{ \sum_{j\conRight k} U_2^{(jk)} - \sum_{j\conLeft k} U_2^{(jk)} }. \labelthis\label{eq:QGVertexCondition}
	\end{align*}
\end{subequations}
Determining the solution to \eqref{eq:QGRawSystem} requires us to find the function $u_3$ and the edge-functions $U_2^{(jk)}$.
However we can eliminate the $U_2^{(jk)}$ from the system \eqref{eq:QGRawSystem} via substitution of \eqref{eq:QGPhi2} into \eqref{eq:QGPhi3}, and use of corollary \ref{cory:DivFree-TangGradsConditions}(v), to obtain
\begin{subequations} \label{eq:CurlCurl-ScalarQGProblem}
	\begin{align}
		-\bracs{ \diff{}{y} + \rmi\qm_{jk} }^2 u_3^{(jk)} &= \bracs{\omega^2-\wavenumber^2}u,
		\qquad &\text{on each } I_{jk}, \\
		u_3 \text{ is continuous at } & v_j \qquad &\forall v_j\in\vertSet, \\
		\sum_{j\con k}\bracs{\pdiff{}{n}+\rmi\qm_{jk}}u_3^{(jk)}(v_j) &= \alpha_j\bracs{\omega^2-\wavenumber^2}u_3(v_j), \qquad &\forall v_j\in\vertSet.
	\end{align}
\end{subequations}
The system \eqref{eq:SingularCurlEquation} has reduced to the scalar system obtained in section \ref{sec:ScalarDerivation}, except with $\sqrt{\omega^2-\wavenumber^2}$ playing the role of the spectral parameter.
We have already surveyed and explored the methods one can employ for studying \eqref{eq:CurlCurl-ScalarQGProblem} in chapter \ref{ch:ScalarSystem}, so whilst we know how to solve this problem, there is nothing new to be done concerning this process (save extracting the band-gap plot for $\wavenumber$ and $\omega$).
To this end, the discussion of section \ref{sec:ScalarDiscussion} covers the techniques one can employ through the study of the $M$-matrix \tstk{although maybe make a nice pretty graph to show that it can be done?}.

It is both regrettable and surprising from an investigative perspective that our analysis of \eqref{eq:SingularCurlEquation} and the associated function spaces has resulted in us obtaining what is essentially the acoustic approximation.
Even when we do not consider oblique waves (for which $\kappa=0)$), we still find that we have this reduction to a scalar system, which does not occur for the classical Maxwell system or curl of the curl equation.
Having said this, we have ultimately still managed to obtain the solvable (and realisable) system \eqref{eq:CurlCurl-ScalarQGProblem} through the analysis of sections \ref{sec:CC-CurlAnalysis} and \ref{sec:DivFreeCondition}, and non-trivial manipulations of unfamiliar objects like $\ktcurl{\dddmes}u$.
This demonstrates that our approach to the postulation and solution of variational problems on singular structures is consistent, and provides a coherent framework for the handling of such problems.
We will continue to discuss the implications of \eqref{eq:CurlCurl-ScalarQGProblem}, and give an a posteriori understanding as to why we have obtained it, in section \ref{sec:CC-Discussion}.

\subsection{Remarks on the Calder\'on Operator} \label{ssec:CalderonOp}
Whilst \eqref{eq:SingularCurlEquation} reduces to the acoustic approximation, there are further parallels that can be drawn between the (classical) curl-of-the-curl problem and the system \eqref{eq:QGRawSystem}.
One of our key motivating factors and themes throughout this work has been retaining an intuitive link between the spaces and objects necessary for studying variational problems on singular structures and their classical counterparts.
Further to this, the Dirichlet-to-Neumann operator (\tstk{you likely said this in the theory section?}also called the $M$-operator) associated with the problem \eqref{eq:QGRawSystem} is a direct analogue of the Calder\'{o}n operator $\mathcal{C}$, the Dirichlet-to-Neumann operator for the curl of the curl equation.
The Dirichlet and Neumann maps for the classical curl of the curl problem motivate the definitions for their counterparts associated to the problem \eqref{eq:QGRawSystem}; and we will observe that the resulting maps form a boundary triple, providing us with an $M$-operator and thus a graph analogue of $\mathcal{C}$.

\tstk{might be the first time we talk about this, might move it into the $M$-matrix section of QGs, or even into the introductory section of this chapter if it reads better. Also did we mention boundary triples before??}
Let us quickly introduce the Calder\'{o}n operator; let $D\subset\reals^3$ be a domain, and define $\mathcal{A}$ as the curl of the curl operator
\begin{align*}
	\dom\bracs{\mathcal{A}} &= \clbracs{ u\in H^2_{\mathrm{curl}}(D) \setVert \hat{n}\wedge\curl{}u\vert_{\partial D} = m }, \\
	\mathcal{A}u &= \curl{}\bracs{\curl{}u},
\end{align*}
and the Dirichlet and Neumann maps ($\dmap$ and $\nmap$ respectively) by
\begin{align} \label{eq:ClassicalEM-DNMaps}
	\dmap u = u\vert_{\partial D}, \qquad
	\nmap u = \hat{n}\wedge\curl{u}\vert_{\partial D}.
\end{align}
We can quickly validate Green's identity:
\begin{align*}
	\integral{D}{ \mathcal{A}u \cdot \overline{v} - u \cdot \overline{\mathcal{A}v} }{x}
	&= \integral{D}{ \curl{}\bracs{\curl{}u}\cdot\overline{v} - u\cdot\overline{\curl{}\bracs{\curl{}v}} }{x} \\
	&= \integral{D}{ \curl{}u\cdot\overline{\curl{}v} - \curl{}u\cdot\overline{\curl{}v} }{x} \\
	&\quad + \integral{\partial D}{ \hat{n}\wedge\curl{}u\cdot\overline{v} - u\cdot\hat{n}\wedge\overline{\curl{}v} }{S} \\
	&= \integral{\partial D}{ \nmap u \cdot \overline{\dmap v} - \dmap u \cdot \overline{\nmap v} }{S},
\end{align*}
so $\bracs{\ltwo{\partial D}{S}^3, \dmap, \nmap}$ is a boundary triple for the operator $\mathcal{A}$, and we can talk about its associated Dirichlet-to-Neumann map.
This is the Calder\'{o}n operator $\mathcal{C}$, acting as
\begin{align*}
	u\vert_{\partial D} \rightarrow \hat{n}\wedge\bracs{\curl{}u}\vert_{\partial D}.
\end{align*}

Now let us turn to the quantum graph problem \eqref{eq:QGRawSystem}.
Given the definition of $\dmap$ in \eqref{eq:ClassicalEM-DNMaps}, we should expect that
\begin{align} \label{eq:DGMapDef}
	\dgmap u &= 
	\begin{pmatrix}
		u\bracs{v_1} \\ u\bracs{v_2} \\ \vdots \\ u\bracs{v_N}
	\end{pmatrix}
	\in\complex^{3N},
\end{align}
where we have stacked the 3-vectors on top of each other and set $N=\abs{\vertSet}$.
As for $\ngmap u$, this should be the analogue of $\nmap$ in \eqref{eq:ClassicalEM-DNMaps} --- only now the boundary of our domain is the vertices of $\graph$.
Define the functions \tstk{might be worth moving into our usual setup assumption for ease of use?}
\begin{align*}
	\sgn_{jk}: \clbracs{v_j, v_k} \rightarrow \clbracs{-1,0,1}, 
	&\qquad
	\sgn_{jk}(x) = \begin{cases} -1 & x=v_j, \\ 1 & x=v_k, \end{cases}
	&\qquad
	\hat{\sigma}_{jk} &= \sgn_{jk}\widehat{e}_{jk},
\end{align*}
so $\hat{\sigma}_{jk}$ is the ``exterior normal" to the edge $I_{jk}$.
The classical action of $\nmap$ in \eqref{eq:ClassicalEM-DNMaps} then suggests the following candidate for $\ngmap$:
\begin{align} \label{eq:NGMapDef}
	\ngmap u &= 
	\begin{pmatrix}
		\sum_{1\con k} \hat{\sigma}_{1k}\bracs{v_1}\wedge\ktcurl{\dddmes}u\bracs{v_1} \\
		\sum_{2\con k} \hat{\sigma}_{2k}\bracs{v_2}\wedge\ktcurl{\dddmes}u\bracs{v_2} \\
		\vdots \\
		\sum_{N\con k} \hat{\sigma}_{Nk}\bracs{v_N}\wedge\ktcurl{\dddmes}u\bracs{v_N}
	\end{pmatrix}
	\in\complex^{3N},
\end{align}
where we have again stacked the 3-vectors vertically.
From our analysis of $\kt$-tangential curls (corollary \ref{cory:TangCurlEdgeRotated}), on each edge $I_{jk}$ we have
\begin{align*}
	\widehat{e}_{jk}\wedge\ktcurl{\dddmes}u &= \bracs{ \bracs{ u_3^{(jk)} }' + \rmi\qm_{jk}u_3^{(jk)} - \rmi\wavenumber U_2^{(jk)} }\widehat{x}_3,
\end{align*}
and noticing that 
\begin{align*}
	\sum_{j\conRight k} U_2^{(kj)}\bracs{v_j} - \sum_{j\conLeft k} U_2^{(jk)}\bracs{v_j} &=
	\sum_{j\con k} \sgn_{jk}(v_j) U_2^{(jk)}\bracs{v_j},
\end{align*} 
we have for each $v_j\in\vertSet$,
\begin{align*}
	\sum_{j\con k} \hat{\sigma}_{jk}\bracs{v_j} \wedge \ktcurl{\dddmes}u\bracs{v_j}
	&= \bracs{ 
	\rmi\wavenumber\sum_{j\con k} \sgn_{jk}(v_j) U_2^{(jk)}\bracs{v_j}
	- \sum_{j\con k}\bracs{\pdiff{}{n} + \rmi\qm_{jk}}u_3^{(jk)}\bracs{v_j}
	} \widehat{x}_3.
\end{align*}
We observe that the vertex conditions for the system \eqref{eq:QGRawSystem} can be written as
\begin{align} \label{eq:VertConditionExplicit}
	\sum_{j\con k} \hat{\sigma}_{jk}\bracs{v_j} \wedge \ktcurl{\dddmes}u\bracs{v_j}
	&= \alpha_j\omega^2 u\bracs{v_j},
	\qquad \forall v_j\in\vertSet.
\end{align}
We can identify \eqref{eq:VertConditionExplicit} as being of the form \eqref{eq:DispersiveBC} \tstk{does this appear in the theory or chapter 3?} where
\begin{align*}
	\tilde{\alpha} = 
	\mathrm{diag}\bracs{\alpha_1, \alpha_1, \alpha_1, \alpha_2, \alpha_2, \alpha_2, ..., \alpha_N, \alpha_N, \alpha_N} \in \complex^{3N\times 3N},
\end{align*}
and $\dgmap, \ngmap$ are as in \eqref{eq:DGMapDef}, \eqref{eq:NGMapDef}.
To complete the analogy, define the operator $\ag$ via the action
\begin{align*}
	\ag u &= 
	\begin{pmatrix}
		\sqbracs{ \rmi\wavenumber\bracs{\diff{}{y} + \rmi\qm_{jk} }u_3^{(jk)} + \wavenumber^2 U_2^{(jk)} }e_{jk}
		+ U_1^{(jk)} n_{jk} \\
		- \bracs{\diff{}{y} + \rmi\qm_{jk} }^2 u_3^{(jk)} + \rmi\wavenumber \bracs{\diff{}{y} + \rmi\qm_{jk} }U_2^{(jk)}
	\end{pmatrix}
\end{align*}
on each edge, where $\mathrm{dom}\bracs{\ag}$ consists of all functions $u$ with the following properties:
\begin{align*}
	u\in\mathrm{dom}\bracs{\ag} \quad\Leftrightarrow\quad &
	\begin{cases}
	u\in L^2\bracs{\graph}\times L^2\bracs{\graph}\times H^2\bracs{\graph}, \\
	U_2^{(jk)}\in \gradSob{I_{jk}}{y}, & \forall I_{jk}\in\edgeSet, \\
	u_3 \text{ is continuous at } v_j, & \forall v_j\in\vertSet, \\
	u_1(v_j) = u_2(v_j) = 0, & \forall v_j\in\vertSet, \\
	\text{\eqref{eq:QGVertexCondition} is satisfied at } v_j, & \forall v_j\in\vertSet.
	\end{cases}
\end{align*}
The eigenvalue problem for the operator $\ag$ is then \eqref{eq:QGRawSystem} and we have that
\begin{align*}
	\integral{I_{jk}}{ \ag u \cdot \overline{v} }{y} - \integral{I_{jk}}{ u \cdot \overline{\ag v} }{y}
	&= \sqbracs{ -u'_3 v_3 + u_3 v_3' - 2\rmi\qm_{jk}u_3 v_3 + \rmi\wavenumber\bracs{U_2 v_3 + u_3 V_2} }_{v_j}^{v_k} \\
	&= -\sqbracs{ \overline{v}_3\bracs{ \bracs{\diff{}{y} + \rmi\qm_{jk} }u_3 - \rmi\wavenumber U_2 } }_{v_j}^{v_k} \\
	&\qquad + \sqbracs{ u_3\overline{\bracs{ \bracs{\diff{}{y} + \rmi\qm_{jk} }v_3 - \rmi\wavenumber V_2 }} }_{v_j}^{v_k},
\end{align*}
implying
\begin{align*}
	&\ip{\ag u}{v}_{L^2\bracs{\graph}^3} - \ip{u}{\ag v}_{L^2\bracs{\graph}^3} \\
	&\quad = \sum_{v_j\in\vertSet}\sum_{j\conLeft k} \integral{I_{jk}}{ \ag u \cdot \overline{v} - u \cdot \overline{\ag v} }{y} \\
	&\quad = \sum_{v_j\in\vertSet}\sum_{j\conLeft k} -\sqbracs{ \overline{v}_3\bracs{ \bracs{\diff{}{y} + \rmi\qm_{jk} }u_3 - \rmi\wavenumber U_2 } }_{v_j}^{v_k}
	+ \sqbracs{ u_3\overline{\bracs{ \bracs{\diff{}{y} + \rmi\qm_{jk} }v_3 - \rmi\wavenumber V_2 }} }_{v_j}^{v_k} \\
	&\quad = \sum_{v_j\in\vertSet} u_3\bracs{v_j}\overline{\bracs{ \sum_{j\con k}\bracs{\pdiff{}{n} + \rmi\qm_{jk}}v_3 - \rmi\wavenumber\sum_{j\con k} \sgn_{jk}(v_j) V_2^{(jk)}\bracs{v_j} }} \\
	&\quad + \sum_{v_j\in\vertSet} \overline{v}_3\bracs{v_j}\bracs{ \sum_{j\con k}\bracs{\pdiff{}{n} + \rmi\qm_{jk}}u_3 - \rmi\wavenumber\sum_{j\con k} \sgn_{jk}(v_j) U_2^{(jk)}\bracs{v_j} } \\
	&\quad = \ngmap u \cdot \overline{\dgmap v} - \dgmap u \cdot \overline{\ngmap v}
	= \ip{\ngmap u}{\dgmap v}_{\complex^{3N}} - \ip{\dgmap u}{\ngmap v}_{\complex^{3N}},
\end{align*}
and so the Green's identity holds. \tstk{do we even define a boundary triple in the QG chapter? If so, saying "green's identity" doesn't make much sense!}
Therefore, $\bracs{\complex^{3N}, \dgmap, \ngmap}$ is a boundary triple for the operator $\ag$.
Given \eqref{eq:ClassicalEM-DNMaps} served as the motivation for the definitions \eqref{eq:DGMapDef} and \eqref{eq:NGMapDef}, the $M$-operator associated with \eqref{eq:QGRawSystem} really is a ``graph version" of the Calder\'on operator for the problem \eqref{eq:SingularCurlEquation}.