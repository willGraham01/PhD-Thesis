\section{Derivation of quantum graph problem} \label{sec:3DSystemDerivation}
We now derive the system \eqref{eq:3DQGFullSystem} from \eqref{eq:PeriodCellCurlCurlStrongForm}.
Let $u=\bracs{u_1,u_2,u_3}^\top\in\ktcurlSob{\ddom}{\dddmes}$ and let $\Phi=\bracs{\phi_1,\phi_2,\phi_3}^\top\in\psmooth{\ddom}^3$.
Also define for each $I_{jk}\in\edgeSet$,
\begin{align*}
	U^{(jk)} := R_{jk} \begin{pmatrix} u_1^{(jk)} \\ u_2^{(jk)} \end{pmatrix},
	\qquad
	\Psi^{(jk)} := R_{jk} \begin{pmatrix} \phi_1^{(jk)} \\ \phi_2^{(jk)} \end{pmatrix}.
\end{align*}
We use an overhead tilde to denote composition with $r_{jk}$, and for a function $v$ with $\widetilde{v}^{(jk)}\in\ktgradSob{\sqbracs{0,l_{jk}}}{y}$ write $\bracs{v^{(jk)}}' := \bracs{\widetilde{v}^{(jk)}}' \circ r_{jk}^{-1}$.
Finally, for a given quasi-momentum $\qm$, set $\qm_{jk} = \bracs{ R_{jk}\qm }_2 = \qm\cdot e_{jk}$.
%, and let
%\begin{align*}
%	\ktcurlSobDivFree{\ddom}{\dddmes} &=
%	\clbracs{ u\in\ktcurlSob{\ddom}{\dddmes} \setVert u \text{ is } \kt\text{-divergence-free with respect to } \dddmes}.
%\end{align*}
A function $u\in\ktcurlSob{\ddom}{\dddmes}$ is a solution to \eqref{eq:PeriodCellCurlCurlStrongForm} if
\begin{align} \label{eq:PeriodCellCurlCurlWeakForm}
	\integral{\ddom}{ \ktcurl{\dddmes}u\cdot\overline{\ktcurl{\dddmes}\Phi} }{\dddmes} 
	&= \omega^2 \integral{\ddom}{ u\cdot\overline{\Phi} }{\dddmes},
	\quad\forall \Phi\in\psmooth{\ddom}^3.
\end{align}
We immediately have that $u$ is $\kt$-divergence free (in the sense of definition \ref{def:DivFree-TangGradients}), since for any $v\in\ktgradSob{\ddom}{\dddmes}$ we can take an approximating sequence $\Phi^n$ for $v$ and obtain
\begin{align*}
	\omega^2\integral{\ddom}{ u\cdot\overline{\ktgrad_{\dddmes}v} }{\dddmes}
	&= \omega^2 \lim_{n\rightarrow\infty} \integral{\ddom}{ u\cdot\overline{\ktgrad\Phi^n} }{\dddmes} \\
	&= \lim_{n\rightarrow\infty} \integral{\ddom}{ \ktcurl{\dddmes}u\cdot\overline{\ktcurl{\dddmes}\ktgrad\Phi^n} }{\dddmes}
	= 0,
\end{align*}
through using lemma \ref{lem:CurlOfGradSmoothFunctions}.
Note that we can similarly show that $u$ is orthogonal to all gradients of zero via the same trick, however we will see that the additional information (the conditions (i) and (iv) in proposition \ref{prop:DivFree-AllGradsConditions}) is obtained more directly in what follows.

Now let us fix an edge $I_{jk}$ and take $\Phi$ such that $\supp\Phi\cap\graph\subset I_{jk}^{\circ}$.
The equation \eqref{eq:PeriodCellCurlCurlWeakForm} reduces to
\begin{align*}
	\omega^2 \integral{I_{jk}}{ u\cdot\overline{\Phi} }{\lambda_{jk}}
	&= \integral{I_{jk}}{ \ktcurl{\ddmes}u\cdot\overline{\ktcurl{\ddmes}\Phi} }{\lambda_{jk}} \\
	&= \integral{I_{jk}}{ \bracs{ \bracs{u_3^{(jk)}}' + \rmi\qm_{jk} u_3^{(jk)} - \rmi\wavenumber U_2^{(jk)} }\overline{\bracs{ \bracs{\phi_3^{(jk)}}' + \rmi\qm_{jk} \phi_3^{(jk)} - \rmi\wavenumber \Psi_2^{(jk)} }} }{\lambda_{jk}}.
\end{align*}
Using the change of variables $r_{jk}$ we obtain
\begin{align*}
	\int_0^{l_{jk}} & \bracs{ \bracs{\widetilde{u}_3^{(jk)}}' + \rmi\qm_{jk} \widetilde{u}_3^{(jk)} - \rmi\wavenumber \widetilde{U}_2^{(jk)} } \bracs{ \bracs{\overline{\widetilde{\phi}}_3^{(jk)}}' - \rmi\qm_{jk} \overline{\widetilde{\phi}}_3^{(jk)} + \rmi\wavenumber \overline{\widetilde{\Psi}}_2^{(jk)} } \ \md y \\
	&= \omega^2 \int_0^{l_{jk}} \widetilde{U}_1^{(jk)}\overline{\widetilde{\Psi}}_1^{(jk)} + \widetilde{U}_2^{(jk)}\overline{\widetilde{\Psi}}_2^{(jk)} + \widetilde{u}_3^{(jk)}\overline{\widetilde{\phi}}_3^{(jk)} \ \md y.
\end{align*}
Now suppose we have $\varphi_1,\varphi_2\in\csmooth{\sqbracs{0,l_{jk}}}$.
Since the system
\begin{align*}
	\begin{pmatrix} \phi_1 \\ \phi_2 \end{pmatrix} &= R_{jk}^\top \begin{pmatrix} \varphi_1 \\ \varphi_2 \end{pmatrix}
\end{align*}
has a unique solution, we can construct a $\Phi\in\csmooth{\ddom}$ with $\supp\Phi\cap\graph\subset I_{jk}^{\circ}$, $\widetilde{\Psi}_1^{(jk)}=\varphi_1$, and $\widetilde{\Psi}_2^{(jk)}=\varphi_2$.
We thus observe that
\begin{align*}
	\int_0^{l_{jk}} & \bracs{ \bracs{\widetilde{u}_3^{(jk)}}' + \rmi\qm_{jk} \widetilde{u}_3^{(jk)} - \rmi\wavenumber \widetilde{U}_2^{(jk)} } \bracs{ \overline{\varphi_3}' - \rmi\qm_{jk} \overline{\varphi_3} + \rmi\wavenumber \overline{\varphi_2} } \ \md y \\
	&= \omega^2 \int_0^{l_{jk}} \widetilde{U}_1^{(jk)}\overline{\varphi_1} + \widetilde{U}_2^{(jk)}\overline{\varphi_2} + \widetilde{u}_3^{(jk)}\overline{\varphi_3} \ \md y
\end{align*}
holds for any $\varphi_1,\varphi_2,\varphi_3\in\csmooth{\sqbracs{0,l_{jk}}}$.
Therefore it must hold that for any $\varphi\in\csmooth{\sqbracs{0,l_{jk}}}$,
\begin{subequations}
	\begin{align*}
		0 &= \int_0^{l_{jk}} \overline{\varphi} \widetilde{U}_1^{(jk)} \ \md y, \labelthis\label{eq:CurlCurlWeakFormPhi1} \\
		0 &= \int_0^{l_{jk}} \overline{\varphi} \bracs{ \rmi\wavenumber\bracs{\widetilde{u}_3^{(jk)}}' + \bracs{\wavenumber^2 - \omega^2}\widetilde{U}_2^{(jk)} - \wavenumber\qm_{jk}\widetilde{u}_3^{(jk)}  } \ \md y, \labelthis\label{eq:CurlCurlWeakFormPhi2} \\
		0 &= \int_0^{l_{jk}} \overline{\varphi}' \bracs{ \bracs{\widetilde{u}_3^{(jk)}}'
		- \rmi\wavenumber\widetilde{U}_2^{(jk)} + \rmi\qm_{jk}\widetilde{u}_3^{(jk)} } \\
		&\qquad -\rmi\qm_{jk}\overline{\varphi}\bracs{ \bracs{\widetilde{u}_3^{(jk)}}' - \rmi\wavenumber\bracs{\widetilde{U}_2^{(jk)}}' + \rmi\qm_{jk}\widetilde{u}_3^{(jk)} }
		- \omega^2 \widetilde{u}_3^{(jk)}\overline{\varphi} \ \md y. \labelthis\label{eq:CurlCurlWeakFormPhi3}
	\end{align*}
\end{subequations}
The equation \eqref{eq:CurlCurlWeakFormPhi1} implies that $U_1^{(jk)}=0$ (almost everywhere) on $I_{jk}$, whilst \eqref{eq:CurlCurlWeakFormPhi2} implies that
\begin{align*}
	\rmi\wavenumber\bracs{\diff{}{y} + \rmi\qm_{jk}}\widetilde{u}_3^{(jk)} + \wavenumber^2\widetilde{U}_2^{(jk)} &= \omega^2\widetilde{U}_2^{(jk)},
\end{align*}
on $I_{jk}$.
Given that $u$ is divergence-free, we know that $\widetilde{U}_2^{(jk)}$ is weakly differentiable in the $\gradSob{\sqbracs{0,l_{jk}}}{y}$-sense and can manipulate \eqref{eq:CurlCurlWeakFormPhi3} to obtain
\begin{align*}
	\int_0^{l_{jk}} \overline{\varphi}' \widetilde{u}'_{3,jk} \ \md y
	&= -\int_0^{l_{jk}} \overline{\psi} \bracs{ \rmi\wavenumber\widetilde{U}'_{2,jk} - \wavenumber\qm_{jk}\widetilde{U}_{2,jk} - 2\rmi\qm_{jk}\widetilde{u}'_{3,jk} + \qm_{jk}^2\widetilde{u}_{3,jk} - \omega^2\widetilde{u}_{3,jk} } \ \md y.,
\end{align*}
for all $\varphi\in\csmooth{\sqbracs{0,l_{jk}}}$.
Thus we can conclude that $\widetilde{u}_3\in\gradgradSob{\sqbracs{0,l_{jk}}}{y}$, from which we deduce that
\begin{align*}
	-\bracs{ \diff{}{y} + \rmi\qm_{jk} }^2\widetilde{u}_3^{(jk)} + \rmi\wavenumber\bracs{ \diff{}{y} + \rmi\qm_{jk} }\widetilde{U}_2^{(jk)} &= \omega^2 \widetilde{u}_3^{(jk)}
\end{align*}
on $I_{jk}$.
This provides us with a system of coupled ODEs on each of the edges $I_{jk}$ (and the information that the component $\widetilde{U}_1^{(jk)}=0$ on each edge).

Now we look at how these edge ODEs are coupled at the vertices.
Fix a vertex $v_j$ and consider a $\Phi\in\csmooth{\ddom}$ whose support contains $v_j$ in its interior, and no other vertices of $\graph$.
Testing against such $\Phi$ in \eqref{eq:PeriodCellCurlCurlWeakForm} implies that
\begin{align*}
	\alpha_j\omega^2 u(v_j)\cdot\overline{\Phi}(v_j)
	&= \integral{\ddom}{ \ktcurl{\ddmes}u\cdot\overline{\ktcurl{\ddmes}\Phi} - \omega^2 u\cdot\overline{\Phi} }{\ddmes} \\
	&= \sum_{j\con k}\integral{I_{jk}}{ \bracs{ \bracs{u_3^{(jk)}}' + \rmi\qm_{jk}u_3^{(jk)} - \rmi\wavenumber U_2^{(jk)}} \bracs{ \bracs{\overline{\phi}_3^{(jk)}}' - \rmi\qm_{jk}\overline{\phi}_3^{(jk)} + \rmi\wavenumber \overline{\Psi}_2^{(jk)}} \\
	&\qquad - \omega^2 U_1^{(jk)}\overline{\Psi}_1 - \omega^2 U_2^{(jk)}\overline{\Psi}_2 - \omega^2 u_3^{(jk)}\overline{\phi}_3 }{\lambda_{jk}} \\
	&= \sum_{j\con k}\int_0^{l_{jk}} \bracs{ \bracs{\widetilde{u}_3^{(jk)}}' + \rmi\qm_{jk}\widetilde{u}_3^{(jk)} - \rmi\wavenumber \widetilde{U}_2^{(jk)}} \bracs{ \bracs{\overline{\widetilde{\phi}}_3^{(jk)}}' - \rmi\qm_{jk}\overline{\widetilde{\phi}}_3^{(jk)} + \rmi\wavenumber \overline{\widetilde{\Psi}}_2^{(jk)}} \\
	&\qquad - \bracs{\bracs{\widetilde{u}_3^{(jk)}}' + \rmi\qm_{jk}\widetilde{u}_3^{(jk)} - \rmi\wavenumber \widetilde{U}_2^{(jk)}}\rmi\wavenumber\overline{\widetilde{\Psi}_2^{(jk)}} - \omega^2 \widetilde{u}_3^{(jk)}\overline{\widetilde{\phi}}_3 \ \md y \\
	&= -\sum_{j\con k}\int_0^{l_{jk}} \overline{\widetilde{\phi}}_3
	\sqbracs{ \bracs{\widetilde{u}_3^{(jk)}}'' + 2\rmi\qm_{jk}\bracs{\widetilde{u}_3^{(jk)}}' + \bracs{\rmi\qm_{jk}}^2\widetilde{u}_3^{(jk)} + \omega^2\widetilde{u}_3^{(jk)} } \\
	&\qquad - \overline{\widetilde{\phi}}_3\sqbracs{\rmi\wavenumber\bracs{ \bracs{\widetilde{U}_2^{(jk)}}' + \rmi\qm_{jk}\widetilde{U}_2^{(jk)} } } \ \md y \\
	&\qquad + \overline{\phi}_3(v_j)\sqbracs{ \sum_{j\con k}\bracs{\pdiff{}{n}+\rmi\qm_{jk}}u_3^{(jk)}(v_j) + \sum_{j\conRight k}U_2^{(kj)}(v_j) - \sum_{j\conLeft k}U_2^{(jk)}(v_j) } \\
	&= \overline{\phi}_3(v_j)\sqbracs{ \sum_{j\con k}\bracs{\pdiff{}{n}+\rmi\qm_{jk}}u_3^{(jk)}(v_j) + \sum_{j\conRight k}U_2^{(kj)}(v_j) - \sum_{j\conLeft k}U_2^{(jk)}(v_j) },
\end{align*}
where $U_2^{(jk)}(v_j)$ is the trace of $U_2^{(jk)}$ to the vertex $v_j$.
Identifying that this holds for all $\phi_1(v_j), \phi_2(v_j), \phi_3(v_j)\in\complex$, we must conclude that
\begin{align*}
	u_1(v_j) &= 0, \\
	u_2(v_j) &= 0, \\
	\alpha_j\omega^2u_3(v_j) &= \sum_{j\con k}\bracs{\pdiff{}{n}+\rmi\qm_{jk}}u_3^{(jk)}(v_j) + \sum_{j\conRight k}U_2^{(kj)}(v_j) - \sum_{j\conLeft k}U_2^{(jk)}(v_j).
\end{align*}
We are thus presented with the following set of equations on each edge $I_{jk}$;
\begin{subequations} \label{eq:QGRawSystem}
	\begin{align*}
		\widetilde{U}_1^{(jk)} &= 0 \\
		\rmi\wavenumber \bracs{ \diff{}{y} + \rmi\qm_{jk} }\widetilde{u}_3^{(jk)} + \wavenumber^2\widetilde{U}_2^{(jk)} &= \omega^2\widetilde{U}_2^{(jk)}, \labelthis\label{eq:QGPhi2} \\
		-\bracs{ \diff{}{y} + \rmi\qm_{jk} }^2\widetilde{u}_3^{(jk)} + \rmi\wavenumber\bracs{ \diff{}{y} + \rmi\qm_{jk} }\widetilde{U}_2^{(jk)} &= \omega^2 \widetilde{u}_3^{(jk)}, \labelthis\label{eq:QGPhi3} 
	\end{align*}
	complemented by the vertex conditions
	\begin{align*}
		\widetilde{u}_3 \text{ is continuous at } v_j &\quad\forall v_j\in\vertSet, \labelthis\label{eq:QGContinuity} \\
		u_1(v_j) &= 0, \\
		u_2(v_j) &= 0, \\
		\alpha_j \omega^2 u_3\bracs{v_j}
		&= \sum_{j\con k} \bracs{ \pdiff{}{n} + \rmi\qm_{jk} }u_3^{(jk)}\bracs{v_j} - \rmi\wavenumber\bracs{ \sum_{j\conRight k} U_2^{(jk)} - \sum_{j\conLeft k} U_2^{(jk)} }. \labelthis\label{eq:QGVertexCondition}
	\end{align*}
\end{subequations}

Determining the solution to \eqref{eq:QGRawSystem} requires us to find the function $u_3$ and the edge-functions $U_2^{(jk)}$.
However we can eliminate the $U_2^{(jk)}$ from the system \eqref{eq:QGRawSystem} via substitution of \eqref{eq:QGPhi2} into \eqref{eq:QGPhi3}, and use of corollary \ref{cory:DivFree-TangGradsConditions}(v), to obtain
\begin{subequations} \label{eq:CurlCurl-ScalarQGProblem}
	\begin{align}
		-\bracs{ \diff{}{y} + \rmi\qm_{jk} }^2 u_3^{(jk)} &= \bracs{\omega^2-\wavenumber^2}u,
		\qquad &\text{on each } I_{jk}, \\
		u_3 \text{ is continuous at } & v_j \qquad &\forall v_j\in\vertSet, \\
		\sum_{j\con k}\bracs{\pdiff{}{n}+\rmi\qm_{jk}}u_3^{(jk)}(v_j) &= \alpha_j\bracs{\omega^2-\wavenumber^2}u_3(v_j), \qquad &\forall v_j\in\vertSet.
	\end{align}
\end{subequations}
That is, \tstk{curl of curl equation} simply reduces to the scalar system obtained in section \ref{sec:ScalarDerivation}, except with $\omega^2-\wavenumber^2$ playing the role of the spectral parameter.
We will elaborate further on the implications of this result in section \tstk{discussion}.

\tstk{remarks: the fact that you have obtained a (scalar) equation for the "polarised" Maxwell by a series of nontrivial manipulations on the curl demonstrates that your approach is sound overall and gives further credibility to the work on the first part of the thesis. This "polarised Maxwell" chapter will not have much new stuff on the numerics side though, as we have got back to the scalar case, which will have been looked at numerically in the preceding chapter.}

\tstk{UNCHANGED FROM PREVIOUS NOTES!!!! Also needs an introductory sentence, and will probably go in the discussion section of this chapter.}

\subsection{Remarks on the Calder\'on Operator} \label{ssec:CalderonOp}
In this section we look to draw parallels between the classical curl-of-the-curl problem (described by \eqref{eq:CurlCurlEqn} on a suitable domain with boundary conditions) for a polarised electromagnetic field, and the system \eqref{eq:QGRawSystem}.
It is not known whether the problem \eqref{eq:PeriodCellCurlCurlStrongForm} is the ``limit" of a thin-structure problem with thick vertices, in the sense of \tstk{scalar SS problem to scalar TS problem, via KZ and EP}.
However, we can demonstrate that the $M$-operator associated to the (operator which defines the) problem \eqref{eq:QGRawSystem} (hence \eqref{eq:PeriodCellCurlCurlStrongForm}) is a direct analogue of the Calder\'on operator for problems like \eqref{eq:CurlCurlEqn}.
This will be is done by showing that the Dirichlet and Neumann maps for the classical problem motivate ``natural" definitions for their counterparts for the problem \eqref{eq:QGRawSystem}, and that the resulting maps form a boundary triple, providing us with an $M$-operator.
We will also see that the vertex condition \eqref{eq:QGVertexCondition} can be written in a familiar form \tstk{see the BC in scalar paper, or \eqref{eq:DispersiveBC}} relating the Dirichlet and Neumann maps, implying a similar solving approach to \tstk{scalar discussion chapter} can be undertaken.

First, we quickly review/ reintroduce the Calder\'on operator in the classical setting. \tstk{might be the first time we talk about this, or we might move it into the $M$-matrix section of QGs, or even into the introductory section of this chapter.}
Let $\dddom\subset\reals^3$ be a domain, and consider the curl-of-the-curl problem (or polarised Maxwell system)
\begin{subequations} \label{eq:Maxwell3D}
	\begin{align} 
		\curl{}\bracs{\curl{}u}u - \beta u = 0 &\qquad\text{in } \dddom, \\
		\hat{n}\wedge\curl{}u = m &\qquad\text{on } \partial\dddom,
	\end{align}
\end{subequations}
where $\beta>0$ and $m$ is a given function, and $\hat{n}$ is the exterior normal to the surface $\partial\dddom$.
The Calder\'on operator associated to the problem \eqref{eq:Maxwell3D} is then the operator $\mathcal{C}$ that acts on solutions $u$ to \eqref{eq:Maxwell3D}, sending
\begin{align*}
	u\vert_{\partial\dddom} \rightarrow \hat{n}\wedge\bracs{\curl{}u}\vert_{\partial\dddom}.
\end{align*}
Defining $\mathcal{A}$ as the operator
\begin{align*}
	\mathrm{dom}\bracs{\mathcal{A}} &= \clbracs{ u\in H^2_{\mathrm{curl}}(\dddom) \setVert \hat{n}\wedge\curl{}u\vert_{\partial\dddom} = m }, \\
	\mathcal{A}u &= \curl{}\bracs{\curl{}u},
\end{align*}
and the Dirichlet and Neumann maps ($\dmap$ and $\nmap$ respectively) by
\begin{align} \label{eq:ClassicalEM-DNMaps}
	\dmap u = u\vert_{\partial\dddom}, \qquad
	\nmap u = \hat{n}\wedge\curl{u}\vert_{\partial\dddom},
\end{align}
we can validate Green's identity for the triple $\bracs{\ltwo{\partial\dddom}{S}^3, \dmap, \nmap}$:
\begin{align*}
	\integral{\dddom}{ \mathcal{A}u \cdot \overline{v} - u \cdot \overline{\mathcal{A}v} }{x}
	&= \integral{\dddom}{ \curl{\curl{u}}\cdot\overline{v} - u\cdot\overline{\curl{\curl{v}}} }{x} \\
	&= \integral{\dddom}{ \curl{u}\cdot\overline{\curl{v}} - \curl{u}\cdot\overline{\curl{v}} }{x} \\
	&\quad + \integral{\partial\dddom}{ \hat{n}\wedge\curl{u}\cdot\overline{v} - u\cdot\hat{n}\wedge\overline{\curl{v}} }{S} \\
	&= \integral{\partial\dddom}{ \nmap u \cdot \overline{\dmap v} - \dmap u \cdot \overline{\nmap v} }{S}.
\end{align*}
We can thus conclude that $\bracs{\ltwo{\partial\dddom}{S}^3, \dmap, \nmap}$ is a boundary triple for the operator $\mathcal{A}$.
The Calder\'on operator is the corresponding Dirichlet-to-Neumann map, or $M$-operator, \tstk{QG chapter} for the problem \eqref{eq:Maxwell3D}.

Now let us return to the quantum graph problem \eqref{eq:QGRawSystem}.
If we expect \eqref{eq:SingStrucCurlCurl} to be some ``limit" of a thin-structure problem with thick vertices, then we expect that the vertex condition \eqref{eq:QGVertexCondition} will be of the form
\begin{align} \label{eq:DispersiveBC}
	\dgmap u &= -\omega^2 \tilde{\alpha} \ngmap u,
\end{align}
which is the form of the vertex conditions in \tstk{scalar problem}.
In the context of \eqref{eq:QGRawSystem}, we should expect $\tilde{\alpha}$ be akin to the diagonal matrix of coupling constants (rather than precisely the matrix $\alpha$), whilst $\dgmap, \ngmap$ will be the Dirichlet and Neumann maps for the quantum graph problem \eqref{eq:QGRawSystem}.

Given the definition of $\dmap$ in \eqref{eq:ClassicalEM-DNMaps}, we should expect that
\begin{align} \label{eq:DGMapDef}
	\dgmap u &= 
	\begin{pmatrix}
		u\bracs{v_1} \\ u\bracs{v_2} \\ \vdots \\ u\bracs{v_N}
	\end{pmatrix}
	\in\complex^{3N},
\end{align}
where we have stacked the 3-vectors on top of each other and set $N=\abs{\vertSet}$.
As for $\ngmap u$, this should be the analogue of $\nmap$ in \eqref{eq:ClassicalEM-DNMaps} --- only now the boundary of our domain is the vertices of $\graph$.
Define the functions \tstk{might be worth moving into our usual setup assumption for ease of use?}
\begin{align*}
	\sgn_{jk}: \clbracs{v_j, v_k} \rightarrow \clbracs{-1,0,1}, 
	&\qquad
	\sgn_{jk}(x) = \begin{cases} -1 & x=v_j, \\ 1 & x=v_k, \end{cases}
	&\qquad
	\hat{\sigma}_{jk} &= \sgn_{jk}\widehat{e}_{jk},
\end{align*}
so $\hat{\sigma}_{jk}$ is the ``exterior normal" to the edge $I_{jk}$.
The natural candidate for $\ngmap$ is then 
\begin{align} \label{eq:NGMapDef}
	\ngmap u &= 
	\begin{pmatrix}
		\sum_{1\con k} \hat{\sigma}_{1k}\wedge\ktcurl{\dddmes}u\vert_{v_1} \\
		\sum_{2\con k} \hat{\sigma}_{2k}\wedge\ktcurl{\dddmes}u\vert_{v_2} \\
		\vdots \\
		\sum_{N\con k} \hat{\sigma}_{Nk}\wedge\ktcurl{\dddmes}u\vert_{v_N}
	\end{pmatrix}
	\in\complex^{3N},
\end{align}
where we have again stacked the 3-vectors vertically.
From our analysis of $\kt$-tangential curls \tstk{section}, we know that
\begin{align*}
	\ktcurl{\dddmes}u &= \bracs{ \bracs{ u_3^{(jk)} }' + \rmi\qm_{jk}u_3^{(jk)} - \rmi\wavenumber U_2^{(jk)} }\widehat{n}_{jk},
\end{align*}
on each edge $I_{jk}$.
Therefore, 
\begin{align*}
	\widehat{e}_{jk}\wedge\ktcurl{\dddmes}u &= -
	\begin{pmatrix} 
	0 \\
	0 \\
	\bracs{ u_3^{(jk)} }' + \rmi\qm_{jk}u_3^{(jk)} - \rmi\wavenumber U_2^{(jk)}
	\end{pmatrix},
\end{align*}
on $I_{jk}$, and hence (for a fixed $v_j\in\vertSet$)
\begin{align*}
	\sum_{j\con k} \hat{\sigma}_{jk} \ \wedge \ &\ktcurl{\dddmes}u\vert_{v_j} = \\ 
	&\begin{pmatrix}
	0 \\
	0 \\	
	- \sum_{j\con k}\bracs{\pdiff{}{n} + \rmi\qm_{jk}}u_3^{(jk)}\bracs{v_j}
	+ \rmi\wavenumber\bracs{ \sum_{j\conRight k} U_2^{(kj)}\bracs{v_j} - \sum_{j\conLeft k} U_2^{(jk)}\bracs{v_j} }
	\end{pmatrix}.
\end{align*}
\tstk{we could write
\begin{align*}
	\sum_{j\conRight k} U_2^{(kj)}\bracs{v_j} - \sum_{j\conLeft k} U_2^{(jk)}\bracs{v_j} &=
	\sum_{j\con k} \sgn_{jk}U_2^{(jk)}\bracs{v_j}
\end{align*} 
using our definitions and conventions from the QG chapter.}
The vertex conditions for the system \eqref{eq:QGRawSystem} can be written as
\begin{align} \label{eq:VertConditionExplicit}
	\alpha_j\omega^2 u\bracs{v_j} &=
	\begin{pmatrix}
	0 \\
	0 \\	
	\bracs{\pdiff{}{n} + \rmi\qm_{jk}}u_3^{(jk)}\bracs{v_j}
	- \rmi\wavenumber\bracs{ \sum_{j\conRight k} U_2^{(kj)}\bracs{v_j} - \sum_{j\conLeft k} U_2^{(jk)}\bracs{v_j} }
	\end{pmatrix},
\end{align}
at each $v_j\in\vertSet$ --- note that the first two components are just the conditions $u_1\bracs{v_j}=u_2\bracs{v_j}=0$.
We can identify \eqref{eq:VertConditionExplicit} as being of the form \eqref{eq:DispersiveBC} where
\begin{align*}
	\tilde{\alpha} = 
	\mathrm{diag}\bracs{\alpha_1, \alpha_1, \alpha_1, \alpha_2, \alpha_2, \alpha_2, ..., \alpha_N, \alpha_N, \alpha_N} \in \complex^{3N\times 3N},
\end{align*}
and $\dgmap, \ngmap$ are as in \eqref{eq:DGMapDef}, \eqref{eq:NGMapDef}.
To complete the analogy, define the operator $\ag$ via the action
\begin{align*}
	\ag u &= 
	\begin{pmatrix}
		\sqbracs{ \rmi\wavenumber\bracs{\diff{}{y} + \rmi\qm_{jk} }u_3^{(jk)} + \wavenumber^2 U_2^{(jk)} }e_{jk}
		+ U_1^{(jk)} n_{jk} \\
		- \bracs{\diff{}{y} + \rmi\qm_{jk} }^2 u_3^{(jk)} + \rmi\wavenumber \bracs{\diff{}{y} + \rmi\qm_{jk} }U_2^{(jk)}
	\end{pmatrix}
\end{align*}
on each edge, where $\mathrm{dom}\bracs{\ag}$ consists of all functions $u$ with the following properties:
\begin{align*}
	u\in\mathrm{dom}\bracs{\ag} \quad\Leftrightarrow\quad &
	\begin{cases}
	u\in L^2\bracs{\graph}\times L^2\bracs{\graph}\times H^2\bracs{\graph}, \\
	\begin{pmatrix} u_1 \\ u_2 \end{pmatrix}\cdot e_{jk}\in \gradSob{I_{jk}}{y}, & \forall I_{jk}\in\edgeSet, \\
	u \text{ is continuous at } v_j, & \forall v_j\in\vertSet, \\
	\text{\eqref{eq:VertConditionExplicit} is satisfied at } v_j, & \forall v_j\in\vertSet.
	\end{cases}
\end{align*}
Then we have that
\begin{align*}
	\integral{I_{jk}}{ \ag u \cdot \overline{v} }{y} - \integral{I_{jk}}{ u \cdot \overline{\ag v} }{y}
	&= \sqbracs{ -u'_3 v_3 + u_3 v_3' - 2\rmi\qm_{jk}u_3 v_3 + \rmi\wavenumber\bracs{U_2 v_3 + u_3 V_2} }_{v_j}^{v_k} \\
	&= -\sqbracs{ \overline{v}_3\bracs{ \bracs{\diff{}{y} + \rmi\qm_{jk} }u_3 - \rmi\wavenumber U_2 } }_{v_j}^{v_k} \\
	&\qquad + \sqbracs{ u_3\overline{\bracs{ \bracs{\diff{}{y} + \rmi\qm_{jk} }v_3 - \rmi\wavenumber V_2 }} }_{v_j}^{v_k}.
\end{align*}
Which implies that
\begin{align*}
	&\ip{\ag u}{v}_{L^2\bracs{\graph}^3} - \ip{u}{\ag v}_{L^2\bracs{\graph}^3}
	= \sum_{v_j\in\vertSet}\sum_{j\conLeft k} \integral{I_{jk}}{ \ag u \cdot \overline{v} - u \cdot \overline{\ag v} }{y} \\
	&\quad = \sum_{v_j\in\vertSet}\sum_{j\conLeft k} -\sqbracs{ \overline{v}_3\bracs{ \bracs{\diff{}{y} + \rmi\qm_{jk} }u_3 - \rmi\wavenumber U_2 } }_{v_j}^{v_k}
	+ \sqbracs{ u_3\overline{\bracs{ \bracs{\diff{}{y} + \rmi\qm_{jk} }v_3 - \rmi\wavenumber V_2 }} }_{v_j}^{v_k} \\
	&\quad = \sum_{v_j\in\vertSet} u_3\bracs{v_j}\overline{\bracs{ \sum_{j\con k}\bracs{\pdiff{}{n} + \rmi\qm_{jk}}v_3 - \rmi\wavenumber\bracs{ \sum_{j\conRight k} V_2^{(kj)}\bracs{v_j} - \sum_{j\conLeft k} V_2^{(jk)}\bracs{v_j} } }} \\
	&\quad + \sum_{v_j\in\vertSet} \overline{v}_3\bracs{v_j}\bracs{ \sum_{j\con k}\bracs{\pdiff{}{n} + \rmi\qm_{jk}}u_3 - \rmi\wavenumber\bracs{ \sum_{j\conRight k} U_2^{(kj)}\bracs{v_j} - \sum_{j\conLeft k} U_2^{(jk)}\bracs{v_j} } } \\
	&\quad = \ngmap u \cdot \overline{\dgmap v} - \dgmap u \cdot \overline{\ngmap v}
	= \ip{\ngmap u}{\dgmap v}_{\complex^{3N}} - \ip{\dgmap u}{\ngmap v}_{\complex^{3N}},
\end{align*}
and so Green's identity holds. \tstk{do we even define a boundary triple in the QG chapter? If so, saying "green's identity" doesn't make much sense!}
Therefore, $\bracs{\complex^{3N}, \dgmap, \ngmap}$ is a boundary triple for the operator $\ag$.
Given the motivations for the definitions \eqref{eq:DGMapDef} and \eqref{eq:NGMapDef}, the $M$-operator associated with \eqref{eq:QGRawSystem} can be thought of as an analogue (or ``graph-version") of the Calder\'on operator for the problem \eqref{eq:SingStrucCurlCurl}.