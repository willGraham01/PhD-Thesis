\section{Discussion and Research Outlook} \label{sec:CC-Discussion}
In this chapter we have performed a detailed analysis of the equation \eqref{eq:SingularCurlEquation} and the associated function spaces.
By extending the definitions and techniques from chapter \ref{ch:ScalarSystem}, we have been able to provide a definition for a curl-like object in one dimension (the tangential curl), and understand the properties of said object on our singular structure.
Complementing this analysis is the analysis and discussion of section \ref{sec:DivFreeCondition}, on the condition of being divergence free in our singular structure context.
It is possible to define a ``strong" notion of the divergence of a vector field in a similar manner to gradients and curls, \tstk{for which de Rham works}, but forces us to accept that the divergence of any vector field is the zero function.
Were we considering a problem in which the divergence appeared as an interior operator, this approach would be necessary for our understanding of the variational form of said problem, as we would need to explicitly work with the object $\ktdiv{\dddmes}u$.
However the problem \eqref{eq:SingularCurlEquation} does not require us to provide such a ``strong" definition of the divergence, and we instead interpret the divergence free condition weakly, as is done classically.
This enables us to characterise divergence-free vector fields, and its utility extends into exploration of the first order Maxwell system, which we discuss shortly.

Our efforts to turn \eqref{eq:SingularCurlEquation} into a more tractable problem in section \ref{sec:3DSystemDerivation} culminate in us arriving at the acoustic approximation.
This marks a departure from the classical setting when one considers the curl of the curl equation \eqref{eq:Intro-CurlCurlEqns} on a periodic medium in the $\bracs{x_1,x_2}$-plane that is extruded into $x_3$.
In such a setup, we only obtain the acoustic approximation from \eqref{eq:Intro-CurlCurlEqns} under the assumption of oblique wave propagation --- when the propagation constant in one of the axial directions is zero.
This would imply that we should only obtain \eqref{eq:SingularScalarWaveEqn} from \eqref{eq:SingularCurlEquation} when we set $\wavenumber=0$, however we obtain \eqref{eq:SingularScalarWaveEqn} regardless of the value of $\wavenumber$, and even have $\wavenumber$ appearing as part of an effective spectral parameter in $\omega^2-\wavenumber^2$.
The reason for this reduction lies in the nature of our variational problem itself --- specifically the dynamics induced by the tangential curls.
Let us imagine we now have a wave $u = u_0\e^{\rmi\vec{k}\cdot x}$ propagating in our material with wave-vector $\vec{k} = \qm_1\widehat{x}_1 + \qm_2\widehat{x}_2 + \wavenumber\widehat{x}_3$.
For each edge $I_{jk}$ (or plane $P_{jk}$) we can write this wave-vector using the local frame of reference provided by $\widehat{e}_{jk}$ and $\widehat{n}_{jk}$, obtaining $\vec{k}_{jk} := \qm_{jk}\widehat{e}_{jk} + \qm_{jk}^\perp\widehat{n}_{jk} + \wavenumber\widehat{x}_3$, where $\qm_{jk}^\perp := \qm\cdot n_{jk}$.
Now $\ktcurl{\dddmes}u$ on $I_{jk}$ is directed in the $\widehat{n}_{jk}$ direction, effectively only inducing dynamics or changes in $u$ in the $\widehat{e}_{jk}$ and $\widehat{x}_3$ directions.
There is no propagation out of the plane $P_{jk}$ as a result, which corresponds to when we have the component of $\vec{k}$ in the $\widehat{n}_{jk}$ direction being zero --- $\vec{k}\cdot\widehat{n}_{jk}=0$, which happens only if $\qm_{jk}^\perp=0$.
By analogy with the classical setting, one component of our wave-vector being zero will then result in the curl of the curl equation reducing to the acoustic approximation.
This occurs for each plane $P_{jk}$, and thus we find that our edge ODEs always reduce to the (singular analogue of the) acoustic approximation.
We also highlight that this is a direct consequence of the behaviour of the tangential curls on the singular structure geometry --- a union of planes induced by the extrusion of $\hat{\graph}$ --- which we have chosen to study.
There is no way to avoid such a reduction since it is tied to the nature of our singular structure being \emph{singular} --- we are always going to effectively ``loose" the ability to observe wave propagation in one direction.

This last observation does provide an idea that might prevent this collapse to the acoustic approximation:  we need part of our domain to be non-singular, which can be done by ``filling" the regions between the edges $I_{jk}$ with some background material.
Doing so will move us closer towards a true ``fibre-like" geometry, where we have a background material interlaced with periodic inclusions, and then extruded into three dimensions.
Pursuit of this idea naturally leads us on to the work in chapter \ref{ch:SingInc}.
Before moving on however, we highlight some further considerations and open problems that have been bought to the forefront of our attention in light of the results of this section.

\subsection{Open Questions} \label{ssec:CC-OpenQuestions}
Obtaining \eqref{eq:SingularScalarWaveEqn} from \eqref{eq:SingularCurlEquation} ultimately fulfils one of the objectives established at the start of this chapter: to provide a candidate for (the spectrum of) the limit of the curl of the curl equation on a thin structure as the thickness tends to zero.
With the success of our analogous variational approach in chapter \ref{ch:ScalarSystem} in obtaining the known limit of the acoustic equation on thin structures, we can put forward \eqref{eq:QGRawSystem} as a candidate for the limit of the aforementioned curl of the curl problems.
We have highlighted already that determining this limit (or the limit of the spectra) is currently an open problem in the literature.
By providing a candidate for the limit we motivate and invite analysis akin to \cite{kuchment2001convergence, kuchment2003asymptotics, exner2005convergence}; either to confirm that sending the thickness of a structure to zero results in the same effective problem as the acoustic setting, or to indicate the effects that are present in the thin structure setting that do not disappear in the singular limit but which are not captured by our variational problems.
The former being confirmed would further cement our approach via singular measures as a tool for probing limits of thin structures, and provide a useful justification for physical modelling of such structures in an electromagnetic context.
It could also motivate further analysis into the first order Maxwell system in such limits, of which the curl of the curl equation a consequence, similar to how the acoustic approximation is a consequence of a particular case of the curl of the curl equations. 
Confirmation of the latter case would also provide a number of insights into why our classically-motivated variational approach fails.
It may simply be the case that \eqref{eq:SingularCurlEquation} is not the correct analogue of \eqref{eq:Intro-CurlCurlEqns}, and with adjustments any effects that are lost can be recovered --- with the explanation as to how and why lying in the aforementioned analysis.
Or it may be that there is an alternative approach to defining gradients and curls to the one we have adopted, which is not so restrictive and preserves any effects our current approach looses.
In any event, this would shed further light onto the behaviour of the curl of the curl operator on increasingly fine thin structures.

Given that our investigation into the curl of the curl equation has resulted in us obtaining the acoustic approximation, it is natural to consider stepping back to a variational problem for the general Maxwell system;
\begin{subequations} \label{eq:CC-Maxwell1stOrder}
	\begin{align}
		\ktdiv{\dddmes} E = 0,
		&\qquad
		\ktcurl{*}E = \rmi\omega\mu_m H, \\
		\ktdiv{\dddmes} H = 0,
		&\qquad
		\ktcurl{*}H = -\rmi\omega\epsilon_m E,
	\end{align}
\end{subequations}
for the purpose of illustration, we will assume the material parameters are constants. 
A thorough analysis would establish a formal definition for the Maxwell operator, the eigenvalue problem for which corresponds to \eqref{eq:CC-Maxwell1stOrder}, which we interpret as the problem of finding divergence-free $E,H\in\pltwo{\ddom}{\dddmes}^3$ such that
\begin{subequations} \label{eq:CC-Maxwell1stOrderVariational}
	\begin{align} 
		\integral{\ddom}{ E\cdot\overline{\ktcurl{}\phi} - \rmi\omega\mu_m H\cdot\overline{\phi} }{\dddmes} &= 0, 
		\qquad &\forall \phi\in\psmooth{\ddom}^3, \\
		\integral{\ddom}{ H\cdot\overline{\ktcurl{}\psi} + \rmi\omega\epsilon_m E\cdot\overline{\psi} }{\dddmes} &= 0,
		\qquad &\forall \psi\in\psmooth{\ddom}^3.
	\end{align}
\end{subequations}
One has to answer a number of new questions when attempting to examine the first order Maxwell operator.
Definition \ref{def:DivFree-AllGradients} translates directly into the first order context, but we write the ``curls" in \eqref{eq:CC-Maxwell1stOrder} with a subscript asterisk to distinguish them from the tangential curls that belong to fields in $\ktcurlSob{\ddom}{\dddmes}$.
Unlike in \eqref{eq:SingularCurlEquation}, the curl appears as a first-order operator in \eqref{eq:CC-Maxwell1stOrder}, and so there is no guarantee that the object $\ktcurl{*}E$ coincides with $E\in\ktcurlSob{\ddom}{\dddmes}$.
Indeed, the objects $\ktcurl{*}E$ and $\ktcurl{*}H$ in \eqref{eq:CC-Maxwell1stOrder} are only really ``defined" upon determining a solution $E,H$ to \eqref{eq:CC-Maxwell1stOrderVariational}.
Circumnavigating these issues and performing an analysis of the resulting system is the natural next step in pursuit of understanding the Maxwell system on singular structures, and the behaviour of the modes supported by such structures.
There is also the related question of whether the singular curl of the curl equation \eqref{eq:SingularCurlEquation} would still be a consequence of the more general (singular) Maxwell system \eqref{eq:CC-Maxwell1stOrder} that is defined.
More precisely, is it true that a solution $E,H$ to \eqref{eq:CC-Maxwell1stOrder} gives rise to a solution $\bracs{u, \ktcurl{\dddmes}u }\in\ktcurlSob{\ddom}{\dddmes}$ to \eqref{eq:SingularCurlEquation}? \tstk{here}