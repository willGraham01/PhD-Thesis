\section{Discussion and Further Studies} \label{sec:CC-Discussion}

\subsection{On the Reduction to a Scalar System} \label{ssec:ReductionToScalar}
\tstk{always get scalar because one qm propagation constant is zero...}


\subsection{Remarks on the Calder\'on Operator} \label{ssec:CalderonOp}
\tstk{comment at end - Whilst it is not known whether the problem \eqref{eq:PeriodCellCurlCurlStrongForm} is the ``limit" of a thin-structure problem with thick vertices, in the sense of \tstk{scalar SS problem to scalar TS problem, via KZ and EP}, the motivation for our consideration of such variational problems stems from the coincidence? Might go in a future directions/ open questions subsection instead}

In this section we look to draw further parallels between the curl-of-the-curl problem on a classical domain and the system \eqref{eq:QGRawSystem}.
One of our key motivating factors and themes throughout this work has been retaining an intuitive link between the spaces and objects necessary for studying variational problems on singular structures and their classical counterparts.
Further to this, we will demonstrate that the Dirichlet-to-Neumann operator (the so-called $M$-operator) associated with the problem \eqref{eq:QGRawSystem} is a direct analogue of the Calder\'on operator, that is the Dirichlet-to-Neumann operator for the curl of the curl equation.
The Dirichlet and Neumann maps for the classical curl of the curl problem motivate the definitions for their counterparts associated to the problem \eqref{eq:QGRawSystem}; and we will observe that the resulting maps form a boundary triple, providing us with an $M$-operator and thus a graph analogue of $\mathcal{C}$.

\tstk{might be the first time we talk about this, might move it into the $M$-matrix section of QGs, or even into the introductory section of this chapter if it reads better. Also did we mention boundary triples before??}
Let us quickly introduce the Calder\'{o}n operator; let $D\subset\reals^3$ be a domain, and define $\mathcal{A}$ as the curl of the curl operator
\begin{align*}
	\dom\bracs{\mathcal{A}} &= \clbracs{ u\in H^2_{\mathrm{curl}}(D) \setVert \hat{n}\wedge\curl{}u\vert_{\partial D} = m }, \\
	\mathcal{A}u &= \curl{}\bracs{\curl{}u},
\end{align*}
and the Dirichlet and Neumann maps ($\dmap$ and $\nmap$ respectively) by
\begin{align} \label{eq:ClassicalEM-DNMaps}
	\dmap u = u\vert_{\partial D}, \qquad
	\nmap u = \hat{n}\wedge\curl{u}\vert_{\partial D}.
\end{align}
We can quickly validate Green's identity:
\begin{align*}
	\integral{D}{ \mathcal{A}u \cdot \overline{v} - u \cdot \overline{\mathcal{A}v} }{x}
	&= \integral{D}{ \curl{}\bracs{\curl{}u}\cdot\overline{v} - u\cdot\overline{\curl{}\bracs{\curl{}v}} }{x} \\
	&= \integral{D}{ \curl{}u\cdot\overline{\curl{}v} - \curl{}u\cdot\overline{\curl{}v} }{x} \\
	&\quad + \integral{\partial D}{ \hat{n}\wedge\curl{}u\cdot\overline{v} - u\cdot\hat{n}\wedge\overline{\curl{}v} }{S} \\
	&= \integral{\partial D}{ \nmap u \cdot \overline{\dmap v} - \dmap u \cdot \overline{\nmap v} }{S},
\end{align*}
so $\bracs{\ltwo{\partial D}{S}^3, \dmap, \nmap}$ is a boundary triple for the operator $\mathcal{A}$, and we can talk about its associated Dirichlet-to-Neumann map (or $M$-operator).
This is the Calder\'{o}n operator $\mathcal{C}$, that acts on solutions $u$ to \eqref{eq:Maxwell3D}, sending
\begin{align*}
	u\vert_{\partial D} \rightarrow \hat{n}\wedge\bracs{\curl{}u}\vert_{\partial D}.
\end{align*}

Now let us turn to the quantum graph problem \eqref{eq:QGRawSystem}.
%If we expect \eqref{eq:SingStrucCurlCurl} to be some ``limit" of a thin-structure problem with thick vertices, then we expect that the vertex condition \eqref{eq:QGVertexCondition} will be of the form
%\begin{align} \label{eq:DispersiveBC}
%	\dgmap u &= -\omega^2 \tilde{\alpha} \ngmap u,
%\end{align}
%which is the form of the vertex conditions in \tstk{scalar problem}.
%In the context of \eqref{eq:QGRawSystem}, we should expect $\tilde{\alpha}$ be akin to the diagonal matrix of coupling constants (rather than precisely the matrix $\alpha$), whilst $\dgmap, \ngmap$ will be the Dirichlet and Neumann maps for the quantum graph problem \eqref{eq:QGRawSystem}.
Given the definition of $\dmap$ in \eqref{eq:ClassicalEM-DNMaps}, we should expect that
\begin{align} \label{eq:DGMapDef}
	\dgmap u &= 
	\begin{pmatrix}
		u\bracs{v_1} \\ u\bracs{v_2} \\ \vdots \\ u\bracs{v_N}
	\end{pmatrix}
	\in\complex^{3N},
\end{align}
where we have stacked the 3-vectors on top of each other and set $N=\abs{\vertSet}$.
As for $\ngmap u$, this should be the analogue of $\nmap$ in \eqref{eq:ClassicalEM-DNMaps} --- only now the boundary of our domain is the vertices of $\graph$.
Define the functions \tstk{might be worth moving into our usual setup assumption for ease of use?}
\begin{align*}
	\sgn_{jk}: \clbracs{v_j, v_k} \rightarrow \clbracs{-1,0,1}, 
	&\qquad
	\sgn_{jk}(x) = \begin{cases} -1 & x=v_j, \\ 1 & x=v_k, \end{cases}
	&\qquad
	\hat{\sigma}_{jk} &= \sgn_{jk}\widehat{e}_{jk},
\end{align*}
so $\hat{\sigma}_{jk}$ is the ``exterior normal" to the edge $I_{jk}$.
The classical action of $\nmap$ in \eqref{eq:ClassicalEM-DNMaps} then suggests the following candidate for $\ngmap$:
\begin{align} \label{eq:NGMapDef}
	\ngmap u &= 
	\begin{pmatrix}
		\sum_{1\con k} \hat{\sigma}_{1k}\wedge\ktcurl{\dddmes}u\bracs{v_1} \\
		\sum_{2\con k} \hat{\sigma}_{2k}\wedge\ktcurl{\dddmes}u\bracs{v_2} \\
		\vdots \\
		\sum_{N\con k} \hat{\sigma}_{Nk}\wedge\ktcurl{\dddmes}u\bracs{v_N}
	\end{pmatrix}
	\in\complex^{3N},
\end{align}
where we have again stacked the 3-vectors vertically.
From our analysis of $\kt$-tangential curls (corollary \ref{cory:TangCurlEdgeRotated}), we know that
\begin{align*}
	\ktcurl{\dddmes}u &= \bracs{ \bracs{ u_3^{(jk)} }' + \rmi\qm_{jk}u_3^{(jk)} - \rmi\wavenumber U_2^{(jk)} }\widehat{n}_{jk},
\end{align*}
on each edge $I_{jk}$.
Therefore, 
\begin{align*}
	\widehat{e}_{jk}\wedge\ktcurl{\dddmes}u &= \bracs{ \bracs{ u_3^{(jk)} }' + \rmi\qm_{jk}u_3^{(jk)} - \rmi\wavenumber U_2^{(jk)} }\widehat{x}_3
\end{align*}
on $I_{jk}$, and noticing that 
\begin{align*}
	\sum_{j\conRight k} U_2^{(kj)}\bracs{v_j} - \sum_{j\conLeft k} U_2^{(jk)}\bracs{v_j} &=
	\sum_{j\con k} \sgn_{jk}(v_j) U_2^{(jk)}\bracs{v_j},
\end{align*} 
we have for each $v_j\in\vertSet$,
\begin{align*}
	\sum_{j\con k} \hat{\sigma}_{jk}\bracs{v_j} \wedge \ktcurl{\dddmes}u\bracs{v_j}
	&= \bracs{ 
	\rmi\wavenumber\sum_{j\con k} \sgn_{jk}(v_j) U_2^{(jk)}\bracs{v_j}
	- \sum_{j\con k}\bracs{\pdiff{}{n} + \rmi\qm_{jk}}u_3^{(jk)}\bracs{v_j}
	} \widehat{x}_3.
\end{align*}
As such, we can now notice that the vertex conditions for the system \eqref{eq:QGRawSystem} can be written as
\begin{align} \label{eq:VertConditionExplicit}
	\sum_{j\con k} \hat{\sigma}_{jk}\bracs{v_j} \wedge \ktcurl{\dddmes}u\bracs{v_j}
	&= \alpha_j\omega^2 u\bracs{v_j},
	\qquad \forall v_j\in\vertSet.
\end{align}
We can identify \eqref{eq:VertConditionExplicit} as being of the form \eqref{eq:DispersiveBC} where
\begin{align*}
	\tilde{\alpha} = 
	\mathrm{diag}\bracs{\alpha_1, \alpha_1, \alpha_1, \alpha_2, \alpha_2, \alpha_2, ..., \alpha_N, \alpha_N, \alpha_N} \in \complex^{3N\times 3N},
\end{align*}
and $\dgmap, \ngmap$ are as in \eqref{eq:DGMapDef}, \eqref{eq:NGMapDef}.
To complete the analogy, define the operator $\ag$ via the action
\begin{align*}
	\ag u &= 
	\begin{pmatrix}
		\sqbracs{ \rmi\wavenumber\bracs{\diff{}{y} + \rmi\qm_{jk} }u_3^{(jk)} + \wavenumber^2 U_2^{(jk)} }e_{jk}
		+ U_1^{(jk)} n_{jk} \\
		- \bracs{\diff{}{y} + \rmi\qm_{jk} }^2 u_3^{(jk)} + \rmi\wavenumber \bracs{\diff{}{y} + \rmi\qm_{jk} }U_2^{(jk)}
	\end{pmatrix}
\end{align*}
on each edge, where $\mathrm{dom}\bracs{\ag}$ consists of all functions $u$ with the following properties:
\begin{align*}
	u\in\mathrm{dom}\bracs{\ag} \quad\Leftrightarrow\quad &
	\begin{cases}
	u\in L^2\bracs{\graph}\times L^2\bracs{\graph}\times H^2\bracs{\graph}, \\
	U_2^{(jk)}\in \gradSob{I_{jk}}{y}, & \forall I_{jk}\in\edgeSet, \\
	u_3 \text{ is continuous at } v_j, & \forall v_j\in\vertSet, \\
	u_1(v_j) = u_2(v_j) = 0, & \forall v_j\in\vertSet, \\
	\text{\eqref{eq:QGVertexCondition} is satisfied at } v_j, & \forall v_j\in\vertSet.
	\end{cases}
\end{align*}
The eigenvalue problem for the operator $\ag$ is then \eqref{eq:QGRawSystem} and we have that
\begin{align*}
	\integral{I_{jk}}{ \ag u \cdot \overline{v} }{y} - \integral{I_{jk}}{ u \cdot \overline{\ag v} }{y}
	&= \sqbracs{ -u'_3 v_3 + u_3 v_3' - 2\rmi\qm_{jk}u_3 v_3 + \rmi\wavenumber\bracs{U_2 v_3 + u_3 V_2} }_{v_j}^{v_k} \\
	&= -\sqbracs{ \overline{v}_3\bracs{ \bracs{\diff{}{y} + \rmi\qm_{jk} }u_3 - \rmi\wavenumber U_2 } }_{v_j}^{v_k} \\
	&\qquad + \sqbracs{ u_3\overline{\bracs{ \bracs{\diff{}{y} + \rmi\qm_{jk} }v_3 - \rmi\wavenumber V_2 }} }_{v_j}^{v_k},
\end{align*}
implying
\begin{align*}
	&\ip{\ag u}{v}_{L^2\bracs{\graph}^3} - \ip{u}{\ag v}_{L^2\bracs{\graph}^3} \\
	&\quad = \sum_{v_j\in\vertSet}\sum_{j\conLeft k} \integral{I_{jk}}{ \ag u \cdot \overline{v} - u \cdot \overline{\ag v} }{y} \\
	&\quad = \sum_{v_j\in\vertSet}\sum_{j\conLeft k} -\sqbracs{ \overline{v}_3\bracs{ \bracs{\diff{}{y} + \rmi\qm_{jk} }u_3 - \rmi\wavenumber U_2 } }_{v_j}^{v_k}
	+ \sqbracs{ u_3\overline{\bracs{ \bracs{\diff{}{y} + \rmi\qm_{jk} }v_3 - \rmi\wavenumber V_2 }} }_{v_j}^{v_k} \\
	&\quad = \sum_{v_j\in\vertSet} u_3\bracs{v_j}\overline{\bracs{ \sum_{j\con k}\bracs{\pdiff{}{n} + \rmi\qm_{jk}}v_3 - \rmi\wavenumber\sum_{j\con k} \sgn_{jk}(v_j) V_2^{(jk)}\bracs{v_j} }} \\
	&\quad + \sum_{v_j\in\vertSet} \overline{v}_3\bracs{v_j}\bracs{ \sum_{j\con k}\bracs{\pdiff{}{n} + \rmi\qm_{jk}}u_3 - \rmi\wavenumber\sum_{j\con k} \sgn_{jk}(v_j) U_2^{(jk)}\bracs{v_j} } \\
	&\quad = \ngmap u \cdot \overline{\dgmap v} - \dgmap u \cdot \overline{\ngmap v}
	= \ip{\ngmap u}{\dgmap v}_{\complex^{3N}} - \ip{\dgmap u}{\ngmap v}_{\complex^{3N}},
\end{align*}
and so the Green's identity holds. \tstk{do we even define a boundary triple in the QG chapter? If so, saying "green's identity" doesn't make much sense!}
Therefore, $\bracs{\complex^{3N}, \dgmap, \ngmap}$ is a boundary triple for the operator $\ag$.
Given \eqref{eq:ClassicalEM-DNMaps} served as the motivations for the definitions \eqref{eq:DGMapDef} and \eqref{eq:NGMapDef}, the $M$-operator associated with \eqref{eq:QGRawSystem} really is a ``graph version" of the Calder\'on operator for the problem \eqref{eq:SingStrucCurlCurl}.