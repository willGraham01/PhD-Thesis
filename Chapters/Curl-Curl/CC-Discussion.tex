\section{Discussion and Research Outlook} \label{sec:CC-Discussion}
Our efforts to find a more tractable form for the problem \eqref{eq:SingularCurlEquation} have culminated in us arriving at the same system that described the acoustic approximation on a singular structure.
This marks a departure from the classical setting when one considers the curl of the curl equation; we only obtain the acoustic approximation from \eqref{eq:Intro-CurlCurlEqns} under the assumption of oblique wave propagation --- when the propagation constant in one of the axial directions is zero.
This would imply that we should only obtain \eqref{eq:SingularScalarWaveEqn} from \eqref{eq:SingularCurlEquation} when we set $\wavenumber=0$, however we obtain \eqref{eq:SingularScalarWaveEqn} regardless of the value of $\wavenumber$.
The reason for this reduction lies in the nature of our variational problem itself --- specifically the tangential curls themselves.
Let us imagine we have a wave $u = u_0\e^{\rmi\vec{k}\cdot x}$ propagating in our material with wave-vector $\vec{k} = \qm_1\widehat{x}_1 + \qm_2\widehat{x}_2 + \wavenumber\widehat{x}_3$.
For each edge $I_{jk}$ (or plane $P_{jk}$) we can write this wave-vector using the local frame of reference provided by $\widehat{e}_{jk}$ and $\widehat{n}_{jk}$, obtaining $\vec{k} := \qm_{jk}\widehat{e}_{jk} + \qm_{jk}^\perp\widehat{n}_{jk} + \wavenumber\widehat{x}_3$, where $\qm_{jk}^\perp := \qm\cdot n_{jk}$.
The tangential curl $\ktcurl{\dddmes}u$ on $I_{jk}$ is directed in the $\widehat{n}_{jk}$ direction, implying that any changes in $u$ are only occurring in the $\widehat{e}_{jk}$ and $\widehat{x}_3$ directions.
Therefore, there is no propagation out of the plane $P_{jk}$, which corresponds to when we have the component of $\vec{k}$ in the $\widehat{n}_{jk}$ direction being zero --- $\vec{k}\cdot\widehat{n}_{jk}=0$, which happens only if $\qm_{jk}^\perp=0$.
By analogy with the classical setting, one component of our wave-vector being zero results in the curl-of-the-curl equation reducing to the acoustic approximation.
This occurs on each plane $P_{jk}$, and thus we find that our edge ODEs always reduce to the (singular analogue of the) acoustic approximation.
We also stress that this is a direct consequence of the behaviour that the measure $\dddmes$ bestows upon the tangential curls.
There is no way to avoid such a reduction since it is inherent to our singular structure being \emph{singular}.
This last observation does provide an idea that might prevent this reduction to the acoustic approximation: we need part of our domain to be non-singular, which can be achieved by ``filling" the regions between the edges $I_{jk}$ with some background material.
Doing this will move us towards considering a true ``fibre-like" geometry, where we have a background material interlaced with periodic inclusions forming a cross section, which is then extruded into three dimensions.
Pursuit of this idea naturally leads us on to the work in chapter \ref{ch:SingInc}, where we do precisely this --- introduce a background material surrounding our singular structure.

Despite the rather disappointing discovery that our candidate for the curl-of-the-curl equation \eqref{eq:SingularCurlEquation} reduces to the same system as the singular acoustic approximation, the analysis we have performed demonstrates that our approach via singular measures provides a rigid framework for the postulation and solution of variational problems with respect to measures.
Our extension of the definitions and techniques from chapter \ref{ch:ScalarSystem} has enabled us to provide an explicit form for curl-like objects in one dimension (the tangential curl), and understand the properties of said object on our singular structure.
We are also able to demonstrate that the quantum graph problem \eqref{eq:QGRawSystem} derived from our variational starting point bears resemblance to the classical curl-of-the-curl equation, and is solvable via the methods of section \ref{sec:ScalarDiscussion}.
Obtaining \eqref{eq:SingularScalarWaveEqn} from \eqref{eq:SingularCurlEquation} additionally fulfils one of the objectives we set out to achieve: to provide a candidate for the problem that arises in the zero-thickness limit of a thin-structure domain on which the curl-of-the-curl equation is posed.
It has already been highlighted that determining this limit (or the limit of the spectra) is currently an open problem in the literature.
By providing \eqref{eq:SingularCurlEquation} and from it \eqref{eq:QGRawSystem} as a candidate for this limit, we hope to motivate and invite analysis for the curl-of-the-curl equation akin to that conducted in the studies \cite{kuchment2001convergence, kuchment2003asymptotics, exner2005convergence, post2012spectral} for the Neumann Laplacian on thin structures.
The outcome of such a study would either confirm that sending the thickness of a thin structure to zero does indeed result in the curl-of-the-curl equation reducing to the same problem as the acoustic approximation, or indicate the effects that are present in the zero-thickness limit but which are not captured by the corresponding variational problems on the limiting singular structure.
Confirmation of the former would justify our approach utilising singular measures and classically-motivated variational problems as a tool for probing zero-thickness limits of thin structures, and provide a useful justification for physical modelling of such structures in an electromagnetic context.
The latter case being confirmed would also provide a number of insights into why our classically-motivated variational approach fails.
It may be the case that \eqref{eq:SingularCurlEquation} is not the correct analogue of \eqref{eq:Intro-CurlCurlEqns}, and there are considerations that we have overlooked when assuming that the ``visual" limit provides enough information to model the zero-thickness limit correctly.
In any event, we have provided a starting point for future investigations into the zero-thickness limit of the curl-of-the-curl equation, and demonstrated that we can realise our variational problems on singular structures as more tractable systems that are explicitly solvable.

\subsection{Considerations for the study of the first order Maxwell system} \label{ssec:CC-1stOrderMaxwell}
Prior to obtaining the result of section \ref{sec:3DSystemDerivation}, we had expected that any behaviour of the solutions to (or eigenvalues of) \eqref{eq:SingularScalarWaveEqn} would be obtainable from \eqref{eq:SingularCurlEquation}.
Whilst this is certainly the case, we were also expecting behaviours in \eqref{eq:SingularCurlEquation} that were \emph{not} realisable from \eqref{eq:SingularScalarWaveEqn}.
Given that \eqref{eq:Intro-CurlCurlEqns} are themselves particular consequences of the more general (first-order) Maxwell system; it is natural for us to ask whether we can define an analogue of the first order Maxwell system on our singular structures, whether reduction to \eqref{eq:SingularCurlEquation} follows as a direct consequence, or whether there is behaviour exhibited by the first order system that is not captured by \eqref{eq:SingularCurlEquation}.
The first order equations that we would need to assign a meaning to in pursuit of this are,
\begin{subequations} \label{eq:CC-Maxwell1stOrder}
	\begin{align}
		\ktdiv{*} E = 0,
		&\qquad
		\ktcurl{*}E = \rmi\omega\mu_m H, \label{eq:CC-Maxwell1stOrderCurlE} \\
		\ktdiv{*} H = 0,
		&\qquad
		\ktcurl{*}H = -\rmi\omega\epsilon_m E, \label{eq:CC-Maxwell1stOrderCurlH}
	\end{align}
\end{subequations}
for the purpose of this discussion, we will assume the material parameters are constant and we are working on the domain $\ddom$.
By analogy with the classical case, we would need to interpret \eqref{eq:CC-Maxwell1stOrder} as a variational problem, for example the problem of finding divergence-free $E,H\in\pltwo{\ddom}{\dddmes}^3$ such that
\begin{subequations} \label{eq:CC-Maxwell1stOrderVariational}
	\begin{align} 
		\integral{\ddom}{ E\cdot\overline{\ktcurl{}\phi} - \rmi\omega\mu_m H\cdot\overline{\phi} }{\dddmes} &= 0, 
		\qquad &\forall \phi\in\psmooth{\ddom}^3, \\
		\integral{\ddom}{ H\cdot\overline{\ktcurl{}\psi} + \rmi\omega\epsilon_m E\cdot\overline{\psi} }{\dddmes} &= 0,
		\qquad &\forall \psi\in\psmooth{\ddom}^3.
	\end{align}
\end{subequations}
We note that we have placed asterisks on the ``curls" in \eqref{eq:CC-Maxwell1stOrder} in order to make a distinction between these and the tangential curls that appear in \eqref{eq:SingularCurlEquation}.
There is indeed a distinction between $\ktcurl{*}E$ of a vector field $E\in\pltwo{\ddom}{\dddmes}^3$, which is only defined by finding a field $H$ that satisfies \eqref{eq:CC-Maxwell1stOrderCurlE}, and $\ktcurl{\dddmes}E$ for $E\in\ktcurlSob{\ddom}{\dddmes}$, which defined through approximations by smooth functions.
This is because the curl no longer appears as an interior operator in \eqref{eq:CC-Maxwell1stOrder}, unlike \eqref{eq:SingularCurlEquation} where we need to define $\ktcurl{\dddmes}u$ independently of our problem so that we can then understand the variational problem \eqref{eq:SingularCurlEquation-VariationalForm}.
This raises the question of under what additional conditions does a solution $E,H$ to \eqref{eq:CC-Maxwell1stOrder} give rise to a solution $\bracs{v, \ktcurl{\dddmes}v }\in\ktcurlSob{\ddom}{\dddmes}$ to \eqref{eq:SingularCurlEquation}?
More precisely, what additional conditions need to be imposed upon the more general system \eqref{eq:CC-Maxwell1stOrder} to obtain \eqref{eq:SingularCurlEquation}?
If such conditions exist, this answers the question as to whether there are additional behaviours exhibited by (solutions to) the first order singular Maxwell system that are not captured by the singular curl-of-the-curl equation.
Conversely, \eqref{eq:CC-Maxwell1stOrder} collapsing to \eqref{eq:SingularCurlEquation} in all cases would imply that the acoustic approximation \eqref{eq:SingularScalarWaveEqn} provides a complete description of electromagnetism on singular structures.